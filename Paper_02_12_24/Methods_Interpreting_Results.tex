%%%%%
\subsection{Interpreting Model Results}
\label{interpreting}

%%%
\subsubsection{Finding $\theta$ for Each Budgetary Decision Metric}
\label{finding_theta}

For each budgetary metric and each model we need to find a corresponding decision threshold $\theta$.  For each new crash notification the model outputs a prediction value $p$, and if $p \ge \theta$ the system recommends immediately dispatching an ambulance.  

For each of the 713,566 samples we use from the CRSS dataset we have a target value $y \in \{0,1\}$ that tells whether the sample is in the Neg or Pos class.  Each model returns for each of those historical samples a value $p \in [0,1]$ that indicates whether the model predicts that the sample is in the positive and negative class.  For each model we need to find the value of $p$ that corresponds to each of the budgetary decision criteria to set as the decision threshold $\theta$.  

To illustrate our methods for finding $\theta$ for each budgetary decision metric and each model we use the results of the Random Forest Classifier on the hard features, which returns values  $p \in [0.10619,0.311074]$.  Table \ref{RFC_Hard_0_Slices}  shows some of the results.  We have rounded $p$ to four decimal places and sorted model results by increasing $p$, giving the number of elements of the negative and positive classes for each $p$.  From Neg and Pos we can calculate the TN, FP, FN, and TP we would have if we chose $\theta$ to be that value of $p$.  The number of negative and positive samples are constant at $N=605,610$ and $P=107,956$. TN is the cumulative sum of Neg, and $\text{FP} = \text{N} - \text{TN}$; similarly for FN and TP.  

In Table \ref{RFC_Hard_0_Slices}, we have shown the results for four bands of $p$.  The first and fourth are the beginning and end of the range of $p$.  The second range shows where $\text{FP/P} \approx 0.05$, the first budgetary decision metric.  The third range illustrates that while Precision generally increases from 0.15 (the class ratio) to 1, it is not perfectly nondecreasing, even with $p$ rounded to four decimal places.  Note in the fourth column, $m$Prob, the values generally increase from 0 to 1 but not in a smooth way.  We have to significantly smooth that metric to find the value of $p$ where it equals 50\%.  The full table is at 
\verb|Analyze_Proba/RFC_Hard_Run_0_Round_4_Rolling_Intervals|.

Our {\bf first budgetary decision metric}, $\text{FP}/\text{P}$, is a nonincreasing function of $p$ because $P$ is constant and FP is P minus the cumulative sum of nonnegative integers Neg.  Because $\text{FP}/\text{P}$ is a nonincreasing function of $p$, we can choose $\theta$ to be the value of $p$ where $\text{FP}/\text{P}$ is closest to $0.05$.  For this run of the Random Forest Classifier with the hard features we can choose $\theta = 0.2580$, and a second run with a different random seed gave $\theta = 0.2583$, so we can say with some confidence that we should choose $\theta = 0.258$ for this model.  

\begin{table}
\caption{
	\normalsize\normalfont
	Metrics on $p$ Output of Random Forest Classifier on the Hard Features.  Table accompanies \S\ref{finding_theta}
}
\label{RFC_Hard_0_Slices}

{\normalsize
\normalfont
\begin{tabular}{lrrlrrrrlll}
\toprule
	\multicolumn{1}{c}{$p$} &     
	\multicolumn{1}{c}{Neg} &   
	\multicolumn{1}{c}{Pos} & 
	\multicolumn{1}{c}{$m$Prob} &     
	\multicolumn{1}{c}{TN} &      
	\multicolumn{1}{c}{FP} &      
	\multicolumn{1}{c}{FN} &      
	\multicolumn{1}{c}{TP} &   
	\multicolumn{1}{c}{Prec} &    
	\multicolumn{1}{c}{Rec} &   
	\multicolumn{1}{c}{FP/P} \\
\midrule
0.1062 & 1 & 0 & 0.0000 & 1 & 605,609 & 0 & 107,956 & 0.1513 & 1.0000 & 5.6098\cr
0.1063 & 86 & 2 & 0.0227 & 87 & 605,523 & 2 & 107,954 & 0.1513  & 1.0000 & 5.609\cr
0.1064 & 51 & 0 & 0.0000 & 138 & 605,472 & 2 & 107,954 & 0.1513 & 1.0000 & 5.6085\cr
%0.1065 & 110 & 0 & 0 & 248 & 605,362 & 2 & 107,954 & 0.1513 & 5.6075\cr
\multicolumn{1}{c}{\vdots} & 
\multicolumn{1}{r}{\vdots} & 
\multicolumn{1}{r}{\vdots} & 
\multicolumn{1}{l}{\vdots} & 
\multicolumn{1}{r}{\vdots} & 
\multicolumn{1}{r}{\vdots} & 
\multicolumn{1}{r}{\vdots} & 
\multicolumn{1}{c}{\vdots} & 
\multicolumn{1}{c}{\vdots} & 
%\multicolumn{1}{c}{\vdots} & 
\multicolumn{1}{c}{\vdots} \cr

0.2579 & 33 & 40 & 0.5479 & 600,161 & 5,449 & 102,100 & 5,856 & 0.518 & 0.0542 & 0.0505\cr
0.258 & 39 & 26 & 0.4000 & 600,200 & 5,410 & 102,126 & 5,830 & 0.5187 & 0.0540 & 0.0501\cr
0.2581 & 39 & 25 & 0.3906 & 600,239 & 5,371 & 102,151 & 5,805 & 0.5194 & 0.0538 & 0.0498\cr
0.2582 & 42 & 31 & 0.4247 & 600,281 & 5,329 & 102,182 & 5,774 & 0.5200 & 0.0535 & 0.0494\cr

\multicolumn{1}{c}{\vdots} & 
\multicolumn{1}{r}{\vdots} & 
\multicolumn{1}{r}{\vdots} & 
\multicolumn{1}{l}{\vdots} & 
\multicolumn{1}{r}{\vdots} & 
\multicolumn{1}{r}{\vdots} & 
\multicolumn{1}{c}{\vdots} & 
\multicolumn{1}{r}{\vdots} & 
\multicolumn{1}{c}{\vdots} & 
%\multicolumn{1}{c}{\vdots} & 
\multicolumn{1}{c}{\vdots} \cr

0.2858 & 4 & 8 & 0.6667 & 605,179 & 431 & 106,858 & 1,098 & 0.7181 & 0.0102 & 0.0040\cr
0.2859 & 1 & 9 & 0.9000 & 605,180 & 430 & 106,867 & 1,089 & 0.7169 & 0.0101 & 0.0040\cr
0.286 & 4 & 11 & 0.7333 & 605,184 & 426 & 106,878 & 1,078 & 0.7168 & 0.0100 & 0.0039\cr
0.2861 & 1 & 10 & 0.9091 & 605,185 & 425 & 106,888 & 1,068 & 0.7153 & 0.0099 & 0.0039\cr
0.2862 & 5 & 6 & 0.5455 & 605,190 & 420 & 106,894 & 1,062 & 0.7166 & 0.0098 & 0.0039\cr


\multicolumn{1}{c}{\vdots} & 
\multicolumn{1}{r}{\vdots} & 
\multicolumn{1}{r}{\vdots} & 
\multicolumn{1}{l}{\vdots} & 
\multicolumn{1}{r}{\vdots} & 
\multicolumn{1}{r}{\vdots} & 
\multicolumn{1}{c}{\vdots} & 
\multicolumn{1}{r}{\vdots} & 
\multicolumn{1}{c}{\vdots} & 
%\multicolumn{1}{c}{\vdots} & 
\multicolumn{1}{c}{\vdots} \cr

%0.3083 & 0 & 1 & 1 & 605,609 & 1 & 107,950 & 6 & 0.8571 & 0\cr
%0.3086 & 1 & 0 & 0 & 605,610 & 0 & 107,950 & 6 & 1 & 0\cr
%0.3087 & 0 & 1 & 1 & 605,610 & 0 & 107,951 & 5 & 1 & 0\cr
%0.3092 & 0 & 1 & 1 & 605,610 & 0 & 107,952 & 4 & 1 & 0\cr
0.3097 & 0 & 1 & 1.0000 & 605,610 & 0 & 107,953 & 3 & 1.0000 & 0.0000 & 0.0000\cr
0.3108 & 0 & 1 & 1.0000 & 605,610 & 0 & 107,954 & 2 & 1.0000 & 0.0000 & 0.0000\cr
0.3111 & 0 & 2 & 1.0000 & 605,610 & 0 & 107,956 & 0  & nan & 0.0000 & 0.0000\cr

\bottomrule
\end{tabular}
}
\end{table}

\FloatBarrier

Table \ref{RFC_Hard_0_Slices} shows that our {\bf second budgetary decision metric}, $\text{Precision} = \text{TP}/(\text{FP} + \text{TP})$, generally increases from 0.15 to 1 (0.15 is the class ratio $\text{P}/(\text{N} + \text{P})$), but is not an increasing function of $p$ at that level of granularity.  

To go from one $p$ to the next, or from one band of $p$ to the next, 

\begin{equation} \label{eq:delta_p}\hfil
\frac{\Delta \text{TP}}{\Delta p} = -\text{Neg} 
\qquad \text{ and }\qquad
\frac{\Delta \text{FP}}{\Delta p} = -\text{Pos} 
\end{equation}

\FloatBarrier

For large changes in $p$, $\text{Prec}(p)$ is an increasing function, but what do we mean by ``large''?  In Equation \ref{eq:delta_p_2} we assume that $\text{Prec}(p)$ increases on an interval of length $\Delta p$, then we derive a relationship that tells us the proportion of Neg to Pos we need in that interval:  

\begin{equation} \label{eq:delta_p_2}\hfil
\begin{aligned}
	\text{Prec}(p) &< \text{Prec}(p + \Delta p) \cr
	\frac{\text{TP}}{\text{FP} + \text{TP}} &< 
	\frac{\text{TP} - \text{Neg}}{(\text{FP} - \text{Pos}) + (\text{TP} - \text{Neg})} \cr 
	&\vdots \cr
	\text{FP}\cdot \text{Pos} &< \text{TP} \cdot \text{Neg} \cr
	\frac{\text{FP}}{\text{TP}} &< \frac{\text{Neg}}{\text{Pos}} \cr
\end{aligned}
\end{equation}

\FloatBarrier

Having no elements of the negative class in a $p$-interval would make precision decrease, so we cut the $p$-values into bands with some minimum number of elements of each class.  For each model, the \verb|Keras/Analyze_Proba| directory has a spreadsheet like Table \ref{RFC_Hard_50_Slices} for minimum number of class elements 5, 10, 25, 50, 100, 200, 400, 800, 1600, 3200, and 6400.  An opportunity for future research (\S\ref{simplifying_assumptions}) is to develop a method to cut the $p$ output into the smallest possible bands such that $\text{Prec}(p)$ is an increasing function.  


In Table \ref{RFC_Hard_50_Slices}, we have cut $p$ into bands such that each band has at least fifty elements of each of the negative and positive classes.  The $p$ values in the table are the average of the minimum and maximum values of $p$ in the band.  The full table is at \verb|Analyze_Proba/RFC_Easy_Run_0_50_Slices.csv|.

At this level of granularity and in this range of $p$, the precision is an increasing function of $p$, and we can say that to get $\text{Precision} = 2/3$ we should set $\theta \approx 0.276$.   Another run of the same model with different random seed gave $\theta \approx 0.279$.



\begin{table}
\caption{
	\normalsize\normalfont
	Metrics on $p$ Output of Random Forest Classifier on the Hard Features with Minimum of 50 Elements of Each Class in Each Band.  Table accompanies \S\ref{finding_theta}
}
\label{RFC_Hard_50_Slices}

{\normalsize
\normalfont
\begin{tabular}{lrrrrrrrlll}
\toprule
	\multicolumn{1}{c}{$p$} &     
	\multicolumn{1}{c}{Neg} &   
	\multicolumn{1}{c}{Pos} & 
	\multicolumn{1}{c}{$m$Prob} &     
	\multicolumn{1}{c}{TN} &      
	\multicolumn{1}{c}{FP} &      
	\multicolumn{1}{c}{FN} &      
	\multicolumn{1}{c}{TP} &   
	\multicolumn{1}{c}{Prec} &    
	\multicolumn{1}{c}{Rec} &   
	\multicolumn{1}{c}{FP/P} \\
\midrule
0.27565 & 50 & 51 & 0.505 & 604,454 & 1,156 & 105,702 & 2,254 & 0.661 & 0.0209 & 0.0107 \cr
0.27608 & 50 & 59 & 0.5413 & 604,504 & 1,106 & 105,761 & 2,195 & 0.665 & 0.0203 & 0.0102 \cr
0.27667 & 50 & 79 & 0.6124 & 604,554 & 1,056 & 105,840 & 2,116 & 0.6671 & 0.0196 & 0.0098 \cr
0.27732 & 56 & 50 & 0.4717 & 604,610 & 1,000 & 105,890 & 2,066 & 0.6738 & 0.0191 & 0.0093 \cr
0.27798 & 50 & 80 & 0.6154 & 604,660 & 950 & 105,970 & 1,986 & 0.6764 & 0.0184 & 0.0088 \cr
0.27869 & 50 & 67 & 0.5726 & 604,710 & 900 & 106,037 & 1,919 & 0.6807 & 0.0178 & 0.0083 \cr
\bottomrule
\end{tabular}
}
\end{table}

\FloatBarrier

For the the {\bf third budgetary decision criterion}, that the minimum probability that each immediately dispatched ambulance is needed is at least 50\%, with the metric $m\text{Prob} = \text{Pos}/(\text{Neg}+\text{Pos}) \ge 0.50$, equivalent to $\text{Pos} \ge \text{Neg}$.  

Table \ref{RFC_Hard_6400_Slices} shows that, if we zoom out to where each band has at least 6,400 elements in each class, cutting $p$ into only seventeen intervals, $m\text{Prob}$ generally increases, but even at this high level the metric decreases between $p=0.1606$ and $p = 0.1689$.  This third metric is highly volatile.  


\begin{table}
\caption{
	\normalsize\normalfont
	Metrics on $p$ Output of Random Forest Classifier on the Hard Features with Minimum of 6400 Elements of Each Class in Each Band.  Table accompanies \S\ref{finding_theta}
}
\label{RFC_Hard_6400_Slices}

{\normalsize
\normalfont
\begin{tabular}{lrrrrrrrlll}
\toprule
	\multicolumn{1}{c}{$p$} &     
	\multicolumn{1}{c}{Neg} &   
	\multicolumn{1}{c}{Pos} & 
	\multicolumn{1}{c}{$m$Prob} &     
	\multicolumn{1}{c}{TN} &      
	\multicolumn{1}{c}{FP} &      
	\multicolumn{1}{c}{FN} &      
	\multicolumn{1}{c}{TP} &   
	\multicolumn{1}{c}{Prec} &    
	\multicolumn{1}{c}{Rec} &   
	\multicolumn{1}{c}{FP/P} \\
\midrule
0.1166 & 132,612 & 5,396 & 0.0391 & 132,612 & 472,998 & 5,396 & 102,560 & 0.1782 & 0.95 & 4.3814 \cr
0.1293 & 79,551 & 6,400 & 0.0745 & 212,163 & 393,447 & 11,796 & 96,160 & 0.1964 & 0.8907 & 3.6445 \cr
0.1336 & 57,673 & 6,400 & 0.0999 & 269,836 & 335,774 & 18,196 & 89,760 & 0.2109 & 0.8314 & 3.1103 \cr
0.137 & 42,946 & 6,400 & 0.1297 & 312,782 & 292,828 & 24,596 & 83,360 & 0.2216 & 0.7722 & 2.7125 \cr
0.1399 & 36,922 & 6,400 & 0.1477 & 349,704 & 255,906 & 30,996 & 76,960 & 0.2312 & 0.7129 & 2.3705 \cr
0.1432 & 35,947 & 6,400 & 0.1511 & 385,651 & 219,959 & 37,396 & 70,560 & 0.2429 & 0.6536 & 2.0375 \cr
0.1469 & 32,431 & 6,400 & 0.1648 & 418,082 & 187,528 & 43,796 & 64,160 & 0.2549 & 0.5943 & 1.7371 \cr
0.1509 & 30,786 & 6,400 & 0.1721 & 448,868 & 156,742 & 50,196 & 57,760 & 0.2693 & 0.535 & 1.4519 \cr
0.1553 & 25,477 & 6,401 & 0.2008 & 474,345 & 131,265 & 56,597 & 51,359 & 0.2812 & 0.4757 & 1.2159 \cr
0.1606 & 23,300 & 6,400 & 0.2155 & 497,645 & 107,965 & 62,997 & 44,959 & 0.294 & 0.4165 & 1.0001 \cr
0.1689 & 26,951 & 6,400 & 0.1919 & 524,596 & 81,014 & 69,397 & 38,559 & 0.3225 & 0.3572 & 0.7504 \cr
0.1806 & 21,160 & 6,400 & 0.2322 & 545,756 & 59,854 & 75,797 & 32,159 & 0.3495 & 0.2979 & 0.5544 \cr
0.1979 & 17,362 & 6,400 & 0.2693 & 563,118 & 42,492 & 82,197 & 25,759 & 0.3774 & 0.2386 & 0.3936 \cr
0.2199 & 16,065 & 6,400 & 0.2849 & 579,183 & 26,427 & 88,597 & 19,359 & 0.4228 & 0.1793 & 0.2448 \cr
0.2366 & 10,969 & 6,400 & 0.3685 & 590,152 & 15,458 & 94,997 & 12,959 & 0.456 & 0.12 & 0.1432 \cr
0.249 & 9,058 & 6,400 & 0.414 & 599,210 & 6,400 & 101,397 & 6,559 & 0.5061 & 0.0608 & 0.0593 \cr
0.2835 & 6,400 & 6,559 & 0.5061 & 605,610 & 0 & 107,956 & 0 & nan & 0 & 0 \cr
\bottomrule
\end{tabular}
}
\end{table}

\FloatBarrier

We tried two ways to estimate where $m\text{Prob} = 0.50$, and both gave the same result.  

Method 1:  If we slice the $p$ results into bands with at least 100 elements of each class, partly shown in Table \ref{RFC_Hard_100_Slices}, there is no band with $p<0.2692$ with $m\text{Prob} \ge 0.5$ and no band with $p > 0.2733$ with $m\text{Prob} \le 0.5$, so we can say with some confidence that if we chose $\theta \approx 0.271$, each ambulance immediately dispatched has at least a 50\% chance of being needed.


\begin{table}
\caption{
	\normalsize\normalfont
	Metrics on $p$ Output of Random Forest Classifier on the Hard Features with Minimum of 100 Elements of Each Class in Each Band.  Table accompanies \S\ref{finding_theta}
}
\label{RFC_Hard_100_Slices}

{\normalsize
\normalfont
\begin{tabular}{lrrrrrrrlll}
\toprule
	\multicolumn{1}{c}{$p$} &     
	\multicolumn{1}{c}{Neg} &   
	\multicolumn{1}{c}{Pos} & 
	\multicolumn{1}{c}{$m$Prob} &     
	\multicolumn{1}{c}{TN} &      
	\multicolumn{1}{c}{FP} &      
	\multicolumn{1}{c}{FN} &      
	\multicolumn{1}{c}{TP} &   
	\multicolumn{1}{c}{Prec} &    
	\multicolumn{1}{c}{Rec} &   
	\multicolumn{1}{c}{FP/P} \\
\midrule
0.2692 & 118 & 100 & 0.4587 & 603,564 & 2,046 & 104,738 & 3,218 & 0.6113 & 0.0298 & 0.019 \cr
0.2698 & 100 & 101 & 0.5025 & 603,664 & 1,946 & 104,839 & 3,117 & 0.6156 & 0.0289 & 0.018 \cr
0.2704 & 127 & 100 & 0.4405 & 603,791 & 1,819 & 104,939 & 3,017 & 0.6239 & 0.0279 & 0.0168 \cr
0.2711 & 108 & 100 & 0.4808 & 603,899 & 1,711 & 105,039 & 2,917 & 0.6303 & 0.027 & 0.0158 \cr
0.2718 & 111 & 100 & 0.4739 & 604,010 & 1,600 & 105,139 & 2,817 & 0.6378 & 0.0261 & 0.0148 \cr
0.2725 & 100 & 109 & 0.5215 & 604,110 & 1,500 & 105,248 & 2,708 & 0.6435 & 0.0251 & 0.0139 \cr
0.2733 & 100 & 149 & 0.5984 & 604,210 & 1,400 & 105,397 & 2,559 & 0.6464 & 0.0237 & 0.013 \cr
\bottomrule
\end{tabular}
}
\end{table}

\FloatBarrier


Method 2:  We used a rolling sum of the Neg and Pos to intervals with enough Neg and Pos to smooth out the metric $m$Prob.  With $p$ rounded to four decimal places we still have volatility at each value of $p$, and the metric gets close to the target value (0.50) many times over a large range.  If we take the rolling average of the ten or twenty rows centered at that value of $p$, we see less volatility, but even in this small interval the $m$Prob increases and decreases.  With a window size of 20, however, the only values of $p$ where the metric is close to the target are in this range, so we can choose $\theta = 0.272$, and our two methods agree.  The full chart, up to a window of size 2,000, is in \verb|Analyze_Proba/RFC_Hard_Run_0_Round_4_Rolling_Intervals|.




\begin{table}
\caption{
	\normalsize\normalfont
	Smoothing $m$Prob with Rolling Sums of Different Window Sizes.
	Table accompanies \S\ref{finding_theta}
}
\label{RFC_Hard_Run_0_Round_4_Rolling_Intervals}

{\normalsize
\normalfont
\begin{tabular}{c | rrl | rrl | rrl | rrl}
\toprule
	\multicolumn{1}{c}{Window Size} &
	\multicolumn{3}{c}{1} &
	\multicolumn{3}{c}{10} &
	\multicolumn{3}{c}{20} &
	\multicolumn{3}{c}{50} \cr\cline{1-1}

	\multicolumn{1}{c}{$p$} &     
	\multicolumn{1}{|c}{Neg} &   
	\multicolumn{1}{c}{Pos} & 
	\multicolumn{1}{c}{$m$Prob} & 
	\multicolumn{1}{|c}{Neg} &   
	\multicolumn{1}{c}{Pos} & 
	\multicolumn{1}{c}{$m$Prob} & 
	\multicolumn{1}{|c}{Neg} &   
	\multicolumn{1}{c}{Pos} & 
	\multicolumn{1}{c}{$m$Prob} & 
	\multicolumn{1}{|c}{Neg} &   
	\multicolumn{1}{c}{Pos} & 
	\multicolumn{1}{c}{$m$Prob} \cr
\midrule
0.2717 & 14 & 22 & 0.6111 & 144 & 166 & 0.4645 & 303 & 320 & 0.4864 & 754 & 771 & 0.4944 \cr
0.2718 & 12 & 18 & 0.6 & 153 & 159 & 0.4904 & 310 & 315 & 0.496 & 754 & 768 & 0.4954 \cr
0.2719 & 14 & 16 & 0.5333 & 143 & 164 & 0.4658 & 307 & 323 & 0.4873 & 758 & 757 & 0.5003 \cr
0.2720 & 12 & 13 & 0.52 & 152 & 156 & 0.4935 & 300 & 323 & 0.4815 & 756 & 745 & 0.5037 \cr
0.2721 & 26 & 13 & 0.3333 & 161 & 150 & 0.5177 & 296 & 312 & 0.4868 & 768 & 737 & 0.5103 \cr
0.2722 & 10 & 22 & 0.6875 & 159 & 149 & 0.5162 & 306 & 296 & 0.5083 & 755 & 726 & 0.5098 \cr
0.2723 & 17 & 7 & 0.2917 & 159 & 147 & 0.5196 & 309 & 291 & 0.515 & 756 & 717 & 0.5132 \cr
%0.2724 & 16 & 21 & 0.5676 & 151 & 153 & 0.4967 & 308 & 290 & 0.5151 & 755 & 702 & 0.5182 \cr
\hline
\multicolumn{6}{c}{} &&&&&&\cr
\multicolumn{6}{l}{
	Range of $p$ where $\left| m\text{Prob} - 0.50 \right| < 0.01$
} &&&&&&\cr
\multicolumn{6}{c}{} &&&&&&\cr

& \multicolumn{2}{r}{min($p$)} & 0.2472 
& \multicolumn{2}{r}{min($p$)} & 0.2718 
& \multicolumn{2}{r}{min($p$)} & 0.2718 
& \multicolumn{2}{r}{min($p$)} & 0.2711
\cr 
& \multicolumn{2}{r}{max($p$)} & 0.3076
& \multicolumn{2}{r}{max($p$)} & 0.2726 
& \multicolumn{2}{r}{max($p$)} & 0.2722 
& \multicolumn{2}{r}{max($p$)} & 0.2722
\cr

\bottomrule
\end{tabular}
}
\end{table}

\FloatBarrier


\begin{comment}
Method 2:  This method is worked out in the 

\verb|RFC_Hard_0_Slices_Find_theta_Budget_Criterion_3.csv| file in the \verb|/Analysis_Spreadsheets/| folder.  

The Random Forest Classifier on the Hard features returns 458,530 unique values of $p$ with between 1 and 83 samples per value of $p$.  For each value of $p$ we know how many samples are in the Neg and Pos classes.  Order the $p$ values increasing.  For each value of $p$, make an interval with the thousand nearest values of $p$ (500 above and 500 below), and find the total number of elements of the positive and negative classes in that neighborhood.  Using those Neg and Pos values, calculate  $\text{Pos}/(\text{Neg}+\text{Pos})$ for each $p$.  Then see whether there is a small interval where $\text{Pos}/(\text{Neg}+\text{Pos}) \approx 0.50$.  We chose ``within 0.001'' to mean ``approximately.''

The first value of $p$ with $|\text{Pos}/(\text{Neg}+\text{Pos}) - 0.50| < 0.001$ was at $p=0.271584$, and the last at $p = 0.27174$.  In that range of 49 values of $p$, 21 of them satisfied the criterion.  Given those results, it is reasonable to say that $\theta \approx 0.272$ satisfies the criterion.  
\end{comment}

%%%
\subsubsection{Choosing the Best Model for each Budgetary Decision Metric}
\label{choosing_model}

For each budgetary constraint we want to find the model that, within the constraint, immediately dispatches the most ambulances to crash persons who need them, which is given as a proportion in Recall = TP/P.  Using as an example our first budgetary constraint, $\text{FP}/\text{P} = 0.05$, we need to find, for each model whether there exists a value of $p$ where $\text{FP}/\text{P}$ is close to 0.05, then find the best $\theta$ interval in that neighborhood.  Of those valid results, find the model that gives the highest Recall.  

Table \ref{FP_P_0_05_hard} shows the best results for each model algorithm.  Within these, the Balanced Random Forest Classifier gives the best results, immediately dispatching the largest proportion (11.67\%) of needed ambulances while staying within the budgetary constraint.  The KerasClassifier with the Binary Focal Crossentropy loss function is a close second, and those two are clearly better than the other six models.  

The Balanced Bagging model here is an interesting case that we cover in the next section, \S \ref{Methods_Model_Failure}.

\begin{table}[h]
\caption{\normalfont\normalsize Comparing Models:  Best results for each algorithm on the Hard features for budgetary criterion $\text{FP}/\text{P}$ closest to $0.05$.  Table accompanies \S\ref{choosing_model}}
\label{FP_P_0_05_hard}

{\normalfont\normalsize
\begin{tabular}{lcclrrrrll}
\toprule
	Algorithm & 
	\multicolumn{1}{c}{$\alpha$} & 
	\multicolumn{1}{c}{$\gamma$} & 
	\multicolumn{1}{c}{$p$} & 
	\multicolumn{1}{c}{TN} & 
	\multicolumn{1}{c}{FP} & 
	\multicolumn{1}{c}{FN} & 
	\multicolumn{1}{c}{TP} & 
	\multicolumn{1}{c}{Recall} & 
	\multicolumn{1}{c}{$\text{FP} / \text{P}$} 
\cr
\noalign{\vskip 2pt}
\hline
\noalign{\vskip 2pt}

Bal RF & 0.5 &  & 0.86 & 600,464 & 5,146 & 95,359 & 12,597 & 0.1167 & 0.048\cr
%Bal RF & 0.85 &  & 0.86 & 600,587 & 5,023 & 95,845 & 12,111 & 0.047\cr
Keras & 0.5 & 0.0 & 0.598 & 600,212 & 5,398 & 97,029 & 10,927 & 0.1012 & 0.05\cr
%Keras & 0.5 & 2.0 & 0.547 & 600,212 & 5,398 & 97,110 & 10,846 & 0.05\cr
%Keras & 0.5 & 1.0 & 0.56 & 600,212 & 5,398 & 97,282 & 10,674 & 0.05\cr
%Keras & 0.85 & 0.0 & 0.899 & 600,212 & 5,398 & 97,408 & 10,548 & 0.05\cr
Log Reg & 0.5 &  & 0.549 & 600,212 & 5,398 & 100,764 & 7,192 & 0.0666 & 0.05\cr
RUSBoost &  &  & 0.5 & 600,212 & 5,398 & 101,041 & 6,915 & 0.0641 & 0.05\cr
%Log Reg &  &  & 0.883 & 600,212 & 5,398 & 101,124 & 6,832 & 0.05\cr
AdaBoost &  &  & 0.501 & 600,212 & 5,398 & 101,131 & 6,825 & 0.0632 & 0.05\cr
Bal Bag &  &  & 0.9 & 602,185 & 3,425 & 101,272 & 6,684 & 0.0619 & 0.032\cr
RF &  &  & 0.258 & 600,212 & 5,398 & 102,135 & 5,821 & 0.0539 & 0.05\cr
Easy Ens &  &  & 0.549 & 600,155 & 5,455 & 102,322 & 5,634 & 0.0522 & 0.051\cr
\bottomrule
\end{tabular}
}
\end{table}

\FloatBarrier

%%%
\subsubsection{Detecting Model Failure}
\label{Methods_Model_Failure}

We have found two ways a model can fail to give a decision threshold $\theta$ that satisfies the budgetary criteria.  The second one we consider is the usual problem, that the model does not sufficiently separate the positive and negative classes to give enough correct recommendations.  The other cause is more subtle, and we consider it first because we just saw it in Table \ref{FP_P_0_05_hard}, that for the Balanced Bagging algorithm, the value of FP/P closest to 0.05 (FP/P = 0.032) is not close to 0.05, and the cause of the failure is the numerics in the algorithm.

The RUSBoost algorithm on the hard features only gives values of $p$ in a tiny range, $p \in [0.499, 0.5011]$, but in that range of 00021 the 713,566 samples return 706,938 different values of $p$.  That's about as close to ``continuous'' as a discrete function can get.  Digging down into the results returned by the Balanced Bagging algorithm on the hard features, 99\% of the values of $p$ are rounded to the nearest tenth, making the results extremely discrete.  In Table \ref{BalBag_Hard_0_Slices}, going from $p=0.897$ to $p=0.9$ we add 19,052 new samples that change the FP/P value from 0.11 to 0.03.   The model results give no way to find a budgetary decision threshold $\theta$ that gives us a 5\% increase in the number of ambulances dispatched to automated notifications from cell phones.  This model may have too much volatility to be useful.  

The Balanced Random Forest algorithm on the hard features has a similar quirk, that 93\% of its $p$ values are rounded to the nearest hundredth, which explains the $\text{FP}/\text{P} = 0.048$ as the closest to 0.05 in Table \ref{FP_P_0_05_hard}.  These two algorithms, Bal Bag and Bal RF, interestingly, had more detail in the Medium Features results and even more detail for the Easy features, which is the opposite of what most of the other algorithms did.  The AdaBoost, Keras, Log Reg, and RUSBoost all had, from the 713,566 samples, about 700,000 unique values of $p$ for the hard features, 650,000 for the medium features, and 150,000 for the easy features.  A breakdown of those numbers is given in 

\noindent\verb|Value_Counts_y_proba.csv| in the \verb|Keras/Analyze_Proba| folder, and the value counts of $p$ are given for each algorithm in files such as \verb|BalBag_Hard_Value_Counts.csv|.


\begin{table}[h]
\caption{\normalfont\normalsize Model Failure:  Balanced Bagging Algorithm on the Hard Features with FP/P closest to 0.05.  Table accompanies \S\ref{Methods_Model_Failure}}
\label{BalBag_Hard_0_Slices}

{\normalfont\normalsize
\begin{tabular}{lrrrrrrl}
\toprule
	\multicolumn{1}{c}{$p$} & 
	\multicolumn{1}{c}{Neg} & 
	\multicolumn{1}{c}{Pos} & 
	\multicolumn{1}{c}{TN} & 
	\multicolumn{1}{c}{FP} & 
	\multicolumn{1}{c}{FN} & 
	\multicolumn{1}{c}{TP} & 
	\multicolumn{1}{c}{$\text{FP} / \text{P}$} 
\cr
\noalign{\vskip 2pt}
\hline
\noalign{\vskip 2pt}

0.896 & 1 & 0 & 593,725 & 11,885 & 90,679 & 17,277 & 0.110091 \cr
0.897 & 1 & 0 & 593,726 & 11,884 & 90,679 & 17,277 & 0.110082 \cr
0.9 & 8,459 & 10,593 & 602,185 & 3,425 & 101,272 & 6,684 & 0.031726 \cr
0.908 & 1 & 1 & 602,186 & 3,424 & 101,273 & 6,683 & 0.031717 \cr
0.913 & 1 & 0 & 602,187 & 3,423 & 101,273 & 6,683 & 0.031707 \cr

\bottomrule
\end{tabular}
}
\end{table}

\FloatBarrier

The second kind of model failure is the usual problem with machine learning models, that the algorithm is not successful in detecting enough of a pattern in the data to build a robust model that separates the positive and negative classes well.  There are two likely reasons for the failure.  One is that the particular algorithm with its hyperparameters is not suited to the kind of patterns in the data, and it is for this reason that we have tried several models with different class and focal weights.  A second possible reason is that there just isn't enough of a pattern in the dataset, and it is for this reason that we have tested the Easy, Medium, and Hard feature sets.  

As an example of this kind of model failure, consider the Random Forest Classifier on the Medium features, shown in Table \ref{RFC_Medium_400_Slices}.  Here the $p$ values are cut into bands with at least one hundred elements of each of the Neg and Pos classes; the $p$ values in the first column are the average of the minimum and maximum $p$ value in the band.  

The Precision values generally increase with $p$, and the table shows the end of the $p$ range; there are no values of Precision close to $2/3$, not because it doesn't theoretically exist, but the model is not robust enough to sufficiently separate the positive and negative classes.  If the governmental leaders decided to change to $\text{Precision} = 1/2$, we could choose $\theta = 0.240$, but of the crash persons with an automated notification from a cell phone who needed an ambulance, we would only be immediately dispatching ambulances to $\text{Recall} = 1\%$ of them.  Scaling that to a medium-sized city, the administrative overhead of the program might outweigh any benefit.  Given a finite budget, a responsible government needs to choose the most cost-effective programs, and an immediate dispatch recommendation system might not be one of them.  

We can see the problem also in a histogram of the model output, in Figure \ref{RFC_Medium_Zoom_Figure}, where we compare the right tails of the output of the Random Forest model on the Medium features with the Example model from Figure \ref{intro_ideal}.  The model output has a long tail to the right, and zooming in on that tail, we see that in none of those intervals are there more elements of the positive class than the negative class.  

Note that this range of $p$ in Table \ref{RFC_Medium_400_Slices} is also where $\text{Pos}/(\text{Neg} + \text{Pos})$ crosses 0.50, meaning that there is a 50\% marginal probability that the last ambulance sent is needed, so we would have the same problem for that metric, that the model would recommend immediately dispatching ambulances to only about 1\% of the automated notifications that needed an ambulance.  

Fortunately we do have models with better results on the Medium features set.  



\begin{table}
\caption{
	\normalsize\normalfont
	Metrics on Partial $p$ Output of Random Forest Classifier on the Medium Features with Minimum of 400 Elements of Each Class in Each Band.  Table accompanies \S\ref{Methods_Model_Failure}
}
\label{RFC_Medium_400_Slices}

{\normalsize
\normalfont
\begin{tabular}{lrrlrrrrlll}
\toprule
	\multicolumn{1}{c}{$p$} &     
	\multicolumn{1}{c}{Neg} &   
	\multicolumn{1}{c}{Pos} & 
	\multicolumn{1}{c}{$m$Prob} &     
	\multicolumn{1}{c}{TN} &      
	\multicolumn{1}{c}{FP} &      
	\multicolumn{1}{c}{FN} &      
	\multicolumn{1}{c}{TP} &   
	\multicolumn{1}{c}{Prec} &    
	\multicolumn{1}{c}{Rec} &    
	\multicolumn{1}{c}{FP/P} \\
\midrule
0.2386 & 415 & 401 & 0.491 & 603,883 & 1,727 & 106,283 & 1,673 & 0.492 & 0.016 & 0.016\cr
0.2405 & 460 & 400 & 0.465 & 604,343 & 1,267 & 106,683 & 1,273 & 0.501 & 0.012 & 0.0117\cr
0.2429 & 466 & 400 & 0.462 & 604,809 & 801 & 107,083 & 873 & 0.522 & 0.008 & 0.0074\cr
0.2465 & 400 & 421 & 0.513 & 605,209 & 401 & 107,504 & 452 & 0.53 & 0.004 & 0.0037\cr
0.2599 & 401 & 452 & 0.53 & 605,610 & 0 & 107,956 & 0 & nan & 0 & 0\cr 
\bottomrule
\end{tabular}
}
\end{table}

\FloatBarrier

%%%
\begin{figure}[h]



\noindent\begin{tabular}{@{\hspace{0pt}}p{2.3in} @{\hspace{-6pt}}p{4.6in} }
	\vskip 0pt
	
	\hfil {\normalfont\normalsize Full range of $p$}
	
	
	%% Creator: Matplotlib, PGF backend
%%
%% To include the figure in your LaTeX document, write
%%   \input{<filename>.pgf}
%%
%% Make sure the required packages are loaded in your preamble
%%   \usepackage{pgf}
%%
%% Also ensure that all the required font packages are loaded; for instance,
%% the lmodern package is sometimes necessary when using math font.
%%   \usepackage{lmodern}
%%
%% Figures using additional raster images can only be included by \input if
%% they are in the same directory as the main LaTeX file. For loading figures
%% from other directories you can use the `import` package
%%   \usepackage{import}
%%
%% and then include the figures with
%%   \import{<path to file>}{<filename>.pgf}
%%
%% Matplotlib used the following preamble
%%   
%%   \usepackage{fontspec}
%%   \makeatletter\@ifpackageloaded{underscore}{}{\usepackage[strings]{underscore}}\makeatother
%%
\begingroup%
\makeatletter%
\begin{pgfpicture}%
\pgfpathrectangle{\pgfpointorigin}{\pgfqpoint{2.130764in}{1.654444in}}%
\pgfusepath{use as bounding box, clip}%
\begin{pgfscope}%
\pgfsetbuttcap%
\pgfsetmiterjoin%
\definecolor{currentfill}{rgb}{1.000000,1.000000,1.000000}%
\pgfsetfillcolor{currentfill}%
\pgfsetlinewidth{0.000000pt}%
\definecolor{currentstroke}{rgb}{1.000000,1.000000,1.000000}%
\pgfsetstrokecolor{currentstroke}%
\pgfsetdash{}{0pt}%
\pgfpathmoveto{\pgfqpoint{0.000000in}{0.000000in}}%
\pgfpathlineto{\pgfqpoint{2.130764in}{0.000000in}}%
\pgfpathlineto{\pgfqpoint{2.130764in}{1.654444in}}%
\pgfpathlineto{\pgfqpoint{0.000000in}{1.654444in}}%
\pgfpathlineto{\pgfqpoint{0.000000in}{0.000000in}}%
\pgfpathclose%
\pgfusepath{fill}%
\end{pgfscope}%
\begin{pgfscope}%
\pgfsetbuttcap%
\pgfsetmiterjoin%
\definecolor{currentfill}{rgb}{1.000000,1.000000,1.000000}%
\pgfsetfillcolor{currentfill}%
\pgfsetlinewidth{0.000000pt}%
\definecolor{currentstroke}{rgb}{0.000000,0.000000,0.000000}%
\pgfsetstrokecolor{currentstroke}%
\pgfsetstrokeopacity{0.000000}%
\pgfsetdash{}{0pt}%
\pgfpathmoveto{\pgfqpoint{0.465000in}{0.449444in}}%
\pgfpathlineto{\pgfqpoint{2.015000in}{0.449444in}}%
\pgfpathlineto{\pgfqpoint{2.015000in}{1.604444in}}%
\pgfpathlineto{\pgfqpoint{0.465000in}{1.604444in}}%
\pgfpathlineto{\pgfqpoint{0.465000in}{0.449444in}}%
\pgfpathclose%
\pgfusepath{fill}%
\end{pgfscope}%
\begin{pgfscope}%
\pgfpathrectangle{\pgfqpoint{0.465000in}{0.449444in}}{\pgfqpoint{1.550000in}{1.155000in}}%
\pgfusepath{clip}%
\pgfsetbuttcap%
\pgfsetmiterjoin%
\pgfsetlinewidth{1.003750pt}%
\definecolor{currentstroke}{rgb}{0.000000,0.000000,0.000000}%
\pgfsetstrokecolor{currentstroke}%
\pgfsetdash{}{0pt}%
\pgfpathmoveto{\pgfqpoint{0.455000in}{0.449444in}}%
\pgfpathlineto{\pgfqpoint{0.500551in}{0.449444in}}%
\pgfpathlineto{\pgfqpoint{0.500551in}{0.449444in}}%
\pgfpathlineto{\pgfqpoint{0.455000in}{0.449444in}}%
\pgfusepath{stroke}%
\end{pgfscope}%
\begin{pgfscope}%
\pgfpathrectangle{\pgfqpoint{0.465000in}{0.449444in}}{\pgfqpoint{1.550000in}{1.155000in}}%
\pgfusepath{clip}%
\pgfsetbuttcap%
\pgfsetmiterjoin%
\pgfsetlinewidth{1.003750pt}%
\definecolor{currentstroke}{rgb}{0.000000,0.000000,0.000000}%
\pgfsetstrokecolor{currentstroke}%
\pgfsetdash{}{0pt}%
\pgfpathmoveto{\pgfqpoint{0.585872in}{0.449444in}}%
\pgfpathlineto{\pgfqpoint{0.642753in}{0.449444in}}%
\pgfpathlineto{\pgfqpoint{0.642753in}{0.987712in}}%
\pgfpathlineto{\pgfqpoint{0.585872in}{0.987712in}}%
\pgfpathlineto{\pgfqpoint{0.585872in}{0.449444in}}%
\pgfpathclose%
\pgfusepath{stroke}%
\end{pgfscope}%
\begin{pgfscope}%
\pgfpathrectangle{\pgfqpoint{0.465000in}{0.449444in}}{\pgfqpoint{1.550000in}{1.155000in}}%
\pgfusepath{clip}%
\pgfsetbuttcap%
\pgfsetmiterjoin%
\pgfsetlinewidth{1.003750pt}%
\definecolor{currentstroke}{rgb}{0.000000,0.000000,0.000000}%
\pgfsetstrokecolor{currentstroke}%
\pgfsetdash{}{0pt}%
\pgfpathmoveto{\pgfqpoint{0.728074in}{0.449444in}}%
\pgfpathlineto{\pgfqpoint{0.784954in}{0.449444in}}%
\pgfpathlineto{\pgfqpoint{0.784954in}{1.549444in}}%
\pgfpathlineto{\pgfqpoint{0.728074in}{1.549444in}}%
\pgfpathlineto{\pgfqpoint{0.728074in}{0.449444in}}%
\pgfpathclose%
\pgfusepath{stroke}%
\end{pgfscope}%
\begin{pgfscope}%
\pgfpathrectangle{\pgfqpoint{0.465000in}{0.449444in}}{\pgfqpoint{1.550000in}{1.155000in}}%
\pgfusepath{clip}%
\pgfsetbuttcap%
\pgfsetmiterjoin%
\pgfsetlinewidth{1.003750pt}%
\definecolor{currentstroke}{rgb}{0.000000,0.000000,0.000000}%
\pgfsetstrokecolor{currentstroke}%
\pgfsetdash{}{0pt}%
\pgfpathmoveto{\pgfqpoint{0.870276in}{0.449444in}}%
\pgfpathlineto{\pgfqpoint{0.927156in}{0.449444in}}%
\pgfpathlineto{\pgfqpoint{0.927156in}{1.361823in}}%
\pgfpathlineto{\pgfqpoint{0.870276in}{1.361823in}}%
\pgfpathlineto{\pgfqpoint{0.870276in}{0.449444in}}%
\pgfpathclose%
\pgfusepath{stroke}%
\end{pgfscope}%
\begin{pgfscope}%
\pgfpathrectangle{\pgfqpoint{0.465000in}{0.449444in}}{\pgfqpoint{1.550000in}{1.155000in}}%
\pgfusepath{clip}%
\pgfsetbuttcap%
\pgfsetmiterjoin%
\pgfsetlinewidth{1.003750pt}%
\definecolor{currentstroke}{rgb}{0.000000,0.000000,0.000000}%
\pgfsetstrokecolor{currentstroke}%
\pgfsetdash{}{0pt}%
\pgfpathmoveto{\pgfqpoint{1.012477in}{0.449444in}}%
\pgfpathlineto{\pgfqpoint{1.069358in}{0.449444in}}%
\pgfpathlineto{\pgfqpoint{1.069358in}{0.927699in}}%
\pgfpathlineto{\pgfqpoint{1.012477in}{0.927699in}}%
\pgfpathlineto{\pgfqpoint{1.012477in}{0.449444in}}%
\pgfpathclose%
\pgfusepath{stroke}%
\end{pgfscope}%
\begin{pgfscope}%
\pgfpathrectangle{\pgfqpoint{0.465000in}{0.449444in}}{\pgfqpoint{1.550000in}{1.155000in}}%
\pgfusepath{clip}%
\pgfsetbuttcap%
\pgfsetmiterjoin%
\pgfsetlinewidth{1.003750pt}%
\definecolor{currentstroke}{rgb}{0.000000,0.000000,0.000000}%
\pgfsetstrokecolor{currentstroke}%
\pgfsetdash{}{0pt}%
\pgfpathmoveto{\pgfqpoint{1.154679in}{0.449444in}}%
\pgfpathlineto{\pgfqpoint{1.211560in}{0.449444in}}%
\pgfpathlineto{\pgfqpoint{1.211560in}{0.551904in}}%
\pgfpathlineto{\pgfqpoint{1.154679in}{0.551904in}}%
\pgfpathlineto{\pgfqpoint{1.154679in}{0.449444in}}%
\pgfpathclose%
\pgfusepath{stroke}%
\end{pgfscope}%
\begin{pgfscope}%
\pgfpathrectangle{\pgfqpoint{0.465000in}{0.449444in}}{\pgfqpoint{1.550000in}{1.155000in}}%
\pgfusepath{clip}%
\pgfsetbuttcap%
\pgfsetmiterjoin%
\pgfsetlinewidth{1.003750pt}%
\definecolor{currentstroke}{rgb}{0.000000,0.000000,0.000000}%
\pgfsetstrokecolor{currentstroke}%
\pgfsetdash{}{0pt}%
\pgfpathmoveto{\pgfqpoint{1.296881in}{0.449444in}}%
\pgfpathlineto{\pgfqpoint{1.353762in}{0.449444in}}%
\pgfpathlineto{\pgfqpoint{1.353762in}{0.540809in}}%
\pgfpathlineto{\pgfqpoint{1.296881in}{0.540809in}}%
\pgfpathlineto{\pgfqpoint{1.296881in}{0.449444in}}%
\pgfpathclose%
\pgfusepath{stroke}%
\end{pgfscope}%
\begin{pgfscope}%
\pgfpathrectangle{\pgfqpoint{0.465000in}{0.449444in}}{\pgfqpoint{1.550000in}{1.155000in}}%
\pgfusepath{clip}%
\pgfsetbuttcap%
\pgfsetmiterjoin%
\pgfsetlinewidth{1.003750pt}%
\definecolor{currentstroke}{rgb}{0.000000,0.000000,0.000000}%
\pgfsetstrokecolor{currentstroke}%
\pgfsetdash{}{0pt}%
\pgfpathmoveto{\pgfqpoint{1.439083in}{0.449444in}}%
\pgfpathlineto{\pgfqpoint{1.495964in}{0.449444in}}%
\pgfpathlineto{\pgfqpoint{1.495964in}{0.521455in}}%
\pgfpathlineto{\pgfqpoint{1.439083in}{0.521455in}}%
\pgfpathlineto{\pgfqpoint{1.439083in}{0.449444in}}%
\pgfpathclose%
\pgfusepath{stroke}%
\end{pgfscope}%
\begin{pgfscope}%
\pgfpathrectangle{\pgfqpoint{0.465000in}{0.449444in}}{\pgfqpoint{1.550000in}{1.155000in}}%
\pgfusepath{clip}%
\pgfsetbuttcap%
\pgfsetmiterjoin%
\pgfsetlinewidth{1.003750pt}%
\definecolor{currentstroke}{rgb}{0.000000,0.000000,0.000000}%
\pgfsetstrokecolor{currentstroke}%
\pgfsetdash{}{0pt}%
\pgfpathmoveto{\pgfqpoint{1.581285in}{0.449444in}}%
\pgfpathlineto{\pgfqpoint{1.638165in}{0.449444in}}%
\pgfpathlineto{\pgfqpoint{1.638165in}{0.491155in}}%
\pgfpathlineto{\pgfqpoint{1.581285in}{0.491155in}}%
\pgfpathlineto{\pgfqpoint{1.581285in}{0.449444in}}%
\pgfpathclose%
\pgfusepath{stroke}%
\end{pgfscope}%
\begin{pgfscope}%
\pgfpathrectangle{\pgfqpoint{0.465000in}{0.449444in}}{\pgfqpoint{1.550000in}{1.155000in}}%
\pgfusepath{clip}%
\pgfsetbuttcap%
\pgfsetmiterjoin%
\pgfsetlinewidth{1.003750pt}%
\definecolor{currentstroke}{rgb}{0.000000,0.000000,0.000000}%
\pgfsetstrokecolor{currentstroke}%
\pgfsetdash{}{0pt}%
\pgfpathmoveto{\pgfqpoint{1.723487in}{0.449444in}}%
\pgfpathlineto{\pgfqpoint{1.780367in}{0.449444in}}%
\pgfpathlineto{\pgfqpoint{1.780367in}{0.465851in}}%
\pgfpathlineto{\pgfqpoint{1.723487in}{0.465851in}}%
\pgfpathlineto{\pgfqpoint{1.723487in}{0.449444in}}%
\pgfpathclose%
\pgfusepath{stroke}%
\end{pgfscope}%
\begin{pgfscope}%
\pgfpathrectangle{\pgfqpoint{0.465000in}{0.449444in}}{\pgfqpoint{1.550000in}{1.155000in}}%
\pgfusepath{clip}%
\pgfsetbuttcap%
\pgfsetmiterjoin%
\pgfsetlinewidth{1.003750pt}%
\definecolor{currentstroke}{rgb}{0.000000,0.000000,0.000000}%
\pgfsetstrokecolor{currentstroke}%
\pgfsetdash{}{0pt}%
\pgfpathmoveto{\pgfqpoint{1.865688in}{0.449444in}}%
\pgfpathlineto{\pgfqpoint{1.922569in}{0.449444in}}%
\pgfpathlineto{\pgfqpoint{1.922569in}{0.451211in}}%
\pgfpathlineto{\pgfqpoint{1.865688in}{0.451211in}}%
\pgfpathlineto{\pgfqpoint{1.865688in}{0.449444in}}%
\pgfpathclose%
\pgfusepath{stroke}%
\end{pgfscope}%
\begin{pgfscope}%
\pgfpathrectangle{\pgfqpoint{0.465000in}{0.449444in}}{\pgfqpoint{1.550000in}{1.155000in}}%
\pgfusepath{clip}%
\pgfsetbuttcap%
\pgfsetmiterjoin%
\definecolor{currentfill}{rgb}{0.000000,0.000000,0.000000}%
\pgfsetfillcolor{currentfill}%
\pgfsetlinewidth{0.000000pt}%
\definecolor{currentstroke}{rgb}{0.000000,0.000000,0.000000}%
\pgfsetstrokecolor{currentstroke}%
\pgfsetstrokeopacity{0.000000}%
\pgfsetdash{}{0pt}%
\pgfpathmoveto{\pgfqpoint{0.500551in}{0.449444in}}%
\pgfpathlineto{\pgfqpoint{0.557431in}{0.449444in}}%
\pgfpathlineto{\pgfqpoint{0.557431in}{0.449444in}}%
\pgfpathlineto{\pgfqpoint{0.500551in}{0.449444in}}%
\pgfpathlineto{\pgfqpoint{0.500551in}{0.449444in}}%
\pgfpathclose%
\pgfusepath{fill}%
\end{pgfscope}%
\begin{pgfscope}%
\pgfpathrectangle{\pgfqpoint{0.465000in}{0.449444in}}{\pgfqpoint{1.550000in}{1.155000in}}%
\pgfusepath{clip}%
\pgfsetbuttcap%
\pgfsetmiterjoin%
\definecolor{currentfill}{rgb}{0.000000,0.000000,0.000000}%
\pgfsetfillcolor{currentfill}%
\pgfsetlinewidth{0.000000pt}%
\definecolor{currentstroke}{rgb}{0.000000,0.000000,0.000000}%
\pgfsetstrokecolor{currentstroke}%
\pgfsetstrokeopacity{0.000000}%
\pgfsetdash{}{0pt}%
\pgfpathmoveto{\pgfqpoint{0.642753in}{0.449444in}}%
\pgfpathlineto{\pgfqpoint{0.699633in}{0.449444in}}%
\pgfpathlineto{\pgfqpoint{0.699633in}{0.480165in}}%
\pgfpathlineto{\pgfqpoint{0.642753in}{0.480165in}}%
\pgfpathlineto{\pgfqpoint{0.642753in}{0.449444in}}%
\pgfpathclose%
\pgfusepath{fill}%
\end{pgfscope}%
\begin{pgfscope}%
\pgfpathrectangle{\pgfqpoint{0.465000in}{0.449444in}}{\pgfqpoint{1.550000in}{1.155000in}}%
\pgfusepath{clip}%
\pgfsetbuttcap%
\pgfsetmiterjoin%
\definecolor{currentfill}{rgb}{0.000000,0.000000,0.000000}%
\pgfsetfillcolor{currentfill}%
\pgfsetlinewidth{0.000000pt}%
\definecolor{currentstroke}{rgb}{0.000000,0.000000,0.000000}%
\pgfsetstrokecolor{currentstroke}%
\pgfsetstrokeopacity{0.000000}%
\pgfsetdash{}{0pt}%
\pgfpathmoveto{\pgfqpoint{0.784954in}{0.449444in}}%
\pgfpathlineto{\pgfqpoint{0.841835in}{0.449444in}}%
\pgfpathlineto{\pgfqpoint{0.841835in}{0.569530in}}%
\pgfpathlineto{\pgfqpoint{0.784954in}{0.569530in}}%
\pgfpathlineto{\pgfqpoint{0.784954in}{0.449444in}}%
\pgfpathclose%
\pgfusepath{fill}%
\end{pgfscope}%
\begin{pgfscope}%
\pgfpathrectangle{\pgfqpoint{0.465000in}{0.449444in}}{\pgfqpoint{1.550000in}{1.155000in}}%
\pgfusepath{clip}%
\pgfsetbuttcap%
\pgfsetmiterjoin%
\definecolor{currentfill}{rgb}{0.000000,0.000000,0.000000}%
\pgfsetfillcolor{currentfill}%
\pgfsetlinewidth{0.000000pt}%
\definecolor{currentstroke}{rgb}{0.000000,0.000000,0.000000}%
\pgfsetstrokecolor{currentstroke}%
\pgfsetstrokeopacity{0.000000}%
\pgfsetdash{}{0pt}%
\pgfpathmoveto{\pgfqpoint{0.927156in}{0.449444in}}%
\pgfpathlineto{\pgfqpoint{0.984037in}{0.449444in}}%
\pgfpathlineto{\pgfqpoint{0.984037in}{0.610787in}}%
\pgfpathlineto{\pgfqpoint{0.927156in}{0.610787in}}%
\pgfpathlineto{\pgfqpoint{0.927156in}{0.449444in}}%
\pgfpathclose%
\pgfusepath{fill}%
\end{pgfscope}%
\begin{pgfscope}%
\pgfpathrectangle{\pgfqpoint{0.465000in}{0.449444in}}{\pgfqpoint{1.550000in}{1.155000in}}%
\pgfusepath{clip}%
\pgfsetbuttcap%
\pgfsetmiterjoin%
\definecolor{currentfill}{rgb}{0.000000,0.000000,0.000000}%
\pgfsetfillcolor{currentfill}%
\pgfsetlinewidth{0.000000pt}%
\definecolor{currentstroke}{rgb}{0.000000,0.000000,0.000000}%
\pgfsetstrokecolor{currentstroke}%
\pgfsetstrokeopacity{0.000000}%
\pgfsetdash{}{0pt}%
\pgfpathmoveto{\pgfqpoint{1.069358in}{0.449444in}}%
\pgfpathlineto{\pgfqpoint{1.126239in}{0.449444in}}%
\pgfpathlineto{\pgfqpoint{1.126239in}{0.573103in}}%
\pgfpathlineto{\pgfqpoint{1.069358in}{0.573103in}}%
\pgfpathlineto{\pgfqpoint{1.069358in}{0.449444in}}%
\pgfpathclose%
\pgfusepath{fill}%
\end{pgfscope}%
\begin{pgfscope}%
\pgfpathrectangle{\pgfqpoint{0.465000in}{0.449444in}}{\pgfqpoint{1.550000in}{1.155000in}}%
\pgfusepath{clip}%
\pgfsetbuttcap%
\pgfsetmiterjoin%
\definecolor{currentfill}{rgb}{0.000000,0.000000,0.000000}%
\pgfsetfillcolor{currentfill}%
\pgfsetlinewidth{0.000000pt}%
\definecolor{currentstroke}{rgb}{0.000000,0.000000,0.000000}%
\pgfsetstrokecolor{currentstroke}%
\pgfsetstrokeopacity{0.000000}%
\pgfsetdash{}{0pt}%
\pgfpathmoveto{\pgfqpoint{1.211560in}{0.449444in}}%
\pgfpathlineto{\pgfqpoint{1.268441in}{0.449444in}}%
\pgfpathlineto{\pgfqpoint{1.268441in}{0.482746in}}%
\pgfpathlineto{\pgfqpoint{1.211560in}{0.482746in}}%
\pgfpathlineto{\pgfqpoint{1.211560in}{0.449444in}}%
\pgfpathclose%
\pgfusepath{fill}%
\end{pgfscope}%
\begin{pgfscope}%
\pgfpathrectangle{\pgfqpoint{0.465000in}{0.449444in}}{\pgfqpoint{1.550000in}{1.155000in}}%
\pgfusepath{clip}%
\pgfsetbuttcap%
\pgfsetmiterjoin%
\definecolor{currentfill}{rgb}{0.000000,0.000000,0.000000}%
\pgfsetfillcolor{currentfill}%
\pgfsetlinewidth{0.000000pt}%
\definecolor{currentstroke}{rgb}{0.000000,0.000000,0.000000}%
\pgfsetstrokecolor{currentstroke}%
\pgfsetstrokeopacity{0.000000}%
\pgfsetdash{}{0pt}%
\pgfpathmoveto{\pgfqpoint{1.353762in}{0.449444in}}%
\pgfpathlineto{\pgfqpoint{1.410642in}{0.449444in}}%
\pgfpathlineto{\pgfqpoint{1.410642in}{0.488369in}}%
\pgfpathlineto{\pgfqpoint{1.353762in}{0.488369in}}%
\pgfpathlineto{\pgfqpoint{1.353762in}{0.449444in}}%
\pgfpathclose%
\pgfusepath{fill}%
\end{pgfscope}%
\begin{pgfscope}%
\pgfpathrectangle{\pgfqpoint{0.465000in}{0.449444in}}{\pgfqpoint{1.550000in}{1.155000in}}%
\pgfusepath{clip}%
\pgfsetbuttcap%
\pgfsetmiterjoin%
\definecolor{currentfill}{rgb}{0.000000,0.000000,0.000000}%
\pgfsetfillcolor{currentfill}%
\pgfsetlinewidth{0.000000pt}%
\definecolor{currentstroke}{rgb}{0.000000,0.000000,0.000000}%
\pgfsetstrokecolor{currentstroke}%
\pgfsetstrokeopacity{0.000000}%
\pgfsetdash{}{0pt}%
\pgfpathmoveto{\pgfqpoint{1.495964in}{0.449444in}}%
\pgfpathlineto{\pgfqpoint{1.552844in}{0.449444in}}%
\pgfpathlineto{\pgfqpoint{1.552844in}{0.490607in}}%
\pgfpathlineto{\pgfqpoint{1.495964in}{0.490607in}}%
\pgfpathlineto{\pgfqpoint{1.495964in}{0.449444in}}%
\pgfpathclose%
\pgfusepath{fill}%
\end{pgfscope}%
\begin{pgfscope}%
\pgfpathrectangle{\pgfqpoint{0.465000in}{0.449444in}}{\pgfqpoint{1.550000in}{1.155000in}}%
\pgfusepath{clip}%
\pgfsetbuttcap%
\pgfsetmiterjoin%
\definecolor{currentfill}{rgb}{0.000000,0.000000,0.000000}%
\pgfsetfillcolor{currentfill}%
\pgfsetlinewidth{0.000000pt}%
\definecolor{currentstroke}{rgb}{0.000000,0.000000,0.000000}%
\pgfsetstrokecolor{currentstroke}%
\pgfsetstrokeopacity{0.000000}%
\pgfsetdash{}{0pt}%
\pgfpathmoveto{\pgfqpoint{1.638165in}{0.449444in}}%
\pgfpathlineto{\pgfqpoint{1.695046in}{0.449444in}}%
\pgfpathlineto{\pgfqpoint{1.695046in}{0.481738in}}%
\pgfpathlineto{\pgfqpoint{1.638165in}{0.481738in}}%
\pgfpathlineto{\pgfqpoint{1.638165in}{0.449444in}}%
\pgfpathclose%
\pgfusepath{fill}%
\end{pgfscope}%
\begin{pgfscope}%
\pgfpathrectangle{\pgfqpoint{0.465000in}{0.449444in}}{\pgfqpoint{1.550000in}{1.155000in}}%
\pgfusepath{clip}%
\pgfsetbuttcap%
\pgfsetmiterjoin%
\definecolor{currentfill}{rgb}{0.000000,0.000000,0.000000}%
\pgfsetfillcolor{currentfill}%
\pgfsetlinewidth{0.000000pt}%
\definecolor{currentstroke}{rgb}{0.000000,0.000000,0.000000}%
\pgfsetstrokecolor{currentstroke}%
\pgfsetstrokeopacity{0.000000}%
\pgfsetdash{}{0pt}%
\pgfpathmoveto{\pgfqpoint{1.780367in}{0.449444in}}%
\pgfpathlineto{\pgfqpoint{1.837248in}{0.449444in}}%
\pgfpathlineto{\pgfqpoint{1.837248in}{0.464373in}}%
\pgfpathlineto{\pgfqpoint{1.780367in}{0.464373in}}%
\pgfpathlineto{\pgfqpoint{1.780367in}{0.449444in}}%
\pgfpathclose%
\pgfusepath{fill}%
\end{pgfscope}%
\begin{pgfscope}%
\pgfpathrectangle{\pgfqpoint{0.465000in}{0.449444in}}{\pgfqpoint{1.550000in}{1.155000in}}%
\pgfusepath{clip}%
\pgfsetbuttcap%
\pgfsetmiterjoin%
\definecolor{currentfill}{rgb}{0.000000,0.000000,0.000000}%
\pgfsetfillcolor{currentfill}%
\pgfsetlinewidth{0.000000pt}%
\definecolor{currentstroke}{rgb}{0.000000,0.000000,0.000000}%
\pgfsetstrokecolor{currentstroke}%
\pgfsetstrokeopacity{0.000000}%
\pgfsetdash{}{0pt}%
\pgfpathmoveto{\pgfqpoint{1.922569in}{0.449444in}}%
\pgfpathlineto{\pgfqpoint{1.979450in}{0.449444in}}%
\pgfpathlineto{\pgfqpoint{1.979450in}{0.451023in}}%
\pgfpathlineto{\pgfqpoint{1.922569in}{0.451023in}}%
\pgfpathlineto{\pgfqpoint{1.922569in}{0.449444in}}%
\pgfpathclose%
\pgfusepath{fill}%
\end{pgfscope}%
\begin{pgfscope}%
\pgfsetbuttcap%
\pgfsetroundjoin%
\definecolor{currentfill}{rgb}{0.000000,0.000000,0.000000}%
\pgfsetfillcolor{currentfill}%
\pgfsetlinewidth{0.803000pt}%
\definecolor{currentstroke}{rgb}{0.000000,0.000000,0.000000}%
\pgfsetstrokecolor{currentstroke}%
\pgfsetdash{}{0pt}%
\pgfsys@defobject{currentmarker}{\pgfqpoint{0.000000in}{-0.048611in}}{\pgfqpoint{0.000000in}{0.000000in}}{%
\pgfpathmoveto{\pgfqpoint{0.000000in}{0.000000in}}%
\pgfpathlineto{\pgfqpoint{0.000000in}{-0.048611in}}%
\pgfusepath{stroke,fill}%
}%
\begin{pgfscope}%
\pgfsys@transformshift{0.500551in}{0.449444in}%
\pgfsys@useobject{currentmarker}{}%
\end{pgfscope}%
\end{pgfscope}%
\begin{pgfscope}%
\definecolor{textcolor}{rgb}{0.000000,0.000000,0.000000}%
\pgfsetstrokecolor{textcolor}%
\pgfsetfillcolor{textcolor}%
\pgftext[x=0.500551in,y=0.352222in,,top]{\color{textcolor}\rmfamily\fontsize{10.000000}{12.000000}\selectfont 0.108}%
\end{pgfscope}%
\begin{pgfscope}%
\pgfsetbuttcap%
\pgfsetroundjoin%
\definecolor{currentfill}{rgb}{0.000000,0.000000,0.000000}%
\pgfsetfillcolor{currentfill}%
\pgfsetlinewidth{0.803000pt}%
\definecolor{currentstroke}{rgb}{0.000000,0.000000,0.000000}%
\pgfsetstrokecolor{currentstroke}%
\pgfsetdash{}{0pt}%
\pgfsys@defobject{currentmarker}{\pgfqpoint{0.000000in}{-0.048611in}}{\pgfqpoint{0.000000in}{0.000000in}}{%
\pgfpathmoveto{\pgfqpoint{0.000000in}{0.000000in}}%
\pgfpathlineto{\pgfqpoint{0.000000in}{-0.048611in}}%
\pgfusepath{stroke,fill}%
}%
\begin{pgfscope}%
\pgfsys@transformshift{1.211560in}{0.449444in}%
\pgfsys@useobject{currentmarker}{}%
\end{pgfscope}%
\end{pgfscope}%
\begin{pgfscope}%
\definecolor{textcolor}{rgb}{0.000000,0.000000,0.000000}%
\pgfsetstrokecolor{textcolor}%
\pgfsetfillcolor{textcolor}%
\pgftext[x=1.211560in,y=0.352222in,,top]{\color{textcolor}\rmfamily\fontsize{10.000000}{12.000000}\selectfont 0.196}%
\end{pgfscope}%
\begin{pgfscope}%
\pgfsetbuttcap%
\pgfsetroundjoin%
\definecolor{currentfill}{rgb}{0.000000,0.000000,0.000000}%
\pgfsetfillcolor{currentfill}%
\pgfsetlinewidth{0.803000pt}%
\definecolor{currentstroke}{rgb}{0.000000,0.000000,0.000000}%
\pgfsetstrokecolor{currentstroke}%
\pgfsetdash{}{0pt}%
\pgfsys@defobject{currentmarker}{\pgfqpoint{0.000000in}{-0.048611in}}{\pgfqpoint{0.000000in}{0.000000in}}{%
\pgfpathmoveto{\pgfqpoint{0.000000in}{0.000000in}}%
\pgfpathlineto{\pgfqpoint{0.000000in}{-0.048611in}}%
\pgfusepath{stroke,fill}%
}%
\begin{pgfscope}%
\pgfsys@transformshift{1.922569in}{0.449444in}%
\pgfsys@useobject{currentmarker}{}%
\end{pgfscope}%
\end{pgfscope}%
\begin{pgfscope}%
\definecolor{textcolor}{rgb}{0.000000,0.000000,0.000000}%
\pgfsetstrokecolor{textcolor}%
\pgfsetfillcolor{textcolor}%
\pgftext[x=1.922569in,y=0.352222in,,top]{\color{textcolor}\rmfamily\fontsize{10.000000}{12.000000}\selectfont 0.285}%
\end{pgfscope}%
\begin{pgfscope}%
\definecolor{textcolor}{rgb}{0.000000,0.000000,0.000000}%
\pgfsetstrokecolor{textcolor}%
\pgfsetfillcolor{textcolor}%
\pgftext[x=1.240000in,y=0.173333in,,top]{\color{textcolor}\rmfamily\fontsize{10.000000}{12.000000}\selectfont \(\displaystyle p\)}%
\end{pgfscope}%
\begin{pgfscope}%
\pgfsetbuttcap%
\pgfsetroundjoin%
\definecolor{currentfill}{rgb}{0.000000,0.000000,0.000000}%
\pgfsetfillcolor{currentfill}%
\pgfsetlinewidth{0.803000pt}%
\definecolor{currentstroke}{rgb}{0.000000,0.000000,0.000000}%
\pgfsetstrokecolor{currentstroke}%
\pgfsetdash{}{0pt}%
\pgfsys@defobject{currentmarker}{\pgfqpoint{-0.048611in}{0.000000in}}{\pgfqpoint{-0.000000in}{0.000000in}}{%
\pgfpathmoveto{\pgfqpoint{-0.000000in}{0.000000in}}%
\pgfpathlineto{\pgfqpoint{-0.048611in}{0.000000in}}%
\pgfusepath{stroke,fill}%
}%
\begin{pgfscope}%
\pgfsys@transformshift{0.465000in}{0.449444in}%
\pgfsys@useobject{currentmarker}{}%
\end{pgfscope}%
\end{pgfscope}%
\begin{pgfscope}%
\definecolor{textcolor}{rgb}{0.000000,0.000000,0.000000}%
\pgfsetstrokecolor{textcolor}%
\pgfsetfillcolor{textcolor}%
\pgftext[x=0.298333in, y=0.401250in, left, base]{\color{textcolor}\rmfamily\fontsize{10.000000}{12.000000}\selectfont \(\displaystyle {0}\)}%
\end{pgfscope}%
\begin{pgfscope}%
\pgfsetbuttcap%
\pgfsetroundjoin%
\definecolor{currentfill}{rgb}{0.000000,0.000000,0.000000}%
\pgfsetfillcolor{currentfill}%
\pgfsetlinewidth{0.803000pt}%
\definecolor{currentstroke}{rgb}{0.000000,0.000000,0.000000}%
\pgfsetstrokecolor{currentstroke}%
\pgfsetdash{}{0pt}%
\pgfsys@defobject{currentmarker}{\pgfqpoint{-0.048611in}{0.000000in}}{\pgfqpoint{-0.000000in}{0.000000in}}{%
\pgfpathmoveto{\pgfqpoint{-0.000000in}{0.000000in}}%
\pgfpathlineto{\pgfqpoint{-0.048611in}{0.000000in}}%
\pgfusepath{stroke,fill}%
}%
\begin{pgfscope}%
\pgfsys@transformshift{0.465000in}{0.844708in}%
\pgfsys@useobject{currentmarker}{}%
\end{pgfscope}%
\end{pgfscope}%
\begin{pgfscope}%
\definecolor{textcolor}{rgb}{0.000000,0.000000,0.000000}%
\pgfsetstrokecolor{textcolor}%
\pgfsetfillcolor{textcolor}%
\pgftext[x=0.228889in, y=0.796513in, left, base]{\color{textcolor}\rmfamily\fontsize{10.000000}{12.000000}\selectfont \(\displaystyle {10}\)}%
\end{pgfscope}%
\begin{pgfscope}%
\pgfsetbuttcap%
\pgfsetroundjoin%
\definecolor{currentfill}{rgb}{0.000000,0.000000,0.000000}%
\pgfsetfillcolor{currentfill}%
\pgfsetlinewidth{0.803000pt}%
\definecolor{currentstroke}{rgb}{0.000000,0.000000,0.000000}%
\pgfsetstrokecolor{currentstroke}%
\pgfsetdash{}{0pt}%
\pgfsys@defobject{currentmarker}{\pgfqpoint{-0.048611in}{0.000000in}}{\pgfqpoint{-0.000000in}{0.000000in}}{%
\pgfpathmoveto{\pgfqpoint{-0.000000in}{0.000000in}}%
\pgfpathlineto{\pgfqpoint{-0.048611in}{0.000000in}}%
\pgfusepath{stroke,fill}%
}%
\begin{pgfscope}%
\pgfsys@transformshift{0.465000in}{1.239972in}%
\pgfsys@useobject{currentmarker}{}%
\end{pgfscope}%
\end{pgfscope}%
\begin{pgfscope}%
\definecolor{textcolor}{rgb}{0.000000,0.000000,0.000000}%
\pgfsetstrokecolor{textcolor}%
\pgfsetfillcolor{textcolor}%
\pgftext[x=0.228889in, y=1.191777in, left, base]{\color{textcolor}\rmfamily\fontsize{10.000000}{12.000000}\selectfont \(\displaystyle {20}\)}%
\end{pgfscope}%
\begin{pgfscope}%
\definecolor{textcolor}{rgb}{0.000000,0.000000,0.000000}%
\pgfsetstrokecolor{textcolor}%
\pgfsetfillcolor{textcolor}%
\pgftext[x=0.173333in,y=1.026944in,,bottom,rotate=90.000000]{\color{textcolor}\rmfamily\fontsize{10.000000}{12.000000}\selectfont Percent of Data Set}%
\end{pgfscope}%
\begin{pgfscope}%
\pgfsetrectcap%
\pgfsetmiterjoin%
\pgfsetlinewidth{0.803000pt}%
\definecolor{currentstroke}{rgb}{0.000000,0.000000,0.000000}%
\pgfsetstrokecolor{currentstroke}%
\pgfsetdash{}{0pt}%
\pgfpathmoveto{\pgfqpoint{0.465000in}{0.449444in}}%
\pgfpathlineto{\pgfqpoint{0.465000in}{1.604444in}}%
\pgfusepath{stroke}%
\end{pgfscope}%
\begin{pgfscope}%
\pgfsetrectcap%
\pgfsetmiterjoin%
\pgfsetlinewidth{0.803000pt}%
\definecolor{currentstroke}{rgb}{0.000000,0.000000,0.000000}%
\pgfsetstrokecolor{currentstroke}%
\pgfsetdash{}{0pt}%
\pgfpathmoveto{\pgfqpoint{2.015000in}{0.449444in}}%
\pgfpathlineto{\pgfqpoint{2.015000in}{1.604444in}}%
\pgfusepath{stroke}%
\end{pgfscope}%
\begin{pgfscope}%
\pgfsetrectcap%
\pgfsetmiterjoin%
\pgfsetlinewidth{0.803000pt}%
\definecolor{currentstroke}{rgb}{0.000000,0.000000,0.000000}%
\pgfsetstrokecolor{currentstroke}%
\pgfsetdash{}{0pt}%
\pgfpathmoveto{\pgfqpoint{0.465000in}{0.449444in}}%
\pgfpathlineto{\pgfqpoint{2.015000in}{0.449444in}}%
\pgfusepath{stroke}%
\end{pgfscope}%
\begin{pgfscope}%
\pgfsetrectcap%
\pgfsetmiterjoin%
\pgfsetlinewidth{0.803000pt}%
\definecolor{currentstroke}{rgb}{0.000000,0.000000,0.000000}%
\pgfsetstrokecolor{currentstroke}%
\pgfsetdash{}{0pt}%
\pgfpathmoveto{\pgfqpoint{0.465000in}{1.604444in}}%
\pgfpathlineto{\pgfqpoint{2.015000in}{1.604444in}}%
\pgfusepath{stroke}%
\end{pgfscope}%
\begin{pgfscope}%
\pgfsetbuttcap%
\pgfsetmiterjoin%
\definecolor{currentfill}{rgb}{1.000000,1.000000,1.000000}%
\pgfsetfillcolor{currentfill}%
\pgfsetfillopacity{0.800000}%
\pgfsetlinewidth{1.003750pt}%
\definecolor{currentstroke}{rgb}{0.800000,0.800000,0.800000}%
\pgfsetstrokecolor{currentstroke}%
\pgfsetstrokeopacity{0.800000}%
\pgfsetdash{}{0pt}%
\pgfpathmoveto{\pgfqpoint{1.238056in}{1.104445in}}%
\pgfpathlineto{\pgfqpoint{1.917778in}{1.104445in}}%
\pgfpathquadraticcurveto{\pgfqpoint{1.945556in}{1.104445in}}{\pgfqpoint{1.945556in}{1.132222in}}%
\pgfpathlineto{\pgfqpoint{1.945556in}{1.507222in}}%
\pgfpathquadraticcurveto{\pgfqpoint{1.945556in}{1.535000in}}{\pgfqpoint{1.917778in}{1.535000in}}%
\pgfpathlineto{\pgfqpoint{1.238056in}{1.535000in}}%
\pgfpathquadraticcurveto{\pgfqpoint{1.210278in}{1.535000in}}{\pgfqpoint{1.210278in}{1.507222in}}%
\pgfpathlineto{\pgfqpoint{1.210278in}{1.132222in}}%
\pgfpathquadraticcurveto{\pgfqpoint{1.210278in}{1.104445in}}{\pgfqpoint{1.238056in}{1.104445in}}%
\pgfpathlineto{\pgfqpoint{1.238056in}{1.104445in}}%
\pgfpathclose%
\pgfusepath{stroke,fill}%
\end{pgfscope}%
\begin{pgfscope}%
\pgfsetbuttcap%
\pgfsetmiterjoin%
\pgfsetlinewidth{1.003750pt}%
\definecolor{currentstroke}{rgb}{0.000000,0.000000,0.000000}%
\pgfsetstrokecolor{currentstroke}%
\pgfsetdash{}{0pt}%
\pgfpathmoveto{\pgfqpoint{1.265834in}{1.382222in}}%
\pgfpathlineto{\pgfqpoint{1.543611in}{1.382222in}}%
\pgfpathlineto{\pgfqpoint{1.543611in}{1.479444in}}%
\pgfpathlineto{\pgfqpoint{1.265834in}{1.479444in}}%
\pgfpathlineto{\pgfqpoint{1.265834in}{1.382222in}}%
\pgfpathclose%
\pgfusepath{stroke}%
\end{pgfscope}%
\begin{pgfscope}%
\definecolor{textcolor}{rgb}{0.000000,0.000000,0.000000}%
\pgfsetstrokecolor{textcolor}%
\pgfsetfillcolor{textcolor}%
\pgftext[x=1.654722in,y=1.382222in,left,base]{\color{textcolor}\rmfamily\fontsize{10.000000}{12.000000}\selectfont Neg}%
\end{pgfscope}%
\begin{pgfscope}%
\pgfsetbuttcap%
\pgfsetmiterjoin%
\definecolor{currentfill}{rgb}{0.000000,0.000000,0.000000}%
\pgfsetfillcolor{currentfill}%
\pgfsetlinewidth{0.000000pt}%
\definecolor{currentstroke}{rgb}{0.000000,0.000000,0.000000}%
\pgfsetstrokecolor{currentstroke}%
\pgfsetstrokeopacity{0.000000}%
\pgfsetdash{}{0pt}%
\pgfpathmoveto{\pgfqpoint{1.265834in}{1.186944in}}%
\pgfpathlineto{\pgfqpoint{1.543611in}{1.186944in}}%
\pgfpathlineto{\pgfqpoint{1.543611in}{1.284167in}}%
\pgfpathlineto{\pgfqpoint{1.265834in}{1.284167in}}%
\pgfpathlineto{\pgfqpoint{1.265834in}{1.186944in}}%
\pgfpathclose%
\pgfusepath{fill}%
\end{pgfscope}%
\begin{pgfscope}%
\definecolor{textcolor}{rgb}{0.000000,0.000000,0.000000}%
\pgfsetstrokecolor{textcolor}%
\pgfsetfillcolor{textcolor}%
\pgftext[x=1.654722in,y=1.186944in,left,base]{\color{textcolor}\rmfamily\fontsize{10.000000}{12.000000}\selectfont Pos}%
\end{pgfscope}%
\end{pgfpicture}%
\makeatother%
\endgroup%
	
&
	\vskip 0pt
	
	\hfil {\normalfont\normalsize Right tail of $p$ for the Random Forest Model}
		
	%% Creator: Matplotlib, PGF backend
%%
%% To include the figure in your LaTeX document, write
%%   \input{<filename>.pgf}
%%
%% Make sure the required packages are loaded in your preamble
%%   \usepackage{pgf}
%%
%% Also ensure that all the required font packages are loaded; for instance,
%% the lmodern package is sometimes necessary when using math font.
%%   \usepackage{lmodern}
%%
%% Figures using additional raster images can only be included by \input if
%% they are in the same directory as the main LaTeX file. For loading figures
%% from other directories you can use the `import` package
%%   \usepackage{import}
%%
%% and then include the figures with
%%   \import{<path to file>}{<filename>.pgf}
%%
%% Matplotlib used the following preamble
%%   
%%   \usepackage{fontspec}
%%   \makeatletter\@ifpackageloaded{underscore}{}{\usepackage[strings]{underscore}}\makeatother
%%
\begingroup%
\makeatletter%
\begin{pgfpicture}%
\pgfpathrectangle{\pgfpointorigin}{\pgfqpoint{4.160257in}{1.654444in}}%
\pgfusepath{use as bounding box, clip}%
\begin{pgfscope}%
\pgfsetbuttcap%
\pgfsetmiterjoin%
\definecolor{currentfill}{rgb}{1.000000,1.000000,1.000000}%
\pgfsetfillcolor{currentfill}%
\pgfsetlinewidth{0.000000pt}%
\definecolor{currentstroke}{rgb}{1.000000,1.000000,1.000000}%
\pgfsetstrokecolor{currentstroke}%
\pgfsetdash{}{0pt}%
\pgfpathmoveto{\pgfqpoint{0.000000in}{0.000000in}}%
\pgfpathlineto{\pgfqpoint{4.160257in}{0.000000in}}%
\pgfpathlineto{\pgfqpoint{4.160257in}{1.654444in}}%
\pgfpathlineto{\pgfqpoint{0.000000in}{1.654444in}}%
\pgfpathlineto{\pgfqpoint{0.000000in}{0.000000in}}%
\pgfpathclose%
\pgfusepath{fill}%
\end{pgfscope}%
\begin{pgfscope}%
\pgfsetbuttcap%
\pgfsetmiterjoin%
\definecolor{currentfill}{rgb}{1.000000,1.000000,1.000000}%
\pgfsetfillcolor{currentfill}%
\pgfsetlinewidth{0.000000pt}%
\definecolor{currentstroke}{rgb}{0.000000,0.000000,0.000000}%
\pgfsetstrokecolor{currentstroke}%
\pgfsetstrokeopacity{0.000000}%
\pgfsetdash{}{0pt}%
\pgfpathmoveto{\pgfqpoint{0.573025in}{0.449444in}}%
\pgfpathlineto{\pgfqpoint{4.060525in}{0.449444in}}%
\pgfpathlineto{\pgfqpoint{4.060525in}{1.604444in}}%
\pgfpathlineto{\pgfqpoint{0.573025in}{1.604444in}}%
\pgfpathlineto{\pgfqpoint{0.573025in}{0.449444in}}%
\pgfpathclose%
\pgfusepath{fill}%
\end{pgfscope}%
\begin{pgfscope}%
\pgfpathrectangle{\pgfqpoint{0.573025in}{0.449444in}}{\pgfqpoint{3.487500in}{1.155000in}}%
\pgfusepath{clip}%
\pgfsetbuttcap%
\pgfsetmiterjoin%
\pgfsetlinewidth{1.003750pt}%
\definecolor{currentstroke}{rgb}{0.000000,0.000000,0.000000}%
\pgfsetstrokecolor{currentstroke}%
\pgfsetdash{}{0pt}%
\pgfpathmoveto{\pgfqpoint{0.563025in}{0.449444in}}%
\pgfpathlineto{\pgfqpoint{0.614742in}{0.449444in}}%
\pgfpathlineto{\pgfqpoint{0.614742in}{0.449444in}}%
\pgfpathlineto{\pgfqpoint{0.563025in}{0.449444in}}%
\pgfusepath{stroke}%
\end{pgfscope}%
\begin{pgfscope}%
\pgfpathrectangle{\pgfqpoint{0.573025in}{0.449444in}}{\pgfqpoint{3.487500in}{1.155000in}}%
\pgfusepath{clip}%
\pgfsetbuttcap%
\pgfsetmiterjoin%
\pgfsetlinewidth{1.003750pt}%
\definecolor{currentstroke}{rgb}{0.000000,0.000000,0.000000}%
\pgfsetstrokecolor{currentstroke}%
\pgfsetdash{}{0pt}%
\pgfpathmoveto{\pgfqpoint{0.714861in}{0.449444in}}%
\pgfpathlineto{\pgfqpoint{0.781608in}{0.449444in}}%
\pgfpathlineto{\pgfqpoint{0.781608in}{1.138177in}}%
\pgfpathlineto{\pgfqpoint{0.714861in}{1.138177in}}%
\pgfpathlineto{\pgfqpoint{0.714861in}{0.449444in}}%
\pgfpathclose%
\pgfusepath{stroke}%
\end{pgfscope}%
\begin{pgfscope}%
\pgfpathrectangle{\pgfqpoint{0.573025in}{0.449444in}}{\pgfqpoint{3.487500in}{1.155000in}}%
\pgfusepath{clip}%
\pgfsetbuttcap%
\pgfsetmiterjoin%
\pgfsetlinewidth{1.003750pt}%
\definecolor{currentstroke}{rgb}{0.000000,0.000000,0.000000}%
\pgfsetstrokecolor{currentstroke}%
\pgfsetdash{}{0pt}%
\pgfpathmoveto{\pgfqpoint{0.881727in}{0.449444in}}%
\pgfpathlineto{\pgfqpoint{0.948474in}{0.449444in}}%
\pgfpathlineto{\pgfqpoint{0.948474in}{1.549444in}}%
\pgfpathlineto{\pgfqpoint{0.881727in}{1.549444in}}%
\pgfpathlineto{\pgfqpoint{0.881727in}{0.449444in}}%
\pgfpathclose%
\pgfusepath{stroke}%
\end{pgfscope}%
\begin{pgfscope}%
\pgfpathrectangle{\pgfqpoint{0.573025in}{0.449444in}}{\pgfqpoint{3.487500in}{1.155000in}}%
\pgfusepath{clip}%
\pgfsetbuttcap%
\pgfsetmiterjoin%
\pgfsetlinewidth{1.003750pt}%
\definecolor{currentstroke}{rgb}{0.000000,0.000000,0.000000}%
\pgfsetstrokecolor{currentstroke}%
\pgfsetdash{}{0pt}%
\pgfpathmoveto{\pgfqpoint{1.048593in}{0.449444in}}%
\pgfpathlineto{\pgfqpoint{1.115340in}{0.449444in}}%
\pgfpathlineto{\pgfqpoint{1.115340in}{1.312777in}}%
\pgfpathlineto{\pgfqpoint{1.048593in}{1.312777in}}%
\pgfpathlineto{\pgfqpoint{1.048593in}{0.449444in}}%
\pgfpathclose%
\pgfusepath{stroke}%
\end{pgfscope}%
\begin{pgfscope}%
\pgfpathrectangle{\pgfqpoint{0.573025in}{0.449444in}}{\pgfqpoint{3.487500in}{1.155000in}}%
\pgfusepath{clip}%
\pgfsetbuttcap%
\pgfsetmiterjoin%
\pgfsetlinewidth{1.003750pt}%
\definecolor{currentstroke}{rgb}{0.000000,0.000000,0.000000}%
\pgfsetstrokecolor{currentstroke}%
\pgfsetdash{}{0pt}%
\pgfpathmoveto{\pgfqpoint{1.215459in}{0.449444in}}%
\pgfpathlineto{\pgfqpoint{1.282206in}{0.449444in}}%
\pgfpathlineto{\pgfqpoint{1.282206in}{0.986586in}}%
\pgfpathlineto{\pgfqpoint{1.215459in}{0.986586in}}%
\pgfpathlineto{\pgfqpoint{1.215459in}{0.449444in}}%
\pgfpathclose%
\pgfusepath{stroke}%
\end{pgfscope}%
\begin{pgfscope}%
\pgfpathrectangle{\pgfqpoint{0.573025in}{0.449444in}}{\pgfqpoint{3.487500in}{1.155000in}}%
\pgfusepath{clip}%
\pgfsetbuttcap%
\pgfsetmiterjoin%
\pgfsetlinewidth{1.003750pt}%
\definecolor{currentstroke}{rgb}{0.000000,0.000000,0.000000}%
\pgfsetstrokecolor{currentstroke}%
\pgfsetdash{}{0pt}%
\pgfpathmoveto{\pgfqpoint{1.382326in}{0.449444in}}%
\pgfpathlineto{\pgfqpoint{1.449072in}{0.449444in}}%
\pgfpathlineto{\pgfqpoint{1.449072in}{1.147265in}}%
\pgfpathlineto{\pgfqpoint{1.382326in}{1.147265in}}%
\pgfpathlineto{\pgfqpoint{1.382326in}{0.449444in}}%
\pgfpathclose%
\pgfusepath{stroke}%
\end{pgfscope}%
\begin{pgfscope}%
\pgfpathrectangle{\pgfqpoint{0.573025in}{0.449444in}}{\pgfqpoint{3.487500in}{1.155000in}}%
\pgfusepath{clip}%
\pgfsetbuttcap%
\pgfsetmiterjoin%
\pgfsetlinewidth{1.003750pt}%
\definecolor{currentstroke}{rgb}{0.000000,0.000000,0.000000}%
\pgfsetstrokecolor{currentstroke}%
\pgfsetdash{}{0pt}%
\pgfpathmoveto{\pgfqpoint{1.549192in}{0.449444in}}%
\pgfpathlineto{\pgfqpoint{1.615938in}{0.449444in}}%
\pgfpathlineto{\pgfqpoint{1.615938in}{1.084231in}}%
\pgfpathlineto{\pgfqpoint{1.549192in}{1.084231in}}%
\pgfpathlineto{\pgfqpoint{1.549192in}{0.449444in}}%
\pgfpathclose%
\pgfusepath{stroke}%
\end{pgfscope}%
\begin{pgfscope}%
\pgfpathrectangle{\pgfqpoint{0.573025in}{0.449444in}}{\pgfqpoint{3.487500in}{1.155000in}}%
\pgfusepath{clip}%
\pgfsetbuttcap%
\pgfsetmiterjoin%
\pgfsetlinewidth{1.003750pt}%
\definecolor{currentstroke}{rgb}{0.000000,0.000000,0.000000}%
\pgfsetstrokecolor{currentstroke}%
\pgfsetdash{}{0pt}%
\pgfpathmoveto{\pgfqpoint{1.716058in}{0.449444in}}%
\pgfpathlineto{\pgfqpoint{1.782804in}{0.449444in}}%
\pgfpathlineto{\pgfqpoint{1.782804in}{1.004761in}}%
\pgfpathlineto{\pgfqpoint{1.716058in}{1.004761in}}%
\pgfpathlineto{\pgfqpoint{1.716058in}{0.449444in}}%
\pgfpathclose%
\pgfusepath{stroke}%
\end{pgfscope}%
\begin{pgfscope}%
\pgfpathrectangle{\pgfqpoint{0.573025in}{0.449444in}}{\pgfqpoint{3.487500in}{1.155000in}}%
\pgfusepath{clip}%
\pgfsetbuttcap%
\pgfsetmiterjoin%
\pgfsetlinewidth{1.003750pt}%
\definecolor{currentstroke}{rgb}{0.000000,0.000000,0.000000}%
\pgfsetstrokecolor{currentstroke}%
\pgfsetdash{}{0pt}%
\pgfpathmoveto{\pgfqpoint{1.882924in}{0.449444in}}%
\pgfpathlineto{\pgfqpoint{1.949670in}{0.449444in}}%
\pgfpathlineto{\pgfqpoint{1.949670in}{1.075143in}}%
\pgfpathlineto{\pgfqpoint{1.882924in}{1.075143in}}%
\pgfpathlineto{\pgfqpoint{1.882924in}{0.449444in}}%
\pgfpathclose%
\pgfusepath{stroke}%
\end{pgfscope}%
\begin{pgfscope}%
\pgfpathrectangle{\pgfqpoint{0.573025in}{0.449444in}}{\pgfqpoint{3.487500in}{1.155000in}}%
\pgfusepath{clip}%
\pgfsetbuttcap%
\pgfsetmiterjoin%
\pgfsetlinewidth{1.003750pt}%
\definecolor{currentstroke}{rgb}{0.000000,0.000000,0.000000}%
\pgfsetstrokecolor{currentstroke}%
\pgfsetdash{}{0pt}%
\pgfpathmoveto{\pgfqpoint{2.049790in}{0.449444in}}%
\pgfpathlineto{\pgfqpoint{2.116536in}{0.449444in}}%
\pgfpathlineto{\pgfqpoint{2.116536in}{0.934380in}}%
\pgfpathlineto{\pgfqpoint{2.049790in}{0.934380in}}%
\pgfpathlineto{\pgfqpoint{2.049790in}{0.449444in}}%
\pgfpathclose%
\pgfusepath{stroke}%
\end{pgfscope}%
\begin{pgfscope}%
\pgfpathrectangle{\pgfqpoint{0.573025in}{0.449444in}}{\pgfqpoint{3.487500in}{1.155000in}}%
\pgfusepath{clip}%
\pgfsetbuttcap%
\pgfsetmiterjoin%
\pgfsetlinewidth{1.003750pt}%
\definecolor{currentstroke}{rgb}{0.000000,0.000000,0.000000}%
\pgfsetstrokecolor{currentstroke}%
\pgfsetdash{}{0pt}%
\pgfpathmoveto{\pgfqpoint{2.216656in}{0.449444in}}%
\pgfpathlineto{\pgfqpoint{2.283402in}{0.449444in}}%
\pgfpathlineto{\pgfqpoint{2.283402in}{0.825908in}}%
\pgfpathlineto{\pgfqpoint{2.216656in}{0.825908in}}%
\pgfpathlineto{\pgfqpoint{2.216656in}{0.449444in}}%
\pgfpathclose%
\pgfusepath{stroke}%
\end{pgfscope}%
\begin{pgfscope}%
\pgfpathrectangle{\pgfqpoint{0.573025in}{0.449444in}}{\pgfqpoint{3.487500in}{1.155000in}}%
\pgfusepath{clip}%
\pgfsetbuttcap%
\pgfsetmiterjoin%
\pgfsetlinewidth{1.003750pt}%
\definecolor{currentstroke}{rgb}{0.000000,0.000000,0.000000}%
\pgfsetstrokecolor{currentstroke}%
\pgfsetdash{}{0pt}%
\pgfpathmoveto{\pgfqpoint{2.383522in}{0.449444in}}%
\pgfpathlineto{\pgfqpoint{2.450268in}{0.449444in}}%
\pgfpathlineto{\pgfqpoint{2.450268in}{0.793810in}}%
\pgfpathlineto{\pgfqpoint{2.383522in}{0.793810in}}%
\pgfpathlineto{\pgfqpoint{2.383522in}{0.449444in}}%
\pgfpathclose%
\pgfusepath{stroke}%
\end{pgfscope}%
\begin{pgfscope}%
\pgfpathrectangle{\pgfqpoint{0.573025in}{0.449444in}}{\pgfqpoint{3.487500in}{1.155000in}}%
\pgfusepath{clip}%
\pgfsetbuttcap%
\pgfsetmiterjoin%
\pgfsetlinewidth{1.003750pt}%
\definecolor{currentstroke}{rgb}{0.000000,0.000000,0.000000}%
\pgfsetstrokecolor{currentstroke}%
\pgfsetdash{}{0pt}%
\pgfpathmoveto{\pgfqpoint{2.550388in}{0.449444in}}%
\pgfpathlineto{\pgfqpoint{2.617134in}{0.449444in}}%
\pgfpathlineto{\pgfqpoint{2.617134in}{0.699646in}}%
\pgfpathlineto{\pgfqpoint{2.550388in}{0.699646in}}%
\pgfpathlineto{\pgfqpoint{2.550388in}{0.449444in}}%
\pgfpathclose%
\pgfusepath{stroke}%
\end{pgfscope}%
\begin{pgfscope}%
\pgfpathrectangle{\pgfqpoint{0.573025in}{0.449444in}}{\pgfqpoint{3.487500in}{1.155000in}}%
\pgfusepath{clip}%
\pgfsetbuttcap%
\pgfsetmiterjoin%
\pgfsetlinewidth{1.003750pt}%
\definecolor{currentstroke}{rgb}{0.000000,0.000000,0.000000}%
\pgfsetstrokecolor{currentstroke}%
\pgfsetdash{}{0pt}%
\pgfpathmoveto{\pgfqpoint{2.717254in}{0.449444in}}%
\pgfpathlineto{\pgfqpoint{2.784000in}{0.449444in}}%
\pgfpathlineto{\pgfqpoint{2.784000in}{0.674897in}}%
\pgfpathlineto{\pgfqpoint{2.717254in}{0.674897in}}%
\pgfpathlineto{\pgfqpoint{2.717254in}{0.449444in}}%
\pgfpathclose%
\pgfusepath{stroke}%
\end{pgfscope}%
\begin{pgfscope}%
\pgfpathrectangle{\pgfqpoint{0.573025in}{0.449444in}}{\pgfqpoint{3.487500in}{1.155000in}}%
\pgfusepath{clip}%
\pgfsetbuttcap%
\pgfsetmiterjoin%
\pgfsetlinewidth{1.003750pt}%
\definecolor{currentstroke}{rgb}{0.000000,0.000000,0.000000}%
\pgfsetstrokecolor{currentstroke}%
\pgfsetdash{}{0pt}%
\pgfpathmoveto{\pgfqpoint{2.884120in}{0.449444in}}%
\pgfpathlineto{\pgfqpoint{2.950866in}{0.449444in}}%
\pgfpathlineto{\pgfqpoint{2.950866in}{0.604322in}}%
\pgfpathlineto{\pgfqpoint{2.884120in}{0.604322in}}%
\pgfpathlineto{\pgfqpoint{2.884120in}{0.449444in}}%
\pgfpathclose%
\pgfusepath{stroke}%
\end{pgfscope}%
\begin{pgfscope}%
\pgfpathrectangle{\pgfqpoint{0.573025in}{0.449444in}}{\pgfqpoint{3.487500in}{1.155000in}}%
\pgfusepath{clip}%
\pgfsetbuttcap%
\pgfsetmiterjoin%
\pgfsetlinewidth{1.003750pt}%
\definecolor{currentstroke}{rgb}{0.000000,0.000000,0.000000}%
\pgfsetstrokecolor{currentstroke}%
\pgfsetdash{}{0pt}%
\pgfpathmoveto{\pgfqpoint{3.050986in}{0.449444in}}%
\pgfpathlineto{\pgfqpoint{3.117732in}{0.449444in}}%
\pgfpathlineto{\pgfqpoint{3.117732in}{0.573965in}}%
\pgfpathlineto{\pgfqpoint{3.050986in}{0.573965in}}%
\pgfpathlineto{\pgfqpoint{3.050986in}{0.449444in}}%
\pgfpathclose%
\pgfusepath{stroke}%
\end{pgfscope}%
\begin{pgfscope}%
\pgfpathrectangle{\pgfqpoint{0.573025in}{0.449444in}}{\pgfqpoint{3.487500in}{1.155000in}}%
\pgfusepath{clip}%
\pgfsetbuttcap%
\pgfsetmiterjoin%
\pgfsetlinewidth{1.003750pt}%
\definecolor{currentstroke}{rgb}{0.000000,0.000000,0.000000}%
\pgfsetstrokecolor{currentstroke}%
\pgfsetdash{}{0pt}%
\pgfpathmoveto{\pgfqpoint{3.217852in}{0.449444in}}%
\pgfpathlineto{\pgfqpoint{3.284598in}{0.449444in}}%
\pgfpathlineto{\pgfqpoint{3.284598in}{0.517312in}}%
\pgfpathlineto{\pgfqpoint{3.217852in}{0.517312in}}%
\pgfpathlineto{\pgfqpoint{3.217852in}{0.449444in}}%
\pgfpathclose%
\pgfusepath{stroke}%
\end{pgfscope}%
\begin{pgfscope}%
\pgfpathrectangle{\pgfqpoint{0.573025in}{0.449444in}}{\pgfqpoint{3.487500in}{1.155000in}}%
\pgfusepath{clip}%
\pgfsetbuttcap%
\pgfsetmiterjoin%
\pgfsetlinewidth{1.003750pt}%
\definecolor{currentstroke}{rgb}{0.000000,0.000000,0.000000}%
\pgfsetstrokecolor{currentstroke}%
\pgfsetdash{}{0pt}%
\pgfpathmoveto{\pgfqpoint{3.384718in}{0.449444in}}%
\pgfpathlineto{\pgfqpoint{3.451464in}{0.449444in}}%
\pgfpathlineto{\pgfqpoint{3.451464in}{0.485022in}}%
\pgfpathlineto{\pgfqpoint{3.384718in}{0.485022in}}%
\pgfpathlineto{\pgfqpoint{3.384718in}{0.449444in}}%
\pgfpathclose%
\pgfusepath{stroke}%
\end{pgfscope}%
\begin{pgfscope}%
\pgfpathrectangle{\pgfqpoint{0.573025in}{0.449444in}}{\pgfqpoint{3.487500in}{1.155000in}}%
\pgfusepath{clip}%
\pgfsetbuttcap%
\pgfsetmiterjoin%
\pgfsetlinewidth{1.003750pt}%
\definecolor{currentstroke}{rgb}{0.000000,0.000000,0.000000}%
\pgfsetstrokecolor{currentstroke}%
\pgfsetdash{}{0pt}%
\pgfpathmoveto{\pgfqpoint{3.551584in}{0.449444in}}%
\pgfpathlineto{\pgfqpoint{3.618330in}{0.449444in}}%
\pgfpathlineto{\pgfqpoint{3.618330in}{0.472067in}}%
\pgfpathlineto{\pgfqpoint{3.551584in}{0.472067in}}%
\pgfpathlineto{\pgfqpoint{3.551584in}{0.449444in}}%
\pgfpathclose%
\pgfusepath{stroke}%
\end{pgfscope}%
\begin{pgfscope}%
\pgfpathrectangle{\pgfqpoint{0.573025in}{0.449444in}}{\pgfqpoint{3.487500in}{1.155000in}}%
\pgfusepath{clip}%
\pgfsetbuttcap%
\pgfsetmiterjoin%
\pgfsetlinewidth{1.003750pt}%
\definecolor{currentstroke}{rgb}{0.000000,0.000000,0.000000}%
\pgfsetstrokecolor{currentstroke}%
\pgfsetdash{}{0pt}%
\pgfpathmoveto{\pgfqpoint{3.718450in}{0.449444in}}%
\pgfpathlineto{\pgfqpoint{3.785196in}{0.449444in}}%
\pgfpathlineto{\pgfqpoint{3.785196in}{0.452538in}}%
\pgfpathlineto{\pgfqpoint{3.718450in}{0.452538in}}%
\pgfpathlineto{\pgfqpoint{3.718450in}{0.449444in}}%
\pgfpathclose%
\pgfusepath{stroke}%
\end{pgfscope}%
\begin{pgfscope}%
\pgfpathrectangle{\pgfqpoint{0.573025in}{0.449444in}}{\pgfqpoint{3.487500in}{1.155000in}}%
\pgfusepath{clip}%
\pgfsetbuttcap%
\pgfsetmiterjoin%
\pgfsetlinewidth{1.003750pt}%
\definecolor{currentstroke}{rgb}{0.000000,0.000000,0.000000}%
\pgfsetstrokecolor{currentstroke}%
\pgfsetdash{}{0pt}%
\pgfpathmoveto{\pgfqpoint{3.885316in}{0.449444in}}%
\pgfpathlineto{\pgfqpoint{3.952062in}{0.449444in}}%
\pgfpathlineto{\pgfqpoint{3.952062in}{0.449831in}}%
\pgfpathlineto{\pgfqpoint{3.885316in}{0.449831in}}%
\pgfpathlineto{\pgfqpoint{3.885316in}{0.449444in}}%
\pgfpathclose%
\pgfusepath{stroke}%
\end{pgfscope}%
\begin{pgfscope}%
\pgfpathrectangle{\pgfqpoint{0.573025in}{0.449444in}}{\pgfqpoint{3.487500in}{1.155000in}}%
\pgfusepath{clip}%
\pgfsetbuttcap%
\pgfsetmiterjoin%
\definecolor{currentfill}{rgb}{0.000000,0.000000,0.000000}%
\pgfsetfillcolor{currentfill}%
\pgfsetlinewidth{0.000000pt}%
\definecolor{currentstroke}{rgb}{0.000000,0.000000,0.000000}%
\pgfsetstrokecolor{currentstroke}%
\pgfsetstrokeopacity{0.000000}%
\pgfsetdash{}{0pt}%
\pgfpathmoveto{\pgfqpoint{0.614742in}{0.449444in}}%
\pgfpathlineto{\pgfqpoint{0.681488in}{0.449444in}}%
\pgfpathlineto{\pgfqpoint{0.681488in}{0.449444in}}%
\pgfpathlineto{\pgfqpoint{0.614742in}{0.449444in}}%
\pgfpathlineto{\pgfqpoint{0.614742in}{0.449444in}}%
\pgfpathclose%
\pgfusepath{fill}%
\end{pgfscope}%
\begin{pgfscope}%
\pgfpathrectangle{\pgfqpoint{0.573025in}{0.449444in}}{\pgfqpoint{3.487500in}{1.155000in}}%
\pgfusepath{clip}%
\pgfsetbuttcap%
\pgfsetmiterjoin%
\definecolor{currentfill}{rgb}{0.000000,0.000000,0.000000}%
\pgfsetfillcolor{currentfill}%
\pgfsetlinewidth{0.000000pt}%
\definecolor{currentstroke}{rgb}{0.000000,0.000000,0.000000}%
\pgfsetstrokecolor{currentstroke}%
\pgfsetstrokeopacity{0.000000}%
\pgfsetdash{}{0pt}%
\pgfpathmoveto{\pgfqpoint{0.781608in}{0.449444in}}%
\pgfpathlineto{\pgfqpoint{0.848354in}{0.449444in}}%
\pgfpathlineto{\pgfqpoint{0.848354in}{0.664649in}}%
\pgfpathlineto{\pgfqpoint{0.781608in}{0.664649in}}%
\pgfpathlineto{\pgfqpoint{0.781608in}{0.449444in}}%
\pgfpathclose%
\pgfusepath{fill}%
\end{pgfscope}%
\begin{pgfscope}%
\pgfpathrectangle{\pgfqpoint{0.573025in}{0.449444in}}{\pgfqpoint{3.487500in}{1.155000in}}%
\pgfusepath{clip}%
\pgfsetbuttcap%
\pgfsetmiterjoin%
\definecolor{currentfill}{rgb}{0.000000,0.000000,0.000000}%
\pgfsetfillcolor{currentfill}%
\pgfsetlinewidth{0.000000pt}%
\definecolor{currentstroke}{rgb}{0.000000,0.000000,0.000000}%
\pgfsetstrokecolor{currentstroke}%
\pgfsetstrokeopacity{0.000000}%
\pgfsetdash{}{0pt}%
\pgfpathmoveto{\pgfqpoint{0.948474in}{0.449444in}}%
\pgfpathlineto{\pgfqpoint{1.015220in}{0.449444in}}%
\pgfpathlineto{\pgfqpoint{1.015220in}{0.908857in}}%
\pgfpathlineto{\pgfqpoint{0.948474in}{0.908857in}}%
\pgfpathlineto{\pgfqpoint{0.948474in}{0.449444in}}%
\pgfpathclose%
\pgfusepath{fill}%
\end{pgfscope}%
\begin{pgfscope}%
\pgfpathrectangle{\pgfqpoint{0.573025in}{0.449444in}}{\pgfqpoint{3.487500in}{1.155000in}}%
\pgfusepath{clip}%
\pgfsetbuttcap%
\pgfsetmiterjoin%
\definecolor{currentfill}{rgb}{0.000000,0.000000,0.000000}%
\pgfsetfillcolor{currentfill}%
\pgfsetlinewidth{0.000000pt}%
\definecolor{currentstroke}{rgb}{0.000000,0.000000,0.000000}%
\pgfsetstrokecolor{currentstroke}%
\pgfsetstrokeopacity{0.000000}%
\pgfsetdash{}{0pt}%
\pgfpathmoveto{\pgfqpoint{1.115340in}{0.449444in}}%
\pgfpathlineto{\pgfqpoint{1.182086in}{0.449444in}}%
\pgfpathlineto{\pgfqpoint{1.182086in}{0.878307in}}%
\pgfpathlineto{\pgfqpoint{1.115340in}{0.878307in}}%
\pgfpathlineto{\pgfqpoint{1.115340in}{0.449444in}}%
\pgfpathclose%
\pgfusepath{fill}%
\end{pgfscope}%
\begin{pgfscope}%
\pgfpathrectangle{\pgfqpoint{0.573025in}{0.449444in}}{\pgfqpoint{3.487500in}{1.155000in}}%
\pgfusepath{clip}%
\pgfsetbuttcap%
\pgfsetmiterjoin%
\definecolor{currentfill}{rgb}{0.000000,0.000000,0.000000}%
\pgfsetfillcolor{currentfill}%
\pgfsetlinewidth{0.000000pt}%
\definecolor{currentstroke}{rgb}{0.000000,0.000000,0.000000}%
\pgfsetstrokecolor{currentstroke}%
\pgfsetstrokeopacity{0.000000}%
\pgfsetdash{}{0pt}%
\pgfpathmoveto{\pgfqpoint{1.282206in}{0.449444in}}%
\pgfpathlineto{\pgfqpoint{1.348952in}{0.449444in}}%
\pgfpathlineto{\pgfqpoint{1.348952in}{0.704674in}}%
\pgfpathlineto{\pgfqpoint{1.282206in}{0.704674in}}%
\pgfpathlineto{\pgfqpoint{1.282206in}{0.449444in}}%
\pgfpathclose%
\pgfusepath{fill}%
\end{pgfscope}%
\begin{pgfscope}%
\pgfpathrectangle{\pgfqpoint{0.573025in}{0.449444in}}{\pgfqpoint{3.487500in}{1.155000in}}%
\pgfusepath{clip}%
\pgfsetbuttcap%
\pgfsetmiterjoin%
\definecolor{currentfill}{rgb}{0.000000,0.000000,0.000000}%
\pgfsetfillcolor{currentfill}%
\pgfsetlinewidth{0.000000pt}%
\definecolor{currentstroke}{rgb}{0.000000,0.000000,0.000000}%
\pgfsetstrokecolor{currentstroke}%
\pgfsetstrokeopacity{0.000000}%
\pgfsetdash{}{0pt}%
\pgfpathmoveto{\pgfqpoint{1.449072in}{0.449444in}}%
\pgfpathlineto{\pgfqpoint{1.515818in}{0.449444in}}%
\pgfpathlineto{\pgfqpoint{1.515818in}{0.809279in}}%
\pgfpathlineto{\pgfqpoint{1.449072in}{0.809279in}}%
\pgfpathlineto{\pgfqpoint{1.449072in}{0.449444in}}%
\pgfpathclose%
\pgfusepath{fill}%
\end{pgfscope}%
\begin{pgfscope}%
\pgfpathrectangle{\pgfqpoint{0.573025in}{0.449444in}}{\pgfqpoint{3.487500in}{1.155000in}}%
\pgfusepath{clip}%
\pgfsetbuttcap%
\pgfsetmiterjoin%
\definecolor{currentfill}{rgb}{0.000000,0.000000,0.000000}%
\pgfsetfillcolor{currentfill}%
\pgfsetlinewidth{0.000000pt}%
\definecolor{currentstroke}{rgb}{0.000000,0.000000,0.000000}%
\pgfsetstrokecolor{currentstroke}%
\pgfsetstrokeopacity{0.000000}%
\pgfsetdash{}{0pt}%
\pgfpathmoveto{\pgfqpoint{1.615938in}{0.449444in}}%
\pgfpathlineto{\pgfqpoint{1.682684in}{0.449444in}}%
\pgfpathlineto{\pgfqpoint{1.682684in}{0.800191in}}%
\pgfpathlineto{\pgfqpoint{1.615938in}{0.800191in}}%
\pgfpathlineto{\pgfqpoint{1.615938in}{0.449444in}}%
\pgfpathclose%
\pgfusepath{fill}%
\end{pgfscope}%
\begin{pgfscope}%
\pgfpathrectangle{\pgfqpoint{0.573025in}{0.449444in}}{\pgfqpoint{3.487500in}{1.155000in}}%
\pgfusepath{clip}%
\pgfsetbuttcap%
\pgfsetmiterjoin%
\definecolor{currentfill}{rgb}{0.000000,0.000000,0.000000}%
\pgfsetfillcolor{currentfill}%
\pgfsetlinewidth{0.000000pt}%
\definecolor{currentstroke}{rgb}{0.000000,0.000000,0.000000}%
\pgfsetstrokecolor{currentstroke}%
\pgfsetstrokeopacity{0.000000}%
\pgfsetdash{}{0pt}%
\pgfpathmoveto{\pgfqpoint{1.782804in}{0.449444in}}%
\pgfpathlineto{\pgfqpoint{1.849550in}{0.449444in}}%
\pgfpathlineto{\pgfqpoint{1.849550in}{0.768288in}}%
\pgfpathlineto{\pgfqpoint{1.782804in}{0.768288in}}%
\pgfpathlineto{\pgfqpoint{1.782804in}{0.449444in}}%
\pgfpathclose%
\pgfusepath{fill}%
\end{pgfscope}%
\begin{pgfscope}%
\pgfpathrectangle{\pgfqpoint{0.573025in}{0.449444in}}{\pgfqpoint{3.487500in}{1.155000in}}%
\pgfusepath{clip}%
\pgfsetbuttcap%
\pgfsetmiterjoin%
\definecolor{currentfill}{rgb}{0.000000,0.000000,0.000000}%
\pgfsetfillcolor{currentfill}%
\pgfsetlinewidth{0.000000pt}%
\definecolor{currentstroke}{rgb}{0.000000,0.000000,0.000000}%
\pgfsetstrokecolor{currentstroke}%
\pgfsetstrokeopacity{0.000000}%
\pgfsetdash{}{0pt}%
\pgfpathmoveto{\pgfqpoint{1.949670in}{0.449444in}}%
\pgfpathlineto{\pgfqpoint{2.016416in}{0.449444in}}%
\pgfpathlineto{\pgfqpoint{2.016416in}{0.856844in}}%
\pgfpathlineto{\pgfqpoint{1.949670in}{0.856844in}}%
\pgfpathlineto{\pgfqpoint{1.949670in}{0.449444in}}%
\pgfpathclose%
\pgfusepath{fill}%
\end{pgfscope}%
\begin{pgfscope}%
\pgfpathrectangle{\pgfqpoint{0.573025in}{0.449444in}}{\pgfqpoint{3.487500in}{1.155000in}}%
\pgfusepath{clip}%
\pgfsetbuttcap%
\pgfsetmiterjoin%
\definecolor{currentfill}{rgb}{0.000000,0.000000,0.000000}%
\pgfsetfillcolor{currentfill}%
\pgfsetlinewidth{0.000000pt}%
\definecolor{currentstroke}{rgb}{0.000000,0.000000,0.000000}%
\pgfsetstrokecolor{currentstroke}%
\pgfsetstrokeopacity{0.000000}%
\pgfsetdash{}{0pt}%
\pgfpathmoveto{\pgfqpoint{2.116536in}{0.449444in}}%
\pgfpathlineto{\pgfqpoint{2.183282in}{0.449444in}}%
\pgfpathlineto{\pgfqpoint{2.183282in}{0.826294in}}%
\pgfpathlineto{\pgfqpoint{2.116536in}{0.826294in}}%
\pgfpathlineto{\pgfqpoint{2.116536in}{0.449444in}}%
\pgfpathclose%
\pgfusepath{fill}%
\end{pgfscope}%
\begin{pgfscope}%
\pgfpathrectangle{\pgfqpoint{0.573025in}{0.449444in}}{\pgfqpoint{3.487500in}{1.155000in}}%
\pgfusepath{clip}%
\pgfsetbuttcap%
\pgfsetmiterjoin%
\definecolor{currentfill}{rgb}{0.000000,0.000000,0.000000}%
\pgfsetfillcolor{currentfill}%
\pgfsetlinewidth{0.000000pt}%
\definecolor{currentstroke}{rgb}{0.000000,0.000000,0.000000}%
\pgfsetstrokecolor{currentstroke}%
\pgfsetstrokeopacity{0.000000}%
\pgfsetdash{}{0pt}%
\pgfpathmoveto{\pgfqpoint{2.283402in}{0.449444in}}%
\pgfpathlineto{\pgfqpoint{2.350148in}{0.449444in}}%
\pgfpathlineto{\pgfqpoint{2.350148in}{0.730583in}}%
\pgfpathlineto{\pgfqpoint{2.283402in}{0.730583in}}%
\pgfpathlineto{\pgfqpoint{2.283402in}{0.449444in}}%
\pgfpathclose%
\pgfusepath{fill}%
\end{pgfscope}%
\begin{pgfscope}%
\pgfpathrectangle{\pgfqpoint{0.573025in}{0.449444in}}{\pgfqpoint{3.487500in}{1.155000in}}%
\pgfusepath{clip}%
\pgfsetbuttcap%
\pgfsetmiterjoin%
\definecolor{currentfill}{rgb}{0.000000,0.000000,0.000000}%
\pgfsetfillcolor{currentfill}%
\pgfsetlinewidth{0.000000pt}%
\definecolor{currentstroke}{rgb}{0.000000,0.000000,0.000000}%
\pgfsetstrokecolor{currentstroke}%
\pgfsetstrokeopacity{0.000000}%
\pgfsetdash{}{0pt}%
\pgfpathmoveto{\pgfqpoint{2.450268in}{0.449444in}}%
\pgfpathlineto{\pgfqpoint{2.517015in}{0.449444in}}%
\pgfpathlineto{\pgfqpoint{2.517015in}{0.706994in}}%
\pgfpathlineto{\pgfqpoint{2.450268in}{0.706994in}}%
\pgfpathlineto{\pgfqpoint{2.450268in}{0.449444in}}%
\pgfpathclose%
\pgfusepath{fill}%
\end{pgfscope}%
\begin{pgfscope}%
\pgfpathrectangle{\pgfqpoint{0.573025in}{0.449444in}}{\pgfqpoint{3.487500in}{1.155000in}}%
\pgfusepath{clip}%
\pgfsetbuttcap%
\pgfsetmiterjoin%
\definecolor{currentfill}{rgb}{0.000000,0.000000,0.000000}%
\pgfsetfillcolor{currentfill}%
\pgfsetlinewidth{0.000000pt}%
\definecolor{currentstroke}{rgb}{0.000000,0.000000,0.000000}%
\pgfsetstrokecolor{currentstroke}%
\pgfsetstrokeopacity{0.000000}%
\pgfsetdash{}{0pt}%
\pgfpathmoveto{\pgfqpoint{2.617134in}{0.449444in}}%
\pgfpathlineto{\pgfqpoint{2.683881in}{0.449444in}}%
\pgfpathlineto{\pgfqpoint{2.683881in}{0.661169in}}%
\pgfpathlineto{\pgfqpoint{2.617134in}{0.661169in}}%
\pgfpathlineto{\pgfqpoint{2.617134in}{0.449444in}}%
\pgfpathclose%
\pgfusepath{fill}%
\end{pgfscope}%
\begin{pgfscope}%
\pgfpathrectangle{\pgfqpoint{0.573025in}{0.449444in}}{\pgfqpoint{3.487500in}{1.155000in}}%
\pgfusepath{clip}%
\pgfsetbuttcap%
\pgfsetmiterjoin%
\definecolor{currentfill}{rgb}{0.000000,0.000000,0.000000}%
\pgfsetfillcolor{currentfill}%
\pgfsetlinewidth{0.000000pt}%
\definecolor{currentstroke}{rgb}{0.000000,0.000000,0.000000}%
\pgfsetstrokecolor{currentstroke}%
\pgfsetstrokeopacity{0.000000}%
\pgfsetdash{}{0pt}%
\pgfpathmoveto{\pgfqpoint{2.784000in}{0.449444in}}%
\pgfpathlineto{\pgfqpoint{2.850747in}{0.449444in}}%
\pgfpathlineto{\pgfqpoint{2.850747in}{0.652854in}}%
\pgfpathlineto{\pgfqpoint{2.784000in}{0.652854in}}%
\pgfpathlineto{\pgfqpoint{2.784000in}{0.449444in}}%
\pgfpathclose%
\pgfusepath{fill}%
\end{pgfscope}%
\begin{pgfscope}%
\pgfpathrectangle{\pgfqpoint{0.573025in}{0.449444in}}{\pgfqpoint{3.487500in}{1.155000in}}%
\pgfusepath{clip}%
\pgfsetbuttcap%
\pgfsetmiterjoin%
\definecolor{currentfill}{rgb}{0.000000,0.000000,0.000000}%
\pgfsetfillcolor{currentfill}%
\pgfsetlinewidth{0.000000pt}%
\definecolor{currentstroke}{rgb}{0.000000,0.000000,0.000000}%
\pgfsetstrokecolor{currentstroke}%
\pgfsetstrokeopacity{0.000000}%
\pgfsetdash{}{0pt}%
\pgfpathmoveto{\pgfqpoint{2.950866in}{0.449444in}}%
\pgfpathlineto{\pgfqpoint{3.017613in}{0.449444in}}%
\pgfpathlineto{\pgfqpoint{3.017613in}{0.585567in}}%
\pgfpathlineto{\pgfqpoint{2.950866in}{0.585567in}}%
\pgfpathlineto{\pgfqpoint{2.950866in}{0.449444in}}%
\pgfpathclose%
\pgfusepath{fill}%
\end{pgfscope}%
\begin{pgfscope}%
\pgfpathrectangle{\pgfqpoint{0.573025in}{0.449444in}}{\pgfqpoint{3.487500in}{1.155000in}}%
\pgfusepath{clip}%
\pgfsetbuttcap%
\pgfsetmiterjoin%
\definecolor{currentfill}{rgb}{0.000000,0.000000,0.000000}%
\pgfsetfillcolor{currentfill}%
\pgfsetlinewidth{0.000000pt}%
\definecolor{currentstroke}{rgb}{0.000000,0.000000,0.000000}%
\pgfsetstrokecolor{currentstroke}%
\pgfsetstrokeopacity{0.000000}%
\pgfsetdash{}{0pt}%
\pgfpathmoveto{\pgfqpoint{3.117732in}{0.449444in}}%
\pgfpathlineto{\pgfqpoint{3.184479in}{0.449444in}}%
\pgfpathlineto{\pgfqpoint{3.184479in}{0.564297in}}%
\pgfpathlineto{\pgfqpoint{3.117732in}{0.564297in}}%
\pgfpathlineto{\pgfqpoint{3.117732in}{0.449444in}}%
\pgfpathclose%
\pgfusepath{fill}%
\end{pgfscope}%
\begin{pgfscope}%
\pgfpathrectangle{\pgfqpoint{0.573025in}{0.449444in}}{\pgfqpoint{3.487500in}{1.155000in}}%
\pgfusepath{clip}%
\pgfsetbuttcap%
\pgfsetmiterjoin%
\definecolor{currentfill}{rgb}{0.000000,0.000000,0.000000}%
\pgfsetfillcolor{currentfill}%
\pgfsetlinewidth{0.000000pt}%
\definecolor{currentstroke}{rgb}{0.000000,0.000000,0.000000}%
\pgfsetstrokecolor{currentstroke}%
\pgfsetstrokeopacity{0.000000}%
\pgfsetdash{}{0pt}%
\pgfpathmoveto{\pgfqpoint{3.284598in}{0.449444in}}%
\pgfpathlineto{\pgfqpoint{3.351345in}{0.449444in}}%
\pgfpathlineto{\pgfqpoint{3.351345in}{0.516152in}}%
\pgfpathlineto{\pgfqpoint{3.284598in}{0.516152in}}%
\pgfpathlineto{\pgfqpoint{3.284598in}{0.449444in}}%
\pgfpathclose%
\pgfusepath{fill}%
\end{pgfscope}%
\begin{pgfscope}%
\pgfpathrectangle{\pgfqpoint{0.573025in}{0.449444in}}{\pgfqpoint{3.487500in}{1.155000in}}%
\pgfusepath{clip}%
\pgfsetbuttcap%
\pgfsetmiterjoin%
\definecolor{currentfill}{rgb}{0.000000,0.000000,0.000000}%
\pgfsetfillcolor{currentfill}%
\pgfsetlinewidth{0.000000pt}%
\definecolor{currentstroke}{rgb}{0.000000,0.000000,0.000000}%
\pgfsetstrokecolor{currentstroke}%
\pgfsetstrokeopacity{0.000000}%
\pgfsetdash{}{0pt}%
\pgfpathmoveto{\pgfqpoint{3.451464in}{0.449444in}}%
\pgfpathlineto{\pgfqpoint{3.518211in}{0.449444in}}%
\pgfpathlineto{\pgfqpoint{3.518211in}{0.482895in}}%
\pgfpathlineto{\pgfqpoint{3.451464in}{0.482895in}}%
\pgfpathlineto{\pgfqpoint{3.451464in}{0.449444in}}%
\pgfpathclose%
\pgfusepath{fill}%
\end{pgfscope}%
\begin{pgfscope}%
\pgfpathrectangle{\pgfqpoint{0.573025in}{0.449444in}}{\pgfqpoint{3.487500in}{1.155000in}}%
\pgfusepath{clip}%
\pgfsetbuttcap%
\pgfsetmiterjoin%
\definecolor{currentfill}{rgb}{0.000000,0.000000,0.000000}%
\pgfsetfillcolor{currentfill}%
\pgfsetlinewidth{0.000000pt}%
\definecolor{currentstroke}{rgb}{0.000000,0.000000,0.000000}%
\pgfsetstrokecolor{currentstroke}%
\pgfsetstrokeopacity{0.000000}%
\pgfsetdash{}{0pt}%
\pgfpathmoveto{\pgfqpoint{3.618330in}{0.449444in}}%
\pgfpathlineto{\pgfqpoint{3.685077in}{0.449444in}}%
\pgfpathlineto{\pgfqpoint{3.685077in}{0.466266in}}%
\pgfpathlineto{\pgfqpoint{3.618330in}{0.466266in}}%
\pgfpathlineto{\pgfqpoint{3.618330in}{0.449444in}}%
\pgfpathclose%
\pgfusepath{fill}%
\end{pgfscope}%
\begin{pgfscope}%
\pgfpathrectangle{\pgfqpoint{0.573025in}{0.449444in}}{\pgfqpoint{3.487500in}{1.155000in}}%
\pgfusepath{clip}%
\pgfsetbuttcap%
\pgfsetmiterjoin%
\definecolor{currentfill}{rgb}{0.000000,0.000000,0.000000}%
\pgfsetfillcolor{currentfill}%
\pgfsetlinewidth{0.000000pt}%
\definecolor{currentstroke}{rgb}{0.000000,0.000000,0.000000}%
\pgfsetstrokecolor{currentstroke}%
\pgfsetstrokeopacity{0.000000}%
\pgfsetdash{}{0pt}%
\pgfpathmoveto{\pgfqpoint{3.785196in}{0.449444in}}%
\pgfpathlineto{\pgfqpoint{3.851943in}{0.449444in}}%
\pgfpathlineto{\pgfqpoint{3.851943in}{0.454085in}}%
\pgfpathlineto{\pgfqpoint{3.785196in}{0.454085in}}%
\pgfpathlineto{\pgfqpoint{3.785196in}{0.449444in}}%
\pgfpathclose%
\pgfusepath{fill}%
\end{pgfscope}%
\begin{pgfscope}%
\pgfpathrectangle{\pgfqpoint{0.573025in}{0.449444in}}{\pgfqpoint{3.487500in}{1.155000in}}%
\pgfusepath{clip}%
\pgfsetbuttcap%
\pgfsetmiterjoin%
\definecolor{currentfill}{rgb}{0.000000,0.000000,0.000000}%
\pgfsetfillcolor{currentfill}%
\pgfsetlinewidth{0.000000pt}%
\definecolor{currentstroke}{rgb}{0.000000,0.000000,0.000000}%
\pgfsetstrokecolor{currentstroke}%
\pgfsetstrokeopacity{0.000000}%
\pgfsetdash{}{0pt}%
\pgfpathmoveto{\pgfqpoint{3.952062in}{0.449444in}}%
\pgfpathlineto{\pgfqpoint{4.018809in}{0.449444in}}%
\pgfpathlineto{\pgfqpoint{4.018809in}{0.449638in}}%
\pgfpathlineto{\pgfqpoint{3.952062in}{0.449638in}}%
\pgfpathlineto{\pgfqpoint{3.952062in}{0.449444in}}%
\pgfpathclose%
\pgfusepath{fill}%
\end{pgfscope}%
\begin{pgfscope}%
\pgfsetbuttcap%
\pgfsetroundjoin%
\definecolor{currentfill}{rgb}{0.000000,0.000000,0.000000}%
\pgfsetfillcolor{currentfill}%
\pgfsetlinewidth{0.803000pt}%
\definecolor{currentstroke}{rgb}{0.000000,0.000000,0.000000}%
\pgfsetstrokecolor{currentstroke}%
\pgfsetdash{}{0pt}%
\pgfsys@defobject{currentmarker}{\pgfqpoint{0.000000in}{-0.048611in}}{\pgfqpoint{0.000000in}{0.000000in}}{%
\pgfpathmoveto{\pgfqpoint{0.000000in}{0.000000in}}%
\pgfpathlineto{\pgfqpoint{0.000000in}{-0.048611in}}%
\pgfusepath{stroke,fill}%
}%
\begin{pgfscope}%
\pgfsys@transformshift{0.614742in}{0.449444in}%
\pgfsys@useobject{currentmarker}{}%
\end{pgfscope}%
\end{pgfscope}%
\begin{pgfscope}%
\definecolor{textcolor}{rgb}{0.000000,0.000000,0.000000}%
\pgfsetstrokecolor{textcolor}%
\pgfsetfillcolor{textcolor}%
\pgftext[x=0.614742in,y=0.352222in,,top]{\color{textcolor}\rmfamily\fontsize{10.000000}{12.000000}\selectfont 0.196}%
\end{pgfscope}%
\begin{pgfscope}%
\pgfsetbuttcap%
\pgfsetroundjoin%
\definecolor{currentfill}{rgb}{0.000000,0.000000,0.000000}%
\pgfsetfillcolor{currentfill}%
\pgfsetlinewidth{0.803000pt}%
\definecolor{currentstroke}{rgb}{0.000000,0.000000,0.000000}%
\pgfsetstrokecolor{currentstroke}%
\pgfsetdash{}{0pt}%
\pgfsys@defobject{currentmarker}{\pgfqpoint{0.000000in}{-0.048611in}}{\pgfqpoint{0.000000in}{0.000000in}}{%
\pgfpathmoveto{\pgfqpoint{0.000000in}{0.000000in}}%
\pgfpathlineto{\pgfqpoint{0.000000in}{-0.048611in}}%
\pgfusepath{stroke,fill}%
}%
\begin{pgfscope}%
\pgfsys@transformshift{1.282206in}{0.449444in}%
\pgfsys@useobject{currentmarker}{}%
\end{pgfscope}%
\end{pgfscope}%
\begin{pgfscope}%
\definecolor{textcolor}{rgb}{0.000000,0.000000,0.000000}%
\pgfsetstrokecolor{textcolor}%
\pgfsetfillcolor{textcolor}%
\pgftext[x=1.282206in,y=0.352222in,,top]{\color{textcolor}\rmfamily\fontsize{10.000000}{12.000000}\selectfont 0.214}%
\end{pgfscope}%
\begin{pgfscope}%
\pgfsetbuttcap%
\pgfsetroundjoin%
\definecolor{currentfill}{rgb}{0.000000,0.000000,0.000000}%
\pgfsetfillcolor{currentfill}%
\pgfsetlinewidth{0.803000pt}%
\definecolor{currentstroke}{rgb}{0.000000,0.000000,0.000000}%
\pgfsetstrokecolor{currentstroke}%
\pgfsetdash{}{0pt}%
\pgfsys@defobject{currentmarker}{\pgfqpoint{0.000000in}{-0.048611in}}{\pgfqpoint{0.000000in}{0.000000in}}{%
\pgfpathmoveto{\pgfqpoint{0.000000in}{0.000000in}}%
\pgfpathlineto{\pgfqpoint{0.000000in}{-0.048611in}}%
\pgfusepath{stroke,fill}%
}%
\begin{pgfscope}%
\pgfsys@transformshift{1.949670in}{0.449444in}%
\pgfsys@useobject{currentmarker}{}%
\end{pgfscope}%
\end{pgfscope}%
\begin{pgfscope}%
\definecolor{textcolor}{rgb}{0.000000,0.000000,0.000000}%
\pgfsetstrokecolor{textcolor}%
\pgfsetfillcolor{textcolor}%
\pgftext[x=1.949670in,y=0.352222in,,top]{\color{textcolor}\rmfamily\fontsize{10.000000}{12.000000}\selectfont 0.232}%
\end{pgfscope}%
\begin{pgfscope}%
\pgfsetbuttcap%
\pgfsetroundjoin%
\definecolor{currentfill}{rgb}{0.000000,0.000000,0.000000}%
\pgfsetfillcolor{currentfill}%
\pgfsetlinewidth{0.803000pt}%
\definecolor{currentstroke}{rgb}{0.000000,0.000000,0.000000}%
\pgfsetstrokecolor{currentstroke}%
\pgfsetdash{}{0pt}%
\pgfsys@defobject{currentmarker}{\pgfqpoint{0.000000in}{-0.048611in}}{\pgfqpoint{0.000000in}{0.000000in}}{%
\pgfpathmoveto{\pgfqpoint{0.000000in}{0.000000in}}%
\pgfpathlineto{\pgfqpoint{0.000000in}{-0.048611in}}%
\pgfusepath{stroke,fill}%
}%
\begin{pgfscope}%
\pgfsys@transformshift{2.617134in}{0.449444in}%
\pgfsys@useobject{currentmarker}{}%
\end{pgfscope}%
\end{pgfscope}%
\begin{pgfscope}%
\definecolor{textcolor}{rgb}{0.000000,0.000000,0.000000}%
\pgfsetstrokecolor{textcolor}%
\pgfsetfillcolor{textcolor}%
\pgftext[x=2.617134in,y=0.352222in,,top]{\color{textcolor}\rmfamily\fontsize{10.000000}{12.000000}\selectfont 0.249}%
\end{pgfscope}%
\begin{pgfscope}%
\pgfsetbuttcap%
\pgfsetroundjoin%
\definecolor{currentfill}{rgb}{0.000000,0.000000,0.000000}%
\pgfsetfillcolor{currentfill}%
\pgfsetlinewidth{0.803000pt}%
\definecolor{currentstroke}{rgb}{0.000000,0.000000,0.000000}%
\pgfsetstrokecolor{currentstroke}%
\pgfsetdash{}{0pt}%
\pgfsys@defobject{currentmarker}{\pgfqpoint{0.000000in}{-0.048611in}}{\pgfqpoint{0.000000in}{0.000000in}}{%
\pgfpathmoveto{\pgfqpoint{0.000000in}{0.000000in}}%
\pgfpathlineto{\pgfqpoint{0.000000in}{-0.048611in}}%
\pgfusepath{stroke,fill}%
}%
\begin{pgfscope}%
\pgfsys@transformshift{3.284598in}{0.449444in}%
\pgfsys@useobject{currentmarker}{}%
\end{pgfscope}%
\end{pgfscope}%
\begin{pgfscope}%
\definecolor{textcolor}{rgb}{0.000000,0.000000,0.000000}%
\pgfsetstrokecolor{textcolor}%
\pgfsetfillcolor{textcolor}%
\pgftext[x=3.284598in,y=0.352222in,,top]{\color{textcolor}\rmfamily\fontsize{10.000000}{12.000000}\selectfont 0.267}%
\end{pgfscope}%
\begin{pgfscope}%
\pgfsetbuttcap%
\pgfsetroundjoin%
\definecolor{currentfill}{rgb}{0.000000,0.000000,0.000000}%
\pgfsetfillcolor{currentfill}%
\pgfsetlinewidth{0.803000pt}%
\definecolor{currentstroke}{rgb}{0.000000,0.000000,0.000000}%
\pgfsetstrokecolor{currentstroke}%
\pgfsetdash{}{0pt}%
\pgfsys@defobject{currentmarker}{\pgfqpoint{0.000000in}{-0.048611in}}{\pgfqpoint{0.000000in}{0.000000in}}{%
\pgfpathmoveto{\pgfqpoint{0.000000in}{0.000000in}}%
\pgfpathlineto{\pgfqpoint{0.000000in}{-0.048611in}}%
\pgfusepath{stroke,fill}%
}%
\begin{pgfscope}%
\pgfsys@transformshift{3.952062in}{0.449444in}%
\pgfsys@useobject{currentmarker}{}%
\end{pgfscope}%
\end{pgfscope}%
\begin{pgfscope}%
\definecolor{textcolor}{rgb}{0.000000,0.000000,0.000000}%
\pgfsetstrokecolor{textcolor}%
\pgfsetfillcolor{textcolor}%
\pgftext[x=3.952062in,y=0.352222in,,top]{\color{textcolor}\rmfamily\fontsize{10.000000}{12.000000}\selectfont 0.285}%
\end{pgfscope}%
\begin{pgfscope}%
\definecolor{textcolor}{rgb}{0.000000,0.000000,0.000000}%
\pgfsetstrokecolor{textcolor}%
\pgfsetfillcolor{textcolor}%
\pgftext[x=2.316775in,y=0.173333in,,top]{\color{textcolor}\rmfamily\fontsize{10.000000}{12.000000}\selectfont \(\displaystyle p\)}%
\end{pgfscope}%
\begin{pgfscope}%
\pgfsetbuttcap%
\pgfsetroundjoin%
\definecolor{currentfill}{rgb}{0.000000,0.000000,0.000000}%
\pgfsetfillcolor{currentfill}%
\pgfsetlinewidth{0.803000pt}%
\definecolor{currentstroke}{rgb}{0.000000,0.000000,0.000000}%
\pgfsetstrokecolor{currentstroke}%
\pgfsetdash{}{0pt}%
\pgfsys@defobject{currentmarker}{\pgfqpoint{-0.048611in}{0.000000in}}{\pgfqpoint{-0.000000in}{0.000000in}}{%
\pgfpathmoveto{\pgfqpoint{-0.000000in}{0.000000in}}%
\pgfpathlineto{\pgfqpoint{-0.048611in}{0.000000in}}%
\pgfusepath{stroke,fill}%
}%
\begin{pgfscope}%
\pgfsys@transformshift{0.573025in}{0.449444in}%
\pgfsys@useobject{currentmarker}{}%
\end{pgfscope}%
\end{pgfscope}%
\begin{pgfscope}%
\definecolor{textcolor}{rgb}{0.000000,0.000000,0.000000}%
\pgfsetstrokecolor{textcolor}%
\pgfsetfillcolor{textcolor}%
\pgftext[x=0.228889in, y=0.401250in, left, base]{\color{textcolor}\rmfamily\fontsize{10.000000}{12.000000}\selectfont \(\displaystyle {0.00}\)}%
\end{pgfscope}%
\begin{pgfscope}%
\pgfsetbuttcap%
\pgfsetroundjoin%
\definecolor{currentfill}{rgb}{0.000000,0.000000,0.000000}%
\pgfsetfillcolor{currentfill}%
\pgfsetlinewidth{0.803000pt}%
\definecolor{currentstroke}{rgb}{0.000000,0.000000,0.000000}%
\pgfsetstrokecolor{currentstroke}%
\pgfsetdash{}{0pt}%
\pgfsys@defobject{currentmarker}{\pgfqpoint{-0.048611in}{0.000000in}}{\pgfqpoint{-0.000000in}{0.000000in}}{%
\pgfpathmoveto{\pgfqpoint{-0.000000in}{0.000000in}}%
\pgfpathlineto{\pgfqpoint{-0.048611in}{0.000000in}}%
\pgfusepath{stroke,fill}%
}%
\begin{pgfscope}%
\pgfsys@transformshift{0.573025in}{0.794374in}%
\pgfsys@useobject{currentmarker}{}%
\end{pgfscope}%
\end{pgfscope}%
\begin{pgfscope}%
\definecolor{textcolor}{rgb}{0.000000,0.000000,0.000000}%
\pgfsetstrokecolor{textcolor}%
\pgfsetfillcolor{textcolor}%
\pgftext[x=0.228889in, y=0.746180in, left, base]{\color{textcolor}\rmfamily\fontsize{10.000000}{12.000000}\selectfont \(\displaystyle {0.25}\)}%
\end{pgfscope}%
\begin{pgfscope}%
\pgfsetbuttcap%
\pgfsetroundjoin%
\definecolor{currentfill}{rgb}{0.000000,0.000000,0.000000}%
\pgfsetfillcolor{currentfill}%
\pgfsetlinewidth{0.803000pt}%
\definecolor{currentstroke}{rgb}{0.000000,0.000000,0.000000}%
\pgfsetstrokecolor{currentstroke}%
\pgfsetdash{}{0pt}%
\pgfsys@defobject{currentmarker}{\pgfqpoint{-0.048611in}{0.000000in}}{\pgfqpoint{-0.000000in}{0.000000in}}{%
\pgfpathmoveto{\pgfqpoint{-0.000000in}{0.000000in}}%
\pgfpathlineto{\pgfqpoint{-0.048611in}{0.000000in}}%
\pgfusepath{stroke,fill}%
}%
\begin{pgfscope}%
\pgfsys@transformshift{0.573025in}{1.139304in}%
\pgfsys@useobject{currentmarker}{}%
\end{pgfscope}%
\end{pgfscope}%
\begin{pgfscope}%
\definecolor{textcolor}{rgb}{0.000000,0.000000,0.000000}%
\pgfsetstrokecolor{textcolor}%
\pgfsetfillcolor{textcolor}%
\pgftext[x=0.228889in, y=1.091110in, left, base]{\color{textcolor}\rmfamily\fontsize{10.000000}{12.000000}\selectfont \(\displaystyle {0.50}\)}%
\end{pgfscope}%
\begin{pgfscope}%
\pgfsetbuttcap%
\pgfsetroundjoin%
\definecolor{currentfill}{rgb}{0.000000,0.000000,0.000000}%
\pgfsetfillcolor{currentfill}%
\pgfsetlinewidth{0.803000pt}%
\definecolor{currentstroke}{rgb}{0.000000,0.000000,0.000000}%
\pgfsetstrokecolor{currentstroke}%
\pgfsetdash{}{0pt}%
\pgfsys@defobject{currentmarker}{\pgfqpoint{-0.048611in}{0.000000in}}{\pgfqpoint{-0.000000in}{0.000000in}}{%
\pgfpathmoveto{\pgfqpoint{-0.000000in}{0.000000in}}%
\pgfpathlineto{\pgfqpoint{-0.048611in}{0.000000in}}%
\pgfusepath{stroke,fill}%
}%
\begin{pgfscope}%
\pgfsys@transformshift{0.573025in}{1.484234in}%
\pgfsys@useobject{currentmarker}{}%
\end{pgfscope}%
\end{pgfscope}%
\begin{pgfscope}%
\definecolor{textcolor}{rgb}{0.000000,0.000000,0.000000}%
\pgfsetstrokecolor{textcolor}%
\pgfsetfillcolor{textcolor}%
\pgftext[x=0.228889in, y=1.436040in, left, base]{\color{textcolor}\rmfamily\fontsize{10.000000}{12.000000}\selectfont \(\displaystyle {0.75}\)}%
\end{pgfscope}%
\begin{pgfscope}%
\definecolor{textcolor}{rgb}{0.000000,0.000000,0.000000}%
\pgfsetstrokecolor{textcolor}%
\pgfsetfillcolor{textcolor}%
\pgftext[x=0.173333in,y=1.026944in,,bottom,rotate=90.000000]{\color{textcolor}\rmfamily\fontsize{10.000000}{12.000000}\selectfont Percent of Data Set}%
\end{pgfscope}%
\begin{pgfscope}%
\pgfsetrectcap%
\pgfsetmiterjoin%
\pgfsetlinewidth{0.803000pt}%
\definecolor{currentstroke}{rgb}{0.000000,0.000000,0.000000}%
\pgfsetstrokecolor{currentstroke}%
\pgfsetdash{}{0pt}%
\pgfpathmoveto{\pgfqpoint{0.573025in}{0.449444in}}%
\pgfpathlineto{\pgfqpoint{0.573025in}{1.604444in}}%
\pgfusepath{stroke}%
\end{pgfscope}%
\begin{pgfscope}%
\pgfsetrectcap%
\pgfsetmiterjoin%
\pgfsetlinewidth{0.803000pt}%
\definecolor{currentstroke}{rgb}{0.000000,0.000000,0.000000}%
\pgfsetstrokecolor{currentstroke}%
\pgfsetdash{}{0pt}%
\pgfpathmoveto{\pgfqpoint{4.060525in}{0.449444in}}%
\pgfpathlineto{\pgfqpoint{4.060525in}{1.604444in}}%
\pgfusepath{stroke}%
\end{pgfscope}%
\begin{pgfscope}%
\pgfsetrectcap%
\pgfsetmiterjoin%
\pgfsetlinewidth{0.803000pt}%
\definecolor{currentstroke}{rgb}{0.000000,0.000000,0.000000}%
\pgfsetstrokecolor{currentstroke}%
\pgfsetdash{}{0pt}%
\pgfpathmoveto{\pgfqpoint{0.573025in}{0.449444in}}%
\pgfpathlineto{\pgfqpoint{4.060525in}{0.449444in}}%
\pgfusepath{stroke}%
\end{pgfscope}%
\begin{pgfscope}%
\pgfsetrectcap%
\pgfsetmiterjoin%
\pgfsetlinewidth{0.803000pt}%
\definecolor{currentstroke}{rgb}{0.000000,0.000000,0.000000}%
\pgfsetstrokecolor{currentstroke}%
\pgfsetdash{}{0pt}%
\pgfpathmoveto{\pgfqpoint{0.573025in}{1.604444in}}%
\pgfpathlineto{\pgfqpoint{4.060525in}{1.604444in}}%
\pgfusepath{stroke}%
\end{pgfscope}%
\begin{pgfscope}%
\pgfsetbuttcap%
\pgfsetmiterjoin%
\definecolor{currentfill}{rgb}{1.000000,1.000000,1.000000}%
\pgfsetfillcolor{currentfill}%
\pgfsetfillopacity{0.800000}%
\pgfsetlinewidth{1.003750pt}%
\definecolor{currentstroke}{rgb}{0.800000,0.800000,0.800000}%
\pgfsetstrokecolor{currentstroke}%
\pgfsetstrokeopacity{0.800000}%
\pgfsetdash{}{0pt}%
\pgfpathmoveto{\pgfqpoint{3.283581in}{1.104445in}}%
\pgfpathlineto{\pgfqpoint{3.963303in}{1.104445in}}%
\pgfpathquadraticcurveto{\pgfqpoint{3.991081in}{1.104445in}}{\pgfqpoint{3.991081in}{1.132222in}}%
\pgfpathlineto{\pgfqpoint{3.991081in}{1.507222in}}%
\pgfpathquadraticcurveto{\pgfqpoint{3.991081in}{1.535000in}}{\pgfqpoint{3.963303in}{1.535000in}}%
\pgfpathlineto{\pgfqpoint{3.283581in}{1.535000in}}%
\pgfpathquadraticcurveto{\pgfqpoint{3.255803in}{1.535000in}}{\pgfqpoint{3.255803in}{1.507222in}}%
\pgfpathlineto{\pgfqpoint{3.255803in}{1.132222in}}%
\pgfpathquadraticcurveto{\pgfqpoint{3.255803in}{1.104445in}}{\pgfqpoint{3.283581in}{1.104445in}}%
\pgfpathlineto{\pgfqpoint{3.283581in}{1.104445in}}%
\pgfpathclose%
\pgfusepath{stroke,fill}%
\end{pgfscope}%
\begin{pgfscope}%
\pgfsetbuttcap%
\pgfsetmiterjoin%
\pgfsetlinewidth{1.003750pt}%
\definecolor{currentstroke}{rgb}{0.000000,0.000000,0.000000}%
\pgfsetstrokecolor{currentstroke}%
\pgfsetdash{}{0pt}%
\pgfpathmoveto{\pgfqpoint{3.311359in}{1.382222in}}%
\pgfpathlineto{\pgfqpoint{3.589136in}{1.382222in}}%
\pgfpathlineto{\pgfqpoint{3.589136in}{1.479444in}}%
\pgfpathlineto{\pgfqpoint{3.311359in}{1.479444in}}%
\pgfpathlineto{\pgfqpoint{3.311359in}{1.382222in}}%
\pgfpathclose%
\pgfusepath{stroke}%
\end{pgfscope}%
\begin{pgfscope}%
\definecolor{textcolor}{rgb}{0.000000,0.000000,0.000000}%
\pgfsetstrokecolor{textcolor}%
\pgfsetfillcolor{textcolor}%
\pgftext[x=3.700248in,y=1.382222in,left,base]{\color{textcolor}\rmfamily\fontsize{10.000000}{12.000000}\selectfont Neg}%
\end{pgfscope}%
\begin{pgfscope}%
\pgfsetbuttcap%
\pgfsetmiterjoin%
\definecolor{currentfill}{rgb}{0.000000,0.000000,0.000000}%
\pgfsetfillcolor{currentfill}%
\pgfsetlinewidth{0.000000pt}%
\definecolor{currentstroke}{rgb}{0.000000,0.000000,0.000000}%
\pgfsetstrokecolor{currentstroke}%
\pgfsetstrokeopacity{0.000000}%
\pgfsetdash{}{0pt}%
\pgfpathmoveto{\pgfqpoint{3.311359in}{1.186944in}}%
\pgfpathlineto{\pgfqpoint{3.589136in}{1.186944in}}%
\pgfpathlineto{\pgfqpoint{3.589136in}{1.284167in}}%
\pgfpathlineto{\pgfqpoint{3.311359in}{1.284167in}}%
\pgfpathlineto{\pgfqpoint{3.311359in}{1.186944in}}%
\pgfpathclose%
\pgfusepath{fill}%
\end{pgfscope}%
\begin{pgfscope}%
\definecolor{textcolor}{rgb}{0.000000,0.000000,0.000000}%
\pgfsetstrokecolor{textcolor}%
\pgfsetfillcolor{textcolor}%
\pgftext[x=3.700248in,y=1.186944in,left,base]{\color{textcolor}\rmfamily\fontsize{10.000000}{12.000000}\selectfont Pos}%
\end{pgfscope}%
\end{pgfpicture}%
\makeatother%
\endgroup%
	
\cr
	\vskip 0pt
	
	\hfil {\normalfont\normalsize Full range of $p$}
	
	
	%% Creator: Matplotlib, PGF backend
%%
%% To include the figure in your LaTeX document, write
%%   \input{<filename>.pgf}
%%
%% Make sure the required packages are loaded in your preamble
%%   \usepackage{pgf}
%%
%% Also ensure that all the required font packages are loaded; for instance,
%% the lmodern package is sometimes necessary when using math font.
%%   \usepackage{lmodern}
%%
%% Figures using additional raster images can only be included by \input if
%% they are in the same directory as the main LaTeX file. For loading figures
%% from other directories you can use the `import` package
%%   \usepackage{import}
%%
%% and then include the figures with
%%   \import{<path to file>}{<filename>.pgf}
%%
%% Matplotlib used the following preamble
%%   
%%   \usepackage{fontspec}
%%   \makeatletter\@ifpackageloaded{underscore}{}{\usepackage[strings]{underscore}}\makeatother
%%
\begingroup%
\makeatletter%
\begin{pgfpicture}%
\pgfpathrectangle{\pgfpointorigin}{\pgfqpoint{2.153750in}{1.654444in}}%
\pgfusepath{use as bounding box, clip}%
\begin{pgfscope}%
\pgfsetbuttcap%
\pgfsetmiterjoin%
\definecolor{currentfill}{rgb}{1.000000,1.000000,1.000000}%
\pgfsetfillcolor{currentfill}%
\pgfsetlinewidth{0.000000pt}%
\definecolor{currentstroke}{rgb}{1.000000,1.000000,1.000000}%
\pgfsetstrokecolor{currentstroke}%
\pgfsetdash{}{0pt}%
\pgfpathmoveto{\pgfqpoint{0.000000in}{0.000000in}}%
\pgfpathlineto{\pgfqpoint{2.153750in}{0.000000in}}%
\pgfpathlineto{\pgfqpoint{2.153750in}{1.654444in}}%
\pgfpathlineto{\pgfqpoint{0.000000in}{1.654444in}}%
\pgfpathlineto{\pgfqpoint{0.000000in}{0.000000in}}%
\pgfpathclose%
\pgfusepath{fill}%
\end{pgfscope}%
\begin{pgfscope}%
\pgfsetbuttcap%
\pgfsetmiterjoin%
\definecolor{currentfill}{rgb}{1.000000,1.000000,1.000000}%
\pgfsetfillcolor{currentfill}%
\pgfsetlinewidth{0.000000pt}%
\definecolor{currentstroke}{rgb}{0.000000,0.000000,0.000000}%
\pgfsetstrokecolor{currentstroke}%
\pgfsetstrokeopacity{0.000000}%
\pgfsetdash{}{0pt}%
\pgfpathmoveto{\pgfqpoint{0.465000in}{0.449444in}}%
\pgfpathlineto{\pgfqpoint{2.015000in}{0.449444in}}%
\pgfpathlineto{\pgfqpoint{2.015000in}{1.604444in}}%
\pgfpathlineto{\pgfqpoint{0.465000in}{1.604444in}}%
\pgfpathlineto{\pgfqpoint{0.465000in}{0.449444in}}%
\pgfpathclose%
\pgfusepath{fill}%
\end{pgfscope}%
\begin{pgfscope}%
\pgfpathrectangle{\pgfqpoint{0.465000in}{0.449444in}}{\pgfqpoint{1.550000in}{1.155000in}}%
\pgfusepath{clip}%
\pgfsetbuttcap%
\pgfsetmiterjoin%
\pgfsetlinewidth{1.003750pt}%
\definecolor{currentstroke}{rgb}{0.000000,0.000000,0.000000}%
\pgfsetstrokecolor{currentstroke}%
\pgfsetdash{}{0pt}%
\pgfpathmoveto{\pgfqpoint{0.455000in}{0.449444in}}%
\pgfpathlineto{\pgfqpoint{0.502805in}{0.449444in}}%
\pgfpathlineto{\pgfqpoint{0.502805in}{0.590081in}}%
\pgfpathlineto{\pgfqpoint{0.455000in}{0.590081in}}%
\pgfusepath{stroke}%
\end{pgfscope}%
\begin{pgfscope}%
\pgfpathrectangle{\pgfqpoint{0.465000in}{0.449444in}}{\pgfqpoint{1.550000in}{1.155000in}}%
\pgfusepath{clip}%
\pgfsetbuttcap%
\pgfsetmiterjoin%
\pgfsetlinewidth{1.003750pt}%
\definecolor{currentstroke}{rgb}{0.000000,0.000000,0.000000}%
\pgfsetstrokecolor{currentstroke}%
\pgfsetdash{}{0pt}%
\pgfpathmoveto{\pgfqpoint{0.593537in}{0.449444in}}%
\pgfpathlineto{\pgfqpoint{0.654025in}{0.449444in}}%
\pgfpathlineto{\pgfqpoint{0.654025in}{1.204573in}}%
\pgfpathlineto{\pgfqpoint{0.593537in}{1.204573in}}%
\pgfpathlineto{\pgfqpoint{0.593537in}{0.449444in}}%
\pgfpathclose%
\pgfusepath{stroke}%
\end{pgfscope}%
\begin{pgfscope}%
\pgfpathrectangle{\pgfqpoint{0.465000in}{0.449444in}}{\pgfqpoint{1.550000in}{1.155000in}}%
\pgfusepath{clip}%
\pgfsetbuttcap%
\pgfsetmiterjoin%
\pgfsetlinewidth{1.003750pt}%
\definecolor{currentstroke}{rgb}{0.000000,0.000000,0.000000}%
\pgfsetstrokecolor{currentstroke}%
\pgfsetdash{}{0pt}%
\pgfpathmoveto{\pgfqpoint{0.744756in}{0.449444in}}%
\pgfpathlineto{\pgfqpoint{0.805244in}{0.449444in}}%
\pgfpathlineto{\pgfqpoint{0.805244in}{1.549444in}}%
\pgfpathlineto{\pgfqpoint{0.744756in}{1.549444in}}%
\pgfpathlineto{\pgfqpoint{0.744756in}{0.449444in}}%
\pgfpathclose%
\pgfusepath{stroke}%
\end{pgfscope}%
\begin{pgfscope}%
\pgfpathrectangle{\pgfqpoint{0.465000in}{0.449444in}}{\pgfqpoint{1.550000in}{1.155000in}}%
\pgfusepath{clip}%
\pgfsetbuttcap%
\pgfsetmiterjoin%
\pgfsetlinewidth{1.003750pt}%
\definecolor{currentstroke}{rgb}{0.000000,0.000000,0.000000}%
\pgfsetstrokecolor{currentstroke}%
\pgfsetdash{}{0pt}%
\pgfpathmoveto{\pgfqpoint{0.895976in}{0.449444in}}%
\pgfpathlineto{\pgfqpoint{0.956464in}{0.449444in}}%
\pgfpathlineto{\pgfqpoint{0.956464in}{1.447255in}}%
\pgfpathlineto{\pgfqpoint{0.895976in}{1.447255in}}%
\pgfpathlineto{\pgfqpoint{0.895976in}{0.449444in}}%
\pgfpathclose%
\pgfusepath{stroke}%
\end{pgfscope}%
\begin{pgfscope}%
\pgfpathrectangle{\pgfqpoint{0.465000in}{0.449444in}}{\pgfqpoint{1.550000in}{1.155000in}}%
\pgfusepath{clip}%
\pgfsetbuttcap%
\pgfsetmiterjoin%
\pgfsetlinewidth{1.003750pt}%
\definecolor{currentstroke}{rgb}{0.000000,0.000000,0.000000}%
\pgfsetstrokecolor{currentstroke}%
\pgfsetdash{}{0pt}%
\pgfpathmoveto{\pgfqpoint{1.047195in}{0.449444in}}%
\pgfpathlineto{\pgfqpoint{1.107683in}{0.449444in}}%
\pgfpathlineto{\pgfqpoint{1.107683in}{1.160058in}}%
\pgfpathlineto{\pgfqpoint{1.047195in}{1.160058in}}%
\pgfpathlineto{\pgfqpoint{1.047195in}{0.449444in}}%
\pgfpathclose%
\pgfusepath{stroke}%
\end{pgfscope}%
\begin{pgfscope}%
\pgfpathrectangle{\pgfqpoint{0.465000in}{0.449444in}}{\pgfqpoint{1.550000in}{1.155000in}}%
\pgfusepath{clip}%
\pgfsetbuttcap%
\pgfsetmiterjoin%
\pgfsetlinewidth{1.003750pt}%
\definecolor{currentstroke}{rgb}{0.000000,0.000000,0.000000}%
\pgfsetstrokecolor{currentstroke}%
\pgfsetdash{}{0pt}%
\pgfpathmoveto{\pgfqpoint{1.198415in}{0.449444in}}%
\pgfpathlineto{\pgfqpoint{1.258903in}{0.449444in}}%
\pgfpathlineto{\pgfqpoint{1.258903in}{0.885778in}}%
\pgfpathlineto{\pgfqpoint{1.198415in}{0.885778in}}%
\pgfpathlineto{\pgfqpoint{1.198415in}{0.449444in}}%
\pgfpathclose%
\pgfusepath{stroke}%
\end{pgfscope}%
\begin{pgfscope}%
\pgfpathrectangle{\pgfqpoint{0.465000in}{0.449444in}}{\pgfqpoint{1.550000in}{1.155000in}}%
\pgfusepath{clip}%
\pgfsetbuttcap%
\pgfsetmiterjoin%
\pgfsetlinewidth{1.003750pt}%
\definecolor{currentstroke}{rgb}{0.000000,0.000000,0.000000}%
\pgfsetstrokecolor{currentstroke}%
\pgfsetdash{}{0pt}%
\pgfpathmoveto{\pgfqpoint{1.349634in}{0.449444in}}%
\pgfpathlineto{\pgfqpoint{1.410122in}{0.449444in}}%
\pgfpathlineto{\pgfqpoint{1.410122in}{0.731988in}}%
\pgfpathlineto{\pgfqpoint{1.349634in}{0.731988in}}%
\pgfpathlineto{\pgfqpoint{1.349634in}{0.449444in}}%
\pgfpathclose%
\pgfusepath{stroke}%
\end{pgfscope}%
\begin{pgfscope}%
\pgfpathrectangle{\pgfqpoint{0.465000in}{0.449444in}}{\pgfqpoint{1.550000in}{1.155000in}}%
\pgfusepath{clip}%
\pgfsetbuttcap%
\pgfsetmiterjoin%
\pgfsetlinewidth{1.003750pt}%
\definecolor{currentstroke}{rgb}{0.000000,0.000000,0.000000}%
\pgfsetstrokecolor{currentstroke}%
\pgfsetdash{}{0pt}%
\pgfpathmoveto{\pgfqpoint{1.500854in}{0.449444in}}%
\pgfpathlineto{\pgfqpoint{1.561342in}{0.449444in}}%
\pgfpathlineto{\pgfqpoint{1.561342in}{0.577885in}}%
\pgfpathlineto{\pgfqpoint{1.500854in}{0.577885in}}%
\pgfpathlineto{\pgfqpoint{1.500854in}{0.449444in}}%
\pgfpathclose%
\pgfusepath{stroke}%
\end{pgfscope}%
\begin{pgfscope}%
\pgfpathrectangle{\pgfqpoint{0.465000in}{0.449444in}}{\pgfqpoint{1.550000in}{1.155000in}}%
\pgfusepath{clip}%
\pgfsetbuttcap%
\pgfsetmiterjoin%
\pgfsetlinewidth{1.003750pt}%
\definecolor{currentstroke}{rgb}{0.000000,0.000000,0.000000}%
\pgfsetstrokecolor{currentstroke}%
\pgfsetdash{}{0pt}%
\pgfpathmoveto{\pgfqpoint{1.652073in}{0.449444in}}%
\pgfpathlineto{\pgfqpoint{1.712561in}{0.449444in}}%
\pgfpathlineto{\pgfqpoint{1.712561in}{0.513416in}}%
\pgfpathlineto{\pgfqpoint{1.652073in}{0.513416in}}%
\pgfpathlineto{\pgfqpoint{1.652073in}{0.449444in}}%
\pgfpathclose%
\pgfusepath{stroke}%
\end{pgfscope}%
\begin{pgfscope}%
\pgfpathrectangle{\pgfqpoint{0.465000in}{0.449444in}}{\pgfqpoint{1.550000in}{1.155000in}}%
\pgfusepath{clip}%
\pgfsetbuttcap%
\pgfsetmiterjoin%
\pgfsetlinewidth{1.003750pt}%
\definecolor{currentstroke}{rgb}{0.000000,0.000000,0.000000}%
\pgfsetstrokecolor{currentstroke}%
\pgfsetdash{}{0pt}%
\pgfpathmoveto{\pgfqpoint{1.803293in}{0.449444in}}%
\pgfpathlineto{\pgfqpoint{1.863781in}{0.449444in}}%
\pgfpathlineto{\pgfqpoint{1.863781in}{0.467616in}}%
\pgfpathlineto{\pgfqpoint{1.803293in}{0.467616in}}%
\pgfpathlineto{\pgfqpoint{1.803293in}{0.449444in}}%
\pgfpathclose%
\pgfusepath{stroke}%
\end{pgfscope}%
\begin{pgfscope}%
\pgfpathrectangle{\pgfqpoint{0.465000in}{0.449444in}}{\pgfqpoint{1.550000in}{1.155000in}}%
\pgfusepath{clip}%
\pgfsetbuttcap%
\pgfsetmiterjoin%
\definecolor{currentfill}{rgb}{0.000000,0.000000,0.000000}%
\pgfsetfillcolor{currentfill}%
\pgfsetlinewidth{0.000000pt}%
\definecolor{currentstroke}{rgb}{0.000000,0.000000,0.000000}%
\pgfsetstrokecolor{currentstroke}%
\pgfsetstrokeopacity{0.000000}%
\pgfsetdash{}{0pt}%
\pgfpathmoveto{\pgfqpoint{0.502805in}{0.449444in}}%
\pgfpathlineto{\pgfqpoint{0.563293in}{0.449444in}}%
\pgfpathlineto{\pgfqpoint{0.563293in}{0.456162in}}%
\pgfpathlineto{\pgfqpoint{0.502805in}{0.456162in}}%
\pgfpathlineto{\pgfqpoint{0.502805in}{0.449444in}}%
\pgfpathclose%
\pgfusepath{fill}%
\end{pgfscope}%
\begin{pgfscope}%
\pgfpathrectangle{\pgfqpoint{0.465000in}{0.449444in}}{\pgfqpoint{1.550000in}{1.155000in}}%
\pgfusepath{clip}%
\pgfsetbuttcap%
\pgfsetmiterjoin%
\definecolor{currentfill}{rgb}{0.000000,0.000000,0.000000}%
\pgfsetfillcolor{currentfill}%
\pgfsetlinewidth{0.000000pt}%
\definecolor{currentstroke}{rgb}{0.000000,0.000000,0.000000}%
\pgfsetstrokecolor{currentstroke}%
\pgfsetstrokeopacity{0.000000}%
\pgfsetdash{}{0pt}%
\pgfpathmoveto{\pgfqpoint{0.654025in}{0.449444in}}%
\pgfpathlineto{\pgfqpoint{0.714512in}{0.449444in}}%
\pgfpathlineto{\pgfqpoint{0.714512in}{0.462803in}}%
\pgfpathlineto{\pgfqpoint{0.654025in}{0.462803in}}%
\pgfpathlineto{\pgfqpoint{0.654025in}{0.449444in}}%
\pgfpathclose%
\pgfusepath{fill}%
\end{pgfscope}%
\begin{pgfscope}%
\pgfpathrectangle{\pgfqpoint{0.465000in}{0.449444in}}{\pgfqpoint{1.550000in}{1.155000in}}%
\pgfusepath{clip}%
\pgfsetbuttcap%
\pgfsetmiterjoin%
\definecolor{currentfill}{rgb}{0.000000,0.000000,0.000000}%
\pgfsetfillcolor{currentfill}%
\pgfsetlinewidth{0.000000pt}%
\definecolor{currentstroke}{rgb}{0.000000,0.000000,0.000000}%
\pgfsetstrokecolor{currentstroke}%
\pgfsetstrokeopacity{0.000000}%
\pgfsetdash{}{0pt}%
\pgfpathmoveto{\pgfqpoint{0.805244in}{0.449444in}}%
\pgfpathlineto{\pgfqpoint{0.865732in}{0.449444in}}%
\pgfpathlineto{\pgfqpoint{0.865732in}{0.475374in}}%
\pgfpathlineto{\pgfqpoint{0.805244in}{0.475374in}}%
\pgfpathlineto{\pgfqpoint{0.805244in}{0.449444in}}%
\pgfpathclose%
\pgfusepath{fill}%
\end{pgfscope}%
\begin{pgfscope}%
\pgfpathrectangle{\pgfqpoint{0.465000in}{0.449444in}}{\pgfqpoint{1.550000in}{1.155000in}}%
\pgfusepath{clip}%
\pgfsetbuttcap%
\pgfsetmiterjoin%
\definecolor{currentfill}{rgb}{0.000000,0.000000,0.000000}%
\pgfsetfillcolor{currentfill}%
\pgfsetlinewidth{0.000000pt}%
\definecolor{currentstroke}{rgb}{0.000000,0.000000,0.000000}%
\pgfsetstrokecolor{currentstroke}%
\pgfsetstrokeopacity{0.000000}%
\pgfsetdash{}{0pt}%
\pgfpathmoveto{\pgfqpoint{0.956464in}{0.449444in}}%
\pgfpathlineto{\pgfqpoint{1.016951in}{0.449444in}}%
\pgfpathlineto{\pgfqpoint{1.016951in}{0.496652in}}%
\pgfpathlineto{\pgfqpoint{0.956464in}{0.496652in}}%
\pgfpathlineto{\pgfqpoint{0.956464in}{0.449444in}}%
\pgfpathclose%
\pgfusepath{fill}%
\end{pgfscope}%
\begin{pgfscope}%
\pgfpathrectangle{\pgfqpoint{0.465000in}{0.449444in}}{\pgfqpoint{1.550000in}{1.155000in}}%
\pgfusepath{clip}%
\pgfsetbuttcap%
\pgfsetmiterjoin%
\definecolor{currentfill}{rgb}{0.000000,0.000000,0.000000}%
\pgfsetfillcolor{currentfill}%
\pgfsetlinewidth{0.000000pt}%
\definecolor{currentstroke}{rgb}{0.000000,0.000000,0.000000}%
\pgfsetstrokecolor{currentstroke}%
\pgfsetstrokeopacity{0.000000}%
\pgfsetdash{}{0pt}%
\pgfpathmoveto{\pgfqpoint{1.107683in}{0.449444in}}%
\pgfpathlineto{\pgfqpoint{1.168171in}{0.449444in}}%
\pgfpathlineto{\pgfqpoint{1.168171in}{0.530884in}}%
\pgfpathlineto{\pgfqpoint{1.107683in}{0.530884in}}%
\pgfpathlineto{\pgfqpoint{1.107683in}{0.449444in}}%
\pgfpathclose%
\pgfusepath{fill}%
\end{pgfscope}%
\begin{pgfscope}%
\pgfpathrectangle{\pgfqpoint{0.465000in}{0.449444in}}{\pgfqpoint{1.550000in}{1.155000in}}%
\pgfusepath{clip}%
\pgfsetbuttcap%
\pgfsetmiterjoin%
\definecolor{currentfill}{rgb}{0.000000,0.000000,0.000000}%
\pgfsetfillcolor{currentfill}%
\pgfsetlinewidth{0.000000pt}%
\definecolor{currentstroke}{rgb}{0.000000,0.000000,0.000000}%
\pgfsetstrokecolor{currentstroke}%
\pgfsetstrokeopacity{0.000000}%
\pgfsetdash{}{0pt}%
\pgfpathmoveto{\pgfqpoint{1.258903in}{0.449444in}}%
\pgfpathlineto{\pgfqpoint{1.319391in}{0.449444in}}%
\pgfpathlineto{\pgfqpoint{1.319391in}{0.576975in}}%
\pgfpathlineto{\pgfqpoint{1.258903in}{0.576975in}}%
\pgfpathlineto{\pgfqpoint{1.258903in}{0.449444in}}%
\pgfpathclose%
\pgfusepath{fill}%
\end{pgfscope}%
\begin{pgfscope}%
\pgfpathrectangle{\pgfqpoint{0.465000in}{0.449444in}}{\pgfqpoint{1.550000in}{1.155000in}}%
\pgfusepath{clip}%
\pgfsetbuttcap%
\pgfsetmiterjoin%
\definecolor{currentfill}{rgb}{0.000000,0.000000,0.000000}%
\pgfsetfillcolor{currentfill}%
\pgfsetlinewidth{0.000000pt}%
\definecolor{currentstroke}{rgb}{0.000000,0.000000,0.000000}%
\pgfsetstrokecolor{currentstroke}%
\pgfsetstrokeopacity{0.000000}%
\pgfsetdash{}{0pt}%
\pgfpathmoveto{\pgfqpoint{1.410122in}{0.449444in}}%
\pgfpathlineto{\pgfqpoint{1.470610in}{0.449444in}}%
\pgfpathlineto{\pgfqpoint{1.470610in}{0.624175in}}%
\pgfpathlineto{\pgfqpoint{1.410122in}{0.624175in}}%
\pgfpathlineto{\pgfqpoint{1.410122in}{0.449444in}}%
\pgfpathclose%
\pgfusepath{fill}%
\end{pgfscope}%
\begin{pgfscope}%
\pgfpathrectangle{\pgfqpoint{0.465000in}{0.449444in}}{\pgfqpoint{1.550000in}{1.155000in}}%
\pgfusepath{clip}%
\pgfsetbuttcap%
\pgfsetmiterjoin%
\definecolor{currentfill}{rgb}{0.000000,0.000000,0.000000}%
\pgfsetfillcolor{currentfill}%
\pgfsetlinewidth{0.000000pt}%
\definecolor{currentstroke}{rgb}{0.000000,0.000000,0.000000}%
\pgfsetstrokecolor{currentstroke}%
\pgfsetstrokeopacity{0.000000}%
\pgfsetdash{}{0pt}%
\pgfpathmoveto{\pgfqpoint{1.561342in}{0.449444in}}%
\pgfpathlineto{\pgfqpoint{1.621830in}{0.449444in}}%
\pgfpathlineto{\pgfqpoint{1.621830in}{0.643181in}}%
\pgfpathlineto{\pgfqpoint{1.561342in}{0.643181in}}%
\pgfpathlineto{\pgfqpoint{1.561342in}{0.449444in}}%
\pgfpathclose%
\pgfusepath{fill}%
\end{pgfscope}%
\begin{pgfscope}%
\pgfpathrectangle{\pgfqpoint{0.465000in}{0.449444in}}{\pgfqpoint{1.550000in}{1.155000in}}%
\pgfusepath{clip}%
\pgfsetbuttcap%
\pgfsetmiterjoin%
\definecolor{currentfill}{rgb}{0.000000,0.000000,0.000000}%
\pgfsetfillcolor{currentfill}%
\pgfsetlinewidth{0.000000pt}%
\definecolor{currentstroke}{rgb}{0.000000,0.000000,0.000000}%
\pgfsetstrokecolor{currentstroke}%
\pgfsetstrokeopacity{0.000000}%
\pgfsetdash{}{0pt}%
\pgfpathmoveto{\pgfqpoint{1.712561in}{0.449444in}}%
\pgfpathlineto{\pgfqpoint{1.773049in}{0.449444in}}%
\pgfpathlineto{\pgfqpoint{1.773049in}{0.579591in}}%
\pgfpathlineto{\pgfqpoint{1.712561in}{0.579591in}}%
\pgfpathlineto{\pgfqpoint{1.712561in}{0.449444in}}%
\pgfpathclose%
\pgfusepath{fill}%
\end{pgfscope}%
\begin{pgfscope}%
\pgfpathrectangle{\pgfqpoint{0.465000in}{0.449444in}}{\pgfqpoint{1.550000in}{1.155000in}}%
\pgfusepath{clip}%
\pgfsetbuttcap%
\pgfsetmiterjoin%
\definecolor{currentfill}{rgb}{0.000000,0.000000,0.000000}%
\pgfsetfillcolor{currentfill}%
\pgfsetlinewidth{0.000000pt}%
\definecolor{currentstroke}{rgb}{0.000000,0.000000,0.000000}%
\pgfsetstrokecolor{currentstroke}%
\pgfsetstrokeopacity{0.000000}%
\pgfsetdash{}{0pt}%
\pgfpathmoveto{\pgfqpoint{1.863781in}{0.449444in}}%
\pgfpathlineto{\pgfqpoint{1.924269in}{0.449444in}}%
\pgfpathlineto{\pgfqpoint{1.924269in}{0.474640in}}%
\pgfpathlineto{\pgfqpoint{1.863781in}{0.474640in}}%
\pgfpathlineto{\pgfqpoint{1.863781in}{0.449444in}}%
\pgfpathclose%
\pgfusepath{fill}%
\end{pgfscope}%
\begin{pgfscope}%
\pgfsetbuttcap%
\pgfsetroundjoin%
\definecolor{currentfill}{rgb}{0.000000,0.000000,0.000000}%
\pgfsetfillcolor{currentfill}%
\pgfsetlinewidth{0.803000pt}%
\definecolor{currentstroke}{rgb}{0.000000,0.000000,0.000000}%
\pgfsetstrokecolor{currentstroke}%
\pgfsetdash{}{0pt}%
\pgfsys@defobject{currentmarker}{\pgfqpoint{0.000000in}{-0.048611in}}{\pgfqpoint{0.000000in}{0.000000in}}{%
\pgfpathmoveto{\pgfqpoint{0.000000in}{0.000000in}}%
\pgfpathlineto{\pgfqpoint{0.000000in}{-0.048611in}}%
\pgfusepath{stroke,fill}%
}%
\begin{pgfscope}%
\pgfsys@transformshift{0.502805in}{0.449444in}%
\pgfsys@useobject{currentmarker}{}%
\end{pgfscope}%
\end{pgfscope}%
\begin{pgfscope}%
\definecolor{textcolor}{rgb}{0.000000,0.000000,0.000000}%
\pgfsetstrokecolor{textcolor}%
\pgfsetfillcolor{textcolor}%
\pgftext[x=0.502805in,y=0.352222in,,top]{\color{textcolor}\rmfamily\fontsize{10.000000}{12.000000}\selectfont 0.0}%
\end{pgfscope}%
\begin{pgfscope}%
\pgfsetbuttcap%
\pgfsetroundjoin%
\definecolor{currentfill}{rgb}{0.000000,0.000000,0.000000}%
\pgfsetfillcolor{currentfill}%
\pgfsetlinewidth{0.803000pt}%
\definecolor{currentstroke}{rgb}{0.000000,0.000000,0.000000}%
\pgfsetstrokecolor{currentstroke}%
\pgfsetdash{}{0pt}%
\pgfsys@defobject{currentmarker}{\pgfqpoint{0.000000in}{-0.048611in}}{\pgfqpoint{0.000000in}{0.000000in}}{%
\pgfpathmoveto{\pgfqpoint{0.000000in}{0.000000in}}%
\pgfpathlineto{\pgfqpoint{0.000000in}{-0.048611in}}%
\pgfusepath{stroke,fill}%
}%
\begin{pgfscope}%
\pgfsys@transformshift{0.880854in}{0.449444in}%
\pgfsys@useobject{currentmarker}{}%
\end{pgfscope}%
\end{pgfscope}%
\begin{pgfscope}%
\definecolor{textcolor}{rgb}{0.000000,0.000000,0.000000}%
\pgfsetstrokecolor{textcolor}%
\pgfsetfillcolor{textcolor}%
\pgftext[x=0.880854in,y=0.352222in,,top]{\color{textcolor}\rmfamily\fontsize{10.000000}{12.000000}\selectfont 0.25}%
\end{pgfscope}%
\begin{pgfscope}%
\pgfsetbuttcap%
\pgfsetroundjoin%
\definecolor{currentfill}{rgb}{0.000000,0.000000,0.000000}%
\pgfsetfillcolor{currentfill}%
\pgfsetlinewidth{0.803000pt}%
\definecolor{currentstroke}{rgb}{0.000000,0.000000,0.000000}%
\pgfsetstrokecolor{currentstroke}%
\pgfsetdash{}{0pt}%
\pgfsys@defobject{currentmarker}{\pgfqpoint{0.000000in}{-0.048611in}}{\pgfqpoint{0.000000in}{0.000000in}}{%
\pgfpathmoveto{\pgfqpoint{0.000000in}{0.000000in}}%
\pgfpathlineto{\pgfqpoint{0.000000in}{-0.048611in}}%
\pgfusepath{stroke,fill}%
}%
\begin{pgfscope}%
\pgfsys@transformshift{1.258903in}{0.449444in}%
\pgfsys@useobject{currentmarker}{}%
\end{pgfscope}%
\end{pgfscope}%
\begin{pgfscope}%
\definecolor{textcolor}{rgb}{0.000000,0.000000,0.000000}%
\pgfsetstrokecolor{textcolor}%
\pgfsetfillcolor{textcolor}%
\pgftext[x=1.258903in,y=0.352222in,,top]{\color{textcolor}\rmfamily\fontsize{10.000000}{12.000000}\selectfont 0.5}%
\end{pgfscope}%
\begin{pgfscope}%
\pgfsetbuttcap%
\pgfsetroundjoin%
\definecolor{currentfill}{rgb}{0.000000,0.000000,0.000000}%
\pgfsetfillcolor{currentfill}%
\pgfsetlinewidth{0.803000pt}%
\definecolor{currentstroke}{rgb}{0.000000,0.000000,0.000000}%
\pgfsetstrokecolor{currentstroke}%
\pgfsetdash{}{0pt}%
\pgfsys@defobject{currentmarker}{\pgfqpoint{0.000000in}{-0.048611in}}{\pgfqpoint{0.000000in}{0.000000in}}{%
\pgfpathmoveto{\pgfqpoint{0.000000in}{0.000000in}}%
\pgfpathlineto{\pgfqpoint{0.000000in}{-0.048611in}}%
\pgfusepath{stroke,fill}%
}%
\begin{pgfscope}%
\pgfsys@transformshift{1.636951in}{0.449444in}%
\pgfsys@useobject{currentmarker}{}%
\end{pgfscope}%
\end{pgfscope}%
\begin{pgfscope}%
\definecolor{textcolor}{rgb}{0.000000,0.000000,0.000000}%
\pgfsetstrokecolor{textcolor}%
\pgfsetfillcolor{textcolor}%
\pgftext[x=1.636951in,y=0.352222in,,top]{\color{textcolor}\rmfamily\fontsize{10.000000}{12.000000}\selectfont 0.75}%
\end{pgfscope}%
\begin{pgfscope}%
\pgfsetbuttcap%
\pgfsetroundjoin%
\definecolor{currentfill}{rgb}{0.000000,0.000000,0.000000}%
\pgfsetfillcolor{currentfill}%
\pgfsetlinewidth{0.803000pt}%
\definecolor{currentstroke}{rgb}{0.000000,0.000000,0.000000}%
\pgfsetstrokecolor{currentstroke}%
\pgfsetdash{}{0pt}%
\pgfsys@defobject{currentmarker}{\pgfqpoint{0.000000in}{-0.048611in}}{\pgfqpoint{0.000000in}{0.000000in}}{%
\pgfpathmoveto{\pgfqpoint{0.000000in}{0.000000in}}%
\pgfpathlineto{\pgfqpoint{0.000000in}{-0.048611in}}%
\pgfusepath{stroke,fill}%
}%
\begin{pgfscope}%
\pgfsys@transformshift{2.015000in}{0.449444in}%
\pgfsys@useobject{currentmarker}{}%
\end{pgfscope}%
\end{pgfscope}%
\begin{pgfscope}%
\definecolor{textcolor}{rgb}{0.000000,0.000000,0.000000}%
\pgfsetstrokecolor{textcolor}%
\pgfsetfillcolor{textcolor}%
\pgftext[x=2.015000in,y=0.352222in,,top]{\color{textcolor}\rmfamily\fontsize{10.000000}{12.000000}\selectfont 1.0}%
\end{pgfscope}%
\begin{pgfscope}%
\definecolor{textcolor}{rgb}{0.000000,0.000000,0.000000}%
\pgfsetstrokecolor{textcolor}%
\pgfsetfillcolor{textcolor}%
\pgftext[x=1.240000in,y=0.173333in,,top]{\color{textcolor}\rmfamily\fontsize{10.000000}{12.000000}\selectfont \(\displaystyle p\)}%
\end{pgfscope}%
\begin{pgfscope}%
\pgfsetbuttcap%
\pgfsetroundjoin%
\definecolor{currentfill}{rgb}{0.000000,0.000000,0.000000}%
\pgfsetfillcolor{currentfill}%
\pgfsetlinewidth{0.803000pt}%
\definecolor{currentstroke}{rgb}{0.000000,0.000000,0.000000}%
\pgfsetstrokecolor{currentstroke}%
\pgfsetdash{}{0pt}%
\pgfsys@defobject{currentmarker}{\pgfqpoint{-0.048611in}{0.000000in}}{\pgfqpoint{-0.000000in}{0.000000in}}{%
\pgfpathmoveto{\pgfqpoint{-0.000000in}{0.000000in}}%
\pgfpathlineto{\pgfqpoint{-0.048611in}{0.000000in}}%
\pgfusepath{stroke,fill}%
}%
\begin{pgfscope}%
\pgfsys@transformshift{0.465000in}{0.449444in}%
\pgfsys@useobject{currentmarker}{}%
\end{pgfscope}%
\end{pgfscope}%
\begin{pgfscope}%
\definecolor{textcolor}{rgb}{0.000000,0.000000,0.000000}%
\pgfsetstrokecolor{textcolor}%
\pgfsetfillcolor{textcolor}%
\pgftext[x=0.298333in, y=0.401250in, left, base]{\color{textcolor}\rmfamily\fontsize{10.000000}{12.000000}\selectfont \(\displaystyle {0}\)}%
\end{pgfscope}%
\begin{pgfscope}%
\pgfsetbuttcap%
\pgfsetroundjoin%
\definecolor{currentfill}{rgb}{0.000000,0.000000,0.000000}%
\pgfsetfillcolor{currentfill}%
\pgfsetlinewidth{0.803000pt}%
\definecolor{currentstroke}{rgb}{0.000000,0.000000,0.000000}%
\pgfsetstrokecolor{currentstroke}%
\pgfsetdash{}{0pt}%
\pgfsys@defobject{currentmarker}{\pgfqpoint{-0.048611in}{0.000000in}}{\pgfqpoint{-0.000000in}{0.000000in}}{%
\pgfpathmoveto{\pgfqpoint{-0.000000in}{0.000000in}}%
\pgfpathlineto{\pgfqpoint{-0.048611in}{0.000000in}}%
\pgfusepath{stroke,fill}%
}%
\begin{pgfscope}%
\pgfsys@transformshift{0.465000in}{0.995409in}%
\pgfsys@useobject{currentmarker}{}%
\end{pgfscope}%
\end{pgfscope}%
\begin{pgfscope}%
\definecolor{textcolor}{rgb}{0.000000,0.000000,0.000000}%
\pgfsetstrokecolor{textcolor}%
\pgfsetfillcolor{textcolor}%
\pgftext[x=0.228889in, y=0.947214in, left, base]{\color{textcolor}\rmfamily\fontsize{10.000000}{12.000000}\selectfont \(\displaystyle {10}\)}%
\end{pgfscope}%
\begin{pgfscope}%
\pgfsetbuttcap%
\pgfsetroundjoin%
\definecolor{currentfill}{rgb}{0.000000,0.000000,0.000000}%
\pgfsetfillcolor{currentfill}%
\pgfsetlinewidth{0.803000pt}%
\definecolor{currentstroke}{rgb}{0.000000,0.000000,0.000000}%
\pgfsetstrokecolor{currentstroke}%
\pgfsetdash{}{0pt}%
\pgfsys@defobject{currentmarker}{\pgfqpoint{-0.048611in}{0.000000in}}{\pgfqpoint{-0.000000in}{0.000000in}}{%
\pgfpathmoveto{\pgfqpoint{-0.000000in}{0.000000in}}%
\pgfpathlineto{\pgfqpoint{-0.048611in}{0.000000in}}%
\pgfusepath{stroke,fill}%
}%
\begin{pgfscope}%
\pgfsys@transformshift{0.465000in}{1.541374in}%
\pgfsys@useobject{currentmarker}{}%
\end{pgfscope}%
\end{pgfscope}%
\begin{pgfscope}%
\definecolor{textcolor}{rgb}{0.000000,0.000000,0.000000}%
\pgfsetstrokecolor{textcolor}%
\pgfsetfillcolor{textcolor}%
\pgftext[x=0.228889in, y=1.493179in, left, base]{\color{textcolor}\rmfamily\fontsize{10.000000}{12.000000}\selectfont \(\displaystyle {20}\)}%
\end{pgfscope}%
\begin{pgfscope}%
\definecolor{textcolor}{rgb}{0.000000,0.000000,0.000000}%
\pgfsetstrokecolor{textcolor}%
\pgfsetfillcolor{textcolor}%
\pgftext[x=0.173333in,y=1.026944in,,bottom,rotate=90.000000]{\color{textcolor}\rmfamily\fontsize{10.000000}{12.000000}\selectfont Percent of Data Set}%
\end{pgfscope}%
\begin{pgfscope}%
\pgfsetrectcap%
\pgfsetmiterjoin%
\pgfsetlinewidth{0.803000pt}%
\definecolor{currentstroke}{rgb}{0.000000,0.000000,0.000000}%
\pgfsetstrokecolor{currentstroke}%
\pgfsetdash{}{0pt}%
\pgfpathmoveto{\pgfqpoint{0.465000in}{0.449444in}}%
\pgfpathlineto{\pgfqpoint{0.465000in}{1.604444in}}%
\pgfusepath{stroke}%
\end{pgfscope}%
\begin{pgfscope}%
\pgfsetrectcap%
\pgfsetmiterjoin%
\pgfsetlinewidth{0.803000pt}%
\definecolor{currentstroke}{rgb}{0.000000,0.000000,0.000000}%
\pgfsetstrokecolor{currentstroke}%
\pgfsetdash{}{0pt}%
\pgfpathmoveto{\pgfqpoint{2.015000in}{0.449444in}}%
\pgfpathlineto{\pgfqpoint{2.015000in}{1.604444in}}%
\pgfusepath{stroke}%
\end{pgfscope}%
\begin{pgfscope}%
\pgfsetrectcap%
\pgfsetmiterjoin%
\pgfsetlinewidth{0.803000pt}%
\definecolor{currentstroke}{rgb}{0.000000,0.000000,0.000000}%
\pgfsetstrokecolor{currentstroke}%
\pgfsetdash{}{0pt}%
\pgfpathmoveto{\pgfqpoint{0.465000in}{0.449444in}}%
\pgfpathlineto{\pgfqpoint{2.015000in}{0.449444in}}%
\pgfusepath{stroke}%
\end{pgfscope}%
\begin{pgfscope}%
\pgfsetrectcap%
\pgfsetmiterjoin%
\pgfsetlinewidth{0.803000pt}%
\definecolor{currentstroke}{rgb}{0.000000,0.000000,0.000000}%
\pgfsetstrokecolor{currentstroke}%
\pgfsetdash{}{0pt}%
\pgfpathmoveto{\pgfqpoint{0.465000in}{1.604444in}}%
\pgfpathlineto{\pgfqpoint{2.015000in}{1.604444in}}%
\pgfusepath{stroke}%
\end{pgfscope}%
\begin{pgfscope}%
\pgfsetbuttcap%
\pgfsetmiterjoin%
\definecolor{currentfill}{rgb}{1.000000,1.000000,1.000000}%
\pgfsetfillcolor{currentfill}%
\pgfsetfillopacity{0.800000}%
\pgfsetlinewidth{1.003750pt}%
\definecolor{currentstroke}{rgb}{0.800000,0.800000,0.800000}%
\pgfsetstrokecolor{currentstroke}%
\pgfsetstrokeopacity{0.800000}%
\pgfsetdash{}{0pt}%
\pgfpathmoveto{\pgfqpoint{1.238056in}{1.104445in}}%
\pgfpathlineto{\pgfqpoint{1.917778in}{1.104445in}}%
\pgfpathquadraticcurveto{\pgfqpoint{1.945556in}{1.104445in}}{\pgfqpoint{1.945556in}{1.132222in}}%
\pgfpathlineto{\pgfqpoint{1.945556in}{1.507222in}}%
\pgfpathquadraticcurveto{\pgfqpoint{1.945556in}{1.535000in}}{\pgfqpoint{1.917778in}{1.535000in}}%
\pgfpathlineto{\pgfqpoint{1.238056in}{1.535000in}}%
\pgfpathquadraticcurveto{\pgfqpoint{1.210278in}{1.535000in}}{\pgfqpoint{1.210278in}{1.507222in}}%
\pgfpathlineto{\pgfqpoint{1.210278in}{1.132222in}}%
\pgfpathquadraticcurveto{\pgfqpoint{1.210278in}{1.104445in}}{\pgfqpoint{1.238056in}{1.104445in}}%
\pgfpathlineto{\pgfqpoint{1.238056in}{1.104445in}}%
\pgfpathclose%
\pgfusepath{stroke,fill}%
\end{pgfscope}%
\begin{pgfscope}%
\pgfsetbuttcap%
\pgfsetmiterjoin%
\pgfsetlinewidth{1.003750pt}%
\definecolor{currentstroke}{rgb}{0.000000,0.000000,0.000000}%
\pgfsetstrokecolor{currentstroke}%
\pgfsetdash{}{0pt}%
\pgfpathmoveto{\pgfqpoint{1.265834in}{1.382222in}}%
\pgfpathlineto{\pgfqpoint{1.543611in}{1.382222in}}%
\pgfpathlineto{\pgfqpoint{1.543611in}{1.479444in}}%
\pgfpathlineto{\pgfqpoint{1.265834in}{1.479444in}}%
\pgfpathlineto{\pgfqpoint{1.265834in}{1.382222in}}%
\pgfpathclose%
\pgfusepath{stroke}%
\end{pgfscope}%
\begin{pgfscope}%
\definecolor{textcolor}{rgb}{0.000000,0.000000,0.000000}%
\pgfsetstrokecolor{textcolor}%
\pgfsetfillcolor{textcolor}%
\pgftext[x=1.654722in,y=1.382222in,left,base]{\color{textcolor}\rmfamily\fontsize{10.000000}{12.000000}\selectfont Neg}%
\end{pgfscope}%
\begin{pgfscope}%
\pgfsetbuttcap%
\pgfsetmiterjoin%
\definecolor{currentfill}{rgb}{0.000000,0.000000,0.000000}%
\pgfsetfillcolor{currentfill}%
\pgfsetlinewidth{0.000000pt}%
\definecolor{currentstroke}{rgb}{0.000000,0.000000,0.000000}%
\pgfsetstrokecolor{currentstroke}%
\pgfsetstrokeopacity{0.000000}%
\pgfsetdash{}{0pt}%
\pgfpathmoveto{\pgfqpoint{1.265834in}{1.186944in}}%
\pgfpathlineto{\pgfqpoint{1.543611in}{1.186944in}}%
\pgfpathlineto{\pgfqpoint{1.543611in}{1.284167in}}%
\pgfpathlineto{\pgfqpoint{1.265834in}{1.284167in}}%
\pgfpathlineto{\pgfqpoint{1.265834in}{1.186944in}}%
\pgfpathclose%
\pgfusepath{fill}%
\end{pgfscope}%
\begin{pgfscope}%
\definecolor{textcolor}{rgb}{0.000000,0.000000,0.000000}%
\pgfsetstrokecolor{textcolor}%
\pgfsetfillcolor{textcolor}%
\pgftext[x=1.654722in,y=1.186944in,left,base]{\color{textcolor}\rmfamily\fontsize{10.000000}{12.000000}\selectfont Pos}%
\end{pgfscope}%
\end{pgfpicture}%
\makeatother%
\endgroup%
	
&
	\vskip 0pt
	
	\hfil {\normalfont\normalsize Right tail of $p$ for the Example Model}
		
	\hskip 12pt%% Creator: Matplotlib, PGF backend
%%
%% To include the figure in your LaTeX document, write
%%   \input{<filename>.pgf}
%%
%% Make sure the required packages are loaded in your preamble
%%   \usepackage{pgf}
%%
%% Also ensure that all the required font packages are loaded; for instance,
%% the lmodern package is sometimes necessary when using math font.
%%   \usepackage{lmodern}
%%
%% Figures using additional raster images can only be included by \input if
%% they are in the same directory as the main LaTeX file. For loading figures
%% from other directories you can use the `import` package
%%   \usepackage{import}
%%
%% and then include the figures with
%%   \import{<path to file>}{<filename>.pgf}
%%
%% Matplotlib used the following preamble
%%   
%%   \usepackage{fontspec}
%%   \makeatletter\@ifpackageloaded{underscore}{}{\usepackage[strings]{underscore}}\makeatother
%%
\begingroup%
\makeatletter%
\begin{pgfpicture}%
\pgfpathrectangle{\pgfpointorigin}{\pgfqpoint{4.006578in}{1.703778in}}%
\pgfusepath{use as bounding box, clip}%
\begin{pgfscope}%
\pgfsetbuttcap%
\pgfsetmiterjoin%
\definecolor{currentfill}{rgb}{1.000000,1.000000,1.000000}%
\pgfsetfillcolor{currentfill}%
\pgfsetlinewidth{0.000000pt}%
\definecolor{currentstroke}{rgb}{1.000000,1.000000,1.000000}%
\pgfsetstrokecolor{currentstroke}%
\pgfsetdash{}{0pt}%
\pgfpathmoveto{\pgfqpoint{0.000000in}{0.000000in}}%
\pgfpathlineto{\pgfqpoint{4.006578in}{0.000000in}}%
\pgfpathlineto{\pgfqpoint{4.006578in}{1.703777in}}%
\pgfpathlineto{\pgfqpoint{0.000000in}{1.703777in}}%
\pgfpathlineto{\pgfqpoint{0.000000in}{0.000000in}}%
\pgfpathclose%
\pgfusepath{fill}%
\end{pgfscope}%
\begin{pgfscope}%
\pgfsetbuttcap%
\pgfsetmiterjoin%
\definecolor{currentfill}{rgb}{1.000000,1.000000,1.000000}%
\pgfsetfillcolor{currentfill}%
\pgfsetlinewidth{0.000000pt}%
\definecolor{currentstroke}{rgb}{0.000000,0.000000,0.000000}%
\pgfsetstrokecolor{currentstroke}%
\pgfsetstrokeopacity{0.000000}%
\pgfsetdash{}{0pt}%
\pgfpathmoveto{\pgfqpoint{0.432374in}{0.498777in}}%
\pgfpathlineto{\pgfqpoint{3.919874in}{0.498777in}}%
\pgfpathlineto{\pgfqpoint{3.919874in}{1.653777in}}%
\pgfpathlineto{\pgfqpoint{0.432374in}{1.653777in}}%
\pgfpathlineto{\pgfqpoint{0.432374in}{0.498777in}}%
\pgfpathclose%
\pgfusepath{fill}%
\end{pgfscope}%
\begin{pgfscope}%
\pgfpathrectangle{\pgfqpoint{0.432374in}{0.498777in}}{\pgfqpoint{3.487500in}{1.155000in}}%
\pgfusepath{clip}%
\pgfsetbuttcap%
\pgfsetmiterjoin%
\pgfsetlinewidth{1.003750pt}%
\definecolor{currentstroke}{rgb}{0.000000,0.000000,0.000000}%
\pgfsetstrokecolor{currentstroke}%
\pgfsetdash{}{0pt}%
\pgfpathmoveto{\pgfqpoint{0.422374in}{0.498777in}}%
\pgfpathlineto{\pgfqpoint{0.474091in}{0.498777in}}%
\pgfpathlineto{\pgfqpoint{0.474091in}{0.498777in}}%
\pgfpathlineto{\pgfqpoint{0.422374in}{0.498777in}}%
\pgfusepath{stroke}%
\end{pgfscope}%
\begin{pgfscope}%
\pgfpathrectangle{\pgfqpoint{0.432374in}{0.498777in}}{\pgfqpoint{3.487500in}{1.155000in}}%
\pgfusepath{clip}%
\pgfsetbuttcap%
\pgfsetmiterjoin%
\pgfsetlinewidth{1.003750pt}%
\definecolor{currentstroke}{rgb}{0.000000,0.000000,0.000000}%
\pgfsetstrokecolor{currentstroke}%
\pgfsetdash{}{0pt}%
\pgfpathmoveto{\pgfqpoint{0.574210in}{0.498777in}}%
\pgfpathlineto{\pgfqpoint{0.640957in}{0.498777in}}%
\pgfpathlineto{\pgfqpoint{0.640957in}{1.598777in}}%
\pgfpathlineto{\pgfqpoint{0.574210in}{1.598777in}}%
\pgfpathlineto{\pgfqpoint{0.574210in}{0.498777in}}%
\pgfpathclose%
\pgfusepath{stroke}%
\end{pgfscope}%
\begin{pgfscope}%
\pgfpathrectangle{\pgfqpoint{0.432374in}{0.498777in}}{\pgfqpoint{3.487500in}{1.155000in}}%
\pgfusepath{clip}%
\pgfsetbuttcap%
\pgfsetmiterjoin%
\pgfsetlinewidth{1.003750pt}%
\definecolor{currentstroke}{rgb}{0.000000,0.000000,0.000000}%
\pgfsetstrokecolor{currentstroke}%
\pgfsetdash{}{0pt}%
\pgfpathmoveto{\pgfqpoint{0.741076in}{0.498777in}}%
\pgfpathlineto{\pgfqpoint{0.807823in}{0.498777in}}%
\pgfpathlineto{\pgfqpoint{0.807823in}{1.462126in}}%
\pgfpathlineto{\pgfqpoint{0.741076in}{1.462126in}}%
\pgfpathlineto{\pgfqpoint{0.741076in}{0.498777in}}%
\pgfpathclose%
\pgfusepath{stroke}%
\end{pgfscope}%
\begin{pgfscope}%
\pgfpathrectangle{\pgfqpoint{0.432374in}{0.498777in}}{\pgfqpoint{3.487500in}{1.155000in}}%
\pgfusepath{clip}%
\pgfsetbuttcap%
\pgfsetmiterjoin%
\pgfsetlinewidth{1.003750pt}%
\definecolor{currentstroke}{rgb}{0.000000,0.000000,0.000000}%
\pgfsetstrokecolor{currentstroke}%
\pgfsetdash{}{0pt}%
\pgfpathmoveto{\pgfqpoint{0.907942in}{0.498777in}}%
\pgfpathlineto{\pgfqpoint{0.974689in}{0.498777in}}%
\pgfpathlineto{\pgfqpoint{0.974689in}{1.346168in}}%
\pgfpathlineto{\pgfqpoint{0.907942in}{1.346168in}}%
\pgfpathlineto{\pgfqpoint{0.907942in}{0.498777in}}%
\pgfpathclose%
\pgfusepath{stroke}%
\end{pgfscope}%
\begin{pgfscope}%
\pgfpathrectangle{\pgfqpoint{0.432374in}{0.498777in}}{\pgfqpoint{3.487500in}{1.155000in}}%
\pgfusepath{clip}%
\pgfsetbuttcap%
\pgfsetmiterjoin%
\pgfsetlinewidth{1.003750pt}%
\definecolor{currentstroke}{rgb}{0.000000,0.000000,0.000000}%
\pgfsetstrokecolor{currentstroke}%
\pgfsetdash{}{0pt}%
\pgfpathmoveto{\pgfqpoint{1.074808in}{0.498777in}}%
\pgfpathlineto{\pgfqpoint{1.141555in}{0.498777in}}%
\pgfpathlineto{\pgfqpoint{1.141555in}{1.241550in}}%
\pgfpathlineto{\pgfqpoint{1.074808in}{1.241550in}}%
\pgfpathlineto{\pgfqpoint{1.074808in}{0.498777in}}%
\pgfpathclose%
\pgfusepath{stroke}%
\end{pgfscope}%
\begin{pgfscope}%
\pgfpathrectangle{\pgfqpoint{0.432374in}{0.498777in}}{\pgfqpoint{3.487500in}{1.155000in}}%
\pgfusepath{clip}%
\pgfsetbuttcap%
\pgfsetmiterjoin%
\pgfsetlinewidth{1.003750pt}%
\definecolor{currentstroke}{rgb}{0.000000,0.000000,0.000000}%
\pgfsetstrokecolor{currentstroke}%
\pgfsetdash{}{0pt}%
\pgfpathmoveto{\pgfqpoint{1.241674in}{0.498777in}}%
\pgfpathlineto{\pgfqpoint{1.308421in}{0.498777in}}%
\pgfpathlineto{\pgfqpoint{1.308421in}{1.450851in}}%
\pgfpathlineto{\pgfqpoint{1.241674in}{1.450851in}}%
\pgfpathlineto{\pgfqpoint{1.241674in}{0.498777in}}%
\pgfpathclose%
\pgfusepath{stroke}%
\end{pgfscope}%
\begin{pgfscope}%
\pgfpathrectangle{\pgfqpoint{0.432374in}{0.498777in}}{\pgfqpoint{3.487500in}{1.155000in}}%
\pgfusepath{clip}%
\pgfsetbuttcap%
\pgfsetmiterjoin%
\pgfsetlinewidth{1.003750pt}%
\definecolor{currentstroke}{rgb}{0.000000,0.000000,0.000000}%
\pgfsetstrokecolor{currentstroke}%
\pgfsetdash{}{0pt}%
\pgfpathmoveto{\pgfqpoint{1.408540in}{0.498777in}}%
\pgfpathlineto{\pgfqpoint{1.475287in}{0.498777in}}%
\pgfpathlineto{\pgfqpoint{1.475287in}{1.046279in}}%
\pgfpathlineto{\pgfqpoint{1.408540in}{1.046279in}}%
\pgfpathlineto{\pgfqpoint{1.408540in}{0.498777in}}%
\pgfpathclose%
\pgfusepath{stroke}%
\end{pgfscope}%
\begin{pgfscope}%
\pgfpathrectangle{\pgfqpoint{0.432374in}{0.498777in}}{\pgfqpoint{3.487500in}{1.155000in}}%
\pgfusepath{clip}%
\pgfsetbuttcap%
\pgfsetmiterjoin%
\pgfsetlinewidth{1.003750pt}%
\definecolor{currentstroke}{rgb}{0.000000,0.000000,0.000000}%
\pgfsetstrokecolor{currentstroke}%
\pgfsetdash{}{0pt}%
\pgfpathmoveto{\pgfqpoint{1.575406in}{0.498777in}}%
\pgfpathlineto{\pgfqpoint{1.642153in}{0.498777in}}%
\pgfpathlineto{\pgfqpoint{1.642153in}{0.962418in}}%
\pgfpathlineto{\pgfqpoint{1.575406in}{0.962418in}}%
\pgfpathlineto{\pgfqpoint{1.575406in}{0.498777in}}%
\pgfpathclose%
\pgfusepath{stroke}%
\end{pgfscope}%
\begin{pgfscope}%
\pgfpathrectangle{\pgfqpoint{0.432374in}{0.498777in}}{\pgfqpoint{3.487500in}{1.155000in}}%
\pgfusepath{clip}%
\pgfsetbuttcap%
\pgfsetmiterjoin%
\pgfsetlinewidth{1.003750pt}%
\definecolor{currentstroke}{rgb}{0.000000,0.000000,0.000000}%
\pgfsetstrokecolor{currentstroke}%
\pgfsetdash{}{0pt}%
\pgfpathmoveto{\pgfqpoint{1.742272in}{0.498777in}}%
\pgfpathlineto{\pgfqpoint{1.809019in}{0.498777in}}%
\pgfpathlineto{\pgfqpoint{1.809019in}{0.901363in}}%
\pgfpathlineto{\pgfqpoint{1.742272in}{0.901363in}}%
\pgfpathlineto{\pgfqpoint{1.742272in}{0.498777in}}%
\pgfpathclose%
\pgfusepath{stroke}%
\end{pgfscope}%
\begin{pgfscope}%
\pgfpathrectangle{\pgfqpoint{0.432374in}{0.498777in}}{\pgfqpoint{3.487500in}{1.155000in}}%
\pgfusepath{clip}%
\pgfsetbuttcap%
\pgfsetmiterjoin%
\pgfsetlinewidth{1.003750pt}%
\definecolor{currentstroke}{rgb}{0.000000,0.000000,0.000000}%
\pgfsetstrokecolor{currentstroke}%
\pgfsetdash{}{0pt}%
\pgfpathmoveto{\pgfqpoint{1.909138in}{0.498777in}}%
\pgfpathlineto{\pgfqpoint{1.975885in}{0.498777in}}%
\pgfpathlineto{\pgfqpoint{1.975885in}{0.838964in}}%
\pgfpathlineto{\pgfqpoint{1.909138in}{0.838964in}}%
\pgfpathlineto{\pgfqpoint{1.909138in}{0.498777in}}%
\pgfpathclose%
\pgfusepath{stroke}%
\end{pgfscope}%
\begin{pgfscope}%
\pgfpathrectangle{\pgfqpoint{0.432374in}{0.498777in}}{\pgfqpoint{3.487500in}{1.155000in}}%
\pgfusepath{clip}%
\pgfsetbuttcap%
\pgfsetmiterjoin%
\pgfsetlinewidth{1.003750pt}%
\definecolor{currentstroke}{rgb}{0.000000,0.000000,0.000000}%
\pgfsetstrokecolor{currentstroke}%
\pgfsetdash{}{0pt}%
\pgfpathmoveto{\pgfqpoint{2.076004in}{0.498777in}}%
\pgfpathlineto{\pgfqpoint{2.142751in}{0.498777in}}%
\pgfpathlineto{\pgfqpoint{2.142751in}{0.785726in}}%
\pgfpathlineto{\pgfqpoint{2.076004in}{0.785726in}}%
\pgfpathlineto{\pgfqpoint{2.076004in}{0.498777in}}%
\pgfpathclose%
\pgfusepath{stroke}%
\end{pgfscope}%
\begin{pgfscope}%
\pgfpathrectangle{\pgfqpoint{0.432374in}{0.498777in}}{\pgfqpoint{3.487500in}{1.155000in}}%
\pgfusepath{clip}%
\pgfsetbuttcap%
\pgfsetmiterjoin%
\pgfsetlinewidth{1.003750pt}%
\definecolor{currentstroke}{rgb}{0.000000,0.000000,0.000000}%
\pgfsetstrokecolor{currentstroke}%
\pgfsetdash{}{0pt}%
\pgfpathmoveto{\pgfqpoint{2.242870in}{0.498777in}}%
\pgfpathlineto{\pgfqpoint{2.309617in}{0.498777in}}%
\pgfpathlineto{\pgfqpoint{2.309617in}{0.740752in}}%
\pgfpathlineto{\pgfqpoint{2.242870in}{0.740752in}}%
\pgfpathlineto{\pgfqpoint{2.242870in}{0.498777in}}%
\pgfpathclose%
\pgfusepath{stroke}%
\end{pgfscope}%
\begin{pgfscope}%
\pgfpathrectangle{\pgfqpoint{0.432374in}{0.498777in}}{\pgfqpoint{3.487500in}{1.155000in}}%
\pgfusepath{clip}%
\pgfsetbuttcap%
\pgfsetmiterjoin%
\pgfsetlinewidth{1.003750pt}%
\definecolor{currentstroke}{rgb}{0.000000,0.000000,0.000000}%
\pgfsetstrokecolor{currentstroke}%
\pgfsetdash{}{0pt}%
\pgfpathmoveto{\pgfqpoint{2.409736in}{0.498777in}}%
\pgfpathlineto{\pgfqpoint{2.476483in}{0.498777in}}%
\pgfpathlineto{\pgfqpoint{2.476483in}{0.705132in}}%
\pgfpathlineto{\pgfqpoint{2.409736in}{0.705132in}}%
\pgfpathlineto{\pgfqpoint{2.409736in}{0.498777in}}%
\pgfpathclose%
\pgfusepath{stroke}%
\end{pgfscope}%
\begin{pgfscope}%
\pgfpathrectangle{\pgfqpoint{0.432374in}{0.498777in}}{\pgfqpoint{3.487500in}{1.155000in}}%
\pgfusepath{clip}%
\pgfsetbuttcap%
\pgfsetmiterjoin%
\pgfsetlinewidth{1.003750pt}%
\definecolor{currentstroke}{rgb}{0.000000,0.000000,0.000000}%
\pgfsetstrokecolor{currentstroke}%
\pgfsetdash{}{0pt}%
\pgfpathmoveto{\pgfqpoint{2.576602in}{0.498777in}}%
\pgfpathlineto{\pgfqpoint{2.643349in}{0.498777in}}%
\pgfpathlineto{\pgfqpoint{2.643349in}{0.672138in}}%
\pgfpathlineto{\pgfqpoint{2.576602in}{0.672138in}}%
\pgfpathlineto{\pgfqpoint{2.576602in}{0.498777in}}%
\pgfpathclose%
\pgfusepath{stroke}%
\end{pgfscope}%
\begin{pgfscope}%
\pgfpathrectangle{\pgfqpoint{0.432374in}{0.498777in}}{\pgfqpoint{3.487500in}{1.155000in}}%
\pgfusepath{clip}%
\pgfsetbuttcap%
\pgfsetmiterjoin%
\pgfsetlinewidth{1.003750pt}%
\definecolor{currentstroke}{rgb}{0.000000,0.000000,0.000000}%
\pgfsetstrokecolor{currentstroke}%
\pgfsetdash{}{0pt}%
\pgfpathmoveto{\pgfqpoint{2.743469in}{0.498777in}}%
\pgfpathlineto{\pgfqpoint{2.810215in}{0.498777in}}%
\pgfpathlineto{\pgfqpoint{2.810215in}{0.640490in}}%
\pgfpathlineto{\pgfqpoint{2.743469in}{0.640490in}}%
\pgfpathlineto{\pgfqpoint{2.743469in}{0.498777in}}%
\pgfpathclose%
\pgfusepath{stroke}%
\end{pgfscope}%
\begin{pgfscope}%
\pgfpathrectangle{\pgfqpoint{0.432374in}{0.498777in}}{\pgfqpoint{3.487500in}{1.155000in}}%
\pgfusepath{clip}%
\pgfsetbuttcap%
\pgfsetmiterjoin%
\pgfsetlinewidth{1.003750pt}%
\definecolor{currentstroke}{rgb}{0.000000,0.000000,0.000000}%
\pgfsetstrokecolor{currentstroke}%
\pgfsetdash{}{0pt}%
\pgfpathmoveto{\pgfqpoint{2.910335in}{0.498777in}}%
\pgfpathlineto{\pgfqpoint{2.977081in}{0.498777in}}%
\pgfpathlineto{\pgfqpoint{2.977081in}{0.620630in}}%
\pgfpathlineto{\pgfqpoint{2.910335in}{0.620630in}}%
\pgfpathlineto{\pgfqpoint{2.910335in}{0.498777in}}%
\pgfpathclose%
\pgfusepath{stroke}%
\end{pgfscope}%
\begin{pgfscope}%
\pgfpathrectangle{\pgfqpoint{0.432374in}{0.498777in}}{\pgfqpoint{3.487500in}{1.155000in}}%
\pgfusepath{clip}%
\pgfsetbuttcap%
\pgfsetmiterjoin%
\pgfsetlinewidth{1.003750pt}%
\definecolor{currentstroke}{rgb}{0.000000,0.000000,0.000000}%
\pgfsetstrokecolor{currentstroke}%
\pgfsetdash{}{0pt}%
\pgfpathmoveto{\pgfqpoint{3.077201in}{0.498777in}}%
\pgfpathlineto{\pgfqpoint{3.143947in}{0.498777in}}%
\pgfpathlineto{\pgfqpoint{3.143947in}{0.597502in}}%
\pgfpathlineto{\pgfqpoint{3.077201in}{0.597502in}}%
\pgfpathlineto{\pgfqpoint{3.077201in}{0.498777in}}%
\pgfpathclose%
\pgfusepath{stroke}%
\end{pgfscope}%
\begin{pgfscope}%
\pgfpathrectangle{\pgfqpoint{0.432374in}{0.498777in}}{\pgfqpoint{3.487500in}{1.155000in}}%
\pgfusepath{clip}%
\pgfsetbuttcap%
\pgfsetmiterjoin%
\pgfsetlinewidth{1.003750pt}%
\definecolor{currentstroke}{rgb}{0.000000,0.000000,0.000000}%
\pgfsetstrokecolor{currentstroke}%
\pgfsetdash{}{0pt}%
\pgfpathmoveto{\pgfqpoint{3.244067in}{0.498777in}}%
\pgfpathlineto{\pgfqpoint{3.310813in}{0.498777in}}%
\pgfpathlineto{\pgfqpoint{3.310813in}{0.581037in}}%
\pgfpathlineto{\pgfqpoint{3.244067in}{0.581037in}}%
\pgfpathlineto{\pgfqpoint{3.244067in}{0.498777in}}%
\pgfpathclose%
\pgfusepath{stroke}%
\end{pgfscope}%
\begin{pgfscope}%
\pgfpathrectangle{\pgfqpoint{0.432374in}{0.498777in}}{\pgfqpoint{3.487500in}{1.155000in}}%
\pgfusepath{clip}%
\pgfsetbuttcap%
\pgfsetmiterjoin%
\pgfsetlinewidth{1.003750pt}%
\definecolor{currentstroke}{rgb}{0.000000,0.000000,0.000000}%
\pgfsetstrokecolor{currentstroke}%
\pgfsetdash{}{0pt}%
\pgfpathmoveto{\pgfqpoint{3.410933in}{0.498777in}}%
\pgfpathlineto{\pgfqpoint{3.477679in}{0.498777in}}%
\pgfpathlineto{\pgfqpoint{3.477679in}{0.568673in}}%
\pgfpathlineto{\pgfqpoint{3.410933in}{0.568673in}}%
\pgfpathlineto{\pgfqpoint{3.410933in}{0.498777in}}%
\pgfpathclose%
\pgfusepath{stroke}%
\end{pgfscope}%
\begin{pgfscope}%
\pgfpathrectangle{\pgfqpoint{0.432374in}{0.498777in}}{\pgfqpoint{3.487500in}{1.155000in}}%
\pgfusepath{clip}%
\pgfsetbuttcap%
\pgfsetmiterjoin%
\pgfsetlinewidth{1.003750pt}%
\definecolor{currentstroke}{rgb}{0.000000,0.000000,0.000000}%
\pgfsetstrokecolor{currentstroke}%
\pgfsetdash{}{0pt}%
\pgfpathmoveto{\pgfqpoint{3.577799in}{0.498777in}}%
\pgfpathlineto{\pgfqpoint{3.644545in}{0.498777in}}%
\pgfpathlineto{\pgfqpoint{3.644545in}{0.498777in}}%
\pgfpathlineto{\pgfqpoint{3.577799in}{0.498777in}}%
\pgfpathlineto{\pgfqpoint{3.577799in}{0.498777in}}%
\pgfpathclose%
\pgfusepath{stroke}%
\end{pgfscope}%
\begin{pgfscope}%
\pgfpathrectangle{\pgfqpoint{0.432374in}{0.498777in}}{\pgfqpoint{3.487500in}{1.155000in}}%
\pgfusepath{clip}%
\pgfsetbuttcap%
\pgfsetmiterjoin%
\pgfsetlinewidth{1.003750pt}%
\definecolor{currentstroke}{rgb}{0.000000,0.000000,0.000000}%
\pgfsetstrokecolor{currentstroke}%
\pgfsetdash{}{0pt}%
\pgfpathmoveto{\pgfqpoint{3.744665in}{0.498777in}}%
\pgfpathlineto{\pgfqpoint{3.811411in}{0.498777in}}%
\pgfpathlineto{\pgfqpoint{3.811411in}{0.498777in}}%
\pgfpathlineto{\pgfqpoint{3.744665in}{0.498777in}}%
\pgfpathlineto{\pgfqpoint{3.744665in}{0.498777in}}%
\pgfpathclose%
\pgfusepath{stroke}%
\end{pgfscope}%
\begin{pgfscope}%
\pgfpathrectangle{\pgfqpoint{0.432374in}{0.498777in}}{\pgfqpoint{3.487500in}{1.155000in}}%
\pgfusepath{clip}%
\pgfsetbuttcap%
\pgfsetmiterjoin%
\definecolor{currentfill}{rgb}{0.000000,0.000000,0.000000}%
\pgfsetfillcolor{currentfill}%
\pgfsetlinewidth{0.000000pt}%
\definecolor{currentstroke}{rgb}{0.000000,0.000000,0.000000}%
\pgfsetstrokecolor{currentstroke}%
\pgfsetstrokeopacity{0.000000}%
\pgfsetdash{}{0pt}%
\pgfpathmoveto{\pgfqpoint{0.474091in}{0.498777in}}%
\pgfpathlineto{\pgfqpoint{0.540837in}{0.498777in}}%
\pgfpathlineto{\pgfqpoint{0.540837in}{0.498777in}}%
\pgfpathlineto{\pgfqpoint{0.474091in}{0.498777in}}%
\pgfpathlineto{\pgfqpoint{0.474091in}{0.498777in}}%
\pgfpathclose%
\pgfusepath{fill}%
\end{pgfscope}%
\begin{pgfscope}%
\pgfpathrectangle{\pgfqpoint{0.432374in}{0.498777in}}{\pgfqpoint{3.487500in}{1.155000in}}%
\pgfusepath{clip}%
\pgfsetbuttcap%
\pgfsetmiterjoin%
\definecolor{currentfill}{rgb}{0.000000,0.000000,0.000000}%
\pgfsetfillcolor{currentfill}%
\pgfsetlinewidth{0.000000pt}%
\definecolor{currentstroke}{rgb}{0.000000,0.000000,0.000000}%
\pgfsetstrokecolor{currentstroke}%
\pgfsetstrokeopacity{0.000000}%
\pgfsetdash{}{0pt}%
\pgfpathmoveto{\pgfqpoint{0.640957in}{0.498777in}}%
\pgfpathlineto{\pgfqpoint{0.707703in}{0.498777in}}%
\pgfpathlineto{\pgfqpoint{0.707703in}{0.724287in}}%
\pgfpathlineto{\pgfqpoint{0.640957in}{0.724287in}}%
\pgfpathlineto{\pgfqpoint{0.640957in}{0.498777in}}%
\pgfpathclose%
\pgfusepath{fill}%
\end{pgfscope}%
\begin{pgfscope}%
\pgfpathrectangle{\pgfqpoint{0.432374in}{0.498777in}}{\pgfqpoint{3.487500in}{1.155000in}}%
\pgfusepath{clip}%
\pgfsetbuttcap%
\pgfsetmiterjoin%
\definecolor{currentfill}{rgb}{0.000000,0.000000,0.000000}%
\pgfsetfillcolor{currentfill}%
\pgfsetlinewidth{0.000000pt}%
\definecolor{currentstroke}{rgb}{0.000000,0.000000,0.000000}%
\pgfsetstrokecolor{currentstroke}%
\pgfsetstrokeopacity{0.000000}%
\pgfsetdash{}{0pt}%
\pgfpathmoveto{\pgfqpoint{0.807823in}{0.498777in}}%
\pgfpathlineto{\pgfqpoint{0.874569in}{0.498777in}}%
\pgfpathlineto{\pgfqpoint{0.874569in}{0.755231in}}%
\pgfpathlineto{\pgfqpoint{0.807823in}{0.755231in}}%
\pgfpathlineto{\pgfqpoint{0.807823in}{0.498777in}}%
\pgfpathclose%
\pgfusepath{fill}%
\end{pgfscope}%
\begin{pgfscope}%
\pgfpathrectangle{\pgfqpoint{0.432374in}{0.498777in}}{\pgfqpoint{3.487500in}{1.155000in}}%
\pgfusepath{clip}%
\pgfsetbuttcap%
\pgfsetmiterjoin%
\definecolor{currentfill}{rgb}{0.000000,0.000000,0.000000}%
\pgfsetfillcolor{currentfill}%
\pgfsetlinewidth{0.000000pt}%
\definecolor{currentstroke}{rgb}{0.000000,0.000000,0.000000}%
\pgfsetstrokecolor{currentstroke}%
\pgfsetstrokeopacity{0.000000}%
\pgfsetdash{}{0pt}%
\pgfpathmoveto{\pgfqpoint{0.974689in}{0.498777in}}%
\pgfpathlineto{\pgfqpoint{1.041435in}{0.498777in}}%
\pgfpathlineto{\pgfqpoint{1.041435in}{0.778422in}}%
\pgfpathlineto{\pgfqpoint{0.974689in}{0.778422in}}%
\pgfpathlineto{\pgfqpoint{0.974689in}{0.498777in}}%
\pgfpathclose%
\pgfusepath{fill}%
\end{pgfscope}%
\begin{pgfscope}%
\pgfpathrectangle{\pgfqpoint{0.432374in}{0.498777in}}{\pgfqpoint{3.487500in}{1.155000in}}%
\pgfusepath{clip}%
\pgfsetbuttcap%
\pgfsetmiterjoin%
\definecolor{currentfill}{rgb}{0.000000,0.000000,0.000000}%
\pgfsetfillcolor{currentfill}%
\pgfsetlinewidth{0.000000pt}%
\definecolor{currentstroke}{rgb}{0.000000,0.000000,0.000000}%
\pgfsetstrokecolor{currentstroke}%
\pgfsetstrokeopacity{0.000000}%
\pgfsetdash{}{0pt}%
\pgfpathmoveto{\pgfqpoint{1.141555in}{0.498777in}}%
\pgfpathlineto{\pgfqpoint{1.208301in}{0.498777in}}%
\pgfpathlineto{\pgfqpoint{1.208301in}{0.805009in}}%
\pgfpathlineto{\pgfqpoint{1.141555in}{0.805009in}}%
\pgfpathlineto{\pgfqpoint{1.141555in}{0.498777in}}%
\pgfpathclose%
\pgfusepath{fill}%
\end{pgfscope}%
\begin{pgfscope}%
\pgfpathrectangle{\pgfqpoint{0.432374in}{0.498777in}}{\pgfqpoint{3.487500in}{1.155000in}}%
\pgfusepath{clip}%
\pgfsetbuttcap%
\pgfsetmiterjoin%
\definecolor{currentfill}{rgb}{0.000000,0.000000,0.000000}%
\pgfsetfillcolor{currentfill}%
\pgfsetlinewidth{0.000000pt}%
\definecolor{currentstroke}{rgb}{0.000000,0.000000,0.000000}%
\pgfsetstrokecolor{currentstroke}%
\pgfsetstrokeopacity{0.000000}%
\pgfsetdash{}{0pt}%
\pgfpathmoveto{\pgfqpoint{1.308421in}{0.498777in}}%
\pgfpathlineto{\pgfqpoint{1.375167in}{0.498777in}}%
\pgfpathlineto{\pgfqpoint{1.375167in}{0.837426in}}%
\pgfpathlineto{\pgfqpoint{1.308421in}{0.837426in}}%
\pgfpathlineto{\pgfqpoint{1.308421in}{0.498777in}}%
\pgfpathclose%
\pgfusepath{fill}%
\end{pgfscope}%
\begin{pgfscope}%
\pgfpathrectangle{\pgfqpoint{0.432374in}{0.498777in}}{\pgfqpoint{3.487500in}{1.155000in}}%
\pgfusepath{clip}%
\pgfsetbuttcap%
\pgfsetmiterjoin%
\definecolor{currentfill}{rgb}{0.000000,0.000000,0.000000}%
\pgfsetfillcolor{currentfill}%
\pgfsetlinewidth{0.000000pt}%
\definecolor{currentstroke}{rgb}{0.000000,0.000000,0.000000}%
\pgfsetstrokecolor{currentstroke}%
\pgfsetstrokeopacity{0.000000}%
\pgfsetdash{}{0pt}%
\pgfpathmoveto{\pgfqpoint{1.475287in}{0.498777in}}%
\pgfpathlineto{\pgfqpoint{1.542033in}{0.498777in}}%
\pgfpathlineto{\pgfqpoint{1.542033in}{0.855493in}}%
\pgfpathlineto{\pgfqpoint{1.475287in}{0.855493in}}%
\pgfpathlineto{\pgfqpoint{1.475287in}{0.498777in}}%
\pgfpathclose%
\pgfusepath{fill}%
\end{pgfscope}%
\begin{pgfscope}%
\pgfpathrectangle{\pgfqpoint{0.432374in}{0.498777in}}{\pgfqpoint{3.487500in}{1.155000in}}%
\pgfusepath{clip}%
\pgfsetbuttcap%
\pgfsetmiterjoin%
\definecolor{currentfill}{rgb}{0.000000,0.000000,0.000000}%
\pgfsetfillcolor{currentfill}%
\pgfsetlinewidth{0.000000pt}%
\definecolor{currentstroke}{rgb}{0.000000,0.000000,0.000000}%
\pgfsetstrokecolor{currentstroke}%
\pgfsetstrokeopacity{0.000000}%
\pgfsetdash{}{0pt}%
\pgfpathmoveto{\pgfqpoint{1.642153in}{0.498777in}}%
\pgfpathlineto{\pgfqpoint{1.708899in}{0.498777in}}%
\pgfpathlineto{\pgfqpoint{1.708899in}{0.877531in}}%
\pgfpathlineto{\pgfqpoint{1.642153in}{0.877531in}}%
\pgfpathlineto{\pgfqpoint{1.642153in}{0.498777in}}%
\pgfpathclose%
\pgfusepath{fill}%
\end{pgfscope}%
\begin{pgfscope}%
\pgfpathrectangle{\pgfqpoint{0.432374in}{0.498777in}}{\pgfqpoint{3.487500in}{1.155000in}}%
\pgfusepath{clip}%
\pgfsetbuttcap%
\pgfsetmiterjoin%
\definecolor{currentfill}{rgb}{0.000000,0.000000,0.000000}%
\pgfsetfillcolor{currentfill}%
\pgfsetlinewidth{0.000000pt}%
\definecolor{currentstroke}{rgb}{0.000000,0.000000,0.000000}%
\pgfsetstrokecolor{currentstroke}%
\pgfsetstrokeopacity{0.000000}%
\pgfsetdash{}{0pt}%
\pgfpathmoveto{\pgfqpoint{1.809019in}{0.498777in}}%
\pgfpathlineto{\pgfqpoint{1.875765in}{0.498777in}}%
\pgfpathlineto{\pgfqpoint{1.875765in}{0.887718in}}%
\pgfpathlineto{\pgfqpoint{1.809019in}{0.887718in}}%
\pgfpathlineto{\pgfqpoint{1.809019in}{0.498777in}}%
\pgfpathclose%
\pgfusepath{fill}%
\end{pgfscope}%
\begin{pgfscope}%
\pgfpathrectangle{\pgfqpoint{0.432374in}{0.498777in}}{\pgfqpoint{3.487500in}{1.155000in}}%
\pgfusepath{clip}%
\pgfsetbuttcap%
\pgfsetmiterjoin%
\definecolor{currentfill}{rgb}{0.000000,0.000000,0.000000}%
\pgfsetfillcolor{currentfill}%
\pgfsetlinewidth{0.000000pt}%
\definecolor{currentstroke}{rgb}{0.000000,0.000000,0.000000}%
\pgfsetstrokecolor{currentstroke}%
\pgfsetstrokeopacity{0.000000}%
\pgfsetdash{}{0pt}%
\pgfpathmoveto{\pgfqpoint{1.975885in}{0.498777in}}%
\pgfpathlineto{\pgfqpoint{2.042631in}{0.498777in}}%
\pgfpathlineto{\pgfqpoint{2.042631in}{0.949348in}}%
\pgfpathlineto{\pgfqpoint{1.975885in}{0.949348in}}%
\pgfpathlineto{\pgfqpoint{1.975885in}{0.498777in}}%
\pgfpathclose%
\pgfusepath{fill}%
\end{pgfscope}%
\begin{pgfscope}%
\pgfpathrectangle{\pgfqpoint{0.432374in}{0.498777in}}{\pgfqpoint{3.487500in}{1.155000in}}%
\pgfusepath{clip}%
\pgfsetbuttcap%
\pgfsetmiterjoin%
\definecolor{currentfill}{rgb}{0.000000,0.000000,0.000000}%
\pgfsetfillcolor{currentfill}%
\pgfsetlinewidth{0.000000pt}%
\definecolor{currentstroke}{rgb}{0.000000,0.000000,0.000000}%
\pgfsetstrokecolor{currentstroke}%
\pgfsetstrokeopacity{0.000000}%
\pgfsetdash{}{0pt}%
\pgfpathmoveto{\pgfqpoint{2.142751in}{0.498777in}}%
\pgfpathlineto{\pgfqpoint{2.209497in}{0.498777in}}%
\pgfpathlineto{\pgfqpoint{2.209497in}{0.901556in}}%
\pgfpathlineto{\pgfqpoint{2.142751in}{0.901556in}}%
\pgfpathlineto{\pgfqpoint{2.142751in}{0.498777in}}%
\pgfpathclose%
\pgfusepath{fill}%
\end{pgfscope}%
\begin{pgfscope}%
\pgfpathrectangle{\pgfqpoint{0.432374in}{0.498777in}}{\pgfqpoint{3.487500in}{1.155000in}}%
\pgfusepath{clip}%
\pgfsetbuttcap%
\pgfsetmiterjoin%
\definecolor{currentfill}{rgb}{0.000000,0.000000,0.000000}%
\pgfsetfillcolor{currentfill}%
\pgfsetlinewidth{0.000000pt}%
\definecolor{currentstroke}{rgb}{0.000000,0.000000,0.000000}%
\pgfsetstrokecolor{currentstroke}%
\pgfsetstrokeopacity{0.000000}%
\pgfsetdash{}{0pt}%
\pgfpathmoveto{\pgfqpoint{2.309617in}{0.498777in}}%
\pgfpathlineto{\pgfqpoint{2.376363in}{0.498777in}}%
\pgfpathlineto{\pgfqpoint{2.376363in}{0.896687in}}%
\pgfpathlineto{\pgfqpoint{2.309617in}{0.896687in}}%
\pgfpathlineto{\pgfqpoint{2.309617in}{0.498777in}}%
\pgfpathclose%
\pgfusepath{fill}%
\end{pgfscope}%
\begin{pgfscope}%
\pgfpathrectangle{\pgfqpoint{0.432374in}{0.498777in}}{\pgfqpoint{3.487500in}{1.155000in}}%
\pgfusepath{clip}%
\pgfsetbuttcap%
\pgfsetmiterjoin%
\definecolor{currentfill}{rgb}{0.000000,0.000000,0.000000}%
\pgfsetfillcolor{currentfill}%
\pgfsetlinewidth{0.000000pt}%
\definecolor{currentstroke}{rgb}{0.000000,0.000000,0.000000}%
\pgfsetstrokecolor{currentstroke}%
\pgfsetstrokeopacity{0.000000}%
\pgfsetdash{}{0pt}%
\pgfpathmoveto{\pgfqpoint{2.476483in}{0.498777in}}%
\pgfpathlineto{\pgfqpoint{2.543229in}{0.498777in}}%
\pgfpathlineto{\pgfqpoint{2.543229in}{0.869715in}}%
\pgfpathlineto{\pgfqpoint{2.476483in}{0.869715in}}%
\pgfpathlineto{\pgfqpoint{2.476483in}{0.498777in}}%
\pgfpathclose%
\pgfusepath{fill}%
\end{pgfscope}%
\begin{pgfscope}%
\pgfpathrectangle{\pgfqpoint{0.432374in}{0.498777in}}{\pgfqpoint{3.487500in}{1.155000in}}%
\pgfusepath{clip}%
\pgfsetbuttcap%
\pgfsetmiterjoin%
\definecolor{currentfill}{rgb}{0.000000,0.000000,0.000000}%
\pgfsetfillcolor{currentfill}%
\pgfsetlinewidth{0.000000pt}%
\definecolor{currentstroke}{rgb}{0.000000,0.000000,0.000000}%
\pgfsetstrokecolor{currentstroke}%
\pgfsetstrokeopacity{0.000000}%
\pgfsetdash{}{0pt}%
\pgfpathmoveto{\pgfqpoint{2.643349in}{0.498777in}}%
\pgfpathlineto{\pgfqpoint{2.710095in}{0.498777in}}%
\pgfpathlineto{\pgfqpoint{2.710095in}{0.849150in}}%
\pgfpathlineto{\pgfqpoint{2.643349in}{0.849150in}}%
\pgfpathlineto{\pgfqpoint{2.643349in}{0.498777in}}%
\pgfpathclose%
\pgfusepath{fill}%
\end{pgfscope}%
\begin{pgfscope}%
\pgfpathrectangle{\pgfqpoint{0.432374in}{0.498777in}}{\pgfqpoint{3.487500in}{1.155000in}}%
\pgfusepath{clip}%
\pgfsetbuttcap%
\pgfsetmiterjoin%
\definecolor{currentfill}{rgb}{0.000000,0.000000,0.000000}%
\pgfsetfillcolor{currentfill}%
\pgfsetlinewidth{0.000000pt}%
\definecolor{currentstroke}{rgb}{0.000000,0.000000,0.000000}%
\pgfsetstrokecolor{currentstroke}%
\pgfsetstrokeopacity{0.000000}%
\pgfsetdash{}{0pt}%
\pgfpathmoveto{\pgfqpoint{2.810215in}{0.498777in}}%
\pgfpathlineto{\pgfqpoint{2.876961in}{0.498777in}}%
\pgfpathlineto{\pgfqpoint{2.876961in}{0.799500in}}%
\pgfpathlineto{\pgfqpoint{2.810215in}{0.799500in}}%
\pgfpathlineto{\pgfqpoint{2.810215in}{0.498777in}}%
\pgfpathclose%
\pgfusepath{fill}%
\end{pgfscope}%
\begin{pgfscope}%
\pgfpathrectangle{\pgfqpoint{0.432374in}{0.498777in}}{\pgfqpoint{3.487500in}{1.155000in}}%
\pgfusepath{clip}%
\pgfsetbuttcap%
\pgfsetmiterjoin%
\definecolor{currentfill}{rgb}{0.000000,0.000000,0.000000}%
\pgfsetfillcolor{currentfill}%
\pgfsetlinewidth{0.000000pt}%
\definecolor{currentstroke}{rgb}{0.000000,0.000000,0.000000}%
\pgfsetstrokecolor{currentstroke}%
\pgfsetstrokeopacity{0.000000}%
\pgfsetdash{}{0pt}%
\pgfpathmoveto{\pgfqpoint{2.977081in}{0.498777in}}%
\pgfpathlineto{\pgfqpoint{3.043827in}{0.498777in}}%
\pgfpathlineto{\pgfqpoint{3.043827in}{0.749849in}}%
\pgfpathlineto{\pgfqpoint{2.977081in}{0.749849in}}%
\pgfpathlineto{\pgfqpoint{2.977081in}{0.498777in}}%
\pgfpathclose%
\pgfusepath{fill}%
\end{pgfscope}%
\begin{pgfscope}%
\pgfpathrectangle{\pgfqpoint{0.432374in}{0.498777in}}{\pgfqpoint{3.487500in}{1.155000in}}%
\pgfusepath{clip}%
\pgfsetbuttcap%
\pgfsetmiterjoin%
\definecolor{currentfill}{rgb}{0.000000,0.000000,0.000000}%
\pgfsetfillcolor{currentfill}%
\pgfsetlinewidth{0.000000pt}%
\definecolor{currentstroke}{rgb}{0.000000,0.000000,0.000000}%
\pgfsetstrokecolor{currentstroke}%
\pgfsetstrokeopacity{0.000000}%
\pgfsetdash{}{0pt}%
\pgfpathmoveto{\pgfqpoint{3.143947in}{0.498777in}}%
\pgfpathlineto{\pgfqpoint{3.210693in}{0.498777in}}%
\pgfpathlineto{\pgfqpoint{3.210693in}{0.686360in}}%
\pgfpathlineto{\pgfqpoint{3.143947in}{0.686360in}}%
\pgfpathlineto{\pgfqpoint{3.143947in}{0.498777in}}%
\pgfpathclose%
\pgfusepath{fill}%
\end{pgfscope}%
\begin{pgfscope}%
\pgfpathrectangle{\pgfqpoint{0.432374in}{0.498777in}}{\pgfqpoint{3.487500in}{1.155000in}}%
\pgfusepath{clip}%
\pgfsetbuttcap%
\pgfsetmiterjoin%
\definecolor{currentfill}{rgb}{0.000000,0.000000,0.000000}%
\pgfsetfillcolor{currentfill}%
\pgfsetlinewidth{0.000000pt}%
\definecolor{currentstroke}{rgb}{0.000000,0.000000,0.000000}%
\pgfsetstrokecolor{currentstroke}%
\pgfsetstrokeopacity{0.000000}%
\pgfsetdash{}{0pt}%
\pgfpathmoveto{\pgfqpoint{3.310813in}{0.498777in}}%
\pgfpathlineto{\pgfqpoint{3.377559in}{0.498777in}}%
\pgfpathlineto{\pgfqpoint{3.377559in}{0.622167in}}%
\pgfpathlineto{\pgfqpoint{3.310813in}{0.622167in}}%
\pgfpathlineto{\pgfqpoint{3.310813in}{0.498777in}}%
\pgfpathclose%
\pgfusepath{fill}%
\end{pgfscope}%
\begin{pgfscope}%
\pgfpathrectangle{\pgfqpoint{0.432374in}{0.498777in}}{\pgfqpoint{3.487500in}{1.155000in}}%
\pgfusepath{clip}%
\pgfsetbuttcap%
\pgfsetmiterjoin%
\definecolor{currentfill}{rgb}{0.000000,0.000000,0.000000}%
\pgfsetfillcolor{currentfill}%
\pgfsetlinewidth{0.000000pt}%
\definecolor{currentstroke}{rgb}{0.000000,0.000000,0.000000}%
\pgfsetstrokecolor{currentstroke}%
\pgfsetstrokeopacity{0.000000}%
\pgfsetdash{}{0pt}%
\pgfpathmoveto{\pgfqpoint{3.477679in}{0.498777in}}%
\pgfpathlineto{\pgfqpoint{3.544425in}{0.498777in}}%
\pgfpathlineto{\pgfqpoint{3.544425in}{0.561754in}}%
\pgfpathlineto{\pgfqpoint{3.477679in}{0.561754in}}%
\pgfpathlineto{\pgfqpoint{3.477679in}{0.498777in}}%
\pgfpathclose%
\pgfusepath{fill}%
\end{pgfscope}%
\begin{pgfscope}%
\pgfpathrectangle{\pgfqpoint{0.432374in}{0.498777in}}{\pgfqpoint{3.487500in}{1.155000in}}%
\pgfusepath{clip}%
\pgfsetbuttcap%
\pgfsetmiterjoin%
\definecolor{currentfill}{rgb}{0.000000,0.000000,0.000000}%
\pgfsetfillcolor{currentfill}%
\pgfsetlinewidth{0.000000pt}%
\definecolor{currentstroke}{rgb}{0.000000,0.000000,0.000000}%
\pgfsetstrokecolor{currentstroke}%
\pgfsetstrokeopacity{0.000000}%
\pgfsetdash{}{0pt}%
\pgfpathmoveto{\pgfqpoint{3.644545in}{0.498777in}}%
\pgfpathlineto{\pgfqpoint{3.711291in}{0.498777in}}%
\pgfpathlineto{\pgfqpoint{3.711291in}{0.521008in}}%
\pgfpathlineto{\pgfqpoint{3.644545in}{0.521008in}}%
\pgfpathlineto{\pgfqpoint{3.644545in}{0.498777in}}%
\pgfpathclose%
\pgfusepath{fill}%
\end{pgfscope}%
\begin{pgfscope}%
\pgfpathrectangle{\pgfqpoint{0.432374in}{0.498777in}}{\pgfqpoint{3.487500in}{1.155000in}}%
\pgfusepath{clip}%
\pgfsetbuttcap%
\pgfsetmiterjoin%
\definecolor{currentfill}{rgb}{0.000000,0.000000,0.000000}%
\pgfsetfillcolor{currentfill}%
\pgfsetlinewidth{0.000000pt}%
\definecolor{currentstroke}{rgb}{0.000000,0.000000,0.000000}%
\pgfsetstrokecolor{currentstroke}%
\pgfsetstrokeopacity{0.000000}%
\pgfsetdash{}{0pt}%
\pgfpathmoveto{\pgfqpoint{3.811411in}{0.498777in}}%
\pgfpathlineto{\pgfqpoint{3.878158in}{0.498777in}}%
\pgfpathlineto{\pgfqpoint{3.878158in}{0.501148in}}%
\pgfpathlineto{\pgfqpoint{3.811411in}{0.501148in}}%
\pgfpathlineto{\pgfqpoint{3.811411in}{0.498777in}}%
\pgfpathclose%
\pgfusepath{fill}%
\end{pgfscope}%
\begin{pgfscope}%
\pgfsetbuttcap%
\pgfsetroundjoin%
\definecolor{currentfill}{rgb}{0.000000,0.000000,0.000000}%
\pgfsetfillcolor{currentfill}%
\pgfsetlinewidth{0.803000pt}%
\definecolor{currentstroke}{rgb}{0.000000,0.000000,0.000000}%
\pgfsetstrokecolor{currentstroke}%
\pgfsetdash{}{0pt}%
\pgfsys@defobject{currentmarker}{\pgfqpoint{0.000000in}{-0.048611in}}{\pgfqpoint{0.000000in}{0.000000in}}{%
\pgfpathmoveto{\pgfqpoint{0.000000in}{0.000000in}}%
\pgfpathlineto{\pgfqpoint{0.000000in}{-0.048611in}}%
\pgfusepath{stroke,fill}%
}%
\begin{pgfscope}%
\pgfsys@transformshift{0.474091in}{0.498777in}%
\pgfsys@useobject{currentmarker}{}%
\end{pgfscope}%
\end{pgfscope}%
\begin{pgfscope}%
\definecolor{textcolor}{rgb}{0.000000,0.000000,0.000000}%
\pgfsetstrokecolor{textcolor}%
\pgfsetfillcolor{textcolor}%
\pgftext[x=0.474091in,y=0.401555in,,top]{\color{textcolor}\rmfamily\fontsize{12.000000}{14.400000}\selectfont 0.50}%
\end{pgfscope}%
\begin{pgfscope}%
\pgfsetbuttcap%
\pgfsetroundjoin%
\definecolor{currentfill}{rgb}{0.000000,0.000000,0.000000}%
\pgfsetfillcolor{currentfill}%
\pgfsetlinewidth{0.803000pt}%
\definecolor{currentstroke}{rgb}{0.000000,0.000000,0.000000}%
\pgfsetstrokecolor{currentstroke}%
\pgfsetdash{}{0pt}%
\pgfsys@defobject{currentmarker}{\pgfqpoint{0.000000in}{-0.048611in}}{\pgfqpoint{0.000000in}{0.000000in}}{%
\pgfpathmoveto{\pgfqpoint{0.000000in}{0.000000in}}%
\pgfpathlineto{\pgfqpoint{0.000000in}{-0.048611in}}%
\pgfusepath{stroke,fill}%
}%
\begin{pgfscope}%
\pgfsys@transformshift{1.141555in}{0.498777in}%
\pgfsys@useobject{currentmarker}{}%
\end{pgfscope}%
\end{pgfscope}%
\begin{pgfscope}%
\definecolor{textcolor}{rgb}{0.000000,0.000000,0.000000}%
\pgfsetstrokecolor{textcolor}%
\pgfsetfillcolor{textcolor}%
\pgftext[x=1.141555in,y=0.401555in,,top]{\color{textcolor}\rmfamily\fontsize{12.000000}{14.400000}\selectfont 0.60}%
\end{pgfscope}%
\begin{pgfscope}%
\pgfsetbuttcap%
\pgfsetroundjoin%
\definecolor{currentfill}{rgb}{0.000000,0.000000,0.000000}%
\pgfsetfillcolor{currentfill}%
\pgfsetlinewidth{0.803000pt}%
\definecolor{currentstroke}{rgb}{0.000000,0.000000,0.000000}%
\pgfsetstrokecolor{currentstroke}%
\pgfsetdash{}{0pt}%
\pgfsys@defobject{currentmarker}{\pgfqpoint{0.000000in}{-0.048611in}}{\pgfqpoint{0.000000in}{0.000000in}}{%
\pgfpathmoveto{\pgfqpoint{0.000000in}{0.000000in}}%
\pgfpathlineto{\pgfqpoint{0.000000in}{-0.048611in}}%
\pgfusepath{stroke,fill}%
}%
\begin{pgfscope}%
\pgfsys@transformshift{1.809019in}{0.498777in}%
\pgfsys@useobject{currentmarker}{}%
\end{pgfscope}%
\end{pgfscope}%
\begin{pgfscope}%
\definecolor{textcolor}{rgb}{0.000000,0.000000,0.000000}%
\pgfsetstrokecolor{textcolor}%
\pgfsetfillcolor{textcolor}%
\pgftext[x=1.809019in,y=0.401555in,,top]{\color{textcolor}\rmfamily\fontsize{12.000000}{14.400000}\selectfont 0.70}%
\end{pgfscope}%
\begin{pgfscope}%
\pgfsetbuttcap%
\pgfsetroundjoin%
\definecolor{currentfill}{rgb}{0.000000,0.000000,0.000000}%
\pgfsetfillcolor{currentfill}%
\pgfsetlinewidth{0.803000pt}%
\definecolor{currentstroke}{rgb}{0.000000,0.000000,0.000000}%
\pgfsetstrokecolor{currentstroke}%
\pgfsetdash{}{0pt}%
\pgfsys@defobject{currentmarker}{\pgfqpoint{0.000000in}{-0.048611in}}{\pgfqpoint{0.000000in}{0.000000in}}{%
\pgfpathmoveto{\pgfqpoint{0.000000in}{0.000000in}}%
\pgfpathlineto{\pgfqpoint{0.000000in}{-0.048611in}}%
\pgfusepath{stroke,fill}%
}%
\begin{pgfscope}%
\pgfsys@transformshift{2.476483in}{0.498777in}%
\pgfsys@useobject{currentmarker}{}%
\end{pgfscope}%
\end{pgfscope}%
\begin{pgfscope}%
\definecolor{textcolor}{rgb}{0.000000,0.000000,0.000000}%
\pgfsetstrokecolor{textcolor}%
\pgfsetfillcolor{textcolor}%
\pgftext[x=2.476483in,y=0.401555in,,top]{\color{textcolor}\rmfamily\fontsize{12.000000}{14.400000}\selectfont 0.80}%
\end{pgfscope}%
\begin{pgfscope}%
\pgfsetbuttcap%
\pgfsetroundjoin%
\definecolor{currentfill}{rgb}{0.000000,0.000000,0.000000}%
\pgfsetfillcolor{currentfill}%
\pgfsetlinewidth{0.803000pt}%
\definecolor{currentstroke}{rgb}{0.000000,0.000000,0.000000}%
\pgfsetstrokecolor{currentstroke}%
\pgfsetdash{}{0pt}%
\pgfsys@defobject{currentmarker}{\pgfqpoint{0.000000in}{-0.048611in}}{\pgfqpoint{0.000000in}{0.000000in}}{%
\pgfpathmoveto{\pgfqpoint{0.000000in}{0.000000in}}%
\pgfpathlineto{\pgfqpoint{0.000000in}{-0.048611in}}%
\pgfusepath{stroke,fill}%
}%
\begin{pgfscope}%
\pgfsys@transformshift{3.143947in}{0.498777in}%
\pgfsys@useobject{currentmarker}{}%
\end{pgfscope}%
\end{pgfscope}%
\begin{pgfscope}%
\definecolor{textcolor}{rgb}{0.000000,0.000000,0.000000}%
\pgfsetstrokecolor{textcolor}%
\pgfsetfillcolor{textcolor}%
\pgftext[x=3.143947in,y=0.401555in,,top]{\color{textcolor}\rmfamily\fontsize{12.000000}{14.400000}\selectfont 0.90}%
\end{pgfscope}%
\begin{pgfscope}%
\pgfsetbuttcap%
\pgfsetroundjoin%
\definecolor{currentfill}{rgb}{0.000000,0.000000,0.000000}%
\pgfsetfillcolor{currentfill}%
\pgfsetlinewidth{0.803000pt}%
\definecolor{currentstroke}{rgb}{0.000000,0.000000,0.000000}%
\pgfsetstrokecolor{currentstroke}%
\pgfsetdash{}{0pt}%
\pgfsys@defobject{currentmarker}{\pgfqpoint{0.000000in}{-0.048611in}}{\pgfqpoint{0.000000in}{0.000000in}}{%
\pgfpathmoveto{\pgfqpoint{0.000000in}{0.000000in}}%
\pgfpathlineto{\pgfqpoint{0.000000in}{-0.048611in}}%
\pgfusepath{stroke,fill}%
}%
\begin{pgfscope}%
\pgfsys@transformshift{3.811411in}{0.498777in}%
\pgfsys@useobject{currentmarker}{}%
\end{pgfscope}%
\end{pgfscope}%
\begin{pgfscope}%
\definecolor{textcolor}{rgb}{0.000000,0.000000,0.000000}%
\pgfsetstrokecolor{textcolor}%
\pgfsetfillcolor{textcolor}%
\pgftext[x=3.811411in,y=0.401555in,,top]{\color{textcolor}\rmfamily\fontsize{12.000000}{14.400000}\selectfont 1.00}%
\end{pgfscope}%
\begin{pgfscope}%
\definecolor{textcolor}{rgb}{0.000000,0.000000,0.000000}%
\pgfsetstrokecolor{textcolor}%
\pgfsetfillcolor{textcolor}%
\pgftext[x=2.176124in,y=0.198000in,,top]{\color{textcolor}\rmfamily\fontsize{12.000000}{14.400000}\selectfont \(\displaystyle p\)}%
\end{pgfscope}%
\begin{pgfscope}%
\pgfsetbuttcap%
\pgfsetroundjoin%
\definecolor{currentfill}{rgb}{0.000000,0.000000,0.000000}%
\pgfsetfillcolor{currentfill}%
\pgfsetlinewidth{0.803000pt}%
\definecolor{currentstroke}{rgb}{0.000000,0.000000,0.000000}%
\pgfsetstrokecolor{currentstroke}%
\pgfsetdash{}{0pt}%
\pgfsys@defobject{currentmarker}{\pgfqpoint{-0.048611in}{0.000000in}}{\pgfqpoint{-0.000000in}{0.000000in}}{%
\pgfpathmoveto{\pgfqpoint{-0.000000in}{0.000000in}}%
\pgfpathlineto{\pgfqpoint{-0.048611in}{0.000000in}}%
\pgfusepath{stroke,fill}%
}%
\begin{pgfscope}%
\pgfsys@transformshift{0.432374in}{0.498777in}%
\pgfsys@useobject{currentmarker}{}%
\end{pgfscope}%
\end{pgfscope}%
\begin{pgfscope}%
\definecolor{textcolor}{rgb}{0.000000,0.000000,0.000000}%
\pgfsetstrokecolor{textcolor}%
\pgfsetfillcolor{textcolor}%
\pgftext[x=0.253555in, y=0.440944in, left, base]{\color{textcolor}\rmfamily\fontsize{12.000000}{14.400000}\selectfont \(\displaystyle {0}\)}%
\end{pgfscope}%
\begin{pgfscope}%
\pgfsetbuttcap%
\pgfsetroundjoin%
\definecolor{currentfill}{rgb}{0.000000,0.000000,0.000000}%
\pgfsetfillcolor{currentfill}%
\pgfsetlinewidth{0.803000pt}%
\definecolor{currentstroke}{rgb}{0.000000,0.000000,0.000000}%
\pgfsetstrokecolor{currentstroke}%
\pgfsetdash{}{0pt}%
\pgfsys@defobject{currentmarker}{\pgfqpoint{-0.048611in}{0.000000in}}{\pgfqpoint{-0.000000in}{0.000000in}}{%
\pgfpathmoveto{\pgfqpoint{-0.000000in}{0.000000in}}%
\pgfpathlineto{\pgfqpoint{-0.048611in}{0.000000in}}%
\pgfusepath{stroke,fill}%
}%
\begin{pgfscope}%
\pgfsys@transformshift{0.432374in}{0.955925in}%
\pgfsys@useobject{currentmarker}{}%
\end{pgfscope}%
\end{pgfscope}%
\begin{pgfscope}%
\definecolor{textcolor}{rgb}{0.000000,0.000000,0.000000}%
\pgfsetstrokecolor{textcolor}%
\pgfsetfillcolor{textcolor}%
\pgftext[x=0.253555in, y=0.898092in, left, base]{\color{textcolor}\rmfamily\fontsize{12.000000}{14.400000}\selectfont \(\displaystyle {1}\)}%
\end{pgfscope}%
\begin{pgfscope}%
\pgfsetbuttcap%
\pgfsetroundjoin%
\definecolor{currentfill}{rgb}{0.000000,0.000000,0.000000}%
\pgfsetfillcolor{currentfill}%
\pgfsetlinewidth{0.803000pt}%
\definecolor{currentstroke}{rgb}{0.000000,0.000000,0.000000}%
\pgfsetstrokecolor{currentstroke}%
\pgfsetdash{}{0pt}%
\pgfsys@defobject{currentmarker}{\pgfqpoint{-0.048611in}{0.000000in}}{\pgfqpoint{-0.000000in}{0.000000in}}{%
\pgfpathmoveto{\pgfqpoint{-0.000000in}{0.000000in}}%
\pgfpathlineto{\pgfqpoint{-0.048611in}{0.000000in}}%
\pgfusepath{stroke,fill}%
}%
\begin{pgfscope}%
\pgfsys@transformshift{0.432374in}{1.413073in}%
\pgfsys@useobject{currentmarker}{}%
\end{pgfscope}%
\end{pgfscope}%
\begin{pgfscope}%
\definecolor{textcolor}{rgb}{0.000000,0.000000,0.000000}%
\pgfsetstrokecolor{textcolor}%
\pgfsetfillcolor{textcolor}%
\pgftext[x=0.253555in, y=1.355240in, left, base]{\color{textcolor}\rmfamily\fontsize{12.000000}{14.400000}\selectfont \(\displaystyle {2}\)}%
\end{pgfscope}%
\begin{pgfscope}%
\definecolor{textcolor}{rgb}{0.000000,0.000000,0.000000}%
\pgfsetstrokecolor{textcolor}%
\pgfsetfillcolor{textcolor}%
\pgftext[x=0.198000in,y=1.076277in,,bottom,rotate=90.000000]{\color{textcolor}\rmfamily\fontsize{12.000000}{14.400000}\selectfont Percent of Dataset}%
\end{pgfscope}%
\begin{pgfscope}%
\pgfsetrectcap%
\pgfsetmiterjoin%
\pgfsetlinewidth{0.803000pt}%
\definecolor{currentstroke}{rgb}{0.000000,0.000000,0.000000}%
\pgfsetstrokecolor{currentstroke}%
\pgfsetdash{}{0pt}%
\pgfpathmoveto{\pgfqpoint{0.432374in}{0.498777in}}%
\pgfpathlineto{\pgfqpoint{0.432374in}{1.653777in}}%
\pgfusepath{stroke}%
\end{pgfscope}%
\begin{pgfscope}%
\pgfsetrectcap%
\pgfsetmiterjoin%
\pgfsetlinewidth{0.803000pt}%
\definecolor{currentstroke}{rgb}{0.000000,0.000000,0.000000}%
\pgfsetstrokecolor{currentstroke}%
\pgfsetdash{}{0pt}%
\pgfpathmoveto{\pgfqpoint{3.919874in}{0.498777in}}%
\pgfpathlineto{\pgfqpoint{3.919874in}{1.653777in}}%
\pgfusepath{stroke}%
\end{pgfscope}%
\begin{pgfscope}%
\pgfsetrectcap%
\pgfsetmiterjoin%
\pgfsetlinewidth{0.803000pt}%
\definecolor{currentstroke}{rgb}{0.000000,0.000000,0.000000}%
\pgfsetstrokecolor{currentstroke}%
\pgfsetdash{}{0pt}%
\pgfpathmoveto{\pgfqpoint{0.432374in}{0.498777in}}%
\pgfpathlineto{\pgfqpoint{3.919874in}{0.498777in}}%
\pgfusepath{stroke}%
\end{pgfscope}%
\begin{pgfscope}%
\pgfsetrectcap%
\pgfsetmiterjoin%
\pgfsetlinewidth{0.803000pt}%
\definecolor{currentstroke}{rgb}{0.000000,0.000000,0.000000}%
\pgfsetstrokecolor{currentstroke}%
\pgfsetdash{}{0pt}%
\pgfpathmoveto{\pgfqpoint{0.432374in}{1.653777in}}%
\pgfpathlineto{\pgfqpoint{3.919874in}{1.653777in}}%
\pgfusepath{stroke}%
\end{pgfscope}%
\begin{pgfscope}%
\pgfsetbuttcap%
\pgfsetmiterjoin%
\definecolor{currentfill}{rgb}{1.000000,1.000000,1.000000}%
\pgfsetfillcolor{currentfill}%
\pgfsetfillopacity{0.800000}%
\pgfsetlinewidth{1.003750pt}%
\definecolor{currentstroke}{rgb}{0.800000,0.800000,0.800000}%
\pgfsetstrokecolor{currentstroke}%
\pgfsetstrokeopacity{0.800000}%
\pgfsetdash{}{0pt}%
\pgfpathmoveto{\pgfqpoint{2.993374in}{1.053944in}}%
\pgfpathlineto{\pgfqpoint{3.803207in}{1.053944in}}%
\pgfpathquadraticcurveto{\pgfqpoint{3.836541in}{1.053944in}}{\pgfqpoint{3.836541in}{1.087278in}}%
\pgfpathlineto{\pgfqpoint{3.836541in}{1.537111in}}%
\pgfpathquadraticcurveto{\pgfqpoint{3.836541in}{1.570444in}}{\pgfqpoint{3.803207in}{1.570444in}}%
\pgfpathlineto{\pgfqpoint{2.993374in}{1.570444in}}%
\pgfpathquadraticcurveto{\pgfqpoint{2.960041in}{1.570444in}}{\pgfqpoint{2.960041in}{1.537111in}}%
\pgfpathlineto{\pgfqpoint{2.960041in}{1.087278in}}%
\pgfpathquadraticcurveto{\pgfqpoint{2.960041in}{1.053944in}}{\pgfqpoint{2.993374in}{1.053944in}}%
\pgfpathlineto{\pgfqpoint{2.993374in}{1.053944in}}%
\pgfpathclose%
\pgfusepath{stroke,fill}%
\end{pgfscope}%
\begin{pgfscope}%
\pgfsetbuttcap%
\pgfsetmiterjoin%
\pgfsetlinewidth{1.003750pt}%
\definecolor{currentstroke}{rgb}{0.000000,0.000000,0.000000}%
\pgfsetstrokecolor{currentstroke}%
\pgfsetdash{}{0pt}%
\pgfpathmoveto{\pgfqpoint{3.026707in}{1.387111in}}%
\pgfpathlineto{\pgfqpoint{3.360041in}{1.387111in}}%
\pgfpathlineto{\pgfqpoint{3.360041in}{1.503777in}}%
\pgfpathlineto{\pgfqpoint{3.026707in}{1.503777in}}%
\pgfpathlineto{\pgfqpoint{3.026707in}{1.387111in}}%
\pgfpathclose%
\pgfusepath{stroke}%
\end{pgfscope}%
\begin{pgfscope}%
\definecolor{textcolor}{rgb}{0.000000,0.000000,0.000000}%
\pgfsetstrokecolor{textcolor}%
\pgfsetfillcolor{textcolor}%
\pgftext[x=3.493374in,y=1.387111in,left,base]{\color{textcolor}\rmfamily\fontsize{12.000000}{14.400000}\selectfont Neg}%
\end{pgfscope}%
\begin{pgfscope}%
\pgfsetbuttcap%
\pgfsetmiterjoin%
\definecolor{currentfill}{rgb}{0.000000,0.000000,0.000000}%
\pgfsetfillcolor{currentfill}%
\pgfsetlinewidth{0.000000pt}%
\definecolor{currentstroke}{rgb}{0.000000,0.000000,0.000000}%
\pgfsetstrokecolor{currentstroke}%
\pgfsetstrokeopacity{0.000000}%
\pgfsetdash{}{0pt}%
\pgfpathmoveto{\pgfqpoint{3.026707in}{1.152944in}}%
\pgfpathlineto{\pgfqpoint{3.360041in}{1.152944in}}%
\pgfpathlineto{\pgfqpoint{3.360041in}{1.269611in}}%
\pgfpathlineto{\pgfqpoint{3.026707in}{1.269611in}}%
\pgfpathlineto{\pgfqpoint{3.026707in}{1.152944in}}%
\pgfpathclose%
\pgfusepath{fill}%
\end{pgfscope}%
\begin{pgfscope}%
\definecolor{textcolor}{rgb}{0.000000,0.000000,0.000000}%
\pgfsetstrokecolor{textcolor}%
\pgfsetfillcolor{textcolor}%
\pgftext[x=3.493374in,y=1.152944in,left,base]{\color{textcolor}\rmfamily\fontsize{12.000000}{14.400000}\selectfont Pos}%
\end{pgfscope}%
\end{pgfpicture}%
\makeatother%
\endgroup%
	
\cr
\end{tabular}

	  \caption{\normalfont\normalsize Random Forest Classifier on the Medium Features (top) and Example model (bottom).  Figure accompanies \S\ref{Methods_Model_Failure}}\label{RFC_Medium_Zoom_Figure}
\end{figure}

\FloatBarrier



%%%%%
\subsubsection{Numerical Precision}
\label{Numerical_Precision}

We ran all of the models twice with different random seeds.  As an example of the difference in model results, consider the first budgetary decision metric, the values of $p$ that make FP/P closest to 0.05, because that metric is the most stable.  If there is a difference in those results between two runs of the same model for FP/P, not only can we only claim that much numerical precision in reporting our results for that metric, but we may need to limit ourselves to even less numerical precision in reporting results for the other budgetary decision metrics.  

Table \ref{FP_P_0_05_Numerical_Precision} shows the models with the largest changes between runs.  The median $\Delta p$ was 0.0003, the median $\Delta$ TP was 57, and the median $\Delta$ FP was 1, giving median $\Delta$ FP/P of $0.0000093$.


\begin{table}
\caption{
	\normalsize\normalfont
	Comparison of values of $p$, TP, FP, and FP/P between Runs of the Same Model with Different Random Seeds.
  Table accompanies \S\ref{Methods_Model_Failure}
}
\label{FP_P_0_05_Numerical_Precision}

{\normalsize
\normalfont
\begin{tabular}{lcclrrrrlll}
\toprule
	\multicolumn{1}{c}{Model} &     
	\multicolumn{1}{c}{Features} &    
	\multicolumn{1}{c}{Run} &     	 
	\multicolumn{1}{c}{$p$} &     
	\multicolumn{1}{c}{Neg} &   
	\multicolumn{1}{c}{Pos} & 
	\multicolumn{1}{c}{FP} &      
	\multicolumn{1}{c}{TP} &   
	\multicolumn{1}{c}{FP/P} &
	\multicolumn{1}{c}{Notes} \\
\midrule

Keras $\alpha = 0.5$, $\gamma = 1.0$  & Medium & 0 & 0.50556 & 1 & 0 & 5,398 & 6,294 & 0.05 & \multirow{2}{*}{ $\Delta p = 0.012$}\cr
Keras $\alpha = 0.5$, $\gamma = 1.0$  & Medium &  1 & 0.51764 & 1 & 0 & 5,398 & 6,105 & 0.05\cr\hline
Keras $\alpha = 0.5$, $\gamma = 1.0 $ & Hard & 0  & 0.56429 & 1 & 0 & 5,398 & 10,946 & 0.05 & \multirow{2}{*}{$\Delta \text{TP} = 409$}\cr
Keras $\alpha = 0.5$, $\gamma = 1.0$  & Hard & 1 & 0.55914 & 1 & 0 & 5,398 & 10,537 & 0.05\cr\hline
Easy Ens & Easy & 0 & 0.52846 & 263 & 107 & 5,286 & 2,682 & 0.049& $\Delta \text{FP} = 180$ \cr
Easy Ens & Easy & 1 & 0.52866 & 27 & 15 & 5,466 & 2,741 & 0.0506 & $\Delta \text{FP/P} = 0.0016$\cr

\bottomrule
\end{tabular}
}
\end{table}

\FloatBarrier












