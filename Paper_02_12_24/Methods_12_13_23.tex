%%%%% Methods

%%%
\subsection{Overview}

\begin{itemize}
	\item The dataset
	\item Binning the features
	\item Imputing missing data
	\item Order of Operations in Binning and Imputing
	\item Handling imbalanced data
	\item Choosing features for ``Easy,'' ``Medium,'' and ``Hard'' data sets
	\item Building models
	\item Choosing metrics to interpret the results
	\item Interpreting the results
\end{itemize}

%%%%%
\subsection{Metrics Old and New}

In our work, we are trying to build a binary classification model to predict, based on immediately available information, whether the crash person needs an ambulance.  To build a supervised-learning model, we start with labeled data, a dataset of crashes for which we know whether the person needed an ambulance.  We separate the data into two sets, a training set and test set.  A machine learning algorithm uses the training set to build (``learn'') a model that will return, for each crash person in a set, a value $p \in [0,1]$ that is a proxy for the probability that the crash person needs an ambulance.  We choose a decision threshold $\theta$ (by default $\theta=0.5$) and classify samples (crash persons) with $p>\theta$ as  ``Needs ambulance'' and samples with $p < \theta$ as ``Does not need ambulance.''  Then we apply the model to the test set, which the algorithm did not see in the learning phase.  The test results tell us how many of the crash person who needed an ambulance or did not need an ambulance were correctly classified, and we organize them in a confusion matrix.  

\

\hfil\begin{tabular}{cc|c|c|}
	&\multicolumn{1}{c}{}& \multicolumn{2}{c}{Prediction} \cr
	&\multicolumn{1}{c}{} & \multicolumn{1}{c}{Predicted Negative (PN)} & \multicolumn{1}{c}{Predicted Positive (PP)} \cr\cline{3-4}
	\multirow{2}{*}{\rotatebox[origin=c]{90}{Actual}}&Negative (N) &
True Negative (TN) & False Positive (FP)
	\vrule width 0pt height 10pt depth 2pt \cr\cline{3-4}
	&Positive (P) & False Negative (FN) & True Positive (TP) 
	\vrule width 0pt height 10pt depth 2pt \cr\cline{3-4}
\end{tabular}

\vskip 12pt

Obviously we would prefer a model that returned no false positives or false negatives, but barring that, we would like to come as close as possible, and we need to define ``close'' in a measurable way.  

\

\hfil\begin{tabular}{r >{$\displaystyle}c<{$\vrule width 0pt height 16pt depth 12pt} p{3in}}
	Metric & \text{Formula} & Meaning \cr\hline
	\bf Accuracy & \frac{ \text{TN} + \text{TP} }{ \text{TN} + \text{FP} + \text{FN} + \text{TP}} & Proportion correctly classified \cr
	\bf Precision & \frac{ \text{TP} }{ \text{TP} + \text{FP}} = \frac{ \text{TP} }{ \text{PP} } & Proportion of immediately-dispatched ambulances that are needed \cr
	\bf Recall & \frac{ \text{TP} }{ \text{TP} + \text{FN}} = \frac{ \text{TP} }{ \text{P} } & Proportion of needed ambulances that are immediately dispatched \cr
	\bf F1 & \frac{ 2 }{ \displaystyle \frac{1}{\text{Precision}} + \frac{1}{\text{Recall}} }  & Harmonic mean of Precision and Recall \\[3em]
	& = \frac{ 2 \cdot \text{Precision} \cdot \text{Recall} }{ \text{Precision} + \text{Recall} } \cr
\end{tabular}

\vskip 12pt

Accuracy and precision depend on the class balance, but recall does not.  Most reported crashes are fender benders that do not require an ambulance, so in our dataset, $\text{P} \ll \text{N}$.  If we artificially balance the dataset so that $\text{P} = \text{N}$, either by resampling the data or using class weights in the loss function, then we would change the proportions in accuracy and precision.  In the literature there is a ``balanced accuracy'' that is widely used.  One could make a ``balanced precision'' in the same way, which would lead to a ``balanced F1,'' but we have not seen those used in the literature.  Recall is not affected by class balance because all of its terms (TP and FN) are in the positive class.  

In combining precision and recall to make F1, why do we choose the harmonic mean instead of the arithmetic or geometric mean?  There is a Gmean in the literature, but it is not used as often.    If $a$ and $b$ are such that $0 < a < b$, then $$a < \text{Harm}(a,b) < \text{Geo}(a,b) < \text{Arith}(a,b) < b$$    The harmonic mean leans towards the worse of the two metrics (precision and recall), much like least squares regression emphasizes the points furthest from the line, telling us not only their central tendency but also whether the two metrics are in balance.
  
\

The predicted $p$ values, when paired with the actual $N=0$ and $P=1$ values, combine to make the False Positive Rate and True Positive Rate values that become the parameterized curve called (for historical reasons) the Receiver Operating Characteristic (ROC) curve.  The area under this curve (AUC) indicates how well the model has separated the positive and negative classes.  

The graphs below are for an artificial dataset to illustrate the metrics.  The histogram shows the number of elements of the positive and negative classes in ten ranges of $p$.  The dataset is imbalanced like crash data, with far more crash persons not needing an ambulance than needing one.  

The ROC curve is a parameterization over $p$ of the true positive rate versus the false positive rate, with $p=1$ in the lower left and $p=0$ in the upper right.  If the model perfectly separated the positive and negative classes, the parameterized curve would form a Gamma ($\Gamma$) from $(0,0)$ to $(0,1)$ to $(1,1)$ given $\text{AUC} = 1$.  Random noise would follow the dashed diagonal and give $\text{AUC} = 0.5$, as illustrated further below.  

\

{\bf Ideal Model}  
In this model, most of the negative class samples are on the left (TN, $p<0.5$) and most of the positive class samples are on the right (TP, $p>0.5$), which is what we want.  Under this model, we would immediately dispatch ambulances to 77.7\% of the crash persons who needed one (recall), and 38.7\% of the ambulances we immediately dispatched would be needed (precision).  The harmonic mean of those two numbers is $\text{F1} = 0.517$.  The area under the curve, which increases with how well the model has separated the two classes, is 0.840.

\

\verb|Ideal|

%%%
\parbox{\linewidth}{
\noindent\begin{tabular}{@{\hspace{-6pt}}p{2.3in} @{\hspace{-6pt}}p{2.0in} p{1.8in}}
	\vskip 0pt
	\qquad \qquad Raw Model Output
	
	%% Creator: Matplotlib, PGF backend
%%
%% To include the figure in your LaTeX document, write
%%   \input{<filename>.pgf}
%%
%% Make sure the required packages are loaded in your preamble
%%   \usepackage{pgf}
%%
%% Also ensure that all the required font packages are loaded; for instance,
%% the lmodern package is sometimes necessary when using math font.
%%   \usepackage{lmodern}
%%
%% Figures using additional raster images can only be included by \input if
%% they are in the same directory as the main LaTeX file. For loading figures
%% from other directories you can use the `import` package
%%   \usepackage{import}
%%
%% and then include the figures with
%%   \import{<path to file>}{<filename>.pgf}
%%
%% Matplotlib used the following preamble
%%   
%%   \usepackage{fontspec}
%%   \makeatletter\@ifpackageloaded{underscore}{}{\usepackage[strings]{underscore}}\makeatother
%%
\begingroup%
\makeatletter%
\begin{pgfpicture}%
\pgfpathrectangle{\pgfpointorigin}{\pgfqpoint{2.153750in}{1.654444in}}%
\pgfusepath{use as bounding box, clip}%
\begin{pgfscope}%
\pgfsetbuttcap%
\pgfsetmiterjoin%
\definecolor{currentfill}{rgb}{1.000000,1.000000,1.000000}%
\pgfsetfillcolor{currentfill}%
\pgfsetlinewidth{0.000000pt}%
\definecolor{currentstroke}{rgb}{1.000000,1.000000,1.000000}%
\pgfsetstrokecolor{currentstroke}%
\pgfsetdash{}{0pt}%
\pgfpathmoveto{\pgfqpoint{0.000000in}{0.000000in}}%
\pgfpathlineto{\pgfqpoint{2.153750in}{0.000000in}}%
\pgfpathlineto{\pgfqpoint{2.153750in}{1.654444in}}%
\pgfpathlineto{\pgfqpoint{0.000000in}{1.654444in}}%
\pgfpathlineto{\pgfqpoint{0.000000in}{0.000000in}}%
\pgfpathclose%
\pgfusepath{fill}%
\end{pgfscope}%
\begin{pgfscope}%
\pgfsetbuttcap%
\pgfsetmiterjoin%
\definecolor{currentfill}{rgb}{1.000000,1.000000,1.000000}%
\pgfsetfillcolor{currentfill}%
\pgfsetlinewidth{0.000000pt}%
\definecolor{currentstroke}{rgb}{0.000000,0.000000,0.000000}%
\pgfsetstrokecolor{currentstroke}%
\pgfsetstrokeopacity{0.000000}%
\pgfsetdash{}{0pt}%
\pgfpathmoveto{\pgfqpoint{0.465000in}{0.449444in}}%
\pgfpathlineto{\pgfqpoint{2.015000in}{0.449444in}}%
\pgfpathlineto{\pgfqpoint{2.015000in}{1.604444in}}%
\pgfpathlineto{\pgfqpoint{0.465000in}{1.604444in}}%
\pgfpathlineto{\pgfqpoint{0.465000in}{0.449444in}}%
\pgfpathclose%
\pgfusepath{fill}%
\end{pgfscope}%
\begin{pgfscope}%
\pgfpathrectangle{\pgfqpoint{0.465000in}{0.449444in}}{\pgfqpoint{1.550000in}{1.155000in}}%
\pgfusepath{clip}%
\pgfsetbuttcap%
\pgfsetmiterjoin%
\pgfsetlinewidth{1.003750pt}%
\definecolor{currentstroke}{rgb}{0.000000,0.000000,0.000000}%
\pgfsetstrokecolor{currentstroke}%
\pgfsetdash{}{0pt}%
\pgfpathmoveto{\pgfqpoint{0.455000in}{0.449444in}}%
\pgfpathlineto{\pgfqpoint{0.502805in}{0.449444in}}%
\pgfpathlineto{\pgfqpoint{0.502805in}{0.590081in}}%
\pgfpathlineto{\pgfqpoint{0.455000in}{0.590081in}}%
\pgfusepath{stroke}%
\end{pgfscope}%
\begin{pgfscope}%
\pgfpathrectangle{\pgfqpoint{0.465000in}{0.449444in}}{\pgfqpoint{1.550000in}{1.155000in}}%
\pgfusepath{clip}%
\pgfsetbuttcap%
\pgfsetmiterjoin%
\pgfsetlinewidth{1.003750pt}%
\definecolor{currentstroke}{rgb}{0.000000,0.000000,0.000000}%
\pgfsetstrokecolor{currentstroke}%
\pgfsetdash{}{0pt}%
\pgfpathmoveto{\pgfqpoint{0.593537in}{0.449444in}}%
\pgfpathlineto{\pgfqpoint{0.654025in}{0.449444in}}%
\pgfpathlineto{\pgfqpoint{0.654025in}{1.204573in}}%
\pgfpathlineto{\pgfqpoint{0.593537in}{1.204573in}}%
\pgfpathlineto{\pgfqpoint{0.593537in}{0.449444in}}%
\pgfpathclose%
\pgfusepath{stroke}%
\end{pgfscope}%
\begin{pgfscope}%
\pgfpathrectangle{\pgfqpoint{0.465000in}{0.449444in}}{\pgfqpoint{1.550000in}{1.155000in}}%
\pgfusepath{clip}%
\pgfsetbuttcap%
\pgfsetmiterjoin%
\pgfsetlinewidth{1.003750pt}%
\definecolor{currentstroke}{rgb}{0.000000,0.000000,0.000000}%
\pgfsetstrokecolor{currentstroke}%
\pgfsetdash{}{0pt}%
\pgfpathmoveto{\pgfqpoint{0.744756in}{0.449444in}}%
\pgfpathlineto{\pgfqpoint{0.805244in}{0.449444in}}%
\pgfpathlineto{\pgfqpoint{0.805244in}{1.549444in}}%
\pgfpathlineto{\pgfqpoint{0.744756in}{1.549444in}}%
\pgfpathlineto{\pgfqpoint{0.744756in}{0.449444in}}%
\pgfpathclose%
\pgfusepath{stroke}%
\end{pgfscope}%
\begin{pgfscope}%
\pgfpathrectangle{\pgfqpoint{0.465000in}{0.449444in}}{\pgfqpoint{1.550000in}{1.155000in}}%
\pgfusepath{clip}%
\pgfsetbuttcap%
\pgfsetmiterjoin%
\pgfsetlinewidth{1.003750pt}%
\definecolor{currentstroke}{rgb}{0.000000,0.000000,0.000000}%
\pgfsetstrokecolor{currentstroke}%
\pgfsetdash{}{0pt}%
\pgfpathmoveto{\pgfqpoint{0.895976in}{0.449444in}}%
\pgfpathlineto{\pgfqpoint{0.956464in}{0.449444in}}%
\pgfpathlineto{\pgfqpoint{0.956464in}{1.447255in}}%
\pgfpathlineto{\pgfqpoint{0.895976in}{1.447255in}}%
\pgfpathlineto{\pgfqpoint{0.895976in}{0.449444in}}%
\pgfpathclose%
\pgfusepath{stroke}%
\end{pgfscope}%
\begin{pgfscope}%
\pgfpathrectangle{\pgfqpoint{0.465000in}{0.449444in}}{\pgfqpoint{1.550000in}{1.155000in}}%
\pgfusepath{clip}%
\pgfsetbuttcap%
\pgfsetmiterjoin%
\pgfsetlinewidth{1.003750pt}%
\definecolor{currentstroke}{rgb}{0.000000,0.000000,0.000000}%
\pgfsetstrokecolor{currentstroke}%
\pgfsetdash{}{0pt}%
\pgfpathmoveto{\pgfqpoint{1.047195in}{0.449444in}}%
\pgfpathlineto{\pgfqpoint{1.107683in}{0.449444in}}%
\pgfpathlineto{\pgfqpoint{1.107683in}{1.160058in}}%
\pgfpathlineto{\pgfqpoint{1.047195in}{1.160058in}}%
\pgfpathlineto{\pgfqpoint{1.047195in}{0.449444in}}%
\pgfpathclose%
\pgfusepath{stroke}%
\end{pgfscope}%
\begin{pgfscope}%
\pgfpathrectangle{\pgfqpoint{0.465000in}{0.449444in}}{\pgfqpoint{1.550000in}{1.155000in}}%
\pgfusepath{clip}%
\pgfsetbuttcap%
\pgfsetmiterjoin%
\pgfsetlinewidth{1.003750pt}%
\definecolor{currentstroke}{rgb}{0.000000,0.000000,0.000000}%
\pgfsetstrokecolor{currentstroke}%
\pgfsetdash{}{0pt}%
\pgfpathmoveto{\pgfqpoint{1.198415in}{0.449444in}}%
\pgfpathlineto{\pgfqpoint{1.258903in}{0.449444in}}%
\pgfpathlineto{\pgfqpoint{1.258903in}{0.885778in}}%
\pgfpathlineto{\pgfqpoint{1.198415in}{0.885778in}}%
\pgfpathlineto{\pgfqpoint{1.198415in}{0.449444in}}%
\pgfpathclose%
\pgfusepath{stroke}%
\end{pgfscope}%
\begin{pgfscope}%
\pgfpathrectangle{\pgfqpoint{0.465000in}{0.449444in}}{\pgfqpoint{1.550000in}{1.155000in}}%
\pgfusepath{clip}%
\pgfsetbuttcap%
\pgfsetmiterjoin%
\pgfsetlinewidth{1.003750pt}%
\definecolor{currentstroke}{rgb}{0.000000,0.000000,0.000000}%
\pgfsetstrokecolor{currentstroke}%
\pgfsetdash{}{0pt}%
\pgfpathmoveto{\pgfqpoint{1.349634in}{0.449444in}}%
\pgfpathlineto{\pgfqpoint{1.410122in}{0.449444in}}%
\pgfpathlineto{\pgfqpoint{1.410122in}{0.731988in}}%
\pgfpathlineto{\pgfqpoint{1.349634in}{0.731988in}}%
\pgfpathlineto{\pgfqpoint{1.349634in}{0.449444in}}%
\pgfpathclose%
\pgfusepath{stroke}%
\end{pgfscope}%
\begin{pgfscope}%
\pgfpathrectangle{\pgfqpoint{0.465000in}{0.449444in}}{\pgfqpoint{1.550000in}{1.155000in}}%
\pgfusepath{clip}%
\pgfsetbuttcap%
\pgfsetmiterjoin%
\pgfsetlinewidth{1.003750pt}%
\definecolor{currentstroke}{rgb}{0.000000,0.000000,0.000000}%
\pgfsetstrokecolor{currentstroke}%
\pgfsetdash{}{0pt}%
\pgfpathmoveto{\pgfqpoint{1.500854in}{0.449444in}}%
\pgfpathlineto{\pgfqpoint{1.561342in}{0.449444in}}%
\pgfpathlineto{\pgfqpoint{1.561342in}{0.577885in}}%
\pgfpathlineto{\pgfqpoint{1.500854in}{0.577885in}}%
\pgfpathlineto{\pgfqpoint{1.500854in}{0.449444in}}%
\pgfpathclose%
\pgfusepath{stroke}%
\end{pgfscope}%
\begin{pgfscope}%
\pgfpathrectangle{\pgfqpoint{0.465000in}{0.449444in}}{\pgfqpoint{1.550000in}{1.155000in}}%
\pgfusepath{clip}%
\pgfsetbuttcap%
\pgfsetmiterjoin%
\pgfsetlinewidth{1.003750pt}%
\definecolor{currentstroke}{rgb}{0.000000,0.000000,0.000000}%
\pgfsetstrokecolor{currentstroke}%
\pgfsetdash{}{0pt}%
\pgfpathmoveto{\pgfqpoint{1.652073in}{0.449444in}}%
\pgfpathlineto{\pgfqpoint{1.712561in}{0.449444in}}%
\pgfpathlineto{\pgfqpoint{1.712561in}{0.513416in}}%
\pgfpathlineto{\pgfqpoint{1.652073in}{0.513416in}}%
\pgfpathlineto{\pgfqpoint{1.652073in}{0.449444in}}%
\pgfpathclose%
\pgfusepath{stroke}%
\end{pgfscope}%
\begin{pgfscope}%
\pgfpathrectangle{\pgfqpoint{0.465000in}{0.449444in}}{\pgfqpoint{1.550000in}{1.155000in}}%
\pgfusepath{clip}%
\pgfsetbuttcap%
\pgfsetmiterjoin%
\pgfsetlinewidth{1.003750pt}%
\definecolor{currentstroke}{rgb}{0.000000,0.000000,0.000000}%
\pgfsetstrokecolor{currentstroke}%
\pgfsetdash{}{0pt}%
\pgfpathmoveto{\pgfqpoint{1.803293in}{0.449444in}}%
\pgfpathlineto{\pgfqpoint{1.863781in}{0.449444in}}%
\pgfpathlineto{\pgfqpoint{1.863781in}{0.467616in}}%
\pgfpathlineto{\pgfqpoint{1.803293in}{0.467616in}}%
\pgfpathlineto{\pgfqpoint{1.803293in}{0.449444in}}%
\pgfpathclose%
\pgfusepath{stroke}%
\end{pgfscope}%
\begin{pgfscope}%
\pgfpathrectangle{\pgfqpoint{0.465000in}{0.449444in}}{\pgfqpoint{1.550000in}{1.155000in}}%
\pgfusepath{clip}%
\pgfsetbuttcap%
\pgfsetmiterjoin%
\definecolor{currentfill}{rgb}{0.000000,0.000000,0.000000}%
\pgfsetfillcolor{currentfill}%
\pgfsetlinewidth{0.000000pt}%
\definecolor{currentstroke}{rgb}{0.000000,0.000000,0.000000}%
\pgfsetstrokecolor{currentstroke}%
\pgfsetstrokeopacity{0.000000}%
\pgfsetdash{}{0pt}%
\pgfpathmoveto{\pgfqpoint{0.502805in}{0.449444in}}%
\pgfpathlineto{\pgfqpoint{0.563293in}{0.449444in}}%
\pgfpathlineto{\pgfqpoint{0.563293in}{0.456162in}}%
\pgfpathlineto{\pgfqpoint{0.502805in}{0.456162in}}%
\pgfpathlineto{\pgfqpoint{0.502805in}{0.449444in}}%
\pgfpathclose%
\pgfusepath{fill}%
\end{pgfscope}%
\begin{pgfscope}%
\pgfpathrectangle{\pgfqpoint{0.465000in}{0.449444in}}{\pgfqpoint{1.550000in}{1.155000in}}%
\pgfusepath{clip}%
\pgfsetbuttcap%
\pgfsetmiterjoin%
\definecolor{currentfill}{rgb}{0.000000,0.000000,0.000000}%
\pgfsetfillcolor{currentfill}%
\pgfsetlinewidth{0.000000pt}%
\definecolor{currentstroke}{rgb}{0.000000,0.000000,0.000000}%
\pgfsetstrokecolor{currentstroke}%
\pgfsetstrokeopacity{0.000000}%
\pgfsetdash{}{0pt}%
\pgfpathmoveto{\pgfqpoint{0.654025in}{0.449444in}}%
\pgfpathlineto{\pgfqpoint{0.714512in}{0.449444in}}%
\pgfpathlineto{\pgfqpoint{0.714512in}{0.462803in}}%
\pgfpathlineto{\pgfqpoint{0.654025in}{0.462803in}}%
\pgfpathlineto{\pgfqpoint{0.654025in}{0.449444in}}%
\pgfpathclose%
\pgfusepath{fill}%
\end{pgfscope}%
\begin{pgfscope}%
\pgfpathrectangle{\pgfqpoint{0.465000in}{0.449444in}}{\pgfqpoint{1.550000in}{1.155000in}}%
\pgfusepath{clip}%
\pgfsetbuttcap%
\pgfsetmiterjoin%
\definecolor{currentfill}{rgb}{0.000000,0.000000,0.000000}%
\pgfsetfillcolor{currentfill}%
\pgfsetlinewidth{0.000000pt}%
\definecolor{currentstroke}{rgb}{0.000000,0.000000,0.000000}%
\pgfsetstrokecolor{currentstroke}%
\pgfsetstrokeopacity{0.000000}%
\pgfsetdash{}{0pt}%
\pgfpathmoveto{\pgfqpoint{0.805244in}{0.449444in}}%
\pgfpathlineto{\pgfqpoint{0.865732in}{0.449444in}}%
\pgfpathlineto{\pgfqpoint{0.865732in}{0.475374in}}%
\pgfpathlineto{\pgfqpoint{0.805244in}{0.475374in}}%
\pgfpathlineto{\pgfqpoint{0.805244in}{0.449444in}}%
\pgfpathclose%
\pgfusepath{fill}%
\end{pgfscope}%
\begin{pgfscope}%
\pgfpathrectangle{\pgfqpoint{0.465000in}{0.449444in}}{\pgfqpoint{1.550000in}{1.155000in}}%
\pgfusepath{clip}%
\pgfsetbuttcap%
\pgfsetmiterjoin%
\definecolor{currentfill}{rgb}{0.000000,0.000000,0.000000}%
\pgfsetfillcolor{currentfill}%
\pgfsetlinewidth{0.000000pt}%
\definecolor{currentstroke}{rgb}{0.000000,0.000000,0.000000}%
\pgfsetstrokecolor{currentstroke}%
\pgfsetstrokeopacity{0.000000}%
\pgfsetdash{}{0pt}%
\pgfpathmoveto{\pgfqpoint{0.956464in}{0.449444in}}%
\pgfpathlineto{\pgfqpoint{1.016951in}{0.449444in}}%
\pgfpathlineto{\pgfqpoint{1.016951in}{0.496652in}}%
\pgfpathlineto{\pgfqpoint{0.956464in}{0.496652in}}%
\pgfpathlineto{\pgfqpoint{0.956464in}{0.449444in}}%
\pgfpathclose%
\pgfusepath{fill}%
\end{pgfscope}%
\begin{pgfscope}%
\pgfpathrectangle{\pgfqpoint{0.465000in}{0.449444in}}{\pgfqpoint{1.550000in}{1.155000in}}%
\pgfusepath{clip}%
\pgfsetbuttcap%
\pgfsetmiterjoin%
\definecolor{currentfill}{rgb}{0.000000,0.000000,0.000000}%
\pgfsetfillcolor{currentfill}%
\pgfsetlinewidth{0.000000pt}%
\definecolor{currentstroke}{rgb}{0.000000,0.000000,0.000000}%
\pgfsetstrokecolor{currentstroke}%
\pgfsetstrokeopacity{0.000000}%
\pgfsetdash{}{0pt}%
\pgfpathmoveto{\pgfqpoint{1.107683in}{0.449444in}}%
\pgfpathlineto{\pgfqpoint{1.168171in}{0.449444in}}%
\pgfpathlineto{\pgfqpoint{1.168171in}{0.530884in}}%
\pgfpathlineto{\pgfqpoint{1.107683in}{0.530884in}}%
\pgfpathlineto{\pgfqpoint{1.107683in}{0.449444in}}%
\pgfpathclose%
\pgfusepath{fill}%
\end{pgfscope}%
\begin{pgfscope}%
\pgfpathrectangle{\pgfqpoint{0.465000in}{0.449444in}}{\pgfqpoint{1.550000in}{1.155000in}}%
\pgfusepath{clip}%
\pgfsetbuttcap%
\pgfsetmiterjoin%
\definecolor{currentfill}{rgb}{0.000000,0.000000,0.000000}%
\pgfsetfillcolor{currentfill}%
\pgfsetlinewidth{0.000000pt}%
\definecolor{currentstroke}{rgb}{0.000000,0.000000,0.000000}%
\pgfsetstrokecolor{currentstroke}%
\pgfsetstrokeopacity{0.000000}%
\pgfsetdash{}{0pt}%
\pgfpathmoveto{\pgfqpoint{1.258903in}{0.449444in}}%
\pgfpathlineto{\pgfqpoint{1.319391in}{0.449444in}}%
\pgfpathlineto{\pgfqpoint{1.319391in}{0.576975in}}%
\pgfpathlineto{\pgfqpoint{1.258903in}{0.576975in}}%
\pgfpathlineto{\pgfqpoint{1.258903in}{0.449444in}}%
\pgfpathclose%
\pgfusepath{fill}%
\end{pgfscope}%
\begin{pgfscope}%
\pgfpathrectangle{\pgfqpoint{0.465000in}{0.449444in}}{\pgfqpoint{1.550000in}{1.155000in}}%
\pgfusepath{clip}%
\pgfsetbuttcap%
\pgfsetmiterjoin%
\definecolor{currentfill}{rgb}{0.000000,0.000000,0.000000}%
\pgfsetfillcolor{currentfill}%
\pgfsetlinewidth{0.000000pt}%
\definecolor{currentstroke}{rgb}{0.000000,0.000000,0.000000}%
\pgfsetstrokecolor{currentstroke}%
\pgfsetstrokeopacity{0.000000}%
\pgfsetdash{}{0pt}%
\pgfpathmoveto{\pgfqpoint{1.410122in}{0.449444in}}%
\pgfpathlineto{\pgfqpoint{1.470610in}{0.449444in}}%
\pgfpathlineto{\pgfqpoint{1.470610in}{0.624175in}}%
\pgfpathlineto{\pgfqpoint{1.410122in}{0.624175in}}%
\pgfpathlineto{\pgfqpoint{1.410122in}{0.449444in}}%
\pgfpathclose%
\pgfusepath{fill}%
\end{pgfscope}%
\begin{pgfscope}%
\pgfpathrectangle{\pgfqpoint{0.465000in}{0.449444in}}{\pgfqpoint{1.550000in}{1.155000in}}%
\pgfusepath{clip}%
\pgfsetbuttcap%
\pgfsetmiterjoin%
\definecolor{currentfill}{rgb}{0.000000,0.000000,0.000000}%
\pgfsetfillcolor{currentfill}%
\pgfsetlinewidth{0.000000pt}%
\definecolor{currentstroke}{rgb}{0.000000,0.000000,0.000000}%
\pgfsetstrokecolor{currentstroke}%
\pgfsetstrokeopacity{0.000000}%
\pgfsetdash{}{0pt}%
\pgfpathmoveto{\pgfqpoint{1.561342in}{0.449444in}}%
\pgfpathlineto{\pgfqpoint{1.621830in}{0.449444in}}%
\pgfpathlineto{\pgfqpoint{1.621830in}{0.643181in}}%
\pgfpathlineto{\pgfqpoint{1.561342in}{0.643181in}}%
\pgfpathlineto{\pgfqpoint{1.561342in}{0.449444in}}%
\pgfpathclose%
\pgfusepath{fill}%
\end{pgfscope}%
\begin{pgfscope}%
\pgfpathrectangle{\pgfqpoint{0.465000in}{0.449444in}}{\pgfqpoint{1.550000in}{1.155000in}}%
\pgfusepath{clip}%
\pgfsetbuttcap%
\pgfsetmiterjoin%
\definecolor{currentfill}{rgb}{0.000000,0.000000,0.000000}%
\pgfsetfillcolor{currentfill}%
\pgfsetlinewidth{0.000000pt}%
\definecolor{currentstroke}{rgb}{0.000000,0.000000,0.000000}%
\pgfsetstrokecolor{currentstroke}%
\pgfsetstrokeopacity{0.000000}%
\pgfsetdash{}{0pt}%
\pgfpathmoveto{\pgfqpoint{1.712561in}{0.449444in}}%
\pgfpathlineto{\pgfqpoint{1.773049in}{0.449444in}}%
\pgfpathlineto{\pgfqpoint{1.773049in}{0.579591in}}%
\pgfpathlineto{\pgfqpoint{1.712561in}{0.579591in}}%
\pgfpathlineto{\pgfqpoint{1.712561in}{0.449444in}}%
\pgfpathclose%
\pgfusepath{fill}%
\end{pgfscope}%
\begin{pgfscope}%
\pgfpathrectangle{\pgfqpoint{0.465000in}{0.449444in}}{\pgfqpoint{1.550000in}{1.155000in}}%
\pgfusepath{clip}%
\pgfsetbuttcap%
\pgfsetmiterjoin%
\definecolor{currentfill}{rgb}{0.000000,0.000000,0.000000}%
\pgfsetfillcolor{currentfill}%
\pgfsetlinewidth{0.000000pt}%
\definecolor{currentstroke}{rgb}{0.000000,0.000000,0.000000}%
\pgfsetstrokecolor{currentstroke}%
\pgfsetstrokeopacity{0.000000}%
\pgfsetdash{}{0pt}%
\pgfpathmoveto{\pgfqpoint{1.863781in}{0.449444in}}%
\pgfpathlineto{\pgfqpoint{1.924269in}{0.449444in}}%
\pgfpathlineto{\pgfqpoint{1.924269in}{0.474640in}}%
\pgfpathlineto{\pgfqpoint{1.863781in}{0.474640in}}%
\pgfpathlineto{\pgfqpoint{1.863781in}{0.449444in}}%
\pgfpathclose%
\pgfusepath{fill}%
\end{pgfscope}%
\begin{pgfscope}%
\pgfsetbuttcap%
\pgfsetroundjoin%
\definecolor{currentfill}{rgb}{0.000000,0.000000,0.000000}%
\pgfsetfillcolor{currentfill}%
\pgfsetlinewidth{0.803000pt}%
\definecolor{currentstroke}{rgb}{0.000000,0.000000,0.000000}%
\pgfsetstrokecolor{currentstroke}%
\pgfsetdash{}{0pt}%
\pgfsys@defobject{currentmarker}{\pgfqpoint{0.000000in}{-0.048611in}}{\pgfqpoint{0.000000in}{0.000000in}}{%
\pgfpathmoveto{\pgfqpoint{0.000000in}{0.000000in}}%
\pgfpathlineto{\pgfqpoint{0.000000in}{-0.048611in}}%
\pgfusepath{stroke,fill}%
}%
\begin{pgfscope}%
\pgfsys@transformshift{0.502805in}{0.449444in}%
\pgfsys@useobject{currentmarker}{}%
\end{pgfscope}%
\end{pgfscope}%
\begin{pgfscope}%
\definecolor{textcolor}{rgb}{0.000000,0.000000,0.000000}%
\pgfsetstrokecolor{textcolor}%
\pgfsetfillcolor{textcolor}%
\pgftext[x=0.502805in,y=0.352222in,,top]{\color{textcolor}\rmfamily\fontsize{10.000000}{12.000000}\selectfont 0.0}%
\end{pgfscope}%
\begin{pgfscope}%
\pgfsetbuttcap%
\pgfsetroundjoin%
\definecolor{currentfill}{rgb}{0.000000,0.000000,0.000000}%
\pgfsetfillcolor{currentfill}%
\pgfsetlinewidth{0.803000pt}%
\definecolor{currentstroke}{rgb}{0.000000,0.000000,0.000000}%
\pgfsetstrokecolor{currentstroke}%
\pgfsetdash{}{0pt}%
\pgfsys@defobject{currentmarker}{\pgfqpoint{0.000000in}{-0.048611in}}{\pgfqpoint{0.000000in}{0.000000in}}{%
\pgfpathmoveto{\pgfqpoint{0.000000in}{0.000000in}}%
\pgfpathlineto{\pgfqpoint{0.000000in}{-0.048611in}}%
\pgfusepath{stroke,fill}%
}%
\begin{pgfscope}%
\pgfsys@transformshift{0.880854in}{0.449444in}%
\pgfsys@useobject{currentmarker}{}%
\end{pgfscope}%
\end{pgfscope}%
\begin{pgfscope}%
\definecolor{textcolor}{rgb}{0.000000,0.000000,0.000000}%
\pgfsetstrokecolor{textcolor}%
\pgfsetfillcolor{textcolor}%
\pgftext[x=0.880854in,y=0.352222in,,top]{\color{textcolor}\rmfamily\fontsize{10.000000}{12.000000}\selectfont 0.25}%
\end{pgfscope}%
\begin{pgfscope}%
\pgfsetbuttcap%
\pgfsetroundjoin%
\definecolor{currentfill}{rgb}{0.000000,0.000000,0.000000}%
\pgfsetfillcolor{currentfill}%
\pgfsetlinewidth{0.803000pt}%
\definecolor{currentstroke}{rgb}{0.000000,0.000000,0.000000}%
\pgfsetstrokecolor{currentstroke}%
\pgfsetdash{}{0pt}%
\pgfsys@defobject{currentmarker}{\pgfqpoint{0.000000in}{-0.048611in}}{\pgfqpoint{0.000000in}{0.000000in}}{%
\pgfpathmoveto{\pgfqpoint{0.000000in}{0.000000in}}%
\pgfpathlineto{\pgfqpoint{0.000000in}{-0.048611in}}%
\pgfusepath{stroke,fill}%
}%
\begin{pgfscope}%
\pgfsys@transformshift{1.258903in}{0.449444in}%
\pgfsys@useobject{currentmarker}{}%
\end{pgfscope}%
\end{pgfscope}%
\begin{pgfscope}%
\definecolor{textcolor}{rgb}{0.000000,0.000000,0.000000}%
\pgfsetstrokecolor{textcolor}%
\pgfsetfillcolor{textcolor}%
\pgftext[x=1.258903in,y=0.352222in,,top]{\color{textcolor}\rmfamily\fontsize{10.000000}{12.000000}\selectfont 0.5}%
\end{pgfscope}%
\begin{pgfscope}%
\pgfsetbuttcap%
\pgfsetroundjoin%
\definecolor{currentfill}{rgb}{0.000000,0.000000,0.000000}%
\pgfsetfillcolor{currentfill}%
\pgfsetlinewidth{0.803000pt}%
\definecolor{currentstroke}{rgb}{0.000000,0.000000,0.000000}%
\pgfsetstrokecolor{currentstroke}%
\pgfsetdash{}{0pt}%
\pgfsys@defobject{currentmarker}{\pgfqpoint{0.000000in}{-0.048611in}}{\pgfqpoint{0.000000in}{0.000000in}}{%
\pgfpathmoveto{\pgfqpoint{0.000000in}{0.000000in}}%
\pgfpathlineto{\pgfqpoint{0.000000in}{-0.048611in}}%
\pgfusepath{stroke,fill}%
}%
\begin{pgfscope}%
\pgfsys@transformshift{1.636951in}{0.449444in}%
\pgfsys@useobject{currentmarker}{}%
\end{pgfscope}%
\end{pgfscope}%
\begin{pgfscope}%
\definecolor{textcolor}{rgb}{0.000000,0.000000,0.000000}%
\pgfsetstrokecolor{textcolor}%
\pgfsetfillcolor{textcolor}%
\pgftext[x=1.636951in,y=0.352222in,,top]{\color{textcolor}\rmfamily\fontsize{10.000000}{12.000000}\selectfont 0.75}%
\end{pgfscope}%
\begin{pgfscope}%
\pgfsetbuttcap%
\pgfsetroundjoin%
\definecolor{currentfill}{rgb}{0.000000,0.000000,0.000000}%
\pgfsetfillcolor{currentfill}%
\pgfsetlinewidth{0.803000pt}%
\definecolor{currentstroke}{rgb}{0.000000,0.000000,0.000000}%
\pgfsetstrokecolor{currentstroke}%
\pgfsetdash{}{0pt}%
\pgfsys@defobject{currentmarker}{\pgfqpoint{0.000000in}{-0.048611in}}{\pgfqpoint{0.000000in}{0.000000in}}{%
\pgfpathmoveto{\pgfqpoint{0.000000in}{0.000000in}}%
\pgfpathlineto{\pgfqpoint{0.000000in}{-0.048611in}}%
\pgfusepath{stroke,fill}%
}%
\begin{pgfscope}%
\pgfsys@transformshift{2.015000in}{0.449444in}%
\pgfsys@useobject{currentmarker}{}%
\end{pgfscope}%
\end{pgfscope}%
\begin{pgfscope}%
\definecolor{textcolor}{rgb}{0.000000,0.000000,0.000000}%
\pgfsetstrokecolor{textcolor}%
\pgfsetfillcolor{textcolor}%
\pgftext[x=2.015000in,y=0.352222in,,top]{\color{textcolor}\rmfamily\fontsize{10.000000}{12.000000}\selectfont 1.0}%
\end{pgfscope}%
\begin{pgfscope}%
\definecolor{textcolor}{rgb}{0.000000,0.000000,0.000000}%
\pgfsetstrokecolor{textcolor}%
\pgfsetfillcolor{textcolor}%
\pgftext[x=1.240000in,y=0.173333in,,top]{\color{textcolor}\rmfamily\fontsize{10.000000}{12.000000}\selectfont \(\displaystyle p\)}%
\end{pgfscope}%
\begin{pgfscope}%
\pgfsetbuttcap%
\pgfsetroundjoin%
\definecolor{currentfill}{rgb}{0.000000,0.000000,0.000000}%
\pgfsetfillcolor{currentfill}%
\pgfsetlinewidth{0.803000pt}%
\definecolor{currentstroke}{rgb}{0.000000,0.000000,0.000000}%
\pgfsetstrokecolor{currentstroke}%
\pgfsetdash{}{0pt}%
\pgfsys@defobject{currentmarker}{\pgfqpoint{-0.048611in}{0.000000in}}{\pgfqpoint{-0.000000in}{0.000000in}}{%
\pgfpathmoveto{\pgfqpoint{-0.000000in}{0.000000in}}%
\pgfpathlineto{\pgfqpoint{-0.048611in}{0.000000in}}%
\pgfusepath{stroke,fill}%
}%
\begin{pgfscope}%
\pgfsys@transformshift{0.465000in}{0.449444in}%
\pgfsys@useobject{currentmarker}{}%
\end{pgfscope}%
\end{pgfscope}%
\begin{pgfscope}%
\definecolor{textcolor}{rgb}{0.000000,0.000000,0.000000}%
\pgfsetstrokecolor{textcolor}%
\pgfsetfillcolor{textcolor}%
\pgftext[x=0.298333in, y=0.401250in, left, base]{\color{textcolor}\rmfamily\fontsize{10.000000}{12.000000}\selectfont \(\displaystyle {0}\)}%
\end{pgfscope}%
\begin{pgfscope}%
\pgfsetbuttcap%
\pgfsetroundjoin%
\definecolor{currentfill}{rgb}{0.000000,0.000000,0.000000}%
\pgfsetfillcolor{currentfill}%
\pgfsetlinewidth{0.803000pt}%
\definecolor{currentstroke}{rgb}{0.000000,0.000000,0.000000}%
\pgfsetstrokecolor{currentstroke}%
\pgfsetdash{}{0pt}%
\pgfsys@defobject{currentmarker}{\pgfqpoint{-0.048611in}{0.000000in}}{\pgfqpoint{-0.000000in}{0.000000in}}{%
\pgfpathmoveto{\pgfqpoint{-0.000000in}{0.000000in}}%
\pgfpathlineto{\pgfqpoint{-0.048611in}{0.000000in}}%
\pgfusepath{stroke,fill}%
}%
\begin{pgfscope}%
\pgfsys@transformshift{0.465000in}{0.995409in}%
\pgfsys@useobject{currentmarker}{}%
\end{pgfscope}%
\end{pgfscope}%
\begin{pgfscope}%
\definecolor{textcolor}{rgb}{0.000000,0.000000,0.000000}%
\pgfsetstrokecolor{textcolor}%
\pgfsetfillcolor{textcolor}%
\pgftext[x=0.228889in, y=0.947214in, left, base]{\color{textcolor}\rmfamily\fontsize{10.000000}{12.000000}\selectfont \(\displaystyle {10}\)}%
\end{pgfscope}%
\begin{pgfscope}%
\pgfsetbuttcap%
\pgfsetroundjoin%
\definecolor{currentfill}{rgb}{0.000000,0.000000,0.000000}%
\pgfsetfillcolor{currentfill}%
\pgfsetlinewidth{0.803000pt}%
\definecolor{currentstroke}{rgb}{0.000000,0.000000,0.000000}%
\pgfsetstrokecolor{currentstroke}%
\pgfsetdash{}{0pt}%
\pgfsys@defobject{currentmarker}{\pgfqpoint{-0.048611in}{0.000000in}}{\pgfqpoint{-0.000000in}{0.000000in}}{%
\pgfpathmoveto{\pgfqpoint{-0.000000in}{0.000000in}}%
\pgfpathlineto{\pgfqpoint{-0.048611in}{0.000000in}}%
\pgfusepath{stroke,fill}%
}%
\begin{pgfscope}%
\pgfsys@transformshift{0.465000in}{1.541374in}%
\pgfsys@useobject{currentmarker}{}%
\end{pgfscope}%
\end{pgfscope}%
\begin{pgfscope}%
\definecolor{textcolor}{rgb}{0.000000,0.000000,0.000000}%
\pgfsetstrokecolor{textcolor}%
\pgfsetfillcolor{textcolor}%
\pgftext[x=0.228889in, y=1.493179in, left, base]{\color{textcolor}\rmfamily\fontsize{10.000000}{12.000000}\selectfont \(\displaystyle {20}\)}%
\end{pgfscope}%
\begin{pgfscope}%
\definecolor{textcolor}{rgb}{0.000000,0.000000,0.000000}%
\pgfsetstrokecolor{textcolor}%
\pgfsetfillcolor{textcolor}%
\pgftext[x=0.173333in,y=1.026944in,,bottom,rotate=90.000000]{\color{textcolor}\rmfamily\fontsize{10.000000}{12.000000}\selectfont Percent of Data Set}%
\end{pgfscope}%
\begin{pgfscope}%
\pgfsetrectcap%
\pgfsetmiterjoin%
\pgfsetlinewidth{0.803000pt}%
\definecolor{currentstroke}{rgb}{0.000000,0.000000,0.000000}%
\pgfsetstrokecolor{currentstroke}%
\pgfsetdash{}{0pt}%
\pgfpathmoveto{\pgfqpoint{0.465000in}{0.449444in}}%
\pgfpathlineto{\pgfqpoint{0.465000in}{1.604444in}}%
\pgfusepath{stroke}%
\end{pgfscope}%
\begin{pgfscope}%
\pgfsetrectcap%
\pgfsetmiterjoin%
\pgfsetlinewidth{0.803000pt}%
\definecolor{currentstroke}{rgb}{0.000000,0.000000,0.000000}%
\pgfsetstrokecolor{currentstroke}%
\pgfsetdash{}{0pt}%
\pgfpathmoveto{\pgfqpoint{2.015000in}{0.449444in}}%
\pgfpathlineto{\pgfqpoint{2.015000in}{1.604444in}}%
\pgfusepath{stroke}%
\end{pgfscope}%
\begin{pgfscope}%
\pgfsetrectcap%
\pgfsetmiterjoin%
\pgfsetlinewidth{0.803000pt}%
\definecolor{currentstroke}{rgb}{0.000000,0.000000,0.000000}%
\pgfsetstrokecolor{currentstroke}%
\pgfsetdash{}{0pt}%
\pgfpathmoveto{\pgfqpoint{0.465000in}{0.449444in}}%
\pgfpathlineto{\pgfqpoint{2.015000in}{0.449444in}}%
\pgfusepath{stroke}%
\end{pgfscope}%
\begin{pgfscope}%
\pgfsetrectcap%
\pgfsetmiterjoin%
\pgfsetlinewidth{0.803000pt}%
\definecolor{currentstroke}{rgb}{0.000000,0.000000,0.000000}%
\pgfsetstrokecolor{currentstroke}%
\pgfsetdash{}{0pt}%
\pgfpathmoveto{\pgfqpoint{0.465000in}{1.604444in}}%
\pgfpathlineto{\pgfqpoint{2.015000in}{1.604444in}}%
\pgfusepath{stroke}%
\end{pgfscope}%
\begin{pgfscope}%
\pgfsetbuttcap%
\pgfsetmiterjoin%
\definecolor{currentfill}{rgb}{1.000000,1.000000,1.000000}%
\pgfsetfillcolor{currentfill}%
\pgfsetfillopacity{0.800000}%
\pgfsetlinewidth{1.003750pt}%
\definecolor{currentstroke}{rgb}{0.800000,0.800000,0.800000}%
\pgfsetstrokecolor{currentstroke}%
\pgfsetstrokeopacity{0.800000}%
\pgfsetdash{}{0pt}%
\pgfpathmoveto{\pgfqpoint{1.238056in}{1.104445in}}%
\pgfpathlineto{\pgfqpoint{1.917778in}{1.104445in}}%
\pgfpathquadraticcurveto{\pgfqpoint{1.945556in}{1.104445in}}{\pgfqpoint{1.945556in}{1.132222in}}%
\pgfpathlineto{\pgfqpoint{1.945556in}{1.507222in}}%
\pgfpathquadraticcurveto{\pgfqpoint{1.945556in}{1.535000in}}{\pgfqpoint{1.917778in}{1.535000in}}%
\pgfpathlineto{\pgfqpoint{1.238056in}{1.535000in}}%
\pgfpathquadraticcurveto{\pgfqpoint{1.210278in}{1.535000in}}{\pgfqpoint{1.210278in}{1.507222in}}%
\pgfpathlineto{\pgfqpoint{1.210278in}{1.132222in}}%
\pgfpathquadraticcurveto{\pgfqpoint{1.210278in}{1.104445in}}{\pgfqpoint{1.238056in}{1.104445in}}%
\pgfpathlineto{\pgfqpoint{1.238056in}{1.104445in}}%
\pgfpathclose%
\pgfusepath{stroke,fill}%
\end{pgfscope}%
\begin{pgfscope}%
\pgfsetbuttcap%
\pgfsetmiterjoin%
\pgfsetlinewidth{1.003750pt}%
\definecolor{currentstroke}{rgb}{0.000000,0.000000,0.000000}%
\pgfsetstrokecolor{currentstroke}%
\pgfsetdash{}{0pt}%
\pgfpathmoveto{\pgfqpoint{1.265834in}{1.382222in}}%
\pgfpathlineto{\pgfqpoint{1.543611in}{1.382222in}}%
\pgfpathlineto{\pgfqpoint{1.543611in}{1.479444in}}%
\pgfpathlineto{\pgfqpoint{1.265834in}{1.479444in}}%
\pgfpathlineto{\pgfqpoint{1.265834in}{1.382222in}}%
\pgfpathclose%
\pgfusepath{stroke}%
\end{pgfscope}%
\begin{pgfscope}%
\definecolor{textcolor}{rgb}{0.000000,0.000000,0.000000}%
\pgfsetstrokecolor{textcolor}%
\pgfsetfillcolor{textcolor}%
\pgftext[x=1.654722in,y=1.382222in,left,base]{\color{textcolor}\rmfamily\fontsize{10.000000}{12.000000}\selectfont Neg}%
\end{pgfscope}%
\begin{pgfscope}%
\pgfsetbuttcap%
\pgfsetmiterjoin%
\definecolor{currentfill}{rgb}{0.000000,0.000000,0.000000}%
\pgfsetfillcolor{currentfill}%
\pgfsetlinewidth{0.000000pt}%
\definecolor{currentstroke}{rgb}{0.000000,0.000000,0.000000}%
\pgfsetstrokecolor{currentstroke}%
\pgfsetstrokeopacity{0.000000}%
\pgfsetdash{}{0pt}%
\pgfpathmoveto{\pgfqpoint{1.265834in}{1.186944in}}%
\pgfpathlineto{\pgfqpoint{1.543611in}{1.186944in}}%
\pgfpathlineto{\pgfqpoint{1.543611in}{1.284167in}}%
\pgfpathlineto{\pgfqpoint{1.265834in}{1.284167in}}%
\pgfpathlineto{\pgfqpoint{1.265834in}{1.186944in}}%
\pgfpathclose%
\pgfusepath{fill}%
\end{pgfscope}%
\begin{pgfscope}%
\definecolor{textcolor}{rgb}{0.000000,0.000000,0.000000}%
\pgfsetstrokecolor{textcolor}%
\pgfsetfillcolor{textcolor}%
\pgftext[x=1.654722in,y=1.186944in,left,base]{\color{textcolor}\rmfamily\fontsize{10.000000}{12.000000}\selectfont Pos}%
\end{pgfscope}%
\end{pgfpicture}%
\makeatother%
\endgroup%

&
	\vskip 0pt
	\qquad \qquad ROC Curve
	
	%% Creator: Matplotlib, PGF backend
%%
%% To include the figure in your LaTeX document, write
%%   \input{<filename>.pgf}
%%
%% Make sure the required packages are loaded in your preamble
%%   \usepackage{pgf}
%%
%% Also ensure that all the required font packages are loaded; for instance,
%% the lmodern package is sometimes necessary when using math font.
%%   \usepackage{lmodern}
%%
%% Figures using additional raster images can only be included by \input if
%% they are in the same directory as the main LaTeX file. For loading figures
%% from other directories you can use the `import` package
%%   \usepackage{import}
%%
%% and then include the figures with
%%   \import{<path to file>}{<filename>.pgf}
%%
%% Matplotlib used the following preamble
%%   
%%   \usepackage{fontspec}
%%   \makeatletter\@ifpackageloaded{underscore}{}{\usepackage[strings]{underscore}}\makeatother
%%
\begingroup%
\makeatletter%
\begin{pgfpicture}%
\pgfpathrectangle{\pgfpointorigin}{\pgfqpoint{2.221861in}{1.754444in}}%
\pgfusepath{use as bounding box, clip}%
\begin{pgfscope}%
\pgfsetbuttcap%
\pgfsetmiterjoin%
\definecolor{currentfill}{rgb}{1.000000,1.000000,1.000000}%
\pgfsetfillcolor{currentfill}%
\pgfsetlinewidth{0.000000pt}%
\definecolor{currentstroke}{rgb}{1.000000,1.000000,1.000000}%
\pgfsetstrokecolor{currentstroke}%
\pgfsetdash{}{0pt}%
\pgfpathmoveto{\pgfqpoint{0.000000in}{0.000000in}}%
\pgfpathlineto{\pgfqpoint{2.221861in}{0.000000in}}%
\pgfpathlineto{\pgfqpoint{2.221861in}{1.754444in}}%
\pgfpathlineto{\pgfqpoint{0.000000in}{1.754444in}}%
\pgfpathlineto{\pgfqpoint{0.000000in}{0.000000in}}%
\pgfpathclose%
\pgfusepath{fill}%
\end{pgfscope}%
\begin{pgfscope}%
\pgfsetbuttcap%
\pgfsetmiterjoin%
\definecolor{currentfill}{rgb}{1.000000,1.000000,1.000000}%
\pgfsetfillcolor{currentfill}%
\pgfsetlinewidth{0.000000pt}%
\definecolor{currentstroke}{rgb}{0.000000,0.000000,0.000000}%
\pgfsetstrokecolor{currentstroke}%
\pgfsetstrokeopacity{0.000000}%
\pgfsetdash{}{0pt}%
\pgfpathmoveto{\pgfqpoint{0.553581in}{0.499444in}}%
\pgfpathlineto{\pgfqpoint{2.103581in}{0.499444in}}%
\pgfpathlineto{\pgfqpoint{2.103581in}{1.654444in}}%
\pgfpathlineto{\pgfqpoint{0.553581in}{1.654444in}}%
\pgfpathlineto{\pgfqpoint{0.553581in}{0.499444in}}%
\pgfpathclose%
\pgfusepath{fill}%
\end{pgfscope}%
\begin{pgfscope}%
\pgfsetbuttcap%
\pgfsetroundjoin%
\definecolor{currentfill}{rgb}{0.000000,0.000000,0.000000}%
\pgfsetfillcolor{currentfill}%
\pgfsetlinewidth{0.803000pt}%
\definecolor{currentstroke}{rgb}{0.000000,0.000000,0.000000}%
\pgfsetstrokecolor{currentstroke}%
\pgfsetdash{}{0pt}%
\pgfsys@defobject{currentmarker}{\pgfqpoint{0.000000in}{-0.048611in}}{\pgfqpoint{0.000000in}{0.000000in}}{%
\pgfpathmoveto{\pgfqpoint{0.000000in}{0.000000in}}%
\pgfpathlineto{\pgfqpoint{0.000000in}{-0.048611in}}%
\pgfusepath{stroke,fill}%
}%
\begin{pgfscope}%
\pgfsys@transformshift{0.624035in}{0.499444in}%
\pgfsys@useobject{currentmarker}{}%
\end{pgfscope}%
\end{pgfscope}%
\begin{pgfscope}%
\definecolor{textcolor}{rgb}{0.000000,0.000000,0.000000}%
\pgfsetstrokecolor{textcolor}%
\pgfsetfillcolor{textcolor}%
\pgftext[x=0.624035in,y=0.402222in,,top]{\color{textcolor}\rmfamily\fontsize{10.000000}{12.000000}\selectfont \(\displaystyle {0.0}\)}%
\end{pgfscope}%
\begin{pgfscope}%
\pgfsetbuttcap%
\pgfsetroundjoin%
\definecolor{currentfill}{rgb}{0.000000,0.000000,0.000000}%
\pgfsetfillcolor{currentfill}%
\pgfsetlinewidth{0.803000pt}%
\definecolor{currentstroke}{rgb}{0.000000,0.000000,0.000000}%
\pgfsetstrokecolor{currentstroke}%
\pgfsetdash{}{0pt}%
\pgfsys@defobject{currentmarker}{\pgfqpoint{0.000000in}{-0.048611in}}{\pgfqpoint{0.000000in}{0.000000in}}{%
\pgfpathmoveto{\pgfqpoint{0.000000in}{0.000000in}}%
\pgfpathlineto{\pgfqpoint{0.000000in}{-0.048611in}}%
\pgfusepath{stroke,fill}%
}%
\begin{pgfscope}%
\pgfsys@transformshift{1.328581in}{0.499444in}%
\pgfsys@useobject{currentmarker}{}%
\end{pgfscope}%
\end{pgfscope}%
\begin{pgfscope}%
\definecolor{textcolor}{rgb}{0.000000,0.000000,0.000000}%
\pgfsetstrokecolor{textcolor}%
\pgfsetfillcolor{textcolor}%
\pgftext[x=1.328581in,y=0.402222in,,top]{\color{textcolor}\rmfamily\fontsize{10.000000}{12.000000}\selectfont \(\displaystyle {0.5}\)}%
\end{pgfscope}%
\begin{pgfscope}%
\pgfsetbuttcap%
\pgfsetroundjoin%
\definecolor{currentfill}{rgb}{0.000000,0.000000,0.000000}%
\pgfsetfillcolor{currentfill}%
\pgfsetlinewidth{0.803000pt}%
\definecolor{currentstroke}{rgb}{0.000000,0.000000,0.000000}%
\pgfsetstrokecolor{currentstroke}%
\pgfsetdash{}{0pt}%
\pgfsys@defobject{currentmarker}{\pgfqpoint{0.000000in}{-0.048611in}}{\pgfqpoint{0.000000in}{0.000000in}}{%
\pgfpathmoveto{\pgfqpoint{0.000000in}{0.000000in}}%
\pgfpathlineto{\pgfqpoint{0.000000in}{-0.048611in}}%
\pgfusepath{stroke,fill}%
}%
\begin{pgfscope}%
\pgfsys@transformshift{2.033126in}{0.499444in}%
\pgfsys@useobject{currentmarker}{}%
\end{pgfscope}%
\end{pgfscope}%
\begin{pgfscope}%
\definecolor{textcolor}{rgb}{0.000000,0.000000,0.000000}%
\pgfsetstrokecolor{textcolor}%
\pgfsetfillcolor{textcolor}%
\pgftext[x=2.033126in,y=0.402222in,,top]{\color{textcolor}\rmfamily\fontsize{10.000000}{12.000000}\selectfont \(\displaystyle {1.0}\)}%
\end{pgfscope}%
\begin{pgfscope}%
\definecolor{textcolor}{rgb}{0.000000,0.000000,0.000000}%
\pgfsetstrokecolor{textcolor}%
\pgfsetfillcolor{textcolor}%
\pgftext[x=1.328581in,y=0.223333in,,top]{\color{textcolor}\rmfamily\fontsize{10.000000}{12.000000}\selectfont False positive rate}%
\end{pgfscope}%
\begin{pgfscope}%
\pgfsetbuttcap%
\pgfsetroundjoin%
\definecolor{currentfill}{rgb}{0.000000,0.000000,0.000000}%
\pgfsetfillcolor{currentfill}%
\pgfsetlinewidth{0.803000pt}%
\definecolor{currentstroke}{rgb}{0.000000,0.000000,0.000000}%
\pgfsetstrokecolor{currentstroke}%
\pgfsetdash{}{0pt}%
\pgfsys@defobject{currentmarker}{\pgfqpoint{-0.048611in}{0.000000in}}{\pgfqpoint{-0.000000in}{0.000000in}}{%
\pgfpathmoveto{\pgfqpoint{-0.000000in}{0.000000in}}%
\pgfpathlineto{\pgfqpoint{-0.048611in}{0.000000in}}%
\pgfusepath{stroke,fill}%
}%
\begin{pgfscope}%
\pgfsys@transformshift{0.553581in}{0.551944in}%
\pgfsys@useobject{currentmarker}{}%
\end{pgfscope}%
\end{pgfscope}%
\begin{pgfscope}%
\definecolor{textcolor}{rgb}{0.000000,0.000000,0.000000}%
\pgfsetstrokecolor{textcolor}%
\pgfsetfillcolor{textcolor}%
\pgftext[x=0.278889in, y=0.503750in, left, base]{\color{textcolor}\rmfamily\fontsize{10.000000}{12.000000}\selectfont \(\displaystyle {0.0}\)}%
\end{pgfscope}%
\begin{pgfscope}%
\pgfsetbuttcap%
\pgfsetroundjoin%
\definecolor{currentfill}{rgb}{0.000000,0.000000,0.000000}%
\pgfsetfillcolor{currentfill}%
\pgfsetlinewidth{0.803000pt}%
\definecolor{currentstroke}{rgb}{0.000000,0.000000,0.000000}%
\pgfsetstrokecolor{currentstroke}%
\pgfsetdash{}{0pt}%
\pgfsys@defobject{currentmarker}{\pgfqpoint{-0.048611in}{0.000000in}}{\pgfqpoint{-0.000000in}{0.000000in}}{%
\pgfpathmoveto{\pgfqpoint{-0.000000in}{0.000000in}}%
\pgfpathlineto{\pgfqpoint{-0.048611in}{0.000000in}}%
\pgfusepath{stroke,fill}%
}%
\begin{pgfscope}%
\pgfsys@transformshift{0.553581in}{1.076944in}%
\pgfsys@useobject{currentmarker}{}%
\end{pgfscope}%
\end{pgfscope}%
\begin{pgfscope}%
\definecolor{textcolor}{rgb}{0.000000,0.000000,0.000000}%
\pgfsetstrokecolor{textcolor}%
\pgfsetfillcolor{textcolor}%
\pgftext[x=0.278889in, y=1.028750in, left, base]{\color{textcolor}\rmfamily\fontsize{10.000000}{12.000000}\selectfont \(\displaystyle {0.5}\)}%
\end{pgfscope}%
\begin{pgfscope}%
\pgfsetbuttcap%
\pgfsetroundjoin%
\definecolor{currentfill}{rgb}{0.000000,0.000000,0.000000}%
\pgfsetfillcolor{currentfill}%
\pgfsetlinewidth{0.803000pt}%
\definecolor{currentstroke}{rgb}{0.000000,0.000000,0.000000}%
\pgfsetstrokecolor{currentstroke}%
\pgfsetdash{}{0pt}%
\pgfsys@defobject{currentmarker}{\pgfqpoint{-0.048611in}{0.000000in}}{\pgfqpoint{-0.000000in}{0.000000in}}{%
\pgfpathmoveto{\pgfqpoint{-0.000000in}{0.000000in}}%
\pgfpathlineto{\pgfqpoint{-0.048611in}{0.000000in}}%
\pgfusepath{stroke,fill}%
}%
\begin{pgfscope}%
\pgfsys@transformshift{0.553581in}{1.601944in}%
\pgfsys@useobject{currentmarker}{}%
\end{pgfscope}%
\end{pgfscope}%
\begin{pgfscope}%
\definecolor{textcolor}{rgb}{0.000000,0.000000,0.000000}%
\pgfsetstrokecolor{textcolor}%
\pgfsetfillcolor{textcolor}%
\pgftext[x=0.278889in, y=1.553750in, left, base]{\color{textcolor}\rmfamily\fontsize{10.000000}{12.000000}\selectfont \(\displaystyle {1.0}\)}%
\end{pgfscope}%
\begin{pgfscope}%
\definecolor{textcolor}{rgb}{0.000000,0.000000,0.000000}%
\pgfsetstrokecolor{textcolor}%
\pgfsetfillcolor{textcolor}%
\pgftext[x=0.223333in,y=1.076944in,,bottom,rotate=90.000000]{\color{textcolor}\rmfamily\fontsize{10.000000}{12.000000}\selectfont True positive rate}%
\end{pgfscope}%
\begin{pgfscope}%
\pgfpathrectangle{\pgfqpoint{0.553581in}{0.499444in}}{\pgfqpoint{1.550000in}{1.155000in}}%
\pgfusepath{clip}%
\pgfsetbuttcap%
\pgfsetroundjoin%
\pgfsetlinewidth{1.505625pt}%
\definecolor{currentstroke}{rgb}{0.000000,0.000000,0.000000}%
\pgfsetstrokecolor{currentstroke}%
\pgfsetdash{{5.550000pt}{2.400000pt}}{0.000000pt}%
\pgfpathmoveto{\pgfqpoint{0.624035in}{0.551944in}}%
\pgfpathlineto{\pgfqpoint{2.033126in}{1.601944in}}%
\pgfusepath{stroke}%
\end{pgfscope}%
\begin{pgfscope}%
\pgfpathrectangle{\pgfqpoint{0.553581in}{0.499444in}}{\pgfqpoint{1.550000in}{1.155000in}}%
\pgfusepath{clip}%
\pgfsetrectcap%
\pgfsetroundjoin%
\pgfsetlinewidth{1.505625pt}%
\definecolor{currentstroke}{rgb}{0.000000,0.000000,0.000000}%
\pgfsetstrokecolor{currentstroke}%
\pgfsetdash{}{0pt}%
\pgfpathmoveto{\pgfqpoint{0.624035in}{0.551944in}}%
\pgfpathlineto{\pgfqpoint{0.626207in}{0.552574in}}%
\pgfpathlineto{\pgfqpoint{0.627318in}{0.561464in}}%
\pgfpathlineto{\pgfqpoint{0.628014in}{0.562514in}}%
\pgfpathlineto{\pgfqpoint{0.629125in}{0.563634in}}%
\pgfpathlineto{\pgfqpoint{0.629605in}{0.564614in}}%
\pgfpathlineto{\pgfqpoint{0.630699in}{0.567694in}}%
\pgfpathlineto{\pgfqpoint{0.631130in}{0.568604in}}%
\pgfpathlineto{\pgfqpoint{0.632225in}{0.573294in}}%
\pgfpathlineto{\pgfqpoint{0.632772in}{0.574064in}}%
\pgfpathlineto{\pgfqpoint{0.633866in}{0.579944in}}%
\pgfpathlineto{\pgfqpoint{0.634247in}{0.580854in}}%
\pgfpathlineto{\pgfqpoint{0.635358in}{0.585824in}}%
\pgfpathlineto{\pgfqpoint{0.635540in}{0.586874in}}%
\pgfpathlineto{\pgfqpoint{0.636634in}{0.593244in}}%
\pgfpathlineto{\pgfqpoint{0.636833in}{0.594154in}}%
\pgfpathlineto{\pgfqpoint{0.637944in}{0.600734in}}%
\pgfpathlineto{\pgfqpoint{0.638126in}{0.601784in}}%
\pgfpathlineto{\pgfqpoint{0.639187in}{0.608714in}}%
\pgfpathlineto{\pgfqpoint{0.639353in}{0.609694in}}%
\pgfpathlineto{\pgfqpoint{0.640464in}{0.616834in}}%
\pgfpathlineto{\pgfqpoint{0.640712in}{0.617884in}}%
\pgfpathlineto{\pgfqpoint{0.641806in}{0.625864in}}%
\pgfpathlineto{\pgfqpoint{0.642038in}{0.626844in}}%
\pgfpathlineto{\pgfqpoint{0.643149in}{0.635804in}}%
\pgfpathlineto{\pgfqpoint{0.643348in}{0.636854in}}%
\pgfpathlineto{\pgfqpoint{0.644392in}{0.645044in}}%
\pgfpathlineto{\pgfqpoint{0.644641in}{0.645884in}}%
\pgfpathlineto{\pgfqpoint{0.645752in}{0.652394in}}%
\pgfpathlineto{\pgfqpoint{0.645868in}{0.653164in}}%
\pgfpathlineto{\pgfqpoint{0.646979in}{0.661424in}}%
\pgfpathlineto{\pgfqpoint{0.647327in}{0.662264in}}%
\pgfpathlineto{\pgfqpoint{0.648437in}{0.668914in}}%
\pgfpathlineto{\pgfqpoint{0.648587in}{0.669824in}}%
\pgfpathlineto{\pgfqpoint{0.649697in}{0.678294in}}%
\pgfpathlineto{\pgfqpoint{0.649813in}{0.679274in}}%
\pgfpathlineto{\pgfqpoint{0.650924in}{0.687044in}}%
\pgfpathlineto{\pgfqpoint{0.651140in}{0.687884in}}%
\pgfpathlineto{\pgfqpoint{0.652250in}{0.696144in}}%
\pgfpathlineto{\pgfqpoint{0.652333in}{0.697194in}}%
\pgfpathlineto{\pgfqpoint{0.653444in}{0.706434in}}%
\pgfpathlineto{\pgfqpoint{0.653858in}{0.707414in}}%
\pgfpathlineto{\pgfqpoint{0.654969in}{0.716584in}}%
\pgfpathlineto{\pgfqpoint{0.655151in}{0.717564in}}%
\pgfpathlineto{\pgfqpoint{0.656262in}{0.725194in}}%
\pgfpathlineto{\pgfqpoint{0.656444in}{0.726174in}}%
\pgfpathlineto{\pgfqpoint{0.657538in}{0.733244in}}%
\pgfpathlineto{\pgfqpoint{0.657704in}{0.734294in}}%
\pgfpathlineto{\pgfqpoint{0.658815in}{0.743464in}}%
\pgfpathlineto{\pgfqpoint{0.659064in}{0.744514in}}%
\pgfpathlineto{\pgfqpoint{0.660158in}{0.751164in}}%
\pgfpathlineto{\pgfqpoint{0.660290in}{0.751794in}}%
\pgfpathlineto{\pgfqpoint{0.661401in}{0.759354in}}%
\pgfpathlineto{\pgfqpoint{0.661633in}{0.760404in}}%
\pgfpathlineto{\pgfqpoint{0.662744in}{0.769784in}}%
\pgfpathlineto{\pgfqpoint{0.662827in}{0.770344in}}%
\pgfpathlineto{\pgfqpoint{0.663937in}{0.777764in}}%
\pgfpathlineto{\pgfqpoint{0.664120in}{0.778814in}}%
\pgfpathlineto{\pgfqpoint{0.665181in}{0.783854in}}%
\pgfpathlineto{\pgfqpoint{0.665297in}{0.784834in}}%
\pgfpathlineto{\pgfqpoint{0.666407in}{0.795614in}}%
\pgfpathlineto{\pgfqpoint{0.666656in}{0.796664in}}%
\pgfpathlineto{\pgfqpoint{0.667767in}{0.801564in}}%
\pgfpathlineto{\pgfqpoint{0.667899in}{0.802404in}}%
\pgfpathlineto{\pgfqpoint{0.669010in}{0.809334in}}%
\pgfpathlineto{\pgfqpoint{0.669342in}{0.810384in}}%
\pgfpathlineto{\pgfqpoint{0.670452in}{0.818084in}}%
\pgfpathlineto{\pgfqpoint{0.670701in}{0.819134in}}%
\pgfpathlineto{\pgfqpoint{0.671812in}{0.825364in}}%
\pgfpathlineto{\pgfqpoint{0.671911in}{0.826414in}}%
\pgfpathlineto{\pgfqpoint{0.673022in}{0.833694in}}%
\pgfpathlineto{\pgfqpoint{0.673138in}{0.834744in}}%
\pgfpathlineto{\pgfqpoint{0.674215in}{0.838944in}}%
\pgfpathlineto{\pgfqpoint{0.674530in}{0.839994in}}%
\pgfpathlineto{\pgfqpoint{0.675608in}{0.846014in}}%
\pgfpathlineto{\pgfqpoint{0.675790in}{0.846994in}}%
\pgfpathlineto{\pgfqpoint{0.676901in}{0.853854in}}%
\pgfpathlineto{\pgfqpoint{0.677199in}{0.854834in}}%
\pgfpathlineto{\pgfqpoint{0.678310in}{0.859034in}}%
\pgfpathlineto{\pgfqpoint{0.678492in}{0.860084in}}%
\pgfpathlineto{\pgfqpoint{0.679587in}{0.867014in}}%
\pgfpathlineto{\pgfqpoint{0.679868in}{0.867924in}}%
\pgfpathlineto{\pgfqpoint{0.680963in}{0.874504in}}%
\pgfpathlineto{\pgfqpoint{0.681228in}{0.875484in}}%
\pgfpathlineto{\pgfqpoint{0.682338in}{0.882204in}}%
\pgfpathlineto{\pgfqpoint{0.682587in}{0.883254in}}%
\pgfpathlineto{\pgfqpoint{0.683681in}{0.889484in}}%
\pgfpathlineto{\pgfqpoint{0.683980in}{0.890394in}}%
\pgfpathlineto{\pgfqpoint{0.685074in}{0.897394in}}%
\pgfpathlineto{\pgfqpoint{0.685389in}{0.898304in}}%
\pgfpathlineto{\pgfqpoint{0.686450in}{0.904184in}}%
\pgfpathlineto{\pgfqpoint{0.686748in}{0.905094in}}%
\pgfpathlineto{\pgfqpoint{0.687859in}{0.910204in}}%
\pgfpathlineto{\pgfqpoint{0.688074in}{0.911114in}}%
\pgfpathlineto{\pgfqpoint{0.689185in}{0.915034in}}%
\pgfpathlineto{\pgfqpoint{0.689400in}{0.916084in}}%
\pgfpathlineto{\pgfqpoint{0.690511in}{0.922664in}}%
\pgfpathlineto{\pgfqpoint{0.690793in}{0.923714in}}%
\pgfpathlineto{\pgfqpoint{0.691887in}{0.928754in}}%
\pgfpathlineto{\pgfqpoint{0.692318in}{0.929734in}}%
\pgfpathlineto{\pgfqpoint{0.693412in}{0.935334in}}%
\pgfpathlineto{\pgfqpoint{0.693611in}{0.936384in}}%
\pgfpathlineto{\pgfqpoint{0.694705in}{0.940864in}}%
\pgfpathlineto{\pgfqpoint{0.695053in}{0.941914in}}%
\pgfpathlineto{\pgfqpoint{0.696164in}{0.948634in}}%
\pgfpathlineto{\pgfqpoint{0.696645in}{0.949684in}}%
\pgfpathlineto{\pgfqpoint{0.697756in}{0.954164in}}%
\pgfpathlineto{\pgfqpoint{0.697921in}{0.955144in}}%
\pgfpathlineto{\pgfqpoint{0.699015in}{0.960464in}}%
\pgfpathlineto{\pgfqpoint{0.699314in}{0.961304in}}%
\pgfpathlineto{\pgfqpoint{0.700425in}{0.965294in}}%
\pgfpathlineto{\pgfqpoint{0.700657in}{0.966344in}}%
\pgfpathlineto{\pgfqpoint{0.701767in}{0.971314in}}%
\pgfpathlineto{\pgfqpoint{0.702049in}{0.972364in}}%
\pgfpathlineto{\pgfqpoint{0.703094in}{0.977194in}}%
\pgfpathlineto{\pgfqpoint{0.703442in}{0.978104in}}%
\pgfpathlineto{\pgfqpoint{0.704536in}{0.982514in}}%
\pgfpathlineto{\pgfqpoint{0.704917in}{0.983564in}}%
\pgfpathlineto{\pgfqpoint{0.706011in}{0.988464in}}%
\pgfpathlineto{\pgfqpoint{0.706160in}{0.989444in}}%
\pgfpathlineto{\pgfqpoint{0.707271in}{0.993504in}}%
\pgfpathlineto{\pgfqpoint{0.707619in}{0.994414in}}%
\pgfpathlineto{\pgfqpoint{0.708697in}{0.998054in}}%
\pgfpathlineto{\pgfqpoint{0.709061in}{0.999104in}}%
\pgfpathlineto{\pgfqpoint{0.710172in}{1.004564in}}%
\pgfpathlineto{\pgfqpoint{0.710504in}{1.005614in}}%
\pgfpathlineto{\pgfqpoint{0.711614in}{1.009604in}}%
\pgfpathlineto{\pgfqpoint{0.711830in}{1.010584in}}%
\pgfpathlineto{\pgfqpoint{0.712874in}{1.015134in}}%
\pgfpathlineto{\pgfqpoint{0.713123in}{1.016184in}}%
\pgfpathlineto{\pgfqpoint{0.714184in}{1.020244in}}%
\pgfpathlineto{\pgfqpoint{0.714598in}{1.021294in}}%
\pgfpathlineto{\pgfqpoint{0.715709in}{1.024304in}}%
\pgfpathlineto{\pgfqpoint{0.716173in}{1.025354in}}%
\pgfpathlineto{\pgfqpoint{0.717284in}{1.029624in}}%
\pgfpathlineto{\pgfqpoint{0.717466in}{1.030604in}}%
\pgfpathlineto{\pgfqpoint{0.718560in}{1.035014in}}%
\pgfpathlineto{\pgfqpoint{0.718776in}{1.035994in}}%
\pgfpathlineto{\pgfqpoint{0.719853in}{1.041034in}}%
\pgfpathlineto{\pgfqpoint{0.720202in}{1.042084in}}%
\pgfpathlineto{\pgfqpoint{0.721312in}{1.046144in}}%
\pgfpathlineto{\pgfqpoint{0.721694in}{1.047194in}}%
\pgfpathlineto{\pgfqpoint{0.722788in}{1.050904in}}%
\pgfpathlineto{\pgfqpoint{0.723086in}{1.051954in}}%
\pgfpathlineto{\pgfqpoint{0.724180in}{1.056364in}}%
\pgfpathlineto{\pgfqpoint{0.724727in}{1.057414in}}%
\pgfpathlineto{\pgfqpoint{0.725805in}{1.061264in}}%
\pgfpathlineto{\pgfqpoint{0.726236in}{1.062314in}}%
\pgfpathlineto{\pgfqpoint{0.727280in}{1.065744in}}%
\pgfpathlineto{\pgfqpoint{0.727728in}{1.066794in}}%
\pgfpathlineto{\pgfqpoint{0.728805in}{1.069874in}}%
\pgfpathlineto{\pgfqpoint{0.729253in}{1.070924in}}%
\pgfpathlineto{\pgfqpoint{0.730364in}{1.074144in}}%
\pgfpathlineto{\pgfqpoint{0.730695in}{1.075124in}}%
\pgfpathlineto{\pgfqpoint{0.731789in}{1.079044in}}%
\pgfpathlineto{\pgfqpoint{0.732336in}{1.080024in}}%
\pgfpathlineto{\pgfqpoint{0.733447in}{1.084574in}}%
\pgfpathlineto{\pgfqpoint{0.734094in}{1.085624in}}%
\pgfpathlineto{\pgfqpoint{0.735188in}{1.089544in}}%
\pgfpathlineto{\pgfqpoint{0.735552in}{1.090594in}}%
\pgfpathlineto{\pgfqpoint{0.736597in}{1.093674in}}%
\pgfpathlineto{\pgfqpoint{0.737177in}{1.094724in}}%
\pgfpathlineto{\pgfqpoint{0.738221in}{1.097524in}}%
\pgfpathlineto{\pgfqpoint{0.738702in}{1.098574in}}%
\pgfpathlineto{\pgfqpoint{0.739780in}{1.102424in}}%
\pgfpathlineto{\pgfqpoint{0.740128in}{1.103474in}}%
\pgfpathlineto{\pgfqpoint{0.741238in}{1.106624in}}%
\pgfpathlineto{\pgfqpoint{0.741636in}{1.107674in}}%
\pgfpathlineto{\pgfqpoint{0.742697in}{1.111034in}}%
\pgfpathlineto{\pgfqpoint{0.743095in}{1.112084in}}%
\pgfpathlineto{\pgfqpoint{0.744206in}{1.115024in}}%
\pgfpathlineto{\pgfqpoint{0.744769in}{1.116074in}}%
\pgfpathlineto{\pgfqpoint{0.745051in}{1.117054in}}%
\pgfpathlineto{\pgfqpoint{0.745068in}{1.117054in}}%
\pgfpathlineto{\pgfqpoint{0.755943in}{1.118104in}}%
\pgfpathlineto{\pgfqpoint{0.757053in}{1.121814in}}%
\pgfpathlineto{\pgfqpoint{0.757600in}{1.122864in}}%
\pgfpathlineto{\pgfqpoint{0.758678in}{1.125384in}}%
\pgfpathlineto{\pgfqpoint{0.759225in}{1.126434in}}%
\pgfpathlineto{\pgfqpoint{0.760269in}{1.129094in}}%
\pgfpathlineto{\pgfqpoint{0.760817in}{1.130144in}}%
\pgfpathlineto{\pgfqpoint{0.761911in}{1.133224in}}%
\pgfpathlineto{\pgfqpoint{0.762474in}{1.134274in}}%
\pgfpathlineto{\pgfqpoint{0.763535in}{1.137844in}}%
\pgfpathlineto{\pgfqpoint{0.764165in}{1.138824in}}%
\pgfpathlineto{\pgfqpoint{0.765259in}{1.142324in}}%
\pgfpathlineto{\pgfqpoint{0.766055in}{1.143374in}}%
\pgfpathlineto{\pgfqpoint{0.767166in}{1.146314in}}%
\pgfpathlineto{\pgfqpoint{0.767696in}{1.147364in}}%
\pgfpathlineto{\pgfqpoint{0.768790in}{1.149954in}}%
\pgfpathlineto{\pgfqpoint{0.769603in}{1.151004in}}%
\pgfpathlineto{\pgfqpoint{0.770514in}{1.153314in}}%
\pgfpathlineto{\pgfqpoint{0.771161in}{1.154364in}}%
\pgfpathlineto{\pgfqpoint{0.772272in}{1.157374in}}%
\pgfpathlineto{\pgfqpoint{0.772852in}{1.158424in}}%
\pgfpathlineto{\pgfqpoint{0.773946in}{1.160664in}}%
\pgfpathlineto{\pgfqpoint{0.774609in}{1.161714in}}%
\pgfpathlineto{\pgfqpoint{0.775703in}{1.164794in}}%
\pgfpathlineto{\pgfqpoint{0.776267in}{1.165774in}}%
\pgfpathlineto{\pgfqpoint{0.777344in}{1.168714in}}%
\pgfpathlineto{\pgfqpoint{0.777759in}{1.169764in}}%
\pgfpathlineto{\pgfqpoint{0.778836in}{1.172144in}}%
\pgfpathlineto{\pgfqpoint{0.779350in}{1.173124in}}%
\pgfpathlineto{\pgfqpoint{0.780444in}{1.174524in}}%
\pgfpathlineto{\pgfqpoint{0.781008in}{1.175574in}}%
\pgfpathlineto{\pgfqpoint{0.782052in}{1.177254in}}%
\pgfpathlineto{\pgfqpoint{0.782699in}{1.178304in}}%
\pgfpathlineto{\pgfqpoint{0.783793in}{1.180894in}}%
\pgfpathlineto{\pgfqpoint{0.784290in}{1.181874in}}%
\pgfpathlineto{\pgfqpoint{0.785384in}{1.184044in}}%
\pgfpathlineto{\pgfqpoint{0.785981in}{1.185094in}}%
\pgfpathlineto{\pgfqpoint{0.787075in}{1.186914in}}%
\pgfpathlineto{\pgfqpoint{0.787755in}{1.187894in}}%
\pgfpathlineto{\pgfqpoint{0.788866in}{1.189784in}}%
\pgfpathlineto{\pgfqpoint{0.789297in}{1.190764in}}%
\pgfpathlineto{\pgfqpoint{0.790407in}{1.192794in}}%
\pgfpathlineto{\pgfqpoint{0.790689in}{1.193634in}}%
\pgfpathlineto{\pgfqpoint{0.791783in}{1.196084in}}%
\pgfpathlineto{\pgfqpoint{0.792347in}{1.197134in}}%
\pgfpathlineto{\pgfqpoint{0.793358in}{1.199724in}}%
\pgfpathlineto{\pgfqpoint{0.794054in}{1.200774in}}%
\pgfpathlineto{\pgfqpoint{0.795099in}{1.201964in}}%
\pgfpathlineto{\pgfqpoint{0.796044in}{1.203014in}}%
\pgfpathlineto{\pgfqpoint{0.797138in}{1.205534in}}%
\pgfpathlineto{\pgfqpoint{0.797718in}{1.206584in}}%
\pgfpathlineto{\pgfqpoint{0.798812in}{1.208474in}}%
\pgfpathlineto{\pgfqpoint{0.799409in}{1.209524in}}%
\pgfpathlineto{\pgfqpoint{0.800520in}{1.212114in}}%
\pgfpathlineto{\pgfqpoint{0.801382in}{1.213094in}}%
\pgfpathlineto{\pgfqpoint{0.802492in}{1.215894in}}%
\pgfpathlineto{\pgfqpoint{0.803106in}{1.216664in}}%
\pgfpathlineto{\pgfqpoint{0.804134in}{1.218554in}}%
\pgfpathlineto{\pgfqpoint{0.805128in}{1.219604in}}%
\pgfpathlineto{\pgfqpoint{0.806222in}{1.221774in}}%
\pgfpathlineto{\pgfqpoint{0.806786in}{1.222614in}}%
\pgfpathlineto{\pgfqpoint{0.807897in}{1.225484in}}%
\pgfpathlineto{\pgfqpoint{0.808659in}{1.226534in}}%
\pgfpathlineto{\pgfqpoint{0.809753in}{1.228284in}}%
\pgfpathlineto{\pgfqpoint{0.810267in}{1.229334in}}%
\pgfpathlineto{\pgfqpoint{0.811378in}{1.231224in}}%
\pgfpathlineto{\pgfqpoint{0.812654in}{1.232274in}}%
\pgfpathlineto{\pgfqpoint{0.813682in}{1.233744in}}%
\pgfpathlineto{\pgfqpoint{0.814180in}{1.234794in}}%
\pgfpathlineto{\pgfqpoint{0.815290in}{1.236614in}}%
\pgfpathlineto{\pgfqpoint{0.816053in}{1.237664in}}%
\pgfpathlineto{\pgfqpoint{0.816948in}{1.239484in}}%
\pgfpathlineto{\pgfqpoint{0.818109in}{1.240464in}}%
\pgfpathlineto{\pgfqpoint{0.819203in}{1.242424in}}%
\pgfpathlineto{\pgfqpoint{0.819949in}{1.243404in}}%
\pgfpathlineto{\pgfqpoint{0.821043in}{1.245364in}}%
\pgfpathlineto{\pgfqpoint{0.821838in}{1.246414in}}%
\pgfpathlineto{\pgfqpoint{0.822899in}{1.248374in}}%
\pgfpathlineto{\pgfqpoint{0.823828in}{1.249424in}}%
\pgfpathlineto{\pgfqpoint{0.824905in}{1.251314in}}%
\pgfpathlineto{\pgfqpoint{0.826049in}{1.252364in}}%
\pgfpathlineto{\pgfqpoint{0.827110in}{1.254044in}}%
\pgfpathlineto{\pgfqpoint{0.828204in}{1.255024in}}%
\pgfpathlineto{\pgfqpoint{0.829298in}{1.257404in}}%
\pgfpathlineto{\pgfqpoint{0.830094in}{1.258454in}}%
\pgfpathlineto{\pgfqpoint{0.831188in}{1.259784in}}%
\pgfpathlineto{\pgfqpoint{0.831901in}{1.260834in}}%
\pgfpathlineto{\pgfqpoint{0.832979in}{1.261954in}}%
\pgfpathlineto{\pgfqpoint{0.833841in}{1.263004in}}%
\pgfpathlineto{\pgfqpoint{0.834769in}{1.263774in}}%
\pgfpathlineto{\pgfqpoint{0.835697in}{1.264824in}}%
\pgfpathlineto{\pgfqpoint{0.836659in}{1.266014in}}%
\pgfpathlineto{\pgfqpoint{0.838234in}{1.267064in}}%
\pgfpathlineto{\pgfqpoint{0.839295in}{1.268814in}}%
\pgfpathlineto{\pgfqpoint{0.840438in}{1.269864in}}%
\pgfpathlineto{\pgfqpoint{0.841549in}{1.271824in}}%
\pgfpathlineto{\pgfqpoint{0.842594in}{1.272874in}}%
\pgfpathlineto{\pgfqpoint{0.843688in}{1.274414in}}%
\pgfpathlineto{\pgfqpoint{0.844881in}{1.275464in}}%
\pgfpathlineto{\pgfqpoint{0.845992in}{1.277844in}}%
\pgfpathlineto{\pgfqpoint{0.846705in}{1.278894in}}%
\pgfpathlineto{\pgfqpoint{0.847633in}{1.280294in}}%
\pgfpathlineto{\pgfqpoint{0.848644in}{1.281274in}}%
\pgfpathlineto{\pgfqpoint{0.849606in}{1.282954in}}%
\pgfpathlineto{\pgfqpoint{0.850882in}{1.284004in}}%
\pgfpathlineto{\pgfqpoint{0.851960in}{1.285194in}}%
\pgfpathlineto{\pgfqpoint{0.853120in}{1.286244in}}%
\pgfpathlineto{\pgfqpoint{0.854198in}{1.287644in}}%
\pgfpathlineto{\pgfqpoint{0.855176in}{1.288694in}}%
\pgfpathlineto{\pgfqpoint{0.856237in}{1.290234in}}%
\pgfpathlineto{\pgfqpoint{0.856917in}{1.291144in}}%
\pgfpathlineto{\pgfqpoint{0.858027in}{1.292754in}}%
\pgfpathlineto{\pgfqpoint{0.859287in}{1.293804in}}%
\pgfpathlineto{\pgfqpoint{0.860232in}{1.294714in}}%
\pgfpathlineto{\pgfqpoint{0.861160in}{1.295764in}}%
\pgfpathlineto{\pgfqpoint{0.862155in}{1.296814in}}%
\pgfpathlineto{\pgfqpoint{0.863746in}{1.297794in}}%
\pgfpathlineto{\pgfqpoint{0.864807in}{1.299054in}}%
\pgfpathlineto{\pgfqpoint{0.866084in}{1.300104in}}%
\pgfpathlineto{\pgfqpoint{0.867128in}{1.301854in}}%
\pgfpathlineto{\pgfqpoint{0.867775in}{1.302904in}}%
\pgfpathlineto{\pgfqpoint{0.868886in}{1.304164in}}%
\pgfpathlineto{\pgfqpoint{0.869897in}{1.305144in}}%
\pgfpathlineto{\pgfqpoint{0.870991in}{1.306614in}}%
\pgfpathlineto{\pgfqpoint{0.872218in}{1.307664in}}%
\pgfpathlineto{\pgfqpoint{0.873312in}{1.309624in}}%
\pgfpathlineto{\pgfqpoint{0.874572in}{1.310674in}}%
\pgfpathlineto{\pgfqpoint{0.875616in}{1.312284in}}%
\pgfpathlineto{\pgfqpoint{0.876263in}{1.313334in}}%
\pgfpathlineto{\pgfqpoint{0.877373in}{1.314804in}}%
\pgfpathlineto{\pgfqpoint{0.878036in}{1.315854in}}%
\pgfpathlineto{\pgfqpoint{0.879081in}{1.316974in}}%
\pgfpathlineto{\pgfqpoint{0.880191in}{1.318024in}}%
\pgfpathlineto{\pgfqpoint{0.881252in}{1.319564in}}%
\pgfpathlineto{\pgfqpoint{0.882927in}{1.320544in}}%
\pgfpathlineto{\pgfqpoint{0.883954in}{1.321384in}}%
\pgfpathlineto{\pgfqpoint{0.885165in}{1.322434in}}%
\pgfpathlineto{\pgfqpoint{0.886226in}{1.323624in}}%
\pgfpathlineto{\pgfqpoint{0.887121in}{1.324674in}}%
\pgfpathlineto{\pgfqpoint{0.888215in}{1.325864in}}%
\pgfpathlineto{\pgfqpoint{0.889624in}{1.326914in}}%
\pgfpathlineto{\pgfqpoint{0.890619in}{1.328104in}}%
\pgfpathlineto{\pgfqpoint{0.892044in}{1.329154in}}%
\pgfpathlineto{\pgfqpoint{0.892857in}{1.330064in}}%
\pgfpathlineto{\pgfqpoint{0.894100in}{1.331044in}}%
\pgfpathlineto{\pgfqpoint{0.894746in}{1.331814in}}%
\pgfpathlineto{\pgfqpoint{0.896089in}{1.332864in}}%
\pgfpathlineto{\pgfqpoint{0.897200in}{1.333984in}}%
\pgfpathlineto{\pgfqpoint{0.898327in}{1.335034in}}%
\pgfpathlineto{\pgfqpoint{0.899338in}{1.336714in}}%
\pgfpathlineto{\pgfqpoint{0.900267in}{1.337694in}}%
\pgfpathlineto{\pgfqpoint{0.901311in}{1.338534in}}%
\pgfpathlineto{\pgfqpoint{0.902554in}{1.339584in}}%
\pgfpathlineto{\pgfqpoint{0.903649in}{1.340494in}}%
\pgfpathlineto{\pgfqpoint{0.905240in}{1.341544in}}%
\pgfpathlineto{\pgfqpoint{0.906334in}{1.342874in}}%
\pgfpathlineto{\pgfqpoint{0.907743in}{1.343924in}}%
\pgfpathlineto{\pgfqpoint{0.908854in}{1.344764in}}%
\pgfpathlineto{\pgfqpoint{0.910048in}{1.345814in}}%
\pgfpathlineto{\pgfqpoint{0.911142in}{1.346654in}}%
\pgfpathlineto{\pgfqpoint{0.912169in}{1.347704in}}%
\pgfpathlineto{\pgfqpoint{0.913114in}{1.348754in}}%
\pgfpathlineto{\pgfqpoint{0.914242in}{1.349804in}}%
\pgfpathlineto{\pgfqpoint{0.915004in}{1.350924in}}%
\pgfpathlineto{\pgfqpoint{0.916894in}{1.351974in}}%
\pgfpathlineto{\pgfqpoint{0.917988in}{1.353234in}}%
\pgfpathlineto{\pgfqpoint{0.919646in}{1.354284in}}%
\pgfpathlineto{\pgfqpoint{0.920757in}{1.356034in}}%
\pgfpathlineto{\pgfqpoint{0.922166in}{1.357084in}}%
\pgfpathlineto{\pgfqpoint{0.923260in}{1.358204in}}%
\pgfpathlineto{\pgfqpoint{0.925332in}{1.359254in}}%
\pgfpathlineto{\pgfqpoint{0.926376in}{1.360164in}}%
\pgfpathlineto{\pgfqpoint{0.927686in}{1.361214in}}%
\pgfpathlineto{\pgfqpoint{0.928780in}{1.361914in}}%
\pgfpathlineto{\pgfqpoint{0.930886in}{1.362964in}}%
\pgfpathlineto{\pgfqpoint{0.931980in}{1.364294in}}%
\pgfpathlineto{\pgfqpoint{0.932941in}{1.365344in}}%
\pgfpathlineto{\pgfqpoint{0.934035in}{1.366604in}}%
\pgfpathlineto{\pgfqpoint{0.935925in}{1.367654in}}%
\pgfpathlineto{\pgfqpoint{0.937003in}{1.368984in}}%
\pgfpathlineto{\pgfqpoint{0.939307in}{1.370034in}}%
\pgfpathlineto{\pgfqpoint{0.940302in}{1.371014in}}%
\pgfpathlineto{\pgfqpoint{0.941876in}{1.372064in}}%
\pgfpathlineto{\pgfqpoint{0.942871in}{1.372694in}}%
\pgfpathlineto{\pgfqpoint{0.944562in}{1.373674in}}%
\pgfpathlineto{\pgfqpoint{0.945573in}{1.374934in}}%
\pgfpathlineto{\pgfqpoint{0.947314in}{1.375984in}}%
\pgfpathlineto{\pgfqpoint{0.948159in}{1.376824in}}%
\pgfpathlineto{\pgfqpoint{0.950663in}{1.377874in}}%
\pgfpathlineto{\pgfqpoint{0.951707in}{1.378504in}}%
\pgfpathlineto{\pgfqpoint{0.953083in}{1.379484in}}%
\pgfpathlineto{\pgfqpoint{0.954127in}{1.380394in}}%
\pgfpathlineto{\pgfqpoint{0.956183in}{1.381444in}}%
\pgfpathlineto{\pgfqpoint{0.957244in}{1.382914in}}%
\pgfpathlineto{\pgfqpoint{0.958736in}{1.383964in}}%
\pgfpathlineto{\pgfqpoint{0.959714in}{1.384874in}}%
\pgfpathlineto{\pgfqpoint{0.961902in}{1.385924in}}%
\pgfpathlineto{\pgfqpoint{0.962781in}{1.386414in}}%
\pgfpathlineto{\pgfqpoint{0.965599in}{1.387464in}}%
\pgfpathlineto{\pgfqpoint{0.966643in}{1.388514in}}%
\pgfpathlineto{\pgfqpoint{0.968069in}{1.389564in}}%
\pgfpathlineto{\pgfqpoint{0.968931in}{1.390404in}}%
\pgfpathlineto{\pgfqpoint{0.971733in}{1.391454in}}%
\pgfpathlineto{\pgfqpoint{0.972843in}{1.391874in}}%
\pgfpathlineto{\pgfqpoint{0.974286in}{1.392924in}}%
\pgfpathlineto{\pgfqpoint{0.975197in}{1.393554in}}%
\pgfpathlineto{\pgfqpoint{0.977336in}{1.394604in}}%
\pgfpathlineto{\pgfqpoint{0.978413in}{1.395654in}}%
\pgfpathlineto{\pgfqpoint{0.979607in}{1.396704in}}%
\pgfpathlineto{\pgfqpoint{0.980635in}{1.397614in}}%
\pgfpathlineto{\pgfqpoint{0.981729in}{1.398664in}}%
\pgfpathlineto{\pgfqpoint{0.982690in}{1.399224in}}%
\pgfpathlineto{\pgfqpoint{0.984365in}{1.400274in}}%
\pgfpathlineto{\pgfqpoint{0.985343in}{1.401044in}}%
\pgfpathlineto{\pgfqpoint{0.987050in}{1.402094in}}%
\pgfpathlineto{\pgfqpoint{0.987912in}{1.403004in}}%
\pgfpathlineto{\pgfqpoint{0.990051in}{1.404054in}}%
\pgfpathlineto{\pgfqpoint{0.990631in}{1.404474in}}%
\pgfpathlineto{\pgfqpoint{0.992935in}{1.405524in}}%
\pgfpathlineto{\pgfqpoint{0.993847in}{1.406294in}}%
\pgfpathlineto{\pgfqpoint{0.995787in}{1.407344in}}%
\pgfpathlineto{\pgfqpoint{0.996798in}{1.408534in}}%
\pgfpathlineto{\pgfqpoint{0.998456in}{1.409444in}}%
\pgfpathlineto{\pgfqpoint{0.999533in}{1.410424in}}%
\pgfpathlineto{\pgfqpoint{1.001224in}{1.411404in}}%
\pgfpathlineto{\pgfqpoint{1.002235in}{1.412314in}}%
\pgfpathlineto{\pgfqpoint{1.003545in}{1.413364in}}%
\pgfpathlineto{\pgfqpoint{1.004423in}{1.414274in}}%
\pgfpathlineto{\pgfqpoint{1.005882in}{1.415324in}}%
\pgfpathlineto{\pgfqpoint{1.006678in}{1.415954in}}%
\pgfpathlineto{\pgfqpoint{1.009165in}{1.417004in}}%
\pgfpathlineto{\pgfqpoint{1.010226in}{1.417774in}}%
\pgfpathlineto{\pgfqpoint{1.012861in}{1.418824in}}%
\pgfpathlineto{\pgfqpoint{1.013972in}{1.419524in}}%
\pgfpathlineto{\pgfqpoint{1.016442in}{1.420504in}}%
\pgfpathlineto{\pgfqpoint{1.017437in}{1.421274in}}%
\pgfpathlineto{\pgfqpoint{1.019890in}{1.422324in}}%
\pgfpathlineto{\pgfqpoint{1.020852in}{1.423444in}}%
\pgfpathlineto{\pgfqpoint{1.022626in}{1.424494in}}%
\pgfpathlineto{\pgfqpoint{1.023488in}{1.425334in}}%
\pgfpathlineto{\pgfqpoint{1.026405in}{1.426384in}}%
\pgfpathlineto{\pgfqpoint{1.027516in}{1.427224in}}%
\pgfpathlineto{\pgfqpoint{1.029787in}{1.428274in}}%
\pgfpathlineto{\pgfqpoint{1.030831in}{1.429184in}}%
\pgfpathlineto{\pgfqpoint{1.032887in}{1.430234in}}%
\pgfpathlineto{\pgfqpoint{1.033882in}{1.431074in}}%
\pgfpathlineto{\pgfqpoint{1.036667in}{1.432124in}}%
\pgfpathlineto{\pgfqpoint{1.037711in}{1.432894in}}%
\pgfpathlineto{\pgfqpoint{1.041872in}{1.433944in}}%
\pgfpathlineto{\pgfqpoint{1.042834in}{1.434504in}}%
\pgfpathlineto{\pgfqpoint{1.046530in}{1.435554in}}%
\pgfpathlineto{\pgfqpoint{1.047575in}{1.436044in}}%
\pgfpathlineto{\pgfqpoint{1.049514in}{1.437094in}}%
\pgfpathlineto{\pgfqpoint{1.050625in}{1.437794in}}%
\pgfpathlineto{\pgfqpoint{1.053095in}{1.438844in}}%
\pgfpathlineto{\pgfqpoint{1.054090in}{1.439544in}}%
\pgfpathlineto{\pgfqpoint{1.056692in}{1.440594in}}%
\pgfpathlineto{\pgfqpoint{1.057704in}{1.441364in}}%
\pgfpathlineto{\pgfqpoint{1.061301in}{1.442414in}}%
\pgfpathlineto{\pgfqpoint{1.062329in}{1.443044in}}%
\pgfpathlineto{\pgfqpoint{1.065893in}{1.444094in}}%
\pgfpathlineto{\pgfqpoint{1.066987in}{1.444584in}}%
\pgfpathlineto{\pgfqpoint{1.068976in}{1.445634in}}%
\pgfpathlineto{\pgfqpoint{1.070037in}{1.446544in}}%
\pgfpathlineto{\pgfqpoint{1.072491in}{1.447594in}}%
\pgfpathlineto{\pgfqpoint{1.073369in}{1.448224in}}%
\pgfpathlineto{\pgfqpoint{1.075840in}{1.449274in}}%
\pgfpathlineto{\pgfqpoint{1.076652in}{1.449694in}}%
\pgfpathlineto{\pgfqpoint{1.079006in}{1.450744in}}%
\pgfpathlineto{\pgfqpoint{1.079967in}{1.451234in}}%
\pgfpathlineto{\pgfqpoint{1.083001in}{1.452214in}}%
\pgfpathlineto{\pgfqpoint{1.084112in}{1.452914in}}%
\pgfpathlineto{\pgfqpoint{1.087726in}{1.453894in}}%
\pgfpathlineto{\pgfqpoint{1.088438in}{1.454384in}}%
\pgfpathlineto{\pgfqpoint{1.092119in}{1.455434in}}%
\pgfpathlineto{\pgfqpoint{1.093064in}{1.456064in}}%
\pgfpathlineto{\pgfqpoint{1.096346in}{1.457114in}}%
\pgfpathlineto{\pgfqpoint{1.097274in}{1.457674in}}%
\pgfpathlineto{\pgfqpoint{1.100092in}{1.458724in}}%
\pgfpathlineto{\pgfqpoint{1.100689in}{1.459004in}}%
\pgfpathlineto{\pgfqpoint{1.104038in}{1.460054in}}%
\pgfpathlineto{\pgfqpoint{1.105149in}{1.460404in}}%
\pgfpathlineto{\pgfqpoint{1.109989in}{1.461454in}}%
\pgfpathlineto{\pgfqpoint{1.111083in}{1.462154in}}%
\pgfpathlineto{\pgfqpoint{1.114813in}{1.463204in}}%
\pgfpathlineto{\pgfqpoint{1.115858in}{1.463624in}}%
\pgfpathlineto{\pgfqpoint{1.119389in}{1.464674in}}%
\pgfpathlineto{\pgfqpoint{1.120433in}{1.465094in}}%
\pgfpathlineto{\pgfqpoint{1.122754in}{1.466144in}}%
\pgfpathlineto{\pgfqpoint{1.123732in}{1.466494in}}%
\pgfpathlineto{\pgfqpoint{1.127910in}{1.467544in}}%
\pgfpathlineto{\pgfqpoint{1.128556in}{1.468034in}}%
\pgfpathlineto{\pgfqpoint{1.131010in}{1.469084in}}%
\pgfpathlineto{\pgfqpoint{1.131822in}{1.469924in}}%
\pgfpathlineto{\pgfqpoint{1.135187in}{1.470974in}}%
\pgfpathlineto{\pgfqpoint{1.135204in}{1.471184in}}%
\pgfpathlineto{\pgfqpoint{1.140956in}{1.472234in}}%
\pgfpathlineto{\pgfqpoint{1.141735in}{1.472584in}}%
\pgfpathlineto{\pgfqpoint{1.145598in}{1.473634in}}%
\pgfpathlineto{\pgfqpoint{1.146675in}{1.474194in}}%
\pgfpathlineto{\pgfqpoint{1.150803in}{1.475244in}}%
\pgfpathlineto{\pgfqpoint{1.151516in}{1.475874in}}%
\pgfpathlineto{\pgfqpoint{1.156058in}{1.476924in}}%
\pgfpathlineto{\pgfqpoint{1.157152in}{1.477414in}}%
\pgfpathlineto{\pgfqpoint{1.162407in}{1.478464in}}%
\pgfpathlineto{\pgfqpoint{1.163452in}{1.479094in}}%
\pgfpathlineto{\pgfqpoint{1.168060in}{1.480144in}}%
\pgfpathlineto{\pgfqpoint{1.168873in}{1.480494in}}%
\pgfpathlineto{\pgfqpoint{1.171923in}{1.481474in}}%
\pgfpathlineto{\pgfqpoint{1.172835in}{1.481754in}}%
\pgfpathlineto{\pgfqpoint{1.177161in}{1.482804in}}%
\pgfpathlineto{\pgfqpoint{1.178156in}{1.483294in}}%
\pgfpathlineto{\pgfqpoint{1.181223in}{1.484344in}}%
\pgfpathlineto{\pgfqpoint{1.181273in}{1.484554in}}%
\pgfpathlineto{\pgfqpoint{1.188633in}{1.485604in}}%
\pgfpathlineto{\pgfqpoint{1.189512in}{1.485884in}}%
\pgfpathlineto{\pgfqpoint{1.193341in}{1.486934in}}%
\pgfpathlineto{\pgfqpoint{1.194203in}{1.487354in}}%
\pgfpathlineto{\pgfqpoint{1.198928in}{1.488404in}}%
\pgfpathlineto{\pgfqpoint{1.199823in}{1.488964in}}%
\pgfpathlineto{\pgfqpoint{1.201895in}{1.489944in}}%
\pgfpathlineto{\pgfqpoint{1.202343in}{1.490224in}}%
\pgfpathlineto{\pgfqpoint{1.205343in}{1.491274in}}%
\pgfpathlineto{\pgfqpoint{1.206404in}{1.491554in}}%
\pgfpathlineto{\pgfqpoint{1.209670in}{1.492604in}}%
\pgfpathlineto{\pgfqpoint{1.210748in}{1.493024in}}%
\pgfpathlineto{\pgfqpoint{1.215887in}{1.494074in}}%
\pgfpathlineto{\pgfqpoint{1.216865in}{1.494214in}}%
\pgfpathlineto{\pgfqpoint{1.223860in}{1.495264in}}%
\pgfpathlineto{\pgfqpoint{1.224739in}{1.495684in}}%
\pgfpathlineto{\pgfqpoint{1.229994in}{1.496734in}}%
\pgfpathlineto{\pgfqpoint{1.230823in}{1.497014in}}%
\pgfpathlineto{\pgfqpoint{1.234354in}{1.498064in}}%
\pgfpathlineto{\pgfqpoint{1.235117in}{1.498344in}}%
\pgfpathlineto{\pgfqpoint{1.240256in}{1.499394in}}%
\pgfpathlineto{\pgfqpoint{1.241151in}{1.499674in}}%
\pgfpathlineto{\pgfqpoint{1.245610in}{1.500724in}}%
\pgfpathlineto{\pgfqpoint{1.246721in}{1.501144in}}%
\pgfpathlineto{\pgfqpoint{1.248942in}{1.502194in}}%
\pgfpathlineto{\pgfqpoint{1.250053in}{1.502754in}}%
\pgfpathlineto{\pgfqpoint{1.254429in}{1.503804in}}%
\pgfpathlineto{\pgfqpoint{1.254728in}{1.504084in}}%
\pgfpathlineto{\pgfqpoint{1.260845in}{1.505134in}}%
\pgfpathlineto{\pgfqpoint{1.261723in}{1.505414in}}%
\pgfpathlineto{\pgfqpoint{1.266697in}{1.506464in}}%
\pgfpathlineto{\pgfqpoint{1.267807in}{1.506744in}}%
\pgfpathlineto{\pgfqpoint{1.271338in}{1.507724in}}%
\pgfpathlineto{\pgfqpoint{1.272200in}{1.508214in}}%
\pgfpathlineto{\pgfqpoint{1.277936in}{1.509264in}}%
\pgfpathlineto{\pgfqpoint{1.278036in}{1.509404in}}%
\pgfpathlineto{\pgfqpoint{1.284203in}{1.510454in}}%
\pgfpathlineto{\pgfqpoint{1.285131in}{1.510804in}}%
\pgfpathlineto{\pgfqpoint{1.289657in}{1.511854in}}%
\pgfpathlineto{\pgfqpoint{1.290585in}{1.512274in}}%
\pgfpathlineto{\pgfqpoint{1.295940in}{1.513324in}}%
\pgfpathlineto{\pgfqpoint{1.296354in}{1.513534in}}%
\pgfpathlineto{\pgfqpoint{1.299703in}{1.514584in}}%
\pgfpathlineto{\pgfqpoint{1.300664in}{1.514934in}}%
\pgfpathlineto{\pgfqpoint{1.305389in}{1.515984in}}%
\pgfpathlineto{\pgfqpoint{1.306118in}{1.516334in}}%
\pgfpathlineto{\pgfqpoint{1.311290in}{1.517384in}}%
\pgfpathlineto{\pgfqpoint{1.312318in}{1.517734in}}%
\pgfpathlineto{\pgfqpoint{1.318104in}{1.518784in}}%
\pgfpathlineto{\pgfqpoint{1.318899in}{1.519274in}}%
\pgfpathlineto{\pgfqpoint{1.323873in}{1.520324in}}%
\pgfpathlineto{\pgfqpoint{1.324950in}{1.520674in}}%
\pgfpathlineto{\pgfqpoint{1.329575in}{1.521654in}}%
\pgfpathlineto{\pgfqpoint{1.330305in}{1.521864in}}%
\pgfpathlineto{\pgfqpoint{1.336521in}{1.522844in}}%
\pgfpathlineto{\pgfqpoint{1.336836in}{1.523194in}}%
\pgfpathlineto{\pgfqpoint{1.344180in}{1.524244in}}%
\pgfpathlineto{\pgfqpoint{1.345191in}{1.524524in}}%
\pgfpathlineto{\pgfqpoint{1.351905in}{1.525574in}}%
\pgfpathlineto{\pgfqpoint{1.352602in}{1.525994in}}%
\pgfpathlineto{\pgfqpoint{1.359050in}{1.527044in}}%
\pgfpathlineto{\pgfqpoint{1.359962in}{1.527394in}}%
\pgfpathlineto{\pgfqpoint{1.365897in}{1.528444in}}%
\pgfpathlineto{\pgfqpoint{1.366991in}{1.528654in}}%
\pgfpathlineto{\pgfqpoint{1.378463in}{1.529704in}}%
\pgfpathlineto{\pgfqpoint{1.379092in}{1.530054in}}%
\pgfpathlineto{\pgfqpoint{1.386221in}{1.531104in}}%
\pgfpathlineto{\pgfqpoint{1.387298in}{1.531314in}}%
\pgfpathlineto{\pgfqpoint{1.393366in}{1.532364in}}%
\pgfpathlineto{\pgfqpoint{1.393880in}{1.532714in}}%
\pgfpathlineto{\pgfqpoint{1.403760in}{1.533764in}}%
\pgfpathlineto{\pgfqpoint{1.404340in}{1.533904in}}%
\pgfpathlineto{\pgfqpoint{1.412529in}{1.534954in}}%
\pgfpathlineto{\pgfqpoint{1.412994in}{1.535164in}}%
\pgfpathlineto{\pgfqpoint{1.422277in}{1.536214in}}%
\pgfpathlineto{\pgfqpoint{1.422807in}{1.536354in}}%
\pgfpathlineto{\pgfqpoint{1.428460in}{1.537404in}}%
\pgfpathlineto{\pgfqpoint{1.429157in}{1.537614in}}%
\pgfpathlineto{\pgfqpoint{1.435340in}{1.538664in}}%
\pgfpathlineto{\pgfqpoint{1.436351in}{1.539014in}}%
\pgfpathlineto{\pgfqpoint{1.443861in}{1.540064in}}%
\pgfpathlineto{\pgfqpoint{1.443877in}{1.540274in}}%
\pgfpathlineto{\pgfqpoint{1.450940in}{1.541324in}}%
\pgfpathlineto{\pgfqpoint{1.451006in}{1.541464in}}%
\pgfpathlineto{\pgfqpoint{1.462776in}{1.542514in}}%
\pgfpathlineto{\pgfqpoint{1.463439in}{1.542724in}}%
\pgfpathlineto{\pgfqpoint{1.473817in}{1.543774in}}%
\pgfpathlineto{\pgfqpoint{1.474347in}{1.544124in}}%
\pgfpathlineto{\pgfqpoint{1.482304in}{1.545174in}}%
\pgfpathlineto{\pgfqpoint{1.482669in}{1.545314in}}%
\pgfpathlineto{\pgfqpoint{1.493892in}{1.546364in}}%
\pgfpathlineto{\pgfqpoint{1.494257in}{1.546504in}}%
\pgfpathlineto{\pgfqpoint{1.503756in}{1.547554in}}%
\pgfpathlineto{\pgfqpoint{1.503921in}{1.547834in}}%
\pgfpathlineto{\pgfqpoint{1.514382in}{1.548884in}}%
\pgfpathlineto{\pgfqpoint{1.515227in}{1.549234in}}%
\pgfpathlineto{\pgfqpoint{1.524046in}{1.550284in}}%
\pgfpathlineto{\pgfqpoint{1.524378in}{1.550564in}}%
\pgfpathlineto{\pgfqpoint{1.535187in}{1.551614in}}%
\pgfpathlineto{\pgfqpoint{1.535817in}{1.551754in}}%
\pgfpathlineto{\pgfqpoint{1.546012in}{1.552804in}}%
\pgfpathlineto{\pgfqpoint{1.547122in}{1.553154in}}%
\pgfpathlineto{\pgfqpoint{1.557450in}{1.554204in}}%
\pgfpathlineto{\pgfqpoint{1.557450in}{1.554274in}}%
\pgfpathlineto{\pgfqpoint{1.571690in}{1.555324in}}%
\pgfpathlineto{\pgfqpoint{1.571790in}{1.555534in}}%
\pgfpathlineto{\pgfqpoint{1.581405in}{1.556584in}}%
\pgfpathlineto{\pgfqpoint{1.581653in}{1.556794in}}%
\pgfpathlineto{\pgfqpoint{1.595396in}{1.557844in}}%
\pgfpathlineto{\pgfqpoint{1.595844in}{1.558054in}}%
\pgfpathlineto{\pgfqpoint{1.606503in}{1.559104in}}%
\pgfpathlineto{\pgfqpoint{1.607514in}{1.559384in}}%
\pgfpathlineto{\pgfqpoint{1.617610in}{1.560434in}}%
\pgfpathlineto{\pgfqpoint{1.618721in}{1.560644in}}%
\pgfpathlineto{\pgfqpoint{1.630889in}{1.561694in}}%
\pgfpathlineto{\pgfqpoint{1.631883in}{1.561834in}}%
\pgfpathlineto{\pgfqpoint{1.652605in}{1.562884in}}%
\pgfpathlineto{\pgfqpoint{1.653600in}{1.563024in}}%
\pgfpathlineto{\pgfqpoint{1.663812in}{1.564074in}}%
\pgfpathlineto{\pgfqpoint{1.663861in}{1.564214in}}%
\pgfpathlineto{\pgfqpoint{1.672283in}{1.565264in}}%
\pgfpathlineto{\pgfqpoint{1.672382in}{1.565404in}}%
\pgfpathlineto{\pgfqpoint{1.683406in}{1.566454in}}%
\pgfpathlineto{\pgfqpoint{1.684467in}{1.566664in}}%
\pgfpathlineto{\pgfqpoint{1.696586in}{1.567714in}}%
\pgfpathlineto{\pgfqpoint{1.696834in}{1.567854in}}%
\pgfpathlineto{\pgfqpoint{1.710212in}{1.568904in}}%
\pgfpathlineto{\pgfqpoint{1.710909in}{1.569114in}}%
\pgfpathlineto{\pgfqpoint{1.723706in}{1.570164in}}%
\pgfpathlineto{\pgfqpoint{1.724718in}{1.570304in}}%
\pgfpathlineto{\pgfqpoint{1.739090in}{1.571354in}}%
\pgfpathlineto{\pgfqpoint{1.739422in}{1.571494in}}%
\pgfpathlineto{\pgfqpoint{1.756828in}{1.572544in}}%
\pgfpathlineto{\pgfqpoint{1.757591in}{1.572754in}}%
\pgfpathlineto{\pgfqpoint{1.769825in}{1.573804in}}%
\pgfpathlineto{\pgfqpoint{1.770339in}{1.573944in}}%
\pgfpathlineto{\pgfqpoint{1.782242in}{1.574994in}}%
\pgfpathlineto{\pgfqpoint{1.782374in}{1.575134in}}%
\pgfpathlineto{\pgfqpoint{1.795288in}{1.576184in}}%
\pgfpathlineto{\pgfqpoint{1.795868in}{1.576394in}}%
\pgfpathlineto{\pgfqpoint{1.809910in}{1.577444in}}%
\pgfpathlineto{\pgfqpoint{1.810987in}{1.577724in}}%
\pgfpathlineto{\pgfqpoint{1.828360in}{1.578774in}}%
\pgfpathlineto{\pgfqpoint{1.829057in}{1.578984in}}%
\pgfpathlineto{\pgfqpoint{1.842236in}{1.580034in}}%
\pgfpathlineto{\pgfqpoint{1.842368in}{1.580174in}}%
\pgfpathlineto{\pgfqpoint{1.859410in}{1.581224in}}%
\pgfpathlineto{\pgfqpoint{1.859891in}{1.581364in}}%
\pgfpathlineto{\pgfqpoint{1.873683in}{1.582414in}}%
\pgfpathlineto{\pgfqpoint{1.873683in}{1.582484in}}%
\pgfpathlineto{\pgfqpoint{1.894190in}{1.583534in}}%
\pgfpathlineto{\pgfqpoint{1.894637in}{1.583674in}}%
\pgfpathlineto{\pgfqpoint{1.915542in}{1.584724in}}%
\pgfpathlineto{\pgfqpoint{1.916155in}{1.584864in}}%
\pgfpathlineto{\pgfqpoint{1.929334in}{1.585914in}}%
\pgfpathlineto{\pgfqpoint{1.930345in}{1.586054in}}%
\pgfpathlineto{\pgfqpoint{1.945680in}{1.587104in}}%
\pgfpathlineto{\pgfqpoint{1.945680in}{1.587174in}}%
\pgfpathlineto{\pgfqpoint{1.958245in}{1.588224in}}%
\pgfpathlineto{\pgfqpoint{1.959074in}{1.588434in}}%
\pgfpathlineto{\pgfqpoint{1.971541in}{1.589484in}}%
\pgfpathlineto{\pgfqpoint{1.972005in}{1.589624in}}%
\pgfpathlineto{\pgfqpoint{1.980807in}{1.590674in}}%
\pgfpathlineto{\pgfqpoint{1.981636in}{1.590954in}}%
\pgfpathlineto{\pgfqpoint{1.988947in}{1.592004in}}%
\pgfpathlineto{\pgfqpoint{1.988947in}{1.592074in}}%
\pgfpathlineto{\pgfqpoint{1.998313in}{1.593124in}}%
\pgfpathlineto{\pgfqpoint{1.999391in}{1.593334in}}%
\pgfpathlineto{\pgfqpoint{2.006536in}{1.594384in}}%
\pgfpathlineto{\pgfqpoint{2.007414in}{1.594524in}}%
\pgfpathlineto{\pgfqpoint{2.013349in}{1.595574in}}%
\pgfpathlineto{\pgfqpoint{2.014195in}{1.595854in}}%
\pgfpathlineto{\pgfqpoint{2.021936in}{1.596904in}}%
\pgfpathlineto{\pgfqpoint{2.022318in}{1.597044in}}%
\pgfpathlineto{\pgfqpoint{2.028136in}{1.598094in}}%
\pgfpathlineto{\pgfqpoint{2.029230in}{1.598584in}}%
\pgfpathlineto{\pgfqpoint{2.032015in}{1.599634in}}%
\pgfpathlineto{\pgfqpoint{2.033126in}{1.601944in}}%
\pgfpathlineto{\pgfqpoint{2.033126in}{1.601944in}}%
\pgfusepath{stroke}%
\end{pgfscope}%
\begin{pgfscope}%
\pgfsetrectcap%
\pgfsetmiterjoin%
\pgfsetlinewidth{0.803000pt}%
\definecolor{currentstroke}{rgb}{0.000000,0.000000,0.000000}%
\pgfsetstrokecolor{currentstroke}%
\pgfsetdash{}{0pt}%
\pgfpathmoveto{\pgfqpoint{0.553581in}{0.499444in}}%
\pgfpathlineto{\pgfqpoint{0.553581in}{1.654444in}}%
\pgfusepath{stroke}%
\end{pgfscope}%
\begin{pgfscope}%
\pgfsetrectcap%
\pgfsetmiterjoin%
\pgfsetlinewidth{0.803000pt}%
\definecolor{currentstroke}{rgb}{0.000000,0.000000,0.000000}%
\pgfsetstrokecolor{currentstroke}%
\pgfsetdash{}{0pt}%
\pgfpathmoveto{\pgfqpoint{2.103581in}{0.499444in}}%
\pgfpathlineto{\pgfqpoint{2.103581in}{1.654444in}}%
\pgfusepath{stroke}%
\end{pgfscope}%
\begin{pgfscope}%
\pgfsetrectcap%
\pgfsetmiterjoin%
\pgfsetlinewidth{0.803000pt}%
\definecolor{currentstroke}{rgb}{0.000000,0.000000,0.000000}%
\pgfsetstrokecolor{currentstroke}%
\pgfsetdash{}{0pt}%
\pgfpathmoveto{\pgfqpoint{0.553581in}{0.499444in}}%
\pgfpathlineto{\pgfqpoint{2.103581in}{0.499444in}}%
\pgfusepath{stroke}%
\end{pgfscope}%
\begin{pgfscope}%
\pgfsetrectcap%
\pgfsetmiterjoin%
\pgfsetlinewidth{0.803000pt}%
\definecolor{currentstroke}{rgb}{0.000000,0.000000,0.000000}%
\pgfsetstrokecolor{currentstroke}%
\pgfsetdash{}{0pt}%
\pgfpathmoveto{\pgfqpoint{0.553581in}{1.654444in}}%
\pgfpathlineto{\pgfqpoint{2.103581in}{1.654444in}}%
\pgfusepath{stroke}%
\end{pgfscope}%
\begin{pgfscope}%
\pgfsetbuttcap%
\pgfsetmiterjoin%
\definecolor{currentfill}{rgb}{1.000000,1.000000,1.000000}%
\pgfsetfillcolor{currentfill}%
\pgfsetfillopacity{0.800000}%
\pgfsetlinewidth{1.003750pt}%
\definecolor{currentstroke}{rgb}{0.800000,0.800000,0.800000}%
\pgfsetstrokecolor{currentstroke}%
\pgfsetstrokeopacity{0.800000}%
\pgfsetdash{}{0pt}%
\pgfpathmoveto{\pgfqpoint{0.832747in}{0.568889in}}%
\pgfpathlineto{\pgfqpoint{2.006358in}{0.568889in}}%
\pgfpathquadraticcurveto{\pgfqpoint{2.034136in}{0.568889in}}{\pgfqpoint{2.034136in}{0.596666in}}%
\pgfpathlineto{\pgfqpoint{2.034136in}{0.776388in}}%
\pgfpathquadraticcurveto{\pgfqpoint{2.034136in}{0.804166in}}{\pgfqpoint{2.006358in}{0.804166in}}%
\pgfpathlineto{\pgfqpoint{0.832747in}{0.804166in}}%
\pgfpathquadraticcurveto{\pgfqpoint{0.804970in}{0.804166in}}{\pgfqpoint{0.804970in}{0.776388in}}%
\pgfpathlineto{\pgfqpoint{0.804970in}{0.596666in}}%
\pgfpathquadraticcurveto{\pgfqpoint{0.804970in}{0.568889in}}{\pgfqpoint{0.832747in}{0.568889in}}%
\pgfpathlineto{\pgfqpoint{0.832747in}{0.568889in}}%
\pgfpathclose%
\pgfusepath{stroke,fill}%
\end{pgfscope}%
\begin{pgfscope}%
\pgfsetrectcap%
\pgfsetroundjoin%
\pgfsetlinewidth{1.505625pt}%
\definecolor{currentstroke}{rgb}{0.000000,0.000000,0.000000}%
\pgfsetstrokecolor{currentstroke}%
\pgfsetdash{}{0pt}%
\pgfpathmoveto{\pgfqpoint{0.860525in}{0.700000in}}%
\pgfpathlineto{\pgfqpoint{0.999414in}{0.700000in}}%
\pgfpathlineto{\pgfqpoint{1.138303in}{0.700000in}}%
\pgfusepath{stroke}%
\end{pgfscope}%
\begin{pgfscope}%
\definecolor{textcolor}{rgb}{0.000000,0.000000,0.000000}%
\pgfsetstrokecolor{textcolor}%
\pgfsetfillcolor{textcolor}%
\pgftext[x=1.249414in,y=0.651388in,left,base]{\color{textcolor}\rmfamily\fontsize{10.000000}{12.000000}\selectfont AUC=0.840}%
\end{pgfscope}%
\end{pgfpicture}%
\makeatother%
\endgroup%

	
&
	\vskip 0pt
	\begin{tabular}{cc|c|c|}
	&\multicolumn{1}{c}{}& \multicolumn{2}{c}{Prediction} \\[0.4em]
	&\multicolumn{1}{c}{} & \multicolumn{1}{c}{N} & \multicolumn{1}{c}{P} \cr\cline{3-4}
	\multirow{2}{*}{\rotatebox[origin=c]{90}{Actual}}&N &
66.5\% & 18.5\%
	\vrule width 0pt height 10pt depth 2pt \cr\cline{3-4}
	&P & 
3.3\% & 11.7\%
	\vrule width 0pt height 10pt depth 2pt \cr\cline{3-4}
	\end{tabular}

	\hfil\begin{tabular}{ll}
	\cr
	0.781 & Accuracy\cr
	0.387 & Precision \cr
	0.777 & Recall \cr
	0.517 & F1 \cr
	0.840 & AUC \cr
\end{tabular}

\cr
\end{tabular}
} % End parbox


\

{\bf Awful Model}  A model with these results has not successfully separated the positive and negative classes, possibly because the data has no pattern.  

\

\verb|Awful|

%%%
\parbox{\linewidth}{
\noindent\begin{tabular}{@{\hspace{-6pt}}p{2.3in} @{\hspace{-6pt}}p{2.0in} p{1.8in}}
	\vskip 0pt
	\qquad \qquad Raw Model Output
	
	%% Creator: Matplotlib, PGF backend
%%
%% To include the figure in your LaTeX document, write
%%   \input{<filename>.pgf}
%%
%% Make sure the required packages are loaded in your preamble
%%   \usepackage{pgf}
%%
%% Also ensure that all the required font packages are loaded; for instance,
%% the lmodern package is sometimes necessary when using math font.
%%   \usepackage{lmodern}
%%
%% Figures using additional raster images can only be included by \input if
%% they are in the same directory as the main LaTeX file. For loading figures
%% from other directories you can use the `import` package
%%   \usepackage{import}
%%
%% and then include the figures with
%%   \import{<path to file>}{<filename>.pgf}
%%
%% Matplotlib used the following preamble
%%   
%%   \usepackage{fontspec}
%%   \makeatletter\@ifpackageloaded{underscore}{}{\usepackage[strings]{underscore}}\makeatother
%%
\begingroup%
\makeatletter%
\begin{pgfpicture}%
\pgfpathrectangle{\pgfpointorigin}{\pgfqpoint{2.153750in}{1.654444in}}%
\pgfusepath{use as bounding box, clip}%
\begin{pgfscope}%
\pgfsetbuttcap%
\pgfsetmiterjoin%
\definecolor{currentfill}{rgb}{1.000000,1.000000,1.000000}%
\pgfsetfillcolor{currentfill}%
\pgfsetlinewidth{0.000000pt}%
\definecolor{currentstroke}{rgb}{1.000000,1.000000,1.000000}%
\pgfsetstrokecolor{currentstroke}%
\pgfsetdash{}{0pt}%
\pgfpathmoveto{\pgfqpoint{0.000000in}{0.000000in}}%
\pgfpathlineto{\pgfqpoint{2.153750in}{0.000000in}}%
\pgfpathlineto{\pgfqpoint{2.153750in}{1.654444in}}%
\pgfpathlineto{\pgfqpoint{0.000000in}{1.654444in}}%
\pgfpathlineto{\pgfqpoint{0.000000in}{0.000000in}}%
\pgfpathclose%
\pgfusepath{fill}%
\end{pgfscope}%
\begin{pgfscope}%
\pgfsetbuttcap%
\pgfsetmiterjoin%
\definecolor{currentfill}{rgb}{1.000000,1.000000,1.000000}%
\pgfsetfillcolor{currentfill}%
\pgfsetlinewidth{0.000000pt}%
\definecolor{currentstroke}{rgb}{0.000000,0.000000,0.000000}%
\pgfsetstrokecolor{currentstroke}%
\pgfsetstrokeopacity{0.000000}%
\pgfsetdash{}{0pt}%
\pgfpathmoveto{\pgfqpoint{0.465000in}{0.449444in}}%
\pgfpathlineto{\pgfqpoint{2.015000in}{0.449444in}}%
\pgfpathlineto{\pgfqpoint{2.015000in}{1.604444in}}%
\pgfpathlineto{\pgfqpoint{0.465000in}{1.604444in}}%
\pgfpathlineto{\pgfqpoint{0.465000in}{0.449444in}}%
\pgfpathclose%
\pgfusepath{fill}%
\end{pgfscope}%
\begin{pgfscope}%
\pgfpathrectangle{\pgfqpoint{0.465000in}{0.449444in}}{\pgfqpoint{1.550000in}{1.155000in}}%
\pgfusepath{clip}%
\pgfsetbuttcap%
\pgfsetmiterjoin%
\pgfsetlinewidth{1.003750pt}%
\definecolor{currentstroke}{rgb}{0.000000,0.000000,0.000000}%
\pgfsetstrokecolor{currentstroke}%
\pgfsetdash{}{0pt}%
\pgfpathmoveto{\pgfqpoint{0.455000in}{0.449444in}}%
\pgfpathlineto{\pgfqpoint{0.502805in}{0.449444in}}%
\pgfpathlineto{\pgfqpoint{0.502805in}{1.418751in}}%
\pgfpathlineto{\pgfqpoint{0.455000in}{1.418751in}}%
\pgfusepath{stroke}%
\end{pgfscope}%
\begin{pgfscope}%
\pgfpathrectangle{\pgfqpoint{0.465000in}{0.449444in}}{\pgfqpoint{1.550000in}{1.155000in}}%
\pgfusepath{clip}%
\pgfsetbuttcap%
\pgfsetmiterjoin%
\pgfsetlinewidth{1.003750pt}%
\definecolor{currentstroke}{rgb}{0.000000,0.000000,0.000000}%
\pgfsetstrokecolor{currentstroke}%
\pgfsetdash{}{0pt}%
\pgfpathmoveto{\pgfqpoint{0.593537in}{0.449444in}}%
\pgfpathlineto{\pgfqpoint{0.654025in}{0.449444in}}%
\pgfpathlineto{\pgfqpoint{0.654025in}{1.549444in}}%
\pgfpathlineto{\pgfqpoint{0.593537in}{1.549444in}}%
\pgfpathlineto{\pgfqpoint{0.593537in}{0.449444in}}%
\pgfpathclose%
\pgfusepath{stroke}%
\end{pgfscope}%
\begin{pgfscope}%
\pgfpathrectangle{\pgfqpoint{0.465000in}{0.449444in}}{\pgfqpoint{1.550000in}{1.155000in}}%
\pgfusepath{clip}%
\pgfsetbuttcap%
\pgfsetmiterjoin%
\pgfsetlinewidth{1.003750pt}%
\definecolor{currentstroke}{rgb}{0.000000,0.000000,0.000000}%
\pgfsetstrokecolor{currentstroke}%
\pgfsetdash{}{0pt}%
\pgfpathmoveto{\pgfqpoint{0.744756in}{0.449444in}}%
\pgfpathlineto{\pgfqpoint{0.805244in}{0.449444in}}%
\pgfpathlineto{\pgfqpoint{0.805244in}{1.200929in}}%
\pgfpathlineto{\pgfqpoint{0.744756in}{1.200929in}}%
\pgfpathlineto{\pgfqpoint{0.744756in}{0.449444in}}%
\pgfpathclose%
\pgfusepath{stroke}%
\end{pgfscope}%
\begin{pgfscope}%
\pgfpathrectangle{\pgfqpoint{0.465000in}{0.449444in}}{\pgfqpoint{1.550000in}{1.155000in}}%
\pgfusepath{clip}%
\pgfsetbuttcap%
\pgfsetmiterjoin%
\pgfsetlinewidth{1.003750pt}%
\definecolor{currentstroke}{rgb}{0.000000,0.000000,0.000000}%
\pgfsetstrokecolor{currentstroke}%
\pgfsetdash{}{0pt}%
\pgfpathmoveto{\pgfqpoint{0.895976in}{0.449444in}}%
\pgfpathlineto{\pgfqpoint{0.956464in}{0.449444in}}%
\pgfpathlineto{\pgfqpoint{0.956464in}{1.462315in}}%
\pgfpathlineto{\pgfqpoint{0.895976in}{1.462315in}}%
\pgfpathlineto{\pgfqpoint{0.895976in}{0.449444in}}%
\pgfpathclose%
\pgfusepath{stroke}%
\end{pgfscope}%
\begin{pgfscope}%
\pgfpathrectangle{\pgfqpoint{0.465000in}{0.449444in}}{\pgfqpoint{1.550000in}{1.155000in}}%
\pgfusepath{clip}%
\pgfsetbuttcap%
\pgfsetmiterjoin%
\pgfsetlinewidth{1.003750pt}%
\definecolor{currentstroke}{rgb}{0.000000,0.000000,0.000000}%
\pgfsetstrokecolor{currentstroke}%
\pgfsetdash{}{0pt}%
\pgfpathmoveto{\pgfqpoint{1.047195in}{0.449444in}}%
\pgfpathlineto{\pgfqpoint{1.107683in}{0.449444in}}%
\pgfpathlineto{\pgfqpoint{1.107683in}{1.168256in}}%
\pgfpathlineto{\pgfqpoint{1.047195in}{1.168256in}}%
\pgfpathlineto{\pgfqpoint{1.047195in}{0.449444in}}%
\pgfpathclose%
\pgfusepath{stroke}%
\end{pgfscope}%
\begin{pgfscope}%
\pgfpathrectangle{\pgfqpoint{0.465000in}{0.449444in}}{\pgfqpoint{1.550000in}{1.155000in}}%
\pgfusepath{clip}%
\pgfsetbuttcap%
\pgfsetmiterjoin%
\pgfsetlinewidth{1.003750pt}%
\definecolor{currentstroke}{rgb}{0.000000,0.000000,0.000000}%
\pgfsetstrokecolor{currentstroke}%
\pgfsetdash{}{0pt}%
\pgfpathmoveto{\pgfqpoint{1.198415in}{0.449444in}}%
\pgfpathlineto{\pgfqpoint{1.258903in}{0.449444in}}%
\pgfpathlineto{\pgfqpoint{1.258903in}{1.473207in}}%
\pgfpathlineto{\pgfqpoint{1.198415in}{1.473207in}}%
\pgfpathlineto{\pgfqpoint{1.198415in}{0.449444in}}%
\pgfpathclose%
\pgfusepath{stroke}%
\end{pgfscope}%
\begin{pgfscope}%
\pgfpathrectangle{\pgfqpoint{0.465000in}{0.449444in}}{\pgfqpoint{1.550000in}{1.155000in}}%
\pgfusepath{clip}%
\pgfsetbuttcap%
\pgfsetmiterjoin%
\pgfsetlinewidth{1.003750pt}%
\definecolor{currentstroke}{rgb}{0.000000,0.000000,0.000000}%
\pgfsetstrokecolor{currentstroke}%
\pgfsetdash{}{0pt}%
\pgfpathmoveto{\pgfqpoint{1.349634in}{0.449444in}}%
\pgfpathlineto{\pgfqpoint{1.410122in}{0.449444in}}%
\pgfpathlineto{\pgfqpoint{1.410122in}{1.375187in}}%
\pgfpathlineto{\pgfqpoint{1.349634in}{1.375187in}}%
\pgfpathlineto{\pgfqpoint{1.349634in}{0.449444in}}%
\pgfpathclose%
\pgfusepath{stroke}%
\end{pgfscope}%
\begin{pgfscope}%
\pgfpathrectangle{\pgfqpoint{0.465000in}{0.449444in}}{\pgfqpoint{1.550000in}{1.155000in}}%
\pgfusepath{clip}%
\pgfsetbuttcap%
\pgfsetmiterjoin%
\pgfsetlinewidth{1.003750pt}%
\definecolor{currentstroke}{rgb}{0.000000,0.000000,0.000000}%
\pgfsetstrokecolor{currentstroke}%
\pgfsetdash{}{0pt}%
\pgfpathmoveto{\pgfqpoint{1.500854in}{0.449444in}}%
\pgfpathlineto{\pgfqpoint{1.561342in}{0.449444in}}%
\pgfpathlineto{\pgfqpoint{1.561342in}{1.277167in}}%
\pgfpathlineto{\pgfqpoint{1.500854in}{1.277167in}}%
\pgfpathlineto{\pgfqpoint{1.500854in}{0.449444in}}%
\pgfpathclose%
\pgfusepath{stroke}%
\end{pgfscope}%
\begin{pgfscope}%
\pgfpathrectangle{\pgfqpoint{0.465000in}{0.449444in}}{\pgfqpoint{1.550000in}{1.155000in}}%
\pgfusepath{clip}%
\pgfsetbuttcap%
\pgfsetmiterjoin%
\pgfsetlinewidth{1.003750pt}%
\definecolor{currentstroke}{rgb}{0.000000,0.000000,0.000000}%
\pgfsetstrokecolor{currentstroke}%
\pgfsetdash{}{0pt}%
\pgfpathmoveto{\pgfqpoint{1.652073in}{0.449444in}}%
\pgfpathlineto{\pgfqpoint{1.712561in}{0.449444in}}%
\pgfpathlineto{\pgfqpoint{1.712561in}{1.386078in}}%
\pgfpathlineto{\pgfqpoint{1.652073in}{1.386078in}}%
\pgfpathlineto{\pgfqpoint{1.652073in}{0.449444in}}%
\pgfpathclose%
\pgfusepath{stroke}%
\end{pgfscope}%
\begin{pgfscope}%
\pgfpathrectangle{\pgfqpoint{0.465000in}{0.449444in}}{\pgfqpoint{1.550000in}{1.155000in}}%
\pgfusepath{clip}%
\pgfsetbuttcap%
\pgfsetmiterjoin%
\pgfsetlinewidth{1.003750pt}%
\definecolor{currentstroke}{rgb}{0.000000,0.000000,0.000000}%
\pgfsetstrokecolor{currentstroke}%
\pgfsetdash{}{0pt}%
\pgfpathmoveto{\pgfqpoint{1.803293in}{0.449444in}}%
\pgfpathlineto{\pgfqpoint{1.863781in}{0.449444in}}%
\pgfpathlineto{\pgfqpoint{1.863781in}{1.440533in}}%
\pgfpathlineto{\pgfqpoint{1.803293in}{1.440533in}}%
\pgfpathlineto{\pgfqpoint{1.803293in}{0.449444in}}%
\pgfpathclose%
\pgfusepath{stroke}%
\end{pgfscope}%
\begin{pgfscope}%
\pgfpathrectangle{\pgfqpoint{0.465000in}{0.449444in}}{\pgfqpoint{1.550000in}{1.155000in}}%
\pgfusepath{clip}%
\pgfsetbuttcap%
\pgfsetmiterjoin%
\definecolor{currentfill}{rgb}{0.000000,0.000000,0.000000}%
\pgfsetfillcolor{currentfill}%
\pgfsetlinewidth{0.000000pt}%
\definecolor{currentstroke}{rgb}{0.000000,0.000000,0.000000}%
\pgfsetstrokecolor{currentstroke}%
\pgfsetstrokeopacity{0.000000}%
\pgfsetdash{}{0pt}%
\pgfpathmoveto{\pgfqpoint{0.502805in}{0.449444in}}%
\pgfpathlineto{\pgfqpoint{0.563293in}{0.449444in}}%
\pgfpathlineto{\pgfqpoint{0.563293in}{0.547464in}}%
\pgfpathlineto{\pgfqpoint{0.502805in}{0.547464in}}%
\pgfpathlineto{\pgfqpoint{0.502805in}{0.449444in}}%
\pgfpathclose%
\pgfusepath{fill}%
\end{pgfscope}%
\begin{pgfscope}%
\pgfpathrectangle{\pgfqpoint{0.465000in}{0.449444in}}{\pgfqpoint{1.550000in}{1.155000in}}%
\pgfusepath{clip}%
\pgfsetbuttcap%
\pgfsetmiterjoin%
\definecolor{currentfill}{rgb}{0.000000,0.000000,0.000000}%
\pgfsetfillcolor{currentfill}%
\pgfsetlinewidth{0.000000pt}%
\definecolor{currentstroke}{rgb}{0.000000,0.000000,0.000000}%
\pgfsetstrokecolor{currentstroke}%
\pgfsetstrokeopacity{0.000000}%
\pgfsetdash{}{0pt}%
\pgfpathmoveto{\pgfqpoint{0.654025in}{0.449444in}}%
\pgfpathlineto{\pgfqpoint{0.714512in}{0.449444in}}%
\pgfpathlineto{\pgfqpoint{0.714512in}{0.591028in}}%
\pgfpathlineto{\pgfqpoint{0.654025in}{0.591028in}}%
\pgfpathlineto{\pgfqpoint{0.654025in}{0.449444in}}%
\pgfpathclose%
\pgfusepath{fill}%
\end{pgfscope}%
\begin{pgfscope}%
\pgfpathrectangle{\pgfqpoint{0.465000in}{0.449444in}}{\pgfqpoint{1.550000in}{1.155000in}}%
\pgfusepath{clip}%
\pgfsetbuttcap%
\pgfsetmiterjoin%
\definecolor{currentfill}{rgb}{0.000000,0.000000,0.000000}%
\pgfsetfillcolor{currentfill}%
\pgfsetlinewidth{0.000000pt}%
\definecolor{currentstroke}{rgb}{0.000000,0.000000,0.000000}%
\pgfsetstrokecolor{currentstroke}%
\pgfsetstrokeopacity{0.000000}%
\pgfsetdash{}{0pt}%
\pgfpathmoveto{\pgfqpoint{0.805244in}{0.449444in}}%
\pgfpathlineto{\pgfqpoint{0.865732in}{0.449444in}}%
\pgfpathlineto{\pgfqpoint{0.865732in}{0.591028in}}%
\pgfpathlineto{\pgfqpoint{0.805244in}{0.591028in}}%
\pgfpathlineto{\pgfqpoint{0.805244in}{0.449444in}}%
\pgfpathclose%
\pgfusepath{fill}%
\end{pgfscope}%
\begin{pgfscope}%
\pgfpathrectangle{\pgfqpoint{0.465000in}{0.449444in}}{\pgfqpoint{1.550000in}{1.155000in}}%
\pgfusepath{clip}%
\pgfsetbuttcap%
\pgfsetmiterjoin%
\definecolor{currentfill}{rgb}{0.000000,0.000000,0.000000}%
\pgfsetfillcolor{currentfill}%
\pgfsetlinewidth{0.000000pt}%
\definecolor{currentstroke}{rgb}{0.000000,0.000000,0.000000}%
\pgfsetstrokecolor{currentstroke}%
\pgfsetstrokeopacity{0.000000}%
\pgfsetdash{}{0pt}%
\pgfpathmoveto{\pgfqpoint{0.956464in}{0.449444in}}%
\pgfpathlineto{\pgfqpoint{1.016951in}{0.449444in}}%
\pgfpathlineto{\pgfqpoint{1.016951in}{0.601919in}}%
\pgfpathlineto{\pgfqpoint{0.956464in}{0.601919in}}%
\pgfpathlineto{\pgfqpoint{0.956464in}{0.449444in}}%
\pgfpathclose%
\pgfusepath{fill}%
\end{pgfscope}%
\begin{pgfscope}%
\pgfpathrectangle{\pgfqpoint{0.465000in}{0.449444in}}{\pgfqpoint{1.550000in}{1.155000in}}%
\pgfusepath{clip}%
\pgfsetbuttcap%
\pgfsetmiterjoin%
\definecolor{currentfill}{rgb}{0.000000,0.000000,0.000000}%
\pgfsetfillcolor{currentfill}%
\pgfsetlinewidth{0.000000pt}%
\definecolor{currentstroke}{rgb}{0.000000,0.000000,0.000000}%
\pgfsetstrokecolor{currentstroke}%
\pgfsetstrokeopacity{0.000000}%
\pgfsetdash{}{0pt}%
\pgfpathmoveto{\pgfqpoint{1.107683in}{0.449444in}}%
\pgfpathlineto{\pgfqpoint{1.168171in}{0.449444in}}%
\pgfpathlineto{\pgfqpoint{1.168171in}{0.623702in}}%
\pgfpathlineto{\pgfqpoint{1.107683in}{0.623702in}}%
\pgfpathlineto{\pgfqpoint{1.107683in}{0.449444in}}%
\pgfpathclose%
\pgfusepath{fill}%
\end{pgfscope}%
\begin{pgfscope}%
\pgfpathrectangle{\pgfqpoint{0.465000in}{0.449444in}}{\pgfqpoint{1.550000in}{1.155000in}}%
\pgfusepath{clip}%
\pgfsetbuttcap%
\pgfsetmiterjoin%
\definecolor{currentfill}{rgb}{0.000000,0.000000,0.000000}%
\pgfsetfillcolor{currentfill}%
\pgfsetlinewidth{0.000000pt}%
\definecolor{currentstroke}{rgb}{0.000000,0.000000,0.000000}%
\pgfsetstrokecolor{currentstroke}%
\pgfsetstrokeopacity{0.000000}%
\pgfsetdash{}{0pt}%
\pgfpathmoveto{\pgfqpoint{1.258903in}{0.449444in}}%
\pgfpathlineto{\pgfqpoint{1.319391in}{0.449444in}}%
\pgfpathlineto{\pgfqpoint{1.319391in}{0.623702in}}%
\pgfpathlineto{\pgfqpoint{1.258903in}{0.623702in}}%
\pgfpathlineto{\pgfqpoint{1.258903in}{0.449444in}}%
\pgfpathclose%
\pgfusepath{fill}%
\end{pgfscope}%
\begin{pgfscope}%
\pgfpathrectangle{\pgfqpoint{0.465000in}{0.449444in}}{\pgfqpoint{1.550000in}{1.155000in}}%
\pgfusepath{clip}%
\pgfsetbuttcap%
\pgfsetmiterjoin%
\definecolor{currentfill}{rgb}{0.000000,0.000000,0.000000}%
\pgfsetfillcolor{currentfill}%
\pgfsetlinewidth{0.000000pt}%
\definecolor{currentstroke}{rgb}{0.000000,0.000000,0.000000}%
\pgfsetstrokecolor{currentstroke}%
\pgfsetstrokeopacity{0.000000}%
\pgfsetdash{}{0pt}%
\pgfpathmoveto{\pgfqpoint{1.410122in}{0.449444in}}%
\pgfpathlineto{\pgfqpoint{1.470610in}{0.449444in}}%
\pgfpathlineto{\pgfqpoint{1.470610in}{0.547464in}}%
\pgfpathlineto{\pgfqpoint{1.410122in}{0.547464in}}%
\pgfpathlineto{\pgfqpoint{1.410122in}{0.449444in}}%
\pgfpathclose%
\pgfusepath{fill}%
\end{pgfscope}%
\begin{pgfscope}%
\pgfpathrectangle{\pgfqpoint{0.465000in}{0.449444in}}{\pgfqpoint{1.550000in}{1.155000in}}%
\pgfusepath{clip}%
\pgfsetbuttcap%
\pgfsetmiterjoin%
\definecolor{currentfill}{rgb}{0.000000,0.000000,0.000000}%
\pgfsetfillcolor{currentfill}%
\pgfsetlinewidth{0.000000pt}%
\definecolor{currentstroke}{rgb}{0.000000,0.000000,0.000000}%
\pgfsetstrokecolor{currentstroke}%
\pgfsetstrokeopacity{0.000000}%
\pgfsetdash{}{0pt}%
\pgfpathmoveto{\pgfqpoint{1.561342in}{0.449444in}}%
\pgfpathlineto{\pgfqpoint{1.621830in}{0.449444in}}%
\pgfpathlineto{\pgfqpoint{1.621830in}{0.721721in}}%
\pgfpathlineto{\pgfqpoint{1.561342in}{0.721721in}}%
\pgfpathlineto{\pgfqpoint{1.561342in}{0.449444in}}%
\pgfpathclose%
\pgfusepath{fill}%
\end{pgfscope}%
\begin{pgfscope}%
\pgfpathrectangle{\pgfqpoint{0.465000in}{0.449444in}}{\pgfqpoint{1.550000in}{1.155000in}}%
\pgfusepath{clip}%
\pgfsetbuttcap%
\pgfsetmiterjoin%
\definecolor{currentfill}{rgb}{0.000000,0.000000,0.000000}%
\pgfsetfillcolor{currentfill}%
\pgfsetlinewidth{0.000000pt}%
\definecolor{currentstroke}{rgb}{0.000000,0.000000,0.000000}%
\pgfsetstrokecolor{currentstroke}%
\pgfsetstrokeopacity{0.000000}%
\pgfsetdash{}{0pt}%
\pgfpathmoveto{\pgfqpoint{1.712561in}{0.449444in}}%
\pgfpathlineto{\pgfqpoint{1.773049in}{0.449444in}}%
\pgfpathlineto{\pgfqpoint{1.773049in}{0.623702in}}%
\pgfpathlineto{\pgfqpoint{1.712561in}{0.623702in}}%
\pgfpathlineto{\pgfqpoint{1.712561in}{0.449444in}}%
\pgfpathclose%
\pgfusepath{fill}%
\end{pgfscope}%
\begin{pgfscope}%
\pgfpathrectangle{\pgfqpoint{0.465000in}{0.449444in}}{\pgfqpoint{1.550000in}{1.155000in}}%
\pgfusepath{clip}%
\pgfsetbuttcap%
\pgfsetmiterjoin%
\definecolor{currentfill}{rgb}{0.000000,0.000000,0.000000}%
\pgfsetfillcolor{currentfill}%
\pgfsetlinewidth{0.000000pt}%
\definecolor{currentstroke}{rgb}{0.000000,0.000000,0.000000}%
\pgfsetstrokecolor{currentstroke}%
\pgfsetstrokeopacity{0.000000}%
\pgfsetdash{}{0pt}%
\pgfpathmoveto{\pgfqpoint{1.863781in}{0.449444in}}%
\pgfpathlineto{\pgfqpoint{1.924269in}{0.449444in}}%
\pgfpathlineto{\pgfqpoint{1.924269in}{0.656375in}}%
\pgfpathlineto{\pgfqpoint{1.863781in}{0.656375in}}%
\pgfpathlineto{\pgfqpoint{1.863781in}{0.449444in}}%
\pgfpathclose%
\pgfusepath{fill}%
\end{pgfscope}%
\begin{pgfscope}%
\pgfsetbuttcap%
\pgfsetroundjoin%
\definecolor{currentfill}{rgb}{0.000000,0.000000,0.000000}%
\pgfsetfillcolor{currentfill}%
\pgfsetlinewidth{0.803000pt}%
\definecolor{currentstroke}{rgb}{0.000000,0.000000,0.000000}%
\pgfsetstrokecolor{currentstroke}%
\pgfsetdash{}{0pt}%
\pgfsys@defobject{currentmarker}{\pgfqpoint{0.000000in}{-0.048611in}}{\pgfqpoint{0.000000in}{0.000000in}}{%
\pgfpathmoveto{\pgfqpoint{0.000000in}{0.000000in}}%
\pgfpathlineto{\pgfqpoint{0.000000in}{-0.048611in}}%
\pgfusepath{stroke,fill}%
}%
\begin{pgfscope}%
\pgfsys@transformshift{0.502805in}{0.449444in}%
\pgfsys@useobject{currentmarker}{}%
\end{pgfscope}%
\end{pgfscope}%
\begin{pgfscope}%
\definecolor{textcolor}{rgb}{0.000000,0.000000,0.000000}%
\pgfsetstrokecolor{textcolor}%
\pgfsetfillcolor{textcolor}%
\pgftext[x=0.502805in,y=0.352222in,,top]{\color{textcolor}\rmfamily\fontsize{10.000000}{12.000000}\selectfont 0.0}%
\end{pgfscope}%
\begin{pgfscope}%
\pgfsetbuttcap%
\pgfsetroundjoin%
\definecolor{currentfill}{rgb}{0.000000,0.000000,0.000000}%
\pgfsetfillcolor{currentfill}%
\pgfsetlinewidth{0.803000pt}%
\definecolor{currentstroke}{rgb}{0.000000,0.000000,0.000000}%
\pgfsetstrokecolor{currentstroke}%
\pgfsetdash{}{0pt}%
\pgfsys@defobject{currentmarker}{\pgfqpoint{0.000000in}{-0.048611in}}{\pgfqpoint{0.000000in}{0.000000in}}{%
\pgfpathmoveto{\pgfqpoint{0.000000in}{0.000000in}}%
\pgfpathlineto{\pgfqpoint{0.000000in}{-0.048611in}}%
\pgfusepath{stroke,fill}%
}%
\begin{pgfscope}%
\pgfsys@transformshift{0.880854in}{0.449444in}%
\pgfsys@useobject{currentmarker}{}%
\end{pgfscope}%
\end{pgfscope}%
\begin{pgfscope}%
\definecolor{textcolor}{rgb}{0.000000,0.000000,0.000000}%
\pgfsetstrokecolor{textcolor}%
\pgfsetfillcolor{textcolor}%
\pgftext[x=0.880854in,y=0.352222in,,top]{\color{textcolor}\rmfamily\fontsize{10.000000}{12.000000}\selectfont 0.25}%
\end{pgfscope}%
\begin{pgfscope}%
\pgfsetbuttcap%
\pgfsetroundjoin%
\definecolor{currentfill}{rgb}{0.000000,0.000000,0.000000}%
\pgfsetfillcolor{currentfill}%
\pgfsetlinewidth{0.803000pt}%
\definecolor{currentstroke}{rgb}{0.000000,0.000000,0.000000}%
\pgfsetstrokecolor{currentstroke}%
\pgfsetdash{}{0pt}%
\pgfsys@defobject{currentmarker}{\pgfqpoint{0.000000in}{-0.048611in}}{\pgfqpoint{0.000000in}{0.000000in}}{%
\pgfpathmoveto{\pgfqpoint{0.000000in}{0.000000in}}%
\pgfpathlineto{\pgfqpoint{0.000000in}{-0.048611in}}%
\pgfusepath{stroke,fill}%
}%
\begin{pgfscope}%
\pgfsys@transformshift{1.258903in}{0.449444in}%
\pgfsys@useobject{currentmarker}{}%
\end{pgfscope}%
\end{pgfscope}%
\begin{pgfscope}%
\definecolor{textcolor}{rgb}{0.000000,0.000000,0.000000}%
\pgfsetstrokecolor{textcolor}%
\pgfsetfillcolor{textcolor}%
\pgftext[x=1.258903in,y=0.352222in,,top]{\color{textcolor}\rmfamily\fontsize{10.000000}{12.000000}\selectfont 0.5}%
\end{pgfscope}%
\begin{pgfscope}%
\pgfsetbuttcap%
\pgfsetroundjoin%
\definecolor{currentfill}{rgb}{0.000000,0.000000,0.000000}%
\pgfsetfillcolor{currentfill}%
\pgfsetlinewidth{0.803000pt}%
\definecolor{currentstroke}{rgb}{0.000000,0.000000,0.000000}%
\pgfsetstrokecolor{currentstroke}%
\pgfsetdash{}{0pt}%
\pgfsys@defobject{currentmarker}{\pgfqpoint{0.000000in}{-0.048611in}}{\pgfqpoint{0.000000in}{0.000000in}}{%
\pgfpathmoveto{\pgfqpoint{0.000000in}{0.000000in}}%
\pgfpathlineto{\pgfqpoint{0.000000in}{-0.048611in}}%
\pgfusepath{stroke,fill}%
}%
\begin{pgfscope}%
\pgfsys@transformshift{1.636951in}{0.449444in}%
\pgfsys@useobject{currentmarker}{}%
\end{pgfscope}%
\end{pgfscope}%
\begin{pgfscope}%
\definecolor{textcolor}{rgb}{0.000000,0.000000,0.000000}%
\pgfsetstrokecolor{textcolor}%
\pgfsetfillcolor{textcolor}%
\pgftext[x=1.636951in,y=0.352222in,,top]{\color{textcolor}\rmfamily\fontsize{10.000000}{12.000000}\selectfont 0.75}%
\end{pgfscope}%
\begin{pgfscope}%
\pgfsetbuttcap%
\pgfsetroundjoin%
\definecolor{currentfill}{rgb}{0.000000,0.000000,0.000000}%
\pgfsetfillcolor{currentfill}%
\pgfsetlinewidth{0.803000pt}%
\definecolor{currentstroke}{rgb}{0.000000,0.000000,0.000000}%
\pgfsetstrokecolor{currentstroke}%
\pgfsetdash{}{0pt}%
\pgfsys@defobject{currentmarker}{\pgfqpoint{0.000000in}{-0.048611in}}{\pgfqpoint{0.000000in}{0.000000in}}{%
\pgfpathmoveto{\pgfqpoint{0.000000in}{0.000000in}}%
\pgfpathlineto{\pgfqpoint{0.000000in}{-0.048611in}}%
\pgfusepath{stroke,fill}%
}%
\begin{pgfscope}%
\pgfsys@transformshift{2.015000in}{0.449444in}%
\pgfsys@useobject{currentmarker}{}%
\end{pgfscope}%
\end{pgfscope}%
\begin{pgfscope}%
\definecolor{textcolor}{rgb}{0.000000,0.000000,0.000000}%
\pgfsetstrokecolor{textcolor}%
\pgfsetfillcolor{textcolor}%
\pgftext[x=2.015000in,y=0.352222in,,top]{\color{textcolor}\rmfamily\fontsize{10.000000}{12.000000}\selectfont 1.0}%
\end{pgfscope}%
\begin{pgfscope}%
\definecolor{textcolor}{rgb}{0.000000,0.000000,0.000000}%
\pgfsetstrokecolor{textcolor}%
\pgfsetfillcolor{textcolor}%
\pgftext[x=1.240000in,y=0.173333in,,top]{\color{textcolor}\rmfamily\fontsize{10.000000}{12.000000}\selectfont \(\displaystyle p\)}%
\end{pgfscope}%
\begin{pgfscope}%
\pgfsetbuttcap%
\pgfsetroundjoin%
\definecolor{currentfill}{rgb}{0.000000,0.000000,0.000000}%
\pgfsetfillcolor{currentfill}%
\pgfsetlinewidth{0.803000pt}%
\definecolor{currentstroke}{rgb}{0.000000,0.000000,0.000000}%
\pgfsetstrokecolor{currentstroke}%
\pgfsetdash{}{0pt}%
\pgfsys@defobject{currentmarker}{\pgfqpoint{-0.048611in}{0.000000in}}{\pgfqpoint{-0.000000in}{0.000000in}}{%
\pgfpathmoveto{\pgfqpoint{-0.000000in}{0.000000in}}%
\pgfpathlineto{\pgfqpoint{-0.048611in}{0.000000in}}%
\pgfusepath{stroke,fill}%
}%
\begin{pgfscope}%
\pgfsys@transformshift{0.465000in}{0.449444in}%
\pgfsys@useobject{currentmarker}{}%
\end{pgfscope}%
\end{pgfscope}%
\begin{pgfscope}%
\definecolor{textcolor}{rgb}{0.000000,0.000000,0.000000}%
\pgfsetstrokecolor{textcolor}%
\pgfsetfillcolor{textcolor}%
\pgftext[x=0.298333in, y=0.401250in, left, base]{\color{textcolor}\rmfamily\fontsize{10.000000}{12.000000}\selectfont \(\displaystyle {0}\)}%
\end{pgfscope}%
\begin{pgfscope}%
\pgfsetbuttcap%
\pgfsetroundjoin%
\definecolor{currentfill}{rgb}{0.000000,0.000000,0.000000}%
\pgfsetfillcolor{currentfill}%
\pgfsetlinewidth{0.803000pt}%
\definecolor{currentstroke}{rgb}{0.000000,0.000000,0.000000}%
\pgfsetstrokecolor{currentstroke}%
\pgfsetdash{}{0pt}%
\pgfsys@defobject{currentmarker}{\pgfqpoint{-0.048611in}{0.000000in}}{\pgfqpoint{-0.000000in}{0.000000in}}{%
\pgfpathmoveto{\pgfqpoint{-0.000000in}{0.000000in}}%
\pgfpathlineto{\pgfqpoint{-0.048611in}{0.000000in}}%
\pgfusepath{stroke,fill}%
}%
\begin{pgfscope}%
\pgfsys@transformshift{0.465000in}{0.993999in}%
\pgfsys@useobject{currentmarker}{}%
\end{pgfscope}%
\end{pgfscope}%
\begin{pgfscope}%
\definecolor{textcolor}{rgb}{0.000000,0.000000,0.000000}%
\pgfsetstrokecolor{textcolor}%
\pgfsetfillcolor{textcolor}%
\pgftext[x=0.298333in, y=0.945804in, left, base]{\color{textcolor}\rmfamily\fontsize{10.000000}{12.000000}\selectfont \(\displaystyle {5}\)}%
\end{pgfscope}%
\begin{pgfscope}%
\pgfsetbuttcap%
\pgfsetroundjoin%
\definecolor{currentfill}{rgb}{0.000000,0.000000,0.000000}%
\pgfsetfillcolor{currentfill}%
\pgfsetlinewidth{0.803000pt}%
\definecolor{currentstroke}{rgb}{0.000000,0.000000,0.000000}%
\pgfsetstrokecolor{currentstroke}%
\pgfsetdash{}{0pt}%
\pgfsys@defobject{currentmarker}{\pgfqpoint{-0.048611in}{0.000000in}}{\pgfqpoint{-0.000000in}{0.000000in}}{%
\pgfpathmoveto{\pgfqpoint{-0.000000in}{0.000000in}}%
\pgfpathlineto{\pgfqpoint{-0.048611in}{0.000000in}}%
\pgfusepath{stroke,fill}%
}%
\begin{pgfscope}%
\pgfsys@transformshift{0.465000in}{1.538553in}%
\pgfsys@useobject{currentmarker}{}%
\end{pgfscope}%
\end{pgfscope}%
\begin{pgfscope}%
\definecolor{textcolor}{rgb}{0.000000,0.000000,0.000000}%
\pgfsetstrokecolor{textcolor}%
\pgfsetfillcolor{textcolor}%
\pgftext[x=0.228889in, y=1.490359in, left, base]{\color{textcolor}\rmfamily\fontsize{10.000000}{12.000000}\selectfont \(\displaystyle {10}\)}%
\end{pgfscope}%
\begin{pgfscope}%
\definecolor{textcolor}{rgb}{0.000000,0.000000,0.000000}%
\pgfsetstrokecolor{textcolor}%
\pgfsetfillcolor{textcolor}%
\pgftext[x=0.173333in,y=1.026944in,,bottom,rotate=90.000000]{\color{textcolor}\rmfamily\fontsize{10.000000}{12.000000}\selectfont Percent of Data Set}%
\end{pgfscope}%
\begin{pgfscope}%
\pgfsetrectcap%
\pgfsetmiterjoin%
\pgfsetlinewidth{0.803000pt}%
\definecolor{currentstroke}{rgb}{0.000000,0.000000,0.000000}%
\pgfsetstrokecolor{currentstroke}%
\pgfsetdash{}{0pt}%
\pgfpathmoveto{\pgfqpoint{0.465000in}{0.449444in}}%
\pgfpathlineto{\pgfqpoint{0.465000in}{1.604444in}}%
\pgfusepath{stroke}%
\end{pgfscope}%
\begin{pgfscope}%
\pgfsetrectcap%
\pgfsetmiterjoin%
\pgfsetlinewidth{0.803000pt}%
\definecolor{currentstroke}{rgb}{0.000000,0.000000,0.000000}%
\pgfsetstrokecolor{currentstroke}%
\pgfsetdash{}{0pt}%
\pgfpathmoveto{\pgfqpoint{2.015000in}{0.449444in}}%
\pgfpathlineto{\pgfqpoint{2.015000in}{1.604444in}}%
\pgfusepath{stroke}%
\end{pgfscope}%
\begin{pgfscope}%
\pgfsetrectcap%
\pgfsetmiterjoin%
\pgfsetlinewidth{0.803000pt}%
\definecolor{currentstroke}{rgb}{0.000000,0.000000,0.000000}%
\pgfsetstrokecolor{currentstroke}%
\pgfsetdash{}{0pt}%
\pgfpathmoveto{\pgfqpoint{0.465000in}{0.449444in}}%
\pgfpathlineto{\pgfqpoint{2.015000in}{0.449444in}}%
\pgfusepath{stroke}%
\end{pgfscope}%
\begin{pgfscope}%
\pgfsetrectcap%
\pgfsetmiterjoin%
\pgfsetlinewidth{0.803000pt}%
\definecolor{currentstroke}{rgb}{0.000000,0.000000,0.000000}%
\pgfsetstrokecolor{currentstroke}%
\pgfsetdash{}{0pt}%
\pgfpathmoveto{\pgfqpoint{0.465000in}{1.604444in}}%
\pgfpathlineto{\pgfqpoint{2.015000in}{1.604444in}}%
\pgfusepath{stroke}%
\end{pgfscope}%
\begin{pgfscope}%
\pgfsetbuttcap%
\pgfsetmiterjoin%
\definecolor{currentfill}{rgb}{1.000000,1.000000,1.000000}%
\pgfsetfillcolor{currentfill}%
\pgfsetfillopacity{0.800000}%
\pgfsetlinewidth{1.003750pt}%
\definecolor{currentstroke}{rgb}{0.800000,0.800000,0.800000}%
\pgfsetstrokecolor{currentstroke}%
\pgfsetstrokeopacity{0.800000}%
\pgfsetdash{}{0pt}%
\pgfpathmoveto{\pgfqpoint{1.238056in}{1.104445in}}%
\pgfpathlineto{\pgfqpoint{1.917778in}{1.104445in}}%
\pgfpathquadraticcurveto{\pgfqpoint{1.945556in}{1.104445in}}{\pgfqpoint{1.945556in}{1.132222in}}%
\pgfpathlineto{\pgfqpoint{1.945556in}{1.507222in}}%
\pgfpathquadraticcurveto{\pgfqpoint{1.945556in}{1.535000in}}{\pgfqpoint{1.917778in}{1.535000in}}%
\pgfpathlineto{\pgfqpoint{1.238056in}{1.535000in}}%
\pgfpathquadraticcurveto{\pgfqpoint{1.210278in}{1.535000in}}{\pgfqpoint{1.210278in}{1.507222in}}%
\pgfpathlineto{\pgfqpoint{1.210278in}{1.132222in}}%
\pgfpathquadraticcurveto{\pgfqpoint{1.210278in}{1.104445in}}{\pgfqpoint{1.238056in}{1.104445in}}%
\pgfpathlineto{\pgfqpoint{1.238056in}{1.104445in}}%
\pgfpathclose%
\pgfusepath{stroke,fill}%
\end{pgfscope}%
\begin{pgfscope}%
\pgfsetbuttcap%
\pgfsetmiterjoin%
\pgfsetlinewidth{1.003750pt}%
\definecolor{currentstroke}{rgb}{0.000000,0.000000,0.000000}%
\pgfsetstrokecolor{currentstroke}%
\pgfsetdash{}{0pt}%
\pgfpathmoveto{\pgfqpoint{1.265834in}{1.382222in}}%
\pgfpathlineto{\pgfqpoint{1.543611in}{1.382222in}}%
\pgfpathlineto{\pgfqpoint{1.543611in}{1.479444in}}%
\pgfpathlineto{\pgfqpoint{1.265834in}{1.479444in}}%
\pgfpathlineto{\pgfqpoint{1.265834in}{1.382222in}}%
\pgfpathclose%
\pgfusepath{stroke}%
\end{pgfscope}%
\begin{pgfscope}%
\definecolor{textcolor}{rgb}{0.000000,0.000000,0.000000}%
\pgfsetstrokecolor{textcolor}%
\pgfsetfillcolor{textcolor}%
\pgftext[x=1.654722in,y=1.382222in,left,base]{\color{textcolor}\rmfamily\fontsize{10.000000}{12.000000}\selectfont Neg}%
\end{pgfscope}%
\begin{pgfscope}%
\pgfsetbuttcap%
\pgfsetmiterjoin%
\definecolor{currentfill}{rgb}{0.000000,0.000000,0.000000}%
\pgfsetfillcolor{currentfill}%
\pgfsetlinewidth{0.000000pt}%
\definecolor{currentstroke}{rgb}{0.000000,0.000000,0.000000}%
\pgfsetstrokecolor{currentstroke}%
\pgfsetstrokeopacity{0.000000}%
\pgfsetdash{}{0pt}%
\pgfpathmoveto{\pgfqpoint{1.265834in}{1.186944in}}%
\pgfpathlineto{\pgfqpoint{1.543611in}{1.186944in}}%
\pgfpathlineto{\pgfqpoint{1.543611in}{1.284167in}}%
\pgfpathlineto{\pgfqpoint{1.265834in}{1.284167in}}%
\pgfpathlineto{\pgfqpoint{1.265834in}{1.186944in}}%
\pgfpathclose%
\pgfusepath{fill}%
\end{pgfscope}%
\begin{pgfscope}%
\definecolor{textcolor}{rgb}{0.000000,0.000000,0.000000}%
\pgfsetstrokecolor{textcolor}%
\pgfsetfillcolor{textcolor}%
\pgftext[x=1.654722in,y=1.186944in,left,base]{\color{textcolor}\rmfamily\fontsize{10.000000}{12.000000}\selectfont Pos}%
\end{pgfscope}%
\end{pgfpicture}%
\makeatother%
\endgroup%

&
	\vskip 0pt
	\qquad \qquad ROC Curve
	
	%% Creator: Matplotlib, PGF backend
%%
%% To include the figure in your LaTeX document, write
%%   \input{<filename>.pgf}
%%
%% Make sure the required packages are loaded in your preamble
%%   \usepackage{pgf}
%%
%% Also ensure that all the required font packages are loaded; for instance,
%% the lmodern package is sometimes necessary when using math font.
%%   \usepackage{lmodern}
%%
%% Figures using additional raster images can only be included by \input if
%% they are in the same directory as the main LaTeX file. For loading figures
%% from other directories you can use the `import` package
%%   \usepackage{import}
%%
%% and then include the figures with
%%   \import{<path to file>}{<filename>.pgf}
%%
%% Matplotlib used the following preamble
%%   
%%   \usepackage{fontspec}
%%   \makeatletter\@ifpackageloaded{underscore}{}{\usepackage[strings]{underscore}}\makeatother
%%
\begingroup%
\makeatletter%
\begin{pgfpicture}%
\pgfpathrectangle{\pgfpointorigin}{\pgfqpoint{2.121861in}{1.654444in}}%
\pgfusepath{use as bounding box, clip}%
\begin{pgfscope}%
\pgfsetbuttcap%
\pgfsetmiterjoin%
\definecolor{currentfill}{rgb}{1.000000,1.000000,1.000000}%
\pgfsetfillcolor{currentfill}%
\pgfsetlinewidth{0.000000pt}%
\definecolor{currentstroke}{rgb}{1.000000,1.000000,1.000000}%
\pgfsetstrokecolor{currentstroke}%
\pgfsetdash{}{0pt}%
\pgfpathmoveto{\pgfqpoint{0.000000in}{0.000000in}}%
\pgfpathlineto{\pgfqpoint{2.121861in}{0.000000in}}%
\pgfpathlineto{\pgfqpoint{2.121861in}{1.654444in}}%
\pgfpathlineto{\pgfqpoint{0.000000in}{1.654444in}}%
\pgfpathlineto{\pgfqpoint{0.000000in}{0.000000in}}%
\pgfpathclose%
\pgfusepath{fill}%
\end{pgfscope}%
\begin{pgfscope}%
\pgfsetbuttcap%
\pgfsetmiterjoin%
\definecolor{currentfill}{rgb}{1.000000,1.000000,1.000000}%
\pgfsetfillcolor{currentfill}%
\pgfsetlinewidth{0.000000pt}%
\definecolor{currentstroke}{rgb}{0.000000,0.000000,0.000000}%
\pgfsetstrokecolor{currentstroke}%
\pgfsetstrokeopacity{0.000000}%
\pgfsetdash{}{0pt}%
\pgfpathmoveto{\pgfqpoint{0.503581in}{0.449444in}}%
\pgfpathlineto{\pgfqpoint{2.053581in}{0.449444in}}%
\pgfpathlineto{\pgfqpoint{2.053581in}{1.604444in}}%
\pgfpathlineto{\pgfqpoint{0.503581in}{1.604444in}}%
\pgfpathlineto{\pgfqpoint{0.503581in}{0.449444in}}%
\pgfpathclose%
\pgfusepath{fill}%
\end{pgfscope}%
\begin{pgfscope}%
\pgfsetbuttcap%
\pgfsetroundjoin%
\definecolor{currentfill}{rgb}{0.000000,0.000000,0.000000}%
\pgfsetfillcolor{currentfill}%
\pgfsetlinewidth{0.803000pt}%
\definecolor{currentstroke}{rgb}{0.000000,0.000000,0.000000}%
\pgfsetstrokecolor{currentstroke}%
\pgfsetdash{}{0pt}%
\pgfsys@defobject{currentmarker}{\pgfqpoint{0.000000in}{-0.048611in}}{\pgfqpoint{0.000000in}{0.000000in}}{%
\pgfpathmoveto{\pgfqpoint{0.000000in}{0.000000in}}%
\pgfpathlineto{\pgfqpoint{0.000000in}{-0.048611in}}%
\pgfusepath{stroke,fill}%
}%
\begin{pgfscope}%
\pgfsys@transformshift{0.574035in}{0.449444in}%
\pgfsys@useobject{currentmarker}{}%
\end{pgfscope}%
\end{pgfscope}%
\begin{pgfscope}%
\definecolor{textcolor}{rgb}{0.000000,0.000000,0.000000}%
\pgfsetstrokecolor{textcolor}%
\pgfsetfillcolor{textcolor}%
\pgftext[x=0.574035in,y=0.352222in,,top]{\color{textcolor}\rmfamily\fontsize{10.000000}{12.000000}\selectfont \(\displaystyle {0.0}\)}%
\end{pgfscope}%
\begin{pgfscope}%
\pgfsetbuttcap%
\pgfsetroundjoin%
\definecolor{currentfill}{rgb}{0.000000,0.000000,0.000000}%
\pgfsetfillcolor{currentfill}%
\pgfsetlinewidth{0.803000pt}%
\definecolor{currentstroke}{rgb}{0.000000,0.000000,0.000000}%
\pgfsetstrokecolor{currentstroke}%
\pgfsetdash{}{0pt}%
\pgfsys@defobject{currentmarker}{\pgfqpoint{0.000000in}{-0.048611in}}{\pgfqpoint{0.000000in}{0.000000in}}{%
\pgfpathmoveto{\pgfqpoint{0.000000in}{0.000000in}}%
\pgfpathlineto{\pgfqpoint{0.000000in}{-0.048611in}}%
\pgfusepath{stroke,fill}%
}%
\begin{pgfscope}%
\pgfsys@transformshift{1.278581in}{0.449444in}%
\pgfsys@useobject{currentmarker}{}%
\end{pgfscope}%
\end{pgfscope}%
\begin{pgfscope}%
\definecolor{textcolor}{rgb}{0.000000,0.000000,0.000000}%
\pgfsetstrokecolor{textcolor}%
\pgfsetfillcolor{textcolor}%
\pgftext[x=1.278581in,y=0.352222in,,top]{\color{textcolor}\rmfamily\fontsize{10.000000}{12.000000}\selectfont \(\displaystyle {0.5}\)}%
\end{pgfscope}%
\begin{pgfscope}%
\pgfsetbuttcap%
\pgfsetroundjoin%
\definecolor{currentfill}{rgb}{0.000000,0.000000,0.000000}%
\pgfsetfillcolor{currentfill}%
\pgfsetlinewidth{0.803000pt}%
\definecolor{currentstroke}{rgb}{0.000000,0.000000,0.000000}%
\pgfsetstrokecolor{currentstroke}%
\pgfsetdash{}{0pt}%
\pgfsys@defobject{currentmarker}{\pgfqpoint{0.000000in}{-0.048611in}}{\pgfqpoint{0.000000in}{0.000000in}}{%
\pgfpathmoveto{\pgfqpoint{0.000000in}{0.000000in}}%
\pgfpathlineto{\pgfqpoint{0.000000in}{-0.048611in}}%
\pgfusepath{stroke,fill}%
}%
\begin{pgfscope}%
\pgfsys@transformshift{1.983126in}{0.449444in}%
\pgfsys@useobject{currentmarker}{}%
\end{pgfscope}%
\end{pgfscope}%
\begin{pgfscope}%
\definecolor{textcolor}{rgb}{0.000000,0.000000,0.000000}%
\pgfsetstrokecolor{textcolor}%
\pgfsetfillcolor{textcolor}%
\pgftext[x=1.983126in,y=0.352222in,,top]{\color{textcolor}\rmfamily\fontsize{10.000000}{12.000000}\selectfont \(\displaystyle {1.0}\)}%
\end{pgfscope}%
\begin{pgfscope}%
\definecolor{textcolor}{rgb}{0.000000,0.000000,0.000000}%
\pgfsetstrokecolor{textcolor}%
\pgfsetfillcolor{textcolor}%
\pgftext[x=1.278581in,y=0.173333in,,top]{\color{textcolor}\rmfamily\fontsize{10.000000}{12.000000}\selectfont False positive rate}%
\end{pgfscope}%
\begin{pgfscope}%
\pgfsetbuttcap%
\pgfsetroundjoin%
\definecolor{currentfill}{rgb}{0.000000,0.000000,0.000000}%
\pgfsetfillcolor{currentfill}%
\pgfsetlinewidth{0.803000pt}%
\definecolor{currentstroke}{rgb}{0.000000,0.000000,0.000000}%
\pgfsetstrokecolor{currentstroke}%
\pgfsetdash{}{0pt}%
\pgfsys@defobject{currentmarker}{\pgfqpoint{-0.048611in}{0.000000in}}{\pgfqpoint{-0.000000in}{0.000000in}}{%
\pgfpathmoveto{\pgfqpoint{-0.000000in}{0.000000in}}%
\pgfpathlineto{\pgfqpoint{-0.048611in}{0.000000in}}%
\pgfusepath{stroke,fill}%
}%
\begin{pgfscope}%
\pgfsys@transformshift{0.503581in}{0.501944in}%
\pgfsys@useobject{currentmarker}{}%
\end{pgfscope}%
\end{pgfscope}%
\begin{pgfscope}%
\definecolor{textcolor}{rgb}{0.000000,0.000000,0.000000}%
\pgfsetstrokecolor{textcolor}%
\pgfsetfillcolor{textcolor}%
\pgftext[x=0.228889in, y=0.453750in, left, base]{\color{textcolor}\rmfamily\fontsize{10.000000}{12.000000}\selectfont \(\displaystyle {0.0}\)}%
\end{pgfscope}%
\begin{pgfscope}%
\pgfsetbuttcap%
\pgfsetroundjoin%
\definecolor{currentfill}{rgb}{0.000000,0.000000,0.000000}%
\pgfsetfillcolor{currentfill}%
\pgfsetlinewidth{0.803000pt}%
\definecolor{currentstroke}{rgb}{0.000000,0.000000,0.000000}%
\pgfsetstrokecolor{currentstroke}%
\pgfsetdash{}{0pt}%
\pgfsys@defobject{currentmarker}{\pgfqpoint{-0.048611in}{0.000000in}}{\pgfqpoint{-0.000000in}{0.000000in}}{%
\pgfpathmoveto{\pgfqpoint{-0.000000in}{0.000000in}}%
\pgfpathlineto{\pgfqpoint{-0.048611in}{0.000000in}}%
\pgfusepath{stroke,fill}%
}%
\begin{pgfscope}%
\pgfsys@transformshift{0.503581in}{1.026944in}%
\pgfsys@useobject{currentmarker}{}%
\end{pgfscope}%
\end{pgfscope}%
\begin{pgfscope}%
\definecolor{textcolor}{rgb}{0.000000,0.000000,0.000000}%
\pgfsetstrokecolor{textcolor}%
\pgfsetfillcolor{textcolor}%
\pgftext[x=0.228889in, y=0.978750in, left, base]{\color{textcolor}\rmfamily\fontsize{10.000000}{12.000000}\selectfont \(\displaystyle {0.5}\)}%
\end{pgfscope}%
\begin{pgfscope}%
\pgfsetbuttcap%
\pgfsetroundjoin%
\definecolor{currentfill}{rgb}{0.000000,0.000000,0.000000}%
\pgfsetfillcolor{currentfill}%
\pgfsetlinewidth{0.803000pt}%
\definecolor{currentstroke}{rgb}{0.000000,0.000000,0.000000}%
\pgfsetstrokecolor{currentstroke}%
\pgfsetdash{}{0pt}%
\pgfsys@defobject{currentmarker}{\pgfqpoint{-0.048611in}{0.000000in}}{\pgfqpoint{-0.000000in}{0.000000in}}{%
\pgfpathmoveto{\pgfqpoint{-0.000000in}{0.000000in}}%
\pgfpathlineto{\pgfqpoint{-0.048611in}{0.000000in}}%
\pgfusepath{stroke,fill}%
}%
\begin{pgfscope}%
\pgfsys@transformshift{0.503581in}{1.551944in}%
\pgfsys@useobject{currentmarker}{}%
\end{pgfscope}%
\end{pgfscope}%
\begin{pgfscope}%
\definecolor{textcolor}{rgb}{0.000000,0.000000,0.000000}%
\pgfsetstrokecolor{textcolor}%
\pgfsetfillcolor{textcolor}%
\pgftext[x=0.228889in, y=1.503750in, left, base]{\color{textcolor}\rmfamily\fontsize{10.000000}{12.000000}\selectfont \(\displaystyle {1.0}\)}%
\end{pgfscope}%
\begin{pgfscope}%
\definecolor{textcolor}{rgb}{0.000000,0.000000,0.000000}%
\pgfsetstrokecolor{textcolor}%
\pgfsetfillcolor{textcolor}%
\pgftext[x=0.173333in,y=1.026944in,,bottom,rotate=90.000000]{\color{textcolor}\rmfamily\fontsize{10.000000}{12.000000}\selectfont True positive rate}%
\end{pgfscope}%
\begin{pgfscope}%
\pgfpathrectangle{\pgfqpoint{0.503581in}{0.449444in}}{\pgfqpoint{1.550000in}{1.155000in}}%
\pgfusepath{clip}%
\pgfsetbuttcap%
\pgfsetroundjoin%
\pgfsetlinewidth{1.505625pt}%
\definecolor{currentstroke}{rgb}{0.000000,0.000000,0.000000}%
\pgfsetstrokecolor{currentstroke}%
\pgfsetdash{{5.550000pt}{2.400000pt}}{0.000000pt}%
\pgfpathmoveto{\pgfqpoint{0.574035in}{0.501944in}}%
\pgfpathlineto{\pgfqpoint{1.983126in}{1.551944in}}%
\pgfusepath{stroke}%
\end{pgfscope}%
\begin{pgfscope}%
\pgfpathrectangle{\pgfqpoint{0.503581in}{0.449444in}}{\pgfqpoint{1.550000in}{1.155000in}}%
\pgfusepath{clip}%
\pgfsetrectcap%
\pgfsetroundjoin%
\pgfsetlinewidth{1.505625pt}%
\definecolor{currentstroke}{rgb}{0.000000,0.000000,0.000000}%
\pgfsetstrokecolor{currentstroke}%
\pgfsetdash{}{0pt}%
\pgfpathmoveto{\pgfqpoint{0.574035in}{0.501944in}}%
\pgfpathlineto{\pgfqpoint{0.585640in}{0.501944in}}%
\pgfpathlineto{\pgfqpoint{0.585640in}{0.508944in}}%
\pgfpathlineto{\pgfqpoint{0.587297in}{0.508944in}}%
\pgfpathlineto{\pgfqpoint{0.587297in}{0.515944in}}%
\pgfpathlineto{\pgfqpoint{0.600559in}{0.515944in}}%
\pgfpathlineto{\pgfqpoint{0.600559in}{0.529944in}}%
\pgfpathlineto{\pgfqpoint{0.602217in}{0.529944in}}%
\pgfpathlineto{\pgfqpoint{0.602217in}{0.536944in}}%
\pgfpathlineto{\pgfqpoint{0.603875in}{0.536944in}}%
\pgfpathlineto{\pgfqpoint{0.603875in}{0.543944in}}%
\pgfpathlineto{\pgfqpoint{0.605533in}{0.543944in}}%
\pgfpathlineto{\pgfqpoint{0.605533in}{0.550944in}}%
\pgfpathlineto{\pgfqpoint{0.607190in}{0.550944in}}%
\pgfpathlineto{\pgfqpoint{0.607190in}{0.557944in}}%
\pgfpathlineto{\pgfqpoint{0.618795in}{0.557944in}}%
\pgfpathlineto{\pgfqpoint{0.618795in}{0.564944in}}%
\pgfpathlineto{\pgfqpoint{0.625426in}{0.564944in}}%
\pgfpathlineto{\pgfqpoint{0.625426in}{0.571944in}}%
\pgfpathlineto{\pgfqpoint{0.640345in}{0.571944in}}%
\pgfpathlineto{\pgfqpoint{0.640345in}{0.578944in}}%
\pgfpathlineto{\pgfqpoint{0.651950in}{0.578944in}}%
\pgfpathlineto{\pgfqpoint{0.651950in}{0.585944in}}%
\pgfpathlineto{\pgfqpoint{0.673500in}{0.585944in}}%
\pgfpathlineto{\pgfqpoint{0.673500in}{0.599944in}}%
\pgfpathlineto{\pgfqpoint{0.688420in}{0.599944in}}%
\pgfpathlineto{\pgfqpoint{0.688420in}{0.606944in}}%
\pgfpathlineto{\pgfqpoint{0.698367in}{0.606944in}}%
\pgfpathlineto{\pgfqpoint{0.698367in}{0.613944in}}%
\pgfpathlineto{\pgfqpoint{0.701682in}{0.613944in}}%
\pgfpathlineto{\pgfqpoint{0.701682in}{0.620944in}}%
\pgfpathlineto{\pgfqpoint{0.708313in}{0.620944in}}%
\pgfpathlineto{\pgfqpoint{0.708313in}{0.627944in}}%
\pgfpathlineto{\pgfqpoint{0.723233in}{0.627944in}}%
\pgfpathlineto{\pgfqpoint{0.723233in}{0.634944in}}%
\pgfpathlineto{\pgfqpoint{0.736495in}{0.634944in}}%
\pgfpathlineto{\pgfqpoint{0.736495in}{0.641944in}}%
\pgfpathlineto{\pgfqpoint{0.743126in}{0.641944in}}%
\pgfpathlineto{\pgfqpoint{0.743126in}{0.648944in}}%
\pgfpathlineto{\pgfqpoint{0.758046in}{0.648944in}}%
\pgfpathlineto{\pgfqpoint{0.758046in}{0.655944in}}%
\pgfpathlineto{\pgfqpoint{0.784570in}{0.655944in}}%
\pgfpathlineto{\pgfqpoint{0.784570in}{0.669944in}}%
\pgfpathlineto{\pgfqpoint{0.789543in}{0.669944in}}%
\pgfpathlineto{\pgfqpoint{0.789543in}{0.683944in}}%
\pgfpathlineto{\pgfqpoint{0.806121in}{0.683944in}}%
\pgfpathlineto{\pgfqpoint{0.806121in}{0.697944in}}%
\pgfpathlineto{\pgfqpoint{0.816067in}{0.697944in}}%
\pgfpathlineto{\pgfqpoint{0.816067in}{0.704944in}}%
\pgfpathlineto{\pgfqpoint{0.821041in}{0.704944in}}%
\pgfpathlineto{\pgfqpoint{0.821041in}{0.711944in}}%
\pgfpathlineto{\pgfqpoint{0.822698in}{0.711944in}}%
\pgfpathlineto{\pgfqpoint{0.822698in}{0.718944in}}%
\pgfpathlineto{\pgfqpoint{0.827672in}{0.718944in}}%
\pgfpathlineto{\pgfqpoint{0.827672in}{0.725944in}}%
\pgfpathlineto{\pgfqpoint{0.829329in}{0.725944in}}%
\pgfpathlineto{\pgfqpoint{0.829329in}{0.739944in}}%
\pgfpathlineto{\pgfqpoint{0.852538in}{0.739944in}}%
\pgfpathlineto{\pgfqpoint{0.852538in}{0.746944in}}%
\pgfpathlineto{\pgfqpoint{0.877404in}{0.746944in}}%
\pgfpathlineto{\pgfqpoint{0.877404in}{0.760944in}}%
\pgfpathlineto{\pgfqpoint{0.880720in}{0.760944in}}%
\pgfpathlineto{\pgfqpoint{0.880720in}{0.774944in}}%
\pgfpathlineto{\pgfqpoint{0.882377in}{0.774944in}}%
\pgfpathlineto{\pgfqpoint{0.882377in}{0.781944in}}%
\pgfpathlineto{\pgfqpoint{0.889009in}{0.781944in}}%
\pgfpathlineto{\pgfqpoint{0.889009in}{0.788944in}}%
\pgfpathlineto{\pgfqpoint{0.892324in}{0.788944in}}%
\pgfpathlineto{\pgfqpoint{0.892324in}{0.795944in}}%
\pgfpathlineto{\pgfqpoint{0.908902in}{0.795944in}}%
\pgfpathlineto{\pgfqpoint{0.908902in}{0.802944in}}%
\pgfpathlineto{\pgfqpoint{0.910559in}{0.802944in}}%
\pgfpathlineto{\pgfqpoint{0.910559in}{0.809944in}}%
\pgfpathlineto{\pgfqpoint{0.918848in}{0.809944in}}%
\pgfpathlineto{\pgfqpoint{0.918848in}{0.816944in}}%
\pgfpathlineto{\pgfqpoint{0.922164in}{0.816944in}}%
\pgfpathlineto{\pgfqpoint{0.922164in}{0.823944in}}%
\pgfpathlineto{\pgfqpoint{0.923821in}{0.823944in}}%
\pgfpathlineto{\pgfqpoint{0.923821in}{0.830944in}}%
\pgfpathlineto{\pgfqpoint{0.942057in}{0.830944in}}%
\pgfpathlineto{\pgfqpoint{0.942057in}{0.837944in}}%
\pgfpathlineto{\pgfqpoint{0.945372in}{0.837944in}}%
\pgfpathlineto{\pgfqpoint{0.945372in}{0.844944in}}%
\pgfpathlineto{\pgfqpoint{0.947030in}{0.844944in}}%
\pgfpathlineto{\pgfqpoint{0.947030in}{0.851944in}}%
\pgfpathlineto{\pgfqpoint{0.953661in}{0.851944in}}%
\pgfpathlineto{\pgfqpoint{0.953661in}{0.865944in}}%
\pgfpathlineto{\pgfqpoint{0.956976in}{0.865944in}}%
\pgfpathlineto{\pgfqpoint{0.956976in}{0.872944in}}%
\pgfpathlineto{\pgfqpoint{0.960292in}{0.872944in}}%
\pgfpathlineto{\pgfqpoint{0.960292in}{0.886944in}}%
\pgfpathlineto{\pgfqpoint{0.963607in}{0.886944in}}%
\pgfpathlineto{\pgfqpoint{0.963607in}{0.893944in}}%
\pgfpathlineto{\pgfqpoint{0.976869in}{0.893944in}}%
\pgfpathlineto{\pgfqpoint{0.976869in}{0.900944in}}%
\pgfpathlineto{\pgfqpoint{0.981843in}{0.900944in}}%
\pgfpathlineto{\pgfqpoint{0.981843in}{0.907944in}}%
\pgfpathlineto{\pgfqpoint{0.983500in}{0.907944in}}%
\pgfpathlineto{\pgfqpoint{0.983500in}{0.914944in}}%
\pgfpathlineto{\pgfqpoint{0.988474in}{0.914944in}}%
\pgfpathlineto{\pgfqpoint{0.988474in}{0.921944in}}%
\pgfpathlineto{\pgfqpoint{1.031575in}{0.921944in}}%
\pgfpathlineto{\pgfqpoint{1.031575in}{0.928944in}}%
\pgfpathlineto{\pgfqpoint{1.043180in}{0.928944in}}%
\pgfpathlineto{\pgfqpoint{1.043180in}{0.935944in}}%
\pgfpathlineto{\pgfqpoint{1.056442in}{0.935944in}}%
\pgfpathlineto{\pgfqpoint{1.056442in}{0.942944in}}%
\pgfpathlineto{\pgfqpoint{1.064730in}{0.942944in}}%
\pgfpathlineto{\pgfqpoint{1.064730in}{0.949944in}}%
\pgfpathlineto{\pgfqpoint{1.092912in}{0.949944in}}%
\pgfpathlineto{\pgfqpoint{1.092912in}{0.956944in}}%
\pgfpathlineto{\pgfqpoint{1.102859in}{0.956944in}}%
\pgfpathlineto{\pgfqpoint{1.102859in}{0.963944in}}%
\pgfpathlineto{\pgfqpoint{1.111148in}{0.963944in}}%
\pgfpathlineto{\pgfqpoint{1.111148in}{0.970944in}}%
\pgfpathlineto{\pgfqpoint{1.121094in}{0.970944in}}%
\pgfpathlineto{\pgfqpoint{1.121094in}{0.977944in}}%
\pgfpathlineto{\pgfqpoint{1.132698in}{0.977944in}}%
\pgfpathlineto{\pgfqpoint{1.132698in}{0.984944in}}%
\pgfpathlineto{\pgfqpoint{1.134356in}{0.984944in}}%
\pgfpathlineto{\pgfqpoint{1.134356in}{0.991944in}}%
\pgfpathlineto{\pgfqpoint{1.147618in}{0.991944in}}%
\pgfpathlineto{\pgfqpoint{1.147618in}{0.998944in}}%
\pgfpathlineto{\pgfqpoint{1.150934in}{0.998944in}}%
\pgfpathlineto{\pgfqpoint{1.150934in}{1.005944in}}%
\pgfpathlineto{\pgfqpoint{1.165853in}{1.005944in}}%
\pgfpathlineto{\pgfqpoint{1.165853in}{1.012944in}}%
\pgfpathlineto{\pgfqpoint{1.167511in}{1.012944in}}%
\pgfpathlineto{\pgfqpoint{1.167511in}{1.019944in}}%
\pgfpathlineto{\pgfqpoint{1.174142in}{1.019944in}}%
\pgfpathlineto{\pgfqpoint{1.174142in}{1.026944in}}%
\pgfpathlineto{\pgfqpoint{1.185746in}{1.026944in}}%
\pgfpathlineto{\pgfqpoint{1.185746in}{1.033944in}}%
\pgfpathlineto{\pgfqpoint{1.187404in}{1.033944in}}%
\pgfpathlineto{\pgfqpoint{1.187404in}{1.040944in}}%
\pgfpathlineto{\pgfqpoint{1.190720in}{1.040944in}}%
\pgfpathlineto{\pgfqpoint{1.190720in}{1.047944in}}%
\pgfpathlineto{\pgfqpoint{1.200666in}{1.047944in}}%
\pgfpathlineto{\pgfqpoint{1.200666in}{1.054944in}}%
\pgfpathlineto{\pgfqpoint{1.208955in}{1.054944in}}%
\pgfpathlineto{\pgfqpoint{1.208955in}{1.061944in}}%
\pgfpathlineto{\pgfqpoint{1.238795in}{1.061944in}}%
\pgfpathlineto{\pgfqpoint{1.238795in}{1.068944in}}%
\pgfpathlineto{\pgfqpoint{1.243768in}{1.068944in}}%
\pgfpathlineto{\pgfqpoint{1.243768in}{1.075944in}}%
\pgfpathlineto{\pgfqpoint{1.265319in}{1.075944in}}%
\pgfpathlineto{\pgfqpoint{1.265319in}{1.082944in}}%
\pgfpathlineto{\pgfqpoint{1.278581in}{1.082944in}}%
\pgfpathlineto{\pgfqpoint{1.278581in}{1.096944in}}%
\pgfpathlineto{\pgfqpoint{1.313394in}{1.096944in}}%
\pgfpathlineto{\pgfqpoint{1.313394in}{1.117944in}}%
\pgfpathlineto{\pgfqpoint{1.336602in}{1.117944in}}%
\pgfpathlineto{\pgfqpoint{1.336602in}{1.124944in}}%
\pgfpathlineto{\pgfqpoint{1.351522in}{1.124944in}}%
\pgfpathlineto{\pgfqpoint{1.351522in}{1.131944in}}%
\pgfpathlineto{\pgfqpoint{1.353180in}{1.131944in}}%
\pgfpathlineto{\pgfqpoint{1.353180in}{1.138944in}}%
\pgfpathlineto{\pgfqpoint{1.354837in}{1.138944in}}%
\pgfpathlineto{\pgfqpoint{1.354837in}{1.145944in}}%
\pgfpathlineto{\pgfqpoint{1.361468in}{1.145944in}}%
\pgfpathlineto{\pgfqpoint{1.361468in}{1.152944in}}%
\pgfpathlineto{\pgfqpoint{1.363126in}{1.152944in}}%
\pgfpathlineto{\pgfqpoint{1.363126in}{1.159944in}}%
\pgfpathlineto{\pgfqpoint{1.378046in}{1.159944in}}%
\pgfpathlineto{\pgfqpoint{1.378046in}{1.166944in}}%
\pgfpathlineto{\pgfqpoint{1.384677in}{1.166944in}}%
\pgfpathlineto{\pgfqpoint{1.384677in}{1.173944in}}%
\pgfpathlineto{\pgfqpoint{1.394623in}{1.173944in}}%
\pgfpathlineto{\pgfqpoint{1.394623in}{1.187944in}}%
\pgfpathlineto{\pgfqpoint{1.396281in}{1.187944in}}%
\pgfpathlineto{\pgfqpoint{1.396281in}{1.194944in}}%
\pgfpathlineto{\pgfqpoint{1.397939in}{1.194944in}}%
\pgfpathlineto{\pgfqpoint{1.397939in}{1.208944in}}%
\pgfpathlineto{\pgfqpoint{1.399597in}{1.208944in}}%
\pgfpathlineto{\pgfqpoint{1.399597in}{1.215944in}}%
\pgfpathlineto{\pgfqpoint{1.407886in}{1.215944in}}%
\pgfpathlineto{\pgfqpoint{1.407886in}{1.222944in}}%
\pgfpathlineto{\pgfqpoint{1.434410in}{1.222944in}}%
\pgfpathlineto{\pgfqpoint{1.434410in}{1.229944in}}%
\pgfpathlineto{\pgfqpoint{1.454303in}{1.229944in}}%
\pgfpathlineto{\pgfqpoint{1.454303in}{1.236944in}}%
\pgfpathlineto{\pgfqpoint{1.474196in}{1.236944in}}%
\pgfpathlineto{\pgfqpoint{1.474196in}{1.243944in}}%
\pgfpathlineto{\pgfqpoint{1.484142in}{1.243944in}}%
\pgfpathlineto{\pgfqpoint{1.484142in}{1.250944in}}%
\pgfpathlineto{\pgfqpoint{1.485800in}{1.250944in}}%
\pgfpathlineto{\pgfqpoint{1.485800in}{1.257944in}}%
\pgfpathlineto{\pgfqpoint{1.494089in}{1.257944in}}%
\pgfpathlineto{\pgfqpoint{1.494089in}{1.264944in}}%
\pgfpathlineto{\pgfqpoint{1.504035in}{1.264944in}}%
\pgfpathlineto{\pgfqpoint{1.504035in}{1.271944in}}%
\pgfpathlineto{\pgfqpoint{1.505693in}{1.271944in}}%
\pgfpathlineto{\pgfqpoint{1.505693in}{1.278944in}}%
\pgfpathlineto{\pgfqpoint{1.509009in}{1.278944in}}%
\pgfpathlineto{\pgfqpoint{1.509009in}{1.285944in}}%
\pgfpathlineto{\pgfqpoint{1.513982in}{1.285944in}}%
\pgfpathlineto{\pgfqpoint{1.513982in}{1.292944in}}%
\pgfpathlineto{\pgfqpoint{1.543821in}{1.292944in}}%
\pgfpathlineto{\pgfqpoint{1.543821in}{1.299944in}}%
\pgfpathlineto{\pgfqpoint{1.547137in}{1.299944in}}%
\pgfpathlineto{\pgfqpoint{1.547137in}{1.306944in}}%
\pgfpathlineto{\pgfqpoint{1.572003in}{1.306944in}}%
\pgfpathlineto{\pgfqpoint{1.572003in}{1.313944in}}%
\pgfpathlineto{\pgfqpoint{1.578634in}{1.313944in}}%
\pgfpathlineto{\pgfqpoint{1.578634in}{1.320944in}}%
\pgfpathlineto{\pgfqpoint{1.588581in}{1.320944in}}%
\pgfpathlineto{\pgfqpoint{1.588581in}{1.327944in}}%
\pgfpathlineto{\pgfqpoint{1.593554in}{1.327944in}}%
\pgfpathlineto{\pgfqpoint{1.593554in}{1.334944in}}%
\pgfpathlineto{\pgfqpoint{1.598527in}{1.334944in}}%
\pgfpathlineto{\pgfqpoint{1.598527in}{1.341944in}}%
\pgfpathlineto{\pgfqpoint{1.600185in}{1.341944in}}%
\pgfpathlineto{\pgfqpoint{1.600185in}{1.348944in}}%
\pgfpathlineto{\pgfqpoint{1.615105in}{1.348944in}}%
\pgfpathlineto{\pgfqpoint{1.615105in}{1.355944in}}%
\pgfpathlineto{\pgfqpoint{1.639971in}{1.355944in}}%
\pgfpathlineto{\pgfqpoint{1.639971in}{1.362944in}}%
\pgfpathlineto{\pgfqpoint{1.643287in}{1.362944in}}%
\pgfpathlineto{\pgfqpoint{1.643287in}{1.369944in}}%
\pgfpathlineto{\pgfqpoint{1.651575in}{1.369944in}}%
\pgfpathlineto{\pgfqpoint{1.651575in}{1.376944in}}%
\pgfpathlineto{\pgfqpoint{1.668153in}{1.376944in}}%
\pgfpathlineto{\pgfqpoint{1.668153in}{1.397944in}}%
\pgfpathlineto{\pgfqpoint{1.691361in}{1.397944in}}%
\pgfpathlineto{\pgfqpoint{1.691361in}{1.404944in}}%
\pgfpathlineto{\pgfqpoint{1.697992in}{1.404944in}}%
\pgfpathlineto{\pgfqpoint{1.697992in}{1.411944in}}%
\pgfpathlineto{\pgfqpoint{1.717886in}{1.411944in}}%
\pgfpathlineto{\pgfqpoint{1.717886in}{1.425944in}}%
\pgfpathlineto{\pgfqpoint{1.721201in}{1.425944in}}%
\pgfpathlineto{\pgfqpoint{1.721201in}{1.432944in}}%
\pgfpathlineto{\pgfqpoint{1.757672in}{1.432944in}}%
\pgfpathlineto{\pgfqpoint{1.757672in}{1.439944in}}%
\pgfpathlineto{\pgfqpoint{1.770934in}{1.439944in}}%
\pgfpathlineto{\pgfqpoint{1.770934in}{1.446944in}}%
\pgfpathlineto{\pgfqpoint{1.777565in}{1.446944in}}%
\pgfpathlineto{\pgfqpoint{1.777565in}{1.453944in}}%
\pgfpathlineto{\pgfqpoint{1.787511in}{1.453944in}}%
\pgfpathlineto{\pgfqpoint{1.787511in}{1.460944in}}%
\pgfpathlineto{\pgfqpoint{1.825640in}{1.460944in}}%
\pgfpathlineto{\pgfqpoint{1.825640in}{1.467944in}}%
\pgfpathlineto{\pgfqpoint{1.827297in}{1.467944in}}%
\pgfpathlineto{\pgfqpoint{1.827297in}{1.481944in}}%
\pgfpathlineto{\pgfqpoint{1.828955in}{1.481944in}}%
\pgfpathlineto{\pgfqpoint{1.828955in}{1.488944in}}%
\pgfpathlineto{\pgfqpoint{1.835586in}{1.488944in}}%
\pgfpathlineto{\pgfqpoint{1.835586in}{1.495944in}}%
\pgfpathlineto{\pgfqpoint{1.857137in}{1.495944in}}%
\pgfpathlineto{\pgfqpoint{1.857137in}{1.509944in}}%
\pgfpathlineto{\pgfqpoint{1.867083in}{1.509944in}}%
\pgfpathlineto{\pgfqpoint{1.867083in}{1.516944in}}%
\pgfpathlineto{\pgfqpoint{1.896923in}{1.516944in}}%
\pgfpathlineto{\pgfqpoint{1.896923in}{1.523944in}}%
\pgfpathlineto{\pgfqpoint{1.906869in}{1.523944in}}%
\pgfpathlineto{\pgfqpoint{1.906869in}{1.530944in}}%
\pgfpathlineto{\pgfqpoint{1.910185in}{1.530944in}}%
\pgfpathlineto{\pgfqpoint{1.910185in}{1.537944in}}%
\pgfpathlineto{\pgfqpoint{1.911843in}{1.537944in}}%
\pgfpathlineto{\pgfqpoint{1.911843in}{1.544944in}}%
\pgfpathlineto{\pgfqpoint{1.963233in}{1.544944in}}%
\pgfpathlineto{\pgfqpoint{1.963233in}{1.551944in}}%
\pgfpathlineto{\pgfqpoint{1.983126in}{1.551944in}}%
\pgfpathlineto{\pgfqpoint{1.983126in}{1.551944in}}%
\pgfusepath{stroke}%
\end{pgfscope}%
\begin{pgfscope}%
\pgfsetrectcap%
\pgfsetmiterjoin%
\pgfsetlinewidth{0.803000pt}%
\definecolor{currentstroke}{rgb}{0.000000,0.000000,0.000000}%
\pgfsetstrokecolor{currentstroke}%
\pgfsetdash{}{0pt}%
\pgfpathmoveto{\pgfqpoint{0.503581in}{0.449444in}}%
\pgfpathlineto{\pgfqpoint{0.503581in}{1.604444in}}%
\pgfusepath{stroke}%
\end{pgfscope}%
\begin{pgfscope}%
\pgfsetrectcap%
\pgfsetmiterjoin%
\pgfsetlinewidth{0.803000pt}%
\definecolor{currentstroke}{rgb}{0.000000,0.000000,0.000000}%
\pgfsetstrokecolor{currentstroke}%
\pgfsetdash{}{0pt}%
\pgfpathmoveto{\pgfqpoint{2.053581in}{0.449444in}}%
\pgfpathlineto{\pgfqpoint{2.053581in}{1.604444in}}%
\pgfusepath{stroke}%
\end{pgfscope}%
\begin{pgfscope}%
\pgfsetrectcap%
\pgfsetmiterjoin%
\pgfsetlinewidth{0.803000pt}%
\definecolor{currentstroke}{rgb}{0.000000,0.000000,0.000000}%
\pgfsetstrokecolor{currentstroke}%
\pgfsetdash{}{0pt}%
\pgfpathmoveto{\pgfqpoint{0.503581in}{0.449444in}}%
\pgfpathlineto{\pgfqpoint{2.053581in}{0.449444in}}%
\pgfusepath{stroke}%
\end{pgfscope}%
\begin{pgfscope}%
\pgfsetrectcap%
\pgfsetmiterjoin%
\pgfsetlinewidth{0.803000pt}%
\definecolor{currentstroke}{rgb}{0.000000,0.000000,0.000000}%
\pgfsetstrokecolor{currentstroke}%
\pgfsetdash{}{0pt}%
\pgfpathmoveto{\pgfqpoint{0.503581in}{1.604444in}}%
\pgfpathlineto{\pgfqpoint{2.053581in}{1.604444in}}%
\pgfusepath{stroke}%
\end{pgfscope}%
\begin{pgfscope}%
\pgfsetbuttcap%
\pgfsetmiterjoin%
\definecolor{currentfill}{rgb}{1.000000,1.000000,1.000000}%
\pgfsetfillcolor{currentfill}%
\pgfsetfillopacity{0.800000}%
\pgfsetlinewidth{1.003750pt}%
\definecolor{currentstroke}{rgb}{0.800000,0.800000,0.800000}%
\pgfsetstrokecolor{currentstroke}%
\pgfsetstrokeopacity{0.800000}%
\pgfsetdash{}{0pt}%
\pgfpathmoveto{\pgfqpoint{0.782747in}{0.518889in}}%
\pgfpathlineto{\pgfqpoint{1.956358in}{0.518889in}}%
\pgfpathquadraticcurveto{\pgfqpoint{1.984136in}{0.518889in}}{\pgfqpoint{1.984136in}{0.546666in}}%
\pgfpathlineto{\pgfqpoint{1.984136in}{0.726388in}}%
\pgfpathquadraticcurveto{\pgfqpoint{1.984136in}{0.754166in}}{\pgfqpoint{1.956358in}{0.754166in}}%
\pgfpathlineto{\pgfqpoint{0.782747in}{0.754166in}}%
\pgfpathquadraticcurveto{\pgfqpoint{0.754970in}{0.754166in}}{\pgfqpoint{0.754970in}{0.726388in}}%
\pgfpathlineto{\pgfqpoint{0.754970in}{0.546666in}}%
\pgfpathquadraticcurveto{\pgfqpoint{0.754970in}{0.518889in}}{\pgfqpoint{0.782747in}{0.518889in}}%
\pgfpathlineto{\pgfqpoint{0.782747in}{0.518889in}}%
\pgfpathclose%
\pgfusepath{stroke,fill}%
\end{pgfscope}%
\begin{pgfscope}%
\pgfsetrectcap%
\pgfsetroundjoin%
\pgfsetlinewidth{1.505625pt}%
\definecolor{currentstroke}{rgb}{0.000000,0.000000,0.000000}%
\pgfsetstrokecolor{currentstroke}%
\pgfsetdash{}{0pt}%
\pgfpathmoveto{\pgfqpoint{0.810525in}{0.650000in}}%
\pgfpathlineto{\pgfqpoint{0.949414in}{0.650000in}}%
\pgfpathlineto{\pgfqpoint{1.088303in}{0.650000in}}%
\pgfusepath{stroke}%
\end{pgfscope}%
\begin{pgfscope}%
\definecolor{textcolor}{rgb}{0.000000,0.000000,0.000000}%
\pgfsetstrokecolor{textcolor}%
\pgfsetfillcolor{textcolor}%
\pgftext[x=1.199414in,y=0.601388in,left,base]{\color{textcolor}\rmfamily\fontsize{10.000000}{12.000000}\selectfont AUC=0.552}%
\end{pgfscope}%
\end{pgfpicture}%
\makeatother%
\endgroup%

	
&
	\vskip 0pt
	\begin{tabular}{cc|c|c|}
	&\multicolumn{1}{c}{}& \multicolumn{2}{c}{Prediction} \\[0.4em]
	&\multicolumn{1}{c}{} & \multicolumn{1}{c}{N} & \multicolumn{1}{c}{P} \cr\cline{3-4}
%0.551741176	
% 41.8	43.2	6.5	8.5	
% 0.503	0.164410058	0.566666667	0.247818438
	\multirow{2}{*}{\rotatebox[origin=c]{90}{Actual}}&N &
41.8\% & 43.2\%
	\vrule width 0pt height 10pt depth 2pt \cr\cline{3-4}
	&P & 
6.5\% & 8.5\%
	\vrule width 0pt height 10pt depth 2pt \cr\cline{3-4}
	\end{tabular}

	\hfil\begin{tabular}{ll}
	\cr
	0.503 & Accuracy\cr
	0.164 & Precision \cr
	0.567 & Recall \cr
	0.248 & F1 \cr
	0.552 & AUC \cr
\end{tabular}
\cr
\end{tabular}
} % End parbox

\

{\bf Ideal Model Tending Left} Under this model, using decision threshold $\theta = 0.5$, we would immediately dispatch no ambulances.  The model has separated the positive and negative classes just as well ($\text{AUC} = 0.840$), but the entire distribution is pushed to the left.  We did encounter this situation in our work.  

\

\verb|Ideal_Left|

%%%
\parbox{\linewidth}{
\noindent\begin{tabular}{@{\hspace{-6pt}}p{2.3in} @{\hspace{-6pt}}p{2.0in} p{1.8in}}
	\vskip 0pt
	\qquad \qquad Raw Model Output
	
	%% Creator: Matplotlib, PGF backend
%%
%% To include the figure in your LaTeX document, write
%%   \input{<filename>.pgf}
%%
%% Make sure the required packages are loaded in your preamble
%%   \usepackage{pgf}
%%
%% Also ensure that all the required font packages are loaded; for instance,
%% the lmodern package is sometimes necessary when using math font.
%%   \usepackage{lmodern}
%%
%% Figures using additional raster images can only be included by \input if
%% they are in the same directory as the main LaTeX file. For loading figures
%% from other directories you can use the `import` package
%%   \usepackage{import}
%%
%% and then include the figures with
%%   \import{<path to file>}{<filename>.pgf}
%%
%% Matplotlib used the following preamble
%%   
%%   \usepackage{fontspec}
%%   \makeatletter\@ifpackageloaded{underscore}{}{\usepackage[strings]{underscore}}\makeatother
%%
\begingroup%
\makeatletter%
\begin{pgfpicture}%
\pgfpathrectangle{\pgfpointorigin}{\pgfqpoint{2.253750in}{1.754444in}}%
\pgfusepath{use as bounding box, clip}%
\begin{pgfscope}%
\pgfsetbuttcap%
\pgfsetmiterjoin%
\definecolor{currentfill}{rgb}{1.000000,1.000000,1.000000}%
\pgfsetfillcolor{currentfill}%
\pgfsetlinewidth{0.000000pt}%
\definecolor{currentstroke}{rgb}{1.000000,1.000000,1.000000}%
\pgfsetstrokecolor{currentstroke}%
\pgfsetdash{}{0pt}%
\pgfpathmoveto{\pgfqpoint{0.000000in}{0.000000in}}%
\pgfpathlineto{\pgfqpoint{2.253750in}{0.000000in}}%
\pgfpathlineto{\pgfqpoint{2.253750in}{1.754444in}}%
\pgfpathlineto{\pgfqpoint{0.000000in}{1.754444in}}%
\pgfpathlineto{\pgfqpoint{0.000000in}{0.000000in}}%
\pgfpathclose%
\pgfusepath{fill}%
\end{pgfscope}%
\begin{pgfscope}%
\pgfsetbuttcap%
\pgfsetmiterjoin%
\definecolor{currentfill}{rgb}{1.000000,1.000000,1.000000}%
\pgfsetfillcolor{currentfill}%
\pgfsetlinewidth{0.000000pt}%
\definecolor{currentstroke}{rgb}{0.000000,0.000000,0.000000}%
\pgfsetstrokecolor{currentstroke}%
\pgfsetstrokeopacity{0.000000}%
\pgfsetdash{}{0pt}%
\pgfpathmoveto{\pgfqpoint{0.515000in}{0.499444in}}%
\pgfpathlineto{\pgfqpoint{2.065000in}{0.499444in}}%
\pgfpathlineto{\pgfqpoint{2.065000in}{1.654444in}}%
\pgfpathlineto{\pgfqpoint{0.515000in}{1.654444in}}%
\pgfpathlineto{\pgfqpoint{0.515000in}{0.499444in}}%
\pgfpathclose%
\pgfusepath{fill}%
\end{pgfscope}%
\begin{pgfscope}%
\pgfpathrectangle{\pgfqpoint{0.515000in}{0.499444in}}{\pgfqpoint{1.550000in}{1.155000in}}%
\pgfusepath{clip}%
\pgfsetbuttcap%
\pgfsetmiterjoin%
\pgfsetlinewidth{1.003750pt}%
\definecolor{currentstroke}{rgb}{0.000000,0.000000,0.000000}%
\pgfsetstrokecolor{currentstroke}%
\pgfsetdash{}{0pt}%
\pgfpathmoveto{\pgfqpoint{0.505000in}{0.499444in}}%
\pgfpathlineto{\pgfqpoint{0.552805in}{0.499444in}}%
\pgfpathlineto{\pgfqpoint{0.552805in}{0.961883in}}%
\pgfpathlineto{\pgfqpoint{0.505000in}{0.961883in}}%
\pgfusepath{stroke}%
\end{pgfscope}%
\begin{pgfscope}%
\pgfpathrectangle{\pgfqpoint{0.515000in}{0.499444in}}{\pgfqpoint{1.550000in}{1.155000in}}%
\pgfusepath{clip}%
\pgfsetbuttcap%
\pgfsetmiterjoin%
\pgfsetlinewidth{1.003750pt}%
\definecolor{currentstroke}{rgb}{0.000000,0.000000,0.000000}%
\pgfsetstrokecolor{currentstroke}%
\pgfsetdash{}{0pt}%
\pgfpathmoveto{\pgfqpoint{0.643537in}{0.499444in}}%
\pgfpathlineto{\pgfqpoint{0.704025in}{0.499444in}}%
\pgfpathlineto{\pgfqpoint{0.704025in}{1.599444in}}%
\pgfpathlineto{\pgfqpoint{0.643537in}{1.599444in}}%
\pgfpathlineto{\pgfqpoint{0.643537in}{0.499444in}}%
\pgfpathclose%
\pgfusepath{stroke}%
\end{pgfscope}%
\begin{pgfscope}%
\pgfpathrectangle{\pgfqpoint{0.515000in}{0.499444in}}{\pgfqpoint{1.550000in}{1.155000in}}%
\pgfusepath{clip}%
\pgfsetbuttcap%
\pgfsetmiterjoin%
\pgfsetlinewidth{1.003750pt}%
\definecolor{currentstroke}{rgb}{0.000000,0.000000,0.000000}%
\pgfsetstrokecolor{currentstroke}%
\pgfsetdash{}{0pt}%
\pgfpathmoveto{\pgfqpoint{0.794756in}{0.499444in}}%
\pgfpathlineto{\pgfqpoint{0.855244in}{0.499444in}}%
\pgfpathlineto{\pgfqpoint{0.855244in}{1.132630in}}%
\pgfpathlineto{\pgfqpoint{0.794756in}{1.132630in}}%
\pgfpathlineto{\pgfqpoint{0.794756in}{0.499444in}}%
\pgfpathclose%
\pgfusepath{stroke}%
\end{pgfscope}%
\begin{pgfscope}%
\pgfpathrectangle{\pgfqpoint{0.515000in}{0.499444in}}{\pgfqpoint{1.550000in}{1.155000in}}%
\pgfusepath{clip}%
\pgfsetbuttcap%
\pgfsetmiterjoin%
\pgfsetlinewidth{1.003750pt}%
\definecolor{currentstroke}{rgb}{0.000000,0.000000,0.000000}%
\pgfsetstrokecolor{currentstroke}%
\pgfsetdash{}{0pt}%
\pgfpathmoveto{\pgfqpoint{0.945976in}{0.499444in}}%
\pgfpathlineto{\pgfqpoint{1.006464in}{0.499444in}}%
\pgfpathlineto{\pgfqpoint{1.006464in}{0.739458in}}%
\pgfpathlineto{\pgfqpoint{0.945976in}{0.739458in}}%
\pgfpathlineto{\pgfqpoint{0.945976in}{0.499444in}}%
\pgfpathclose%
\pgfusepath{stroke}%
\end{pgfscope}%
\begin{pgfscope}%
\pgfpathrectangle{\pgfqpoint{0.515000in}{0.499444in}}{\pgfqpoint{1.550000in}{1.155000in}}%
\pgfusepath{clip}%
\pgfsetbuttcap%
\pgfsetmiterjoin%
\pgfsetlinewidth{1.003750pt}%
\definecolor{currentstroke}{rgb}{0.000000,0.000000,0.000000}%
\pgfsetstrokecolor{currentstroke}%
\pgfsetdash{}{0pt}%
\pgfpathmoveto{\pgfqpoint{1.097195in}{0.499444in}}%
\pgfpathlineto{\pgfqpoint{1.157683in}{0.499444in}}%
\pgfpathlineto{\pgfqpoint{1.157683in}{0.559668in}}%
\pgfpathlineto{\pgfqpoint{1.097195in}{0.559668in}}%
\pgfpathlineto{\pgfqpoint{1.097195in}{0.499444in}}%
\pgfpathclose%
\pgfusepath{stroke}%
\end{pgfscope}%
\begin{pgfscope}%
\pgfpathrectangle{\pgfqpoint{0.515000in}{0.499444in}}{\pgfqpoint{1.550000in}{1.155000in}}%
\pgfusepath{clip}%
\pgfsetbuttcap%
\pgfsetmiterjoin%
\pgfsetlinewidth{1.003750pt}%
\definecolor{currentstroke}{rgb}{0.000000,0.000000,0.000000}%
\pgfsetstrokecolor{currentstroke}%
\pgfsetdash{}{0pt}%
\pgfpathmoveto{\pgfqpoint{1.248415in}{0.499444in}}%
\pgfpathlineto{\pgfqpoint{1.308903in}{0.499444in}}%
\pgfpathlineto{\pgfqpoint{1.308903in}{0.499444in}}%
\pgfpathlineto{\pgfqpoint{1.248415in}{0.499444in}}%
\pgfpathlineto{\pgfqpoint{1.248415in}{0.499444in}}%
\pgfpathclose%
\pgfusepath{stroke}%
\end{pgfscope}%
\begin{pgfscope}%
\pgfpathrectangle{\pgfqpoint{0.515000in}{0.499444in}}{\pgfqpoint{1.550000in}{1.155000in}}%
\pgfusepath{clip}%
\pgfsetbuttcap%
\pgfsetmiterjoin%
\pgfsetlinewidth{1.003750pt}%
\definecolor{currentstroke}{rgb}{0.000000,0.000000,0.000000}%
\pgfsetstrokecolor{currentstroke}%
\pgfsetdash{}{0pt}%
\pgfpathmoveto{\pgfqpoint{1.399634in}{0.499444in}}%
\pgfpathlineto{\pgfqpoint{1.460122in}{0.499444in}}%
\pgfpathlineto{\pgfqpoint{1.460122in}{0.499444in}}%
\pgfpathlineto{\pgfqpoint{1.399634in}{0.499444in}}%
\pgfpathlineto{\pgfqpoint{1.399634in}{0.499444in}}%
\pgfpathclose%
\pgfusepath{stroke}%
\end{pgfscope}%
\begin{pgfscope}%
\pgfpathrectangle{\pgfqpoint{0.515000in}{0.499444in}}{\pgfqpoint{1.550000in}{1.155000in}}%
\pgfusepath{clip}%
\pgfsetbuttcap%
\pgfsetmiterjoin%
\pgfsetlinewidth{1.003750pt}%
\definecolor{currentstroke}{rgb}{0.000000,0.000000,0.000000}%
\pgfsetstrokecolor{currentstroke}%
\pgfsetdash{}{0pt}%
\pgfpathmoveto{\pgfqpoint{1.550854in}{0.499444in}}%
\pgfpathlineto{\pgfqpoint{1.611342in}{0.499444in}}%
\pgfpathlineto{\pgfqpoint{1.611342in}{0.499444in}}%
\pgfpathlineto{\pgfqpoint{1.550854in}{0.499444in}}%
\pgfpathlineto{\pgfqpoint{1.550854in}{0.499444in}}%
\pgfpathclose%
\pgfusepath{stroke}%
\end{pgfscope}%
\begin{pgfscope}%
\pgfpathrectangle{\pgfqpoint{0.515000in}{0.499444in}}{\pgfqpoint{1.550000in}{1.155000in}}%
\pgfusepath{clip}%
\pgfsetbuttcap%
\pgfsetmiterjoin%
\pgfsetlinewidth{1.003750pt}%
\definecolor{currentstroke}{rgb}{0.000000,0.000000,0.000000}%
\pgfsetstrokecolor{currentstroke}%
\pgfsetdash{}{0pt}%
\pgfpathmoveto{\pgfqpoint{1.702073in}{0.499444in}}%
\pgfpathlineto{\pgfqpoint{1.762561in}{0.499444in}}%
\pgfpathlineto{\pgfqpoint{1.762561in}{0.499444in}}%
\pgfpathlineto{\pgfqpoint{1.702073in}{0.499444in}}%
\pgfpathlineto{\pgfqpoint{1.702073in}{0.499444in}}%
\pgfpathclose%
\pgfusepath{stroke}%
\end{pgfscope}%
\begin{pgfscope}%
\pgfpathrectangle{\pgfqpoint{0.515000in}{0.499444in}}{\pgfqpoint{1.550000in}{1.155000in}}%
\pgfusepath{clip}%
\pgfsetbuttcap%
\pgfsetmiterjoin%
\pgfsetlinewidth{1.003750pt}%
\definecolor{currentstroke}{rgb}{0.000000,0.000000,0.000000}%
\pgfsetstrokecolor{currentstroke}%
\pgfsetdash{}{0pt}%
\pgfpathmoveto{\pgfqpoint{1.853293in}{0.499444in}}%
\pgfpathlineto{\pgfqpoint{1.913781in}{0.499444in}}%
\pgfpathlineto{\pgfqpoint{1.913781in}{0.499444in}}%
\pgfpathlineto{\pgfqpoint{1.853293in}{0.499444in}}%
\pgfpathlineto{\pgfqpoint{1.853293in}{0.499444in}}%
\pgfpathclose%
\pgfusepath{stroke}%
\end{pgfscope}%
\begin{pgfscope}%
\pgfpathrectangle{\pgfqpoint{0.515000in}{0.499444in}}{\pgfqpoint{1.550000in}{1.155000in}}%
\pgfusepath{clip}%
\pgfsetbuttcap%
\pgfsetmiterjoin%
\definecolor{currentfill}{rgb}{0.000000,0.000000,0.000000}%
\pgfsetfillcolor{currentfill}%
\pgfsetlinewidth{0.000000pt}%
\definecolor{currentstroke}{rgb}{0.000000,0.000000,0.000000}%
\pgfsetstrokecolor{currentstroke}%
\pgfsetstrokeopacity{0.000000}%
\pgfsetdash{}{0pt}%
\pgfpathmoveto{\pgfqpoint{0.552805in}{0.499444in}}%
\pgfpathlineto{\pgfqpoint{0.613293in}{0.499444in}}%
\pgfpathlineto{\pgfqpoint{0.613293in}{0.511101in}}%
\pgfpathlineto{\pgfqpoint{0.552805in}{0.511101in}}%
\pgfpathlineto{\pgfqpoint{0.552805in}{0.499444in}}%
\pgfpathclose%
\pgfusepath{fill}%
\end{pgfscope}%
\begin{pgfscope}%
\pgfpathrectangle{\pgfqpoint{0.515000in}{0.499444in}}{\pgfqpoint{1.550000in}{1.155000in}}%
\pgfusepath{clip}%
\pgfsetbuttcap%
\pgfsetmiterjoin%
\definecolor{currentfill}{rgb}{0.000000,0.000000,0.000000}%
\pgfsetfillcolor{currentfill}%
\pgfsetlinewidth{0.000000pt}%
\definecolor{currentstroke}{rgb}{0.000000,0.000000,0.000000}%
\pgfsetstrokecolor{currentstroke}%
\pgfsetstrokeopacity{0.000000}%
\pgfsetdash{}{0pt}%
\pgfpathmoveto{\pgfqpoint{0.704025in}{0.499444in}}%
\pgfpathlineto{\pgfqpoint{0.764512in}{0.499444in}}%
\pgfpathlineto{\pgfqpoint{0.764512in}{0.540846in}}%
\pgfpathlineto{\pgfqpoint{0.704025in}{0.540846in}}%
\pgfpathlineto{\pgfqpoint{0.704025in}{0.499444in}}%
\pgfpathclose%
\pgfusepath{fill}%
\end{pgfscope}%
\begin{pgfscope}%
\pgfpathrectangle{\pgfqpoint{0.515000in}{0.499444in}}{\pgfqpoint{1.550000in}{1.155000in}}%
\pgfusepath{clip}%
\pgfsetbuttcap%
\pgfsetmiterjoin%
\definecolor{currentfill}{rgb}{0.000000,0.000000,0.000000}%
\pgfsetfillcolor{currentfill}%
\pgfsetlinewidth{0.000000pt}%
\definecolor{currentstroke}{rgb}{0.000000,0.000000,0.000000}%
\pgfsetstrokecolor{currentstroke}%
\pgfsetstrokeopacity{0.000000}%
\pgfsetdash{}{0pt}%
\pgfpathmoveto{\pgfqpoint{0.855244in}{0.499444in}}%
\pgfpathlineto{\pgfqpoint{0.915732in}{0.499444in}}%
\pgfpathlineto{\pgfqpoint{0.915732in}{0.615047in}}%
\pgfpathlineto{\pgfqpoint{0.855244in}{0.615047in}}%
\pgfpathlineto{\pgfqpoint{0.855244in}{0.499444in}}%
\pgfpathclose%
\pgfusepath{fill}%
\end{pgfscope}%
\begin{pgfscope}%
\pgfpathrectangle{\pgfqpoint{0.515000in}{0.499444in}}{\pgfqpoint{1.550000in}{1.155000in}}%
\pgfusepath{clip}%
\pgfsetbuttcap%
\pgfsetmiterjoin%
\definecolor{currentfill}{rgb}{0.000000,0.000000,0.000000}%
\pgfsetfillcolor{currentfill}%
\pgfsetlinewidth{0.000000pt}%
\definecolor{currentstroke}{rgb}{0.000000,0.000000,0.000000}%
\pgfsetstrokecolor{currentstroke}%
\pgfsetstrokeopacity{0.000000}%
\pgfsetdash{}{0pt}%
\pgfpathmoveto{\pgfqpoint{1.006464in}{0.499444in}}%
\pgfpathlineto{\pgfqpoint{1.066951in}{0.499444in}}%
\pgfpathlineto{\pgfqpoint{1.066951in}{0.693211in}}%
\pgfpathlineto{\pgfqpoint{1.006464in}{0.693211in}}%
\pgfpathlineto{\pgfqpoint{1.006464in}{0.499444in}}%
\pgfpathclose%
\pgfusepath{fill}%
\end{pgfscope}%
\begin{pgfscope}%
\pgfpathrectangle{\pgfqpoint{0.515000in}{0.499444in}}{\pgfqpoint{1.550000in}{1.155000in}}%
\pgfusepath{clip}%
\pgfsetbuttcap%
\pgfsetmiterjoin%
\definecolor{currentfill}{rgb}{0.000000,0.000000,0.000000}%
\pgfsetfillcolor{currentfill}%
\pgfsetlinewidth{0.000000pt}%
\definecolor{currentstroke}{rgb}{0.000000,0.000000,0.000000}%
\pgfsetstrokecolor{currentstroke}%
\pgfsetstrokeopacity{0.000000}%
\pgfsetdash{}{0pt}%
\pgfpathmoveto{\pgfqpoint{1.157683in}{0.499444in}}%
\pgfpathlineto{\pgfqpoint{1.218171in}{0.499444in}}%
\pgfpathlineto{\pgfqpoint{1.218171in}{0.577462in}}%
\pgfpathlineto{\pgfqpoint{1.157683in}{0.577462in}}%
\pgfpathlineto{\pgfqpoint{1.157683in}{0.499444in}}%
\pgfpathclose%
\pgfusepath{fill}%
\end{pgfscope}%
\begin{pgfscope}%
\pgfpathrectangle{\pgfqpoint{0.515000in}{0.499444in}}{\pgfqpoint{1.550000in}{1.155000in}}%
\pgfusepath{clip}%
\pgfsetbuttcap%
\pgfsetmiterjoin%
\definecolor{currentfill}{rgb}{0.000000,0.000000,0.000000}%
\pgfsetfillcolor{currentfill}%
\pgfsetlinewidth{0.000000pt}%
\definecolor{currentstroke}{rgb}{0.000000,0.000000,0.000000}%
\pgfsetstrokecolor{currentstroke}%
\pgfsetstrokeopacity{0.000000}%
\pgfsetdash{}{0pt}%
\pgfpathmoveto{\pgfqpoint{1.308903in}{0.499444in}}%
\pgfpathlineto{\pgfqpoint{1.369391in}{0.499444in}}%
\pgfpathlineto{\pgfqpoint{1.369391in}{0.499444in}}%
\pgfpathlineto{\pgfqpoint{1.308903in}{0.499444in}}%
\pgfpathlineto{\pgfqpoint{1.308903in}{0.499444in}}%
\pgfpathclose%
\pgfusepath{fill}%
\end{pgfscope}%
\begin{pgfscope}%
\pgfpathrectangle{\pgfqpoint{0.515000in}{0.499444in}}{\pgfqpoint{1.550000in}{1.155000in}}%
\pgfusepath{clip}%
\pgfsetbuttcap%
\pgfsetmiterjoin%
\definecolor{currentfill}{rgb}{0.000000,0.000000,0.000000}%
\pgfsetfillcolor{currentfill}%
\pgfsetlinewidth{0.000000pt}%
\definecolor{currentstroke}{rgb}{0.000000,0.000000,0.000000}%
\pgfsetstrokecolor{currentstroke}%
\pgfsetstrokeopacity{0.000000}%
\pgfsetdash{}{0pt}%
\pgfpathmoveto{\pgfqpoint{1.460122in}{0.499444in}}%
\pgfpathlineto{\pgfqpoint{1.520610in}{0.499444in}}%
\pgfpathlineto{\pgfqpoint{1.520610in}{0.499444in}}%
\pgfpathlineto{\pgfqpoint{1.460122in}{0.499444in}}%
\pgfpathlineto{\pgfqpoint{1.460122in}{0.499444in}}%
\pgfpathclose%
\pgfusepath{fill}%
\end{pgfscope}%
\begin{pgfscope}%
\pgfpathrectangle{\pgfqpoint{0.515000in}{0.499444in}}{\pgfqpoint{1.550000in}{1.155000in}}%
\pgfusepath{clip}%
\pgfsetbuttcap%
\pgfsetmiterjoin%
\definecolor{currentfill}{rgb}{0.000000,0.000000,0.000000}%
\pgfsetfillcolor{currentfill}%
\pgfsetlinewidth{0.000000pt}%
\definecolor{currentstroke}{rgb}{0.000000,0.000000,0.000000}%
\pgfsetstrokecolor{currentstroke}%
\pgfsetstrokeopacity{0.000000}%
\pgfsetdash{}{0pt}%
\pgfpathmoveto{\pgfqpoint{1.611342in}{0.499444in}}%
\pgfpathlineto{\pgfqpoint{1.671830in}{0.499444in}}%
\pgfpathlineto{\pgfqpoint{1.671830in}{0.499444in}}%
\pgfpathlineto{\pgfqpoint{1.611342in}{0.499444in}}%
\pgfpathlineto{\pgfqpoint{1.611342in}{0.499444in}}%
\pgfpathclose%
\pgfusepath{fill}%
\end{pgfscope}%
\begin{pgfscope}%
\pgfpathrectangle{\pgfqpoint{0.515000in}{0.499444in}}{\pgfqpoint{1.550000in}{1.155000in}}%
\pgfusepath{clip}%
\pgfsetbuttcap%
\pgfsetmiterjoin%
\definecolor{currentfill}{rgb}{0.000000,0.000000,0.000000}%
\pgfsetfillcolor{currentfill}%
\pgfsetlinewidth{0.000000pt}%
\definecolor{currentstroke}{rgb}{0.000000,0.000000,0.000000}%
\pgfsetstrokecolor{currentstroke}%
\pgfsetstrokeopacity{0.000000}%
\pgfsetdash{}{0pt}%
\pgfpathmoveto{\pgfqpoint{1.762561in}{0.499444in}}%
\pgfpathlineto{\pgfqpoint{1.823049in}{0.499444in}}%
\pgfpathlineto{\pgfqpoint{1.823049in}{0.499444in}}%
\pgfpathlineto{\pgfqpoint{1.762561in}{0.499444in}}%
\pgfpathlineto{\pgfqpoint{1.762561in}{0.499444in}}%
\pgfpathclose%
\pgfusepath{fill}%
\end{pgfscope}%
\begin{pgfscope}%
\pgfpathrectangle{\pgfqpoint{0.515000in}{0.499444in}}{\pgfqpoint{1.550000in}{1.155000in}}%
\pgfusepath{clip}%
\pgfsetbuttcap%
\pgfsetmiterjoin%
\definecolor{currentfill}{rgb}{0.000000,0.000000,0.000000}%
\pgfsetfillcolor{currentfill}%
\pgfsetlinewidth{0.000000pt}%
\definecolor{currentstroke}{rgb}{0.000000,0.000000,0.000000}%
\pgfsetstrokecolor{currentstroke}%
\pgfsetstrokeopacity{0.000000}%
\pgfsetdash{}{0pt}%
\pgfpathmoveto{\pgfqpoint{1.913781in}{0.499444in}}%
\pgfpathlineto{\pgfqpoint{1.974269in}{0.499444in}}%
\pgfpathlineto{\pgfqpoint{1.974269in}{0.499444in}}%
\pgfpathlineto{\pgfqpoint{1.913781in}{0.499444in}}%
\pgfpathlineto{\pgfqpoint{1.913781in}{0.499444in}}%
\pgfpathclose%
\pgfusepath{fill}%
\end{pgfscope}%
\begin{pgfscope}%
\pgfsetbuttcap%
\pgfsetroundjoin%
\definecolor{currentfill}{rgb}{0.000000,0.000000,0.000000}%
\pgfsetfillcolor{currentfill}%
\pgfsetlinewidth{0.803000pt}%
\definecolor{currentstroke}{rgb}{0.000000,0.000000,0.000000}%
\pgfsetstrokecolor{currentstroke}%
\pgfsetdash{}{0pt}%
\pgfsys@defobject{currentmarker}{\pgfqpoint{0.000000in}{-0.048611in}}{\pgfqpoint{0.000000in}{0.000000in}}{%
\pgfpathmoveto{\pgfqpoint{0.000000in}{0.000000in}}%
\pgfpathlineto{\pgfqpoint{0.000000in}{-0.048611in}}%
\pgfusepath{stroke,fill}%
}%
\begin{pgfscope}%
\pgfsys@transformshift{0.552805in}{0.499444in}%
\pgfsys@useobject{currentmarker}{}%
\end{pgfscope}%
\end{pgfscope}%
\begin{pgfscope}%
\definecolor{textcolor}{rgb}{0.000000,0.000000,0.000000}%
\pgfsetstrokecolor{textcolor}%
\pgfsetfillcolor{textcolor}%
\pgftext[x=0.552805in,y=0.402222in,,top]{\color{textcolor}\rmfamily\fontsize{10.000000}{12.000000}\selectfont 0.0}%
\end{pgfscope}%
\begin{pgfscope}%
\pgfsetbuttcap%
\pgfsetroundjoin%
\definecolor{currentfill}{rgb}{0.000000,0.000000,0.000000}%
\pgfsetfillcolor{currentfill}%
\pgfsetlinewidth{0.803000pt}%
\definecolor{currentstroke}{rgb}{0.000000,0.000000,0.000000}%
\pgfsetstrokecolor{currentstroke}%
\pgfsetdash{}{0pt}%
\pgfsys@defobject{currentmarker}{\pgfqpoint{0.000000in}{-0.048611in}}{\pgfqpoint{0.000000in}{0.000000in}}{%
\pgfpathmoveto{\pgfqpoint{0.000000in}{0.000000in}}%
\pgfpathlineto{\pgfqpoint{0.000000in}{-0.048611in}}%
\pgfusepath{stroke,fill}%
}%
\begin{pgfscope}%
\pgfsys@transformshift{0.930854in}{0.499444in}%
\pgfsys@useobject{currentmarker}{}%
\end{pgfscope}%
\end{pgfscope}%
\begin{pgfscope}%
\definecolor{textcolor}{rgb}{0.000000,0.000000,0.000000}%
\pgfsetstrokecolor{textcolor}%
\pgfsetfillcolor{textcolor}%
\pgftext[x=0.930854in,y=0.402222in,,top]{\color{textcolor}\rmfamily\fontsize{10.000000}{12.000000}\selectfont 0.25}%
\end{pgfscope}%
\begin{pgfscope}%
\pgfsetbuttcap%
\pgfsetroundjoin%
\definecolor{currentfill}{rgb}{0.000000,0.000000,0.000000}%
\pgfsetfillcolor{currentfill}%
\pgfsetlinewidth{0.803000pt}%
\definecolor{currentstroke}{rgb}{0.000000,0.000000,0.000000}%
\pgfsetstrokecolor{currentstroke}%
\pgfsetdash{}{0pt}%
\pgfsys@defobject{currentmarker}{\pgfqpoint{0.000000in}{-0.048611in}}{\pgfqpoint{0.000000in}{0.000000in}}{%
\pgfpathmoveto{\pgfqpoint{0.000000in}{0.000000in}}%
\pgfpathlineto{\pgfqpoint{0.000000in}{-0.048611in}}%
\pgfusepath{stroke,fill}%
}%
\begin{pgfscope}%
\pgfsys@transformshift{1.308903in}{0.499444in}%
\pgfsys@useobject{currentmarker}{}%
\end{pgfscope}%
\end{pgfscope}%
\begin{pgfscope}%
\definecolor{textcolor}{rgb}{0.000000,0.000000,0.000000}%
\pgfsetstrokecolor{textcolor}%
\pgfsetfillcolor{textcolor}%
\pgftext[x=1.308903in,y=0.402222in,,top]{\color{textcolor}\rmfamily\fontsize{10.000000}{12.000000}\selectfont 0.5}%
\end{pgfscope}%
\begin{pgfscope}%
\pgfsetbuttcap%
\pgfsetroundjoin%
\definecolor{currentfill}{rgb}{0.000000,0.000000,0.000000}%
\pgfsetfillcolor{currentfill}%
\pgfsetlinewidth{0.803000pt}%
\definecolor{currentstroke}{rgb}{0.000000,0.000000,0.000000}%
\pgfsetstrokecolor{currentstroke}%
\pgfsetdash{}{0pt}%
\pgfsys@defobject{currentmarker}{\pgfqpoint{0.000000in}{-0.048611in}}{\pgfqpoint{0.000000in}{0.000000in}}{%
\pgfpathmoveto{\pgfqpoint{0.000000in}{0.000000in}}%
\pgfpathlineto{\pgfqpoint{0.000000in}{-0.048611in}}%
\pgfusepath{stroke,fill}%
}%
\begin{pgfscope}%
\pgfsys@transformshift{1.686951in}{0.499444in}%
\pgfsys@useobject{currentmarker}{}%
\end{pgfscope}%
\end{pgfscope}%
\begin{pgfscope}%
\definecolor{textcolor}{rgb}{0.000000,0.000000,0.000000}%
\pgfsetstrokecolor{textcolor}%
\pgfsetfillcolor{textcolor}%
\pgftext[x=1.686951in,y=0.402222in,,top]{\color{textcolor}\rmfamily\fontsize{10.000000}{12.000000}\selectfont 0.75}%
\end{pgfscope}%
\begin{pgfscope}%
\pgfsetbuttcap%
\pgfsetroundjoin%
\definecolor{currentfill}{rgb}{0.000000,0.000000,0.000000}%
\pgfsetfillcolor{currentfill}%
\pgfsetlinewidth{0.803000pt}%
\definecolor{currentstroke}{rgb}{0.000000,0.000000,0.000000}%
\pgfsetstrokecolor{currentstroke}%
\pgfsetdash{}{0pt}%
\pgfsys@defobject{currentmarker}{\pgfqpoint{0.000000in}{-0.048611in}}{\pgfqpoint{0.000000in}{0.000000in}}{%
\pgfpathmoveto{\pgfqpoint{0.000000in}{0.000000in}}%
\pgfpathlineto{\pgfqpoint{0.000000in}{-0.048611in}}%
\pgfusepath{stroke,fill}%
}%
\begin{pgfscope}%
\pgfsys@transformshift{2.065000in}{0.499444in}%
\pgfsys@useobject{currentmarker}{}%
\end{pgfscope}%
\end{pgfscope}%
\begin{pgfscope}%
\definecolor{textcolor}{rgb}{0.000000,0.000000,0.000000}%
\pgfsetstrokecolor{textcolor}%
\pgfsetfillcolor{textcolor}%
\pgftext[x=2.065000in,y=0.402222in,,top]{\color{textcolor}\rmfamily\fontsize{10.000000}{12.000000}\selectfont 1.0}%
\end{pgfscope}%
\begin{pgfscope}%
\definecolor{textcolor}{rgb}{0.000000,0.000000,0.000000}%
\pgfsetstrokecolor{textcolor}%
\pgfsetfillcolor{textcolor}%
\pgftext[x=1.290000in,y=0.223333in,,top]{\color{textcolor}\rmfamily\fontsize{10.000000}{12.000000}\selectfont \(\displaystyle p\)}%
\end{pgfscope}%
\begin{pgfscope}%
\pgfsetbuttcap%
\pgfsetroundjoin%
\definecolor{currentfill}{rgb}{0.000000,0.000000,0.000000}%
\pgfsetfillcolor{currentfill}%
\pgfsetlinewidth{0.803000pt}%
\definecolor{currentstroke}{rgb}{0.000000,0.000000,0.000000}%
\pgfsetstrokecolor{currentstroke}%
\pgfsetdash{}{0pt}%
\pgfsys@defobject{currentmarker}{\pgfqpoint{-0.048611in}{0.000000in}}{\pgfqpoint{-0.000000in}{0.000000in}}{%
\pgfpathmoveto{\pgfqpoint{-0.000000in}{0.000000in}}%
\pgfpathlineto{\pgfqpoint{-0.048611in}{0.000000in}}%
\pgfusepath{stroke,fill}%
}%
\begin{pgfscope}%
\pgfsys@transformshift{0.515000in}{0.499444in}%
\pgfsys@useobject{currentmarker}{}%
\end{pgfscope}%
\end{pgfscope}%
\begin{pgfscope}%
\definecolor{textcolor}{rgb}{0.000000,0.000000,0.000000}%
\pgfsetstrokecolor{textcolor}%
\pgfsetfillcolor{textcolor}%
\pgftext[x=0.348333in, y=0.451250in, left, base]{\color{textcolor}\rmfamily\fontsize{10.000000}{12.000000}\selectfont \(\displaystyle {0}\)}%
\end{pgfscope}%
\begin{pgfscope}%
\pgfsetbuttcap%
\pgfsetroundjoin%
\definecolor{currentfill}{rgb}{0.000000,0.000000,0.000000}%
\pgfsetfillcolor{currentfill}%
\pgfsetlinewidth{0.803000pt}%
\definecolor{currentstroke}{rgb}{0.000000,0.000000,0.000000}%
\pgfsetstrokecolor{currentstroke}%
\pgfsetdash{}{0pt}%
\pgfsys@defobject{currentmarker}{\pgfqpoint{-0.048611in}{0.000000in}}{\pgfqpoint{-0.000000in}{0.000000in}}{%
\pgfpathmoveto{\pgfqpoint{-0.000000in}{0.000000in}}%
\pgfpathlineto{\pgfqpoint{-0.048611in}{0.000000in}}%
\pgfusepath{stroke,fill}%
}%
\begin{pgfscope}%
\pgfsys@transformshift{0.515000in}{0.793075in}%
\pgfsys@useobject{currentmarker}{}%
\end{pgfscope}%
\end{pgfscope}%
\begin{pgfscope}%
\definecolor{textcolor}{rgb}{0.000000,0.000000,0.000000}%
\pgfsetstrokecolor{textcolor}%
\pgfsetfillcolor{textcolor}%
\pgftext[x=0.278889in, y=0.744881in, left, base]{\color{textcolor}\rmfamily\fontsize{10.000000}{12.000000}\selectfont \(\displaystyle {10}\)}%
\end{pgfscope}%
\begin{pgfscope}%
\pgfsetbuttcap%
\pgfsetroundjoin%
\definecolor{currentfill}{rgb}{0.000000,0.000000,0.000000}%
\pgfsetfillcolor{currentfill}%
\pgfsetlinewidth{0.803000pt}%
\definecolor{currentstroke}{rgb}{0.000000,0.000000,0.000000}%
\pgfsetstrokecolor{currentstroke}%
\pgfsetdash{}{0pt}%
\pgfsys@defobject{currentmarker}{\pgfqpoint{-0.048611in}{0.000000in}}{\pgfqpoint{-0.000000in}{0.000000in}}{%
\pgfpathmoveto{\pgfqpoint{-0.000000in}{0.000000in}}%
\pgfpathlineto{\pgfqpoint{-0.048611in}{0.000000in}}%
\pgfusepath{stroke,fill}%
}%
\begin{pgfscope}%
\pgfsys@transformshift{0.515000in}{1.086706in}%
\pgfsys@useobject{currentmarker}{}%
\end{pgfscope}%
\end{pgfscope}%
\begin{pgfscope}%
\definecolor{textcolor}{rgb}{0.000000,0.000000,0.000000}%
\pgfsetstrokecolor{textcolor}%
\pgfsetfillcolor{textcolor}%
\pgftext[x=0.278889in, y=1.038511in, left, base]{\color{textcolor}\rmfamily\fontsize{10.000000}{12.000000}\selectfont \(\displaystyle {20}\)}%
\end{pgfscope}%
\begin{pgfscope}%
\pgfsetbuttcap%
\pgfsetroundjoin%
\definecolor{currentfill}{rgb}{0.000000,0.000000,0.000000}%
\pgfsetfillcolor{currentfill}%
\pgfsetlinewidth{0.803000pt}%
\definecolor{currentstroke}{rgb}{0.000000,0.000000,0.000000}%
\pgfsetstrokecolor{currentstroke}%
\pgfsetdash{}{0pt}%
\pgfsys@defobject{currentmarker}{\pgfqpoint{-0.048611in}{0.000000in}}{\pgfqpoint{-0.000000in}{0.000000in}}{%
\pgfpathmoveto{\pgfqpoint{-0.000000in}{0.000000in}}%
\pgfpathlineto{\pgfqpoint{-0.048611in}{0.000000in}}%
\pgfusepath{stroke,fill}%
}%
\begin{pgfscope}%
\pgfsys@transformshift{0.515000in}{1.380337in}%
\pgfsys@useobject{currentmarker}{}%
\end{pgfscope}%
\end{pgfscope}%
\begin{pgfscope}%
\definecolor{textcolor}{rgb}{0.000000,0.000000,0.000000}%
\pgfsetstrokecolor{textcolor}%
\pgfsetfillcolor{textcolor}%
\pgftext[x=0.278889in, y=1.332142in, left, base]{\color{textcolor}\rmfamily\fontsize{10.000000}{12.000000}\selectfont \(\displaystyle {30}\)}%
\end{pgfscope}%
\begin{pgfscope}%
\definecolor{textcolor}{rgb}{0.000000,0.000000,0.000000}%
\pgfsetstrokecolor{textcolor}%
\pgfsetfillcolor{textcolor}%
\pgftext[x=0.223333in,y=1.076944in,,bottom,rotate=90.000000]{\color{textcolor}\rmfamily\fontsize{10.000000}{12.000000}\selectfont Percent of Data Set}%
\end{pgfscope}%
\begin{pgfscope}%
\pgfsetrectcap%
\pgfsetmiterjoin%
\pgfsetlinewidth{0.803000pt}%
\definecolor{currentstroke}{rgb}{0.000000,0.000000,0.000000}%
\pgfsetstrokecolor{currentstroke}%
\pgfsetdash{}{0pt}%
\pgfpathmoveto{\pgfqpoint{0.515000in}{0.499444in}}%
\pgfpathlineto{\pgfqpoint{0.515000in}{1.654444in}}%
\pgfusepath{stroke}%
\end{pgfscope}%
\begin{pgfscope}%
\pgfsetrectcap%
\pgfsetmiterjoin%
\pgfsetlinewidth{0.803000pt}%
\definecolor{currentstroke}{rgb}{0.000000,0.000000,0.000000}%
\pgfsetstrokecolor{currentstroke}%
\pgfsetdash{}{0pt}%
\pgfpathmoveto{\pgfqpoint{2.065000in}{0.499444in}}%
\pgfpathlineto{\pgfqpoint{2.065000in}{1.654444in}}%
\pgfusepath{stroke}%
\end{pgfscope}%
\begin{pgfscope}%
\pgfsetrectcap%
\pgfsetmiterjoin%
\pgfsetlinewidth{0.803000pt}%
\definecolor{currentstroke}{rgb}{0.000000,0.000000,0.000000}%
\pgfsetstrokecolor{currentstroke}%
\pgfsetdash{}{0pt}%
\pgfpathmoveto{\pgfqpoint{0.515000in}{0.499444in}}%
\pgfpathlineto{\pgfqpoint{2.065000in}{0.499444in}}%
\pgfusepath{stroke}%
\end{pgfscope}%
\begin{pgfscope}%
\pgfsetrectcap%
\pgfsetmiterjoin%
\pgfsetlinewidth{0.803000pt}%
\definecolor{currentstroke}{rgb}{0.000000,0.000000,0.000000}%
\pgfsetstrokecolor{currentstroke}%
\pgfsetdash{}{0pt}%
\pgfpathmoveto{\pgfqpoint{0.515000in}{1.654444in}}%
\pgfpathlineto{\pgfqpoint{2.065000in}{1.654444in}}%
\pgfusepath{stroke}%
\end{pgfscope}%
\begin{pgfscope}%
\pgfsetbuttcap%
\pgfsetmiterjoin%
\definecolor{currentfill}{rgb}{1.000000,1.000000,1.000000}%
\pgfsetfillcolor{currentfill}%
\pgfsetfillopacity{0.800000}%
\pgfsetlinewidth{1.003750pt}%
\definecolor{currentstroke}{rgb}{0.800000,0.800000,0.800000}%
\pgfsetstrokecolor{currentstroke}%
\pgfsetstrokeopacity{0.800000}%
\pgfsetdash{}{0pt}%
\pgfpathmoveto{\pgfqpoint{1.288056in}{1.154445in}}%
\pgfpathlineto{\pgfqpoint{1.967778in}{1.154445in}}%
\pgfpathquadraticcurveto{\pgfqpoint{1.995556in}{1.154445in}}{\pgfqpoint{1.995556in}{1.182222in}}%
\pgfpathlineto{\pgfqpoint{1.995556in}{1.557222in}}%
\pgfpathquadraticcurveto{\pgfqpoint{1.995556in}{1.585000in}}{\pgfqpoint{1.967778in}{1.585000in}}%
\pgfpathlineto{\pgfqpoint{1.288056in}{1.585000in}}%
\pgfpathquadraticcurveto{\pgfqpoint{1.260278in}{1.585000in}}{\pgfqpoint{1.260278in}{1.557222in}}%
\pgfpathlineto{\pgfqpoint{1.260278in}{1.182222in}}%
\pgfpathquadraticcurveto{\pgfqpoint{1.260278in}{1.154445in}}{\pgfqpoint{1.288056in}{1.154445in}}%
\pgfpathlineto{\pgfqpoint{1.288056in}{1.154445in}}%
\pgfpathclose%
\pgfusepath{stroke,fill}%
\end{pgfscope}%
\begin{pgfscope}%
\pgfsetbuttcap%
\pgfsetmiterjoin%
\pgfsetlinewidth{1.003750pt}%
\definecolor{currentstroke}{rgb}{0.000000,0.000000,0.000000}%
\pgfsetstrokecolor{currentstroke}%
\pgfsetdash{}{0pt}%
\pgfpathmoveto{\pgfqpoint{1.315834in}{1.432222in}}%
\pgfpathlineto{\pgfqpoint{1.593611in}{1.432222in}}%
\pgfpathlineto{\pgfqpoint{1.593611in}{1.529444in}}%
\pgfpathlineto{\pgfqpoint{1.315834in}{1.529444in}}%
\pgfpathlineto{\pgfqpoint{1.315834in}{1.432222in}}%
\pgfpathclose%
\pgfusepath{stroke}%
\end{pgfscope}%
\begin{pgfscope}%
\definecolor{textcolor}{rgb}{0.000000,0.000000,0.000000}%
\pgfsetstrokecolor{textcolor}%
\pgfsetfillcolor{textcolor}%
\pgftext[x=1.704722in,y=1.432222in,left,base]{\color{textcolor}\rmfamily\fontsize{10.000000}{12.000000}\selectfont Neg}%
\end{pgfscope}%
\begin{pgfscope}%
\pgfsetbuttcap%
\pgfsetmiterjoin%
\definecolor{currentfill}{rgb}{0.000000,0.000000,0.000000}%
\pgfsetfillcolor{currentfill}%
\pgfsetlinewidth{0.000000pt}%
\definecolor{currentstroke}{rgb}{0.000000,0.000000,0.000000}%
\pgfsetstrokecolor{currentstroke}%
\pgfsetstrokeopacity{0.000000}%
\pgfsetdash{}{0pt}%
\pgfpathmoveto{\pgfqpoint{1.315834in}{1.236944in}}%
\pgfpathlineto{\pgfqpoint{1.593611in}{1.236944in}}%
\pgfpathlineto{\pgfqpoint{1.593611in}{1.334167in}}%
\pgfpathlineto{\pgfqpoint{1.315834in}{1.334167in}}%
\pgfpathlineto{\pgfqpoint{1.315834in}{1.236944in}}%
\pgfpathclose%
\pgfusepath{fill}%
\end{pgfscope}%
\begin{pgfscope}%
\definecolor{textcolor}{rgb}{0.000000,0.000000,0.000000}%
\pgfsetstrokecolor{textcolor}%
\pgfsetfillcolor{textcolor}%
\pgftext[x=1.704722in,y=1.236944in,left,base]{\color{textcolor}\rmfamily\fontsize{10.000000}{12.000000}\selectfont Pos}%
\end{pgfscope}%
\end{pgfpicture}%
\makeatother%
\endgroup%

&
	\vskip 0pt
	\qquad \qquad ROC Curve
	
	%% Creator: Matplotlib, PGF backend
%%
%% To include the figure in your LaTeX document, write
%%   \input{<filename>.pgf}
%%
%% Make sure the required packages are loaded in your preamble
%%   \usepackage{pgf}
%%
%% Also ensure that all the required font packages are loaded; for instance,
%% the lmodern package is sometimes necessary when using math font.
%%   \usepackage{lmodern}
%%
%% Figures using additional raster images can only be included by \input if
%% they are in the same directory as the main LaTeX file. For loading figures
%% from other directories you can use the `import` package
%%   \usepackage{import}
%%
%% and then include the figures with
%%   \import{<path to file>}{<filename>.pgf}
%%
%% Matplotlib used the following preamble
%%   
%%   \usepackage{fontspec}
%%   \makeatletter\@ifpackageloaded{underscore}{}{\usepackage[strings]{underscore}}\makeatother
%%
\begingroup%
\makeatletter%
\begin{pgfpicture}%
\pgfpathrectangle{\pgfpointorigin}{\pgfqpoint{2.221861in}{1.754444in}}%
\pgfusepath{use as bounding box, clip}%
\begin{pgfscope}%
\pgfsetbuttcap%
\pgfsetmiterjoin%
\definecolor{currentfill}{rgb}{1.000000,1.000000,1.000000}%
\pgfsetfillcolor{currentfill}%
\pgfsetlinewidth{0.000000pt}%
\definecolor{currentstroke}{rgb}{1.000000,1.000000,1.000000}%
\pgfsetstrokecolor{currentstroke}%
\pgfsetdash{}{0pt}%
\pgfpathmoveto{\pgfqpoint{0.000000in}{0.000000in}}%
\pgfpathlineto{\pgfqpoint{2.221861in}{0.000000in}}%
\pgfpathlineto{\pgfqpoint{2.221861in}{1.754444in}}%
\pgfpathlineto{\pgfqpoint{0.000000in}{1.754444in}}%
\pgfpathlineto{\pgfqpoint{0.000000in}{0.000000in}}%
\pgfpathclose%
\pgfusepath{fill}%
\end{pgfscope}%
\begin{pgfscope}%
\pgfsetbuttcap%
\pgfsetmiterjoin%
\definecolor{currentfill}{rgb}{1.000000,1.000000,1.000000}%
\pgfsetfillcolor{currentfill}%
\pgfsetlinewidth{0.000000pt}%
\definecolor{currentstroke}{rgb}{0.000000,0.000000,0.000000}%
\pgfsetstrokecolor{currentstroke}%
\pgfsetstrokeopacity{0.000000}%
\pgfsetdash{}{0pt}%
\pgfpathmoveto{\pgfqpoint{0.553581in}{0.499444in}}%
\pgfpathlineto{\pgfqpoint{2.103581in}{0.499444in}}%
\pgfpathlineto{\pgfqpoint{2.103581in}{1.654444in}}%
\pgfpathlineto{\pgfqpoint{0.553581in}{1.654444in}}%
\pgfpathlineto{\pgfqpoint{0.553581in}{0.499444in}}%
\pgfpathclose%
\pgfusepath{fill}%
\end{pgfscope}%
\begin{pgfscope}%
\pgfsetbuttcap%
\pgfsetroundjoin%
\definecolor{currentfill}{rgb}{0.000000,0.000000,0.000000}%
\pgfsetfillcolor{currentfill}%
\pgfsetlinewidth{0.803000pt}%
\definecolor{currentstroke}{rgb}{0.000000,0.000000,0.000000}%
\pgfsetstrokecolor{currentstroke}%
\pgfsetdash{}{0pt}%
\pgfsys@defobject{currentmarker}{\pgfqpoint{0.000000in}{-0.048611in}}{\pgfqpoint{0.000000in}{0.000000in}}{%
\pgfpathmoveto{\pgfqpoint{0.000000in}{0.000000in}}%
\pgfpathlineto{\pgfqpoint{0.000000in}{-0.048611in}}%
\pgfusepath{stroke,fill}%
}%
\begin{pgfscope}%
\pgfsys@transformshift{0.624035in}{0.499444in}%
\pgfsys@useobject{currentmarker}{}%
\end{pgfscope}%
\end{pgfscope}%
\begin{pgfscope}%
\definecolor{textcolor}{rgb}{0.000000,0.000000,0.000000}%
\pgfsetstrokecolor{textcolor}%
\pgfsetfillcolor{textcolor}%
\pgftext[x=0.624035in,y=0.402222in,,top]{\color{textcolor}\rmfamily\fontsize{10.000000}{12.000000}\selectfont \(\displaystyle {0.0}\)}%
\end{pgfscope}%
\begin{pgfscope}%
\pgfsetbuttcap%
\pgfsetroundjoin%
\definecolor{currentfill}{rgb}{0.000000,0.000000,0.000000}%
\pgfsetfillcolor{currentfill}%
\pgfsetlinewidth{0.803000pt}%
\definecolor{currentstroke}{rgb}{0.000000,0.000000,0.000000}%
\pgfsetstrokecolor{currentstroke}%
\pgfsetdash{}{0pt}%
\pgfsys@defobject{currentmarker}{\pgfqpoint{0.000000in}{-0.048611in}}{\pgfqpoint{0.000000in}{0.000000in}}{%
\pgfpathmoveto{\pgfqpoint{0.000000in}{0.000000in}}%
\pgfpathlineto{\pgfqpoint{0.000000in}{-0.048611in}}%
\pgfusepath{stroke,fill}%
}%
\begin{pgfscope}%
\pgfsys@transformshift{1.328581in}{0.499444in}%
\pgfsys@useobject{currentmarker}{}%
\end{pgfscope}%
\end{pgfscope}%
\begin{pgfscope}%
\definecolor{textcolor}{rgb}{0.000000,0.000000,0.000000}%
\pgfsetstrokecolor{textcolor}%
\pgfsetfillcolor{textcolor}%
\pgftext[x=1.328581in,y=0.402222in,,top]{\color{textcolor}\rmfamily\fontsize{10.000000}{12.000000}\selectfont \(\displaystyle {0.5}\)}%
\end{pgfscope}%
\begin{pgfscope}%
\pgfsetbuttcap%
\pgfsetroundjoin%
\definecolor{currentfill}{rgb}{0.000000,0.000000,0.000000}%
\pgfsetfillcolor{currentfill}%
\pgfsetlinewidth{0.803000pt}%
\definecolor{currentstroke}{rgb}{0.000000,0.000000,0.000000}%
\pgfsetstrokecolor{currentstroke}%
\pgfsetdash{}{0pt}%
\pgfsys@defobject{currentmarker}{\pgfqpoint{0.000000in}{-0.048611in}}{\pgfqpoint{0.000000in}{0.000000in}}{%
\pgfpathmoveto{\pgfqpoint{0.000000in}{0.000000in}}%
\pgfpathlineto{\pgfqpoint{0.000000in}{-0.048611in}}%
\pgfusepath{stroke,fill}%
}%
\begin{pgfscope}%
\pgfsys@transformshift{2.033126in}{0.499444in}%
\pgfsys@useobject{currentmarker}{}%
\end{pgfscope}%
\end{pgfscope}%
\begin{pgfscope}%
\definecolor{textcolor}{rgb}{0.000000,0.000000,0.000000}%
\pgfsetstrokecolor{textcolor}%
\pgfsetfillcolor{textcolor}%
\pgftext[x=2.033126in,y=0.402222in,,top]{\color{textcolor}\rmfamily\fontsize{10.000000}{12.000000}\selectfont \(\displaystyle {1.0}\)}%
\end{pgfscope}%
\begin{pgfscope}%
\definecolor{textcolor}{rgb}{0.000000,0.000000,0.000000}%
\pgfsetstrokecolor{textcolor}%
\pgfsetfillcolor{textcolor}%
\pgftext[x=1.328581in,y=0.223333in,,top]{\color{textcolor}\rmfamily\fontsize{10.000000}{12.000000}\selectfont False positive rate}%
\end{pgfscope}%
\begin{pgfscope}%
\pgfsetbuttcap%
\pgfsetroundjoin%
\definecolor{currentfill}{rgb}{0.000000,0.000000,0.000000}%
\pgfsetfillcolor{currentfill}%
\pgfsetlinewidth{0.803000pt}%
\definecolor{currentstroke}{rgb}{0.000000,0.000000,0.000000}%
\pgfsetstrokecolor{currentstroke}%
\pgfsetdash{}{0pt}%
\pgfsys@defobject{currentmarker}{\pgfqpoint{-0.048611in}{0.000000in}}{\pgfqpoint{-0.000000in}{0.000000in}}{%
\pgfpathmoveto{\pgfqpoint{-0.000000in}{0.000000in}}%
\pgfpathlineto{\pgfqpoint{-0.048611in}{0.000000in}}%
\pgfusepath{stroke,fill}%
}%
\begin{pgfscope}%
\pgfsys@transformshift{0.553581in}{0.551944in}%
\pgfsys@useobject{currentmarker}{}%
\end{pgfscope}%
\end{pgfscope}%
\begin{pgfscope}%
\definecolor{textcolor}{rgb}{0.000000,0.000000,0.000000}%
\pgfsetstrokecolor{textcolor}%
\pgfsetfillcolor{textcolor}%
\pgftext[x=0.278889in, y=0.503750in, left, base]{\color{textcolor}\rmfamily\fontsize{10.000000}{12.000000}\selectfont \(\displaystyle {0.0}\)}%
\end{pgfscope}%
\begin{pgfscope}%
\pgfsetbuttcap%
\pgfsetroundjoin%
\definecolor{currentfill}{rgb}{0.000000,0.000000,0.000000}%
\pgfsetfillcolor{currentfill}%
\pgfsetlinewidth{0.803000pt}%
\definecolor{currentstroke}{rgb}{0.000000,0.000000,0.000000}%
\pgfsetstrokecolor{currentstroke}%
\pgfsetdash{}{0pt}%
\pgfsys@defobject{currentmarker}{\pgfqpoint{-0.048611in}{0.000000in}}{\pgfqpoint{-0.000000in}{0.000000in}}{%
\pgfpathmoveto{\pgfqpoint{-0.000000in}{0.000000in}}%
\pgfpathlineto{\pgfqpoint{-0.048611in}{0.000000in}}%
\pgfusepath{stroke,fill}%
}%
\begin{pgfscope}%
\pgfsys@transformshift{0.553581in}{1.076944in}%
\pgfsys@useobject{currentmarker}{}%
\end{pgfscope}%
\end{pgfscope}%
\begin{pgfscope}%
\definecolor{textcolor}{rgb}{0.000000,0.000000,0.000000}%
\pgfsetstrokecolor{textcolor}%
\pgfsetfillcolor{textcolor}%
\pgftext[x=0.278889in, y=1.028750in, left, base]{\color{textcolor}\rmfamily\fontsize{10.000000}{12.000000}\selectfont \(\displaystyle {0.5}\)}%
\end{pgfscope}%
\begin{pgfscope}%
\pgfsetbuttcap%
\pgfsetroundjoin%
\definecolor{currentfill}{rgb}{0.000000,0.000000,0.000000}%
\pgfsetfillcolor{currentfill}%
\pgfsetlinewidth{0.803000pt}%
\definecolor{currentstroke}{rgb}{0.000000,0.000000,0.000000}%
\pgfsetstrokecolor{currentstroke}%
\pgfsetdash{}{0pt}%
\pgfsys@defobject{currentmarker}{\pgfqpoint{-0.048611in}{0.000000in}}{\pgfqpoint{-0.000000in}{0.000000in}}{%
\pgfpathmoveto{\pgfqpoint{-0.000000in}{0.000000in}}%
\pgfpathlineto{\pgfqpoint{-0.048611in}{0.000000in}}%
\pgfusepath{stroke,fill}%
}%
\begin{pgfscope}%
\pgfsys@transformshift{0.553581in}{1.601944in}%
\pgfsys@useobject{currentmarker}{}%
\end{pgfscope}%
\end{pgfscope}%
\begin{pgfscope}%
\definecolor{textcolor}{rgb}{0.000000,0.000000,0.000000}%
\pgfsetstrokecolor{textcolor}%
\pgfsetfillcolor{textcolor}%
\pgftext[x=0.278889in, y=1.553750in, left, base]{\color{textcolor}\rmfamily\fontsize{10.000000}{12.000000}\selectfont \(\displaystyle {1.0}\)}%
\end{pgfscope}%
\begin{pgfscope}%
\definecolor{textcolor}{rgb}{0.000000,0.000000,0.000000}%
\pgfsetstrokecolor{textcolor}%
\pgfsetfillcolor{textcolor}%
\pgftext[x=0.223333in,y=1.076944in,,bottom,rotate=90.000000]{\color{textcolor}\rmfamily\fontsize{10.000000}{12.000000}\selectfont True positive rate}%
\end{pgfscope}%
\begin{pgfscope}%
\pgfpathrectangle{\pgfqpoint{0.553581in}{0.499444in}}{\pgfqpoint{1.550000in}{1.155000in}}%
\pgfusepath{clip}%
\pgfsetbuttcap%
\pgfsetroundjoin%
\pgfsetlinewidth{1.505625pt}%
\definecolor{currentstroke}{rgb}{0.000000,0.000000,0.000000}%
\pgfsetstrokecolor{currentstroke}%
\pgfsetdash{{5.550000pt}{2.400000pt}}{0.000000pt}%
\pgfpathmoveto{\pgfqpoint{0.624035in}{0.551944in}}%
\pgfpathlineto{\pgfqpoint{2.033126in}{1.601944in}}%
\pgfusepath{stroke}%
\end{pgfscope}%
\begin{pgfscope}%
\pgfpathrectangle{\pgfqpoint{0.553581in}{0.499444in}}{\pgfqpoint{1.550000in}{1.155000in}}%
\pgfusepath{clip}%
\pgfsetrectcap%
\pgfsetroundjoin%
\pgfsetlinewidth{1.505625pt}%
\definecolor{currentstroke}{rgb}{0.000000,0.000000,0.000000}%
\pgfsetstrokecolor{currentstroke}%
\pgfsetdash{}{0pt}%
\pgfpathmoveto{\pgfqpoint{0.624035in}{0.551944in}}%
\pgfpathlineto{\pgfqpoint{0.626207in}{0.552574in}}%
\pgfpathlineto{\pgfqpoint{0.627318in}{0.561464in}}%
\pgfpathlineto{\pgfqpoint{0.628014in}{0.562514in}}%
\pgfpathlineto{\pgfqpoint{0.629125in}{0.563634in}}%
\pgfpathlineto{\pgfqpoint{0.629605in}{0.564614in}}%
\pgfpathlineto{\pgfqpoint{0.630699in}{0.567694in}}%
\pgfpathlineto{\pgfqpoint{0.631130in}{0.568604in}}%
\pgfpathlineto{\pgfqpoint{0.632225in}{0.573294in}}%
\pgfpathlineto{\pgfqpoint{0.632772in}{0.574064in}}%
\pgfpathlineto{\pgfqpoint{0.633866in}{0.579944in}}%
\pgfpathlineto{\pgfqpoint{0.634247in}{0.580854in}}%
\pgfpathlineto{\pgfqpoint{0.635358in}{0.585824in}}%
\pgfpathlineto{\pgfqpoint{0.635540in}{0.586874in}}%
\pgfpathlineto{\pgfqpoint{0.636634in}{0.593244in}}%
\pgfpathlineto{\pgfqpoint{0.636833in}{0.594154in}}%
\pgfpathlineto{\pgfqpoint{0.637944in}{0.600734in}}%
\pgfpathlineto{\pgfqpoint{0.638126in}{0.601784in}}%
\pgfpathlineto{\pgfqpoint{0.639187in}{0.608714in}}%
\pgfpathlineto{\pgfqpoint{0.639353in}{0.609694in}}%
\pgfpathlineto{\pgfqpoint{0.640464in}{0.616834in}}%
\pgfpathlineto{\pgfqpoint{0.640712in}{0.617884in}}%
\pgfpathlineto{\pgfqpoint{0.641806in}{0.625864in}}%
\pgfpathlineto{\pgfqpoint{0.642038in}{0.626844in}}%
\pgfpathlineto{\pgfqpoint{0.643149in}{0.635804in}}%
\pgfpathlineto{\pgfqpoint{0.643348in}{0.636854in}}%
\pgfpathlineto{\pgfqpoint{0.644392in}{0.645044in}}%
\pgfpathlineto{\pgfqpoint{0.644641in}{0.645884in}}%
\pgfpathlineto{\pgfqpoint{0.645752in}{0.652394in}}%
\pgfpathlineto{\pgfqpoint{0.645868in}{0.653164in}}%
\pgfpathlineto{\pgfqpoint{0.646979in}{0.661424in}}%
\pgfpathlineto{\pgfqpoint{0.647327in}{0.662264in}}%
\pgfpathlineto{\pgfqpoint{0.648437in}{0.668914in}}%
\pgfpathlineto{\pgfqpoint{0.648587in}{0.669824in}}%
\pgfpathlineto{\pgfqpoint{0.649697in}{0.678294in}}%
\pgfpathlineto{\pgfqpoint{0.649813in}{0.679274in}}%
\pgfpathlineto{\pgfqpoint{0.650924in}{0.687044in}}%
\pgfpathlineto{\pgfqpoint{0.651140in}{0.687884in}}%
\pgfpathlineto{\pgfqpoint{0.652250in}{0.696144in}}%
\pgfpathlineto{\pgfqpoint{0.652333in}{0.697194in}}%
\pgfpathlineto{\pgfqpoint{0.653444in}{0.706434in}}%
\pgfpathlineto{\pgfqpoint{0.653858in}{0.707414in}}%
\pgfpathlineto{\pgfqpoint{0.654969in}{0.716584in}}%
\pgfpathlineto{\pgfqpoint{0.655151in}{0.717564in}}%
\pgfpathlineto{\pgfqpoint{0.656262in}{0.725194in}}%
\pgfpathlineto{\pgfqpoint{0.656444in}{0.726174in}}%
\pgfpathlineto{\pgfqpoint{0.657538in}{0.733244in}}%
\pgfpathlineto{\pgfqpoint{0.657704in}{0.734294in}}%
\pgfpathlineto{\pgfqpoint{0.658815in}{0.743464in}}%
\pgfpathlineto{\pgfqpoint{0.659064in}{0.744514in}}%
\pgfpathlineto{\pgfqpoint{0.660158in}{0.751164in}}%
\pgfpathlineto{\pgfqpoint{0.660290in}{0.751794in}}%
\pgfpathlineto{\pgfqpoint{0.661401in}{0.759354in}}%
\pgfpathlineto{\pgfqpoint{0.661633in}{0.760404in}}%
\pgfpathlineto{\pgfqpoint{0.662744in}{0.769784in}}%
\pgfpathlineto{\pgfqpoint{0.662827in}{0.770344in}}%
\pgfpathlineto{\pgfqpoint{0.663937in}{0.777764in}}%
\pgfpathlineto{\pgfqpoint{0.664120in}{0.778814in}}%
\pgfpathlineto{\pgfqpoint{0.665181in}{0.783854in}}%
\pgfpathlineto{\pgfqpoint{0.665297in}{0.784834in}}%
\pgfpathlineto{\pgfqpoint{0.666407in}{0.795614in}}%
\pgfpathlineto{\pgfqpoint{0.666656in}{0.796664in}}%
\pgfpathlineto{\pgfqpoint{0.667767in}{0.801564in}}%
\pgfpathlineto{\pgfqpoint{0.667899in}{0.802404in}}%
\pgfpathlineto{\pgfqpoint{0.669010in}{0.809334in}}%
\pgfpathlineto{\pgfqpoint{0.669342in}{0.810384in}}%
\pgfpathlineto{\pgfqpoint{0.670452in}{0.818084in}}%
\pgfpathlineto{\pgfqpoint{0.670701in}{0.819134in}}%
\pgfpathlineto{\pgfqpoint{0.671812in}{0.825364in}}%
\pgfpathlineto{\pgfqpoint{0.671911in}{0.826414in}}%
\pgfpathlineto{\pgfqpoint{0.673022in}{0.833694in}}%
\pgfpathlineto{\pgfqpoint{0.673138in}{0.834744in}}%
\pgfpathlineto{\pgfqpoint{0.674215in}{0.838944in}}%
\pgfpathlineto{\pgfqpoint{0.674530in}{0.839994in}}%
\pgfpathlineto{\pgfqpoint{0.675608in}{0.846014in}}%
\pgfpathlineto{\pgfqpoint{0.675790in}{0.846994in}}%
\pgfpathlineto{\pgfqpoint{0.676901in}{0.853854in}}%
\pgfpathlineto{\pgfqpoint{0.677199in}{0.854834in}}%
\pgfpathlineto{\pgfqpoint{0.678310in}{0.859034in}}%
\pgfpathlineto{\pgfqpoint{0.678492in}{0.860084in}}%
\pgfpathlineto{\pgfqpoint{0.679587in}{0.867014in}}%
\pgfpathlineto{\pgfqpoint{0.679868in}{0.867924in}}%
\pgfpathlineto{\pgfqpoint{0.680963in}{0.874504in}}%
\pgfpathlineto{\pgfqpoint{0.681228in}{0.875484in}}%
\pgfpathlineto{\pgfqpoint{0.682338in}{0.882204in}}%
\pgfpathlineto{\pgfqpoint{0.682587in}{0.883254in}}%
\pgfpathlineto{\pgfqpoint{0.683681in}{0.889484in}}%
\pgfpathlineto{\pgfqpoint{0.683980in}{0.890394in}}%
\pgfpathlineto{\pgfqpoint{0.685074in}{0.897394in}}%
\pgfpathlineto{\pgfqpoint{0.685389in}{0.898304in}}%
\pgfpathlineto{\pgfqpoint{0.686450in}{0.904184in}}%
\pgfpathlineto{\pgfqpoint{0.686748in}{0.905094in}}%
\pgfpathlineto{\pgfqpoint{0.687859in}{0.910204in}}%
\pgfpathlineto{\pgfqpoint{0.688074in}{0.911114in}}%
\pgfpathlineto{\pgfqpoint{0.689185in}{0.915034in}}%
\pgfpathlineto{\pgfqpoint{0.689400in}{0.916084in}}%
\pgfpathlineto{\pgfqpoint{0.690511in}{0.922664in}}%
\pgfpathlineto{\pgfqpoint{0.690793in}{0.923714in}}%
\pgfpathlineto{\pgfqpoint{0.691887in}{0.928754in}}%
\pgfpathlineto{\pgfqpoint{0.692318in}{0.929734in}}%
\pgfpathlineto{\pgfqpoint{0.693412in}{0.935334in}}%
\pgfpathlineto{\pgfqpoint{0.693611in}{0.936384in}}%
\pgfpathlineto{\pgfqpoint{0.694705in}{0.940864in}}%
\pgfpathlineto{\pgfqpoint{0.695053in}{0.941914in}}%
\pgfpathlineto{\pgfqpoint{0.696164in}{0.948634in}}%
\pgfpathlineto{\pgfqpoint{0.696645in}{0.949684in}}%
\pgfpathlineto{\pgfqpoint{0.697756in}{0.954164in}}%
\pgfpathlineto{\pgfqpoint{0.697921in}{0.955144in}}%
\pgfpathlineto{\pgfqpoint{0.699015in}{0.960464in}}%
\pgfpathlineto{\pgfqpoint{0.699314in}{0.961304in}}%
\pgfpathlineto{\pgfqpoint{0.700425in}{0.965294in}}%
\pgfpathlineto{\pgfqpoint{0.700657in}{0.966344in}}%
\pgfpathlineto{\pgfqpoint{0.701767in}{0.971314in}}%
\pgfpathlineto{\pgfqpoint{0.702049in}{0.972364in}}%
\pgfpathlineto{\pgfqpoint{0.703094in}{0.977194in}}%
\pgfpathlineto{\pgfqpoint{0.703442in}{0.978104in}}%
\pgfpathlineto{\pgfqpoint{0.704536in}{0.982514in}}%
\pgfpathlineto{\pgfqpoint{0.704917in}{0.983564in}}%
\pgfpathlineto{\pgfqpoint{0.706011in}{0.988464in}}%
\pgfpathlineto{\pgfqpoint{0.706160in}{0.989444in}}%
\pgfpathlineto{\pgfqpoint{0.707271in}{0.993504in}}%
\pgfpathlineto{\pgfqpoint{0.707619in}{0.994414in}}%
\pgfpathlineto{\pgfqpoint{0.708697in}{0.998054in}}%
\pgfpathlineto{\pgfqpoint{0.709061in}{0.999104in}}%
\pgfpathlineto{\pgfqpoint{0.710172in}{1.004564in}}%
\pgfpathlineto{\pgfqpoint{0.710504in}{1.005614in}}%
\pgfpathlineto{\pgfqpoint{0.711614in}{1.009604in}}%
\pgfpathlineto{\pgfqpoint{0.711830in}{1.010584in}}%
\pgfpathlineto{\pgfqpoint{0.712874in}{1.015134in}}%
\pgfpathlineto{\pgfqpoint{0.713123in}{1.016184in}}%
\pgfpathlineto{\pgfqpoint{0.714184in}{1.020244in}}%
\pgfpathlineto{\pgfqpoint{0.714598in}{1.021294in}}%
\pgfpathlineto{\pgfqpoint{0.715709in}{1.024304in}}%
\pgfpathlineto{\pgfqpoint{0.716173in}{1.025354in}}%
\pgfpathlineto{\pgfqpoint{0.717284in}{1.029624in}}%
\pgfpathlineto{\pgfqpoint{0.717466in}{1.030604in}}%
\pgfpathlineto{\pgfqpoint{0.718560in}{1.035014in}}%
\pgfpathlineto{\pgfqpoint{0.718776in}{1.035994in}}%
\pgfpathlineto{\pgfqpoint{0.719853in}{1.041034in}}%
\pgfpathlineto{\pgfqpoint{0.720202in}{1.042084in}}%
\pgfpathlineto{\pgfqpoint{0.721312in}{1.046144in}}%
\pgfpathlineto{\pgfqpoint{0.721694in}{1.047194in}}%
\pgfpathlineto{\pgfqpoint{0.722788in}{1.050904in}}%
\pgfpathlineto{\pgfqpoint{0.723086in}{1.051954in}}%
\pgfpathlineto{\pgfqpoint{0.724180in}{1.056364in}}%
\pgfpathlineto{\pgfqpoint{0.724727in}{1.057414in}}%
\pgfpathlineto{\pgfqpoint{0.725805in}{1.061264in}}%
\pgfpathlineto{\pgfqpoint{0.726236in}{1.062314in}}%
\pgfpathlineto{\pgfqpoint{0.727280in}{1.065744in}}%
\pgfpathlineto{\pgfqpoint{0.727728in}{1.066794in}}%
\pgfpathlineto{\pgfqpoint{0.728805in}{1.069874in}}%
\pgfpathlineto{\pgfqpoint{0.729253in}{1.070924in}}%
\pgfpathlineto{\pgfqpoint{0.730364in}{1.074144in}}%
\pgfpathlineto{\pgfqpoint{0.730695in}{1.075124in}}%
\pgfpathlineto{\pgfqpoint{0.731789in}{1.079044in}}%
\pgfpathlineto{\pgfqpoint{0.732336in}{1.080024in}}%
\pgfpathlineto{\pgfqpoint{0.733447in}{1.084574in}}%
\pgfpathlineto{\pgfqpoint{0.734094in}{1.085624in}}%
\pgfpathlineto{\pgfqpoint{0.735188in}{1.089544in}}%
\pgfpathlineto{\pgfqpoint{0.735552in}{1.090594in}}%
\pgfpathlineto{\pgfqpoint{0.736597in}{1.093674in}}%
\pgfpathlineto{\pgfqpoint{0.737177in}{1.094724in}}%
\pgfpathlineto{\pgfqpoint{0.738221in}{1.097524in}}%
\pgfpathlineto{\pgfqpoint{0.738702in}{1.098574in}}%
\pgfpathlineto{\pgfqpoint{0.739780in}{1.102424in}}%
\pgfpathlineto{\pgfqpoint{0.740128in}{1.103474in}}%
\pgfpathlineto{\pgfqpoint{0.741238in}{1.106624in}}%
\pgfpathlineto{\pgfqpoint{0.741636in}{1.107674in}}%
\pgfpathlineto{\pgfqpoint{0.742697in}{1.111034in}}%
\pgfpathlineto{\pgfqpoint{0.743095in}{1.112084in}}%
\pgfpathlineto{\pgfqpoint{0.744206in}{1.115024in}}%
\pgfpathlineto{\pgfqpoint{0.744769in}{1.116074in}}%
\pgfpathlineto{\pgfqpoint{0.745051in}{1.117054in}}%
\pgfpathlineto{\pgfqpoint{0.745068in}{1.117054in}}%
\pgfpathlineto{\pgfqpoint{0.755943in}{1.118104in}}%
\pgfpathlineto{\pgfqpoint{0.757053in}{1.121814in}}%
\pgfpathlineto{\pgfqpoint{0.757600in}{1.122864in}}%
\pgfpathlineto{\pgfqpoint{0.758678in}{1.125384in}}%
\pgfpathlineto{\pgfqpoint{0.759225in}{1.126434in}}%
\pgfpathlineto{\pgfqpoint{0.760269in}{1.129094in}}%
\pgfpathlineto{\pgfqpoint{0.760817in}{1.130144in}}%
\pgfpathlineto{\pgfqpoint{0.761911in}{1.133224in}}%
\pgfpathlineto{\pgfqpoint{0.762474in}{1.134274in}}%
\pgfpathlineto{\pgfqpoint{0.763535in}{1.137844in}}%
\pgfpathlineto{\pgfqpoint{0.764165in}{1.138824in}}%
\pgfpathlineto{\pgfqpoint{0.765259in}{1.142324in}}%
\pgfpathlineto{\pgfqpoint{0.766055in}{1.143374in}}%
\pgfpathlineto{\pgfqpoint{0.767166in}{1.146314in}}%
\pgfpathlineto{\pgfqpoint{0.767696in}{1.147364in}}%
\pgfpathlineto{\pgfqpoint{0.768790in}{1.149954in}}%
\pgfpathlineto{\pgfqpoint{0.769603in}{1.151004in}}%
\pgfpathlineto{\pgfqpoint{0.770514in}{1.153314in}}%
\pgfpathlineto{\pgfqpoint{0.771161in}{1.154364in}}%
\pgfpathlineto{\pgfqpoint{0.772272in}{1.157374in}}%
\pgfpathlineto{\pgfqpoint{0.772852in}{1.158424in}}%
\pgfpathlineto{\pgfqpoint{0.773946in}{1.160664in}}%
\pgfpathlineto{\pgfqpoint{0.774609in}{1.161714in}}%
\pgfpathlineto{\pgfqpoint{0.775703in}{1.164794in}}%
\pgfpathlineto{\pgfqpoint{0.776267in}{1.165774in}}%
\pgfpathlineto{\pgfqpoint{0.777344in}{1.168714in}}%
\pgfpathlineto{\pgfqpoint{0.777759in}{1.169764in}}%
\pgfpathlineto{\pgfqpoint{0.778836in}{1.172144in}}%
\pgfpathlineto{\pgfqpoint{0.779350in}{1.173124in}}%
\pgfpathlineto{\pgfqpoint{0.780444in}{1.174524in}}%
\pgfpathlineto{\pgfqpoint{0.781008in}{1.175574in}}%
\pgfpathlineto{\pgfqpoint{0.782052in}{1.177254in}}%
\pgfpathlineto{\pgfqpoint{0.782699in}{1.178304in}}%
\pgfpathlineto{\pgfqpoint{0.783793in}{1.180894in}}%
\pgfpathlineto{\pgfqpoint{0.784290in}{1.181874in}}%
\pgfpathlineto{\pgfqpoint{0.785384in}{1.184044in}}%
\pgfpathlineto{\pgfqpoint{0.785981in}{1.185094in}}%
\pgfpathlineto{\pgfqpoint{0.787075in}{1.186914in}}%
\pgfpathlineto{\pgfqpoint{0.787755in}{1.187894in}}%
\pgfpathlineto{\pgfqpoint{0.788866in}{1.189784in}}%
\pgfpathlineto{\pgfqpoint{0.789297in}{1.190764in}}%
\pgfpathlineto{\pgfqpoint{0.790407in}{1.192794in}}%
\pgfpathlineto{\pgfqpoint{0.790689in}{1.193634in}}%
\pgfpathlineto{\pgfqpoint{0.791783in}{1.196084in}}%
\pgfpathlineto{\pgfqpoint{0.792347in}{1.197134in}}%
\pgfpathlineto{\pgfqpoint{0.793358in}{1.199724in}}%
\pgfpathlineto{\pgfqpoint{0.794054in}{1.200774in}}%
\pgfpathlineto{\pgfqpoint{0.795099in}{1.201964in}}%
\pgfpathlineto{\pgfqpoint{0.796044in}{1.203014in}}%
\pgfpathlineto{\pgfqpoint{0.797138in}{1.205534in}}%
\pgfpathlineto{\pgfqpoint{0.797718in}{1.206584in}}%
\pgfpathlineto{\pgfqpoint{0.798812in}{1.208474in}}%
\pgfpathlineto{\pgfqpoint{0.799409in}{1.209524in}}%
\pgfpathlineto{\pgfqpoint{0.800520in}{1.212114in}}%
\pgfpathlineto{\pgfqpoint{0.801382in}{1.213094in}}%
\pgfpathlineto{\pgfqpoint{0.802492in}{1.215894in}}%
\pgfpathlineto{\pgfqpoint{0.803106in}{1.216664in}}%
\pgfpathlineto{\pgfqpoint{0.804134in}{1.218554in}}%
\pgfpathlineto{\pgfqpoint{0.805128in}{1.219604in}}%
\pgfpathlineto{\pgfqpoint{0.806222in}{1.221774in}}%
\pgfpathlineto{\pgfqpoint{0.806786in}{1.222614in}}%
\pgfpathlineto{\pgfqpoint{0.807897in}{1.225484in}}%
\pgfpathlineto{\pgfqpoint{0.808659in}{1.226534in}}%
\pgfpathlineto{\pgfqpoint{0.809753in}{1.228284in}}%
\pgfpathlineto{\pgfqpoint{0.810267in}{1.229334in}}%
\pgfpathlineto{\pgfqpoint{0.811378in}{1.231224in}}%
\pgfpathlineto{\pgfqpoint{0.812654in}{1.232274in}}%
\pgfpathlineto{\pgfqpoint{0.813682in}{1.233744in}}%
\pgfpathlineto{\pgfqpoint{0.814180in}{1.234794in}}%
\pgfpathlineto{\pgfqpoint{0.815290in}{1.236614in}}%
\pgfpathlineto{\pgfqpoint{0.816053in}{1.237664in}}%
\pgfpathlineto{\pgfqpoint{0.816948in}{1.239484in}}%
\pgfpathlineto{\pgfqpoint{0.818109in}{1.240464in}}%
\pgfpathlineto{\pgfqpoint{0.819203in}{1.242424in}}%
\pgfpathlineto{\pgfqpoint{0.819949in}{1.243404in}}%
\pgfpathlineto{\pgfqpoint{0.821043in}{1.245364in}}%
\pgfpathlineto{\pgfqpoint{0.821838in}{1.246414in}}%
\pgfpathlineto{\pgfqpoint{0.822899in}{1.248374in}}%
\pgfpathlineto{\pgfqpoint{0.823828in}{1.249424in}}%
\pgfpathlineto{\pgfqpoint{0.824905in}{1.251314in}}%
\pgfpathlineto{\pgfqpoint{0.826049in}{1.252364in}}%
\pgfpathlineto{\pgfqpoint{0.827110in}{1.254044in}}%
\pgfpathlineto{\pgfqpoint{0.828204in}{1.255024in}}%
\pgfpathlineto{\pgfqpoint{0.829298in}{1.257404in}}%
\pgfpathlineto{\pgfqpoint{0.830094in}{1.258454in}}%
\pgfpathlineto{\pgfqpoint{0.831188in}{1.259784in}}%
\pgfpathlineto{\pgfqpoint{0.831901in}{1.260834in}}%
\pgfpathlineto{\pgfqpoint{0.832979in}{1.261954in}}%
\pgfpathlineto{\pgfqpoint{0.833841in}{1.263004in}}%
\pgfpathlineto{\pgfqpoint{0.834769in}{1.263774in}}%
\pgfpathlineto{\pgfqpoint{0.835697in}{1.264824in}}%
\pgfpathlineto{\pgfqpoint{0.836659in}{1.266014in}}%
\pgfpathlineto{\pgfqpoint{0.838234in}{1.267064in}}%
\pgfpathlineto{\pgfqpoint{0.839295in}{1.268814in}}%
\pgfpathlineto{\pgfqpoint{0.840438in}{1.269864in}}%
\pgfpathlineto{\pgfqpoint{0.841549in}{1.271824in}}%
\pgfpathlineto{\pgfqpoint{0.842594in}{1.272874in}}%
\pgfpathlineto{\pgfqpoint{0.843688in}{1.274414in}}%
\pgfpathlineto{\pgfqpoint{0.844881in}{1.275464in}}%
\pgfpathlineto{\pgfqpoint{0.845992in}{1.277844in}}%
\pgfpathlineto{\pgfqpoint{0.846705in}{1.278894in}}%
\pgfpathlineto{\pgfqpoint{0.847633in}{1.280294in}}%
\pgfpathlineto{\pgfqpoint{0.848644in}{1.281274in}}%
\pgfpathlineto{\pgfqpoint{0.849606in}{1.282954in}}%
\pgfpathlineto{\pgfqpoint{0.850882in}{1.284004in}}%
\pgfpathlineto{\pgfqpoint{0.851960in}{1.285194in}}%
\pgfpathlineto{\pgfqpoint{0.853120in}{1.286244in}}%
\pgfpathlineto{\pgfqpoint{0.854198in}{1.287644in}}%
\pgfpathlineto{\pgfqpoint{0.855176in}{1.288694in}}%
\pgfpathlineto{\pgfqpoint{0.856237in}{1.290234in}}%
\pgfpathlineto{\pgfqpoint{0.856917in}{1.291144in}}%
\pgfpathlineto{\pgfqpoint{0.858027in}{1.292754in}}%
\pgfpathlineto{\pgfqpoint{0.859287in}{1.293804in}}%
\pgfpathlineto{\pgfqpoint{0.860232in}{1.294714in}}%
\pgfpathlineto{\pgfqpoint{0.861160in}{1.295764in}}%
\pgfpathlineto{\pgfqpoint{0.862155in}{1.296814in}}%
\pgfpathlineto{\pgfqpoint{0.863746in}{1.297794in}}%
\pgfpathlineto{\pgfqpoint{0.864807in}{1.299054in}}%
\pgfpathlineto{\pgfqpoint{0.866084in}{1.300104in}}%
\pgfpathlineto{\pgfqpoint{0.867128in}{1.301854in}}%
\pgfpathlineto{\pgfqpoint{0.867775in}{1.302904in}}%
\pgfpathlineto{\pgfqpoint{0.868886in}{1.304164in}}%
\pgfpathlineto{\pgfqpoint{0.869897in}{1.305144in}}%
\pgfpathlineto{\pgfqpoint{0.870991in}{1.306614in}}%
\pgfpathlineto{\pgfqpoint{0.872218in}{1.307664in}}%
\pgfpathlineto{\pgfqpoint{0.873312in}{1.309624in}}%
\pgfpathlineto{\pgfqpoint{0.874572in}{1.310674in}}%
\pgfpathlineto{\pgfqpoint{0.875616in}{1.312284in}}%
\pgfpathlineto{\pgfqpoint{0.876263in}{1.313334in}}%
\pgfpathlineto{\pgfqpoint{0.877373in}{1.314804in}}%
\pgfpathlineto{\pgfqpoint{0.878036in}{1.315854in}}%
\pgfpathlineto{\pgfqpoint{0.879081in}{1.316974in}}%
\pgfpathlineto{\pgfqpoint{0.880191in}{1.318024in}}%
\pgfpathlineto{\pgfqpoint{0.881252in}{1.319564in}}%
\pgfpathlineto{\pgfqpoint{0.882927in}{1.320544in}}%
\pgfpathlineto{\pgfqpoint{0.883954in}{1.321384in}}%
\pgfpathlineto{\pgfqpoint{0.885165in}{1.322434in}}%
\pgfpathlineto{\pgfqpoint{0.886226in}{1.323624in}}%
\pgfpathlineto{\pgfqpoint{0.887121in}{1.324674in}}%
\pgfpathlineto{\pgfqpoint{0.888215in}{1.325864in}}%
\pgfpathlineto{\pgfqpoint{0.889624in}{1.326914in}}%
\pgfpathlineto{\pgfqpoint{0.890619in}{1.328104in}}%
\pgfpathlineto{\pgfqpoint{0.892044in}{1.329154in}}%
\pgfpathlineto{\pgfqpoint{0.892857in}{1.330064in}}%
\pgfpathlineto{\pgfqpoint{0.894100in}{1.331044in}}%
\pgfpathlineto{\pgfqpoint{0.894746in}{1.331814in}}%
\pgfpathlineto{\pgfqpoint{0.896089in}{1.332864in}}%
\pgfpathlineto{\pgfqpoint{0.897200in}{1.333984in}}%
\pgfpathlineto{\pgfqpoint{0.898327in}{1.335034in}}%
\pgfpathlineto{\pgfqpoint{0.899338in}{1.336714in}}%
\pgfpathlineto{\pgfqpoint{0.900267in}{1.337694in}}%
\pgfpathlineto{\pgfqpoint{0.901311in}{1.338534in}}%
\pgfpathlineto{\pgfqpoint{0.902554in}{1.339584in}}%
\pgfpathlineto{\pgfqpoint{0.903649in}{1.340494in}}%
\pgfpathlineto{\pgfqpoint{0.905240in}{1.341544in}}%
\pgfpathlineto{\pgfqpoint{0.906334in}{1.342874in}}%
\pgfpathlineto{\pgfqpoint{0.907743in}{1.343924in}}%
\pgfpathlineto{\pgfqpoint{0.908854in}{1.344764in}}%
\pgfpathlineto{\pgfqpoint{0.910048in}{1.345814in}}%
\pgfpathlineto{\pgfqpoint{0.911142in}{1.346654in}}%
\pgfpathlineto{\pgfqpoint{0.912169in}{1.347704in}}%
\pgfpathlineto{\pgfqpoint{0.913114in}{1.348754in}}%
\pgfpathlineto{\pgfqpoint{0.914242in}{1.349804in}}%
\pgfpathlineto{\pgfqpoint{0.915004in}{1.350924in}}%
\pgfpathlineto{\pgfqpoint{0.916894in}{1.351974in}}%
\pgfpathlineto{\pgfqpoint{0.917988in}{1.353234in}}%
\pgfpathlineto{\pgfqpoint{0.919646in}{1.354284in}}%
\pgfpathlineto{\pgfqpoint{0.920757in}{1.356034in}}%
\pgfpathlineto{\pgfqpoint{0.922166in}{1.357084in}}%
\pgfpathlineto{\pgfqpoint{0.923260in}{1.358204in}}%
\pgfpathlineto{\pgfqpoint{0.925332in}{1.359254in}}%
\pgfpathlineto{\pgfqpoint{0.926376in}{1.360164in}}%
\pgfpathlineto{\pgfqpoint{0.927686in}{1.361214in}}%
\pgfpathlineto{\pgfqpoint{0.928780in}{1.361914in}}%
\pgfpathlineto{\pgfqpoint{0.930886in}{1.362964in}}%
\pgfpathlineto{\pgfqpoint{0.931980in}{1.364294in}}%
\pgfpathlineto{\pgfqpoint{0.932941in}{1.365344in}}%
\pgfpathlineto{\pgfqpoint{0.934035in}{1.366604in}}%
\pgfpathlineto{\pgfqpoint{0.935925in}{1.367654in}}%
\pgfpathlineto{\pgfqpoint{0.937003in}{1.368984in}}%
\pgfpathlineto{\pgfqpoint{0.939307in}{1.370034in}}%
\pgfpathlineto{\pgfqpoint{0.940302in}{1.371014in}}%
\pgfpathlineto{\pgfqpoint{0.941876in}{1.372064in}}%
\pgfpathlineto{\pgfqpoint{0.942871in}{1.372694in}}%
\pgfpathlineto{\pgfqpoint{0.944562in}{1.373674in}}%
\pgfpathlineto{\pgfqpoint{0.945573in}{1.374934in}}%
\pgfpathlineto{\pgfqpoint{0.947314in}{1.375984in}}%
\pgfpathlineto{\pgfqpoint{0.948159in}{1.376824in}}%
\pgfpathlineto{\pgfqpoint{0.950663in}{1.377874in}}%
\pgfpathlineto{\pgfqpoint{0.951707in}{1.378504in}}%
\pgfpathlineto{\pgfqpoint{0.953083in}{1.379484in}}%
\pgfpathlineto{\pgfqpoint{0.954127in}{1.380394in}}%
\pgfpathlineto{\pgfqpoint{0.956183in}{1.381444in}}%
\pgfpathlineto{\pgfqpoint{0.957244in}{1.382914in}}%
\pgfpathlineto{\pgfqpoint{0.958736in}{1.383964in}}%
\pgfpathlineto{\pgfqpoint{0.959714in}{1.384874in}}%
\pgfpathlineto{\pgfqpoint{0.961902in}{1.385924in}}%
\pgfpathlineto{\pgfqpoint{0.962781in}{1.386414in}}%
\pgfpathlineto{\pgfqpoint{0.965599in}{1.387464in}}%
\pgfpathlineto{\pgfqpoint{0.966643in}{1.388514in}}%
\pgfpathlineto{\pgfqpoint{0.968069in}{1.389564in}}%
\pgfpathlineto{\pgfqpoint{0.968931in}{1.390404in}}%
\pgfpathlineto{\pgfqpoint{0.971733in}{1.391454in}}%
\pgfpathlineto{\pgfqpoint{0.972843in}{1.391874in}}%
\pgfpathlineto{\pgfqpoint{0.974286in}{1.392924in}}%
\pgfpathlineto{\pgfqpoint{0.975197in}{1.393554in}}%
\pgfpathlineto{\pgfqpoint{0.977336in}{1.394604in}}%
\pgfpathlineto{\pgfqpoint{0.978413in}{1.395654in}}%
\pgfpathlineto{\pgfqpoint{0.979607in}{1.396704in}}%
\pgfpathlineto{\pgfqpoint{0.980635in}{1.397614in}}%
\pgfpathlineto{\pgfqpoint{0.981729in}{1.398664in}}%
\pgfpathlineto{\pgfqpoint{0.982690in}{1.399224in}}%
\pgfpathlineto{\pgfqpoint{0.984365in}{1.400274in}}%
\pgfpathlineto{\pgfqpoint{0.985343in}{1.401044in}}%
\pgfpathlineto{\pgfqpoint{0.987050in}{1.402094in}}%
\pgfpathlineto{\pgfqpoint{0.987912in}{1.403004in}}%
\pgfpathlineto{\pgfqpoint{0.990051in}{1.404054in}}%
\pgfpathlineto{\pgfqpoint{0.990631in}{1.404474in}}%
\pgfpathlineto{\pgfqpoint{0.992935in}{1.405524in}}%
\pgfpathlineto{\pgfqpoint{0.993847in}{1.406294in}}%
\pgfpathlineto{\pgfqpoint{0.995787in}{1.407344in}}%
\pgfpathlineto{\pgfqpoint{0.996798in}{1.408534in}}%
\pgfpathlineto{\pgfqpoint{0.998456in}{1.409444in}}%
\pgfpathlineto{\pgfqpoint{0.999533in}{1.410424in}}%
\pgfpathlineto{\pgfqpoint{1.001224in}{1.411404in}}%
\pgfpathlineto{\pgfqpoint{1.002235in}{1.412314in}}%
\pgfpathlineto{\pgfqpoint{1.003545in}{1.413364in}}%
\pgfpathlineto{\pgfqpoint{1.004423in}{1.414274in}}%
\pgfpathlineto{\pgfqpoint{1.005882in}{1.415324in}}%
\pgfpathlineto{\pgfqpoint{1.006678in}{1.415954in}}%
\pgfpathlineto{\pgfqpoint{1.009165in}{1.417004in}}%
\pgfpathlineto{\pgfqpoint{1.010226in}{1.417774in}}%
\pgfpathlineto{\pgfqpoint{1.012861in}{1.418824in}}%
\pgfpathlineto{\pgfqpoint{1.013972in}{1.419524in}}%
\pgfpathlineto{\pgfqpoint{1.016442in}{1.420504in}}%
\pgfpathlineto{\pgfqpoint{1.017437in}{1.421274in}}%
\pgfpathlineto{\pgfqpoint{1.019890in}{1.422324in}}%
\pgfpathlineto{\pgfqpoint{1.020852in}{1.423444in}}%
\pgfpathlineto{\pgfqpoint{1.022626in}{1.424494in}}%
\pgfpathlineto{\pgfqpoint{1.023488in}{1.425334in}}%
\pgfpathlineto{\pgfqpoint{1.026405in}{1.426384in}}%
\pgfpathlineto{\pgfqpoint{1.027516in}{1.427224in}}%
\pgfpathlineto{\pgfqpoint{1.029787in}{1.428274in}}%
\pgfpathlineto{\pgfqpoint{1.030831in}{1.429184in}}%
\pgfpathlineto{\pgfqpoint{1.032887in}{1.430234in}}%
\pgfpathlineto{\pgfqpoint{1.033882in}{1.431074in}}%
\pgfpathlineto{\pgfqpoint{1.036667in}{1.432124in}}%
\pgfpathlineto{\pgfqpoint{1.037711in}{1.432894in}}%
\pgfpathlineto{\pgfqpoint{1.041872in}{1.433944in}}%
\pgfpathlineto{\pgfqpoint{1.042834in}{1.434504in}}%
\pgfpathlineto{\pgfqpoint{1.046530in}{1.435554in}}%
\pgfpathlineto{\pgfqpoint{1.047575in}{1.436044in}}%
\pgfpathlineto{\pgfqpoint{1.049514in}{1.437094in}}%
\pgfpathlineto{\pgfqpoint{1.050625in}{1.437794in}}%
\pgfpathlineto{\pgfqpoint{1.053095in}{1.438844in}}%
\pgfpathlineto{\pgfqpoint{1.054090in}{1.439544in}}%
\pgfpathlineto{\pgfqpoint{1.056692in}{1.440594in}}%
\pgfpathlineto{\pgfqpoint{1.057704in}{1.441364in}}%
\pgfpathlineto{\pgfqpoint{1.061301in}{1.442414in}}%
\pgfpathlineto{\pgfqpoint{1.062329in}{1.443044in}}%
\pgfpathlineto{\pgfqpoint{1.065893in}{1.444094in}}%
\pgfpathlineto{\pgfqpoint{1.066987in}{1.444584in}}%
\pgfpathlineto{\pgfqpoint{1.068976in}{1.445634in}}%
\pgfpathlineto{\pgfqpoint{1.070037in}{1.446544in}}%
\pgfpathlineto{\pgfqpoint{1.072491in}{1.447594in}}%
\pgfpathlineto{\pgfqpoint{1.073369in}{1.448224in}}%
\pgfpathlineto{\pgfqpoint{1.075840in}{1.449274in}}%
\pgfpathlineto{\pgfqpoint{1.076652in}{1.449694in}}%
\pgfpathlineto{\pgfqpoint{1.079006in}{1.450744in}}%
\pgfpathlineto{\pgfqpoint{1.079967in}{1.451234in}}%
\pgfpathlineto{\pgfqpoint{1.083001in}{1.452214in}}%
\pgfpathlineto{\pgfqpoint{1.084112in}{1.452914in}}%
\pgfpathlineto{\pgfqpoint{1.087726in}{1.453894in}}%
\pgfpathlineto{\pgfqpoint{1.088438in}{1.454384in}}%
\pgfpathlineto{\pgfqpoint{1.092119in}{1.455434in}}%
\pgfpathlineto{\pgfqpoint{1.093064in}{1.456064in}}%
\pgfpathlineto{\pgfqpoint{1.096346in}{1.457114in}}%
\pgfpathlineto{\pgfqpoint{1.097274in}{1.457674in}}%
\pgfpathlineto{\pgfqpoint{1.100092in}{1.458724in}}%
\pgfpathlineto{\pgfqpoint{1.100689in}{1.459004in}}%
\pgfpathlineto{\pgfqpoint{1.104038in}{1.460054in}}%
\pgfpathlineto{\pgfqpoint{1.105149in}{1.460404in}}%
\pgfpathlineto{\pgfqpoint{1.109989in}{1.461454in}}%
\pgfpathlineto{\pgfqpoint{1.111083in}{1.462154in}}%
\pgfpathlineto{\pgfqpoint{1.114813in}{1.463204in}}%
\pgfpathlineto{\pgfqpoint{1.115858in}{1.463624in}}%
\pgfpathlineto{\pgfqpoint{1.119389in}{1.464674in}}%
\pgfpathlineto{\pgfqpoint{1.120433in}{1.465094in}}%
\pgfpathlineto{\pgfqpoint{1.122754in}{1.466144in}}%
\pgfpathlineto{\pgfqpoint{1.123732in}{1.466494in}}%
\pgfpathlineto{\pgfqpoint{1.127910in}{1.467544in}}%
\pgfpathlineto{\pgfqpoint{1.128556in}{1.468034in}}%
\pgfpathlineto{\pgfqpoint{1.131010in}{1.469084in}}%
\pgfpathlineto{\pgfqpoint{1.131822in}{1.469924in}}%
\pgfpathlineto{\pgfqpoint{1.135187in}{1.470974in}}%
\pgfpathlineto{\pgfqpoint{1.135204in}{1.471184in}}%
\pgfpathlineto{\pgfqpoint{1.140956in}{1.472234in}}%
\pgfpathlineto{\pgfqpoint{1.141735in}{1.472584in}}%
\pgfpathlineto{\pgfqpoint{1.145598in}{1.473634in}}%
\pgfpathlineto{\pgfqpoint{1.146675in}{1.474194in}}%
\pgfpathlineto{\pgfqpoint{1.150803in}{1.475244in}}%
\pgfpathlineto{\pgfqpoint{1.151516in}{1.475874in}}%
\pgfpathlineto{\pgfqpoint{1.156058in}{1.476924in}}%
\pgfpathlineto{\pgfqpoint{1.157152in}{1.477414in}}%
\pgfpathlineto{\pgfqpoint{1.162407in}{1.478464in}}%
\pgfpathlineto{\pgfqpoint{1.163452in}{1.479094in}}%
\pgfpathlineto{\pgfqpoint{1.168060in}{1.480144in}}%
\pgfpathlineto{\pgfqpoint{1.168873in}{1.480494in}}%
\pgfpathlineto{\pgfqpoint{1.171923in}{1.481474in}}%
\pgfpathlineto{\pgfqpoint{1.172835in}{1.481754in}}%
\pgfpathlineto{\pgfqpoint{1.177161in}{1.482804in}}%
\pgfpathlineto{\pgfqpoint{1.178156in}{1.483294in}}%
\pgfpathlineto{\pgfqpoint{1.181223in}{1.484344in}}%
\pgfpathlineto{\pgfqpoint{1.181273in}{1.484554in}}%
\pgfpathlineto{\pgfqpoint{1.188633in}{1.485604in}}%
\pgfpathlineto{\pgfqpoint{1.189512in}{1.485884in}}%
\pgfpathlineto{\pgfqpoint{1.193341in}{1.486934in}}%
\pgfpathlineto{\pgfqpoint{1.194203in}{1.487354in}}%
\pgfpathlineto{\pgfqpoint{1.198928in}{1.488404in}}%
\pgfpathlineto{\pgfqpoint{1.199823in}{1.488964in}}%
\pgfpathlineto{\pgfqpoint{1.201895in}{1.489944in}}%
\pgfpathlineto{\pgfqpoint{1.202343in}{1.490224in}}%
\pgfpathlineto{\pgfqpoint{1.205343in}{1.491274in}}%
\pgfpathlineto{\pgfqpoint{1.206404in}{1.491554in}}%
\pgfpathlineto{\pgfqpoint{1.209670in}{1.492604in}}%
\pgfpathlineto{\pgfqpoint{1.210748in}{1.493024in}}%
\pgfpathlineto{\pgfqpoint{1.215887in}{1.494074in}}%
\pgfpathlineto{\pgfqpoint{1.216865in}{1.494214in}}%
\pgfpathlineto{\pgfqpoint{1.223860in}{1.495264in}}%
\pgfpathlineto{\pgfqpoint{1.224739in}{1.495684in}}%
\pgfpathlineto{\pgfqpoint{1.229994in}{1.496734in}}%
\pgfpathlineto{\pgfqpoint{1.230823in}{1.497014in}}%
\pgfpathlineto{\pgfqpoint{1.234354in}{1.498064in}}%
\pgfpathlineto{\pgfqpoint{1.235117in}{1.498344in}}%
\pgfpathlineto{\pgfqpoint{1.240256in}{1.499394in}}%
\pgfpathlineto{\pgfqpoint{1.241151in}{1.499674in}}%
\pgfpathlineto{\pgfqpoint{1.245610in}{1.500724in}}%
\pgfpathlineto{\pgfqpoint{1.246721in}{1.501144in}}%
\pgfpathlineto{\pgfqpoint{1.248942in}{1.502194in}}%
\pgfpathlineto{\pgfqpoint{1.250053in}{1.502754in}}%
\pgfpathlineto{\pgfqpoint{1.254429in}{1.503804in}}%
\pgfpathlineto{\pgfqpoint{1.254728in}{1.504084in}}%
\pgfpathlineto{\pgfqpoint{1.260845in}{1.505134in}}%
\pgfpathlineto{\pgfqpoint{1.261723in}{1.505414in}}%
\pgfpathlineto{\pgfqpoint{1.266697in}{1.506464in}}%
\pgfpathlineto{\pgfqpoint{1.267807in}{1.506744in}}%
\pgfpathlineto{\pgfqpoint{1.271338in}{1.507724in}}%
\pgfpathlineto{\pgfqpoint{1.272200in}{1.508214in}}%
\pgfpathlineto{\pgfqpoint{1.277936in}{1.509264in}}%
\pgfpathlineto{\pgfqpoint{1.278036in}{1.509404in}}%
\pgfpathlineto{\pgfqpoint{1.284203in}{1.510454in}}%
\pgfpathlineto{\pgfqpoint{1.285131in}{1.510804in}}%
\pgfpathlineto{\pgfqpoint{1.289657in}{1.511854in}}%
\pgfpathlineto{\pgfqpoint{1.290585in}{1.512274in}}%
\pgfpathlineto{\pgfqpoint{1.295940in}{1.513324in}}%
\pgfpathlineto{\pgfqpoint{1.296354in}{1.513534in}}%
\pgfpathlineto{\pgfqpoint{1.299703in}{1.514584in}}%
\pgfpathlineto{\pgfqpoint{1.300664in}{1.514934in}}%
\pgfpathlineto{\pgfqpoint{1.305389in}{1.515984in}}%
\pgfpathlineto{\pgfqpoint{1.306118in}{1.516334in}}%
\pgfpathlineto{\pgfqpoint{1.311290in}{1.517384in}}%
\pgfpathlineto{\pgfqpoint{1.312318in}{1.517734in}}%
\pgfpathlineto{\pgfqpoint{1.318104in}{1.518784in}}%
\pgfpathlineto{\pgfqpoint{1.318899in}{1.519274in}}%
\pgfpathlineto{\pgfqpoint{1.323873in}{1.520324in}}%
\pgfpathlineto{\pgfqpoint{1.324950in}{1.520674in}}%
\pgfpathlineto{\pgfqpoint{1.329575in}{1.521654in}}%
\pgfpathlineto{\pgfqpoint{1.330305in}{1.521864in}}%
\pgfpathlineto{\pgfqpoint{1.336521in}{1.522844in}}%
\pgfpathlineto{\pgfqpoint{1.336836in}{1.523194in}}%
\pgfpathlineto{\pgfqpoint{1.344180in}{1.524244in}}%
\pgfpathlineto{\pgfqpoint{1.345191in}{1.524524in}}%
\pgfpathlineto{\pgfqpoint{1.351905in}{1.525574in}}%
\pgfpathlineto{\pgfqpoint{1.352602in}{1.525994in}}%
\pgfpathlineto{\pgfqpoint{1.359050in}{1.527044in}}%
\pgfpathlineto{\pgfqpoint{1.359962in}{1.527394in}}%
\pgfpathlineto{\pgfqpoint{1.365897in}{1.528444in}}%
\pgfpathlineto{\pgfqpoint{1.366991in}{1.528654in}}%
\pgfpathlineto{\pgfqpoint{1.378463in}{1.529704in}}%
\pgfpathlineto{\pgfqpoint{1.379092in}{1.530054in}}%
\pgfpathlineto{\pgfqpoint{1.386221in}{1.531104in}}%
\pgfpathlineto{\pgfqpoint{1.387298in}{1.531314in}}%
\pgfpathlineto{\pgfqpoint{1.393366in}{1.532364in}}%
\pgfpathlineto{\pgfqpoint{1.393880in}{1.532714in}}%
\pgfpathlineto{\pgfqpoint{1.403760in}{1.533764in}}%
\pgfpathlineto{\pgfqpoint{1.404340in}{1.533904in}}%
\pgfpathlineto{\pgfqpoint{1.412529in}{1.534954in}}%
\pgfpathlineto{\pgfqpoint{1.412994in}{1.535164in}}%
\pgfpathlineto{\pgfqpoint{1.422277in}{1.536214in}}%
\pgfpathlineto{\pgfqpoint{1.422807in}{1.536354in}}%
\pgfpathlineto{\pgfqpoint{1.428460in}{1.537404in}}%
\pgfpathlineto{\pgfqpoint{1.429157in}{1.537614in}}%
\pgfpathlineto{\pgfqpoint{1.435340in}{1.538664in}}%
\pgfpathlineto{\pgfqpoint{1.436351in}{1.539014in}}%
\pgfpathlineto{\pgfqpoint{1.443861in}{1.540064in}}%
\pgfpathlineto{\pgfqpoint{1.443877in}{1.540274in}}%
\pgfpathlineto{\pgfqpoint{1.450940in}{1.541324in}}%
\pgfpathlineto{\pgfqpoint{1.451006in}{1.541464in}}%
\pgfpathlineto{\pgfqpoint{1.462776in}{1.542514in}}%
\pgfpathlineto{\pgfqpoint{1.463439in}{1.542724in}}%
\pgfpathlineto{\pgfqpoint{1.473817in}{1.543774in}}%
\pgfpathlineto{\pgfqpoint{1.474347in}{1.544124in}}%
\pgfpathlineto{\pgfqpoint{1.482304in}{1.545174in}}%
\pgfpathlineto{\pgfqpoint{1.482669in}{1.545314in}}%
\pgfpathlineto{\pgfqpoint{1.493892in}{1.546364in}}%
\pgfpathlineto{\pgfqpoint{1.494257in}{1.546504in}}%
\pgfpathlineto{\pgfqpoint{1.503756in}{1.547554in}}%
\pgfpathlineto{\pgfqpoint{1.503921in}{1.547834in}}%
\pgfpathlineto{\pgfqpoint{1.514382in}{1.548884in}}%
\pgfpathlineto{\pgfqpoint{1.515227in}{1.549234in}}%
\pgfpathlineto{\pgfqpoint{1.524046in}{1.550284in}}%
\pgfpathlineto{\pgfqpoint{1.524378in}{1.550564in}}%
\pgfpathlineto{\pgfqpoint{1.535187in}{1.551614in}}%
\pgfpathlineto{\pgfqpoint{1.535817in}{1.551754in}}%
\pgfpathlineto{\pgfqpoint{1.546012in}{1.552804in}}%
\pgfpathlineto{\pgfqpoint{1.547122in}{1.553154in}}%
\pgfpathlineto{\pgfqpoint{1.557450in}{1.554204in}}%
\pgfpathlineto{\pgfqpoint{1.557450in}{1.554274in}}%
\pgfpathlineto{\pgfqpoint{1.571690in}{1.555324in}}%
\pgfpathlineto{\pgfqpoint{1.571790in}{1.555534in}}%
\pgfpathlineto{\pgfqpoint{1.581405in}{1.556584in}}%
\pgfpathlineto{\pgfqpoint{1.581653in}{1.556794in}}%
\pgfpathlineto{\pgfqpoint{1.595396in}{1.557844in}}%
\pgfpathlineto{\pgfqpoint{1.595844in}{1.558054in}}%
\pgfpathlineto{\pgfqpoint{1.606503in}{1.559104in}}%
\pgfpathlineto{\pgfqpoint{1.607514in}{1.559384in}}%
\pgfpathlineto{\pgfqpoint{1.617610in}{1.560434in}}%
\pgfpathlineto{\pgfqpoint{1.618721in}{1.560644in}}%
\pgfpathlineto{\pgfqpoint{1.630889in}{1.561694in}}%
\pgfpathlineto{\pgfqpoint{1.631883in}{1.561834in}}%
\pgfpathlineto{\pgfqpoint{1.652605in}{1.562884in}}%
\pgfpathlineto{\pgfqpoint{1.653600in}{1.563024in}}%
\pgfpathlineto{\pgfqpoint{1.663812in}{1.564074in}}%
\pgfpathlineto{\pgfqpoint{1.663861in}{1.564214in}}%
\pgfpathlineto{\pgfqpoint{1.672283in}{1.565264in}}%
\pgfpathlineto{\pgfqpoint{1.672382in}{1.565404in}}%
\pgfpathlineto{\pgfqpoint{1.683406in}{1.566454in}}%
\pgfpathlineto{\pgfqpoint{1.684467in}{1.566664in}}%
\pgfpathlineto{\pgfqpoint{1.696586in}{1.567714in}}%
\pgfpathlineto{\pgfqpoint{1.696834in}{1.567854in}}%
\pgfpathlineto{\pgfqpoint{1.710212in}{1.568904in}}%
\pgfpathlineto{\pgfqpoint{1.710909in}{1.569114in}}%
\pgfpathlineto{\pgfqpoint{1.723706in}{1.570164in}}%
\pgfpathlineto{\pgfqpoint{1.724718in}{1.570304in}}%
\pgfpathlineto{\pgfqpoint{1.739090in}{1.571354in}}%
\pgfpathlineto{\pgfqpoint{1.739422in}{1.571494in}}%
\pgfpathlineto{\pgfqpoint{1.756828in}{1.572544in}}%
\pgfpathlineto{\pgfqpoint{1.757591in}{1.572754in}}%
\pgfpathlineto{\pgfqpoint{1.769825in}{1.573804in}}%
\pgfpathlineto{\pgfqpoint{1.770339in}{1.573944in}}%
\pgfpathlineto{\pgfqpoint{1.782242in}{1.574994in}}%
\pgfpathlineto{\pgfqpoint{1.782374in}{1.575134in}}%
\pgfpathlineto{\pgfqpoint{1.795288in}{1.576184in}}%
\pgfpathlineto{\pgfqpoint{1.795868in}{1.576394in}}%
\pgfpathlineto{\pgfqpoint{1.809910in}{1.577444in}}%
\pgfpathlineto{\pgfqpoint{1.810987in}{1.577724in}}%
\pgfpathlineto{\pgfqpoint{1.828360in}{1.578774in}}%
\pgfpathlineto{\pgfqpoint{1.829057in}{1.578984in}}%
\pgfpathlineto{\pgfqpoint{1.842236in}{1.580034in}}%
\pgfpathlineto{\pgfqpoint{1.842368in}{1.580174in}}%
\pgfpathlineto{\pgfqpoint{1.859410in}{1.581224in}}%
\pgfpathlineto{\pgfqpoint{1.859891in}{1.581364in}}%
\pgfpathlineto{\pgfqpoint{1.873683in}{1.582414in}}%
\pgfpathlineto{\pgfqpoint{1.873683in}{1.582484in}}%
\pgfpathlineto{\pgfqpoint{1.894190in}{1.583534in}}%
\pgfpathlineto{\pgfqpoint{1.894637in}{1.583674in}}%
\pgfpathlineto{\pgfqpoint{1.915542in}{1.584724in}}%
\pgfpathlineto{\pgfqpoint{1.916155in}{1.584864in}}%
\pgfpathlineto{\pgfqpoint{1.929334in}{1.585914in}}%
\pgfpathlineto{\pgfqpoint{1.930345in}{1.586054in}}%
\pgfpathlineto{\pgfqpoint{1.945680in}{1.587104in}}%
\pgfpathlineto{\pgfqpoint{1.945680in}{1.587174in}}%
\pgfpathlineto{\pgfqpoint{1.958245in}{1.588224in}}%
\pgfpathlineto{\pgfqpoint{1.959074in}{1.588434in}}%
\pgfpathlineto{\pgfqpoint{1.971541in}{1.589484in}}%
\pgfpathlineto{\pgfqpoint{1.972005in}{1.589624in}}%
\pgfpathlineto{\pgfqpoint{1.980807in}{1.590674in}}%
\pgfpathlineto{\pgfqpoint{1.981636in}{1.590954in}}%
\pgfpathlineto{\pgfqpoint{1.988947in}{1.592004in}}%
\pgfpathlineto{\pgfqpoint{1.988947in}{1.592074in}}%
\pgfpathlineto{\pgfqpoint{1.998313in}{1.593124in}}%
\pgfpathlineto{\pgfqpoint{1.999391in}{1.593334in}}%
\pgfpathlineto{\pgfqpoint{2.006536in}{1.594384in}}%
\pgfpathlineto{\pgfqpoint{2.007414in}{1.594524in}}%
\pgfpathlineto{\pgfqpoint{2.013349in}{1.595574in}}%
\pgfpathlineto{\pgfqpoint{2.014195in}{1.595854in}}%
\pgfpathlineto{\pgfqpoint{2.021936in}{1.596904in}}%
\pgfpathlineto{\pgfqpoint{2.022318in}{1.597044in}}%
\pgfpathlineto{\pgfqpoint{2.028136in}{1.598094in}}%
\pgfpathlineto{\pgfqpoint{2.029230in}{1.598584in}}%
\pgfpathlineto{\pgfqpoint{2.032015in}{1.599634in}}%
\pgfpathlineto{\pgfqpoint{2.033126in}{1.601944in}}%
\pgfpathlineto{\pgfqpoint{2.033126in}{1.601944in}}%
\pgfusepath{stroke}%
\end{pgfscope}%
\begin{pgfscope}%
\pgfsetrectcap%
\pgfsetmiterjoin%
\pgfsetlinewidth{0.803000pt}%
\definecolor{currentstroke}{rgb}{0.000000,0.000000,0.000000}%
\pgfsetstrokecolor{currentstroke}%
\pgfsetdash{}{0pt}%
\pgfpathmoveto{\pgfqpoint{0.553581in}{0.499444in}}%
\pgfpathlineto{\pgfqpoint{0.553581in}{1.654444in}}%
\pgfusepath{stroke}%
\end{pgfscope}%
\begin{pgfscope}%
\pgfsetrectcap%
\pgfsetmiterjoin%
\pgfsetlinewidth{0.803000pt}%
\definecolor{currentstroke}{rgb}{0.000000,0.000000,0.000000}%
\pgfsetstrokecolor{currentstroke}%
\pgfsetdash{}{0pt}%
\pgfpathmoveto{\pgfqpoint{2.103581in}{0.499444in}}%
\pgfpathlineto{\pgfqpoint{2.103581in}{1.654444in}}%
\pgfusepath{stroke}%
\end{pgfscope}%
\begin{pgfscope}%
\pgfsetrectcap%
\pgfsetmiterjoin%
\pgfsetlinewidth{0.803000pt}%
\definecolor{currentstroke}{rgb}{0.000000,0.000000,0.000000}%
\pgfsetstrokecolor{currentstroke}%
\pgfsetdash{}{0pt}%
\pgfpathmoveto{\pgfqpoint{0.553581in}{0.499444in}}%
\pgfpathlineto{\pgfqpoint{2.103581in}{0.499444in}}%
\pgfusepath{stroke}%
\end{pgfscope}%
\begin{pgfscope}%
\pgfsetrectcap%
\pgfsetmiterjoin%
\pgfsetlinewidth{0.803000pt}%
\definecolor{currentstroke}{rgb}{0.000000,0.000000,0.000000}%
\pgfsetstrokecolor{currentstroke}%
\pgfsetdash{}{0pt}%
\pgfpathmoveto{\pgfqpoint{0.553581in}{1.654444in}}%
\pgfpathlineto{\pgfqpoint{2.103581in}{1.654444in}}%
\pgfusepath{stroke}%
\end{pgfscope}%
\begin{pgfscope}%
\pgfsetbuttcap%
\pgfsetmiterjoin%
\definecolor{currentfill}{rgb}{1.000000,1.000000,1.000000}%
\pgfsetfillcolor{currentfill}%
\pgfsetfillopacity{0.800000}%
\pgfsetlinewidth{1.003750pt}%
\definecolor{currentstroke}{rgb}{0.800000,0.800000,0.800000}%
\pgfsetstrokecolor{currentstroke}%
\pgfsetstrokeopacity{0.800000}%
\pgfsetdash{}{0pt}%
\pgfpathmoveto{\pgfqpoint{0.832747in}{0.568889in}}%
\pgfpathlineto{\pgfqpoint{2.006358in}{0.568889in}}%
\pgfpathquadraticcurveto{\pgfqpoint{2.034136in}{0.568889in}}{\pgfqpoint{2.034136in}{0.596666in}}%
\pgfpathlineto{\pgfqpoint{2.034136in}{0.776388in}}%
\pgfpathquadraticcurveto{\pgfqpoint{2.034136in}{0.804166in}}{\pgfqpoint{2.006358in}{0.804166in}}%
\pgfpathlineto{\pgfqpoint{0.832747in}{0.804166in}}%
\pgfpathquadraticcurveto{\pgfqpoint{0.804970in}{0.804166in}}{\pgfqpoint{0.804970in}{0.776388in}}%
\pgfpathlineto{\pgfqpoint{0.804970in}{0.596666in}}%
\pgfpathquadraticcurveto{\pgfqpoint{0.804970in}{0.568889in}}{\pgfqpoint{0.832747in}{0.568889in}}%
\pgfpathlineto{\pgfqpoint{0.832747in}{0.568889in}}%
\pgfpathclose%
\pgfusepath{stroke,fill}%
\end{pgfscope}%
\begin{pgfscope}%
\pgfsetrectcap%
\pgfsetroundjoin%
\pgfsetlinewidth{1.505625pt}%
\definecolor{currentstroke}{rgb}{0.000000,0.000000,0.000000}%
\pgfsetstrokecolor{currentstroke}%
\pgfsetdash{}{0pt}%
\pgfpathmoveto{\pgfqpoint{0.860525in}{0.700000in}}%
\pgfpathlineto{\pgfqpoint{0.999414in}{0.700000in}}%
\pgfpathlineto{\pgfqpoint{1.138303in}{0.700000in}}%
\pgfusepath{stroke}%
\end{pgfscope}%
\begin{pgfscope}%
\definecolor{textcolor}{rgb}{0.000000,0.000000,0.000000}%
\pgfsetstrokecolor{textcolor}%
\pgfsetfillcolor{textcolor}%
\pgftext[x=1.249414in,y=0.651388in,left,base]{\color{textcolor}\rmfamily\fontsize{10.000000}{12.000000}\selectfont AUC=0.840}%
\end{pgfscope}%
\end{pgfpicture}%
\makeatother%
\endgroup%

	
&
	\vskip 0pt
	\begin{tabular}{cc|c|c|}
	&\multicolumn{1}{c}{}& \multicolumn{2}{c}{Prediction} \\[0.4em]
	&\multicolumn{1}{c}{} & \multicolumn{1}{c}{N} & \multicolumn{1}{c}{P} \cr\cline{3-4}
	\multirow{2}{*}{\rotatebox[origin=c]{90}{Actual}}&N &
85\% & 0\%
	\vrule width 0pt height 10pt depth 2pt \cr\cline{3-4}
	&P & 
15\% & 0\%
	\vrule width 0pt height 10pt depth 2pt \cr\cline{3-4}
	\end{tabular}

	\hfil\begin{tabular}{ll}
	\cr
	0.85 & Accuracy\cr
	nan & Precision \cr
	0 & Recall \cr
	nan & F1 \cr
	0.840 & AUC \cr
\end{tabular}
\cr
\end{tabular}
} % End parbox

\

{\bf Ideal Model Tending Right} Under this related model, if we were using decision threshold $\theta = 0.5$, we would immediately dispatch an ambulance to every reported crash, which would save lives but be expensive and not possible with existing resources in the short term.  

\

\verb|Ideal_Right|

%%%
\parbox{\linewidth}{
\noindent\begin{tabular}{@{\hspace{-6pt}}p{2.3in} @{\hspace{-6pt}}p{2.0in} p{1.8in}}
	\vskip 0pt
	\qquad \qquad Raw Model Output
	
	%% Creator: Matplotlib, PGF backend
%%
%% To include the figure in your LaTeX document, write
%%   \input{<filename>.pgf}
%%
%% Make sure the required packages are loaded in your preamble
%%   \usepackage{pgf}
%%
%% Also ensure that all the required font packages are loaded; for instance,
%% the lmodern package is sometimes necessary when using math font.
%%   \usepackage{lmodern}
%%
%% Figures using additional raster images can only be included by \input if
%% they are in the same directory as the main LaTeX file. For loading figures
%% from other directories you can use the `import` package
%%   \usepackage{import}
%%
%% and then include the figures with
%%   \import{<path to file>}{<filename>.pgf}
%%
%% Matplotlib used the following preamble
%%   
%%   \usepackage{fontspec}
%%   \makeatletter\@ifpackageloaded{underscore}{}{\usepackage[strings]{underscore}}\makeatother
%%
\begingroup%
\makeatletter%
\begin{pgfpicture}%
\pgfpathrectangle{\pgfpointorigin}{\pgfqpoint{2.253750in}{1.754444in}}%
\pgfusepath{use as bounding box, clip}%
\begin{pgfscope}%
\pgfsetbuttcap%
\pgfsetmiterjoin%
\definecolor{currentfill}{rgb}{1.000000,1.000000,1.000000}%
\pgfsetfillcolor{currentfill}%
\pgfsetlinewidth{0.000000pt}%
\definecolor{currentstroke}{rgb}{1.000000,1.000000,1.000000}%
\pgfsetstrokecolor{currentstroke}%
\pgfsetdash{}{0pt}%
\pgfpathmoveto{\pgfqpoint{0.000000in}{0.000000in}}%
\pgfpathlineto{\pgfqpoint{2.253750in}{0.000000in}}%
\pgfpathlineto{\pgfqpoint{2.253750in}{1.754444in}}%
\pgfpathlineto{\pgfqpoint{0.000000in}{1.754444in}}%
\pgfpathlineto{\pgfqpoint{0.000000in}{0.000000in}}%
\pgfpathclose%
\pgfusepath{fill}%
\end{pgfscope}%
\begin{pgfscope}%
\pgfsetbuttcap%
\pgfsetmiterjoin%
\definecolor{currentfill}{rgb}{1.000000,1.000000,1.000000}%
\pgfsetfillcolor{currentfill}%
\pgfsetlinewidth{0.000000pt}%
\definecolor{currentstroke}{rgb}{0.000000,0.000000,0.000000}%
\pgfsetstrokecolor{currentstroke}%
\pgfsetstrokeopacity{0.000000}%
\pgfsetdash{}{0pt}%
\pgfpathmoveto{\pgfqpoint{0.515000in}{0.499444in}}%
\pgfpathlineto{\pgfqpoint{2.065000in}{0.499444in}}%
\pgfpathlineto{\pgfqpoint{2.065000in}{1.654444in}}%
\pgfpathlineto{\pgfqpoint{0.515000in}{1.654444in}}%
\pgfpathlineto{\pgfqpoint{0.515000in}{0.499444in}}%
\pgfpathclose%
\pgfusepath{fill}%
\end{pgfscope}%
\begin{pgfscope}%
\pgfpathrectangle{\pgfqpoint{0.515000in}{0.499444in}}{\pgfqpoint{1.550000in}{1.155000in}}%
\pgfusepath{clip}%
\pgfsetbuttcap%
\pgfsetmiterjoin%
\pgfsetlinewidth{1.003750pt}%
\definecolor{currentstroke}{rgb}{0.000000,0.000000,0.000000}%
\pgfsetstrokecolor{currentstroke}%
\pgfsetdash{}{0pt}%
\pgfpathmoveto{\pgfqpoint{0.505000in}{0.499444in}}%
\pgfpathlineto{\pgfqpoint{0.552805in}{0.499444in}}%
\pgfpathlineto{\pgfqpoint{0.552805in}{0.499444in}}%
\pgfpathlineto{\pgfqpoint{0.505000in}{0.499444in}}%
\pgfusepath{stroke}%
\end{pgfscope}%
\begin{pgfscope}%
\pgfpathrectangle{\pgfqpoint{0.515000in}{0.499444in}}{\pgfqpoint{1.550000in}{1.155000in}}%
\pgfusepath{clip}%
\pgfsetbuttcap%
\pgfsetmiterjoin%
\pgfsetlinewidth{1.003750pt}%
\definecolor{currentstroke}{rgb}{0.000000,0.000000,0.000000}%
\pgfsetstrokecolor{currentstroke}%
\pgfsetdash{}{0pt}%
\pgfpathmoveto{\pgfqpoint{0.643537in}{0.499444in}}%
\pgfpathlineto{\pgfqpoint{0.704025in}{0.499444in}}%
\pgfpathlineto{\pgfqpoint{0.704025in}{0.499444in}}%
\pgfpathlineto{\pgfqpoint{0.643537in}{0.499444in}}%
\pgfpathlineto{\pgfqpoint{0.643537in}{0.499444in}}%
\pgfpathclose%
\pgfusepath{stroke}%
\end{pgfscope}%
\begin{pgfscope}%
\pgfpathrectangle{\pgfqpoint{0.515000in}{0.499444in}}{\pgfqpoint{1.550000in}{1.155000in}}%
\pgfusepath{clip}%
\pgfsetbuttcap%
\pgfsetmiterjoin%
\pgfsetlinewidth{1.003750pt}%
\definecolor{currentstroke}{rgb}{0.000000,0.000000,0.000000}%
\pgfsetstrokecolor{currentstroke}%
\pgfsetdash{}{0pt}%
\pgfpathmoveto{\pgfqpoint{0.794756in}{0.499444in}}%
\pgfpathlineto{\pgfqpoint{0.855244in}{0.499444in}}%
\pgfpathlineto{\pgfqpoint{0.855244in}{0.499444in}}%
\pgfpathlineto{\pgfqpoint{0.794756in}{0.499444in}}%
\pgfpathlineto{\pgfqpoint{0.794756in}{0.499444in}}%
\pgfpathclose%
\pgfusepath{stroke}%
\end{pgfscope}%
\begin{pgfscope}%
\pgfpathrectangle{\pgfqpoint{0.515000in}{0.499444in}}{\pgfqpoint{1.550000in}{1.155000in}}%
\pgfusepath{clip}%
\pgfsetbuttcap%
\pgfsetmiterjoin%
\pgfsetlinewidth{1.003750pt}%
\definecolor{currentstroke}{rgb}{0.000000,0.000000,0.000000}%
\pgfsetstrokecolor{currentstroke}%
\pgfsetdash{}{0pt}%
\pgfpathmoveto{\pgfqpoint{0.945976in}{0.499444in}}%
\pgfpathlineto{\pgfqpoint{1.006464in}{0.499444in}}%
\pgfpathlineto{\pgfqpoint{1.006464in}{0.499444in}}%
\pgfpathlineto{\pgfqpoint{0.945976in}{0.499444in}}%
\pgfpathlineto{\pgfqpoint{0.945976in}{0.499444in}}%
\pgfpathclose%
\pgfusepath{stroke}%
\end{pgfscope}%
\begin{pgfscope}%
\pgfpathrectangle{\pgfqpoint{0.515000in}{0.499444in}}{\pgfqpoint{1.550000in}{1.155000in}}%
\pgfusepath{clip}%
\pgfsetbuttcap%
\pgfsetmiterjoin%
\pgfsetlinewidth{1.003750pt}%
\definecolor{currentstroke}{rgb}{0.000000,0.000000,0.000000}%
\pgfsetstrokecolor{currentstroke}%
\pgfsetdash{}{0pt}%
\pgfpathmoveto{\pgfqpoint{1.097195in}{0.499444in}}%
\pgfpathlineto{\pgfqpoint{1.157683in}{0.499444in}}%
\pgfpathlineto{\pgfqpoint{1.157683in}{0.499444in}}%
\pgfpathlineto{\pgfqpoint{1.097195in}{0.499444in}}%
\pgfpathlineto{\pgfqpoint{1.097195in}{0.499444in}}%
\pgfpathclose%
\pgfusepath{stroke}%
\end{pgfscope}%
\begin{pgfscope}%
\pgfpathrectangle{\pgfqpoint{0.515000in}{0.499444in}}{\pgfqpoint{1.550000in}{1.155000in}}%
\pgfusepath{clip}%
\pgfsetbuttcap%
\pgfsetmiterjoin%
\pgfsetlinewidth{1.003750pt}%
\definecolor{currentstroke}{rgb}{0.000000,0.000000,0.000000}%
\pgfsetstrokecolor{currentstroke}%
\pgfsetdash{}{0pt}%
\pgfpathmoveto{\pgfqpoint{1.248415in}{0.499444in}}%
\pgfpathlineto{\pgfqpoint{1.308903in}{0.499444in}}%
\pgfpathlineto{\pgfqpoint{1.308903in}{0.961883in}}%
\pgfpathlineto{\pgfqpoint{1.248415in}{0.961883in}}%
\pgfpathlineto{\pgfqpoint{1.248415in}{0.499444in}}%
\pgfpathclose%
\pgfusepath{stroke}%
\end{pgfscope}%
\begin{pgfscope}%
\pgfpathrectangle{\pgfqpoint{0.515000in}{0.499444in}}{\pgfqpoint{1.550000in}{1.155000in}}%
\pgfusepath{clip}%
\pgfsetbuttcap%
\pgfsetmiterjoin%
\pgfsetlinewidth{1.003750pt}%
\definecolor{currentstroke}{rgb}{0.000000,0.000000,0.000000}%
\pgfsetstrokecolor{currentstroke}%
\pgfsetdash{}{0pt}%
\pgfpathmoveto{\pgfqpoint{1.399634in}{0.499444in}}%
\pgfpathlineto{\pgfqpoint{1.460122in}{0.499444in}}%
\pgfpathlineto{\pgfqpoint{1.460122in}{1.599444in}}%
\pgfpathlineto{\pgfqpoint{1.399634in}{1.599444in}}%
\pgfpathlineto{\pgfqpoint{1.399634in}{0.499444in}}%
\pgfpathclose%
\pgfusepath{stroke}%
\end{pgfscope}%
\begin{pgfscope}%
\pgfpathrectangle{\pgfqpoint{0.515000in}{0.499444in}}{\pgfqpoint{1.550000in}{1.155000in}}%
\pgfusepath{clip}%
\pgfsetbuttcap%
\pgfsetmiterjoin%
\pgfsetlinewidth{1.003750pt}%
\definecolor{currentstroke}{rgb}{0.000000,0.000000,0.000000}%
\pgfsetstrokecolor{currentstroke}%
\pgfsetdash{}{0pt}%
\pgfpathmoveto{\pgfqpoint{1.550854in}{0.499444in}}%
\pgfpathlineto{\pgfqpoint{1.611342in}{0.499444in}}%
\pgfpathlineto{\pgfqpoint{1.611342in}{1.132630in}}%
\pgfpathlineto{\pgfqpoint{1.550854in}{1.132630in}}%
\pgfpathlineto{\pgfqpoint{1.550854in}{0.499444in}}%
\pgfpathclose%
\pgfusepath{stroke}%
\end{pgfscope}%
\begin{pgfscope}%
\pgfpathrectangle{\pgfqpoint{0.515000in}{0.499444in}}{\pgfqpoint{1.550000in}{1.155000in}}%
\pgfusepath{clip}%
\pgfsetbuttcap%
\pgfsetmiterjoin%
\pgfsetlinewidth{1.003750pt}%
\definecolor{currentstroke}{rgb}{0.000000,0.000000,0.000000}%
\pgfsetstrokecolor{currentstroke}%
\pgfsetdash{}{0pt}%
\pgfpathmoveto{\pgfqpoint{1.702073in}{0.499444in}}%
\pgfpathlineto{\pgfqpoint{1.762561in}{0.499444in}}%
\pgfpathlineto{\pgfqpoint{1.762561in}{0.739458in}}%
\pgfpathlineto{\pgfqpoint{1.702073in}{0.739458in}}%
\pgfpathlineto{\pgfqpoint{1.702073in}{0.499444in}}%
\pgfpathclose%
\pgfusepath{stroke}%
\end{pgfscope}%
\begin{pgfscope}%
\pgfpathrectangle{\pgfqpoint{0.515000in}{0.499444in}}{\pgfqpoint{1.550000in}{1.155000in}}%
\pgfusepath{clip}%
\pgfsetbuttcap%
\pgfsetmiterjoin%
\pgfsetlinewidth{1.003750pt}%
\definecolor{currentstroke}{rgb}{0.000000,0.000000,0.000000}%
\pgfsetstrokecolor{currentstroke}%
\pgfsetdash{}{0pt}%
\pgfpathmoveto{\pgfqpoint{1.853293in}{0.499444in}}%
\pgfpathlineto{\pgfqpoint{1.913781in}{0.499444in}}%
\pgfpathlineto{\pgfqpoint{1.913781in}{0.559668in}}%
\pgfpathlineto{\pgfqpoint{1.853293in}{0.559668in}}%
\pgfpathlineto{\pgfqpoint{1.853293in}{0.499444in}}%
\pgfpathclose%
\pgfusepath{stroke}%
\end{pgfscope}%
\begin{pgfscope}%
\pgfpathrectangle{\pgfqpoint{0.515000in}{0.499444in}}{\pgfqpoint{1.550000in}{1.155000in}}%
\pgfusepath{clip}%
\pgfsetbuttcap%
\pgfsetmiterjoin%
\definecolor{currentfill}{rgb}{0.000000,0.000000,0.000000}%
\pgfsetfillcolor{currentfill}%
\pgfsetlinewidth{0.000000pt}%
\definecolor{currentstroke}{rgb}{0.000000,0.000000,0.000000}%
\pgfsetstrokecolor{currentstroke}%
\pgfsetstrokeopacity{0.000000}%
\pgfsetdash{}{0pt}%
\pgfpathmoveto{\pgfqpoint{0.552805in}{0.499444in}}%
\pgfpathlineto{\pgfqpoint{0.613293in}{0.499444in}}%
\pgfpathlineto{\pgfqpoint{0.613293in}{0.499444in}}%
\pgfpathlineto{\pgfqpoint{0.552805in}{0.499444in}}%
\pgfpathlineto{\pgfqpoint{0.552805in}{0.499444in}}%
\pgfpathclose%
\pgfusepath{fill}%
\end{pgfscope}%
\begin{pgfscope}%
\pgfpathrectangle{\pgfqpoint{0.515000in}{0.499444in}}{\pgfqpoint{1.550000in}{1.155000in}}%
\pgfusepath{clip}%
\pgfsetbuttcap%
\pgfsetmiterjoin%
\definecolor{currentfill}{rgb}{0.000000,0.000000,0.000000}%
\pgfsetfillcolor{currentfill}%
\pgfsetlinewidth{0.000000pt}%
\definecolor{currentstroke}{rgb}{0.000000,0.000000,0.000000}%
\pgfsetstrokecolor{currentstroke}%
\pgfsetstrokeopacity{0.000000}%
\pgfsetdash{}{0pt}%
\pgfpathmoveto{\pgfqpoint{0.704025in}{0.499444in}}%
\pgfpathlineto{\pgfqpoint{0.764512in}{0.499444in}}%
\pgfpathlineto{\pgfqpoint{0.764512in}{0.499444in}}%
\pgfpathlineto{\pgfqpoint{0.704025in}{0.499444in}}%
\pgfpathlineto{\pgfqpoint{0.704025in}{0.499444in}}%
\pgfpathclose%
\pgfusepath{fill}%
\end{pgfscope}%
\begin{pgfscope}%
\pgfpathrectangle{\pgfqpoint{0.515000in}{0.499444in}}{\pgfqpoint{1.550000in}{1.155000in}}%
\pgfusepath{clip}%
\pgfsetbuttcap%
\pgfsetmiterjoin%
\definecolor{currentfill}{rgb}{0.000000,0.000000,0.000000}%
\pgfsetfillcolor{currentfill}%
\pgfsetlinewidth{0.000000pt}%
\definecolor{currentstroke}{rgb}{0.000000,0.000000,0.000000}%
\pgfsetstrokecolor{currentstroke}%
\pgfsetstrokeopacity{0.000000}%
\pgfsetdash{}{0pt}%
\pgfpathmoveto{\pgfqpoint{0.855244in}{0.499444in}}%
\pgfpathlineto{\pgfqpoint{0.915732in}{0.499444in}}%
\pgfpathlineto{\pgfqpoint{0.915732in}{0.499444in}}%
\pgfpathlineto{\pgfqpoint{0.855244in}{0.499444in}}%
\pgfpathlineto{\pgfqpoint{0.855244in}{0.499444in}}%
\pgfpathclose%
\pgfusepath{fill}%
\end{pgfscope}%
\begin{pgfscope}%
\pgfpathrectangle{\pgfqpoint{0.515000in}{0.499444in}}{\pgfqpoint{1.550000in}{1.155000in}}%
\pgfusepath{clip}%
\pgfsetbuttcap%
\pgfsetmiterjoin%
\definecolor{currentfill}{rgb}{0.000000,0.000000,0.000000}%
\pgfsetfillcolor{currentfill}%
\pgfsetlinewidth{0.000000pt}%
\definecolor{currentstroke}{rgb}{0.000000,0.000000,0.000000}%
\pgfsetstrokecolor{currentstroke}%
\pgfsetstrokeopacity{0.000000}%
\pgfsetdash{}{0pt}%
\pgfpathmoveto{\pgfqpoint{1.006464in}{0.499444in}}%
\pgfpathlineto{\pgfqpoint{1.066951in}{0.499444in}}%
\pgfpathlineto{\pgfqpoint{1.066951in}{0.499444in}}%
\pgfpathlineto{\pgfqpoint{1.006464in}{0.499444in}}%
\pgfpathlineto{\pgfqpoint{1.006464in}{0.499444in}}%
\pgfpathclose%
\pgfusepath{fill}%
\end{pgfscope}%
\begin{pgfscope}%
\pgfpathrectangle{\pgfqpoint{0.515000in}{0.499444in}}{\pgfqpoint{1.550000in}{1.155000in}}%
\pgfusepath{clip}%
\pgfsetbuttcap%
\pgfsetmiterjoin%
\definecolor{currentfill}{rgb}{0.000000,0.000000,0.000000}%
\pgfsetfillcolor{currentfill}%
\pgfsetlinewidth{0.000000pt}%
\definecolor{currentstroke}{rgb}{0.000000,0.000000,0.000000}%
\pgfsetstrokecolor{currentstroke}%
\pgfsetstrokeopacity{0.000000}%
\pgfsetdash{}{0pt}%
\pgfpathmoveto{\pgfqpoint{1.157683in}{0.499444in}}%
\pgfpathlineto{\pgfqpoint{1.218171in}{0.499444in}}%
\pgfpathlineto{\pgfqpoint{1.218171in}{0.499444in}}%
\pgfpathlineto{\pgfqpoint{1.157683in}{0.499444in}}%
\pgfpathlineto{\pgfqpoint{1.157683in}{0.499444in}}%
\pgfpathclose%
\pgfusepath{fill}%
\end{pgfscope}%
\begin{pgfscope}%
\pgfpathrectangle{\pgfqpoint{0.515000in}{0.499444in}}{\pgfqpoint{1.550000in}{1.155000in}}%
\pgfusepath{clip}%
\pgfsetbuttcap%
\pgfsetmiterjoin%
\definecolor{currentfill}{rgb}{0.000000,0.000000,0.000000}%
\pgfsetfillcolor{currentfill}%
\pgfsetlinewidth{0.000000pt}%
\definecolor{currentstroke}{rgb}{0.000000,0.000000,0.000000}%
\pgfsetstrokecolor{currentstroke}%
\pgfsetstrokeopacity{0.000000}%
\pgfsetdash{}{0pt}%
\pgfpathmoveto{\pgfqpoint{1.308903in}{0.499444in}}%
\pgfpathlineto{\pgfqpoint{1.369391in}{0.499444in}}%
\pgfpathlineto{\pgfqpoint{1.369391in}{0.511101in}}%
\pgfpathlineto{\pgfqpoint{1.308903in}{0.511101in}}%
\pgfpathlineto{\pgfqpoint{1.308903in}{0.499444in}}%
\pgfpathclose%
\pgfusepath{fill}%
\end{pgfscope}%
\begin{pgfscope}%
\pgfpathrectangle{\pgfqpoint{0.515000in}{0.499444in}}{\pgfqpoint{1.550000in}{1.155000in}}%
\pgfusepath{clip}%
\pgfsetbuttcap%
\pgfsetmiterjoin%
\definecolor{currentfill}{rgb}{0.000000,0.000000,0.000000}%
\pgfsetfillcolor{currentfill}%
\pgfsetlinewidth{0.000000pt}%
\definecolor{currentstroke}{rgb}{0.000000,0.000000,0.000000}%
\pgfsetstrokecolor{currentstroke}%
\pgfsetstrokeopacity{0.000000}%
\pgfsetdash{}{0pt}%
\pgfpathmoveto{\pgfqpoint{1.460122in}{0.499444in}}%
\pgfpathlineto{\pgfqpoint{1.520610in}{0.499444in}}%
\pgfpathlineto{\pgfqpoint{1.520610in}{0.540846in}}%
\pgfpathlineto{\pgfqpoint{1.460122in}{0.540846in}}%
\pgfpathlineto{\pgfqpoint{1.460122in}{0.499444in}}%
\pgfpathclose%
\pgfusepath{fill}%
\end{pgfscope}%
\begin{pgfscope}%
\pgfpathrectangle{\pgfqpoint{0.515000in}{0.499444in}}{\pgfqpoint{1.550000in}{1.155000in}}%
\pgfusepath{clip}%
\pgfsetbuttcap%
\pgfsetmiterjoin%
\definecolor{currentfill}{rgb}{0.000000,0.000000,0.000000}%
\pgfsetfillcolor{currentfill}%
\pgfsetlinewidth{0.000000pt}%
\definecolor{currentstroke}{rgb}{0.000000,0.000000,0.000000}%
\pgfsetstrokecolor{currentstroke}%
\pgfsetstrokeopacity{0.000000}%
\pgfsetdash{}{0pt}%
\pgfpathmoveto{\pgfqpoint{1.611342in}{0.499444in}}%
\pgfpathlineto{\pgfqpoint{1.671830in}{0.499444in}}%
\pgfpathlineto{\pgfqpoint{1.671830in}{0.615047in}}%
\pgfpathlineto{\pgfqpoint{1.611342in}{0.615047in}}%
\pgfpathlineto{\pgfqpoint{1.611342in}{0.499444in}}%
\pgfpathclose%
\pgfusepath{fill}%
\end{pgfscope}%
\begin{pgfscope}%
\pgfpathrectangle{\pgfqpoint{0.515000in}{0.499444in}}{\pgfqpoint{1.550000in}{1.155000in}}%
\pgfusepath{clip}%
\pgfsetbuttcap%
\pgfsetmiterjoin%
\definecolor{currentfill}{rgb}{0.000000,0.000000,0.000000}%
\pgfsetfillcolor{currentfill}%
\pgfsetlinewidth{0.000000pt}%
\definecolor{currentstroke}{rgb}{0.000000,0.000000,0.000000}%
\pgfsetstrokecolor{currentstroke}%
\pgfsetstrokeopacity{0.000000}%
\pgfsetdash{}{0pt}%
\pgfpathmoveto{\pgfqpoint{1.762561in}{0.499444in}}%
\pgfpathlineto{\pgfqpoint{1.823049in}{0.499444in}}%
\pgfpathlineto{\pgfqpoint{1.823049in}{0.693211in}}%
\pgfpathlineto{\pgfqpoint{1.762561in}{0.693211in}}%
\pgfpathlineto{\pgfqpoint{1.762561in}{0.499444in}}%
\pgfpathclose%
\pgfusepath{fill}%
\end{pgfscope}%
\begin{pgfscope}%
\pgfpathrectangle{\pgfqpoint{0.515000in}{0.499444in}}{\pgfqpoint{1.550000in}{1.155000in}}%
\pgfusepath{clip}%
\pgfsetbuttcap%
\pgfsetmiterjoin%
\definecolor{currentfill}{rgb}{0.000000,0.000000,0.000000}%
\pgfsetfillcolor{currentfill}%
\pgfsetlinewidth{0.000000pt}%
\definecolor{currentstroke}{rgb}{0.000000,0.000000,0.000000}%
\pgfsetstrokecolor{currentstroke}%
\pgfsetstrokeopacity{0.000000}%
\pgfsetdash{}{0pt}%
\pgfpathmoveto{\pgfqpoint{1.913781in}{0.499444in}}%
\pgfpathlineto{\pgfqpoint{1.974269in}{0.499444in}}%
\pgfpathlineto{\pgfqpoint{1.974269in}{0.577462in}}%
\pgfpathlineto{\pgfqpoint{1.913781in}{0.577462in}}%
\pgfpathlineto{\pgfqpoint{1.913781in}{0.499444in}}%
\pgfpathclose%
\pgfusepath{fill}%
\end{pgfscope}%
\begin{pgfscope}%
\pgfsetbuttcap%
\pgfsetroundjoin%
\definecolor{currentfill}{rgb}{0.000000,0.000000,0.000000}%
\pgfsetfillcolor{currentfill}%
\pgfsetlinewidth{0.803000pt}%
\definecolor{currentstroke}{rgb}{0.000000,0.000000,0.000000}%
\pgfsetstrokecolor{currentstroke}%
\pgfsetdash{}{0pt}%
\pgfsys@defobject{currentmarker}{\pgfqpoint{0.000000in}{-0.048611in}}{\pgfqpoint{0.000000in}{0.000000in}}{%
\pgfpathmoveto{\pgfqpoint{0.000000in}{0.000000in}}%
\pgfpathlineto{\pgfqpoint{0.000000in}{-0.048611in}}%
\pgfusepath{stroke,fill}%
}%
\begin{pgfscope}%
\pgfsys@transformshift{0.552805in}{0.499444in}%
\pgfsys@useobject{currentmarker}{}%
\end{pgfscope}%
\end{pgfscope}%
\begin{pgfscope}%
\definecolor{textcolor}{rgb}{0.000000,0.000000,0.000000}%
\pgfsetstrokecolor{textcolor}%
\pgfsetfillcolor{textcolor}%
\pgftext[x=0.552805in,y=0.402222in,,top]{\color{textcolor}\rmfamily\fontsize{10.000000}{12.000000}\selectfont 0.0}%
\end{pgfscope}%
\begin{pgfscope}%
\pgfsetbuttcap%
\pgfsetroundjoin%
\definecolor{currentfill}{rgb}{0.000000,0.000000,0.000000}%
\pgfsetfillcolor{currentfill}%
\pgfsetlinewidth{0.803000pt}%
\definecolor{currentstroke}{rgb}{0.000000,0.000000,0.000000}%
\pgfsetstrokecolor{currentstroke}%
\pgfsetdash{}{0pt}%
\pgfsys@defobject{currentmarker}{\pgfqpoint{0.000000in}{-0.048611in}}{\pgfqpoint{0.000000in}{0.000000in}}{%
\pgfpathmoveto{\pgfqpoint{0.000000in}{0.000000in}}%
\pgfpathlineto{\pgfqpoint{0.000000in}{-0.048611in}}%
\pgfusepath{stroke,fill}%
}%
\begin{pgfscope}%
\pgfsys@transformshift{0.930854in}{0.499444in}%
\pgfsys@useobject{currentmarker}{}%
\end{pgfscope}%
\end{pgfscope}%
\begin{pgfscope}%
\definecolor{textcolor}{rgb}{0.000000,0.000000,0.000000}%
\pgfsetstrokecolor{textcolor}%
\pgfsetfillcolor{textcolor}%
\pgftext[x=0.930854in,y=0.402222in,,top]{\color{textcolor}\rmfamily\fontsize{10.000000}{12.000000}\selectfont 0.25}%
\end{pgfscope}%
\begin{pgfscope}%
\pgfsetbuttcap%
\pgfsetroundjoin%
\definecolor{currentfill}{rgb}{0.000000,0.000000,0.000000}%
\pgfsetfillcolor{currentfill}%
\pgfsetlinewidth{0.803000pt}%
\definecolor{currentstroke}{rgb}{0.000000,0.000000,0.000000}%
\pgfsetstrokecolor{currentstroke}%
\pgfsetdash{}{0pt}%
\pgfsys@defobject{currentmarker}{\pgfqpoint{0.000000in}{-0.048611in}}{\pgfqpoint{0.000000in}{0.000000in}}{%
\pgfpathmoveto{\pgfqpoint{0.000000in}{0.000000in}}%
\pgfpathlineto{\pgfqpoint{0.000000in}{-0.048611in}}%
\pgfusepath{stroke,fill}%
}%
\begin{pgfscope}%
\pgfsys@transformshift{1.308903in}{0.499444in}%
\pgfsys@useobject{currentmarker}{}%
\end{pgfscope}%
\end{pgfscope}%
\begin{pgfscope}%
\definecolor{textcolor}{rgb}{0.000000,0.000000,0.000000}%
\pgfsetstrokecolor{textcolor}%
\pgfsetfillcolor{textcolor}%
\pgftext[x=1.308903in,y=0.402222in,,top]{\color{textcolor}\rmfamily\fontsize{10.000000}{12.000000}\selectfont 0.5}%
\end{pgfscope}%
\begin{pgfscope}%
\pgfsetbuttcap%
\pgfsetroundjoin%
\definecolor{currentfill}{rgb}{0.000000,0.000000,0.000000}%
\pgfsetfillcolor{currentfill}%
\pgfsetlinewidth{0.803000pt}%
\definecolor{currentstroke}{rgb}{0.000000,0.000000,0.000000}%
\pgfsetstrokecolor{currentstroke}%
\pgfsetdash{}{0pt}%
\pgfsys@defobject{currentmarker}{\pgfqpoint{0.000000in}{-0.048611in}}{\pgfqpoint{0.000000in}{0.000000in}}{%
\pgfpathmoveto{\pgfqpoint{0.000000in}{0.000000in}}%
\pgfpathlineto{\pgfqpoint{0.000000in}{-0.048611in}}%
\pgfusepath{stroke,fill}%
}%
\begin{pgfscope}%
\pgfsys@transformshift{1.686951in}{0.499444in}%
\pgfsys@useobject{currentmarker}{}%
\end{pgfscope}%
\end{pgfscope}%
\begin{pgfscope}%
\definecolor{textcolor}{rgb}{0.000000,0.000000,0.000000}%
\pgfsetstrokecolor{textcolor}%
\pgfsetfillcolor{textcolor}%
\pgftext[x=1.686951in,y=0.402222in,,top]{\color{textcolor}\rmfamily\fontsize{10.000000}{12.000000}\selectfont 0.75}%
\end{pgfscope}%
\begin{pgfscope}%
\pgfsetbuttcap%
\pgfsetroundjoin%
\definecolor{currentfill}{rgb}{0.000000,0.000000,0.000000}%
\pgfsetfillcolor{currentfill}%
\pgfsetlinewidth{0.803000pt}%
\definecolor{currentstroke}{rgb}{0.000000,0.000000,0.000000}%
\pgfsetstrokecolor{currentstroke}%
\pgfsetdash{}{0pt}%
\pgfsys@defobject{currentmarker}{\pgfqpoint{0.000000in}{-0.048611in}}{\pgfqpoint{0.000000in}{0.000000in}}{%
\pgfpathmoveto{\pgfqpoint{0.000000in}{0.000000in}}%
\pgfpathlineto{\pgfqpoint{0.000000in}{-0.048611in}}%
\pgfusepath{stroke,fill}%
}%
\begin{pgfscope}%
\pgfsys@transformshift{2.065000in}{0.499444in}%
\pgfsys@useobject{currentmarker}{}%
\end{pgfscope}%
\end{pgfscope}%
\begin{pgfscope}%
\definecolor{textcolor}{rgb}{0.000000,0.000000,0.000000}%
\pgfsetstrokecolor{textcolor}%
\pgfsetfillcolor{textcolor}%
\pgftext[x=2.065000in,y=0.402222in,,top]{\color{textcolor}\rmfamily\fontsize{10.000000}{12.000000}\selectfont 1.0}%
\end{pgfscope}%
\begin{pgfscope}%
\definecolor{textcolor}{rgb}{0.000000,0.000000,0.000000}%
\pgfsetstrokecolor{textcolor}%
\pgfsetfillcolor{textcolor}%
\pgftext[x=1.290000in,y=0.223333in,,top]{\color{textcolor}\rmfamily\fontsize{10.000000}{12.000000}\selectfont \(\displaystyle p\)}%
\end{pgfscope}%
\begin{pgfscope}%
\pgfsetbuttcap%
\pgfsetroundjoin%
\definecolor{currentfill}{rgb}{0.000000,0.000000,0.000000}%
\pgfsetfillcolor{currentfill}%
\pgfsetlinewidth{0.803000pt}%
\definecolor{currentstroke}{rgb}{0.000000,0.000000,0.000000}%
\pgfsetstrokecolor{currentstroke}%
\pgfsetdash{}{0pt}%
\pgfsys@defobject{currentmarker}{\pgfqpoint{-0.048611in}{0.000000in}}{\pgfqpoint{-0.000000in}{0.000000in}}{%
\pgfpathmoveto{\pgfqpoint{-0.000000in}{0.000000in}}%
\pgfpathlineto{\pgfqpoint{-0.048611in}{0.000000in}}%
\pgfusepath{stroke,fill}%
}%
\begin{pgfscope}%
\pgfsys@transformshift{0.515000in}{0.499444in}%
\pgfsys@useobject{currentmarker}{}%
\end{pgfscope}%
\end{pgfscope}%
\begin{pgfscope}%
\definecolor{textcolor}{rgb}{0.000000,0.000000,0.000000}%
\pgfsetstrokecolor{textcolor}%
\pgfsetfillcolor{textcolor}%
\pgftext[x=0.348333in, y=0.451250in, left, base]{\color{textcolor}\rmfamily\fontsize{10.000000}{12.000000}\selectfont \(\displaystyle {0}\)}%
\end{pgfscope}%
\begin{pgfscope}%
\pgfsetbuttcap%
\pgfsetroundjoin%
\definecolor{currentfill}{rgb}{0.000000,0.000000,0.000000}%
\pgfsetfillcolor{currentfill}%
\pgfsetlinewidth{0.803000pt}%
\definecolor{currentstroke}{rgb}{0.000000,0.000000,0.000000}%
\pgfsetstrokecolor{currentstroke}%
\pgfsetdash{}{0pt}%
\pgfsys@defobject{currentmarker}{\pgfqpoint{-0.048611in}{0.000000in}}{\pgfqpoint{-0.000000in}{0.000000in}}{%
\pgfpathmoveto{\pgfqpoint{-0.000000in}{0.000000in}}%
\pgfpathlineto{\pgfqpoint{-0.048611in}{0.000000in}}%
\pgfusepath{stroke,fill}%
}%
\begin{pgfscope}%
\pgfsys@transformshift{0.515000in}{0.793075in}%
\pgfsys@useobject{currentmarker}{}%
\end{pgfscope}%
\end{pgfscope}%
\begin{pgfscope}%
\definecolor{textcolor}{rgb}{0.000000,0.000000,0.000000}%
\pgfsetstrokecolor{textcolor}%
\pgfsetfillcolor{textcolor}%
\pgftext[x=0.278889in, y=0.744881in, left, base]{\color{textcolor}\rmfamily\fontsize{10.000000}{12.000000}\selectfont \(\displaystyle {10}\)}%
\end{pgfscope}%
\begin{pgfscope}%
\pgfsetbuttcap%
\pgfsetroundjoin%
\definecolor{currentfill}{rgb}{0.000000,0.000000,0.000000}%
\pgfsetfillcolor{currentfill}%
\pgfsetlinewidth{0.803000pt}%
\definecolor{currentstroke}{rgb}{0.000000,0.000000,0.000000}%
\pgfsetstrokecolor{currentstroke}%
\pgfsetdash{}{0pt}%
\pgfsys@defobject{currentmarker}{\pgfqpoint{-0.048611in}{0.000000in}}{\pgfqpoint{-0.000000in}{0.000000in}}{%
\pgfpathmoveto{\pgfqpoint{-0.000000in}{0.000000in}}%
\pgfpathlineto{\pgfqpoint{-0.048611in}{0.000000in}}%
\pgfusepath{stroke,fill}%
}%
\begin{pgfscope}%
\pgfsys@transformshift{0.515000in}{1.086706in}%
\pgfsys@useobject{currentmarker}{}%
\end{pgfscope}%
\end{pgfscope}%
\begin{pgfscope}%
\definecolor{textcolor}{rgb}{0.000000,0.000000,0.000000}%
\pgfsetstrokecolor{textcolor}%
\pgfsetfillcolor{textcolor}%
\pgftext[x=0.278889in, y=1.038511in, left, base]{\color{textcolor}\rmfamily\fontsize{10.000000}{12.000000}\selectfont \(\displaystyle {20}\)}%
\end{pgfscope}%
\begin{pgfscope}%
\pgfsetbuttcap%
\pgfsetroundjoin%
\definecolor{currentfill}{rgb}{0.000000,0.000000,0.000000}%
\pgfsetfillcolor{currentfill}%
\pgfsetlinewidth{0.803000pt}%
\definecolor{currentstroke}{rgb}{0.000000,0.000000,0.000000}%
\pgfsetstrokecolor{currentstroke}%
\pgfsetdash{}{0pt}%
\pgfsys@defobject{currentmarker}{\pgfqpoint{-0.048611in}{0.000000in}}{\pgfqpoint{-0.000000in}{0.000000in}}{%
\pgfpathmoveto{\pgfqpoint{-0.000000in}{0.000000in}}%
\pgfpathlineto{\pgfqpoint{-0.048611in}{0.000000in}}%
\pgfusepath{stroke,fill}%
}%
\begin{pgfscope}%
\pgfsys@transformshift{0.515000in}{1.380337in}%
\pgfsys@useobject{currentmarker}{}%
\end{pgfscope}%
\end{pgfscope}%
\begin{pgfscope}%
\definecolor{textcolor}{rgb}{0.000000,0.000000,0.000000}%
\pgfsetstrokecolor{textcolor}%
\pgfsetfillcolor{textcolor}%
\pgftext[x=0.278889in, y=1.332142in, left, base]{\color{textcolor}\rmfamily\fontsize{10.000000}{12.000000}\selectfont \(\displaystyle {30}\)}%
\end{pgfscope}%
\begin{pgfscope}%
\definecolor{textcolor}{rgb}{0.000000,0.000000,0.000000}%
\pgfsetstrokecolor{textcolor}%
\pgfsetfillcolor{textcolor}%
\pgftext[x=0.223333in,y=1.076944in,,bottom,rotate=90.000000]{\color{textcolor}\rmfamily\fontsize{10.000000}{12.000000}\selectfont Percent of Data Set}%
\end{pgfscope}%
\begin{pgfscope}%
\pgfsetrectcap%
\pgfsetmiterjoin%
\pgfsetlinewidth{0.803000pt}%
\definecolor{currentstroke}{rgb}{0.000000,0.000000,0.000000}%
\pgfsetstrokecolor{currentstroke}%
\pgfsetdash{}{0pt}%
\pgfpathmoveto{\pgfqpoint{0.515000in}{0.499444in}}%
\pgfpathlineto{\pgfqpoint{0.515000in}{1.654444in}}%
\pgfusepath{stroke}%
\end{pgfscope}%
\begin{pgfscope}%
\pgfsetrectcap%
\pgfsetmiterjoin%
\pgfsetlinewidth{0.803000pt}%
\definecolor{currentstroke}{rgb}{0.000000,0.000000,0.000000}%
\pgfsetstrokecolor{currentstroke}%
\pgfsetdash{}{0pt}%
\pgfpathmoveto{\pgfqpoint{2.065000in}{0.499444in}}%
\pgfpathlineto{\pgfqpoint{2.065000in}{1.654444in}}%
\pgfusepath{stroke}%
\end{pgfscope}%
\begin{pgfscope}%
\pgfsetrectcap%
\pgfsetmiterjoin%
\pgfsetlinewidth{0.803000pt}%
\definecolor{currentstroke}{rgb}{0.000000,0.000000,0.000000}%
\pgfsetstrokecolor{currentstroke}%
\pgfsetdash{}{0pt}%
\pgfpathmoveto{\pgfqpoint{0.515000in}{0.499444in}}%
\pgfpathlineto{\pgfqpoint{2.065000in}{0.499444in}}%
\pgfusepath{stroke}%
\end{pgfscope}%
\begin{pgfscope}%
\pgfsetrectcap%
\pgfsetmiterjoin%
\pgfsetlinewidth{0.803000pt}%
\definecolor{currentstroke}{rgb}{0.000000,0.000000,0.000000}%
\pgfsetstrokecolor{currentstroke}%
\pgfsetdash{}{0pt}%
\pgfpathmoveto{\pgfqpoint{0.515000in}{1.654444in}}%
\pgfpathlineto{\pgfqpoint{2.065000in}{1.654444in}}%
\pgfusepath{stroke}%
\end{pgfscope}%
\begin{pgfscope}%
\pgfsetbuttcap%
\pgfsetmiterjoin%
\definecolor{currentfill}{rgb}{1.000000,1.000000,1.000000}%
\pgfsetfillcolor{currentfill}%
\pgfsetfillopacity{0.800000}%
\pgfsetlinewidth{1.003750pt}%
\definecolor{currentstroke}{rgb}{0.800000,0.800000,0.800000}%
\pgfsetstrokecolor{currentstroke}%
\pgfsetstrokeopacity{0.800000}%
\pgfsetdash{}{0pt}%
\pgfpathmoveto{\pgfqpoint{0.612223in}{1.154445in}}%
\pgfpathlineto{\pgfqpoint{1.291945in}{1.154445in}}%
\pgfpathquadraticcurveto{\pgfqpoint{1.319722in}{1.154445in}}{\pgfqpoint{1.319722in}{1.182222in}}%
\pgfpathlineto{\pgfqpoint{1.319722in}{1.557222in}}%
\pgfpathquadraticcurveto{\pgfqpoint{1.319722in}{1.585000in}}{\pgfqpoint{1.291945in}{1.585000in}}%
\pgfpathlineto{\pgfqpoint{0.612223in}{1.585000in}}%
\pgfpathquadraticcurveto{\pgfqpoint{0.584445in}{1.585000in}}{\pgfqpoint{0.584445in}{1.557222in}}%
\pgfpathlineto{\pgfqpoint{0.584445in}{1.182222in}}%
\pgfpathquadraticcurveto{\pgfqpoint{0.584445in}{1.154445in}}{\pgfqpoint{0.612223in}{1.154445in}}%
\pgfpathlineto{\pgfqpoint{0.612223in}{1.154445in}}%
\pgfpathclose%
\pgfusepath{stroke,fill}%
\end{pgfscope}%
\begin{pgfscope}%
\pgfsetbuttcap%
\pgfsetmiterjoin%
\pgfsetlinewidth{1.003750pt}%
\definecolor{currentstroke}{rgb}{0.000000,0.000000,0.000000}%
\pgfsetstrokecolor{currentstroke}%
\pgfsetdash{}{0pt}%
\pgfpathmoveto{\pgfqpoint{0.640000in}{1.432222in}}%
\pgfpathlineto{\pgfqpoint{0.917778in}{1.432222in}}%
\pgfpathlineto{\pgfqpoint{0.917778in}{1.529444in}}%
\pgfpathlineto{\pgfqpoint{0.640000in}{1.529444in}}%
\pgfpathlineto{\pgfqpoint{0.640000in}{1.432222in}}%
\pgfpathclose%
\pgfusepath{stroke}%
\end{pgfscope}%
\begin{pgfscope}%
\definecolor{textcolor}{rgb}{0.000000,0.000000,0.000000}%
\pgfsetstrokecolor{textcolor}%
\pgfsetfillcolor{textcolor}%
\pgftext[x=1.028889in,y=1.432222in,left,base]{\color{textcolor}\rmfamily\fontsize{10.000000}{12.000000}\selectfont Neg}%
\end{pgfscope}%
\begin{pgfscope}%
\pgfsetbuttcap%
\pgfsetmiterjoin%
\definecolor{currentfill}{rgb}{0.000000,0.000000,0.000000}%
\pgfsetfillcolor{currentfill}%
\pgfsetlinewidth{0.000000pt}%
\definecolor{currentstroke}{rgb}{0.000000,0.000000,0.000000}%
\pgfsetstrokecolor{currentstroke}%
\pgfsetstrokeopacity{0.000000}%
\pgfsetdash{}{0pt}%
\pgfpathmoveto{\pgfqpoint{0.640000in}{1.236944in}}%
\pgfpathlineto{\pgfqpoint{0.917778in}{1.236944in}}%
\pgfpathlineto{\pgfqpoint{0.917778in}{1.334167in}}%
\pgfpathlineto{\pgfqpoint{0.640000in}{1.334167in}}%
\pgfpathlineto{\pgfqpoint{0.640000in}{1.236944in}}%
\pgfpathclose%
\pgfusepath{fill}%
\end{pgfscope}%
\begin{pgfscope}%
\definecolor{textcolor}{rgb}{0.000000,0.000000,0.000000}%
\pgfsetstrokecolor{textcolor}%
\pgfsetfillcolor{textcolor}%
\pgftext[x=1.028889in,y=1.236944in,left,base]{\color{textcolor}\rmfamily\fontsize{10.000000}{12.000000}\selectfont Pos}%
\end{pgfscope}%
\end{pgfpicture}%
\makeatother%
\endgroup%

&
	\vskip 0pt
	\qquad \qquad ROC Curve
	
	%% Creator: Matplotlib, PGF backend
%%
%% To include the figure in your LaTeX document, write
%%   \input{<filename>.pgf}
%%
%% Make sure the required packages are loaded in your preamble
%%   \usepackage{pgf}
%%
%% Also ensure that all the required font packages are loaded; for instance,
%% the lmodern package is sometimes necessary when using math font.
%%   \usepackage{lmodern}
%%
%% Figures using additional raster images can only be included by \input if
%% they are in the same directory as the main LaTeX file. For loading figures
%% from other directories you can use the `import` package
%%   \usepackage{import}
%%
%% and then include the figures with
%%   \import{<path to file>}{<filename>.pgf}
%%
%% Matplotlib used the following preamble
%%   
%%   \usepackage{fontspec}
%%   \makeatletter\@ifpackageloaded{underscore}{}{\usepackage[strings]{underscore}}\makeatother
%%
\begingroup%
\makeatletter%
\begin{pgfpicture}%
\pgfpathrectangle{\pgfpointorigin}{\pgfqpoint{2.221861in}{1.754444in}}%
\pgfusepath{use as bounding box, clip}%
\begin{pgfscope}%
\pgfsetbuttcap%
\pgfsetmiterjoin%
\definecolor{currentfill}{rgb}{1.000000,1.000000,1.000000}%
\pgfsetfillcolor{currentfill}%
\pgfsetlinewidth{0.000000pt}%
\definecolor{currentstroke}{rgb}{1.000000,1.000000,1.000000}%
\pgfsetstrokecolor{currentstroke}%
\pgfsetdash{}{0pt}%
\pgfpathmoveto{\pgfqpoint{0.000000in}{0.000000in}}%
\pgfpathlineto{\pgfqpoint{2.221861in}{0.000000in}}%
\pgfpathlineto{\pgfqpoint{2.221861in}{1.754444in}}%
\pgfpathlineto{\pgfqpoint{0.000000in}{1.754444in}}%
\pgfpathlineto{\pgfqpoint{0.000000in}{0.000000in}}%
\pgfpathclose%
\pgfusepath{fill}%
\end{pgfscope}%
\begin{pgfscope}%
\pgfsetbuttcap%
\pgfsetmiterjoin%
\definecolor{currentfill}{rgb}{1.000000,1.000000,1.000000}%
\pgfsetfillcolor{currentfill}%
\pgfsetlinewidth{0.000000pt}%
\definecolor{currentstroke}{rgb}{0.000000,0.000000,0.000000}%
\pgfsetstrokecolor{currentstroke}%
\pgfsetstrokeopacity{0.000000}%
\pgfsetdash{}{0pt}%
\pgfpathmoveto{\pgfqpoint{0.553581in}{0.499444in}}%
\pgfpathlineto{\pgfqpoint{2.103581in}{0.499444in}}%
\pgfpathlineto{\pgfqpoint{2.103581in}{1.654444in}}%
\pgfpathlineto{\pgfqpoint{0.553581in}{1.654444in}}%
\pgfpathlineto{\pgfqpoint{0.553581in}{0.499444in}}%
\pgfpathclose%
\pgfusepath{fill}%
\end{pgfscope}%
\begin{pgfscope}%
\pgfsetbuttcap%
\pgfsetroundjoin%
\definecolor{currentfill}{rgb}{0.000000,0.000000,0.000000}%
\pgfsetfillcolor{currentfill}%
\pgfsetlinewidth{0.803000pt}%
\definecolor{currentstroke}{rgb}{0.000000,0.000000,0.000000}%
\pgfsetstrokecolor{currentstroke}%
\pgfsetdash{}{0pt}%
\pgfsys@defobject{currentmarker}{\pgfqpoint{0.000000in}{-0.048611in}}{\pgfqpoint{0.000000in}{0.000000in}}{%
\pgfpathmoveto{\pgfqpoint{0.000000in}{0.000000in}}%
\pgfpathlineto{\pgfqpoint{0.000000in}{-0.048611in}}%
\pgfusepath{stroke,fill}%
}%
\begin{pgfscope}%
\pgfsys@transformshift{0.624035in}{0.499444in}%
\pgfsys@useobject{currentmarker}{}%
\end{pgfscope}%
\end{pgfscope}%
\begin{pgfscope}%
\definecolor{textcolor}{rgb}{0.000000,0.000000,0.000000}%
\pgfsetstrokecolor{textcolor}%
\pgfsetfillcolor{textcolor}%
\pgftext[x=0.624035in,y=0.402222in,,top]{\color{textcolor}\rmfamily\fontsize{10.000000}{12.000000}\selectfont \(\displaystyle {0.0}\)}%
\end{pgfscope}%
\begin{pgfscope}%
\pgfsetbuttcap%
\pgfsetroundjoin%
\definecolor{currentfill}{rgb}{0.000000,0.000000,0.000000}%
\pgfsetfillcolor{currentfill}%
\pgfsetlinewidth{0.803000pt}%
\definecolor{currentstroke}{rgb}{0.000000,0.000000,0.000000}%
\pgfsetstrokecolor{currentstroke}%
\pgfsetdash{}{0pt}%
\pgfsys@defobject{currentmarker}{\pgfqpoint{0.000000in}{-0.048611in}}{\pgfqpoint{0.000000in}{0.000000in}}{%
\pgfpathmoveto{\pgfqpoint{0.000000in}{0.000000in}}%
\pgfpathlineto{\pgfqpoint{0.000000in}{-0.048611in}}%
\pgfusepath{stroke,fill}%
}%
\begin{pgfscope}%
\pgfsys@transformshift{1.328581in}{0.499444in}%
\pgfsys@useobject{currentmarker}{}%
\end{pgfscope}%
\end{pgfscope}%
\begin{pgfscope}%
\definecolor{textcolor}{rgb}{0.000000,0.000000,0.000000}%
\pgfsetstrokecolor{textcolor}%
\pgfsetfillcolor{textcolor}%
\pgftext[x=1.328581in,y=0.402222in,,top]{\color{textcolor}\rmfamily\fontsize{10.000000}{12.000000}\selectfont \(\displaystyle {0.5}\)}%
\end{pgfscope}%
\begin{pgfscope}%
\pgfsetbuttcap%
\pgfsetroundjoin%
\definecolor{currentfill}{rgb}{0.000000,0.000000,0.000000}%
\pgfsetfillcolor{currentfill}%
\pgfsetlinewidth{0.803000pt}%
\definecolor{currentstroke}{rgb}{0.000000,0.000000,0.000000}%
\pgfsetstrokecolor{currentstroke}%
\pgfsetdash{}{0pt}%
\pgfsys@defobject{currentmarker}{\pgfqpoint{0.000000in}{-0.048611in}}{\pgfqpoint{0.000000in}{0.000000in}}{%
\pgfpathmoveto{\pgfqpoint{0.000000in}{0.000000in}}%
\pgfpathlineto{\pgfqpoint{0.000000in}{-0.048611in}}%
\pgfusepath{stroke,fill}%
}%
\begin{pgfscope}%
\pgfsys@transformshift{2.033126in}{0.499444in}%
\pgfsys@useobject{currentmarker}{}%
\end{pgfscope}%
\end{pgfscope}%
\begin{pgfscope}%
\definecolor{textcolor}{rgb}{0.000000,0.000000,0.000000}%
\pgfsetstrokecolor{textcolor}%
\pgfsetfillcolor{textcolor}%
\pgftext[x=2.033126in,y=0.402222in,,top]{\color{textcolor}\rmfamily\fontsize{10.000000}{12.000000}\selectfont \(\displaystyle {1.0}\)}%
\end{pgfscope}%
\begin{pgfscope}%
\definecolor{textcolor}{rgb}{0.000000,0.000000,0.000000}%
\pgfsetstrokecolor{textcolor}%
\pgfsetfillcolor{textcolor}%
\pgftext[x=1.328581in,y=0.223333in,,top]{\color{textcolor}\rmfamily\fontsize{10.000000}{12.000000}\selectfont False positive rate}%
\end{pgfscope}%
\begin{pgfscope}%
\pgfsetbuttcap%
\pgfsetroundjoin%
\definecolor{currentfill}{rgb}{0.000000,0.000000,0.000000}%
\pgfsetfillcolor{currentfill}%
\pgfsetlinewidth{0.803000pt}%
\definecolor{currentstroke}{rgb}{0.000000,0.000000,0.000000}%
\pgfsetstrokecolor{currentstroke}%
\pgfsetdash{}{0pt}%
\pgfsys@defobject{currentmarker}{\pgfqpoint{-0.048611in}{0.000000in}}{\pgfqpoint{-0.000000in}{0.000000in}}{%
\pgfpathmoveto{\pgfqpoint{-0.000000in}{0.000000in}}%
\pgfpathlineto{\pgfqpoint{-0.048611in}{0.000000in}}%
\pgfusepath{stroke,fill}%
}%
\begin{pgfscope}%
\pgfsys@transformshift{0.553581in}{0.551944in}%
\pgfsys@useobject{currentmarker}{}%
\end{pgfscope}%
\end{pgfscope}%
\begin{pgfscope}%
\definecolor{textcolor}{rgb}{0.000000,0.000000,0.000000}%
\pgfsetstrokecolor{textcolor}%
\pgfsetfillcolor{textcolor}%
\pgftext[x=0.278889in, y=0.503750in, left, base]{\color{textcolor}\rmfamily\fontsize{10.000000}{12.000000}\selectfont \(\displaystyle {0.0}\)}%
\end{pgfscope}%
\begin{pgfscope}%
\pgfsetbuttcap%
\pgfsetroundjoin%
\definecolor{currentfill}{rgb}{0.000000,0.000000,0.000000}%
\pgfsetfillcolor{currentfill}%
\pgfsetlinewidth{0.803000pt}%
\definecolor{currentstroke}{rgb}{0.000000,0.000000,0.000000}%
\pgfsetstrokecolor{currentstroke}%
\pgfsetdash{}{0pt}%
\pgfsys@defobject{currentmarker}{\pgfqpoint{-0.048611in}{0.000000in}}{\pgfqpoint{-0.000000in}{0.000000in}}{%
\pgfpathmoveto{\pgfqpoint{-0.000000in}{0.000000in}}%
\pgfpathlineto{\pgfqpoint{-0.048611in}{0.000000in}}%
\pgfusepath{stroke,fill}%
}%
\begin{pgfscope}%
\pgfsys@transformshift{0.553581in}{1.076944in}%
\pgfsys@useobject{currentmarker}{}%
\end{pgfscope}%
\end{pgfscope}%
\begin{pgfscope}%
\definecolor{textcolor}{rgb}{0.000000,0.000000,0.000000}%
\pgfsetstrokecolor{textcolor}%
\pgfsetfillcolor{textcolor}%
\pgftext[x=0.278889in, y=1.028750in, left, base]{\color{textcolor}\rmfamily\fontsize{10.000000}{12.000000}\selectfont \(\displaystyle {0.5}\)}%
\end{pgfscope}%
\begin{pgfscope}%
\pgfsetbuttcap%
\pgfsetroundjoin%
\definecolor{currentfill}{rgb}{0.000000,0.000000,0.000000}%
\pgfsetfillcolor{currentfill}%
\pgfsetlinewidth{0.803000pt}%
\definecolor{currentstroke}{rgb}{0.000000,0.000000,0.000000}%
\pgfsetstrokecolor{currentstroke}%
\pgfsetdash{}{0pt}%
\pgfsys@defobject{currentmarker}{\pgfqpoint{-0.048611in}{0.000000in}}{\pgfqpoint{-0.000000in}{0.000000in}}{%
\pgfpathmoveto{\pgfqpoint{-0.000000in}{0.000000in}}%
\pgfpathlineto{\pgfqpoint{-0.048611in}{0.000000in}}%
\pgfusepath{stroke,fill}%
}%
\begin{pgfscope}%
\pgfsys@transformshift{0.553581in}{1.601944in}%
\pgfsys@useobject{currentmarker}{}%
\end{pgfscope}%
\end{pgfscope}%
\begin{pgfscope}%
\definecolor{textcolor}{rgb}{0.000000,0.000000,0.000000}%
\pgfsetstrokecolor{textcolor}%
\pgfsetfillcolor{textcolor}%
\pgftext[x=0.278889in, y=1.553750in, left, base]{\color{textcolor}\rmfamily\fontsize{10.000000}{12.000000}\selectfont \(\displaystyle {1.0}\)}%
\end{pgfscope}%
\begin{pgfscope}%
\definecolor{textcolor}{rgb}{0.000000,0.000000,0.000000}%
\pgfsetstrokecolor{textcolor}%
\pgfsetfillcolor{textcolor}%
\pgftext[x=0.223333in,y=1.076944in,,bottom,rotate=90.000000]{\color{textcolor}\rmfamily\fontsize{10.000000}{12.000000}\selectfont True positive rate}%
\end{pgfscope}%
\begin{pgfscope}%
\pgfpathrectangle{\pgfqpoint{0.553581in}{0.499444in}}{\pgfqpoint{1.550000in}{1.155000in}}%
\pgfusepath{clip}%
\pgfsetbuttcap%
\pgfsetroundjoin%
\pgfsetlinewidth{1.505625pt}%
\definecolor{currentstroke}{rgb}{0.000000,0.000000,0.000000}%
\pgfsetstrokecolor{currentstroke}%
\pgfsetdash{{5.550000pt}{2.400000pt}}{0.000000pt}%
\pgfpathmoveto{\pgfqpoint{0.624035in}{0.551944in}}%
\pgfpathlineto{\pgfqpoint{2.033126in}{1.601944in}}%
\pgfusepath{stroke}%
\end{pgfscope}%
\begin{pgfscope}%
\pgfpathrectangle{\pgfqpoint{0.553581in}{0.499444in}}{\pgfqpoint{1.550000in}{1.155000in}}%
\pgfusepath{clip}%
\pgfsetrectcap%
\pgfsetroundjoin%
\pgfsetlinewidth{1.505625pt}%
\definecolor{currentstroke}{rgb}{0.000000,0.000000,0.000000}%
\pgfsetstrokecolor{currentstroke}%
\pgfsetdash{}{0pt}%
\pgfpathmoveto{\pgfqpoint{0.624035in}{0.551944in}}%
\pgfpathlineto{\pgfqpoint{0.626207in}{0.552574in}}%
\pgfpathlineto{\pgfqpoint{0.627318in}{0.561464in}}%
\pgfpathlineto{\pgfqpoint{0.628014in}{0.562514in}}%
\pgfpathlineto{\pgfqpoint{0.629125in}{0.563634in}}%
\pgfpathlineto{\pgfqpoint{0.629605in}{0.564614in}}%
\pgfpathlineto{\pgfqpoint{0.630699in}{0.567694in}}%
\pgfpathlineto{\pgfqpoint{0.631130in}{0.568604in}}%
\pgfpathlineto{\pgfqpoint{0.632225in}{0.573294in}}%
\pgfpathlineto{\pgfqpoint{0.632772in}{0.574064in}}%
\pgfpathlineto{\pgfqpoint{0.633866in}{0.579944in}}%
\pgfpathlineto{\pgfqpoint{0.634247in}{0.580854in}}%
\pgfpathlineto{\pgfqpoint{0.635358in}{0.585824in}}%
\pgfpathlineto{\pgfqpoint{0.635540in}{0.586874in}}%
\pgfpathlineto{\pgfqpoint{0.636634in}{0.593244in}}%
\pgfpathlineto{\pgfqpoint{0.636833in}{0.594154in}}%
\pgfpathlineto{\pgfqpoint{0.637944in}{0.600734in}}%
\pgfpathlineto{\pgfqpoint{0.638126in}{0.601784in}}%
\pgfpathlineto{\pgfqpoint{0.639187in}{0.608714in}}%
\pgfpathlineto{\pgfqpoint{0.639353in}{0.609694in}}%
\pgfpathlineto{\pgfqpoint{0.640464in}{0.616834in}}%
\pgfpathlineto{\pgfqpoint{0.640712in}{0.617884in}}%
\pgfpathlineto{\pgfqpoint{0.641806in}{0.625864in}}%
\pgfpathlineto{\pgfqpoint{0.642038in}{0.626844in}}%
\pgfpathlineto{\pgfqpoint{0.643149in}{0.635804in}}%
\pgfpathlineto{\pgfqpoint{0.643348in}{0.636854in}}%
\pgfpathlineto{\pgfqpoint{0.644392in}{0.645044in}}%
\pgfpathlineto{\pgfqpoint{0.644641in}{0.645884in}}%
\pgfpathlineto{\pgfqpoint{0.645752in}{0.652394in}}%
\pgfpathlineto{\pgfqpoint{0.645868in}{0.653164in}}%
\pgfpathlineto{\pgfqpoint{0.646979in}{0.661424in}}%
\pgfpathlineto{\pgfqpoint{0.647327in}{0.662264in}}%
\pgfpathlineto{\pgfqpoint{0.648437in}{0.668914in}}%
\pgfpathlineto{\pgfqpoint{0.648587in}{0.669824in}}%
\pgfpathlineto{\pgfqpoint{0.649697in}{0.678294in}}%
\pgfpathlineto{\pgfqpoint{0.649813in}{0.679274in}}%
\pgfpathlineto{\pgfqpoint{0.650924in}{0.687044in}}%
\pgfpathlineto{\pgfqpoint{0.651140in}{0.687884in}}%
\pgfpathlineto{\pgfqpoint{0.652250in}{0.696144in}}%
\pgfpathlineto{\pgfqpoint{0.652333in}{0.697194in}}%
\pgfpathlineto{\pgfqpoint{0.653444in}{0.706434in}}%
\pgfpathlineto{\pgfqpoint{0.653858in}{0.707414in}}%
\pgfpathlineto{\pgfqpoint{0.654969in}{0.716584in}}%
\pgfpathlineto{\pgfqpoint{0.655151in}{0.717564in}}%
\pgfpathlineto{\pgfqpoint{0.656262in}{0.725194in}}%
\pgfpathlineto{\pgfqpoint{0.656444in}{0.726174in}}%
\pgfpathlineto{\pgfqpoint{0.657538in}{0.733244in}}%
\pgfpathlineto{\pgfqpoint{0.657704in}{0.734294in}}%
\pgfpathlineto{\pgfqpoint{0.658815in}{0.743464in}}%
\pgfpathlineto{\pgfqpoint{0.659064in}{0.744514in}}%
\pgfpathlineto{\pgfqpoint{0.660158in}{0.751164in}}%
\pgfpathlineto{\pgfqpoint{0.660290in}{0.751794in}}%
\pgfpathlineto{\pgfqpoint{0.661401in}{0.759354in}}%
\pgfpathlineto{\pgfqpoint{0.661633in}{0.760404in}}%
\pgfpathlineto{\pgfqpoint{0.662744in}{0.769784in}}%
\pgfpathlineto{\pgfqpoint{0.662827in}{0.770344in}}%
\pgfpathlineto{\pgfqpoint{0.663937in}{0.777764in}}%
\pgfpathlineto{\pgfqpoint{0.664120in}{0.778814in}}%
\pgfpathlineto{\pgfqpoint{0.665181in}{0.783854in}}%
\pgfpathlineto{\pgfqpoint{0.665297in}{0.784834in}}%
\pgfpathlineto{\pgfqpoint{0.666407in}{0.795614in}}%
\pgfpathlineto{\pgfqpoint{0.666656in}{0.796664in}}%
\pgfpathlineto{\pgfqpoint{0.667767in}{0.801564in}}%
\pgfpathlineto{\pgfqpoint{0.667899in}{0.802404in}}%
\pgfpathlineto{\pgfqpoint{0.669010in}{0.809334in}}%
\pgfpathlineto{\pgfqpoint{0.669342in}{0.810384in}}%
\pgfpathlineto{\pgfqpoint{0.670452in}{0.818084in}}%
\pgfpathlineto{\pgfqpoint{0.670701in}{0.819134in}}%
\pgfpathlineto{\pgfqpoint{0.671812in}{0.825364in}}%
\pgfpathlineto{\pgfqpoint{0.671911in}{0.826414in}}%
\pgfpathlineto{\pgfqpoint{0.673022in}{0.833694in}}%
\pgfpathlineto{\pgfqpoint{0.673138in}{0.834744in}}%
\pgfpathlineto{\pgfqpoint{0.674215in}{0.838944in}}%
\pgfpathlineto{\pgfqpoint{0.674530in}{0.839994in}}%
\pgfpathlineto{\pgfqpoint{0.675608in}{0.846014in}}%
\pgfpathlineto{\pgfqpoint{0.675790in}{0.846994in}}%
\pgfpathlineto{\pgfqpoint{0.676901in}{0.853854in}}%
\pgfpathlineto{\pgfqpoint{0.677199in}{0.854834in}}%
\pgfpathlineto{\pgfqpoint{0.678310in}{0.859034in}}%
\pgfpathlineto{\pgfqpoint{0.678492in}{0.860084in}}%
\pgfpathlineto{\pgfqpoint{0.679587in}{0.867014in}}%
\pgfpathlineto{\pgfqpoint{0.679868in}{0.867924in}}%
\pgfpathlineto{\pgfqpoint{0.680963in}{0.874504in}}%
\pgfpathlineto{\pgfqpoint{0.681228in}{0.875484in}}%
\pgfpathlineto{\pgfqpoint{0.682338in}{0.882204in}}%
\pgfpathlineto{\pgfqpoint{0.682587in}{0.883254in}}%
\pgfpathlineto{\pgfqpoint{0.683681in}{0.889484in}}%
\pgfpathlineto{\pgfqpoint{0.683980in}{0.890394in}}%
\pgfpathlineto{\pgfqpoint{0.685074in}{0.897394in}}%
\pgfpathlineto{\pgfqpoint{0.685389in}{0.898304in}}%
\pgfpathlineto{\pgfqpoint{0.686450in}{0.904184in}}%
\pgfpathlineto{\pgfqpoint{0.686748in}{0.905094in}}%
\pgfpathlineto{\pgfqpoint{0.687859in}{0.910204in}}%
\pgfpathlineto{\pgfqpoint{0.688074in}{0.911114in}}%
\pgfpathlineto{\pgfqpoint{0.689185in}{0.915034in}}%
\pgfpathlineto{\pgfqpoint{0.689400in}{0.916084in}}%
\pgfpathlineto{\pgfqpoint{0.690511in}{0.922664in}}%
\pgfpathlineto{\pgfqpoint{0.690793in}{0.923714in}}%
\pgfpathlineto{\pgfqpoint{0.691887in}{0.928754in}}%
\pgfpathlineto{\pgfqpoint{0.692318in}{0.929734in}}%
\pgfpathlineto{\pgfqpoint{0.693412in}{0.935334in}}%
\pgfpathlineto{\pgfqpoint{0.693611in}{0.936384in}}%
\pgfpathlineto{\pgfqpoint{0.694705in}{0.940864in}}%
\pgfpathlineto{\pgfqpoint{0.695053in}{0.941914in}}%
\pgfpathlineto{\pgfqpoint{0.696164in}{0.948634in}}%
\pgfpathlineto{\pgfqpoint{0.696645in}{0.949684in}}%
\pgfpathlineto{\pgfqpoint{0.697756in}{0.954164in}}%
\pgfpathlineto{\pgfqpoint{0.697921in}{0.955144in}}%
\pgfpathlineto{\pgfqpoint{0.699015in}{0.960464in}}%
\pgfpathlineto{\pgfqpoint{0.699314in}{0.961304in}}%
\pgfpathlineto{\pgfqpoint{0.700425in}{0.965294in}}%
\pgfpathlineto{\pgfqpoint{0.700657in}{0.966344in}}%
\pgfpathlineto{\pgfqpoint{0.701767in}{0.971314in}}%
\pgfpathlineto{\pgfqpoint{0.702049in}{0.972364in}}%
\pgfpathlineto{\pgfqpoint{0.703094in}{0.977194in}}%
\pgfpathlineto{\pgfqpoint{0.703442in}{0.978104in}}%
\pgfpathlineto{\pgfqpoint{0.704536in}{0.982514in}}%
\pgfpathlineto{\pgfqpoint{0.704917in}{0.983564in}}%
\pgfpathlineto{\pgfqpoint{0.706011in}{0.988464in}}%
\pgfpathlineto{\pgfqpoint{0.706160in}{0.989444in}}%
\pgfpathlineto{\pgfqpoint{0.707271in}{0.993504in}}%
\pgfpathlineto{\pgfqpoint{0.707619in}{0.994414in}}%
\pgfpathlineto{\pgfqpoint{0.708697in}{0.998054in}}%
\pgfpathlineto{\pgfqpoint{0.709061in}{0.999104in}}%
\pgfpathlineto{\pgfqpoint{0.710172in}{1.004564in}}%
\pgfpathlineto{\pgfqpoint{0.710504in}{1.005614in}}%
\pgfpathlineto{\pgfqpoint{0.711614in}{1.009604in}}%
\pgfpathlineto{\pgfqpoint{0.711830in}{1.010584in}}%
\pgfpathlineto{\pgfqpoint{0.712874in}{1.015134in}}%
\pgfpathlineto{\pgfqpoint{0.713123in}{1.016184in}}%
\pgfpathlineto{\pgfqpoint{0.714184in}{1.020244in}}%
\pgfpathlineto{\pgfqpoint{0.714598in}{1.021294in}}%
\pgfpathlineto{\pgfqpoint{0.715709in}{1.024304in}}%
\pgfpathlineto{\pgfqpoint{0.716173in}{1.025354in}}%
\pgfpathlineto{\pgfqpoint{0.717284in}{1.029624in}}%
\pgfpathlineto{\pgfqpoint{0.717466in}{1.030604in}}%
\pgfpathlineto{\pgfqpoint{0.718560in}{1.035014in}}%
\pgfpathlineto{\pgfqpoint{0.718776in}{1.035994in}}%
\pgfpathlineto{\pgfqpoint{0.719853in}{1.041034in}}%
\pgfpathlineto{\pgfqpoint{0.720202in}{1.042084in}}%
\pgfpathlineto{\pgfqpoint{0.721312in}{1.046144in}}%
\pgfpathlineto{\pgfqpoint{0.721694in}{1.047194in}}%
\pgfpathlineto{\pgfqpoint{0.722788in}{1.050904in}}%
\pgfpathlineto{\pgfqpoint{0.723086in}{1.051954in}}%
\pgfpathlineto{\pgfqpoint{0.724180in}{1.056364in}}%
\pgfpathlineto{\pgfqpoint{0.724727in}{1.057414in}}%
\pgfpathlineto{\pgfqpoint{0.725805in}{1.061264in}}%
\pgfpathlineto{\pgfqpoint{0.726236in}{1.062314in}}%
\pgfpathlineto{\pgfqpoint{0.727280in}{1.065744in}}%
\pgfpathlineto{\pgfqpoint{0.727728in}{1.066794in}}%
\pgfpathlineto{\pgfqpoint{0.728805in}{1.069874in}}%
\pgfpathlineto{\pgfqpoint{0.729253in}{1.070924in}}%
\pgfpathlineto{\pgfqpoint{0.730364in}{1.074144in}}%
\pgfpathlineto{\pgfqpoint{0.730695in}{1.075124in}}%
\pgfpathlineto{\pgfqpoint{0.731789in}{1.079044in}}%
\pgfpathlineto{\pgfqpoint{0.732336in}{1.080024in}}%
\pgfpathlineto{\pgfqpoint{0.733447in}{1.084574in}}%
\pgfpathlineto{\pgfqpoint{0.734094in}{1.085624in}}%
\pgfpathlineto{\pgfqpoint{0.735188in}{1.089544in}}%
\pgfpathlineto{\pgfqpoint{0.735552in}{1.090594in}}%
\pgfpathlineto{\pgfqpoint{0.736597in}{1.093674in}}%
\pgfpathlineto{\pgfqpoint{0.737177in}{1.094724in}}%
\pgfpathlineto{\pgfqpoint{0.738221in}{1.097524in}}%
\pgfpathlineto{\pgfqpoint{0.738702in}{1.098574in}}%
\pgfpathlineto{\pgfqpoint{0.739780in}{1.102424in}}%
\pgfpathlineto{\pgfqpoint{0.740128in}{1.103474in}}%
\pgfpathlineto{\pgfqpoint{0.741238in}{1.106624in}}%
\pgfpathlineto{\pgfqpoint{0.741636in}{1.107674in}}%
\pgfpathlineto{\pgfqpoint{0.742697in}{1.111034in}}%
\pgfpathlineto{\pgfqpoint{0.743095in}{1.112084in}}%
\pgfpathlineto{\pgfqpoint{0.744206in}{1.115024in}}%
\pgfpathlineto{\pgfqpoint{0.744769in}{1.116074in}}%
\pgfpathlineto{\pgfqpoint{0.745051in}{1.117054in}}%
\pgfpathlineto{\pgfqpoint{0.745068in}{1.117054in}}%
\pgfpathlineto{\pgfqpoint{0.755943in}{1.118104in}}%
\pgfpathlineto{\pgfqpoint{0.757053in}{1.121814in}}%
\pgfpathlineto{\pgfqpoint{0.757600in}{1.122864in}}%
\pgfpathlineto{\pgfqpoint{0.758678in}{1.125384in}}%
\pgfpathlineto{\pgfqpoint{0.759225in}{1.126434in}}%
\pgfpathlineto{\pgfqpoint{0.760269in}{1.129094in}}%
\pgfpathlineto{\pgfqpoint{0.760817in}{1.130144in}}%
\pgfpathlineto{\pgfqpoint{0.761911in}{1.133224in}}%
\pgfpathlineto{\pgfqpoint{0.762474in}{1.134274in}}%
\pgfpathlineto{\pgfqpoint{0.763535in}{1.137844in}}%
\pgfpathlineto{\pgfqpoint{0.764165in}{1.138824in}}%
\pgfpathlineto{\pgfqpoint{0.765259in}{1.142324in}}%
\pgfpathlineto{\pgfqpoint{0.766055in}{1.143374in}}%
\pgfpathlineto{\pgfqpoint{0.767166in}{1.146314in}}%
\pgfpathlineto{\pgfqpoint{0.767696in}{1.147364in}}%
\pgfpathlineto{\pgfqpoint{0.768790in}{1.149954in}}%
\pgfpathlineto{\pgfqpoint{0.769603in}{1.151004in}}%
\pgfpathlineto{\pgfqpoint{0.770514in}{1.153314in}}%
\pgfpathlineto{\pgfqpoint{0.771161in}{1.154364in}}%
\pgfpathlineto{\pgfqpoint{0.772272in}{1.157374in}}%
\pgfpathlineto{\pgfqpoint{0.772852in}{1.158424in}}%
\pgfpathlineto{\pgfqpoint{0.773946in}{1.160664in}}%
\pgfpathlineto{\pgfqpoint{0.774609in}{1.161714in}}%
\pgfpathlineto{\pgfqpoint{0.775703in}{1.164794in}}%
\pgfpathlineto{\pgfqpoint{0.776267in}{1.165774in}}%
\pgfpathlineto{\pgfqpoint{0.777344in}{1.168714in}}%
\pgfpathlineto{\pgfqpoint{0.777759in}{1.169764in}}%
\pgfpathlineto{\pgfqpoint{0.778836in}{1.172144in}}%
\pgfpathlineto{\pgfqpoint{0.779350in}{1.173124in}}%
\pgfpathlineto{\pgfqpoint{0.780444in}{1.174524in}}%
\pgfpathlineto{\pgfqpoint{0.781008in}{1.175574in}}%
\pgfpathlineto{\pgfqpoint{0.782052in}{1.177254in}}%
\pgfpathlineto{\pgfqpoint{0.782699in}{1.178304in}}%
\pgfpathlineto{\pgfqpoint{0.783793in}{1.180894in}}%
\pgfpathlineto{\pgfqpoint{0.784290in}{1.181874in}}%
\pgfpathlineto{\pgfqpoint{0.785384in}{1.184044in}}%
\pgfpathlineto{\pgfqpoint{0.785981in}{1.185094in}}%
\pgfpathlineto{\pgfqpoint{0.787075in}{1.186914in}}%
\pgfpathlineto{\pgfqpoint{0.787755in}{1.187894in}}%
\pgfpathlineto{\pgfqpoint{0.788866in}{1.189784in}}%
\pgfpathlineto{\pgfqpoint{0.789297in}{1.190764in}}%
\pgfpathlineto{\pgfqpoint{0.790407in}{1.192794in}}%
\pgfpathlineto{\pgfqpoint{0.790689in}{1.193634in}}%
\pgfpathlineto{\pgfqpoint{0.791783in}{1.196084in}}%
\pgfpathlineto{\pgfqpoint{0.792347in}{1.197134in}}%
\pgfpathlineto{\pgfqpoint{0.793358in}{1.199724in}}%
\pgfpathlineto{\pgfqpoint{0.794054in}{1.200774in}}%
\pgfpathlineto{\pgfqpoint{0.795099in}{1.201964in}}%
\pgfpathlineto{\pgfqpoint{0.796044in}{1.203014in}}%
\pgfpathlineto{\pgfqpoint{0.797138in}{1.205534in}}%
\pgfpathlineto{\pgfqpoint{0.797718in}{1.206584in}}%
\pgfpathlineto{\pgfqpoint{0.798812in}{1.208474in}}%
\pgfpathlineto{\pgfqpoint{0.799409in}{1.209524in}}%
\pgfpathlineto{\pgfqpoint{0.800520in}{1.212114in}}%
\pgfpathlineto{\pgfqpoint{0.801382in}{1.213094in}}%
\pgfpathlineto{\pgfqpoint{0.802492in}{1.215894in}}%
\pgfpathlineto{\pgfqpoint{0.803106in}{1.216664in}}%
\pgfpathlineto{\pgfqpoint{0.804134in}{1.218554in}}%
\pgfpathlineto{\pgfqpoint{0.805128in}{1.219604in}}%
\pgfpathlineto{\pgfqpoint{0.806222in}{1.221774in}}%
\pgfpathlineto{\pgfqpoint{0.806786in}{1.222614in}}%
\pgfpathlineto{\pgfqpoint{0.807897in}{1.225484in}}%
\pgfpathlineto{\pgfqpoint{0.808659in}{1.226534in}}%
\pgfpathlineto{\pgfqpoint{0.809753in}{1.228284in}}%
\pgfpathlineto{\pgfqpoint{0.810267in}{1.229334in}}%
\pgfpathlineto{\pgfqpoint{0.811378in}{1.231224in}}%
\pgfpathlineto{\pgfqpoint{0.812654in}{1.232274in}}%
\pgfpathlineto{\pgfqpoint{0.813682in}{1.233744in}}%
\pgfpathlineto{\pgfqpoint{0.814180in}{1.234794in}}%
\pgfpathlineto{\pgfqpoint{0.815290in}{1.236614in}}%
\pgfpathlineto{\pgfqpoint{0.816053in}{1.237664in}}%
\pgfpathlineto{\pgfqpoint{0.816948in}{1.239484in}}%
\pgfpathlineto{\pgfqpoint{0.818109in}{1.240464in}}%
\pgfpathlineto{\pgfqpoint{0.819203in}{1.242424in}}%
\pgfpathlineto{\pgfqpoint{0.819949in}{1.243404in}}%
\pgfpathlineto{\pgfqpoint{0.821043in}{1.245364in}}%
\pgfpathlineto{\pgfqpoint{0.821838in}{1.246414in}}%
\pgfpathlineto{\pgfqpoint{0.822899in}{1.248374in}}%
\pgfpathlineto{\pgfqpoint{0.823828in}{1.249424in}}%
\pgfpathlineto{\pgfqpoint{0.824905in}{1.251314in}}%
\pgfpathlineto{\pgfqpoint{0.826049in}{1.252364in}}%
\pgfpathlineto{\pgfqpoint{0.827110in}{1.254044in}}%
\pgfpathlineto{\pgfqpoint{0.828204in}{1.255024in}}%
\pgfpathlineto{\pgfqpoint{0.829298in}{1.257404in}}%
\pgfpathlineto{\pgfqpoint{0.830094in}{1.258454in}}%
\pgfpathlineto{\pgfqpoint{0.831188in}{1.259784in}}%
\pgfpathlineto{\pgfqpoint{0.831901in}{1.260834in}}%
\pgfpathlineto{\pgfqpoint{0.832979in}{1.261954in}}%
\pgfpathlineto{\pgfqpoint{0.833841in}{1.263004in}}%
\pgfpathlineto{\pgfqpoint{0.834769in}{1.263774in}}%
\pgfpathlineto{\pgfqpoint{0.835697in}{1.264824in}}%
\pgfpathlineto{\pgfqpoint{0.836659in}{1.266014in}}%
\pgfpathlineto{\pgfqpoint{0.838234in}{1.267064in}}%
\pgfpathlineto{\pgfqpoint{0.839295in}{1.268814in}}%
\pgfpathlineto{\pgfqpoint{0.840438in}{1.269864in}}%
\pgfpathlineto{\pgfqpoint{0.841549in}{1.271824in}}%
\pgfpathlineto{\pgfqpoint{0.842594in}{1.272874in}}%
\pgfpathlineto{\pgfqpoint{0.843688in}{1.274414in}}%
\pgfpathlineto{\pgfqpoint{0.844881in}{1.275464in}}%
\pgfpathlineto{\pgfqpoint{0.845992in}{1.277844in}}%
\pgfpathlineto{\pgfqpoint{0.846705in}{1.278894in}}%
\pgfpathlineto{\pgfqpoint{0.847633in}{1.280294in}}%
\pgfpathlineto{\pgfqpoint{0.848644in}{1.281274in}}%
\pgfpathlineto{\pgfqpoint{0.849606in}{1.282954in}}%
\pgfpathlineto{\pgfqpoint{0.850882in}{1.284004in}}%
\pgfpathlineto{\pgfqpoint{0.851960in}{1.285194in}}%
\pgfpathlineto{\pgfqpoint{0.853120in}{1.286244in}}%
\pgfpathlineto{\pgfqpoint{0.854198in}{1.287644in}}%
\pgfpathlineto{\pgfqpoint{0.855176in}{1.288694in}}%
\pgfpathlineto{\pgfqpoint{0.856237in}{1.290234in}}%
\pgfpathlineto{\pgfqpoint{0.856917in}{1.291144in}}%
\pgfpathlineto{\pgfqpoint{0.858027in}{1.292754in}}%
\pgfpathlineto{\pgfqpoint{0.859287in}{1.293804in}}%
\pgfpathlineto{\pgfqpoint{0.860232in}{1.294714in}}%
\pgfpathlineto{\pgfqpoint{0.861160in}{1.295764in}}%
\pgfpathlineto{\pgfqpoint{0.862155in}{1.296814in}}%
\pgfpathlineto{\pgfqpoint{0.863746in}{1.297794in}}%
\pgfpathlineto{\pgfqpoint{0.864807in}{1.299054in}}%
\pgfpathlineto{\pgfqpoint{0.866084in}{1.300104in}}%
\pgfpathlineto{\pgfqpoint{0.867128in}{1.301854in}}%
\pgfpathlineto{\pgfqpoint{0.867775in}{1.302904in}}%
\pgfpathlineto{\pgfqpoint{0.868886in}{1.304164in}}%
\pgfpathlineto{\pgfqpoint{0.869897in}{1.305144in}}%
\pgfpathlineto{\pgfqpoint{0.870991in}{1.306614in}}%
\pgfpathlineto{\pgfqpoint{0.872218in}{1.307664in}}%
\pgfpathlineto{\pgfqpoint{0.873312in}{1.309624in}}%
\pgfpathlineto{\pgfqpoint{0.874572in}{1.310674in}}%
\pgfpathlineto{\pgfqpoint{0.875616in}{1.312284in}}%
\pgfpathlineto{\pgfqpoint{0.876263in}{1.313334in}}%
\pgfpathlineto{\pgfqpoint{0.877373in}{1.314804in}}%
\pgfpathlineto{\pgfqpoint{0.878036in}{1.315854in}}%
\pgfpathlineto{\pgfqpoint{0.879081in}{1.316974in}}%
\pgfpathlineto{\pgfqpoint{0.880191in}{1.318024in}}%
\pgfpathlineto{\pgfqpoint{0.881252in}{1.319564in}}%
\pgfpathlineto{\pgfqpoint{0.882927in}{1.320544in}}%
\pgfpathlineto{\pgfqpoint{0.883954in}{1.321384in}}%
\pgfpathlineto{\pgfqpoint{0.885165in}{1.322434in}}%
\pgfpathlineto{\pgfqpoint{0.886226in}{1.323624in}}%
\pgfpathlineto{\pgfqpoint{0.887121in}{1.324674in}}%
\pgfpathlineto{\pgfqpoint{0.888215in}{1.325864in}}%
\pgfpathlineto{\pgfqpoint{0.889624in}{1.326914in}}%
\pgfpathlineto{\pgfqpoint{0.890619in}{1.328104in}}%
\pgfpathlineto{\pgfqpoint{0.892044in}{1.329154in}}%
\pgfpathlineto{\pgfqpoint{0.892857in}{1.330064in}}%
\pgfpathlineto{\pgfqpoint{0.894100in}{1.331044in}}%
\pgfpathlineto{\pgfqpoint{0.894746in}{1.331814in}}%
\pgfpathlineto{\pgfqpoint{0.896089in}{1.332864in}}%
\pgfpathlineto{\pgfqpoint{0.897200in}{1.333984in}}%
\pgfpathlineto{\pgfqpoint{0.898327in}{1.335034in}}%
\pgfpathlineto{\pgfqpoint{0.899338in}{1.336714in}}%
\pgfpathlineto{\pgfqpoint{0.900267in}{1.337694in}}%
\pgfpathlineto{\pgfqpoint{0.901311in}{1.338534in}}%
\pgfpathlineto{\pgfqpoint{0.902554in}{1.339584in}}%
\pgfpathlineto{\pgfqpoint{0.903649in}{1.340494in}}%
\pgfpathlineto{\pgfqpoint{0.905240in}{1.341544in}}%
\pgfpathlineto{\pgfqpoint{0.906334in}{1.342874in}}%
\pgfpathlineto{\pgfqpoint{0.907743in}{1.343924in}}%
\pgfpathlineto{\pgfqpoint{0.908854in}{1.344764in}}%
\pgfpathlineto{\pgfqpoint{0.910048in}{1.345814in}}%
\pgfpathlineto{\pgfqpoint{0.911142in}{1.346654in}}%
\pgfpathlineto{\pgfqpoint{0.912169in}{1.347704in}}%
\pgfpathlineto{\pgfqpoint{0.913114in}{1.348754in}}%
\pgfpathlineto{\pgfqpoint{0.914242in}{1.349804in}}%
\pgfpathlineto{\pgfqpoint{0.915004in}{1.350924in}}%
\pgfpathlineto{\pgfqpoint{0.916894in}{1.351974in}}%
\pgfpathlineto{\pgfqpoint{0.917988in}{1.353234in}}%
\pgfpathlineto{\pgfqpoint{0.919646in}{1.354284in}}%
\pgfpathlineto{\pgfqpoint{0.920757in}{1.356034in}}%
\pgfpathlineto{\pgfqpoint{0.922166in}{1.357084in}}%
\pgfpathlineto{\pgfqpoint{0.923260in}{1.358204in}}%
\pgfpathlineto{\pgfqpoint{0.925332in}{1.359254in}}%
\pgfpathlineto{\pgfqpoint{0.926376in}{1.360164in}}%
\pgfpathlineto{\pgfqpoint{0.927686in}{1.361214in}}%
\pgfpathlineto{\pgfqpoint{0.928780in}{1.361914in}}%
\pgfpathlineto{\pgfqpoint{0.930886in}{1.362964in}}%
\pgfpathlineto{\pgfqpoint{0.931980in}{1.364294in}}%
\pgfpathlineto{\pgfqpoint{0.932941in}{1.365344in}}%
\pgfpathlineto{\pgfqpoint{0.934035in}{1.366604in}}%
\pgfpathlineto{\pgfqpoint{0.935925in}{1.367654in}}%
\pgfpathlineto{\pgfqpoint{0.937003in}{1.368984in}}%
\pgfpathlineto{\pgfqpoint{0.939307in}{1.370034in}}%
\pgfpathlineto{\pgfqpoint{0.940302in}{1.371014in}}%
\pgfpathlineto{\pgfqpoint{0.941876in}{1.372064in}}%
\pgfpathlineto{\pgfqpoint{0.942871in}{1.372694in}}%
\pgfpathlineto{\pgfqpoint{0.944562in}{1.373674in}}%
\pgfpathlineto{\pgfqpoint{0.945573in}{1.374934in}}%
\pgfpathlineto{\pgfqpoint{0.947314in}{1.375984in}}%
\pgfpathlineto{\pgfqpoint{0.948159in}{1.376824in}}%
\pgfpathlineto{\pgfqpoint{0.950663in}{1.377874in}}%
\pgfpathlineto{\pgfqpoint{0.951707in}{1.378504in}}%
\pgfpathlineto{\pgfqpoint{0.953083in}{1.379484in}}%
\pgfpathlineto{\pgfqpoint{0.954127in}{1.380394in}}%
\pgfpathlineto{\pgfqpoint{0.956183in}{1.381444in}}%
\pgfpathlineto{\pgfqpoint{0.957244in}{1.382914in}}%
\pgfpathlineto{\pgfqpoint{0.958736in}{1.383964in}}%
\pgfpathlineto{\pgfqpoint{0.959714in}{1.384874in}}%
\pgfpathlineto{\pgfqpoint{0.961902in}{1.385924in}}%
\pgfpathlineto{\pgfqpoint{0.962781in}{1.386414in}}%
\pgfpathlineto{\pgfqpoint{0.965599in}{1.387464in}}%
\pgfpathlineto{\pgfqpoint{0.966643in}{1.388514in}}%
\pgfpathlineto{\pgfqpoint{0.968069in}{1.389564in}}%
\pgfpathlineto{\pgfqpoint{0.968931in}{1.390404in}}%
\pgfpathlineto{\pgfqpoint{0.971733in}{1.391454in}}%
\pgfpathlineto{\pgfqpoint{0.972843in}{1.391874in}}%
\pgfpathlineto{\pgfqpoint{0.974286in}{1.392924in}}%
\pgfpathlineto{\pgfqpoint{0.975197in}{1.393554in}}%
\pgfpathlineto{\pgfqpoint{0.977336in}{1.394604in}}%
\pgfpathlineto{\pgfqpoint{0.978413in}{1.395654in}}%
\pgfpathlineto{\pgfqpoint{0.979607in}{1.396704in}}%
\pgfpathlineto{\pgfqpoint{0.980635in}{1.397614in}}%
\pgfpathlineto{\pgfqpoint{0.981729in}{1.398664in}}%
\pgfpathlineto{\pgfqpoint{0.982690in}{1.399224in}}%
\pgfpathlineto{\pgfqpoint{0.984365in}{1.400274in}}%
\pgfpathlineto{\pgfqpoint{0.985343in}{1.401044in}}%
\pgfpathlineto{\pgfqpoint{0.987050in}{1.402094in}}%
\pgfpathlineto{\pgfqpoint{0.987912in}{1.403004in}}%
\pgfpathlineto{\pgfqpoint{0.990051in}{1.404054in}}%
\pgfpathlineto{\pgfqpoint{0.990631in}{1.404474in}}%
\pgfpathlineto{\pgfqpoint{0.992935in}{1.405524in}}%
\pgfpathlineto{\pgfqpoint{0.993847in}{1.406294in}}%
\pgfpathlineto{\pgfqpoint{0.995787in}{1.407344in}}%
\pgfpathlineto{\pgfqpoint{0.996798in}{1.408534in}}%
\pgfpathlineto{\pgfqpoint{0.998456in}{1.409444in}}%
\pgfpathlineto{\pgfqpoint{0.999533in}{1.410424in}}%
\pgfpathlineto{\pgfqpoint{1.001224in}{1.411404in}}%
\pgfpathlineto{\pgfqpoint{1.002235in}{1.412314in}}%
\pgfpathlineto{\pgfqpoint{1.003545in}{1.413364in}}%
\pgfpathlineto{\pgfqpoint{1.004423in}{1.414274in}}%
\pgfpathlineto{\pgfqpoint{1.005882in}{1.415324in}}%
\pgfpathlineto{\pgfqpoint{1.006678in}{1.415954in}}%
\pgfpathlineto{\pgfqpoint{1.009165in}{1.417004in}}%
\pgfpathlineto{\pgfqpoint{1.010226in}{1.417774in}}%
\pgfpathlineto{\pgfqpoint{1.012861in}{1.418824in}}%
\pgfpathlineto{\pgfqpoint{1.013972in}{1.419524in}}%
\pgfpathlineto{\pgfqpoint{1.016442in}{1.420504in}}%
\pgfpathlineto{\pgfqpoint{1.017437in}{1.421274in}}%
\pgfpathlineto{\pgfqpoint{1.019890in}{1.422324in}}%
\pgfpathlineto{\pgfqpoint{1.020852in}{1.423444in}}%
\pgfpathlineto{\pgfqpoint{1.022626in}{1.424494in}}%
\pgfpathlineto{\pgfqpoint{1.023488in}{1.425334in}}%
\pgfpathlineto{\pgfqpoint{1.026405in}{1.426384in}}%
\pgfpathlineto{\pgfqpoint{1.027516in}{1.427224in}}%
\pgfpathlineto{\pgfqpoint{1.029787in}{1.428274in}}%
\pgfpathlineto{\pgfqpoint{1.030831in}{1.429184in}}%
\pgfpathlineto{\pgfqpoint{1.032887in}{1.430234in}}%
\pgfpathlineto{\pgfqpoint{1.033882in}{1.431074in}}%
\pgfpathlineto{\pgfqpoint{1.036667in}{1.432124in}}%
\pgfpathlineto{\pgfqpoint{1.037711in}{1.432894in}}%
\pgfpathlineto{\pgfqpoint{1.041872in}{1.433944in}}%
\pgfpathlineto{\pgfqpoint{1.042834in}{1.434504in}}%
\pgfpathlineto{\pgfqpoint{1.046530in}{1.435554in}}%
\pgfpathlineto{\pgfqpoint{1.047575in}{1.436044in}}%
\pgfpathlineto{\pgfqpoint{1.049514in}{1.437094in}}%
\pgfpathlineto{\pgfqpoint{1.050625in}{1.437794in}}%
\pgfpathlineto{\pgfqpoint{1.053095in}{1.438844in}}%
\pgfpathlineto{\pgfqpoint{1.054090in}{1.439544in}}%
\pgfpathlineto{\pgfqpoint{1.056692in}{1.440594in}}%
\pgfpathlineto{\pgfqpoint{1.057704in}{1.441364in}}%
\pgfpathlineto{\pgfqpoint{1.061301in}{1.442414in}}%
\pgfpathlineto{\pgfqpoint{1.062329in}{1.443044in}}%
\pgfpathlineto{\pgfqpoint{1.065893in}{1.444094in}}%
\pgfpathlineto{\pgfqpoint{1.066987in}{1.444584in}}%
\pgfpathlineto{\pgfqpoint{1.068976in}{1.445634in}}%
\pgfpathlineto{\pgfqpoint{1.070037in}{1.446544in}}%
\pgfpathlineto{\pgfqpoint{1.072491in}{1.447594in}}%
\pgfpathlineto{\pgfqpoint{1.073369in}{1.448224in}}%
\pgfpathlineto{\pgfqpoint{1.075840in}{1.449274in}}%
\pgfpathlineto{\pgfqpoint{1.076652in}{1.449694in}}%
\pgfpathlineto{\pgfqpoint{1.079006in}{1.450744in}}%
\pgfpathlineto{\pgfqpoint{1.079967in}{1.451234in}}%
\pgfpathlineto{\pgfqpoint{1.083001in}{1.452214in}}%
\pgfpathlineto{\pgfqpoint{1.084112in}{1.452914in}}%
\pgfpathlineto{\pgfqpoint{1.087726in}{1.453894in}}%
\pgfpathlineto{\pgfqpoint{1.088438in}{1.454384in}}%
\pgfpathlineto{\pgfqpoint{1.092119in}{1.455434in}}%
\pgfpathlineto{\pgfqpoint{1.093064in}{1.456064in}}%
\pgfpathlineto{\pgfqpoint{1.096346in}{1.457114in}}%
\pgfpathlineto{\pgfqpoint{1.097274in}{1.457674in}}%
\pgfpathlineto{\pgfqpoint{1.100092in}{1.458724in}}%
\pgfpathlineto{\pgfqpoint{1.100689in}{1.459004in}}%
\pgfpathlineto{\pgfqpoint{1.104038in}{1.460054in}}%
\pgfpathlineto{\pgfqpoint{1.105149in}{1.460404in}}%
\pgfpathlineto{\pgfqpoint{1.109989in}{1.461454in}}%
\pgfpathlineto{\pgfqpoint{1.111083in}{1.462154in}}%
\pgfpathlineto{\pgfqpoint{1.114813in}{1.463204in}}%
\pgfpathlineto{\pgfqpoint{1.115858in}{1.463624in}}%
\pgfpathlineto{\pgfqpoint{1.119389in}{1.464674in}}%
\pgfpathlineto{\pgfqpoint{1.120433in}{1.465094in}}%
\pgfpathlineto{\pgfqpoint{1.122754in}{1.466144in}}%
\pgfpathlineto{\pgfqpoint{1.123732in}{1.466494in}}%
\pgfpathlineto{\pgfqpoint{1.127910in}{1.467544in}}%
\pgfpathlineto{\pgfqpoint{1.128556in}{1.468034in}}%
\pgfpathlineto{\pgfqpoint{1.131010in}{1.469084in}}%
\pgfpathlineto{\pgfqpoint{1.131822in}{1.469924in}}%
\pgfpathlineto{\pgfqpoint{1.135187in}{1.470974in}}%
\pgfpathlineto{\pgfqpoint{1.135204in}{1.471184in}}%
\pgfpathlineto{\pgfqpoint{1.140956in}{1.472234in}}%
\pgfpathlineto{\pgfqpoint{1.141735in}{1.472584in}}%
\pgfpathlineto{\pgfqpoint{1.145598in}{1.473634in}}%
\pgfpathlineto{\pgfqpoint{1.146675in}{1.474194in}}%
\pgfpathlineto{\pgfqpoint{1.150803in}{1.475244in}}%
\pgfpathlineto{\pgfqpoint{1.151516in}{1.475874in}}%
\pgfpathlineto{\pgfqpoint{1.156058in}{1.476924in}}%
\pgfpathlineto{\pgfqpoint{1.157152in}{1.477414in}}%
\pgfpathlineto{\pgfqpoint{1.162407in}{1.478464in}}%
\pgfpathlineto{\pgfqpoint{1.163452in}{1.479094in}}%
\pgfpathlineto{\pgfqpoint{1.168060in}{1.480144in}}%
\pgfpathlineto{\pgfqpoint{1.168873in}{1.480494in}}%
\pgfpathlineto{\pgfqpoint{1.171923in}{1.481474in}}%
\pgfpathlineto{\pgfqpoint{1.172835in}{1.481754in}}%
\pgfpathlineto{\pgfqpoint{1.177161in}{1.482804in}}%
\pgfpathlineto{\pgfqpoint{1.178156in}{1.483294in}}%
\pgfpathlineto{\pgfqpoint{1.181223in}{1.484344in}}%
\pgfpathlineto{\pgfqpoint{1.181273in}{1.484554in}}%
\pgfpathlineto{\pgfqpoint{1.188633in}{1.485604in}}%
\pgfpathlineto{\pgfqpoint{1.189512in}{1.485884in}}%
\pgfpathlineto{\pgfqpoint{1.193341in}{1.486934in}}%
\pgfpathlineto{\pgfqpoint{1.194203in}{1.487354in}}%
\pgfpathlineto{\pgfqpoint{1.198928in}{1.488404in}}%
\pgfpathlineto{\pgfqpoint{1.199823in}{1.488964in}}%
\pgfpathlineto{\pgfqpoint{1.201895in}{1.489944in}}%
\pgfpathlineto{\pgfqpoint{1.202343in}{1.490224in}}%
\pgfpathlineto{\pgfqpoint{1.205343in}{1.491274in}}%
\pgfpathlineto{\pgfqpoint{1.206404in}{1.491554in}}%
\pgfpathlineto{\pgfqpoint{1.209670in}{1.492604in}}%
\pgfpathlineto{\pgfqpoint{1.210748in}{1.493024in}}%
\pgfpathlineto{\pgfqpoint{1.215887in}{1.494074in}}%
\pgfpathlineto{\pgfqpoint{1.216865in}{1.494214in}}%
\pgfpathlineto{\pgfqpoint{1.223860in}{1.495264in}}%
\pgfpathlineto{\pgfqpoint{1.224739in}{1.495684in}}%
\pgfpathlineto{\pgfqpoint{1.229994in}{1.496734in}}%
\pgfpathlineto{\pgfqpoint{1.230823in}{1.497014in}}%
\pgfpathlineto{\pgfqpoint{1.234354in}{1.498064in}}%
\pgfpathlineto{\pgfqpoint{1.235117in}{1.498344in}}%
\pgfpathlineto{\pgfqpoint{1.240256in}{1.499394in}}%
\pgfpathlineto{\pgfqpoint{1.241151in}{1.499674in}}%
\pgfpathlineto{\pgfqpoint{1.245610in}{1.500724in}}%
\pgfpathlineto{\pgfqpoint{1.246721in}{1.501144in}}%
\pgfpathlineto{\pgfqpoint{1.248942in}{1.502194in}}%
\pgfpathlineto{\pgfqpoint{1.250053in}{1.502754in}}%
\pgfpathlineto{\pgfqpoint{1.254429in}{1.503804in}}%
\pgfpathlineto{\pgfqpoint{1.254728in}{1.504084in}}%
\pgfpathlineto{\pgfqpoint{1.260845in}{1.505134in}}%
\pgfpathlineto{\pgfqpoint{1.261723in}{1.505414in}}%
\pgfpathlineto{\pgfqpoint{1.266697in}{1.506464in}}%
\pgfpathlineto{\pgfqpoint{1.267807in}{1.506744in}}%
\pgfpathlineto{\pgfqpoint{1.271338in}{1.507724in}}%
\pgfpathlineto{\pgfqpoint{1.272200in}{1.508214in}}%
\pgfpathlineto{\pgfqpoint{1.277936in}{1.509264in}}%
\pgfpathlineto{\pgfqpoint{1.278036in}{1.509404in}}%
\pgfpathlineto{\pgfqpoint{1.284203in}{1.510454in}}%
\pgfpathlineto{\pgfqpoint{1.285131in}{1.510804in}}%
\pgfpathlineto{\pgfqpoint{1.289657in}{1.511854in}}%
\pgfpathlineto{\pgfqpoint{1.290585in}{1.512274in}}%
\pgfpathlineto{\pgfqpoint{1.295940in}{1.513324in}}%
\pgfpathlineto{\pgfqpoint{1.296354in}{1.513534in}}%
\pgfpathlineto{\pgfqpoint{1.299703in}{1.514584in}}%
\pgfpathlineto{\pgfqpoint{1.300664in}{1.514934in}}%
\pgfpathlineto{\pgfqpoint{1.305389in}{1.515984in}}%
\pgfpathlineto{\pgfqpoint{1.306118in}{1.516334in}}%
\pgfpathlineto{\pgfqpoint{1.311290in}{1.517384in}}%
\pgfpathlineto{\pgfqpoint{1.312318in}{1.517734in}}%
\pgfpathlineto{\pgfqpoint{1.318104in}{1.518784in}}%
\pgfpathlineto{\pgfqpoint{1.318899in}{1.519274in}}%
\pgfpathlineto{\pgfqpoint{1.323873in}{1.520324in}}%
\pgfpathlineto{\pgfqpoint{1.324950in}{1.520674in}}%
\pgfpathlineto{\pgfqpoint{1.329575in}{1.521654in}}%
\pgfpathlineto{\pgfqpoint{1.330305in}{1.521864in}}%
\pgfpathlineto{\pgfqpoint{1.336521in}{1.522844in}}%
\pgfpathlineto{\pgfqpoint{1.336836in}{1.523194in}}%
\pgfpathlineto{\pgfqpoint{1.344180in}{1.524244in}}%
\pgfpathlineto{\pgfqpoint{1.345191in}{1.524524in}}%
\pgfpathlineto{\pgfqpoint{1.351905in}{1.525574in}}%
\pgfpathlineto{\pgfqpoint{1.352602in}{1.525994in}}%
\pgfpathlineto{\pgfqpoint{1.359050in}{1.527044in}}%
\pgfpathlineto{\pgfqpoint{1.359962in}{1.527394in}}%
\pgfpathlineto{\pgfqpoint{1.365897in}{1.528444in}}%
\pgfpathlineto{\pgfqpoint{1.366991in}{1.528654in}}%
\pgfpathlineto{\pgfqpoint{1.378463in}{1.529704in}}%
\pgfpathlineto{\pgfqpoint{1.379092in}{1.530054in}}%
\pgfpathlineto{\pgfqpoint{1.386221in}{1.531104in}}%
\pgfpathlineto{\pgfqpoint{1.387298in}{1.531314in}}%
\pgfpathlineto{\pgfqpoint{1.393366in}{1.532364in}}%
\pgfpathlineto{\pgfqpoint{1.393880in}{1.532714in}}%
\pgfpathlineto{\pgfqpoint{1.403760in}{1.533764in}}%
\pgfpathlineto{\pgfqpoint{1.404340in}{1.533904in}}%
\pgfpathlineto{\pgfqpoint{1.412529in}{1.534954in}}%
\pgfpathlineto{\pgfqpoint{1.412994in}{1.535164in}}%
\pgfpathlineto{\pgfqpoint{1.422277in}{1.536214in}}%
\pgfpathlineto{\pgfqpoint{1.422807in}{1.536354in}}%
\pgfpathlineto{\pgfqpoint{1.428460in}{1.537404in}}%
\pgfpathlineto{\pgfqpoint{1.429157in}{1.537614in}}%
\pgfpathlineto{\pgfqpoint{1.435340in}{1.538664in}}%
\pgfpathlineto{\pgfqpoint{1.436351in}{1.539014in}}%
\pgfpathlineto{\pgfqpoint{1.443861in}{1.540064in}}%
\pgfpathlineto{\pgfqpoint{1.443877in}{1.540274in}}%
\pgfpathlineto{\pgfqpoint{1.450940in}{1.541324in}}%
\pgfpathlineto{\pgfqpoint{1.451006in}{1.541464in}}%
\pgfpathlineto{\pgfqpoint{1.462776in}{1.542514in}}%
\pgfpathlineto{\pgfqpoint{1.463439in}{1.542724in}}%
\pgfpathlineto{\pgfqpoint{1.473817in}{1.543774in}}%
\pgfpathlineto{\pgfqpoint{1.474347in}{1.544124in}}%
\pgfpathlineto{\pgfqpoint{1.482304in}{1.545174in}}%
\pgfpathlineto{\pgfqpoint{1.482669in}{1.545314in}}%
\pgfpathlineto{\pgfqpoint{1.493892in}{1.546364in}}%
\pgfpathlineto{\pgfqpoint{1.494257in}{1.546504in}}%
\pgfpathlineto{\pgfqpoint{1.503756in}{1.547554in}}%
\pgfpathlineto{\pgfqpoint{1.503921in}{1.547834in}}%
\pgfpathlineto{\pgfqpoint{1.514382in}{1.548884in}}%
\pgfpathlineto{\pgfqpoint{1.515227in}{1.549234in}}%
\pgfpathlineto{\pgfqpoint{1.524046in}{1.550284in}}%
\pgfpathlineto{\pgfqpoint{1.524378in}{1.550564in}}%
\pgfpathlineto{\pgfqpoint{1.535187in}{1.551614in}}%
\pgfpathlineto{\pgfqpoint{1.535817in}{1.551754in}}%
\pgfpathlineto{\pgfqpoint{1.546012in}{1.552804in}}%
\pgfpathlineto{\pgfqpoint{1.547122in}{1.553154in}}%
\pgfpathlineto{\pgfqpoint{1.557450in}{1.554204in}}%
\pgfpathlineto{\pgfqpoint{1.557450in}{1.554274in}}%
\pgfpathlineto{\pgfqpoint{1.571690in}{1.555324in}}%
\pgfpathlineto{\pgfqpoint{1.571790in}{1.555534in}}%
\pgfpathlineto{\pgfqpoint{1.581405in}{1.556584in}}%
\pgfpathlineto{\pgfqpoint{1.581653in}{1.556794in}}%
\pgfpathlineto{\pgfqpoint{1.595396in}{1.557844in}}%
\pgfpathlineto{\pgfqpoint{1.595844in}{1.558054in}}%
\pgfpathlineto{\pgfqpoint{1.606503in}{1.559104in}}%
\pgfpathlineto{\pgfqpoint{1.607514in}{1.559384in}}%
\pgfpathlineto{\pgfqpoint{1.617610in}{1.560434in}}%
\pgfpathlineto{\pgfqpoint{1.618721in}{1.560644in}}%
\pgfpathlineto{\pgfqpoint{1.630889in}{1.561694in}}%
\pgfpathlineto{\pgfqpoint{1.631883in}{1.561834in}}%
\pgfpathlineto{\pgfqpoint{1.652605in}{1.562884in}}%
\pgfpathlineto{\pgfqpoint{1.653600in}{1.563024in}}%
\pgfpathlineto{\pgfqpoint{1.663812in}{1.564074in}}%
\pgfpathlineto{\pgfqpoint{1.663861in}{1.564214in}}%
\pgfpathlineto{\pgfqpoint{1.672283in}{1.565264in}}%
\pgfpathlineto{\pgfqpoint{1.672382in}{1.565404in}}%
\pgfpathlineto{\pgfqpoint{1.683406in}{1.566454in}}%
\pgfpathlineto{\pgfqpoint{1.684467in}{1.566664in}}%
\pgfpathlineto{\pgfqpoint{1.696586in}{1.567714in}}%
\pgfpathlineto{\pgfqpoint{1.696834in}{1.567854in}}%
\pgfpathlineto{\pgfqpoint{1.710212in}{1.568904in}}%
\pgfpathlineto{\pgfqpoint{1.710909in}{1.569114in}}%
\pgfpathlineto{\pgfqpoint{1.723706in}{1.570164in}}%
\pgfpathlineto{\pgfqpoint{1.724718in}{1.570304in}}%
\pgfpathlineto{\pgfqpoint{1.739090in}{1.571354in}}%
\pgfpathlineto{\pgfqpoint{1.739422in}{1.571494in}}%
\pgfpathlineto{\pgfqpoint{1.756828in}{1.572544in}}%
\pgfpathlineto{\pgfqpoint{1.757591in}{1.572754in}}%
\pgfpathlineto{\pgfqpoint{1.769825in}{1.573804in}}%
\pgfpathlineto{\pgfqpoint{1.770339in}{1.573944in}}%
\pgfpathlineto{\pgfqpoint{1.782242in}{1.574994in}}%
\pgfpathlineto{\pgfqpoint{1.782374in}{1.575134in}}%
\pgfpathlineto{\pgfqpoint{1.795288in}{1.576184in}}%
\pgfpathlineto{\pgfqpoint{1.795868in}{1.576394in}}%
\pgfpathlineto{\pgfqpoint{1.809910in}{1.577444in}}%
\pgfpathlineto{\pgfqpoint{1.810987in}{1.577724in}}%
\pgfpathlineto{\pgfqpoint{1.828360in}{1.578774in}}%
\pgfpathlineto{\pgfqpoint{1.829057in}{1.578984in}}%
\pgfpathlineto{\pgfqpoint{1.842236in}{1.580034in}}%
\pgfpathlineto{\pgfqpoint{1.842368in}{1.580174in}}%
\pgfpathlineto{\pgfqpoint{1.859410in}{1.581224in}}%
\pgfpathlineto{\pgfqpoint{1.859891in}{1.581364in}}%
\pgfpathlineto{\pgfqpoint{1.873683in}{1.582414in}}%
\pgfpathlineto{\pgfqpoint{1.873683in}{1.582484in}}%
\pgfpathlineto{\pgfqpoint{1.894190in}{1.583534in}}%
\pgfpathlineto{\pgfqpoint{1.894637in}{1.583674in}}%
\pgfpathlineto{\pgfqpoint{1.915542in}{1.584724in}}%
\pgfpathlineto{\pgfqpoint{1.916155in}{1.584864in}}%
\pgfpathlineto{\pgfqpoint{1.929334in}{1.585914in}}%
\pgfpathlineto{\pgfqpoint{1.930345in}{1.586054in}}%
\pgfpathlineto{\pgfqpoint{1.945680in}{1.587104in}}%
\pgfpathlineto{\pgfqpoint{1.945680in}{1.587174in}}%
\pgfpathlineto{\pgfqpoint{1.958245in}{1.588224in}}%
\pgfpathlineto{\pgfqpoint{1.959074in}{1.588434in}}%
\pgfpathlineto{\pgfqpoint{1.971541in}{1.589484in}}%
\pgfpathlineto{\pgfqpoint{1.972005in}{1.589624in}}%
\pgfpathlineto{\pgfqpoint{1.980807in}{1.590674in}}%
\pgfpathlineto{\pgfqpoint{1.981636in}{1.590954in}}%
\pgfpathlineto{\pgfqpoint{1.988947in}{1.592004in}}%
\pgfpathlineto{\pgfqpoint{1.988947in}{1.592074in}}%
\pgfpathlineto{\pgfqpoint{1.998313in}{1.593124in}}%
\pgfpathlineto{\pgfqpoint{1.999391in}{1.593334in}}%
\pgfpathlineto{\pgfqpoint{2.006536in}{1.594384in}}%
\pgfpathlineto{\pgfqpoint{2.007414in}{1.594524in}}%
\pgfpathlineto{\pgfqpoint{2.013349in}{1.595574in}}%
\pgfpathlineto{\pgfqpoint{2.014195in}{1.595854in}}%
\pgfpathlineto{\pgfqpoint{2.021936in}{1.596904in}}%
\pgfpathlineto{\pgfqpoint{2.022318in}{1.597044in}}%
\pgfpathlineto{\pgfqpoint{2.028136in}{1.598094in}}%
\pgfpathlineto{\pgfqpoint{2.029230in}{1.598584in}}%
\pgfpathlineto{\pgfqpoint{2.032015in}{1.599634in}}%
\pgfpathlineto{\pgfqpoint{2.033126in}{1.601944in}}%
\pgfpathlineto{\pgfqpoint{2.033126in}{1.601944in}}%
\pgfusepath{stroke}%
\end{pgfscope}%
\begin{pgfscope}%
\pgfsetrectcap%
\pgfsetmiterjoin%
\pgfsetlinewidth{0.803000pt}%
\definecolor{currentstroke}{rgb}{0.000000,0.000000,0.000000}%
\pgfsetstrokecolor{currentstroke}%
\pgfsetdash{}{0pt}%
\pgfpathmoveto{\pgfqpoint{0.553581in}{0.499444in}}%
\pgfpathlineto{\pgfqpoint{0.553581in}{1.654444in}}%
\pgfusepath{stroke}%
\end{pgfscope}%
\begin{pgfscope}%
\pgfsetrectcap%
\pgfsetmiterjoin%
\pgfsetlinewidth{0.803000pt}%
\definecolor{currentstroke}{rgb}{0.000000,0.000000,0.000000}%
\pgfsetstrokecolor{currentstroke}%
\pgfsetdash{}{0pt}%
\pgfpathmoveto{\pgfqpoint{2.103581in}{0.499444in}}%
\pgfpathlineto{\pgfqpoint{2.103581in}{1.654444in}}%
\pgfusepath{stroke}%
\end{pgfscope}%
\begin{pgfscope}%
\pgfsetrectcap%
\pgfsetmiterjoin%
\pgfsetlinewidth{0.803000pt}%
\definecolor{currentstroke}{rgb}{0.000000,0.000000,0.000000}%
\pgfsetstrokecolor{currentstroke}%
\pgfsetdash{}{0pt}%
\pgfpathmoveto{\pgfqpoint{0.553581in}{0.499444in}}%
\pgfpathlineto{\pgfqpoint{2.103581in}{0.499444in}}%
\pgfusepath{stroke}%
\end{pgfscope}%
\begin{pgfscope}%
\pgfsetrectcap%
\pgfsetmiterjoin%
\pgfsetlinewidth{0.803000pt}%
\definecolor{currentstroke}{rgb}{0.000000,0.000000,0.000000}%
\pgfsetstrokecolor{currentstroke}%
\pgfsetdash{}{0pt}%
\pgfpathmoveto{\pgfqpoint{0.553581in}{1.654444in}}%
\pgfpathlineto{\pgfqpoint{2.103581in}{1.654444in}}%
\pgfusepath{stroke}%
\end{pgfscope}%
\begin{pgfscope}%
\pgfsetbuttcap%
\pgfsetmiterjoin%
\definecolor{currentfill}{rgb}{1.000000,1.000000,1.000000}%
\pgfsetfillcolor{currentfill}%
\pgfsetfillopacity{0.800000}%
\pgfsetlinewidth{1.003750pt}%
\definecolor{currentstroke}{rgb}{0.800000,0.800000,0.800000}%
\pgfsetstrokecolor{currentstroke}%
\pgfsetstrokeopacity{0.800000}%
\pgfsetdash{}{0pt}%
\pgfpathmoveto{\pgfqpoint{0.832747in}{0.568889in}}%
\pgfpathlineto{\pgfqpoint{2.006358in}{0.568889in}}%
\pgfpathquadraticcurveto{\pgfqpoint{2.034136in}{0.568889in}}{\pgfqpoint{2.034136in}{0.596666in}}%
\pgfpathlineto{\pgfqpoint{2.034136in}{0.776388in}}%
\pgfpathquadraticcurveto{\pgfqpoint{2.034136in}{0.804166in}}{\pgfqpoint{2.006358in}{0.804166in}}%
\pgfpathlineto{\pgfqpoint{0.832747in}{0.804166in}}%
\pgfpathquadraticcurveto{\pgfqpoint{0.804970in}{0.804166in}}{\pgfqpoint{0.804970in}{0.776388in}}%
\pgfpathlineto{\pgfqpoint{0.804970in}{0.596666in}}%
\pgfpathquadraticcurveto{\pgfqpoint{0.804970in}{0.568889in}}{\pgfqpoint{0.832747in}{0.568889in}}%
\pgfpathlineto{\pgfqpoint{0.832747in}{0.568889in}}%
\pgfpathclose%
\pgfusepath{stroke,fill}%
\end{pgfscope}%
\begin{pgfscope}%
\pgfsetrectcap%
\pgfsetroundjoin%
\pgfsetlinewidth{1.505625pt}%
\definecolor{currentstroke}{rgb}{0.000000,0.000000,0.000000}%
\pgfsetstrokecolor{currentstroke}%
\pgfsetdash{}{0pt}%
\pgfpathmoveto{\pgfqpoint{0.860525in}{0.700000in}}%
\pgfpathlineto{\pgfqpoint{0.999414in}{0.700000in}}%
\pgfpathlineto{\pgfqpoint{1.138303in}{0.700000in}}%
\pgfusepath{stroke}%
\end{pgfscope}%
\begin{pgfscope}%
\definecolor{textcolor}{rgb}{0.000000,0.000000,0.000000}%
\pgfsetstrokecolor{textcolor}%
\pgfsetfillcolor{textcolor}%
\pgftext[x=1.249414in,y=0.651388in,left,base]{\color{textcolor}\rmfamily\fontsize{10.000000}{12.000000}\selectfont AUC=0.840}%
\end{pgfscope}%
\end{pgfpicture}%
\makeatother%
\endgroup%

	
&
	\vskip 0pt
	\begin{tabular}{cc|c|c|}
	&\multicolumn{1}{c}{}& \multicolumn{2}{c}{Prediction} \\[0.4em]
	&\multicolumn{1}{c}{} & \multicolumn{1}{c}{N} & \multicolumn{1}{c}{P} \cr\cline{3-4}
	\multirow{2}{*}{\rotatebox[origin=c]{90}{Actual}}&N &
0\% & 85\%
	\vrule width 0pt height 10pt depth 2pt \cr\cline{3-4}
	&P & 
0\% & 15\%
	\vrule width 0pt height 10pt depth 2pt \cr\cline{3-4}
	\end{tabular}
% 0.839808121
%	0	84.993	0	15.007	
%	0.150068774	0.150068774	1	0.150068774

	\hfil\begin{tabular}{ll}
	\cr
	0.15 & Accuracy\cr
	0.15 & Precision \cr
	1 & Recall \cr
	0.15 & F1 \cr
	0.840 & AUC \cr
\end{tabular}
\cr
\end{tabular}
} % End parbox

\

If a model gives results like the ``Ideal Model Tending Left,'' and we directly apply its recommendations to never immediately dispatch an ambulance, then we wasted our time building a recommendation system.  The model output can still be useful, however, if we approach the data in one of three ways.  

\

{\bf Shift the $p$ Values to the Right}

Here we have shifted all of the $p$ values so that the average of the median $p$ value for the negative class and the median $p$ value for the positive class is at $0.5$.

\

\verb|Ideal_Left_Shifted|

%%%
\parbox{\linewidth}{
\noindent\begin{tabular}{@{\hspace{-6pt}}p{2.3in} @{\hspace{-6pt}}p{2.0in} p{1.8in}}
	\vskip 0pt
	\qquad \qquad Raw Model Output
	
	%% Creator: Matplotlib, PGF backend
%%
%% To include the figure in your LaTeX document, write
%%   \input{<filename>.pgf}
%%
%% Make sure the required packages are loaded in your preamble
%%   \usepackage{pgf}
%%
%% Also ensure that all the required font packages are loaded; for instance,
%% the lmodern package is sometimes necessary when using math font.
%%   \usepackage{lmodern}
%%
%% Figures using additional raster images can only be included by \input if
%% they are in the same directory as the main LaTeX file. For loading figures
%% from other directories you can use the `import` package
%%   \usepackage{import}
%%
%% and then include the figures with
%%   \import{<path to file>}{<filename>.pgf}
%%
%% Matplotlib used the following preamble
%%   
%%   \usepackage{fontspec}
%%   \makeatletter\@ifpackageloaded{underscore}{}{\usepackage[strings]{underscore}}\makeatother
%%
\begingroup%
\makeatletter%
\begin{pgfpicture}%
\pgfpathrectangle{\pgfpointorigin}{\pgfqpoint{2.218304in}{1.703778in}}%
\pgfusepath{use as bounding box, clip}%
\begin{pgfscope}%
\pgfsetbuttcap%
\pgfsetmiterjoin%
\definecolor{currentfill}{rgb}{1.000000,1.000000,1.000000}%
\pgfsetfillcolor{currentfill}%
\pgfsetlinewidth{0.000000pt}%
\definecolor{currentstroke}{rgb}{1.000000,1.000000,1.000000}%
\pgfsetstrokecolor{currentstroke}%
\pgfsetdash{}{0pt}%
\pgfpathmoveto{\pgfqpoint{0.000000in}{0.000000in}}%
\pgfpathlineto{\pgfqpoint{2.218304in}{0.000000in}}%
\pgfpathlineto{\pgfqpoint{2.218304in}{1.703777in}}%
\pgfpathlineto{\pgfqpoint{0.000000in}{1.703777in}}%
\pgfpathlineto{\pgfqpoint{0.000000in}{0.000000in}}%
\pgfpathclose%
\pgfusepath{fill}%
\end{pgfscope}%
\begin{pgfscope}%
\pgfsetbuttcap%
\pgfsetmiterjoin%
\definecolor{currentfill}{rgb}{1.000000,1.000000,1.000000}%
\pgfsetfillcolor{currentfill}%
\pgfsetlinewidth{0.000000pt}%
\definecolor{currentstroke}{rgb}{0.000000,0.000000,0.000000}%
\pgfsetstrokecolor{currentstroke}%
\pgfsetstrokeopacity{0.000000}%
\pgfsetdash{}{0pt}%
\pgfpathmoveto{\pgfqpoint{0.513970in}{0.498777in}}%
\pgfpathlineto{\pgfqpoint{2.063970in}{0.498777in}}%
\pgfpathlineto{\pgfqpoint{2.063970in}{1.653777in}}%
\pgfpathlineto{\pgfqpoint{0.513970in}{1.653777in}}%
\pgfpathlineto{\pgfqpoint{0.513970in}{0.498777in}}%
\pgfpathclose%
\pgfusepath{fill}%
\end{pgfscope}%
\begin{pgfscope}%
\pgfpathrectangle{\pgfqpoint{0.513970in}{0.498777in}}{\pgfqpoint{1.550000in}{1.155000in}}%
\pgfusepath{clip}%
\pgfsetbuttcap%
\pgfsetmiterjoin%
\pgfsetlinewidth{1.003750pt}%
\definecolor{currentstroke}{rgb}{0.000000,0.000000,0.000000}%
\pgfsetstrokecolor{currentstroke}%
\pgfsetdash{}{0pt}%
\pgfpathmoveto{\pgfqpoint{0.503970in}{0.498777in}}%
\pgfpathlineto{\pgfqpoint{0.551775in}{0.498777in}}%
\pgfpathlineto{\pgfqpoint{0.551775in}{0.498777in}}%
\pgfpathlineto{\pgfqpoint{0.503970in}{0.498777in}}%
\pgfusepath{stroke}%
\end{pgfscope}%
\begin{pgfscope}%
\pgfpathrectangle{\pgfqpoint{0.513970in}{0.498777in}}{\pgfqpoint{1.550000in}{1.155000in}}%
\pgfusepath{clip}%
\pgfsetbuttcap%
\pgfsetmiterjoin%
\pgfsetlinewidth{1.003750pt}%
\definecolor{currentstroke}{rgb}{0.000000,0.000000,0.000000}%
\pgfsetstrokecolor{currentstroke}%
\pgfsetdash{}{0pt}%
\pgfpathmoveto{\pgfqpoint{0.642507in}{0.498777in}}%
\pgfpathlineto{\pgfqpoint{0.702995in}{0.498777in}}%
\pgfpathlineto{\pgfqpoint{0.702995in}{0.498777in}}%
\pgfpathlineto{\pgfqpoint{0.642507in}{0.498777in}}%
\pgfpathlineto{\pgfqpoint{0.642507in}{0.498777in}}%
\pgfpathclose%
\pgfusepath{stroke}%
\end{pgfscope}%
\begin{pgfscope}%
\pgfpathrectangle{\pgfqpoint{0.513970in}{0.498777in}}{\pgfqpoint{1.550000in}{1.155000in}}%
\pgfusepath{clip}%
\pgfsetbuttcap%
\pgfsetmiterjoin%
\pgfsetlinewidth{1.003750pt}%
\definecolor{currentstroke}{rgb}{0.000000,0.000000,0.000000}%
\pgfsetstrokecolor{currentstroke}%
\pgfsetdash{}{0pt}%
\pgfpathmoveto{\pgfqpoint{0.793727in}{0.498777in}}%
\pgfpathlineto{\pgfqpoint{0.854214in}{0.498777in}}%
\pgfpathlineto{\pgfqpoint{0.854214in}{0.576988in}}%
\pgfpathlineto{\pgfqpoint{0.793727in}{0.576988in}}%
\pgfpathlineto{\pgfqpoint{0.793727in}{0.498777in}}%
\pgfpathclose%
\pgfusepath{stroke}%
\end{pgfscope}%
\begin{pgfscope}%
\pgfpathrectangle{\pgfqpoint{0.513970in}{0.498777in}}{\pgfqpoint{1.550000in}{1.155000in}}%
\pgfusepath{clip}%
\pgfsetbuttcap%
\pgfsetmiterjoin%
\pgfsetlinewidth{1.003750pt}%
\definecolor{currentstroke}{rgb}{0.000000,0.000000,0.000000}%
\pgfsetstrokecolor{currentstroke}%
\pgfsetdash{}{0pt}%
\pgfpathmoveto{\pgfqpoint{0.944946in}{0.498777in}}%
\pgfpathlineto{\pgfqpoint{1.005434in}{0.498777in}}%
\pgfpathlineto{\pgfqpoint{1.005434in}{1.598777in}}%
\pgfpathlineto{\pgfqpoint{0.944946in}{1.598777in}}%
\pgfpathlineto{\pgfqpoint{0.944946in}{0.498777in}}%
\pgfpathclose%
\pgfusepath{stroke}%
\end{pgfscope}%
\begin{pgfscope}%
\pgfpathrectangle{\pgfqpoint{0.513970in}{0.498777in}}{\pgfqpoint{1.550000in}{1.155000in}}%
\pgfusepath{clip}%
\pgfsetbuttcap%
\pgfsetmiterjoin%
\pgfsetlinewidth{1.003750pt}%
\definecolor{currentstroke}{rgb}{0.000000,0.000000,0.000000}%
\pgfsetstrokecolor{currentstroke}%
\pgfsetdash{}{0pt}%
\pgfpathmoveto{\pgfqpoint{1.096166in}{0.498777in}}%
\pgfpathlineto{\pgfqpoint{1.156653in}{0.498777in}}%
\pgfpathlineto{\pgfqpoint{1.156653in}{1.528039in}}%
\pgfpathlineto{\pgfqpoint{1.096166in}{1.528039in}}%
\pgfpathlineto{\pgfqpoint{1.096166in}{0.498777in}}%
\pgfpathclose%
\pgfusepath{stroke}%
\end{pgfscope}%
\begin{pgfscope}%
\pgfpathrectangle{\pgfqpoint{0.513970in}{0.498777in}}{\pgfqpoint{1.550000in}{1.155000in}}%
\pgfusepath{clip}%
\pgfsetbuttcap%
\pgfsetmiterjoin%
\pgfsetlinewidth{1.003750pt}%
\definecolor{currentstroke}{rgb}{0.000000,0.000000,0.000000}%
\pgfsetstrokecolor{currentstroke}%
\pgfsetdash{}{0pt}%
\pgfpathmoveto{\pgfqpoint{1.247385in}{0.498777in}}%
\pgfpathlineto{\pgfqpoint{1.307873in}{0.498777in}}%
\pgfpathlineto{\pgfqpoint{1.307873in}{0.934340in}}%
\pgfpathlineto{\pgfqpoint{1.247385in}{0.934340in}}%
\pgfpathlineto{\pgfqpoint{1.247385in}{0.498777in}}%
\pgfpathclose%
\pgfusepath{stroke}%
\end{pgfscope}%
\begin{pgfscope}%
\pgfpathrectangle{\pgfqpoint{0.513970in}{0.498777in}}{\pgfqpoint{1.550000in}{1.155000in}}%
\pgfusepath{clip}%
\pgfsetbuttcap%
\pgfsetmiterjoin%
\pgfsetlinewidth{1.003750pt}%
\definecolor{currentstroke}{rgb}{0.000000,0.000000,0.000000}%
\pgfsetstrokecolor{currentstroke}%
\pgfsetdash{}{0pt}%
\pgfpathmoveto{\pgfqpoint{1.398605in}{0.498777in}}%
\pgfpathlineto{\pgfqpoint{1.459092in}{0.498777in}}%
\pgfpathlineto{\pgfqpoint{1.459092in}{0.615629in}}%
\pgfpathlineto{\pgfqpoint{1.398605in}{0.615629in}}%
\pgfpathlineto{\pgfqpoint{1.398605in}{0.498777in}}%
\pgfpathclose%
\pgfusepath{stroke}%
\end{pgfscope}%
\begin{pgfscope}%
\pgfpathrectangle{\pgfqpoint{0.513970in}{0.498777in}}{\pgfqpoint{1.550000in}{1.155000in}}%
\pgfusepath{clip}%
\pgfsetbuttcap%
\pgfsetmiterjoin%
\pgfsetlinewidth{1.003750pt}%
\definecolor{currentstroke}{rgb}{0.000000,0.000000,0.000000}%
\pgfsetstrokecolor{currentstroke}%
\pgfsetdash{}{0pt}%
\pgfpathmoveto{\pgfqpoint{1.549824in}{0.498777in}}%
\pgfpathlineto{\pgfqpoint{1.610312in}{0.498777in}}%
\pgfpathlineto{\pgfqpoint{1.610312in}{0.510250in}}%
\pgfpathlineto{\pgfqpoint{1.549824in}{0.510250in}}%
\pgfpathlineto{\pgfqpoint{1.549824in}{0.498777in}}%
\pgfpathclose%
\pgfusepath{stroke}%
\end{pgfscope}%
\begin{pgfscope}%
\pgfpathrectangle{\pgfqpoint{0.513970in}{0.498777in}}{\pgfqpoint{1.550000in}{1.155000in}}%
\pgfusepath{clip}%
\pgfsetbuttcap%
\pgfsetmiterjoin%
\pgfsetlinewidth{1.003750pt}%
\definecolor{currentstroke}{rgb}{0.000000,0.000000,0.000000}%
\pgfsetstrokecolor{currentstroke}%
\pgfsetdash{}{0pt}%
\pgfpathmoveto{\pgfqpoint{1.701044in}{0.498777in}}%
\pgfpathlineto{\pgfqpoint{1.761531in}{0.498777in}}%
\pgfpathlineto{\pgfqpoint{1.761531in}{0.498777in}}%
\pgfpathlineto{\pgfqpoint{1.701044in}{0.498777in}}%
\pgfpathlineto{\pgfqpoint{1.701044in}{0.498777in}}%
\pgfpathclose%
\pgfusepath{stroke}%
\end{pgfscope}%
\begin{pgfscope}%
\pgfpathrectangle{\pgfqpoint{0.513970in}{0.498777in}}{\pgfqpoint{1.550000in}{1.155000in}}%
\pgfusepath{clip}%
\pgfsetbuttcap%
\pgfsetmiterjoin%
\pgfsetlinewidth{1.003750pt}%
\definecolor{currentstroke}{rgb}{0.000000,0.000000,0.000000}%
\pgfsetstrokecolor{currentstroke}%
\pgfsetdash{}{0pt}%
\pgfpathmoveto{\pgfqpoint{1.852263in}{0.498777in}}%
\pgfpathlineto{\pgfqpoint{1.912751in}{0.498777in}}%
\pgfpathlineto{\pgfqpoint{1.912751in}{0.498777in}}%
\pgfpathlineto{\pgfqpoint{1.852263in}{0.498777in}}%
\pgfpathlineto{\pgfqpoint{1.852263in}{0.498777in}}%
\pgfpathclose%
\pgfusepath{stroke}%
\end{pgfscope}%
\begin{pgfscope}%
\pgfpathrectangle{\pgfqpoint{0.513970in}{0.498777in}}{\pgfqpoint{1.550000in}{1.155000in}}%
\pgfusepath{clip}%
\pgfsetbuttcap%
\pgfsetmiterjoin%
\definecolor{currentfill}{rgb}{0.000000,0.000000,0.000000}%
\pgfsetfillcolor{currentfill}%
\pgfsetlinewidth{0.000000pt}%
\definecolor{currentstroke}{rgb}{0.000000,0.000000,0.000000}%
\pgfsetstrokecolor{currentstroke}%
\pgfsetstrokeopacity{0.000000}%
\pgfsetdash{}{0pt}%
\pgfpathmoveto{\pgfqpoint{0.551775in}{0.498777in}}%
\pgfpathlineto{\pgfqpoint{0.612263in}{0.498777in}}%
\pgfpathlineto{\pgfqpoint{0.612263in}{0.498777in}}%
\pgfpathlineto{\pgfqpoint{0.551775in}{0.498777in}}%
\pgfpathlineto{\pgfqpoint{0.551775in}{0.498777in}}%
\pgfpathclose%
\pgfusepath{fill}%
\end{pgfscope}%
\begin{pgfscope}%
\pgfpathrectangle{\pgfqpoint{0.513970in}{0.498777in}}{\pgfqpoint{1.550000in}{1.155000in}}%
\pgfusepath{clip}%
\pgfsetbuttcap%
\pgfsetmiterjoin%
\definecolor{currentfill}{rgb}{0.000000,0.000000,0.000000}%
\pgfsetfillcolor{currentfill}%
\pgfsetlinewidth{0.000000pt}%
\definecolor{currentstroke}{rgb}{0.000000,0.000000,0.000000}%
\pgfsetstrokecolor{currentstroke}%
\pgfsetstrokeopacity{0.000000}%
\pgfsetdash{}{0pt}%
\pgfpathmoveto{\pgfqpoint{0.702995in}{0.498777in}}%
\pgfpathlineto{\pgfqpoint{0.763483in}{0.498777in}}%
\pgfpathlineto{\pgfqpoint{0.763483in}{0.498777in}}%
\pgfpathlineto{\pgfqpoint{0.702995in}{0.498777in}}%
\pgfpathlineto{\pgfqpoint{0.702995in}{0.498777in}}%
\pgfpathclose%
\pgfusepath{fill}%
\end{pgfscope}%
\begin{pgfscope}%
\pgfpathrectangle{\pgfqpoint{0.513970in}{0.498777in}}{\pgfqpoint{1.550000in}{1.155000in}}%
\pgfusepath{clip}%
\pgfsetbuttcap%
\pgfsetmiterjoin%
\definecolor{currentfill}{rgb}{0.000000,0.000000,0.000000}%
\pgfsetfillcolor{currentfill}%
\pgfsetlinewidth{0.000000pt}%
\definecolor{currentstroke}{rgb}{0.000000,0.000000,0.000000}%
\pgfsetstrokecolor{currentstroke}%
\pgfsetstrokeopacity{0.000000}%
\pgfsetdash{}{0pt}%
\pgfpathmoveto{\pgfqpoint{0.854214in}{0.498777in}}%
\pgfpathlineto{\pgfqpoint{0.914702in}{0.498777in}}%
\pgfpathlineto{\pgfqpoint{0.914702in}{0.502690in}}%
\pgfpathlineto{\pgfqpoint{0.854214in}{0.502690in}}%
\pgfpathlineto{\pgfqpoint{0.854214in}{0.498777in}}%
\pgfpathclose%
\pgfusepath{fill}%
\end{pgfscope}%
\begin{pgfscope}%
\pgfpathrectangle{\pgfqpoint{0.513970in}{0.498777in}}{\pgfqpoint{1.550000in}{1.155000in}}%
\pgfusepath{clip}%
\pgfsetbuttcap%
\pgfsetmiterjoin%
\definecolor{currentfill}{rgb}{0.000000,0.000000,0.000000}%
\pgfsetfillcolor{currentfill}%
\pgfsetlinewidth{0.000000pt}%
\definecolor{currentstroke}{rgb}{0.000000,0.000000,0.000000}%
\pgfsetstrokecolor{currentstroke}%
\pgfsetstrokeopacity{0.000000}%
\pgfsetdash{}{0pt}%
\pgfpathmoveto{\pgfqpoint{1.005434in}{0.498777in}}%
\pgfpathlineto{\pgfqpoint{1.065922in}{0.498777in}}%
\pgfpathlineto{\pgfqpoint{1.065922in}{0.521873in}}%
\pgfpathlineto{\pgfqpoint{1.005434in}{0.521873in}}%
\pgfpathlineto{\pgfqpoint{1.005434in}{0.498777in}}%
\pgfpathclose%
\pgfusepath{fill}%
\end{pgfscope}%
\begin{pgfscope}%
\pgfpathrectangle{\pgfqpoint{0.513970in}{0.498777in}}{\pgfqpoint{1.550000in}{1.155000in}}%
\pgfusepath{clip}%
\pgfsetbuttcap%
\pgfsetmiterjoin%
\definecolor{currentfill}{rgb}{0.000000,0.000000,0.000000}%
\pgfsetfillcolor{currentfill}%
\pgfsetlinewidth{0.000000pt}%
\definecolor{currentstroke}{rgb}{0.000000,0.000000,0.000000}%
\pgfsetstrokecolor{currentstroke}%
\pgfsetstrokeopacity{0.000000}%
\pgfsetdash{}{0pt}%
\pgfpathmoveto{\pgfqpoint{1.156653in}{0.498777in}}%
\pgfpathlineto{\pgfqpoint{1.217141in}{0.498777in}}%
\pgfpathlineto{\pgfqpoint{1.217141in}{0.574934in}}%
\pgfpathlineto{\pgfqpoint{1.156653in}{0.574934in}}%
\pgfpathlineto{\pgfqpoint{1.156653in}{0.498777in}}%
\pgfpathclose%
\pgfusepath{fill}%
\end{pgfscope}%
\begin{pgfscope}%
\pgfpathrectangle{\pgfqpoint{0.513970in}{0.498777in}}{\pgfqpoint{1.550000in}{1.155000in}}%
\pgfusepath{clip}%
\pgfsetbuttcap%
\pgfsetmiterjoin%
\definecolor{currentfill}{rgb}{0.000000,0.000000,0.000000}%
\pgfsetfillcolor{currentfill}%
\pgfsetlinewidth{0.000000pt}%
\definecolor{currentstroke}{rgb}{0.000000,0.000000,0.000000}%
\pgfsetstrokecolor{currentstroke}%
\pgfsetstrokeopacity{0.000000}%
\pgfsetdash{}{0pt}%
\pgfpathmoveto{\pgfqpoint{1.307873in}{0.498777in}}%
\pgfpathlineto{\pgfqpoint{1.368361in}{0.498777in}}%
\pgfpathlineto{\pgfqpoint{1.368361in}{0.678409in}}%
\pgfpathlineto{\pgfqpoint{1.307873in}{0.678409in}}%
\pgfpathlineto{\pgfqpoint{1.307873in}{0.498777in}}%
\pgfpathclose%
\pgfusepath{fill}%
\end{pgfscope}%
\begin{pgfscope}%
\pgfpathrectangle{\pgfqpoint{0.513970in}{0.498777in}}{\pgfqpoint{1.550000in}{1.155000in}}%
\pgfusepath{clip}%
\pgfsetbuttcap%
\pgfsetmiterjoin%
\definecolor{currentfill}{rgb}{0.000000,0.000000,0.000000}%
\pgfsetfillcolor{currentfill}%
\pgfsetlinewidth{0.000000pt}%
\definecolor{currentstroke}{rgb}{0.000000,0.000000,0.000000}%
\pgfsetstrokecolor{currentstroke}%
\pgfsetstrokeopacity{0.000000}%
\pgfsetdash{}{0pt}%
\pgfpathmoveto{\pgfqpoint{1.459092in}{0.498777in}}%
\pgfpathlineto{\pgfqpoint{1.519580in}{0.498777in}}%
\pgfpathlineto{\pgfqpoint{1.519580in}{0.693767in}}%
\pgfpathlineto{\pgfqpoint{1.459092in}{0.693767in}}%
\pgfpathlineto{\pgfqpoint{1.459092in}{0.498777in}}%
\pgfpathclose%
\pgfusepath{fill}%
\end{pgfscope}%
\begin{pgfscope}%
\pgfpathrectangle{\pgfqpoint{0.513970in}{0.498777in}}{\pgfqpoint{1.550000in}{1.155000in}}%
\pgfusepath{clip}%
\pgfsetbuttcap%
\pgfsetmiterjoin%
\definecolor{currentfill}{rgb}{0.000000,0.000000,0.000000}%
\pgfsetfillcolor{currentfill}%
\pgfsetlinewidth{0.000000pt}%
\definecolor{currentstroke}{rgb}{0.000000,0.000000,0.000000}%
\pgfsetstrokecolor{currentstroke}%
\pgfsetstrokeopacity{0.000000}%
\pgfsetdash{}{0pt}%
\pgfpathmoveto{\pgfqpoint{1.610312in}{0.498777in}}%
\pgfpathlineto{\pgfqpoint{1.670800in}{0.498777in}}%
\pgfpathlineto{\pgfqpoint{1.670800in}{0.515014in}}%
\pgfpathlineto{\pgfqpoint{1.610312in}{0.515014in}}%
\pgfpathlineto{\pgfqpoint{1.610312in}{0.498777in}}%
\pgfpathclose%
\pgfusepath{fill}%
\end{pgfscope}%
\begin{pgfscope}%
\pgfpathrectangle{\pgfqpoint{0.513970in}{0.498777in}}{\pgfqpoint{1.550000in}{1.155000in}}%
\pgfusepath{clip}%
\pgfsetbuttcap%
\pgfsetmiterjoin%
\definecolor{currentfill}{rgb}{0.000000,0.000000,0.000000}%
\pgfsetfillcolor{currentfill}%
\pgfsetlinewidth{0.000000pt}%
\definecolor{currentstroke}{rgb}{0.000000,0.000000,0.000000}%
\pgfsetstrokecolor{currentstroke}%
\pgfsetstrokeopacity{0.000000}%
\pgfsetdash{}{0pt}%
\pgfpathmoveto{\pgfqpoint{1.761531in}{0.498777in}}%
\pgfpathlineto{\pgfqpoint{1.822019in}{0.498777in}}%
\pgfpathlineto{\pgfqpoint{1.822019in}{0.498777in}}%
\pgfpathlineto{\pgfqpoint{1.761531in}{0.498777in}}%
\pgfpathlineto{\pgfqpoint{1.761531in}{0.498777in}}%
\pgfpathclose%
\pgfusepath{fill}%
\end{pgfscope}%
\begin{pgfscope}%
\pgfpathrectangle{\pgfqpoint{0.513970in}{0.498777in}}{\pgfqpoint{1.550000in}{1.155000in}}%
\pgfusepath{clip}%
\pgfsetbuttcap%
\pgfsetmiterjoin%
\definecolor{currentfill}{rgb}{0.000000,0.000000,0.000000}%
\pgfsetfillcolor{currentfill}%
\pgfsetlinewidth{0.000000pt}%
\definecolor{currentstroke}{rgb}{0.000000,0.000000,0.000000}%
\pgfsetstrokecolor{currentstroke}%
\pgfsetstrokeopacity{0.000000}%
\pgfsetdash{}{0pt}%
\pgfpathmoveto{\pgfqpoint{1.912751in}{0.498777in}}%
\pgfpathlineto{\pgfqpoint{1.973239in}{0.498777in}}%
\pgfpathlineto{\pgfqpoint{1.973239in}{0.498777in}}%
\pgfpathlineto{\pgfqpoint{1.912751in}{0.498777in}}%
\pgfpathlineto{\pgfqpoint{1.912751in}{0.498777in}}%
\pgfpathclose%
\pgfusepath{fill}%
\end{pgfscope}%
\begin{pgfscope}%
\pgfsetbuttcap%
\pgfsetroundjoin%
\definecolor{currentfill}{rgb}{0.000000,0.000000,0.000000}%
\pgfsetfillcolor{currentfill}%
\pgfsetlinewidth{0.803000pt}%
\definecolor{currentstroke}{rgb}{0.000000,0.000000,0.000000}%
\pgfsetstrokecolor{currentstroke}%
\pgfsetdash{}{0pt}%
\pgfsys@defobject{currentmarker}{\pgfqpoint{0.000000in}{-0.048611in}}{\pgfqpoint{0.000000in}{0.000000in}}{%
\pgfpathmoveto{\pgfqpoint{0.000000in}{0.000000in}}%
\pgfpathlineto{\pgfqpoint{0.000000in}{-0.048611in}}%
\pgfusepath{stroke,fill}%
}%
\begin{pgfscope}%
\pgfsys@transformshift{0.551775in}{0.498777in}%
\pgfsys@useobject{currentmarker}{}%
\end{pgfscope}%
\end{pgfscope}%
\begin{pgfscope}%
\definecolor{textcolor}{rgb}{0.000000,0.000000,0.000000}%
\pgfsetstrokecolor{textcolor}%
\pgfsetfillcolor{textcolor}%
\pgftext[x=0.551775in,y=0.401555in,,top]{\color{textcolor}\rmfamily\fontsize{12.000000}{14.400000}\selectfont 0.0}%
\end{pgfscope}%
\begin{pgfscope}%
\pgfsetbuttcap%
\pgfsetroundjoin%
\definecolor{currentfill}{rgb}{0.000000,0.000000,0.000000}%
\pgfsetfillcolor{currentfill}%
\pgfsetlinewidth{0.803000pt}%
\definecolor{currentstroke}{rgb}{0.000000,0.000000,0.000000}%
\pgfsetstrokecolor{currentstroke}%
\pgfsetdash{}{0pt}%
\pgfsys@defobject{currentmarker}{\pgfqpoint{0.000000in}{-0.048611in}}{\pgfqpoint{0.000000in}{0.000000in}}{%
\pgfpathmoveto{\pgfqpoint{0.000000in}{0.000000in}}%
\pgfpathlineto{\pgfqpoint{0.000000in}{-0.048611in}}%
\pgfusepath{stroke,fill}%
}%
\begin{pgfscope}%
\pgfsys@transformshift{0.929824in}{0.498777in}%
\pgfsys@useobject{currentmarker}{}%
\end{pgfscope}%
\end{pgfscope}%
\begin{pgfscope}%
\definecolor{textcolor}{rgb}{0.000000,0.000000,0.000000}%
\pgfsetstrokecolor{textcolor}%
\pgfsetfillcolor{textcolor}%
\pgftext[x=0.929824in,y=0.401555in,,top]{\color{textcolor}\rmfamily\fontsize{12.000000}{14.400000}\selectfont 0.25}%
\end{pgfscope}%
\begin{pgfscope}%
\pgfsetbuttcap%
\pgfsetroundjoin%
\definecolor{currentfill}{rgb}{0.000000,0.000000,0.000000}%
\pgfsetfillcolor{currentfill}%
\pgfsetlinewidth{0.803000pt}%
\definecolor{currentstroke}{rgb}{0.000000,0.000000,0.000000}%
\pgfsetstrokecolor{currentstroke}%
\pgfsetdash{}{0pt}%
\pgfsys@defobject{currentmarker}{\pgfqpoint{0.000000in}{-0.048611in}}{\pgfqpoint{0.000000in}{0.000000in}}{%
\pgfpathmoveto{\pgfqpoint{0.000000in}{0.000000in}}%
\pgfpathlineto{\pgfqpoint{0.000000in}{-0.048611in}}%
\pgfusepath{stroke,fill}%
}%
\begin{pgfscope}%
\pgfsys@transformshift{1.307873in}{0.498777in}%
\pgfsys@useobject{currentmarker}{}%
\end{pgfscope}%
\end{pgfscope}%
\begin{pgfscope}%
\definecolor{textcolor}{rgb}{0.000000,0.000000,0.000000}%
\pgfsetstrokecolor{textcolor}%
\pgfsetfillcolor{textcolor}%
\pgftext[x=1.307873in,y=0.401555in,,top]{\color{textcolor}\rmfamily\fontsize{12.000000}{14.400000}\selectfont 0.5}%
\end{pgfscope}%
\begin{pgfscope}%
\pgfsetbuttcap%
\pgfsetroundjoin%
\definecolor{currentfill}{rgb}{0.000000,0.000000,0.000000}%
\pgfsetfillcolor{currentfill}%
\pgfsetlinewidth{0.803000pt}%
\definecolor{currentstroke}{rgb}{0.000000,0.000000,0.000000}%
\pgfsetstrokecolor{currentstroke}%
\pgfsetdash{}{0pt}%
\pgfsys@defobject{currentmarker}{\pgfqpoint{0.000000in}{-0.048611in}}{\pgfqpoint{0.000000in}{0.000000in}}{%
\pgfpathmoveto{\pgfqpoint{0.000000in}{0.000000in}}%
\pgfpathlineto{\pgfqpoint{0.000000in}{-0.048611in}}%
\pgfusepath{stroke,fill}%
}%
\begin{pgfscope}%
\pgfsys@transformshift{1.685922in}{0.498777in}%
\pgfsys@useobject{currentmarker}{}%
\end{pgfscope}%
\end{pgfscope}%
\begin{pgfscope}%
\definecolor{textcolor}{rgb}{0.000000,0.000000,0.000000}%
\pgfsetstrokecolor{textcolor}%
\pgfsetfillcolor{textcolor}%
\pgftext[x=1.685922in,y=0.401555in,,top]{\color{textcolor}\rmfamily\fontsize{12.000000}{14.400000}\selectfont 0.75}%
\end{pgfscope}%
\begin{pgfscope}%
\pgfsetbuttcap%
\pgfsetroundjoin%
\definecolor{currentfill}{rgb}{0.000000,0.000000,0.000000}%
\pgfsetfillcolor{currentfill}%
\pgfsetlinewidth{0.803000pt}%
\definecolor{currentstroke}{rgb}{0.000000,0.000000,0.000000}%
\pgfsetstrokecolor{currentstroke}%
\pgfsetdash{}{0pt}%
\pgfsys@defobject{currentmarker}{\pgfqpoint{0.000000in}{-0.048611in}}{\pgfqpoint{0.000000in}{0.000000in}}{%
\pgfpathmoveto{\pgfqpoint{0.000000in}{0.000000in}}%
\pgfpathlineto{\pgfqpoint{0.000000in}{-0.048611in}}%
\pgfusepath{stroke,fill}%
}%
\begin{pgfscope}%
\pgfsys@transformshift{2.063970in}{0.498777in}%
\pgfsys@useobject{currentmarker}{}%
\end{pgfscope}%
\end{pgfscope}%
\begin{pgfscope}%
\definecolor{textcolor}{rgb}{0.000000,0.000000,0.000000}%
\pgfsetstrokecolor{textcolor}%
\pgfsetfillcolor{textcolor}%
\pgftext[x=2.063970in,y=0.401555in,,top]{\color{textcolor}\rmfamily\fontsize{12.000000}{14.400000}\selectfont 1.0}%
\end{pgfscope}%
\begin{pgfscope}%
\definecolor{textcolor}{rgb}{0.000000,0.000000,0.000000}%
\pgfsetstrokecolor{textcolor}%
\pgfsetfillcolor{textcolor}%
\pgftext[x=1.288970in,y=0.198000in,,top]{\color{textcolor}\rmfamily\fontsize{12.000000}{14.400000}\selectfont \(\displaystyle p\)}%
\end{pgfscope}%
\begin{pgfscope}%
\pgfsetbuttcap%
\pgfsetroundjoin%
\definecolor{currentfill}{rgb}{0.000000,0.000000,0.000000}%
\pgfsetfillcolor{currentfill}%
\pgfsetlinewidth{0.803000pt}%
\definecolor{currentstroke}{rgb}{0.000000,0.000000,0.000000}%
\pgfsetstrokecolor{currentstroke}%
\pgfsetdash{}{0pt}%
\pgfsys@defobject{currentmarker}{\pgfqpoint{-0.048611in}{0.000000in}}{\pgfqpoint{-0.000000in}{0.000000in}}{%
\pgfpathmoveto{\pgfqpoint{-0.000000in}{0.000000in}}%
\pgfpathlineto{\pgfqpoint{-0.048611in}{0.000000in}}%
\pgfusepath{stroke,fill}%
}%
\begin{pgfscope}%
\pgfsys@transformshift{0.513970in}{0.498777in}%
\pgfsys@useobject{currentmarker}{}%
\end{pgfscope}%
\end{pgfscope}%
\begin{pgfscope}%
\definecolor{textcolor}{rgb}{0.000000,0.000000,0.000000}%
\pgfsetstrokecolor{textcolor}%
\pgfsetfillcolor{textcolor}%
\pgftext[x=0.335152in, y=0.440944in, left, base]{\color{textcolor}\rmfamily\fontsize{12.000000}{14.400000}\selectfont \(\displaystyle {0}\)}%
\end{pgfscope}%
\begin{pgfscope}%
\pgfsetbuttcap%
\pgfsetroundjoin%
\definecolor{currentfill}{rgb}{0.000000,0.000000,0.000000}%
\pgfsetfillcolor{currentfill}%
\pgfsetlinewidth{0.803000pt}%
\definecolor{currentstroke}{rgb}{0.000000,0.000000,0.000000}%
\pgfsetstrokecolor{currentstroke}%
\pgfsetdash{}{0pt}%
\pgfsys@defobject{currentmarker}{\pgfqpoint{-0.048611in}{0.000000in}}{\pgfqpoint{-0.000000in}{0.000000in}}{%
\pgfpathmoveto{\pgfqpoint{-0.000000in}{0.000000in}}%
\pgfpathlineto{\pgfqpoint{-0.048611in}{0.000000in}}%
\pgfusepath{stroke,fill}%
}%
\begin{pgfscope}%
\pgfsys@transformshift{0.513970in}{1.151854in}%
\pgfsys@useobject{currentmarker}{}%
\end{pgfscope}%
\end{pgfscope}%
\begin{pgfscope}%
\definecolor{textcolor}{rgb}{0.000000,0.000000,0.000000}%
\pgfsetstrokecolor{textcolor}%
\pgfsetfillcolor{textcolor}%
\pgftext[x=0.253555in, y=1.094020in, left, base]{\color{textcolor}\rmfamily\fontsize{12.000000}{14.400000}\selectfont \(\displaystyle {20}\)}%
\end{pgfscope}%
\begin{pgfscope}%
\definecolor{textcolor}{rgb}{0.000000,0.000000,0.000000}%
\pgfsetstrokecolor{textcolor}%
\pgfsetfillcolor{textcolor}%
\pgftext[x=0.198000in,y=1.076277in,,bottom,rotate=90.000000]{\color{textcolor}\rmfamily\fontsize{12.000000}{14.400000}\selectfont Percent of Dataset}%
\end{pgfscope}%
\begin{pgfscope}%
\pgfsetrectcap%
\pgfsetmiterjoin%
\pgfsetlinewidth{0.803000pt}%
\definecolor{currentstroke}{rgb}{0.000000,0.000000,0.000000}%
\pgfsetstrokecolor{currentstroke}%
\pgfsetdash{}{0pt}%
\pgfpathmoveto{\pgfqpoint{0.513970in}{0.498777in}}%
\pgfpathlineto{\pgfqpoint{0.513970in}{1.653777in}}%
\pgfusepath{stroke}%
\end{pgfscope}%
\begin{pgfscope}%
\pgfsetrectcap%
\pgfsetmiterjoin%
\pgfsetlinewidth{0.803000pt}%
\definecolor{currentstroke}{rgb}{0.000000,0.000000,0.000000}%
\pgfsetstrokecolor{currentstroke}%
\pgfsetdash{}{0pt}%
\pgfpathmoveto{\pgfqpoint{2.063970in}{0.498777in}}%
\pgfpathlineto{\pgfqpoint{2.063970in}{1.653777in}}%
\pgfusepath{stroke}%
\end{pgfscope}%
\begin{pgfscope}%
\pgfsetrectcap%
\pgfsetmiterjoin%
\pgfsetlinewidth{0.803000pt}%
\definecolor{currentstroke}{rgb}{0.000000,0.000000,0.000000}%
\pgfsetstrokecolor{currentstroke}%
\pgfsetdash{}{0pt}%
\pgfpathmoveto{\pgfqpoint{0.513970in}{0.498777in}}%
\pgfpathlineto{\pgfqpoint{2.063970in}{0.498777in}}%
\pgfusepath{stroke}%
\end{pgfscope}%
\begin{pgfscope}%
\pgfsetrectcap%
\pgfsetmiterjoin%
\pgfsetlinewidth{0.803000pt}%
\definecolor{currentstroke}{rgb}{0.000000,0.000000,0.000000}%
\pgfsetstrokecolor{currentstroke}%
\pgfsetdash{}{0pt}%
\pgfpathmoveto{\pgfqpoint{0.513970in}{1.653777in}}%
\pgfpathlineto{\pgfqpoint{2.063970in}{1.653777in}}%
\pgfusepath{stroke}%
\end{pgfscope}%
\begin{pgfscope}%
\pgfsetbuttcap%
\pgfsetmiterjoin%
\definecolor{currentfill}{rgb}{1.000000,1.000000,1.000000}%
\pgfsetfillcolor{currentfill}%
\pgfsetfillopacity{0.800000}%
\pgfsetlinewidth{1.003750pt}%
\definecolor{currentstroke}{rgb}{0.800000,0.800000,0.800000}%
\pgfsetstrokecolor{currentstroke}%
\pgfsetstrokeopacity{0.800000}%
\pgfsetdash{}{0pt}%
\pgfpathmoveto{\pgfqpoint{1.137470in}{1.053944in}}%
\pgfpathlineto{\pgfqpoint{1.947304in}{1.053944in}}%
\pgfpathquadraticcurveto{\pgfqpoint{1.980637in}{1.053944in}}{\pgfqpoint{1.980637in}{1.087278in}}%
\pgfpathlineto{\pgfqpoint{1.980637in}{1.537111in}}%
\pgfpathquadraticcurveto{\pgfqpoint{1.980637in}{1.570444in}}{\pgfqpoint{1.947304in}{1.570444in}}%
\pgfpathlineto{\pgfqpoint{1.137470in}{1.570444in}}%
\pgfpathquadraticcurveto{\pgfqpoint{1.104137in}{1.570444in}}{\pgfqpoint{1.104137in}{1.537111in}}%
\pgfpathlineto{\pgfqpoint{1.104137in}{1.087278in}}%
\pgfpathquadraticcurveto{\pgfqpoint{1.104137in}{1.053944in}}{\pgfqpoint{1.137470in}{1.053944in}}%
\pgfpathlineto{\pgfqpoint{1.137470in}{1.053944in}}%
\pgfpathclose%
\pgfusepath{stroke,fill}%
\end{pgfscope}%
\begin{pgfscope}%
\pgfsetbuttcap%
\pgfsetmiterjoin%
\pgfsetlinewidth{1.003750pt}%
\definecolor{currentstroke}{rgb}{0.000000,0.000000,0.000000}%
\pgfsetstrokecolor{currentstroke}%
\pgfsetdash{}{0pt}%
\pgfpathmoveto{\pgfqpoint{1.170804in}{1.387111in}}%
\pgfpathlineto{\pgfqpoint{1.504137in}{1.387111in}}%
\pgfpathlineto{\pgfqpoint{1.504137in}{1.503777in}}%
\pgfpathlineto{\pgfqpoint{1.170804in}{1.503777in}}%
\pgfpathlineto{\pgfqpoint{1.170804in}{1.387111in}}%
\pgfpathclose%
\pgfusepath{stroke}%
\end{pgfscope}%
\begin{pgfscope}%
\definecolor{textcolor}{rgb}{0.000000,0.000000,0.000000}%
\pgfsetstrokecolor{textcolor}%
\pgfsetfillcolor{textcolor}%
\pgftext[x=1.637470in,y=1.387111in,left,base]{\color{textcolor}\rmfamily\fontsize{12.000000}{14.400000}\selectfont Neg}%
\end{pgfscope}%
\begin{pgfscope}%
\pgfsetbuttcap%
\pgfsetmiterjoin%
\definecolor{currentfill}{rgb}{0.000000,0.000000,0.000000}%
\pgfsetfillcolor{currentfill}%
\pgfsetlinewidth{0.000000pt}%
\definecolor{currentstroke}{rgb}{0.000000,0.000000,0.000000}%
\pgfsetstrokecolor{currentstroke}%
\pgfsetstrokeopacity{0.000000}%
\pgfsetdash{}{0pt}%
\pgfpathmoveto{\pgfqpoint{1.170804in}{1.152944in}}%
\pgfpathlineto{\pgfqpoint{1.504137in}{1.152944in}}%
\pgfpathlineto{\pgfqpoint{1.504137in}{1.269611in}}%
\pgfpathlineto{\pgfqpoint{1.170804in}{1.269611in}}%
\pgfpathlineto{\pgfqpoint{1.170804in}{1.152944in}}%
\pgfpathclose%
\pgfusepath{fill}%
\end{pgfscope}%
\begin{pgfscope}%
\definecolor{textcolor}{rgb}{0.000000,0.000000,0.000000}%
\pgfsetstrokecolor{textcolor}%
\pgfsetfillcolor{textcolor}%
\pgftext[x=1.637470in,y=1.152944in,left,base]{\color{textcolor}\rmfamily\fontsize{12.000000}{14.400000}\selectfont Pos}%
\end{pgfscope}%
\end{pgfpicture}%
\makeatother%
\endgroup%

&
	\vskip 0pt
	\qquad \qquad ROC Curve
	
	%% Creator: Matplotlib, PGF backend
%%
%% To include the figure in your LaTeX document, write
%%   \input{<filename>.pgf}
%%
%% Make sure the required packages are loaded in your preamble
%%   \usepackage{pgf}
%%
%% Also ensure that all the required font packages are loaded; for instance,
%% the lmodern package is sometimes necessary when using math font.
%%   \usepackage{lmodern}
%%
%% Figures using additional raster images can only be included by \input if
%% they are in the same directory as the main LaTeX file. For loading figures
%% from other directories you can use the `import` package
%%   \usepackage{import}
%%
%% and then include the figures with
%%   \import{<path to file>}{<filename>.pgf}
%%
%% Matplotlib used the following preamble
%%   
%%   \usepackage{fontspec}
%%   \makeatletter\@ifpackageloaded{underscore}{}{\usepackage[strings]{underscore}}\makeatother
%%
\begingroup%
\makeatletter%
\begin{pgfpicture}%
\pgfpathrectangle{\pgfpointorigin}{\pgfqpoint{2.221861in}{1.754444in}}%
\pgfusepath{use as bounding box, clip}%
\begin{pgfscope}%
\pgfsetbuttcap%
\pgfsetmiterjoin%
\definecolor{currentfill}{rgb}{1.000000,1.000000,1.000000}%
\pgfsetfillcolor{currentfill}%
\pgfsetlinewidth{0.000000pt}%
\definecolor{currentstroke}{rgb}{1.000000,1.000000,1.000000}%
\pgfsetstrokecolor{currentstroke}%
\pgfsetdash{}{0pt}%
\pgfpathmoveto{\pgfqpoint{0.000000in}{0.000000in}}%
\pgfpathlineto{\pgfqpoint{2.221861in}{0.000000in}}%
\pgfpathlineto{\pgfqpoint{2.221861in}{1.754444in}}%
\pgfpathlineto{\pgfqpoint{0.000000in}{1.754444in}}%
\pgfpathlineto{\pgfqpoint{0.000000in}{0.000000in}}%
\pgfpathclose%
\pgfusepath{fill}%
\end{pgfscope}%
\begin{pgfscope}%
\pgfsetbuttcap%
\pgfsetmiterjoin%
\definecolor{currentfill}{rgb}{1.000000,1.000000,1.000000}%
\pgfsetfillcolor{currentfill}%
\pgfsetlinewidth{0.000000pt}%
\definecolor{currentstroke}{rgb}{0.000000,0.000000,0.000000}%
\pgfsetstrokecolor{currentstroke}%
\pgfsetstrokeopacity{0.000000}%
\pgfsetdash{}{0pt}%
\pgfpathmoveto{\pgfqpoint{0.553581in}{0.499444in}}%
\pgfpathlineto{\pgfqpoint{2.103581in}{0.499444in}}%
\pgfpathlineto{\pgfqpoint{2.103581in}{1.654444in}}%
\pgfpathlineto{\pgfqpoint{0.553581in}{1.654444in}}%
\pgfpathlineto{\pgfqpoint{0.553581in}{0.499444in}}%
\pgfpathclose%
\pgfusepath{fill}%
\end{pgfscope}%
\begin{pgfscope}%
\pgfsetbuttcap%
\pgfsetroundjoin%
\definecolor{currentfill}{rgb}{0.000000,0.000000,0.000000}%
\pgfsetfillcolor{currentfill}%
\pgfsetlinewidth{0.803000pt}%
\definecolor{currentstroke}{rgb}{0.000000,0.000000,0.000000}%
\pgfsetstrokecolor{currentstroke}%
\pgfsetdash{}{0pt}%
\pgfsys@defobject{currentmarker}{\pgfqpoint{0.000000in}{-0.048611in}}{\pgfqpoint{0.000000in}{0.000000in}}{%
\pgfpathmoveto{\pgfqpoint{0.000000in}{0.000000in}}%
\pgfpathlineto{\pgfqpoint{0.000000in}{-0.048611in}}%
\pgfusepath{stroke,fill}%
}%
\begin{pgfscope}%
\pgfsys@transformshift{0.624035in}{0.499444in}%
\pgfsys@useobject{currentmarker}{}%
\end{pgfscope}%
\end{pgfscope}%
\begin{pgfscope}%
\definecolor{textcolor}{rgb}{0.000000,0.000000,0.000000}%
\pgfsetstrokecolor{textcolor}%
\pgfsetfillcolor{textcolor}%
\pgftext[x=0.624035in,y=0.402222in,,top]{\color{textcolor}\rmfamily\fontsize{10.000000}{12.000000}\selectfont \(\displaystyle {0.0}\)}%
\end{pgfscope}%
\begin{pgfscope}%
\pgfsetbuttcap%
\pgfsetroundjoin%
\definecolor{currentfill}{rgb}{0.000000,0.000000,0.000000}%
\pgfsetfillcolor{currentfill}%
\pgfsetlinewidth{0.803000pt}%
\definecolor{currentstroke}{rgb}{0.000000,0.000000,0.000000}%
\pgfsetstrokecolor{currentstroke}%
\pgfsetdash{}{0pt}%
\pgfsys@defobject{currentmarker}{\pgfqpoint{0.000000in}{-0.048611in}}{\pgfqpoint{0.000000in}{0.000000in}}{%
\pgfpathmoveto{\pgfqpoint{0.000000in}{0.000000in}}%
\pgfpathlineto{\pgfqpoint{0.000000in}{-0.048611in}}%
\pgfusepath{stroke,fill}%
}%
\begin{pgfscope}%
\pgfsys@transformshift{1.328581in}{0.499444in}%
\pgfsys@useobject{currentmarker}{}%
\end{pgfscope}%
\end{pgfscope}%
\begin{pgfscope}%
\definecolor{textcolor}{rgb}{0.000000,0.000000,0.000000}%
\pgfsetstrokecolor{textcolor}%
\pgfsetfillcolor{textcolor}%
\pgftext[x=1.328581in,y=0.402222in,,top]{\color{textcolor}\rmfamily\fontsize{10.000000}{12.000000}\selectfont \(\displaystyle {0.5}\)}%
\end{pgfscope}%
\begin{pgfscope}%
\pgfsetbuttcap%
\pgfsetroundjoin%
\definecolor{currentfill}{rgb}{0.000000,0.000000,0.000000}%
\pgfsetfillcolor{currentfill}%
\pgfsetlinewidth{0.803000pt}%
\definecolor{currentstroke}{rgb}{0.000000,0.000000,0.000000}%
\pgfsetstrokecolor{currentstroke}%
\pgfsetdash{}{0pt}%
\pgfsys@defobject{currentmarker}{\pgfqpoint{0.000000in}{-0.048611in}}{\pgfqpoint{0.000000in}{0.000000in}}{%
\pgfpathmoveto{\pgfqpoint{0.000000in}{0.000000in}}%
\pgfpathlineto{\pgfqpoint{0.000000in}{-0.048611in}}%
\pgfusepath{stroke,fill}%
}%
\begin{pgfscope}%
\pgfsys@transformshift{2.033126in}{0.499444in}%
\pgfsys@useobject{currentmarker}{}%
\end{pgfscope}%
\end{pgfscope}%
\begin{pgfscope}%
\definecolor{textcolor}{rgb}{0.000000,0.000000,0.000000}%
\pgfsetstrokecolor{textcolor}%
\pgfsetfillcolor{textcolor}%
\pgftext[x=2.033126in,y=0.402222in,,top]{\color{textcolor}\rmfamily\fontsize{10.000000}{12.000000}\selectfont \(\displaystyle {1.0}\)}%
\end{pgfscope}%
\begin{pgfscope}%
\definecolor{textcolor}{rgb}{0.000000,0.000000,0.000000}%
\pgfsetstrokecolor{textcolor}%
\pgfsetfillcolor{textcolor}%
\pgftext[x=1.328581in,y=0.223333in,,top]{\color{textcolor}\rmfamily\fontsize{10.000000}{12.000000}\selectfont False positive rate}%
\end{pgfscope}%
\begin{pgfscope}%
\pgfsetbuttcap%
\pgfsetroundjoin%
\definecolor{currentfill}{rgb}{0.000000,0.000000,0.000000}%
\pgfsetfillcolor{currentfill}%
\pgfsetlinewidth{0.803000pt}%
\definecolor{currentstroke}{rgb}{0.000000,0.000000,0.000000}%
\pgfsetstrokecolor{currentstroke}%
\pgfsetdash{}{0pt}%
\pgfsys@defobject{currentmarker}{\pgfqpoint{-0.048611in}{0.000000in}}{\pgfqpoint{-0.000000in}{0.000000in}}{%
\pgfpathmoveto{\pgfqpoint{-0.000000in}{0.000000in}}%
\pgfpathlineto{\pgfqpoint{-0.048611in}{0.000000in}}%
\pgfusepath{stroke,fill}%
}%
\begin{pgfscope}%
\pgfsys@transformshift{0.553581in}{0.551944in}%
\pgfsys@useobject{currentmarker}{}%
\end{pgfscope}%
\end{pgfscope}%
\begin{pgfscope}%
\definecolor{textcolor}{rgb}{0.000000,0.000000,0.000000}%
\pgfsetstrokecolor{textcolor}%
\pgfsetfillcolor{textcolor}%
\pgftext[x=0.278889in, y=0.503750in, left, base]{\color{textcolor}\rmfamily\fontsize{10.000000}{12.000000}\selectfont \(\displaystyle {0.0}\)}%
\end{pgfscope}%
\begin{pgfscope}%
\pgfsetbuttcap%
\pgfsetroundjoin%
\definecolor{currentfill}{rgb}{0.000000,0.000000,0.000000}%
\pgfsetfillcolor{currentfill}%
\pgfsetlinewidth{0.803000pt}%
\definecolor{currentstroke}{rgb}{0.000000,0.000000,0.000000}%
\pgfsetstrokecolor{currentstroke}%
\pgfsetdash{}{0pt}%
\pgfsys@defobject{currentmarker}{\pgfqpoint{-0.048611in}{0.000000in}}{\pgfqpoint{-0.000000in}{0.000000in}}{%
\pgfpathmoveto{\pgfqpoint{-0.000000in}{0.000000in}}%
\pgfpathlineto{\pgfqpoint{-0.048611in}{0.000000in}}%
\pgfusepath{stroke,fill}%
}%
\begin{pgfscope}%
\pgfsys@transformshift{0.553581in}{1.076944in}%
\pgfsys@useobject{currentmarker}{}%
\end{pgfscope}%
\end{pgfscope}%
\begin{pgfscope}%
\definecolor{textcolor}{rgb}{0.000000,0.000000,0.000000}%
\pgfsetstrokecolor{textcolor}%
\pgfsetfillcolor{textcolor}%
\pgftext[x=0.278889in, y=1.028750in, left, base]{\color{textcolor}\rmfamily\fontsize{10.000000}{12.000000}\selectfont \(\displaystyle {0.5}\)}%
\end{pgfscope}%
\begin{pgfscope}%
\pgfsetbuttcap%
\pgfsetroundjoin%
\definecolor{currentfill}{rgb}{0.000000,0.000000,0.000000}%
\pgfsetfillcolor{currentfill}%
\pgfsetlinewidth{0.803000pt}%
\definecolor{currentstroke}{rgb}{0.000000,0.000000,0.000000}%
\pgfsetstrokecolor{currentstroke}%
\pgfsetdash{}{0pt}%
\pgfsys@defobject{currentmarker}{\pgfqpoint{-0.048611in}{0.000000in}}{\pgfqpoint{-0.000000in}{0.000000in}}{%
\pgfpathmoveto{\pgfqpoint{-0.000000in}{0.000000in}}%
\pgfpathlineto{\pgfqpoint{-0.048611in}{0.000000in}}%
\pgfusepath{stroke,fill}%
}%
\begin{pgfscope}%
\pgfsys@transformshift{0.553581in}{1.601944in}%
\pgfsys@useobject{currentmarker}{}%
\end{pgfscope}%
\end{pgfscope}%
\begin{pgfscope}%
\definecolor{textcolor}{rgb}{0.000000,0.000000,0.000000}%
\pgfsetstrokecolor{textcolor}%
\pgfsetfillcolor{textcolor}%
\pgftext[x=0.278889in, y=1.553750in, left, base]{\color{textcolor}\rmfamily\fontsize{10.000000}{12.000000}\selectfont \(\displaystyle {1.0}\)}%
\end{pgfscope}%
\begin{pgfscope}%
\definecolor{textcolor}{rgb}{0.000000,0.000000,0.000000}%
\pgfsetstrokecolor{textcolor}%
\pgfsetfillcolor{textcolor}%
\pgftext[x=0.223333in,y=1.076944in,,bottom,rotate=90.000000]{\color{textcolor}\rmfamily\fontsize{10.000000}{12.000000}\selectfont True positive rate}%
\end{pgfscope}%
\begin{pgfscope}%
\pgfpathrectangle{\pgfqpoint{0.553581in}{0.499444in}}{\pgfqpoint{1.550000in}{1.155000in}}%
\pgfusepath{clip}%
\pgfsetbuttcap%
\pgfsetroundjoin%
\pgfsetlinewidth{1.505625pt}%
\definecolor{currentstroke}{rgb}{0.000000,0.000000,0.000000}%
\pgfsetstrokecolor{currentstroke}%
\pgfsetdash{{5.550000pt}{2.400000pt}}{0.000000pt}%
\pgfpathmoveto{\pgfqpoint{0.624035in}{0.551944in}}%
\pgfpathlineto{\pgfqpoint{2.033126in}{1.601944in}}%
\pgfusepath{stroke}%
\end{pgfscope}%
\begin{pgfscope}%
\pgfpathrectangle{\pgfqpoint{0.553581in}{0.499444in}}{\pgfqpoint{1.550000in}{1.155000in}}%
\pgfusepath{clip}%
\pgfsetrectcap%
\pgfsetroundjoin%
\pgfsetlinewidth{1.505625pt}%
\definecolor{currentstroke}{rgb}{0.000000,0.000000,0.000000}%
\pgfsetstrokecolor{currentstroke}%
\pgfsetdash{}{0pt}%
\pgfpathmoveto{\pgfqpoint{0.624035in}{0.551944in}}%
\pgfpathlineto{\pgfqpoint{0.626207in}{0.552574in}}%
\pgfpathlineto{\pgfqpoint{0.627318in}{0.561464in}}%
\pgfpathlineto{\pgfqpoint{0.628014in}{0.562514in}}%
\pgfpathlineto{\pgfqpoint{0.629125in}{0.563634in}}%
\pgfpathlineto{\pgfqpoint{0.629605in}{0.564614in}}%
\pgfpathlineto{\pgfqpoint{0.630699in}{0.567694in}}%
\pgfpathlineto{\pgfqpoint{0.631130in}{0.568604in}}%
\pgfpathlineto{\pgfqpoint{0.632225in}{0.573294in}}%
\pgfpathlineto{\pgfqpoint{0.632772in}{0.574064in}}%
\pgfpathlineto{\pgfqpoint{0.633866in}{0.579944in}}%
\pgfpathlineto{\pgfqpoint{0.634247in}{0.580854in}}%
\pgfpathlineto{\pgfqpoint{0.635358in}{0.585824in}}%
\pgfpathlineto{\pgfqpoint{0.635540in}{0.586874in}}%
\pgfpathlineto{\pgfqpoint{0.636634in}{0.593244in}}%
\pgfpathlineto{\pgfqpoint{0.636833in}{0.594154in}}%
\pgfpathlineto{\pgfqpoint{0.637944in}{0.600734in}}%
\pgfpathlineto{\pgfqpoint{0.638126in}{0.601784in}}%
\pgfpathlineto{\pgfqpoint{0.639187in}{0.608714in}}%
\pgfpathlineto{\pgfqpoint{0.639353in}{0.609694in}}%
\pgfpathlineto{\pgfqpoint{0.640464in}{0.616834in}}%
\pgfpathlineto{\pgfqpoint{0.640712in}{0.617884in}}%
\pgfpathlineto{\pgfqpoint{0.641806in}{0.625864in}}%
\pgfpathlineto{\pgfqpoint{0.642038in}{0.626844in}}%
\pgfpathlineto{\pgfqpoint{0.643149in}{0.635804in}}%
\pgfpathlineto{\pgfqpoint{0.643348in}{0.636854in}}%
\pgfpathlineto{\pgfqpoint{0.644392in}{0.645044in}}%
\pgfpathlineto{\pgfqpoint{0.644641in}{0.645884in}}%
\pgfpathlineto{\pgfqpoint{0.645752in}{0.652394in}}%
\pgfpathlineto{\pgfqpoint{0.645868in}{0.653164in}}%
\pgfpathlineto{\pgfqpoint{0.646979in}{0.661424in}}%
\pgfpathlineto{\pgfqpoint{0.647327in}{0.662264in}}%
\pgfpathlineto{\pgfqpoint{0.648437in}{0.668914in}}%
\pgfpathlineto{\pgfqpoint{0.648587in}{0.669824in}}%
\pgfpathlineto{\pgfqpoint{0.649697in}{0.678294in}}%
\pgfpathlineto{\pgfqpoint{0.649813in}{0.679274in}}%
\pgfpathlineto{\pgfqpoint{0.650924in}{0.687044in}}%
\pgfpathlineto{\pgfqpoint{0.651140in}{0.687884in}}%
\pgfpathlineto{\pgfqpoint{0.652250in}{0.696144in}}%
\pgfpathlineto{\pgfqpoint{0.652333in}{0.697194in}}%
\pgfpathlineto{\pgfqpoint{0.653444in}{0.706434in}}%
\pgfpathlineto{\pgfqpoint{0.653858in}{0.707414in}}%
\pgfpathlineto{\pgfqpoint{0.654969in}{0.716584in}}%
\pgfpathlineto{\pgfqpoint{0.655151in}{0.717564in}}%
\pgfpathlineto{\pgfqpoint{0.656262in}{0.725194in}}%
\pgfpathlineto{\pgfqpoint{0.656444in}{0.726174in}}%
\pgfpathlineto{\pgfqpoint{0.657538in}{0.733244in}}%
\pgfpathlineto{\pgfqpoint{0.657704in}{0.734294in}}%
\pgfpathlineto{\pgfqpoint{0.658815in}{0.743464in}}%
\pgfpathlineto{\pgfqpoint{0.659064in}{0.744514in}}%
\pgfpathlineto{\pgfqpoint{0.660158in}{0.751164in}}%
\pgfpathlineto{\pgfqpoint{0.660290in}{0.751794in}}%
\pgfpathlineto{\pgfqpoint{0.661401in}{0.759354in}}%
\pgfpathlineto{\pgfqpoint{0.661633in}{0.760404in}}%
\pgfpathlineto{\pgfqpoint{0.662744in}{0.769784in}}%
\pgfpathlineto{\pgfqpoint{0.662827in}{0.770344in}}%
\pgfpathlineto{\pgfqpoint{0.663937in}{0.777764in}}%
\pgfpathlineto{\pgfqpoint{0.664120in}{0.778814in}}%
\pgfpathlineto{\pgfqpoint{0.665181in}{0.783854in}}%
\pgfpathlineto{\pgfqpoint{0.665297in}{0.784834in}}%
\pgfpathlineto{\pgfqpoint{0.666407in}{0.795614in}}%
\pgfpathlineto{\pgfqpoint{0.666656in}{0.796664in}}%
\pgfpathlineto{\pgfqpoint{0.667767in}{0.801564in}}%
\pgfpathlineto{\pgfqpoint{0.667899in}{0.802404in}}%
\pgfpathlineto{\pgfqpoint{0.669010in}{0.809334in}}%
\pgfpathlineto{\pgfqpoint{0.669342in}{0.810384in}}%
\pgfpathlineto{\pgfqpoint{0.670452in}{0.818084in}}%
\pgfpathlineto{\pgfqpoint{0.670701in}{0.819134in}}%
\pgfpathlineto{\pgfqpoint{0.671812in}{0.825364in}}%
\pgfpathlineto{\pgfqpoint{0.671911in}{0.826414in}}%
\pgfpathlineto{\pgfqpoint{0.673022in}{0.833694in}}%
\pgfpathlineto{\pgfqpoint{0.673138in}{0.834744in}}%
\pgfpathlineto{\pgfqpoint{0.674215in}{0.838944in}}%
\pgfpathlineto{\pgfqpoint{0.674530in}{0.839994in}}%
\pgfpathlineto{\pgfqpoint{0.675608in}{0.846014in}}%
\pgfpathlineto{\pgfqpoint{0.675790in}{0.846994in}}%
\pgfpathlineto{\pgfqpoint{0.676901in}{0.853854in}}%
\pgfpathlineto{\pgfqpoint{0.677199in}{0.854834in}}%
\pgfpathlineto{\pgfqpoint{0.678310in}{0.859034in}}%
\pgfpathlineto{\pgfqpoint{0.678492in}{0.860084in}}%
\pgfpathlineto{\pgfqpoint{0.679587in}{0.867014in}}%
\pgfpathlineto{\pgfqpoint{0.679868in}{0.867924in}}%
\pgfpathlineto{\pgfqpoint{0.680963in}{0.874504in}}%
\pgfpathlineto{\pgfqpoint{0.681228in}{0.875484in}}%
\pgfpathlineto{\pgfqpoint{0.682338in}{0.882204in}}%
\pgfpathlineto{\pgfqpoint{0.682587in}{0.883254in}}%
\pgfpathlineto{\pgfqpoint{0.683681in}{0.889484in}}%
\pgfpathlineto{\pgfqpoint{0.683980in}{0.890394in}}%
\pgfpathlineto{\pgfqpoint{0.685074in}{0.897394in}}%
\pgfpathlineto{\pgfqpoint{0.685389in}{0.898304in}}%
\pgfpathlineto{\pgfqpoint{0.686450in}{0.904184in}}%
\pgfpathlineto{\pgfqpoint{0.686748in}{0.905094in}}%
\pgfpathlineto{\pgfqpoint{0.687859in}{0.910204in}}%
\pgfpathlineto{\pgfqpoint{0.688074in}{0.911114in}}%
\pgfpathlineto{\pgfqpoint{0.689185in}{0.915034in}}%
\pgfpathlineto{\pgfqpoint{0.689400in}{0.916084in}}%
\pgfpathlineto{\pgfqpoint{0.690511in}{0.922664in}}%
\pgfpathlineto{\pgfqpoint{0.690793in}{0.923714in}}%
\pgfpathlineto{\pgfqpoint{0.691887in}{0.928754in}}%
\pgfpathlineto{\pgfqpoint{0.692318in}{0.929734in}}%
\pgfpathlineto{\pgfqpoint{0.693412in}{0.935334in}}%
\pgfpathlineto{\pgfqpoint{0.693611in}{0.936384in}}%
\pgfpathlineto{\pgfqpoint{0.694705in}{0.940864in}}%
\pgfpathlineto{\pgfqpoint{0.695053in}{0.941914in}}%
\pgfpathlineto{\pgfqpoint{0.696164in}{0.948634in}}%
\pgfpathlineto{\pgfqpoint{0.696645in}{0.949684in}}%
\pgfpathlineto{\pgfqpoint{0.697756in}{0.954164in}}%
\pgfpathlineto{\pgfqpoint{0.697921in}{0.955144in}}%
\pgfpathlineto{\pgfqpoint{0.699015in}{0.960464in}}%
\pgfpathlineto{\pgfqpoint{0.699314in}{0.961304in}}%
\pgfpathlineto{\pgfqpoint{0.700425in}{0.965294in}}%
\pgfpathlineto{\pgfqpoint{0.700657in}{0.966344in}}%
\pgfpathlineto{\pgfqpoint{0.701767in}{0.971314in}}%
\pgfpathlineto{\pgfqpoint{0.702049in}{0.972364in}}%
\pgfpathlineto{\pgfqpoint{0.703094in}{0.977194in}}%
\pgfpathlineto{\pgfqpoint{0.703442in}{0.978104in}}%
\pgfpathlineto{\pgfqpoint{0.704536in}{0.982514in}}%
\pgfpathlineto{\pgfqpoint{0.704917in}{0.983564in}}%
\pgfpathlineto{\pgfqpoint{0.706011in}{0.988464in}}%
\pgfpathlineto{\pgfqpoint{0.706160in}{0.989444in}}%
\pgfpathlineto{\pgfqpoint{0.707271in}{0.993504in}}%
\pgfpathlineto{\pgfqpoint{0.707619in}{0.994414in}}%
\pgfpathlineto{\pgfqpoint{0.708697in}{0.998054in}}%
\pgfpathlineto{\pgfqpoint{0.709061in}{0.999104in}}%
\pgfpathlineto{\pgfqpoint{0.710172in}{1.004564in}}%
\pgfpathlineto{\pgfqpoint{0.710504in}{1.005614in}}%
\pgfpathlineto{\pgfqpoint{0.711614in}{1.009604in}}%
\pgfpathlineto{\pgfqpoint{0.711830in}{1.010584in}}%
\pgfpathlineto{\pgfqpoint{0.712874in}{1.015134in}}%
\pgfpathlineto{\pgfqpoint{0.713123in}{1.016184in}}%
\pgfpathlineto{\pgfqpoint{0.714184in}{1.020244in}}%
\pgfpathlineto{\pgfqpoint{0.714598in}{1.021294in}}%
\pgfpathlineto{\pgfqpoint{0.715709in}{1.024304in}}%
\pgfpathlineto{\pgfqpoint{0.716173in}{1.025354in}}%
\pgfpathlineto{\pgfqpoint{0.717284in}{1.029624in}}%
\pgfpathlineto{\pgfqpoint{0.717466in}{1.030604in}}%
\pgfpathlineto{\pgfqpoint{0.718560in}{1.035014in}}%
\pgfpathlineto{\pgfqpoint{0.718776in}{1.035994in}}%
\pgfpathlineto{\pgfqpoint{0.719853in}{1.041034in}}%
\pgfpathlineto{\pgfqpoint{0.720202in}{1.042084in}}%
\pgfpathlineto{\pgfqpoint{0.721312in}{1.046144in}}%
\pgfpathlineto{\pgfqpoint{0.721694in}{1.047194in}}%
\pgfpathlineto{\pgfqpoint{0.722788in}{1.050904in}}%
\pgfpathlineto{\pgfqpoint{0.723086in}{1.051954in}}%
\pgfpathlineto{\pgfqpoint{0.724180in}{1.056364in}}%
\pgfpathlineto{\pgfqpoint{0.724727in}{1.057414in}}%
\pgfpathlineto{\pgfqpoint{0.725805in}{1.061264in}}%
\pgfpathlineto{\pgfqpoint{0.726236in}{1.062314in}}%
\pgfpathlineto{\pgfqpoint{0.727280in}{1.065744in}}%
\pgfpathlineto{\pgfqpoint{0.727728in}{1.066794in}}%
\pgfpathlineto{\pgfqpoint{0.728805in}{1.069874in}}%
\pgfpathlineto{\pgfqpoint{0.729253in}{1.070924in}}%
\pgfpathlineto{\pgfqpoint{0.730364in}{1.074144in}}%
\pgfpathlineto{\pgfqpoint{0.730695in}{1.075124in}}%
\pgfpathlineto{\pgfqpoint{0.731789in}{1.079044in}}%
\pgfpathlineto{\pgfqpoint{0.732336in}{1.080024in}}%
\pgfpathlineto{\pgfqpoint{0.733447in}{1.084574in}}%
\pgfpathlineto{\pgfqpoint{0.734094in}{1.085624in}}%
\pgfpathlineto{\pgfqpoint{0.735188in}{1.089544in}}%
\pgfpathlineto{\pgfqpoint{0.735552in}{1.090594in}}%
\pgfpathlineto{\pgfqpoint{0.736597in}{1.093674in}}%
\pgfpathlineto{\pgfqpoint{0.737177in}{1.094724in}}%
\pgfpathlineto{\pgfqpoint{0.738221in}{1.097524in}}%
\pgfpathlineto{\pgfqpoint{0.738702in}{1.098574in}}%
\pgfpathlineto{\pgfqpoint{0.739780in}{1.102424in}}%
\pgfpathlineto{\pgfqpoint{0.740128in}{1.103474in}}%
\pgfpathlineto{\pgfqpoint{0.741238in}{1.106624in}}%
\pgfpathlineto{\pgfqpoint{0.741636in}{1.107674in}}%
\pgfpathlineto{\pgfqpoint{0.742697in}{1.111034in}}%
\pgfpathlineto{\pgfqpoint{0.743095in}{1.112084in}}%
\pgfpathlineto{\pgfqpoint{0.744206in}{1.115024in}}%
\pgfpathlineto{\pgfqpoint{0.744769in}{1.116074in}}%
\pgfpathlineto{\pgfqpoint{0.745051in}{1.117054in}}%
\pgfpathlineto{\pgfqpoint{0.745068in}{1.117054in}}%
\pgfpathlineto{\pgfqpoint{0.755943in}{1.118104in}}%
\pgfpathlineto{\pgfqpoint{0.757053in}{1.121814in}}%
\pgfpathlineto{\pgfqpoint{0.757600in}{1.122864in}}%
\pgfpathlineto{\pgfqpoint{0.758678in}{1.125384in}}%
\pgfpathlineto{\pgfqpoint{0.759225in}{1.126434in}}%
\pgfpathlineto{\pgfqpoint{0.760269in}{1.129094in}}%
\pgfpathlineto{\pgfqpoint{0.760817in}{1.130144in}}%
\pgfpathlineto{\pgfqpoint{0.761911in}{1.133224in}}%
\pgfpathlineto{\pgfqpoint{0.762474in}{1.134274in}}%
\pgfpathlineto{\pgfqpoint{0.763535in}{1.137844in}}%
\pgfpathlineto{\pgfqpoint{0.764165in}{1.138824in}}%
\pgfpathlineto{\pgfqpoint{0.765259in}{1.142324in}}%
\pgfpathlineto{\pgfqpoint{0.766055in}{1.143374in}}%
\pgfpathlineto{\pgfqpoint{0.767166in}{1.146314in}}%
\pgfpathlineto{\pgfqpoint{0.767696in}{1.147364in}}%
\pgfpathlineto{\pgfqpoint{0.768790in}{1.149954in}}%
\pgfpathlineto{\pgfqpoint{0.769603in}{1.151004in}}%
\pgfpathlineto{\pgfqpoint{0.770514in}{1.153314in}}%
\pgfpathlineto{\pgfqpoint{0.771161in}{1.154364in}}%
\pgfpathlineto{\pgfqpoint{0.772272in}{1.157374in}}%
\pgfpathlineto{\pgfqpoint{0.772852in}{1.158424in}}%
\pgfpathlineto{\pgfqpoint{0.773946in}{1.160664in}}%
\pgfpathlineto{\pgfqpoint{0.774609in}{1.161714in}}%
\pgfpathlineto{\pgfqpoint{0.775703in}{1.164794in}}%
\pgfpathlineto{\pgfqpoint{0.776267in}{1.165774in}}%
\pgfpathlineto{\pgfqpoint{0.777344in}{1.168714in}}%
\pgfpathlineto{\pgfqpoint{0.777759in}{1.169764in}}%
\pgfpathlineto{\pgfqpoint{0.778836in}{1.172144in}}%
\pgfpathlineto{\pgfqpoint{0.779350in}{1.173124in}}%
\pgfpathlineto{\pgfqpoint{0.780444in}{1.174524in}}%
\pgfpathlineto{\pgfqpoint{0.781008in}{1.175574in}}%
\pgfpathlineto{\pgfqpoint{0.782052in}{1.177254in}}%
\pgfpathlineto{\pgfqpoint{0.782699in}{1.178304in}}%
\pgfpathlineto{\pgfqpoint{0.783793in}{1.180894in}}%
\pgfpathlineto{\pgfqpoint{0.784290in}{1.181874in}}%
\pgfpathlineto{\pgfqpoint{0.785384in}{1.184044in}}%
\pgfpathlineto{\pgfqpoint{0.785981in}{1.185094in}}%
\pgfpathlineto{\pgfqpoint{0.787075in}{1.186914in}}%
\pgfpathlineto{\pgfqpoint{0.787755in}{1.187894in}}%
\pgfpathlineto{\pgfqpoint{0.788866in}{1.189784in}}%
\pgfpathlineto{\pgfqpoint{0.789297in}{1.190764in}}%
\pgfpathlineto{\pgfqpoint{0.790407in}{1.192794in}}%
\pgfpathlineto{\pgfqpoint{0.790689in}{1.193634in}}%
\pgfpathlineto{\pgfqpoint{0.791783in}{1.196084in}}%
\pgfpathlineto{\pgfqpoint{0.792347in}{1.197134in}}%
\pgfpathlineto{\pgfqpoint{0.793358in}{1.199724in}}%
\pgfpathlineto{\pgfqpoint{0.794054in}{1.200774in}}%
\pgfpathlineto{\pgfqpoint{0.795099in}{1.201964in}}%
\pgfpathlineto{\pgfqpoint{0.796044in}{1.203014in}}%
\pgfpathlineto{\pgfqpoint{0.797138in}{1.205534in}}%
\pgfpathlineto{\pgfqpoint{0.797718in}{1.206584in}}%
\pgfpathlineto{\pgfqpoint{0.798812in}{1.208474in}}%
\pgfpathlineto{\pgfqpoint{0.799409in}{1.209524in}}%
\pgfpathlineto{\pgfqpoint{0.800520in}{1.212114in}}%
\pgfpathlineto{\pgfqpoint{0.801382in}{1.213094in}}%
\pgfpathlineto{\pgfqpoint{0.802492in}{1.215894in}}%
\pgfpathlineto{\pgfqpoint{0.803106in}{1.216664in}}%
\pgfpathlineto{\pgfqpoint{0.804134in}{1.218554in}}%
\pgfpathlineto{\pgfqpoint{0.805128in}{1.219604in}}%
\pgfpathlineto{\pgfqpoint{0.806222in}{1.221774in}}%
\pgfpathlineto{\pgfqpoint{0.806786in}{1.222614in}}%
\pgfpathlineto{\pgfqpoint{0.807897in}{1.225484in}}%
\pgfpathlineto{\pgfqpoint{0.808659in}{1.226534in}}%
\pgfpathlineto{\pgfqpoint{0.809753in}{1.228284in}}%
\pgfpathlineto{\pgfqpoint{0.810267in}{1.229334in}}%
\pgfpathlineto{\pgfqpoint{0.811378in}{1.231224in}}%
\pgfpathlineto{\pgfqpoint{0.812654in}{1.232274in}}%
\pgfpathlineto{\pgfqpoint{0.813682in}{1.233744in}}%
\pgfpathlineto{\pgfqpoint{0.814180in}{1.234794in}}%
\pgfpathlineto{\pgfqpoint{0.815290in}{1.236614in}}%
\pgfpathlineto{\pgfqpoint{0.816053in}{1.237664in}}%
\pgfpathlineto{\pgfqpoint{0.816948in}{1.239484in}}%
\pgfpathlineto{\pgfqpoint{0.818109in}{1.240464in}}%
\pgfpathlineto{\pgfqpoint{0.819203in}{1.242424in}}%
\pgfpathlineto{\pgfqpoint{0.819949in}{1.243404in}}%
\pgfpathlineto{\pgfqpoint{0.821043in}{1.245364in}}%
\pgfpathlineto{\pgfqpoint{0.821838in}{1.246414in}}%
\pgfpathlineto{\pgfqpoint{0.822899in}{1.248374in}}%
\pgfpathlineto{\pgfqpoint{0.823828in}{1.249424in}}%
\pgfpathlineto{\pgfqpoint{0.824905in}{1.251314in}}%
\pgfpathlineto{\pgfqpoint{0.826049in}{1.252364in}}%
\pgfpathlineto{\pgfqpoint{0.827110in}{1.254044in}}%
\pgfpathlineto{\pgfqpoint{0.828204in}{1.255024in}}%
\pgfpathlineto{\pgfqpoint{0.829298in}{1.257404in}}%
\pgfpathlineto{\pgfqpoint{0.830094in}{1.258454in}}%
\pgfpathlineto{\pgfqpoint{0.831188in}{1.259784in}}%
\pgfpathlineto{\pgfqpoint{0.831901in}{1.260834in}}%
\pgfpathlineto{\pgfqpoint{0.832979in}{1.261954in}}%
\pgfpathlineto{\pgfqpoint{0.833841in}{1.263004in}}%
\pgfpathlineto{\pgfqpoint{0.834769in}{1.263774in}}%
\pgfpathlineto{\pgfqpoint{0.835697in}{1.264824in}}%
\pgfpathlineto{\pgfqpoint{0.836659in}{1.266014in}}%
\pgfpathlineto{\pgfqpoint{0.838234in}{1.267064in}}%
\pgfpathlineto{\pgfqpoint{0.839295in}{1.268814in}}%
\pgfpathlineto{\pgfqpoint{0.840438in}{1.269864in}}%
\pgfpathlineto{\pgfqpoint{0.841549in}{1.271824in}}%
\pgfpathlineto{\pgfqpoint{0.842594in}{1.272874in}}%
\pgfpathlineto{\pgfqpoint{0.843688in}{1.274414in}}%
\pgfpathlineto{\pgfqpoint{0.844881in}{1.275464in}}%
\pgfpathlineto{\pgfqpoint{0.845992in}{1.277844in}}%
\pgfpathlineto{\pgfqpoint{0.846705in}{1.278894in}}%
\pgfpathlineto{\pgfqpoint{0.847633in}{1.280294in}}%
\pgfpathlineto{\pgfqpoint{0.848644in}{1.281274in}}%
\pgfpathlineto{\pgfqpoint{0.849606in}{1.282954in}}%
\pgfpathlineto{\pgfqpoint{0.850882in}{1.284004in}}%
\pgfpathlineto{\pgfqpoint{0.851960in}{1.285194in}}%
\pgfpathlineto{\pgfqpoint{0.853120in}{1.286244in}}%
\pgfpathlineto{\pgfqpoint{0.854198in}{1.287644in}}%
\pgfpathlineto{\pgfqpoint{0.855176in}{1.288694in}}%
\pgfpathlineto{\pgfqpoint{0.856237in}{1.290234in}}%
\pgfpathlineto{\pgfqpoint{0.856917in}{1.291144in}}%
\pgfpathlineto{\pgfqpoint{0.858027in}{1.292754in}}%
\pgfpathlineto{\pgfqpoint{0.859287in}{1.293804in}}%
\pgfpathlineto{\pgfqpoint{0.860232in}{1.294714in}}%
\pgfpathlineto{\pgfqpoint{0.861160in}{1.295764in}}%
\pgfpathlineto{\pgfqpoint{0.862155in}{1.296814in}}%
\pgfpathlineto{\pgfqpoint{0.863746in}{1.297794in}}%
\pgfpathlineto{\pgfqpoint{0.864807in}{1.299054in}}%
\pgfpathlineto{\pgfqpoint{0.866084in}{1.300104in}}%
\pgfpathlineto{\pgfqpoint{0.867128in}{1.301854in}}%
\pgfpathlineto{\pgfqpoint{0.867775in}{1.302904in}}%
\pgfpathlineto{\pgfqpoint{0.868886in}{1.304164in}}%
\pgfpathlineto{\pgfqpoint{0.869897in}{1.305144in}}%
\pgfpathlineto{\pgfqpoint{0.870991in}{1.306614in}}%
\pgfpathlineto{\pgfqpoint{0.872218in}{1.307664in}}%
\pgfpathlineto{\pgfqpoint{0.873312in}{1.309624in}}%
\pgfpathlineto{\pgfqpoint{0.874572in}{1.310674in}}%
\pgfpathlineto{\pgfqpoint{0.875616in}{1.312284in}}%
\pgfpathlineto{\pgfqpoint{0.876263in}{1.313334in}}%
\pgfpathlineto{\pgfqpoint{0.877373in}{1.314804in}}%
\pgfpathlineto{\pgfqpoint{0.878036in}{1.315854in}}%
\pgfpathlineto{\pgfqpoint{0.879081in}{1.316974in}}%
\pgfpathlineto{\pgfqpoint{0.880191in}{1.318024in}}%
\pgfpathlineto{\pgfqpoint{0.881252in}{1.319564in}}%
\pgfpathlineto{\pgfqpoint{0.882927in}{1.320544in}}%
\pgfpathlineto{\pgfqpoint{0.883954in}{1.321384in}}%
\pgfpathlineto{\pgfqpoint{0.885165in}{1.322434in}}%
\pgfpathlineto{\pgfqpoint{0.886226in}{1.323624in}}%
\pgfpathlineto{\pgfqpoint{0.887121in}{1.324674in}}%
\pgfpathlineto{\pgfqpoint{0.888215in}{1.325864in}}%
\pgfpathlineto{\pgfqpoint{0.889624in}{1.326914in}}%
\pgfpathlineto{\pgfqpoint{0.890619in}{1.328104in}}%
\pgfpathlineto{\pgfqpoint{0.892044in}{1.329154in}}%
\pgfpathlineto{\pgfqpoint{0.892857in}{1.330064in}}%
\pgfpathlineto{\pgfqpoint{0.894100in}{1.331044in}}%
\pgfpathlineto{\pgfqpoint{0.894746in}{1.331814in}}%
\pgfpathlineto{\pgfqpoint{0.896089in}{1.332864in}}%
\pgfpathlineto{\pgfqpoint{0.897200in}{1.333984in}}%
\pgfpathlineto{\pgfqpoint{0.898327in}{1.335034in}}%
\pgfpathlineto{\pgfqpoint{0.899338in}{1.336714in}}%
\pgfpathlineto{\pgfqpoint{0.900267in}{1.337694in}}%
\pgfpathlineto{\pgfqpoint{0.901311in}{1.338534in}}%
\pgfpathlineto{\pgfqpoint{0.902554in}{1.339584in}}%
\pgfpathlineto{\pgfqpoint{0.903649in}{1.340494in}}%
\pgfpathlineto{\pgfqpoint{0.905240in}{1.341544in}}%
\pgfpathlineto{\pgfqpoint{0.906334in}{1.342874in}}%
\pgfpathlineto{\pgfqpoint{0.907743in}{1.343924in}}%
\pgfpathlineto{\pgfqpoint{0.908854in}{1.344764in}}%
\pgfpathlineto{\pgfqpoint{0.910048in}{1.345814in}}%
\pgfpathlineto{\pgfqpoint{0.911142in}{1.346654in}}%
\pgfpathlineto{\pgfqpoint{0.912169in}{1.347704in}}%
\pgfpathlineto{\pgfqpoint{0.913114in}{1.348754in}}%
\pgfpathlineto{\pgfqpoint{0.914242in}{1.349804in}}%
\pgfpathlineto{\pgfqpoint{0.915004in}{1.350924in}}%
\pgfpathlineto{\pgfqpoint{0.916894in}{1.351974in}}%
\pgfpathlineto{\pgfqpoint{0.917988in}{1.353234in}}%
\pgfpathlineto{\pgfqpoint{0.919646in}{1.354284in}}%
\pgfpathlineto{\pgfqpoint{0.920757in}{1.356034in}}%
\pgfpathlineto{\pgfqpoint{0.922166in}{1.357084in}}%
\pgfpathlineto{\pgfqpoint{0.923260in}{1.358204in}}%
\pgfpathlineto{\pgfqpoint{0.925332in}{1.359254in}}%
\pgfpathlineto{\pgfqpoint{0.926376in}{1.360164in}}%
\pgfpathlineto{\pgfqpoint{0.927686in}{1.361214in}}%
\pgfpathlineto{\pgfqpoint{0.928780in}{1.361914in}}%
\pgfpathlineto{\pgfqpoint{0.930886in}{1.362964in}}%
\pgfpathlineto{\pgfqpoint{0.931980in}{1.364294in}}%
\pgfpathlineto{\pgfqpoint{0.932941in}{1.365344in}}%
\pgfpathlineto{\pgfqpoint{0.934035in}{1.366604in}}%
\pgfpathlineto{\pgfqpoint{0.935925in}{1.367654in}}%
\pgfpathlineto{\pgfqpoint{0.937003in}{1.368984in}}%
\pgfpathlineto{\pgfqpoint{0.939307in}{1.370034in}}%
\pgfpathlineto{\pgfqpoint{0.940302in}{1.371014in}}%
\pgfpathlineto{\pgfqpoint{0.941876in}{1.372064in}}%
\pgfpathlineto{\pgfqpoint{0.942871in}{1.372694in}}%
\pgfpathlineto{\pgfqpoint{0.944562in}{1.373674in}}%
\pgfpathlineto{\pgfqpoint{0.945573in}{1.374934in}}%
\pgfpathlineto{\pgfqpoint{0.947314in}{1.375984in}}%
\pgfpathlineto{\pgfqpoint{0.948159in}{1.376824in}}%
\pgfpathlineto{\pgfqpoint{0.950663in}{1.377874in}}%
\pgfpathlineto{\pgfqpoint{0.951707in}{1.378504in}}%
\pgfpathlineto{\pgfqpoint{0.953083in}{1.379484in}}%
\pgfpathlineto{\pgfqpoint{0.954127in}{1.380394in}}%
\pgfpathlineto{\pgfqpoint{0.956183in}{1.381444in}}%
\pgfpathlineto{\pgfqpoint{0.957244in}{1.382914in}}%
\pgfpathlineto{\pgfqpoint{0.958736in}{1.383964in}}%
\pgfpathlineto{\pgfqpoint{0.959714in}{1.384874in}}%
\pgfpathlineto{\pgfqpoint{0.961902in}{1.385924in}}%
\pgfpathlineto{\pgfqpoint{0.962781in}{1.386414in}}%
\pgfpathlineto{\pgfqpoint{0.965599in}{1.387464in}}%
\pgfpathlineto{\pgfqpoint{0.966643in}{1.388514in}}%
\pgfpathlineto{\pgfqpoint{0.968069in}{1.389564in}}%
\pgfpathlineto{\pgfqpoint{0.968931in}{1.390404in}}%
\pgfpathlineto{\pgfqpoint{0.971733in}{1.391454in}}%
\pgfpathlineto{\pgfqpoint{0.972843in}{1.391874in}}%
\pgfpathlineto{\pgfqpoint{0.974286in}{1.392924in}}%
\pgfpathlineto{\pgfqpoint{0.975197in}{1.393554in}}%
\pgfpathlineto{\pgfqpoint{0.977336in}{1.394604in}}%
\pgfpathlineto{\pgfqpoint{0.978413in}{1.395654in}}%
\pgfpathlineto{\pgfqpoint{0.979607in}{1.396704in}}%
\pgfpathlineto{\pgfqpoint{0.980635in}{1.397614in}}%
\pgfpathlineto{\pgfqpoint{0.981729in}{1.398664in}}%
\pgfpathlineto{\pgfqpoint{0.982690in}{1.399224in}}%
\pgfpathlineto{\pgfqpoint{0.984365in}{1.400274in}}%
\pgfpathlineto{\pgfqpoint{0.985343in}{1.401044in}}%
\pgfpathlineto{\pgfqpoint{0.987050in}{1.402094in}}%
\pgfpathlineto{\pgfqpoint{0.987912in}{1.403004in}}%
\pgfpathlineto{\pgfqpoint{0.990051in}{1.404054in}}%
\pgfpathlineto{\pgfqpoint{0.990631in}{1.404474in}}%
\pgfpathlineto{\pgfqpoint{0.992935in}{1.405524in}}%
\pgfpathlineto{\pgfqpoint{0.993847in}{1.406294in}}%
\pgfpathlineto{\pgfqpoint{0.995787in}{1.407344in}}%
\pgfpathlineto{\pgfqpoint{0.996798in}{1.408534in}}%
\pgfpathlineto{\pgfqpoint{0.998456in}{1.409444in}}%
\pgfpathlineto{\pgfqpoint{0.999533in}{1.410424in}}%
\pgfpathlineto{\pgfqpoint{1.001224in}{1.411404in}}%
\pgfpathlineto{\pgfqpoint{1.002235in}{1.412314in}}%
\pgfpathlineto{\pgfqpoint{1.003545in}{1.413364in}}%
\pgfpathlineto{\pgfqpoint{1.004423in}{1.414274in}}%
\pgfpathlineto{\pgfqpoint{1.005882in}{1.415324in}}%
\pgfpathlineto{\pgfqpoint{1.006678in}{1.415954in}}%
\pgfpathlineto{\pgfqpoint{1.009165in}{1.417004in}}%
\pgfpathlineto{\pgfqpoint{1.010226in}{1.417774in}}%
\pgfpathlineto{\pgfqpoint{1.012861in}{1.418824in}}%
\pgfpathlineto{\pgfqpoint{1.013972in}{1.419524in}}%
\pgfpathlineto{\pgfqpoint{1.016442in}{1.420504in}}%
\pgfpathlineto{\pgfqpoint{1.017437in}{1.421274in}}%
\pgfpathlineto{\pgfqpoint{1.019890in}{1.422324in}}%
\pgfpathlineto{\pgfqpoint{1.020852in}{1.423444in}}%
\pgfpathlineto{\pgfqpoint{1.022626in}{1.424494in}}%
\pgfpathlineto{\pgfqpoint{1.023488in}{1.425334in}}%
\pgfpathlineto{\pgfqpoint{1.026405in}{1.426384in}}%
\pgfpathlineto{\pgfqpoint{1.027516in}{1.427224in}}%
\pgfpathlineto{\pgfqpoint{1.029787in}{1.428274in}}%
\pgfpathlineto{\pgfqpoint{1.030831in}{1.429184in}}%
\pgfpathlineto{\pgfqpoint{1.032887in}{1.430234in}}%
\pgfpathlineto{\pgfqpoint{1.033882in}{1.431074in}}%
\pgfpathlineto{\pgfqpoint{1.036667in}{1.432124in}}%
\pgfpathlineto{\pgfqpoint{1.037711in}{1.432894in}}%
\pgfpathlineto{\pgfqpoint{1.041872in}{1.433944in}}%
\pgfpathlineto{\pgfqpoint{1.042834in}{1.434504in}}%
\pgfpathlineto{\pgfqpoint{1.046530in}{1.435554in}}%
\pgfpathlineto{\pgfqpoint{1.047575in}{1.436044in}}%
\pgfpathlineto{\pgfqpoint{1.049514in}{1.437094in}}%
\pgfpathlineto{\pgfqpoint{1.050625in}{1.437794in}}%
\pgfpathlineto{\pgfqpoint{1.053095in}{1.438844in}}%
\pgfpathlineto{\pgfqpoint{1.054090in}{1.439544in}}%
\pgfpathlineto{\pgfqpoint{1.056692in}{1.440594in}}%
\pgfpathlineto{\pgfqpoint{1.057704in}{1.441364in}}%
\pgfpathlineto{\pgfqpoint{1.061301in}{1.442414in}}%
\pgfpathlineto{\pgfqpoint{1.062329in}{1.443044in}}%
\pgfpathlineto{\pgfqpoint{1.065893in}{1.444094in}}%
\pgfpathlineto{\pgfqpoint{1.066987in}{1.444584in}}%
\pgfpathlineto{\pgfqpoint{1.068976in}{1.445634in}}%
\pgfpathlineto{\pgfqpoint{1.070037in}{1.446544in}}%
\pgfpathlineto{\pgfqpoint{1.072491in}{1.447594in}}%
\pgfpathlineto{\pgfqpoint{1.073369in}{1.448224in}}%
\pgfpathlineto{\pgfqpoint{1.075840in}{1.449274in}}%
\pgfpathlineto{\pgfqpoint{1.076652in}{1.449694in}}%
\pgfpathlineto{\pgfqpoint{1.079006in}{1.450744in}}%
\pgfpathlineto{\pgfqpoint{1.079967in}{1.451234in}}%
\pgfpathlineto{\pgfqpoint{1.083001in}{1.452214in}}%
\pgfpathlineto{\pgfqpoint{1.084112in}{1.452914in}}%
\pgfpathlineto{\pgfqpoint{1.087726in}{1.453894in}}%
\pgfpathlineto{\pgfqpoint{1.088438in}{1.454384in}}%
\pgfpathlineto{\pgfqpoint{1.092119in}{1.455434in}}%
\pgfpathlineto{\pgfqpoint{1.093064in}{1.456064in}}%
\pgfpathlineto{\pgfqpoint{1.096346in}{1.457114in}}%
\pgfpathlineto{\pgfqpoint{1.097274in}{1.457674in}}%
\pgfpathlineto{\pgfqpoint{1.100092in}{1.458724in}}%
\pgfpathlineto{\pgfqpoint{1.100689in}{1.459004in}}%
\pgfpathlineto{\pgfqpoint{1.104038in}{1.460054in}}%
\pgfpathlineto{\pgfqpoint{1.105149in}{1.460404in}}%
\pgfpathlineto{\pgfqpoint{1.109989in}{1.461454in}}%
\pgfpathlineto{\pgfqpoint{1.111083in}{1.462154in}}%
\pgfpathlineto{\pgfqpoint{1.114813in}{1.463204in}}%
\pgfpathlineto{\pgfqpoint{1.115858in}{1.463624in}}%
\pgfpathlineto{\pgfqpoint{1.119389in}{1.464674in}}%
\pgfpathlineto{\pgfqpoint{1.120433in}{1.465094in}}%
\pgfpathlineto{\pgfqpoint{1.122754in}{1.466144in}}%
\pgfpathlineto{\pgfqpoint{1.123732in}{1.466494in}}%
\pgfpathlineto{\pgfqpoint{1.127910in}{1.467544in}}%
\pgfpathlineto{\pgfqpoint{1.128556in}{1.468034in}}%
\pgfpathlineto{\pgfqpoint{1.131010in}{1.469084in}}%
\pgfpathlineto{\pgfqpoint{1.131822in}{1.469924in}}%
\pgfpathlineto{\pgfqpoint{1.135187in}{1.470974in}}%
\pgfpathlineto{\pgfqpoint{1.135204in}{1.471184in}}%
\pgfpathlineto{\pgfqpoint{1.140956in}{1.472234in}}%
\pgfpathlineto{\pgfqpoint{1.141735in}{1.472584in}}%
\pgfpathlineto{\pgfqpoint{1.145598in}{1.473634in}}%
\pgfpathlineto{\pgfqpoint{1.146675in}{1.474194in}}%
\pgfpathlineto{\pgfqpoint{1.150803in}{1.475244in}}%
\pgfpathlineto{\pgfqpoint{1.151516in}{1.475874in}}%
\pgfpathlineto{\pgfqpoint{1.156058in}{1.476924in}}%
\pgfpathlineto{\pgfqpoint{1.157152in}{1.477414in}}%
\pgfpathlineto{\pgfqpoint{1.162407in}{1.478464in}}%
\pgfpathlineto{\pgfqpoint{1.163452in}{1.479094in}}%
\pgfpathlineto{\pgfqpoint{1.168060in}{1.480144in}}%
\pgfpathlineto{\pgfqpoint{1.168873in}{1.480494in}}%
\pgfpathlineto{\pgfqpoint{1.171923in}{1.481474in}}%
\pgfpathlineto{\pgfqpoint{1.172835in}{1.481754in}}%
\pgfpathlineto{\pgfqpoint{1.177161in}{1.482804in}}%
\pgfpathlineto{\pgfqpoint{1.178156in}{1.483294in}}%
\pgfpathlineto{\pgfqpoint{1.181223in}{1.484344in}}%
\pgfpathlineto{\pgfqpoint{1.181273in}{1.484554in}}%
\pgfpathlineto{\pgfqpoint{1.188633in}{1.485604in}}%
\pgfpathlineto{\pgfqpoint{1.189512in}{1.485884in}}%
\pgfpathlineto{\pgfqpoint{1.193341in}{1.486934in}}%
\pgfpathlineto{\pgfqpoint{1.194203in}{1.487354in}}%
\pgfpathlineto{\pgfqpoint{1.198928in}{1.488404in}}%
\pgfpathlineto{\pgfqpoint{1.199823in}{1.488964in}}%
\pgfpathlineto{\pgfqpoint{1.201895in}{1.489944in}}%
\pgfpathlineto{\pgfqpoint{1.202343in}{1.490224in}}%
\pgfpathlineto{\pgfqpoint{1.205343in}{1.491274in}}%
\pgfpathlineto{\pgfqpoint{1.206404in}{1.491554in}}%
\pgfpathlineto{\pgfqpoint{1.209670in}{1.492604in}}%
\pgfpathlineto{\pgfqpoint{1.210748in}{1.493024in}}%
\pgfpathlineto{\pgfqpoint{1.215887in}{1.494074in}}%
\pgfpathlineto{\pgfqpoint{1.216865in}{1.494214in}}%
\pgfpathlineto{\pgfqpoint{1.223860in}{1.495264in}}%
\pgfpathlineto{\pgfqpoint{1.224739in}{1.495684in}}%
\pgfpathlineto{\pgfqpoint{1.229994in}{1.496734in}}%
\pgfpathlineto{\pgfqpoint{1.230823in}{1.497014in}}%
\pgfpathlineto{\pgfqpoint{1.234354in}{1.498064in}}%
\pgfpathlineto{\pgfqpoint{1.235117in}{1.498344in}}%
\pgfpathlineto{\pgfqpoint{1.240256in}{1.499394in}}%
\pgfpathlineto{\pgfqpoint{1.241151in}{1.499674in}}%
\pgfpathlineto{\pgfqpoint{1.245610in}{1.500724in}}%
\pgfpathlineto{\pgfqpoint{1.246721in}{1.501144in}}%
\pgfpathlineto{\pgfqpoint{1.248942in}{1.502194in}}%
\pgfpathlineto{\pgfqpoint{1.250053in}{1.502754in}}%
\pgfpathlineto{\pgfqpoint{1.254429in}{1.503804in}}%
\pgfpathlineto{\pgfqpoint{1.254728in}{1.504084in}}%
\pgfpathlineto{\pgfqpoint{1.260845in}{1.505134in}}%
\pgfpathlineto{\pgfqpoint{1.261723in}{1.505414in}}%
\pgfpathlineto{\pgfqpoint{1.266697in}{1.506464in}}%
\pgfpathlineto{\pgfqpoint{1.267807in}{1.506744in}}%
\pgfpathlineto{\pgfqpoint{1.271338in}{1.507724in}}%
\pgfpathlineto{\pgfqpoint{1.272200in}{1.508214in}}%
\pgfpathlineto{\pgfqpoint{1.277936in}{1.509264in}}%
\pgfpathlineto{\pgfqpoint{1.278036in}{1.509404in}}%
\pgfpathlineto{\pgfqpoint{1.284203in}{1.510454in}}%
\pgfpathlineto{\pgfqpoint{1.285131in}{1.510804in}}%
\pgfpathlineto{\pgfqpoint{1.289657in}{1.511854in}}%
\pgfpathlineto{\pgfqpoint{1.290585in}{1.512274in}}%
\pgfpathlineto{\pgfqpoint{1.295940in}{1.513324in}}%
\pgfpathlineto{\pgfqpoint{1.296354in}{1.513534in}}%
\pgfpathlineto{\pgfqpoint{1.299703in}{1.514584in}}%
\pgfpathlineto{\pgfqpoint{1.300664in}{1.514934in}}%
\pgfpathlineto{\pgfqpoint{1.305389in}{1.515984in}}%
\pgfpathlineto{\pgfqpoint{1.306118in}{1.516334in}}%
\pgfpathlineto{\pgfqpoint{1.311290in}{1.517384in}}%
\pgfpathlineto{\pgfqpoint{1.312318in}{1.517734in}}%
\pgfpathlineto{\pgfqpoint{1.318104in}{1.518784in}}%
\pgfpathlineto{\pgfqpoint{1.318899in}{1.519274in}}%
\pgfpathlineto{\pgfqpoint{1.323873in}{1.520324in}}%
\pgfpathlineto{\pgfqpoint{1.324950in}{1.520674in}}%
\pgfpathlineto{\pgfqpoint{1.329575in}{1.521654in}}%
\pgfpathlineto{\pgfqpoint{1.330305in}{1.521864in}}%
\pgfpathlineto{\pgfqpoint{1.336521in}{1.522844in}}%
\pgfpathlineto{\pgfqpoint{1.336836in}{1.523194in}}%
\pgfpathlineto{\pgfqpoint{1.344180in}{1.524244in}}%
\pgfpathlineto{\pgfqpoint{1.345191in}{1.524524in}}%
\pgfpathlineto{\pgfqpoint{1.351905in}{1.525574in}}%
\pgfpathlineto{\pgfqpoint{1.352602in}{1.525994in}}%
\pgfpathlineto{\pgfqpoint{1.359050in}{1.527044in}}%
\pgfpathlineto{\pgfqpoint{1.359962in}{1.527394in}}%
\pgfpathlineto{\pgfqpoint{1.365897in}{1.528444in}}%
\pgfpathlineto{\pgfqpoint{1.366991in}{1.528654in}}%
\pgfpathlineto{\pgfqpoint{1.378463in}{1.529704in}}%
\pgfpathlineto{\pgfqpoint{1.379092in}{1.530054in}}%
\pgfpathlineto{\pgfqpoint{1.386221in}{1.531104in}}%
\pgfpathlineto{\pgfqpoint{1.387298in}{1.531314in}}%
\pgfpathlineto{\pgfqpoint{1.393366in}{1.532364in}}%
\pgfpathlineto{\pgfqpoint{1.393880in}{1.532714in}}%
\pgfpathlineto{\pgfqpoint{1.403760in}{1.533764in}}%
\pgfpathlineto{\pgfqpoint{1.404340in}{1.533904in}}%
\pgfpathlineto{\pgfqpoint{1.412529in}{1.534954in}}%
\pgfpathlineto{\pgfqpoint{1.412994in}{1.535164in}}%
\pgfpathlineto{\pgfqpoint{1.422277in}{1.536214in}}%
\pgfpathlineto{\pgfqpoint{1.422807in}{1.536354in}}%
\pgfpathlineto{\pgfqpoint{1.428460in}{1.537404in}}%
\pgfpathlineto{\pgfqpoint{1.429157in}{1.537614in}}%
\pgfpathlineto{\pgfqpoint{1.435340in}{1.538664in}}%
\pgfpathlineto{\pgfqpoint{1.436351in}{1.539014in}}%
\pgfpathlineto{\pgfqpoint{1.443861in}{1.540064in}}%
\pgfpathlineto{\pgfqpoint{1.443877in}{1.540274in}}%
\pgfpathlineto{\pgfqpoint{1.450940in}{1.541324in}}%
\pgfpathlineto{\pgfqpoint{1.451006in}{1.541464in}}%
\pgfpathlineto{\pgfqpoint{1.462776in}{1.542514in}}%
\pgfpathlineto{\pgfqpoint{1.463439in}{1.542724in}}%
\pgfpathlineto{\pgfqpoint{1.473817in}{1.543774in}}%
\pgfpathlineto{\pgfqpoint{1.474347in}{1.544124in}}%
\pgfpathlineto{\pgfqpoint{1.482304in}{1.545174in}}%
\pgfpathlineto{\pgfqpoint{1.482669in}{1.545314in}}%
\pgfpathlineto{\pgfqpoint{1.493892in}{1.546364in}}%
\pgfpathlineto{\pgfqpoint{1.494257in}{1.546504in}}%
\pgfpathlineto{\pgfqpoint{1.503756in}{1.547554in}}%
\pgfpathlineto{\pgfqpoint{1.503921in}{1.547834in}}%
\pgfpathlineto{\pgfqpoint{1.514382in}{1.548884in}}%
\pgfpathlineto{\pgfqpoint{1.515227in}{1.549234in}}%
\pgfpathlineto{\pgfqpoint{1.524046in}{1.550284in}}%
\pgfpathlineto{\pgfqpoint{1.524378in}{1.550564in}}%
\pgfpathlineto{\pgfqpoint{1.535187in}{1.551614in}}%
\pgfpathlineto{\pgfqpoint{1.535817in}{1.551754in}}%
\pgfpathlineto{\pgfqpoint{1.546012in}{1.552804in}}%
\pgfpathlineto{\pgfqpoint{1.547122in}{1.553154in}}%
\pgfpathlineto{\pgfqpoint{1.557450in}{1.554204in}}%
\pgfpathlineto{\pgfqpoint{1.557450in}{1.554274in}}%
\pgfpathlineto{\pgfqpoint{1.571690in}{1.555324in}}%
\pgfpathlineto{\pgfqpoint{1.571790in}{1.555534in}}%
\pgfpathlineto{\pgfqpoint{1.581405in}{1.556584in}}%
\pgfpathlineto{\pgfqpoint{1.581653in}{1.556794in}}%
\pgfpathlineto{\pgfqpoint{1.595396in}{1.557844in}}%
\pgfpathlineto{\pgfqpoint{1.595844in}{1.558054in}}%
\pgfpathlineto{\pgfqpoint{1.606503in}{1.559104in}}%
\pgfpathlineto{\pgfqpoint{1.607514in}{1.559384in}}%
\pgfpathlineto{\pgfqpoint{1.617610in}{1.560434in}}%
\pgfpathlineto{\pgfqpoint{1.618721in}{1.560644in}}%
\pgfpathlineto{\pgfqpoint{1.630889in}{1.561694in}}%
\pgfpathlineto{\pgfqpoint{1.631883in}{1.561834in}}%
\pgfpathlineto{\pgfqpoint{1.652605in}{1.562884in}}%
\pgfpathlineto{\pgfqpoint{1.653600in}{1.563024in}}%
\pgfpathlineto{\pgfqpoint{1.663812in}{1.564074in}}%
\pgfpathlineto{\pgfqpoint{1.663861in}{1.564214in}}%
\pgfpathlineto{\pgfqpoint{1.672283in}{1.565264in}}%
\pgfpathlineto{\pgfqpoint{1.672382in}{1.565404in}}%
\pgfpathlineto{\pgfqpoint{1.683406in}{1.566454in}}%
\pgfpathlineto{\pgfqpoint{1.684467in}{1.566664in}}%
\pgfpathlineto{\pgfqpoint{1.696586in}{1.567714in}}%
\pgfpathlineto{\pgfqpoint{1.696834in}{1.567854in}}%
\pgfpathlineto{\pgfqpoint{1.710212in}{1.568904in}}%
\pgfpathlineto{\pgfqpoint{1.710909in}{1.569114in}}%
\pgfpathlineto{\pgfqpoint{1.723706in}{1.570164in}}%
\pgfpathlineto{\pgfqpoint{1.724718in}{1.570304in}}%
\pgfpathlineto{\pgfqpoint{1.739090in}{1.571354in}}%
\pgfpathlineto{\pgfqpoint{1.739422in}{1.571494in}}%
\pgfpathlineto{\pgfqpoint{1.756828in}{1.572544in}}%
\pgfpathlineto{\pgfqpoint{1.757591in}{1.572754in}}%
\pgfpathlineto{\pgfqpoint{1.769825in}{1.573804in}}%
\pgfpathlineto{\pgfqpoint{1.770339in}{1.573944in}}%
\pgfpathlineto{\pgfqpoint{1.782242in}{1.574994in}}%
\pgfpathlineto{\pgfqpoint{1.782374in}{1.575134in}}%
\pgfpathlineto{\pgfqpoint{1.795288in}{1.576184in}}%
\pgfpathlineto{\pgfqpoint{1.795868in}{1.576394in}}%
\pgfpathlineto{\pgfqpoint{1.809910in}{1.577444in}}%
\pgfpathlineto{\pgfqpoint{1.810987in}{1.577724in}}%
\pgfpathlineto{\pgfqpoint{1.828360in}{1.578774in}}%
\pgfpathlineto{\pgfqpoint{1.829057in}{1.578984in}}%
\pgfpathlineto{\pgfqpoint{1.842236in}{1.580034in}}%
\pgfpathlineto{\pgfqpoint{1.842368in}{1.580174in}}%
\pgfpathlineto{\pgfqpoint{1.859410in}{1.581224in}}%
\pgfpathlineto{\pgfqpoint{1.859891in}{1.581364in}}%
\pgfpathlineto{\pgfqpoint{1.873683in}{1.582414in}}%
\pgfpathlineto{\pgfqpoint{1.873683in}{1.582484in}}%
\pgfpathlineto{\pgfqpoint{1.894190in}{1.583534in}}%
\pgfpathlineto{\pgfqpoint{1.894637in}{1.583674in}}%
\pgfpathlineto{\pgfqpoint{1.915542in}{1.584724in}}%
\pgfpathlineto{\pgfqpoint{1.916155in}{1.584864in}}%
\pgfpathlineto{\pgfqpoint{1.929334in}{1.585914in}}%
\pgfpathlineto{\pgfqpoint{1.930345in}{1.586054in}}%
\pgfpathlineto{\pgfqpoint{1.945680in}{1.587104in}}%
\pgfpathlineto{\pgfqpoint{1.945680in}{1.587174in}}%
\pgfpathlineto{\pgfqpoint{1.958245in}{1.588224in}}%
\pgfpathlineto{\pgfqpoint{1.959074in}{1.588434in}}%
\pgfpathlineto{\pgfqpoint{1.971541in}{1.589484in}}%
\pgfpathlineto{\pgfqpoint{1.972005in}{1.589624in}}%
\pgfpathlineto{\pgfqpoint{1.980807in}{1.590674in}}%
\pgfpathlineto{\pgfqpoint{1.981636in}{1.590954in}}%
\pgfpathlineto{\pgfqpoint{1.988947in}{1.592004in}}%
\pgfpathlineto{\pgfqpoint{1.988947in}{1.592074in}}%
\pgfpathlineto{\pgfqpoint{1.998313in}{1.593124in}}%
\pgfpathlineto{\pgfqpoint{1.999391in}{1.593334in}}%
\pgfpathlineto{\pgfqpoint{2.006536in}{1.594384in}}%
\pgfpathlineto{\pgfqpoint{2.007414in}{1.594524in}}%
\pgfpathlineto{\pgfqpoint{2.013349in}{1.595574in}}%
\pgfpathlineto{\pgfqpoint{2.014195in}{1.595854in}}%
\pgfpathlineto{\pgfqpoint{2.021936in}{1.596904in}}%
\pgfpathlineto{\pgfqpoint{2.022318in}{1.597044in}}%
\pgfpathlineto{\pgfqpoint{2.028136in}{1.598094in}}%
\pgfpathlineto{\pgfqpoint{2.029230in}{1.598584in}}%
\pgfpathlineto{\pgfqpoint{2.032015in}{1.599634in}}%
\pgfpathlineto{\pgfqpoint{2.033126in}{1.601944in}}%
\pgfpathlineto{\pgfqpoint{2.033126in}{1.601944in}}%
\pgfusepath{stroke}%
\end{pgfscope}%
\begin{pgfscope}%
\pgfsetrectcap%
\pgfsetmiterjoin%
\pgfsetlinewidth{0.803000pt}%
\definecolor{currentstroke}{rgb}{0.000000,0.000000,0.000000}%
\pgfsetstrokecolor{currentstroke}%
\pgfsetdash{}{0pt}%
\pgfpathmoveto{\pgfqpoint{0.553581in}{0.499444in}}%
\pgfpathlineto{\pgfqpoint{0.553581in}{1.654444in}}%
\pgfusepath{stroke}%
\end{pgfscope}%
\begin{pgfscope}%
\pgfsetrectcap%
\pgfsetmiterjoin%
\pgfsetlinewidth{0.803000pt}%
\definecolor{currentstroke}{rgb}{0.000000,0.000000,0.000000}%
\pgfsetstrokecolor{currentstroke}%
\pgfsetdash{}{0pt}%
\pgfpathmoveto{\pgfqpoint{2.103581in}{0.499444in}}%
\pgfpathlineto{\pgfqpoint{2.103581in}{1.654444in}}%
\pgfusepath{stroke}%
\end{pgfscope}%
\begin{pgfscope}%
\pgfsetrectcap%
\pgfsetmiterjoin%
\pgfsetlinewidth{0.803000pt}%
\definecolor{currentstroke}{rgb}{0.000000,0.000000,0.000000}%
\pgfsetstrokecolor{currentstroke}%
\pgfsetdash{}{0pt}%
\pgfpathmoveto{\pgfqpoint{0.553581in}{0.499444in}}%
\pgfpathlineto{\pgfqpoint{2.103581in}{0.499444in}}%
\pgfusepath{stroke}%
\end{pgfscope}%
\begin{pgfscope}%
\pgfsetrectcap%
\pgfsetmiterjoin%
\pgfsetlinewidth{0.803000pt}%
\definecolor{currentstroke}{rgb}{0.000000,0.000000,0.000000}%
\pgfsetstrokecolor{currentstroke}%
\pgfsetdash{}{0pt}%
\pgfpathmoveto{\pgfqpoint{0.553581in}{1.654444in}}%
\pgfpathlineto{\pgfqpoint{2.103581in}{1.654444in}}%
\pgfusepath{stroke}%
\end{pgfscope}%
\begin{pgfscope}%
\pgfsetbuttcap%
\pgfsetmiterjoin%
\definecolor{currentfill}{rgb}{1.000000,1.000000,1.000000}%
\pgfsetfillcolor{currentfill}%
\pgfsetfillopacity{0.800000}%
\pgfsetlinewidth{1.003750pt}%
\definecolor{currentstroke}{rgb}{0.800000,0.800000,0.800000}%
\pgfsetstrokecolor{currentstroke}%
\pgfsetstrokeopacity{0.800000}%
\pgfsetdash{}{0pt}%
\pgfpathmoveto{\pgfqpoint{0.832747in}{0.568889in}}%
\pgfpathlineto{\pgfqpoint{2.006358in}{0.568889in}}%
\pgfpathquadraticcurveto{\pgfqpoint{2.034136in}{0.568889in}}{\pgfqpoint{2.034136in}{0.596666in}}%
\pgfpathlineto{\pgfqpoint{2.034136in}{0.776388in}}%
\pgfpathquadraticcurveto{\pgfqpoint{2.034136in}{0.804166in}}{\pgfqpoint{2.006358in}{0.804166in}}%
\pgfpathlineto{\pgfqpoint{0.832747in}{0.804166in}}%
\pgfpathquadraticcurveto{\pgfqpoint{0.804970in}{0.804166in}}{\pgfqpoint{0.804970in}{0.776388in}}%
\pgfpathlineto{\pgfqpoint{0.804970in}{0.596666in}}%
\pgfpathquadraticcurveto{\pgfqpoint{0.804970in}{0.568889in}}{\pgfqpoint{0.832747in}{0.568889in}}%
\pgfpathlineto{\pgfqpoint{0.832747in}{0.568889in}}%
\pgfpathclose%
\pgfusepath{stroke,fill}%
\end{pgfscope}%
\begin{pgfscope}%
\pgfsetrectcap%
\pgfsetroundjoin%
\pgfsetlinewidth{1.505625pt}%
\definecolor{currentstroke}{rgb}{0.000000,0.000000,0.000000}%
\pgfsetstrokecolor{currentstroke}%
\pgfsetdash{}{0pt}%
\pgfpathmoveto{\pgfqpoint{0.860525in}{0.700000in}}%
\pgfpathlineto{\pgfqpoint{0.999414in}{0.700000in}}%
\pgfpathlineto{\pgfqpoint{1.138303in}{0.700000in}}%
\pgfusepath{stroke}%
\end{pgfscope}%
\begin{pgfscope}%
\definecolor{textcolor}{rgb}{0.000000,0.000000,0.000000}%
\pgfsetstrokecolor{textcolor}%
\pgfsetfillcolor{textcolor}%
\pgftext[x=1.249414in,y=0.651388in,left,base]{\color{textcolor}\rmfamily\fontsize{10.000000}{12.000000}\selectfont AUC=0.840}%
\end{pgfscope}%
\end{pgfpicture}%
\makeatother%
\endgroup%

	
&
	\vskip 0pt
	\begin{tabular}{cc|c|c|}
	&\multicolumn{1}{c}{}& \multicolumn{2}{c}{Prediction} \\[0.4em]
	&\multicolumn{1}{c}{} & \multicolumn{1}{c}{N} & \multicolumn{1}{c}{P} \cr\cline{3-4}
	\multirow{2}{*}{\rotatebox[origin=c]{90}{Actual}}&N &
66.3\% & 18.7\%
	\vrule width 0pt height 10pt depth 2pt \cr\cline{3-4}
	&P & 
3.30\% & 11.7\%
	\vrule width 0pt height 10pt depth 2pt \cr\cline{3-4}
	\end{tabular}
% 0.839808121	
%66.303	18.69	3.299	11.707	
% 0.780102823 	0.385141776	0.780135983	0.515686041

	\hfil\begin{tabular}{ll}
	\cr
	0.780 & Accuracy\cr
	0.385 & Precision \cr
	0.780 & Recall \cr
	0.516 & F1 \cr
	0.840 & AUC \cr
\end{tabular}
\cr
\end{tabular}
} % End parbox


\

{\bf Linearly Transform the $p$ Values}, mapping the min to 0.0 and the max to 1.0.  If there are far outliers, perhaps map the 1\% quantile to 0 and the 99\% quantile to 1.0.  This transformation is especially useful for visualizing the data.  

\

\verb|Ideal_Left_Linear_Transform|


%%%
\parbox{\linewidth}{
\noindent\begin{tabular}{@{\hspace{-6pt}}p{2.3in} @{\hspace{-6pt}}p{2.0in} p{1.8in}}
	\vskip 0pt
	\qquad \qquad Raw Model Output
	
	%% Creator: Matplotlib, PGF backend
%%
%% To include the figure in your LaTeX document, write
%%   \input{<filename>.pgf}
%%
%% Make sure the required packages are loaded in your preamble
%%   \usepackage{pgf}
%%
%% Also ensure that all the required font packages are loaded; for instance,
%% the lmodern package is sometimes necessary when using math font.
%%   \usepackage{lmodern}
%%
%% Figures using additional raster images can only be included by \input if
%% they are in the same directory as the main LaTeX file. For loading figures
%% from other directories you can use the `import` package
%%   \usepackage{import}
%%
%% and then include the figures with
%%   \import{<path to file>}{<filename>.pgf}
%%
%% Matplotlib used the following preamble
%%   
%%   \usepackage{fontspec}
%%   \makeatletter\@ifpackageloaded{underscore}{}{\usepackage[strings]{underscore}}\makeatother
%%
\begingroup%
\makeatletter%
\begin{pgfpicture}%
\pgfpathrectangle{\pgfpointorigin}{\pgfqpoint{2.153750in}{1.654444in}}%
\pgfusepath{use as bounding box, clip}%
\begin{pgfscope}%
\pgfsetbuttcap%
\pgfsetmiterjoin%
\definecolor{currentfill}{rgb}{1.000000,1.000000,1.000000}%
\pgfsetfillcolor{currentfill}%
\pgfsetlinewidth{0.000000pt}%
\definecolor{currentstroke}{rgb}{1.000000,1.000000,1.000000}%
\pgfsetstrokecolor{currentstroke}%
\pgfsetdash{}{0pt}%
\pgfpathmoveto{\pgfqpoint{0.000000in}{0.000000in}}%
\pgfpathlineto{\pgfqpoint{2.153750in}{0.000000in}}%
\pgfpathlineto{\pgfqpoint{2.153750in}{1.654444in}}%
\pgfpathlineto{\pgfqpoint{0.000000in}{1.654444in}}%
\pgfpathlineto{\pgfqpoint{0.000000in}{0.000000in}}%
\pgfpathclose%
\pgfusepath{fill}%
\end{pgfscope}%
\begin{pgfscope}%
\pgfsetbuttcap%
\pgfsetmiterjoin%
\definecolor{currentfill}{rgb}{1.000000,1.000000,1.000000}%
\pgfsetfillcolor{currentfill}%
\pgfsetlinewidth{0.000000pt}%
\definecolor{currentstroke}{rgb}{0.000000,0.000000,0.000000}%
\pgfsetstrokecolor{currentstroke}%
\pgfsetstrokeopacity{0.000000}%
\pgfsetdash{}{0pt}%
\pgfpathmoveto{\pgfqpoint{0.465000in}{0.449444in}}%
\pgfpathlineto{\pgfqpoint{2.015000in}{0.449444in}}%
\pgfpathlineto{\pgfqpoint{2.015000in}{1.604444in}}%
\pgfpathlineto{\pgfqpoint{0.465000in}{1.604444in}}%
\pgfpathlineto{\pgfqpoint{0.465000in}{0.449444in}}%
\pgfpathclose%
\pgfusepath{fill}%
\end{pgfscope}%
\begin{pgfscope}%
\pgfpathrectangle{\pgfqpoint{0.465000in}{0.449444in}}{\pgfqpoint{1.550000in}{1.155000in}}%
\pgfusepath{clip}%
\pgfsetbuttcap%
\pgfsetmiterjoin%
\pgfsetlinewidth{1.003750pt}%
\definecolor{currentstroke}{rgb}{0.000000,0.000000,0.000000}%
\pgfsetstrokecolor{currentstroke}%
\pgfsetdash{}{0pt}%
\pgfpathmoveto{\pgfqpoint{0.455000in}{0.449444in}}%
\pgfpathlineto{\pgfqpoint{0.502805in}{0.449444in}}%
\pgfpathlineto{\pgfqpoint{0.502805in}{1.002391in}}%
\pgfpathlineto{\pgfqpoint{0.455000in}{1.002391in}}%
\pgfusepath{stroke}%
\end{pgfscope}%
\begin{pgfscope}%
\pgfpathrectangle{\pgfqpoint{0.465000in}{0.449444in}}{\pgfqpoint{1.550000in}{1.155000in}}%
\pgfusepath{clip}%
\pgfsetbuttcap%
\pgfsetmiterjoin%
\pgfsetlinewidth{1.003750pt}%
\definecolor{currentstroke}{rgb}{0.000000,0.000000,0.000000}%
\pgfsetstrokecolor{currentstroke}%
\pgfsetdash{}{0pt}%
\pgfpathmoveto{\pgfqpoint{0.593537in}{0.449444in}}%
\pgfpathlineto{\pgfqpoint{0.654025in}{0.449444in}}%
\pgfpathlineto{\pgfqpoint{0.654025in}{1.463667in}}%
\pgfpathlineto{\pgfqpoint{0.593537in}{1.463667in}}%
\pgfpathlineto{\pgfqpoint{0.593537in}{0.449444in}}%
\pgfpathclose%
\pgfusepath{stroke}%
\end{pgfscope}%
\begin{pgfscope}%
\pgfpathrectangle{\pgfqpoint{0.465000in}{0.449444in}}{\pgfqpoint{1.550000in}{1.155000in}}%
\pgfusepath{clip}%
\pgfsetbuttcap%
\pgfsetmiterjoin%
\pgfsetlinewidth{1.003750pt}%
\definecolor{currentstroke}{rgb}{0.000000,0.000000,0.000000}%
\pgfsetstrokecolor{currentstroke}%
\pgfsetdash{}{0pt}%
\pgfpathmoveto{\pgfqpoint{0.744756in}{0.449444in}}%
\pgfpathlineto{\pgfqpoint{0.805244in}{0.449444in}}%
\pgfpathlineto{\pgfqpoint{0.805244in}{1.549444in}}%
\pgfpathlineto{\pgfqpoint{0.744756in}{1.549444in}}%
\pgfpathlineto{\pgfqpoint{0.744756in}{0.449444in}}%
\pgfpathclose%
\pgfusepath{stroke}%
\end{pgfscope}%
\begin{pgfscope}%
\pgfpathrectangle{\pgfqpoint{0.465000in}{0.449444in}}{\pgfqpoint{1.550000in}{1.155000in}}%
\pgfusepath{clip}%
\pgfsetbuttcap%
\pgfsetmiterjoin%
\pgfsetlinewidth{1.003750pt}%
\definecolor{currentstroke}{rgb}{0.000000,0.000000,0.000000}%
\pgfsetstrokecolor{currentstroke}%
\pgfsetdash{}{0pt}%
\pgfpathmoveto{\pgfqpoint{0.895976in}{0.449444in}}%
\pgfpathlineto{\pgfqpoint{0.956464in}{0.449444in}}%
\pgfpathlineto{\pgfqpoint{0.956464in}{1.389648in}}%
\pgfpathlineto{\pgfqpoint{0.895976in}{1.389648in}}%
\pgfpathlineto{\pgfqpoint{0.895976in}{0.449444in}}%
\pgfpathclose%
\pgfusepath{stroke}%
\end{pgfscope}%
\begin{pgfscope}%
\pgfpathrectangle{\pgfqpoint{0.465000in}{0.449444in}}{\pgfqpoint{1.550000in}{1.155000in}}%
\pgfusepath{clip}%
\pgfsetbuttcap%
\pgfsetmiterjoin%
\pgfsetlinewidth{1.003750pt}%
\definecolor{currentstroke}{rgb}{0.000000,0.000000,0.000000}%
\pgfsetstrokecolor{currentstroke}%
\pgfsetdash{}{0pt}%
\pgfpathmoveto{\pgfqpoint{1.047195in}{0.449444in}}%
\pgfpathlineto{\pgfqpoint{1.107683in}{0.449444in}}%
\pgfpathlineto{\pgfqpoint{1.107683in}{1.137712in}}%
\pgfpathlineto{\pgfqpoint{1.047195in}{1.137712in}}%
\pgfpathlineto{\pgfqpoint{1.047195in}{0.449444in}}%
\pgfpathclose%
\pgfusepath{stroke}%
\end{pgfscope}%
\begin{pgfscope}%
\pgfpathrectangle{\pgfqpoint{0.465000in}{0.449444in}}{\pgfqpoint{1.550000in}{1.155000in}}%
\pgfusepath{clip}%
\pgfsetbuttcap%
\pgfsetmiterjoin%
\pgfsetlinewidth{1.003750pt}%
\definecolor{currentstroke}{rgb}{0.000000,0.000000,0.000000}%
\pgfsetstrokecolor{currentstroke}%
\pgfsetdash{}{0pt}%
\pgfpathmoveto{\pgfqpoint{1.198415in}{0.449444in}}%
\pgfpathlineto{\pgfqpoint{1.258903in}{0.449444in}}%
\pgfpathlineto{\pgfqpoint{1.258903in}{0.903221in}}%
\pgfpathlineto{\pgfqpoint{1.198415in}{0.903221in}}%
\pgfpathlineto{\pgfqpoint{1.198415in}{0.449444in}}%
\pgfpathclose%
\pgfusepath{stroke}%
\end{pgfscope}%
\begin{pgfscope}%
\pgfpathrectangle{\pgfqpoint{0.465000in}{0.449444in}}{\pgfqpoint{1.550000in}{1.155000in}}%
\pgfusepath{clip}%
\pgfsetbuttcap%
\pgfsetmiterjoin%
\pgfsetlinewidth{1.003750pt}%
\definecolor{currentstroke}{rgb}{0.000000,0.000000,0.000000}%
\pgfsetstrokecolor{currentstroke}%
\pgfsetdash{}{0pt}%
\pgfpathmoveto{\pgfqpoint{1.349634in}{0.449444in}}%
\pgfpathlineto{\pgfqpoint{1.410122in}{0.449444in}}%
\pgfpathlineto{\pgfqpoint{1.410122in}{0.775146in}}%
\pgfpathlineto{\pgfqpoint{1.349634in}{0.775146in}}%
\pgfpathlineto{\pgfqpoint{1.349634in}{0.449444in}}%
\pgfpathclose%
\pgfusepath{stroke}%
\end{pgfscope}%
\begin{pgfscope}%
\pgfpathrectangle{\pgfqpoint{0.465000in}{0.449444in}}{\pgfqpoint{1.550000in}{1.155000in}}%
\pgfusepath{clip}%
\pgfsetbuttcap%
\pgfsetmiterjoin%
\pgfsetlinewidth{1.003750pt}%
\definecolor{currentstroke}{rgb}{0.000000,0.000000,0.000000}%
\pgfsetstrokecolor{currentstroke}%
\pgfsetdash{}{0pt}%
\pgfpathmoveto{\pgfqpoint{1.500854in}{0.449444in}}%
\pgfpathlineto{\pgfqpoint{1.561342in}{0.449444in}}%
\pgfpathlineto{\pgfqpoint{1.561342in}{0.613798in}}%
\pgfpathlineto{\pgfqpoint{1.500854in}{0.613798in}}%
\pgfpathlineto{\pgfqpoint{1.500854in}{0.449444in}}%
\pgfpathclose%
\pgfusepath{stroke}%
\end{pgfscope}%
\begin{pgfscope}%
\pgfpathrectangle{\pgfqpoint{0.465000in}{0.449444in}}{\pgfqpoint{1.550000in}{1.155000in}}%
\pgfusepath{clip}%
\pgfsetbuttcap%
\pgfsetmiterjoin%
\pgfsetlinewidth{1.003750pt}%
\definecolor{currentstroke}{rgb}{0.000000,0.000000,0.000000}%
\pgfsetstrokecolor{currentstroke}%
\pgfsetdash{}{0pt}%
\pgfpathmoveto{\pgfqpoint{1.652073in}{0.449444in}}%
\pgfpathlineto{\pgfqpoint{1.712561in}{0.449444in}}%
\pgfpathlineto{\pgfqpoint{1.712561in}{0.542504in}}%
\pgfpathlineto{\pgfqpoint{1.652073in}{0.542504in}}%
\pgfpathlineto{\pgfqpoint{1.652073in}{0.449444in}}%
\pgfpathclose%
\pgfusepath{stroke}%
\end{pgfscope}%
\begin{pgfscope}%
\pgfpathrectangle{\pgfqpoint{0.465000in}{0.449444in}}{\pgfqpoint{1.550000in}{1.155000in}}%
\pgfusepath{clip}%
\pgfsetbuttcap%
\pgfsetmiterjoin%
\pgfsetlinewidth{1.003750pt}%
\definecolor{currentstroke}{rgb}{0.000000,0.000000,0.000000}%
\pgfsetstrokecolor{currentstroke}%
\pgfsetdash{}{0pt}%
\pgfpathmoveto{\pgfqpoint{1.803293in}{0.449444in}}%
\pgfpathlineto{\pgfqpoint{1.863781in}{0.449444in}}%
\pgfpathlineto{\pgfqpoint{1.863781in}{0.511063in}}%
\pgfpathlineto{\pgfqpoint{1.803293in}{0.511063in}}%
\pgfpathlineto{\pgfqpoint{1.803293in}{0.449444in}}%
\pgfpathclose%
\pgfusepath{stroke}%
\end{pgfscope}%
\begin{pgfscope}%
\pgfpathrectangle{\pgfqpoint{0.465000in}{0.449444in}}{\pgfqpoint{1.550000in}{1.155000in}}%
\pgfusepath{clip}%
\pgfsetbuttcap%
\pgfsetmiterjoin%
\definecolor{currentfill}{rgb}{0.000000,0.000000,0.000000}%
\pgfsetfillcolor{currentfill}%
\pgfsetlinewidth{0.000000pt}%
\definecolor{currentstroke}{rgb}{0.000000,0.000000,0.000000}%
\pgfsetstrokecolor{currentstroke}%
\pgfsetstrokeopacity{0.000000}%
\pgfsetdash{}{0pt}%
\pgfpathmoveto{\pgfqpoint{0.502805in}{0.449444in}}%
\pgfpathlineto{\pgfqpoint{0.563293in}{0.449444in}}%
\pgfpathlineto{\pgfqpoint{0.563293in}{0.460283in}}%
\pgfpathlineto{\pgfqpoint{0.502805in}{0.460283in}}%
\pgfpathlineto{\pgfqpoint{0.502805in}{0.449444in}}%
\pgfpathclose%
\pgfusepath{fill}%
\end{pgfscope}%
\begin{pgfscope}%
\pgfpathrectangle{\pgfqpoint{0.465000in}{0.449444in}}{\pgfqpoint{1.550000in}{1.155000in}}%
\pgfusepath{clip}%
\pgfsetbuttcap%
\pgfsetmiterjoin%
\definecolor{currentfill}{rgb}{0.000000,0.000000,0.000000}%
\pgfsetfillcolor{currentfill}%
\pgfsetlinewidth{0.000000pt}%
\definecolor{currentstroke}{rgb}{0.000000,0.000000,0.000000}%
\pgfsetstrokecolor{currentstroke}%
\pgfsetstrokeopacity{0.000000}%
\pgfsetdash{}{0pt}%
\pgfpathmoveto{\pgfqpoint{0.654025in}{0.449444in}}%
\pgfpathlineto{\pgfqpoint{0.714512in}{0.449444in}}%
\pgfpathlineto{\pgfqpoint{0.714512in}{0.468432in}}%
\pgfpathlineto{\pgfqpoint{0.654025in}{0.468432in}}%
\pgfpathlineto{\pgfqpoint{0.654025in}{0.449444in}}%
\pgfpathclose%
\pgfusepath{fill}%
\end{pgfscope}%
\begin{pgfscope}%
\pgfpathrectangle{\pgfqpoint{0.465000in}{0.449444in}}{\pgfqpoint{1.550000in}{1.155000in}}%
\pgfusepath{clip}%
\pgfsetbuttcap%
\pgfsetmiterjoin%
\definecolor{currentfill}{rgb}{0.000000,0.000000,0.000000}%
\pgfsetfillcolor{currentfill}%
\pgfsetlinewidth{0.000000pt}%
\definecolor{currentstroke}{rgb}{0.000000,0.000000,0.000000}%
\pgfsetstrokecolor{currentstroke}%
\pgfsetstrokeopacity{0.000000}%
\pgfsetdash{}{0pt}%
\pgfpathmoveto{\pgfqpoint{0.805244in}{0.449444in}}%
\pgfpathlineto{\pgfqpoint{0.865732in}{0.449444in}}%
\pgfpathlineto{\pgfqpoint{0.865732in}{0.482717in}}%
\pgfpathlineto{\pgfqpoint{0.805244in}{0.482717in}}%
\pgfpathlineto{\pgfqpoint{0.805244in}{0.449444in}}%
\pgfpathclose%
\pgfusepath{fill}%
\end{pgfscope}%
\begin{pgfscope}%
\pgfpathrectangle{\pgfqpoint{0.465000in}{0.449444in}}{\pgfqpoint{1.550000in}{1.155000in}}%
\pgfusepath{clip}%
\pgfsetbuttcap%
\pgfsetmiterjoin%
\definecolor{currentfill}{rgb}{0.000000,0.000000,0.000000}%
\pgfsetfillcolor{currentfill}%
\pgfsetlinewidth{0.000000pt}%
\definecolor{currentstroke}{rgb}{0.000000,0.000000,0.000000}%
\pgfsetstrokecolor{currentstroke}%
\pgfsetstrokeopacity{0.000000}%
\pgfsetdash{}{0pt}%
\pgfpathmoveto{\pgfqpoint{0.956464in}{0.449444in}}%
\pgfpathlineto{\pgfqpoint{1.016951in}{0.449444in}}%
\pgfpathlineto{\pgfqpoint{1.016951in}{0.502743in}}%
\pgfpathlineto{\pgfqpoint{0.956464in}{0.502743in}}%
\pgfpathlineto{\pgfqpoint{0.956464in}{0.449444in}}%
\pgfpathclose%
\pgfusepath{fill}%
\end{pgfscope}%
\begin{pgfscope}%
\pgfpathrectangle{\pgfqpoint{0.465000in}{0.449444in}}{\pgfqpoint{1.550000in}{1.155000in}}%
\pgfusepath{clip}%
\pgfsetbuttcap%
\pgfsetmiterjoin%
\definecolor{currentfill}{rgb}{0.000000,0.000000,0.000000}%
\pgfsetfillcolor{currentfill}%
\pgfsetlinewidth{0.000000pt}%
\definecolor{currentstroke}{rgb}{0.000000,0.000000,0.000000}%
\pgfsetstrokecolor{currentstroke}%
\pgfsetstrokeopacity{0.000000}%
\pgfsetdash{}{0pt}%
\pgfpathmoveto{\pgfqpoint{1.107683in}{0.449444in}}%
\pgfpathlineto{\pgfqpoint{1.168171in}{0.449444in}}%
\pgfpathlineto{\pgfqpoint{1.168171in}{0.533426in}}%
\pgfpathlineto{\pgfqpoint{1.107683in}{0.533426in}}%
\pgfpathlineto{\pgfqpoint{1.107683in}{0.449444in}}%
\pgfpathclose%
\pgfusepath{fill}%
\end{pgfscope}%
\begin{pgfscope}%
\pgfpathrectangle{\pgfqpoint{0.465000in}{0.449444in}}{\pgfqpoint{1.550000in}{1.155000in}}%
\pgfusepath{clip}%
\pgfsetbuttcap%
\pgfsetmiterjoin%
\definecolor{currentfill}{rgb}{0.000000,0.000000,0.000000}%
\pgfsetfillcolor{currentfill}%
\pgfsetlinewidth{0.000000pt}%
\definecolor{currentstroke}{rgb}{0.000000,0.000000,0.000000}%
\pgfsetstrokecolor{currentstroke}%
\pgfsetstrokeopacity{0.000000}%
\pgfsetdash{}{0pt}%
\pgfpathmoveto{\pgfqpoint{1.258903in}{0.449444in}}%
\pgfpathlineto{\pgfqpoint{1.319391in}{0.449444in}}%
\pgfpathlineto{\pgfqpoint{1.319391in}{0.572998in}}%
\pgfpathlineto{\pgfqpoint{1.258903in}{0.572998in}}%
\pgfpathlineto{\pgfqpoint{1.258903in}{0.449444in}}%
\pgfpathclose%
\pgfusepath{fill}%
\end{pgfscope}%
\begin{pgfscope}%
\pgfpathrectangle{\pgfqpoint{0.465000in}{0.449444in}}{\pgfqpoint{1.550000in}{1.155000in}}%
\pgfusepath{clip}%
\pgfsetbuttcap%
\pgfsetmiterjoin%
\definecolor{currentfill}{rgb}{0.000000,0.000000,0.000000}%
\pgfsetfillcolor{currentfill}%
\pgfsetlinewidth{0.000000pt}%
\definecolor{currentstroke}{rgb}{0.000000,0.000000,0.000000}%
\pgfsetstrokecolor{currentstroke}%
\pgfsetstrokeopacity{0.000000}%
\pgfsetdash{}{0pt}%
\pgfpathmoveto{\pgfqpoint{1.410122in}{0.449444in}}%
\pgfpathlineto{\pgfqpoint{1.470610in}{0.449444in}}%
\pgfpathlineto{\pgfqpoint{1.470610in}{0.615151in}}%
\pgfpathlineto{\pgfqpoint{1.410122in}{0.615151in}}%
\pgfpathlineto{\pgfqpoint{1.410122in}{0.449444in}}%
\pgfpathclose%
\pgfusepath{fill}%
\end{pgfscope}%
\begin{pgfscope}%
\pgfpathrectangle{\pgfqpoint{0.465000in}{0.449444in}}{\pgfqpoint{1.550000in}{1.155000in}}%
\pgfusepath{clip}%
\pgfsetbuttcap%
\pgfsetmiterjoin%
\definecolor{currentfill}{rgb}{0.000000,0.000000,0.000000}%
\pgfsetfillcolor{currentfill}%
\pgfsetlinewidth{0.000000pt}%
\definecolor{currentstroke}{rgb}{0.000000,0.000000,0.000000}%
\pgfsetstrokecolor{currentstroke}%
\pgfsetstrokeopacity{0.000000}%
\pgfsetdash{}{0pt}%
\pgfpathmoveto{\pgfqpoint{1.561342in}{0.449444in}}%
\pgfpathlineto{\pgfqpoint{1.621830in}{0.449444in}}%
\pgfpathlineto{\pgfqpoint{1.621830in}{0.646087in}}%
\pgfpathlineto{\pgfqpoint{1.561342in}{0.646087in}}%
\pgfpathlineto{\pgfqpoint{1.561342in}{0.449444in}}%
\pgfpathclose%
\pgfusepath{fill}%
\end{pgfscope}%
\begin{pgfscope}%
\pgfpathrectangle{\pgfqpoint{0.465000in}{0.449444in}}{\pgfqpoint{1.550000in}{1.155000in}}%
\pgfusepath{clip}%
\pgfsetbuttcap%
\pgfsetmiterjoin%
\definecolor{currentfill}{rgb}{0.000000,0.000000,0.000000}%
\pgfsetfillcolor{currentfill}%
\pgfsetlinewidth{0.000000pt}%
\definecolor{currentstroke}{rgb}{0.000000,0.000000,0.000000}%
\pgfsetstrokecolor{currentstroke}%
\pgfsetstrokeopacity{0.000000}%
\pgfsetdash{}{0pt}%
\pgfpathmoveto{\pgfqpoint{1.712561in}{0.449444in}}%
\pgfpathlineto{\pgfqpoint{1.773049in}{0.449444in}}%
\pgfpathlineto{\pgfqpoint{1.773049in}{0.622055in}}%
\pgfpathlineto{\pgfqpoint{1.712561in}{0.622055in}}%
\pgfpathlineto{\pgfqpoint{1.712561in}{0.449444in}}%
\pgfpathclose%
\pgfusepath{fill}%
\end{pgfscope}%
\begin{pgfscope}%
\pgfpathrectangle{\pgfqpoint{0.465000in}{0.449444in}}{\pgfqpoint{1.550000in}{1.155000in}}%
\pgfusepath{clip}%
\pgfsetbuttcap%
\pgfsetmiterjoin%
\definecolor{currentfill}{rgb}{0.000000,0.000000,0.000000}%
\pgfsetfillcolor{currentfill}%
\pgfsetlinewidth{0.000000pt}%
\definecolor{currentstroke}{rgb}{0.000000,0.000000,0.000000}%
\pgfsetstrokecolor{currentstroke}%
\pgfsetstrokeopacity{0.000000}%
\pgfsetdash{}{0pt}%
\pgfpathmoveto{\pgfqpoint{1.863781in}{0.449444in}}%
\pgfpathlineto{\pgfqpoint{1.924269in}{0.449444in}}%
\pgfpathlineto{\pgfqpoint{1.924269in}{0.559534in}}%
\pgfpathlineto{\pgfqpoint{1.863781in}{0.559534in}}%
\pgfpathlineto{\pgfqpoint{1.863781in}{0.449444in}}%
\pgfpathclose%
\pgfusepath{fill}%
\end{pgfscope}%
\begin{pgfscope}%
\pgfsetbuttcap%
\pgfsetroundjoin%
\definecolor{currentfill}{rgb}{0.000000,0.000000,0.000000}%
\pgfsetfillcolor{currentfill}%
\pgfsetlinewidth{0.803000pt}%
\definecolor{currentstroke}{rgb}{0.000000,0.000000,0.000000}%
\pgfsetstrokecolor{currentstroke}%
\pgfsetdash{}{0pt}%
\pgfsys@defobject{currentmarker}{\pgfqpoint{0.000000in}{-0.048611in}}{\pgfqpoint{0.000000in}{0.000000in}}{%
\pgfpathmoveto{\pgfqpoint{0.000000in}{0.000000in}}%
\pgfpathlineto{\pgfqpoint{0.000000in}{-0.048611in}}%
\pgfusepath{stroke,fill}%
}%
\begin{pgfscope}%
\pgfsys@transformshift{0.502805in}{0.449444in}%
\pgfsys@useobject{currentmarker}{}%
\end{pgfscope}%
\end{pgfscope}%
\begin{pgfscope}%
\definecolor{textcolor}{rgb}{0.000000,0.000000,0.000000}%
\pgfsetstrokecolor{textcolor}%
\pgfsetfillcolor{textcolor}%
\pgftext[x=0.502805in,y=0.352222in,,top]{\color{textcolor}\rmfamily\fontsize{10.000000}{12.000000}\selectfont 0.0}%
\end{pgfscope}%
\begin{pgfscope}%
\pgfsetbuttcap%
\pgfsetroundjoin%
\definecolor{currentfill}{rgb}{0.000000,0.000000,0.000000}%
\pgfsetfillcolor{currentfill}%
\pgfsetlinewidth{0.803000pt}%
\definecolor{currentstroke}{rgb}{0.000000,0.000000,0.000000}%
\pgfsetstrokecolor{currentstroke}%
\pgfsetdash{}{0pt}%
\pgfsys@defobject{currentmarker}{\pgfqpoint{0.000000in}{-0.048611in}}{\pgfqpoint{0.000000in}{0.000000in}}{%
\pgfpathmoveto{\pgfqpoint{0.000000in}{0.000000in}}%
\pgfpathlineto{\pgfqpoint{0.000000in}{-0.048611in}}%
\pgfusepath{stroke,fill}%
}%
\begin{pgfscope}%
\pgfsys@transformshift{0.880854in}{0.449444in}%
\pgfsys@useobject{currentmarker}{}%
\end{pgfscope}%
\end{pgfscope}%
\begin{pgfscope}%
\definecolor{textcolor}{rgb}{0.000000,0.000000,0.000000}%
\pgfsetstrokecolor{textcolor}%
\pgfsetfillcolor{textcolor}%
\pgftext[x=0.880854in,y=0.352222in,,top]{\color{textcolor}\rmfamily\fontsize{10.000000}{12.000000}\selectfont 0.25}%
\end{pgfscope}%
\begin{pgfscope}%
\pgfsetbuttcap%
\pgfsetroundjoin%
\definecolor{currentfill}{rgb}{0.000000,0.000000,0.000000}%
\pgfsetfillcolor{currentfill}%
\pgfsetlinewidth{0.803000pt}%
\definecolor{currentstroke}{rgb}{0.000000,0.000000,0.000000}%
\pgfsetstrokecolor{currentstroke}%
\pgfsetdash{}{0pt}%
\pgfsys@defobject{currentmarker}{\pgfqpoint{0.000000in}{-0.048611in}}{\pgfqpoint{0.000000in}{0.000000in}}{%
\pgfpathmoveto{\pgfqpoint{0.000000in}{0.000000in}}%
\pgfpathlineto{\pgfqpoint{0.000000in}{-0.048611in}}%
\pgfusepath{stroke,fill}%
}%
\begin{pgfscope}%
\pgfsys@transformshift{1.258903in}{0.449444in}%
\pgfsys@useobject{currentmarker}{}%
\end{pgfscope}%
\end{pgfscope}%
\begin{pgfscope}%
\definecolor{textcolor}{rgb}{0.000000,0.000000,0.000000}%
\pgfsetstrokecolor{textcolor}%
\pgfsetfillcolor{textcolor}%
\pgftext[x=1.258903in,y=0.352222in,,top]{\color{textcolor}\rmfamily\fontsize{10.000000}{12.000000}\selectfont 0.5}%
\end{pgfscope}%
\begin{pgfscope}%
\pgfsetbuttcap%
\pgfsetroundjoin%
\definecolor{currentfill}{rgb}{0.000000,0.000000,0.000000}%
\pgfsetfillcolor{currentfill}%
\pgfsetlinewidth{0.803000pt}%
\definecolor{currentstroke}{rgb}{0.000000,0.000000,0.000000}%
\pgfsetstrokecolor{currentstroke}%
\pgfsetdash{}{0pt}%
\pgfsys@defobject{currentmarker}{\pgfqpoint{0.000000in}{-0.048611in}}{\pgfqpoint{0.000000in}{0.000000in}}{%
\pgfpathmoveto{\pgfqpoint{0.000000in}{0.000000in}}%
\pgfpathlineto{\pgfqpoint{0.000000in}{-0.048611in}}%
\pgfusepath{stroke,fill}%
}%
\begin{pgfscope}%
\pgfsys@transformshift{1.636951in}{0.449444in}%
\pgfsys@useobject{currentmarker}{}%
\end{pgfscope}%
\end{pgfscope}%
\begin{pgfscope}%
\definecolor{textcolor}{rgb}{0.000000,0.000000,0.000000}%
\pgfsetstrokecolor{textcolor}%
\pgfsetfillcolor{textcolor}%
\pgftext[x=1.636951in,y=0.352222in,,top]{\color{textcolor}\rmfamily\fontsize{10.000000}{12.000000}\selectfont 0.75}%
\end{pgfscope}%
\begin{pgfscope}%
\pgfsetbuttcap%
\pgfsetroundjoin%
\definecolor{currentfill}{rgb}{0.000000,0.000000,0.000000}%
\pgfsetfillcolor{currentfill}%
\pgfsetlinewidth{0.803000pt}%
\definecolor{currentstroke}{rgb}{0.000000,0.000000,0.000000}%
\pgfsetstrokecolor{currentstroke}%
\pgfsetdash{}{0pt}%
\pgfsys@defobject{currentmarker}{\pgfqpoint{0.000000in}{-0.048611in}}{\pgfqpoint{0.000000in}{0.000000in}}{%
\pgfpathmoveto{\pgfqpoint{0.000000in}{0.000000in}}%
\pgfpathlineto{\pgfqpoint{0.000000in}{-0.048611in}}%
\pgfusepath{stroke,fill}%
}%
\begin{pgfscope}%
\pgfsys@transformshift{2.015000in}{0.449444in}%
\pgfsys@useobject{currentmarker}{}%
\end{pgfscope}%
\end{pgfscope}%
\begin{pgfscope}%
\definecolor{textcolor}{rgb}{0.000000,0.000000,0.000000}%
\pgfsetstrokecolor{textcolor}%
\pgfsetfillcolor{textcolor}%
\pgftext[x=2.015000in,y=0.352222in,,top]{\color{textcolor}\rmfamily\fontsize{10.000000}{12.000000}\selectfont 1.0}%
\end{pgfscope}%
\begin{pgfscope}%
\definecolor{textcolor}{rgb}{0.000000,0.000000,0.000000}%
\pgfsetstrokecolor{textcolor}%
\pgfsetfillcolor{textcolor}%
\pgftext[x=1.240000in,y=0.173333in,,top]{\color{textcolor}\rmfamily\fontsize{10.000000}{12.000000}\selectfont \(\displaystyle p\)}%
\end{pgfscope}%
\begin{pgfscope}%
\pgfsetbuttcap%
\pgfsetroundjoin%
\definecolor{currentfill}{rgb}{0.000000,0.000000,0.000000}%
\pgfsetfillcolor{currentfill}%
\pgfsetlinewidth{0.803000pt}%
\definecolor{currentstroke}{rgb}{0.000000,0.000000,0.000000}%
\pgfsetstrokecolor{currentstroke}%
\pgfsetdash{}{0pt}%
\pgfsys@defobject{currentmarker}{\pgfqpoint{-0.048611in}{0.000000in}}{\pgfqpoint{-0.000000in}{0.000000in}}{%
\pgfpathmoveto{\pgfqpoint{-0.000000in}{0.000000in}}%
\pgfpathlineto{\pgfqpoint{-0.048611in}{0.000000in}}%
\pgfusepath{stroke,fill}%
}%
\begin{pgfscope}%
\pgfsys@transformshift{0.465000in}{0.449444in}%
\pgfsys@useobject{currentmarker}{}%
\end{pgfscope}%
\end{pgfscope}%
\begin{pgfscope}%
\definecolor{textcolor}{rgb}{0.000000,0.000000,0.000000}%
\pgfsetstrokecolor{textcolor}%
\pgfsetfillcolor{textcolor}%
\pgftext[x=0.298333in, y=0.401250in, left, base]{\color{textcolor}\rmfamily\fontsize{10.000000}{12.000000}\selectfont \(\displaystyle {0}\)}%
\end{pgfscope}%
\begin{pgfscope}%
\pgfsetbuttcap%
\pgfsetroundjoin%
\definecolor{currentfill}{rgb}{0.000000,0.000000,0.000000}%
\pgfsetfillcolor{currentfill}%
\pgfsetlinewidth{0.803000pt}%
\definecolor{currentstroke}{rgb}{0.000000,0.000000,0.000000}%
\pgfsetstrokecolor{currentstroke}%
\pgfsetdash{}{0pt}%
\pgfsys@defobject{currentmarker}{\pgfqpoint{-0.048611in}{0.000000in}}{\pgfqpoint{-0.000000in}{0.000000in}}{%
\pgfpathmoveto{\pgfqpoint{-0.000000in}{0.000000in}}%
\pgfpathlineto{\pgfqpoint{-0.048611in}{0.000000in}}%
\pgfusepath{stroke,fill}%
}%
\begin{pgfscope}%
\pgfsys@transformshift{0.465000in}{0.771421in}%
\pgfsys@useobject{currentmarker}{}%
\end{pgfscope}%
\end{pgfscope}%
\begin{pgfscope}%
\definecolor{textcolor}{rgb}{0.000000,0.000000,0.000000}%
\pgfsetstrokecolor{textcolor}%
\pgfsetfillcolor{textcolor}%
\pgftext[x=0.298333in, y=0.723227in, left, base]{\color{textcolor}\rmfamily\fontsize{10.000000}{12.000000}\selectfont \(\displaystyle {5}\)}%
\end{pgfscope}%
\begin{pgfscope}%
\pgfsetbuttcap%
\pgfsetroundjoin%
\definecolor{currentfill}{rgb}{0.000000,0.000000,0.000000}%
\pgfsetfillcolor{currentfill}%
\pgfsetlinewidth{0.803000pt}%
\definecolor{currentstroke}{rgb}{0.000000,0.000000,0.000000}%
\pgfsetstrokecolor{currentstroke}%
\pgfsetdash{}{0pt}%
\pgfsys@defobject{currentmarker}{\pgfqpoint{-0.048611in}{0.000000in}}{\pgfqpoint{-0.000000in}{0.000000in}}{%
\pgfpathmoveto{\pgfqpoint{-0.000000in}{0.000000in}}%
\pgfpathlineto{\pgfqpoint{-0.048611in}{0.000000in}}%
\pgfusepath{stroke,fill}%
}%
\begin{pgfscope}%
\pgfsys@transformshift{0.465000in}{1.093399in}%
\pgfsys@useobject{currentmarker}{}%
\end{pgfscope}%
\end{pgfscope}%
\begin{pgfscope}%
\definecolor{textcolor}{rgb}{0.000000,0.000000,0.000000}%
\pgfsetstrokecolor{textcolor}%
\pgfsetfillcolor{textcolor}%
\pgftext[x=0.228889in, y=1.045204in, left, base]{\color{textcolor}\rmfamily\fontsize{10.000000}{12.000000}\selectfont \(\displaystyle {10}\)}%
\end{pgfscope}%
\begin{pgfscope}%
\pgfsetbuttcap%
\pgfsetroundjoin%
\definecolor{currentfill}{rgb}{0.000000,0.000000,0.000000}%
\pgfsetfillcolor{currentfill}%
\pgfsetlinewidth{0.803000pt}%
\definecolor{currentstroke}{rgb}{0.000000,0.000000,0.000000}%
\pgfsetstrokecolor{currentstroke}%
\pgfsetdash{}{0pt}%
\pgfsys@defobject{currentmarker}{\pgfqpoint{-0.048611in}{0.000000in}}{\pgfqpoint{-0.000000in}{0.000000in}}{%
\pgfpathmoveto{\pgfqpoint{-0.000000in}{0.000000in}}%
\pgfpathlineto{\pgfqpoint{-0.048611in}{0.000000in}}%
\pgfusepath{stroke,fill}%
}%
\begin{pgfscope}%
\pgfsys@transformshift{0.465000in}{1.415376in}%
\pgfsys@useobject{currentmarker}{}%
\end{pgfscope}%
\end{pgfscope}%
\begin{pgfscope}%
\definecolor{textcolor}{rgb}{0.000000,0.000000,0.000000}%
\pgfsetstrokecolor{textcolor}%
\pgfsetfillcolor{textcolor}%
\pgftext[x=0.228889in, y=1.367181in, left, base]{\color{textcolor}\rmfamily\fontsize{10.000000}{12.000000}\selectfont \(\displaystyle {15}\)}%
\end{pgfscope}%
\begin{pgfscope}%
\definecolor{textcolor}{rgb}{0.000000,0.000000,0.000000}%
\pgfsetstrokecolor{textcolor}%
\pgfsetfillcolor{textcolor}%
\pgftext[x=0.173333in,y=1.026944in,,bottom,rotate=90.000000]{\color{textcolor}\rmfamily\fontsize{10.000000}{12.000000}\selectfont Percent of Data Set}%
\end{pgfscope}%
\begin{pgfscope}%
\pgfsetrectcap%
\pgfsetmiterjoin%
\pgfsetlinewidth{0.803000pt}%
\definecolor{currentstroke}{rgb}{0.000000,0.000000,0.000000}%
\pgfsetstrokecolor{currentstroke}%
\pgfsetdash{}{0pt}%
\pgfpathmoveto{\pgfqpoint{0.465000in}{0.449444in}}%
\pgfpathlineto{\pgfqpoint{0.465000in}{1.604444in}}%
\pgfusepath{stroke}%
\end{pgfscope}%
\begin{pgfscope}%
\pgfsetrectcap%
\pgfsetmiterjoin%
\pgfsetlinewidth{0.803000pt}%
\definecolor{currentstroke}{rgb}{0.000000,0.000000,0.000000}%
\pgfsetstrokecolor{currentstroke}%
\pgfsetdash{}{0pt}%
\pgfpathmoveto{\pgfqpoint{2.015000in}{0.449444in}}%
\pgfpathlineto{\pgfqpoint{2.015000in}{1.604444in}}%
\pgfusepath{stroke}%
\end{pgfscope}%
\begin{pgfscope}%
\pgfsetrectcap%
\pgfsetmiterjoin%
\pgfsetlinewidth{0.803000pt}%
\definecolor{currentstroke}{rgb}{0.000000,0.000000,0.000000}%
\pgfsetstrokecolor{currentstroke}%
\pgfsetdash{}{0pt}%
\pgfpathmoveto{\pgfqpoint{0.465000in}{0.449444in}}%
\pgfpathlineto{\pgfqpoint{2.015000in}{0.449444in}}%
\pgfusepath{stroke}%
\end{pgfscope}%
\begin{pgfscope}%
\pgfsetrectcap%
\pgfsetmiterjoin%
\pgfsetlinewidth{0.803000pt}%
\definecolor{currentstroke}{rgb}{0.000000,0.000000,0.000000}%
\pgfsetstrokecolor{currentstroke}%
\pgfsetdash{}{0pt}%
\pgfpathmoveto{\pgfqpoint{0.465000in}{1.604444in}}%
\pgfpathlineto{\pgfqpoint{2.015000in}{1.604444in}}%
\pgfusepath{stroke}%
\end{pgfscope}%
\begin{pgfscope}%
\pgfsetbuttcap%
\pgfsetmiterjoin%
\definecolor{currentfill}{rgb}{1.000000,1.000000,1.000000}%
\pgfsetfillcolor{currentfill}%
\pgfsetfillopacity{0.800000}%
\pgfsetlinewidth{1.003750pt}%
\definecolor{currentstroke}{rgb}{0.800000,0.800000,0.800000}%
\pgfsetstrokecolor{currentstroke}%
\pgfsetstrokeopacity{0.800000}%
\pgfsetdash{}{0pt}%
\pgfpathmoveto{\pgfqpoint{1.238056in}{1.104445in}}%
\pgfpathlineto{\pgfqpoint{1.917778in}{1.104445in}}%
\pgfpathquadraticcurveto{\pgfqpoint{1.945556in}{1.104445in}}{\pgfqpoint{1.945556in}{1.132222in}}%
\pgfpathlineto{\pgfqpoint{1.945556in}{1.507222in}}%
\pgfpathquadraticcurveto{\pgfqpoint{1.945556in}{1.535000in}}{\pgfqpoint{1.917778in}{1.535000in}}%
\pgfpathlineto{\pgfqpoint{1.238056in}{1.535000in}}%
\pgfpathquadraticcurveto{\pgfqpoint{1.210278in}{1.535000in}}{\pgfqpoint{1.210278in}{1.507222in}}%
\pgfpathlineto{\pgfqpoint{1.210278in}{1.132222in}}%
\pgfpathquadraticcurveto{\pgfqpoint{1.210278in}{1.104445in}}{\pgfqpoint{1.238056in}{1.104445in}}%
\pgfpathlineto{\pgfqpoint{1.238056in}{1.104445in}}%
\pgfpathclose%
\pgfusepath{stroke,fill}%
\end{pgfscope}%
\begin{pgfscope}%
\pgfsetbuttcap%
\pgfsetmiterjoin%
\pgfsetlinewidth{1.003750pt}%
\definecolor{currentstroke}{rgb}{0.000000,0.000000,0.000000}%
\pgfsetstrokecolor{currentstroke}%
\pgfsetdash{}{0pt}%
\pgfpathmoveto{\pgfqpoint{1.265834in}{1.382222in}}%
\pgfpathlineto{\pgfqpoint{1.543611in}{1.382222in}}%
\pgfpathlineto{\pgfqpoint{1.543611in}{1.479444in}}%
\pgfpathlineto{\pgfqpoint{1.265834in}{1.479444in}}%
\pgfpathlineto{\pgfqpoint{1.265834in}{1.382222in}}%
\pgfpathclose%
\pgfusepath{stroke}%
\end{pgfscope}%
\begin{pgfscope}%
\definecolor{textcolor}{rgb}{0.000000,0.000000,0.000000}%
\pgfsetstrokecolor{textcolor}%
\pgfsetfillcolor{textcolor}%
\pgftext[x=1.654722in,y=1.382222in,left,base]{\color{textcolor}\rmfamily\fontsize{10.000000}{12.000000}\selectfont Neg}%
\end{pgfscope}%
\begin{pgfscope}%
\pgfsetbuttcap%
\pgfsetmiterjoin%
\definecolor{currentfill}{rgb}{0.000000,0.000000,0.000000}%
\pgfsetfillcolor{currentfill}%
\pgfsetlinewidth{0.000000pt}%
\definecolor{currentstroke}{rgb}{0.000000,0.000000,0.000000}%
\pgfsetstrokecolor{currentstroke}%
\pgfsetstrokeopacity{0.000000}%
\pgfsetdash{}{0pt}%
\pgfpathmoveto{\pgfqpoint{1.265834in}{1.186944in}}%
\pgfpathlineto{\pgfqpoint{1.543611in}{1.186944in}}%
\pgfpathlineto{\pgfqpoint{1.543611in}{1.284167in}}%
\pgfpathlineto{\pgfqpoint{1.265834in}{1.284167in}}%
\pgfpathlineto{\pgfqpoint{1.265834in}{1.186944in}}%
\pgfpathclose%
\pgfusepath{fill}%
\end{pgfscope}%
\begin{pgfscope}%
\definecolor{textcolor}{rgb}{0.000000,0.000000,0.000000}%
\pgfsetstrokecolor{textcolor}%
\pgfsetfillcolor{textcolor}%
\pgftext[x=1.654722in,y=1.186944in,left,base]{\color{textcolor}\rmfamily\fontsize{10.000000}{12.000000}\selectfont Pos}%
\end{pgfscope}%
\end{pgfpicture}%
\makeatother%
\endgroup%

&
	\vskip 0pt
	\qquad \qquad ROC Curve
	
	%% Creator: Matplotlib, PGF backend
%%
%% To include the figure in your LaTeX document, write
%%   \input{<filename>.pgf}
%%
%% Make sure the required packages are loaded in your preamble
%%   \usepackage{pgf}
%%
%% Also ensure that all the required font packages are loaded; for instance,
%% the lmodern package is sometimes necessary when using math font.
%%   \usepackage{lmodern}
%%
%% Figures using additional raster images can only be included by \input if
%% they are in the same directory as the main LaTeX file. For loading figures
%% from other directories you can use the `import` package
%%   \usepackage{import}
%%
%% and then include the figures with
%%   \import{<path to file>}{<filename>.pgf}
%%
%% Matplotlib used the following preamble
%%   
%%   \usepackage{fontspec}
%%   \makeatletter\@ifpackageloaded{underscore}{}{\usepackage[strings]{underscore}}\makeatother
%%
\begingroup%
\makeatletter%
\begin{pgfpicture}%
\pgfpathrectangle{\pgfpointorigin}{\pgfqpoint{2.221861in}{1.754444in}}%
\pgfusepath{use as bounding box, clip}%
\begin{pgfscope}%
\pgfsetbuttcap%
\pgfsetmiterjoin%
\definecolor{currentfill}{rgb}{1.000000,1.000000,1.000000}%
\pgfsetfillcolor{currentfill}%
\pgfsetlinewidth{0.000000pt}%
\definecolor{currentstroke}{rgb}{1.000000,1.000000,1.000000}%
\pgfsetstrokecolor{currentstroke}%
\pgfsetdash{}{0pt}%
\pgfpathmoveto{\pgfqpoint{0.000000in}{0.000000in}}%
\pgfpathlineto{\pgfqpoint{2.221861in}{0.000000in}}%
\pgfpathlineto{\pgfqpoint{2.221861in}{1.754444in}}%
\pgfpathlineto{\pgfqpoint{0.000000in}{1.754444in}}%
\pgfpathlineto{\pgfqpoint{0.000000in}{0.000000in}}%
\pgfpathclose%
\pgfusepath{fill}%
\end{pgfscope}%
\begin{pgfscope}%
\pgfsetbuttcap%
\pgfsetmiterjoin%
\definecolor{currentfill}{rgb}{1.000000,1.000000,1.000000}%
\pgfsetfillcolor{currentfill}%
\pgfsetlinewidth{0.000000pt}%
\definecolor{currentstroke}{rgb}{0.000000,0.000000,0.000000}%
\pgfsetstrokecolor{currentstroke}%
\pgfsetstrokeopacity{0.000000}%
\pgfsetdash{}{0pt}%
\pgfpathmoveto{\pgfqpoint{0.553581in}{0.499444in}}%
\pgfpathlineto{\pgfqpoint{2.103581in}{0.499444in}}%
\pgfpathlineto{\pgfqpoint{2.103581in}{1.654444in}}%
\pgfpathlineto{\pgfqpoint{0.553581in}{1.654444in}}%
\pgfpathlineto{\pgfqpoint{0.553581in}{0.499444in}}%
\pgfpathclose%
\pgfusepath{fill}%
\end{pgfscope}%
\begin{pgfscope}%
\pgfsetbuttcap%
\pgfsetroundjoin%
\definecolor{currentfill}{rgb}{0.000000,0.000000,0.000000}%
\pgfsetfillcolor{currentfill}%
\pgfsetlinewidth{0.803000pt}%
\definecolor{currentstroke}{rgb}{0.000000,0.000000,0.000000}%
\pgfsetstrokecolor{currentstroke}%
\pgfsetdash{}{0pt}%
\pgfsys@defobject{currentmarker}{\pgfqpoint{0.000000in}{-0.048611in}}{\pgfqpoint{0.000000in}{0.000000in}}{%
\pgfpathmoveto{\pgfqpoint{0.000000in}{0.000000in}}%
\pgfpathlineto{\pgfqpoint{0.000000in}{-0.048611in}}%
\pgfusepath{stroke,fill}%
}%
\begin{pgfscope}%
\pgfsys@transformshift{0.624035in}{0.499444in}%
\pgfsys@useobject{currentmarker}{}%
\end{pgfscope}%
\end{pgfscope}%
\begin{pgfscope}%
\definecolor{textcolor}{rgb}{0.000000,0.000000,0.000000}%
\pgfsetstrokecolor{textcolor}%
\pgfsetfillcolor{textcolor}%
\pgftext[x=0.624035in,y=0.402222in,,top]{\color{textcolor}\rmfamily\fontsize{10.000000}{12.000000}\selectfont \(\displaystyle {0.0}\)}%
\end{pgfscope}%
\begin{pgfscope}%
\pgfsetbuttcap%
\pgfsetroundjoin%
\definecolor{currentfill}{rgb}{0.000000,0.000000,0.000000}%
\pgfsetfillcolor{currentfill}%
\pgfsetlinewidth{0.803000pt}%
\definecolor{currentstroke}{rgb}{0.000000,0.000000,0.000000}%
\pgfsetstrokecolor{currentstroke}%
\pgfsetdash{}{0pt}%
\pgfsys@defobject{currentmarker}{\pgfqpoint{0.000000in}{-0.048611in}}{\pgfqpoint{0.000000in}{0.000000in}}{%
\pgfpathmoveto{\pgfqpoint{0.000000in}{0.000000in}}%
\pgfpathlineto{\pgfqpoint{0.000000in}{-0.048611in}}%
\pgfusepath{stroke,fill}%
}%
\begin{pgfscope}%
\pgfsys@transformshift{1.328581in}{0.499444in}%
\pgfsys@useobject{currentmarker}{}%
\end{pgfscope}%
\end{pgfscope}%
\begin{pgfscope}%
\definecolor{textcolor}{rgb}{0.000000,0.000000,0.000000}%
\pgfsetstrokecolor{textcolor}%
\pgfsetfillcolor{textcolor}%
\pgftext[x=1.328581in,y=0.402222in,,top]{\color{textcolor}\rmfamily\fontsize{10.000000}{12.000000}\selectfont \(\displaystyle {0.5}\)}%
\end{pgfscope}%
\begin{pgfscope}%
\pgfsetbuttcap%
\pgfsetroundjoin%
\definecolor{currentfill}{rgb}{0.000000,0.000000,0.000000}%
\pgfsetfillcolor{currentfill}%
\pgfsetlinewidth{0.803000pt}%
\definecolor{currentstroke}{rgb}{0.000000,0.000000,0.000000}%
\pgfsetstrokecolor{currentstroke}%
\pgfsetdash{}{0pt}%
\pgfsys@defobject{currentmarker}{\pgfqpoint{0.000000in}{-0.048611in}}{\pgfqpoint{0.000000in}{0.000000in}}{%
\pgfpathmoveto{\pgfqpoint{0.000000in}{0.000000in}}%
\pgfpathlineto{\pgfqpoint{0.000000in}{-0.048611in}}%
\pgfusepath{stroke,fill}%
}%
\begin{pgfscope}%
\pgfsys@transformshift{2.033126in}{0.499444in}%
\pgfsys@useobject{currentmarker}{}%
\end{pgfscope}%
\end{pgfscope}%
\begin{pgfscope}%
\definecolor{textcolor}{rgb}{0.000000,0.000000,0.000000}%
\pgfsetstrokecolor{textcolor}%
\pgfsetfillcolor{textcolor}%
\pgftext[x=2.033126in,y=0.402222in,,top]{\color{textcolor}\rmfamily\fontsize{10.000000}{12.000000}\selectfont \(\displaystyle {1.0}\)}%
\end{pgfscope}%
\begin{pgfscope}%
\definecolor{textcolor}{rgb}{0.000000,0.000000,0.000000}%
\pgfsetstrokecolor{textcolor}%
\pgfsetfillcolor{textcolor}%
\pgftext[x=1.328581in,y=0.223333in,,top]{\color{textcolor}\rmfamily\fontsize{10.000000}{12.000000}\selectfont False positive rate}%
\end{pgfscope}%
\begin{pgfscope}%
\pgfsetbuttcap%
\pgfsetroundjoin%
\definecolor{currentfill}{rgb}{0.000000,0.000000,0.000000}%
\pgfsetfillcolor{currentfill}%
\pgfsetlinewidth{0.803000pt}%
\definecolor{currentstroke}{rgb}{0.000000,0.000000,0.000000}%
\pgfsetstrokecolor{currentstroke}%
\pgfsetdash{}{0pt}%
\pgfsys@defobject{currentmarker}{\pgfqpoint{-0.048611in}{0.000000in}}{\pgfqpoint{-0.000000in}{0.000000in}}{%
\pgfpathmoveto{\pgfqpoint{-0.000000in}{0.000000in}}%
\pgfpathlineto{\pgfqpoint{-0.048611in}{0.000000in}}%
\pgfusepath{stroke,fill}%
}%
\begin{pgfscope}%
\pgfsys@transformshift{0.553581in}{0.551944in}%
\pgfsys@useobject{currentmarker}{}%
\end{pgfscope}%
\end{pgfscope}%
\begin{pgfscope}%
\definecolor{textcolor}{rgb}{0.000000,0.000000,0.000000}%
\pgfsetstrokecolor{textcolor}%
\pgfsetfillcolor{textcolor}%
\pgftext[x=0.278889in, y=0.503750in, left, base]{\color{textcolor}\rmfamily\fontsize{10.000000}{12.000000}\selectfont \(\displaystyle {0.0}\)}%
\end{pgfscope}%
\begin{pgfscope}%
\pgfsetbuttcap%
\pgfsetroundjoin%
\definecolor{currentfill}{rgb}{0.000000,0.000000,0.000000}%
\pgfsetfillcolor{currentfill}%
\pgfsetlinewidth{0.803000pt}%
\definecolor{currentstroke}{rgb}{0.000000,0.000000,0.000000}%
\pgfsetstrokecolor{currentstroke}%
\pgfsetdash{}{0pt}%
\pgfsys@defobject{currentmarker}{\pgfqpoint{-0.048611in}{0.000000in}}{\pgfqpoint{-0.000000in}{0.000000in}}{%
\pgfpathmoveto{\pgfqpoint{-0.000000in}{0.000000in}}%
\pgfpathlineto{\pgfqpoint{-0.048611in}{0.000000in}}%
\pgfusepath{stroke,fill}%
}%
\begin{pgfscope}%
\pgfsys@transformshift{0.553581in}{1.076944in}%
\pgfsys@useobject{currentmarker}{}%
\end{pgfscope}%
\end{pgfscope}%
\begin{pgfscope}%
\definecolor{textcolor}{rgb}{0.000000,0.000000,0.000000}%
\pgfsetstrokecolor{textcolor}%
\pgfsetfillcolor{textcolor}%
\pgftext[x=0.278889in, y=1.028750in, left, base]{\color{textcolor}\rmfamily\fontsize{10.000000}{12.000000}\selectfont \(\displaystyle {0.5}\)}%
\end{pgfscope}%
\begin{pgfscope}%
\pgfsetbuttcap%
\pgfsetroundjoin%
\definecolor{currentfill}{rgb}{0.000000,0.000000,0.000000}%
\pgfsetfillcolor{currentfill}%
\pgfsetlinewidth{0.803000pt}%
\definecolor{currentstroke}{rgb}{0.000000,0.000000,0.000000}%
\pgfsetstrokecolor{currentstroke}%
\pgfsetdash{}{0pt}%
\pgfsys@defobject{currentmarker}{\pgfqpoint{-0.048611in}{0.000000in}}{\pgfqpoint{-0.000000in}{0.000000in}}{%
\pgfpathmoveto{\pgfqpoint{-0.000000in}{0.000000in}}%
\pgfpathlineto{\pgfqpoint{-0.048611in}{0.000000in}}%
\pgfusepath{stroke,fill}%
}%
\begin{pgfscope}%
\pgfsys@transformshift{0.553581in}{1.601944in}%
\pgfsys@useobject{currentmarker}{}%
\end{pgfscope}%
\end{pgfscope}%
\begin{pgfscope}%
\definecolor{textcolor}{rgb}{0.000000,0.000000,0.000000}%
\pgfsetstrokecolor{textcolor}%
\pgfsetfillcolor{textcolor}%
\pgftext[x=0.278889in, y=1.553750in, left, base]{\color{textcolor}\rmfamily\fontsize{10.000000}{12.000000}\selectfont \(\displaystyle {1.0}\)}%
\end{pgfscope}%
\begin{pgfscope}%
\definecolor{textcolor}{rgb}{0.000000,0.000000,0.000000}%
\pgfsetstrokecolor{textcolor}%
\pgfsetfillcolor{textcolor}%
\pgftext[x=0.223333in,y=1.076944in,,bottom,rotate=90.000000]{\color{textcolor}\rmfamily\fontsize{10.000000}{12.000000}\selectfont True positive rate}%
\end{pgfscope}%
\begin{pgfscope}%
\pgfpathrectangle{\pgfqpoint{0.553581in}{0.499444in}}{\pgfqpoint{1.550000in}{1.155000in}}%
\pgfusepath{clip}%
\pgfsetbuttcap%
\pgfsetroundjoin%
\pgfsetlinewidth{1.505625pt}%
\definecolor{currentstroke}{rgb}{0.000000,0.000000,0.000000}%
\pgfsetstrokecolor{currentstroke}%
\pgfsetdash{{5.550000pt}{2.400000pt}}{0.000000pt}%
\pgfpathmoveto{\pgfqpoint{0.624035in}{0.551944in}}%
\pgfpathlineto{\pgfqpoint{2.033126in}{1.601944in}}%
\pgfusepath{stroke}%
\end{pgfscope}%
\begin{pgfscope}%
\pgfpathrectangle{\pgfqpoint{0.553581in}{0.499444in}}{\pgfqpoint{1.550000in}{1.155000in}}%
\pgfusepath{clip}%
\pgfsetrectcap%
\pgfsetroundjoin%
\pgfsetlinewidth{1.505625pt}%
\definecolor{currentstroke}{rgb}{0.000000,0.000000,0.000000}%
\pgfsetstrokecolor{currentstroke}%
\pgfsetdash{}{0pt}%
\pgfpathmoveto{\pgfqpoint{0.624035in}{0.551944in}}%
\pgfpathlineto{\pgfqpoint{0.634247in}{0.580854in}}%
\pgfpathlineto{\pgfqpoint{0.635358in}{0.585824in}}%
\pgfpathlineto{\pgfqpoint{0.635540in}{0.586874in}}%
\pgfpathlineto{\pgfqpoint{0.636634in}{0.593244in}}%
\pgfpathlineto{\pgfqpoint{0.636833in}{0.594154in}}%
\pgfpathlineto{\pgfqpoint{0.637944in}{0.600734in}}%
\pgfpathlineto{\pgfqpoint{0.638126in}{0.601784in}}%
\pgfpathlineto{\pgfqpoint{0.639187in}{0.608714in}}%
\pgfpathlineto{\pgfqpoint{0.639353in}{0.609694in}}%
\pgfpathlineto{\pgfqpoint{0.640464in}{0.616834in}}%
\pgfpathlineto{\pgfqpoint{0.640712in}{0.617884in}}%
\pgfpathlineto{\pgfqpoint{0.641806in}{0.625864in}}%
\pgfpathlineto{\pgfqpoint{0.642038in}{0.626844in}}%
\pgfpathlineto{\pgfqpoint{0.643149in}{0.635804in}}%
\pgfpathlineto{\pgfqpoint{0.643348in}{0.636854in}}%
\pgfpathlineto{\pgfqpoint{0.644392in}{0.645044in}}%
\pgfpathlineto{\pgfqpoint{0.644641in}{0.645884in}}%
\pgfpathlineto{\pgfqpoint{0.645752in}{0.652394in}}%
\pgfpathlineto{\pgfqpoint{0.645868in}{0.653164in}}%
\pgfpathlineto{\pgfqpoint{0.646979in}{0.661424in}}%
\pgfpathlineto{\pgfqpoint{0.647327in}{0.662264in}}%
\pgfpathlineto{\pgfqpoint{0.648437in}{0.668914in}}%
\pgfpathlineto{\pgfqpoint{0.648587in}{0.669824in}}%
\pgfpathlineto{\pgfqpoint{0.649697in}{0.678294in}}%
\pgfpathlineto{\pgfqpoint{0.649813in}{0.679274in}}%
\pgfpathlineto{\pgfqpoint{0.650924in}{0.687044in}}%
\pgfpathlineto{\pgfqpoint{0.651140in}{0.687884in}}%
\pgfpathlineto{\pgfqpoint{0.652250in}{0.696144in}}%
\pgfpathlineto{\pgfqpoint{0.652333in}{0.697194in}}%
\pgfpathlineto{\pgfqpoint{0.653444in}{0.706434in}}%
\pgfpathlineto{\pgfqpoint{0.653858in}{0.707414in}}%
\pgfpathlineto{\pgfqpoint{0.654969in}{0.716584in}}%
\pgfpathlineto{\pgfqpoint{0.655151in}{0.717564in}}%
\pgfpathlineto{\pgfqpoint{0.656262in}{0.725194in}}%
\pgfpathlineto{\pgfqpoint{0.656444in}{0.726174in}}%
\pgfpathlineto{\pgfqpoint{0.657538in}{0.733244in}}%
\pgfpathlineto{\pgfqpoint{0.657704in}{0.734294in}}%
\pgfpathlineto{\pgfqpoint{0.658815in}{0.743464in}}%
\pgfpathlineto{\pgfqpoint{0.659064in}{0.744514in}}%
\pgfpathlineto{\pgfqpoint{0.660158in}{0.751164in}}%
\pgfpathlineto{\pgfqpoint{0.660290in}{0.751794in}}%
\pgfpathlineto{\pgfqpoint{0.661401in}{0.759354in}}%
\pgfpathlineto{\pgfqpoint{0.661633in}{0.760404in}}%
\pgfpathlineto{\pgfqpoint{0.662744in}{0.769784in}}%
\pgfpathlineto{\pgfqpoint{0.662827in}{0.770344in}}%
\pgfpathlineto{\pgfqpoint{0.663937in}{0.777764in}}%
\pgfpathlineto{\pgfqpoint{0.664120in}{0.778814in}}%
\pgfpathlineto{\pgfqpoint{0.665181in}{0.783854in}}%
\pgfpathlineto{\pgfqpoint{0.665297in}{0.784834in}}%
\pgfpathlineto{\pgfqpoint{0.666407in}{0.795614in}}%
\pgfpathlineto{\pgfqpoint{0.666656in}{0.796664in}}%
\pgfpathlineto{\pgfqpoint{0.667767in}{0.801564in}}%
\pgfpathlineto{\pgfqpoint{0.667899in}{0.802404in}}%
\pgfpathlineto{\pgfqpoint{0.669010in}{0.809334in}}%
\pgfpathlineto{\pgfqpoint{0.669342in}{0.810384in}}%
\pgfpathlineto{\pgfqpoint{0.670452in}{0.818084in}}%
\pgfpathlineto{\pgfqpoint{0.670701in}{0.819134in}}%
\pgfpathlineto{\pgfqpoint{0.671812in}{0.825364in}}%
\pgfpathlineto{\pgfqpoint{0.671911in}{0.826414in}}%
\pgfpathlineto{\pgfqpoint{0.673022in}{0.833694in}}%
\pgfpathlineto{\pgfqpoint{0.673138in}{0.834744in}}%
\pgfpathlineto{\pgfqpoint{0.674215in}{0.838944in}}%
\pgfpathlineto{\pgfqpoint{0.674530in}{0.839994in}}%
\pgfpathlineto{\pgfqpoint{0.675608in}{0.846014in}}%
\pgfpathlineto{\pgfqpoint{0.675790in}{0.846994in}}%
\pgfpathlineto{\pgfqpoint{0.676901in}{0.853854in}}%
\pgfpathlineto{\pgfqpoint{0.677199in}{0.854834in}}%
\pgfpathlineto{\pgfqpoint{0.678310in}{0.859034in}}%
\pgfpathlineto{\pgfqpoint{0.678492in}{0.860084in}}%
\pgfpathlineto{\pgfqpoint{0.679587in}{0.867014in}}%
\pgfpathlineto{\pgfqpoint{0.679868in}{0.867924in}}%
\pgfpathlineto{\pgfqpoint{0.680963in}{0.874504in}}%
\pgfpathlineto{\pgfqpoint{0.681228in}{0.875484in}}%
\pgfpathlineto{\pgfqpoint{0.682338in}{0.882204in}}%
\pgfpathlineto{\pgfqpoint{0.682587in}{0.883254in}}%
\pgfpathlineto{\pgfqpoint{0.683681in}{0.889484in}}%
\pgfpathlineto{\pgfqpoint{0.683980in}{0.890394in}}%
\pgfpathlineto{\pgfqpoint{0.685074in}{0.897394in}}%
\pgfpathlineto{\pgfqpoint{0.685389in}{0.898304in}}%
\pgfpathlineto{\pgfqpoint{0.686450in}{0.904184in}}%
\pgfpathlineto{\pgfqpoint{0.686748in}{0.905094in}}%
\pgfpathlineto{\pgfqpoint{0.687859in}{0.910204in}}%
\pgfpathlineto{\pgfqpoint{0.688074in}{0.911114in}}%
\pgfpathlineto{\pgfqpoint{0.689185in}{0.915034in}}%
\pgfpathlineto{\pgfqpoint{0.689400in}{0.916084in}}%
\pgfpathlineto{\pgfqpoint{0.690511in}{0.922664in}}%
\pgfpathlineto{\pgfqpoint{0.690793in}{0.923714in}}%
\pgfpathlineto{\pgfqpoint{0.691887in}{0.928754in}}%
\pgfpathlineto{\pgfqpoint{0.692318in}{0.929734in}}%
\pgfpathlineto{\pgfqpoint{0.693412in}{0.935334in}}%
\pgfpathlineto{\pgfqpoint{0.693611in}{0.936384in}}%
\pgfpathlineto{\pgfqpoint{0.694705in}{0.940864in}}%
\pgfpathlineto{\pgfqpoint{0.695053in}{0.941914in}}%
\pgfpathlineto{\pgfqpoint{0.696164in}{0.948634in}}%
\pgfpathlineto{\pgfqpoint{0.696645in}{0.949684in}}%
\pgfpathlineto{\pgfqpoint{0.697756in}{0.954164in}}%
\pgfpathlineto{\pgfqpoint{0.697921in}{0.955144in}}%
\pgfpathlineto{\pgfqpoint{0.699015in}{0.960464in}}%
\pgfpathlineto{\pgfqpoint{0.699314in}{0.961304in}}%
\pgfpathlineto{\pgfqpoint{0.700425in}{0.965294in}}%
\pgfpathlineto{\pgfqpoint{0.700657in}{0.966344in}}%
\pgfpathlineto{\pgfqpoint{0.701767in}{0.971314in}}%
\pgfpathlineto{\pgfqpoint{0.702049in}{0.972364in}}%
\pgfpathlineto{\pgfqpoint{0.703094in}{0.977194in}}%
\pgfpathlineto{\pgfqpoint{0.703442in}{0.978104in}}%
\pgfpathlineto{\pgfqpoint{0.704536in}{0.982514in}}%
\pgfpathlineto{\pgfqpoint{0.704917in}{0.983564in}}%
\pgfpathlineto{\pgfqpoint{0.706011in}{0.988464in}}%
\pgfpathlineto{\pgfqpoint{0.706160in}{0.989444in}}%
\pgfpathlineto{\pgfqpoint{0.707271in}{0.993504in}}%
\pgfpathlineto{\pgfqpoint{0.707619in}{0.994414in}}%
\pgfpathlineto{\pgfqpoint{0.708697in}{0.998054in}}%
\pgfpathlineto{\pgfqpoint{0.709061in}{0.999104in}}%
\pgfpathlineto{\pgfqpoint{0.710172in}{1.004564in}}%
\pgfpathlineto{\pgfqpoint{0.710504in}{1.005614in}}%
\pgfpathlineto{\pgfqpoint{0.711614in}{1.009604in}}%
\pgfpathlineto{\pgfqpoint{0.711830in}{1.010584in}}%
\pgfpathlineto{\pgfqpoint{0.712874in}{1.015134in}}%
\pgfpathlineto{\pgfqpoint{0.713123in}{1.016184in}}%
\pgfpathlineto{\pgfqpoint{0.714184in}{1.020244in}}%
\pgfpathlineto{\pgfqpoint{0.714598in}{1.021294in}}%
\pgfpathlineto{\pgfqpoint{0.715709in}{1.024304in}}%
\pgfpathlineto{\pgfqpoint{0.716173in}{1.025354in}}%
\pgfpathlineto{\pgfqpoint{0.717284in}{1.029624in}}%
\pgfpathlineto{\pgfqpoint{0.717466in}{1.030604in}}%
\pgfpathlineto{\pgfqpoint{0.718560in}{1.035014in}}%
\pgfpathlineto{\pgfqpoint{0.718776in}{1.035994in}}%
\pgfpathlineto{\pgfqpoint{0.719853in}{1.041034in}}%
\pgfpathlineto{\pgfqpoint{0.720202in}{1.042084in}}%
\pgfpathlineto{\pgfqpoint{0.721312in}{1.046144in}}%
\pgfpathlineto{\pgfqpoint{0.721694in}{1.047194in}}%
\pgfpathlineto{\pgfqpoint{0.722788in}{1.050904in}}%
\pgfpathlineto{\pgfqpoint{0.723086in}{1.051954in}}%
\pgfpathlineto{\pgfqpoint{0.724180in}{1.056364in}}%
\pgfpathlineto{\pgfqpoint{0.724727in}{1.057414in}}%
\pgfpathlineto{\pgfqpoint{0.725805in}{1.061264in}}%
\pgfpathlineto{\pgfqpoint{0.726236in}{1.062314in}}%
\pgfpathlineto{\pgfqpoint{0.727280in}{1.065744in}}%
\pgfpathlineto{\pgfqpoint{0.727728in}{1.066794in}}%
\pgfpathlineto{\pgfqpoint{0.728805in}{1.069874in}}%
\pgfpathlineto{\pgfqpoint{0.729253in}{1.070924in}}%
\pgfpathlineto{\pgfqpoint{0.730364in}{1.074144in}}%
\pgfpathlineto{\pgfqpoint{0.730695in}{1.075124in}}%
\pgfpathlineto{\pgfqpoint{0.731789in}{1.079044in}}%
\pgfpathlineto{\pgfqpoint{0.732336in}{1.080024in}}%
\pgfpathlineto{\pgfqpoint{0.733447in}{1.084574in}}%
\pgfpathlineto{\pgfqpoint{0.734094in}{1.085624in}}%
\pgfpathlineto{\pgfqpoint{0.735188in}{1.089544in}}%
\pgfpathlineto{\pgfqpoint{0.735552in}{1.090594in}}%
\pgfpathlineto{\pgfqpoint{0.736597in}{1.093674in}}%
\pgfpathlineto{\pgfqpoint{0.737177in}{1.094724in}}%
\pgfpathlineto{\pgfqpoint{0.738221in}{1.097524in}}%
\pgfpathlineto{\pgfqpoint{0.738702in}{1.098574in}}%
\pgfpathlineto{\pgfqpoint{0.739780in}{1.102424in}}%
\pgfpathlineto{\pgfqpoint{0.740128in}{1.103474in}}%
\pgfpathlineto{\pgfqpoint{0.741238in}{1.106624in}}%
\pgfpathlineto{\pgfqpoint{0.741636in}{1.107674in}}%
\pgfpathlineto{\pgfqpoint{0.742697in}{1.111034in}}%
\pgfpathlineto{\pgfqpoint{0.743095in}{1.112084in}}%
\pgfpathlineto{\pgfqpoint{0.744206in}{1.115024in}}%
\pgfpathlineto{\pgfqpoint{0.744769in}{1.116074in}}%
\pgfpathlineto{\pgfqpoint{0.745051in}{1.117054in}}%
\pgfpathlineto{\pgfqpoint{0.745068in}{1.117054in}}%
\pgfpathlineto{\pgfqpoint{0.755943in}{1.118104in}}%
\pgfpathlineto{\pgfqpoint{0.757053in}{1.121814in}}%
\pgfpathlineto{\pgfqpoint{0.757600in}{1.122864in}}%
\pgfpathlineto{\pgfqpoint{0.758678in}{1.125384in}}%
\pgfpathlineto{\pgfqpoint{0.759225in}{1.126434in}}%
\pgfpathlineto{\pgfqpoint{0.760269in}{1.129094in}}%
\pgfpathlineto{\pgfqpoint{0.760817in}{1.130144in}}%
\pgfpathlineto{\pgfqpoint{0.761911in}{1.133224in}}%
\pgfpathlineto{\pgfqpoint{0.762474in}{1.134274in}}%
\pgfpathlineto{\pgfqpoint{0.763535in}{1.137844in}}%
\pgfpathlineto{\pgfqpoint{0.764165in}{1.138824in}}%
\pgfpathlineto{\pgfqpoint{0.765259in}{1.142324in}}%
\pgfpathlineto{\pgfqpoint{0.766055in}{1.143374in}}%
\pgfpathlineto{\pgfqpoint{0.767166in}{1.146314in}}%
\pgfpathlineto{\pgfqpoint{0.767696in}{1.147364in}}%
\pgfpathlineto{\pgfqpoint{0.768790in}{1.149954in}}%
\pgfpathlineto{\pgfqpoint{0.769603in}{1.151004in}}%
\pgfpathlineto{\pgfqpoint{0.770514in}{1.153314in}}%
\pgfpathlineto{\pgfqpoint{0.771161in}{1.154364in}}%
\pgfpathlineto{\pgfqpoint{0.772272in}{1.157374in}}%
\pgfpathlineto{\pgfqpoint{0.772852in}{1.158424in}}%
\pgfpathlineto{\pgfqpoint{0.773946in}{1.160664in}}%
\pgfpathlineto{\pgfqpoint{0.774609in}{1.161714in}}%
\pgfpathlineto{\pgfqpoint{0.775703in}{1.164794in}}%
\pgfpathlineto{\pgfqpoint{0.776267in}{1.165774in}}%
\pgfpathlineto{\pgfqpoint{0.777344in}{1.168714in}}%
\pgfpathlineto{\pgfqpoint{0.777759in}{1.169764in}}%
\pgfpathlineto{\pgfqpoint{0.778836in}{1.172144in}}%
\pgfpathlineto{\pgfqpoint{0.779350in}{1.173124in}}%
\pgfpathlineto{\pgfqpoint{0.780444in}{1.174524in}}%
\pgfpathlineto{\pgfqpoint{0.781008in}{1.175574in}}%
\pgfpathlineto{\pgfqpoint{0.782052in}{1.177254in}}%
\pgfpathlineto{\pgfqpoint{0.782699in}{1.178304in}}%
\pgfpathlineto{\pgfqpoint{0.783793in}{1.180894in}}%
\pgfpathlineto{\pgfqpoint{0.784290in}{1.181874in}}%
\pgfpathlineto{\pgfqpoint{0.785384in}{1.184044in}}%
\pgfpathlineto{\pgfqpoint{0.785981in}{1.185094in}}%
\pgfpathlineto{\pgfqpoint{0.787075in}{1.186914in}}%
\pgfpathlineto{\pgfqpoint{0.787755in}{1.187894in}}%
\pgfpathlineto{\pgfqpoint{0.788866in}{1.189784in}}%
\pgfpathlineto{\pgfqpoint{0.789297in}{1.190764in}}%
\pgfpathlineto{\pgfqpoint{0.790407in}{1.192794in}}%
\pgfpathlineto{\pgfqpoint{0.790689in}{1.193634in}}%
\pgfpathlineto{\pgfqpoint{0.791783in}{1.196084in}}%
\pgfpathlineto{\pgfqpoint{0.792347in}{1.197134in}}%
\pgfpathlineto{\pgfqpoint{0.793358in}{1.199724in}}%
\pgfpathlineto{\pgfqpoint{0.794054in}{1.200774in}}%
\pgfpathlineto{\pgfqpoint{0.795099in}{1.201964in}}%
\pgfpathlineto{\pgfqpoint{0.796044in}{1.203014in}}%
\pgfpathlineto{\pgfqpoint{0.797138in}{1.205534in}}%
\pgfpathlineto{\pgfqpoint{0.797718in}{1.206584in}}%
\pgfpathlineto{\pgfqpoint{0.798812in}{1.208474in}}%
\pgfpathlineto{\pgfqpoint{0.799409in}{1.209524in}}%
\pgfpathlineto{\pgfqpoint{0.800520in}{1.212114in}}%
\pgfpathlineto{\pgfqpoint{0.801382in}{1.213094in}}%
\pgfpathlineto{\pgfqpoint{0.802492in}{1.215894in}}%
\pgfpathlineto{\pgfqpoint{0.803106in}{1.216664in}}%
\pgfpathlineto{\pgfqpoint{0.804134in}{1.218554in}}%
\pgfpathlineto{\pgfqpoint{0.805128in}{1.219604in}}%
\pgfpathlineto{\pgfqpoint{0.806222in}{1.221774in}}%
\pgfpathlineto{\pgfqpoint{0.806786in}{1.222614in}}%
\pgfpathlineto{\pgfqpoint{0.807897in}{1.225484in}}%
\pgfpathlineto{\pgfqpoint{0.808659in}{1.226534in}}%
\pgfpathlineto{\pgfqpoint{0.809753in}{1.228284in}}%
\pgfpathlineto{\pgfqpoint{0.810267in}{1.229334in}}%
\pgfpathlineto{\pgfqpoint{0.811378in}{1.231224in}}%
\pgfpathlineto{\pgfqpoint{0.812654in}{1.232274in}}%
\pgfpathlineto{\pgfqpoint{0.813682in}{1.233744in}}%
\pgfpathlineto{\pgfqpoint{0.814180in}{1.234794in}}%
\pgfpathlineto{\pgfqpoint{0.815290in}{1.236614in}}%
\pgfpathlineto{\pgfqpoint{0.816053in}{1.237664in}}%
\pgfpathlineto{\pgfqpoint{0.816948in}{1.239484in}}%
\pgfpathlineto{\pgfqpoint{0.818109in}{1.240464in}}%
\pgfpathlineto{\pgfqpoint{0.819203in}{1.242424in}}%
\pgfpathlineto{\pgfqpoint{0.819949in}{1.243404in}}%
\pgfpathlineto{\pgfqpoint{0.821043in}{1.245364in}}%
\pgfpathlineto{\pgfqpoint{0.821838in}{1.246414in}}%
\pgfpathlineto{\pgfqpoint{0.822899in}{1.248374in}}%
\pgfpathlineto{\pgfqpoint{0.823828in}{1.249424in}}%
\pgfpathlineto{\pgfqpoint{0.824905in}{1.251314in}}%
\pgfpathlineto{\pgfqpoint{0.826049in}{1.252364in}}%
\pgfpathlineto{\pgfqpoint{0.827110in}{1.254044in}}%
\pgfpathlineto{\pgfqpoint{0.828204in}{1.255024in}}%
\pgfpathlineto{\pgfqpoint{0.829298in}{1.257404in}}%
\pgfpathlineto{\pgfqpoint{0.830094in}{1.258454in}}%
\pgfpathlineto{\pgfqpoint{0.831188in}{1.259784in}}%
\pgfpathlineto{\pgfqpoint{0.831901in}{1.260834in}}%
\pgfpathlineto{\pgfqpoint{0.832979in}{1.261954in}}%
\pgfpathlineto{\pgfqpoint{0.833841in}{1.263004in}}%
\pgfpathlineto{\pgfqpoint{0.834769in}{1.263774in}}%
\pgfpathlineto{\pgfqpoint{0.835697in}{1.264824in}}%
\pgfpathlineto{\pgfqpoint{0.836659in}{1.266014in}}%
\pgfpathlineto{\pgfqpoint{0.838234in}{1.267064in}}%
\pgfpathlineto{\pgfqpoint{0.839295in}{1.268814in}}%
\pgfpathlineto{\pgfqpoint{0.840438in}{1.269864in}}%
\pgfpathlineto{\pgfqpoint{0.841549in}{1.271824in}}%
\pgfpathlineto{\pgfqpoint{0.842594in}{1.272874in}}%
\pgfpathlineto{\pgfqpoint{0.843688in}{1.274414in}}%
\pgfpathlineto{\pgfqpoint{0.844881in}{1.275464in}}%
\pgfpathlineto{\pgfqpoint{0.845992in}{1.277844in}}%
\pgfpathlineto{\pgfqpoint{0.846705in}{1.278894in}}%
\pgfpathlineto{\pgfqpoint{0.847633in}{1.280294in}}%
\pgfpathlineto{\pgfqpoint{0.848644in}{1.281274in}}%
\pgfpathlineto{\pgfqpoint{0.849606in}{1.282954in}}%
\pgfpathlineto{\pgfqpoint{0.850882in}{1.284004in}}%
\pgfpathlineto{\pgfqpoint{0.851960in}{1.285194in}}%
\pgfpathlineto{\pgfqpoint{0.853120in}{1.286244in}}%
\pgfpathlineto{\pgfqpoint{0.854198in}{1.287644in}}%
\pgfpathlineto{\pgfqpoint{0.855176in}{1.288694in}}%
\pgfpathlineto{\pgfqpoint{0.856237in}{1.290234in}}%
\pgfpathlineto{\pgfqpoint{0.856917in}{1.291144in}}%
\pgfpathlineto{\pgfqpoint{0.858027in}{1.292754in}}%
\pgfpathlineto{\pgfqpoint{0.859287in}{1.293804in}}%
\pgfpathlineto{\pgfqpoint{0.860232in}{1.294714in}}%
\pgfpathlineto{\pgfqpoint{0.861160in}{1.295764in}}%
\pgfpathlineto{\pgfqpoint{0.862155in}{1.296814in}}%
\pgfpathlineto{\pgfqpoint{0.863746in}{1.297794in}}%
\pgfpathlineto{\pgfqpoint{0.864807in}{1.299054in}}%
\pgfpathlineto{\pgfqpoint{0.866084in}{1.300104in}}%
\pgfpathlineto{\pgfqpoint{0.867128in}{1.301854in}}%
\pgfpathlineto{\pgfqpoint{0.867775in}{1.302904in}}%
\pgfpathlineto{\pgfqpoint{0.868886in}{1.304164in}}%
\pgfpathlineto{\pgfqpoint{0.869897in}{1.305144in}}%
\pgfpathlineto{\pgfqpoint{0.870991in}{1.306614in}}%
\pgfpathlineto{\pgfqpoint{0.872218in}{1.307664in}}%
\pgfpathlineto{\pgfqpoint{0.873312in}{1.309624in}}%
\pgfpathlineto{\pgfqpoint{0.874572in}{1.310674in}}%
\pgfpathlineto{\pgfqpoint{0.875616in}{1.312284in}}%
\pgfpathlineto{\pgfqpoint{0.876263in}{1.313334in}}%
\pgfpathlineto{\pgfqpoint{0.877373in}{1.314804in}}%
\pgfpathlineto{\pgfqpoint{0.878036in}{1.315854in}}%
\pgfpathlineto{\pgfqpoint{0.879081in}{1.316974in}}%
\pgfpathlineto{\pgfqpoint{0.880191in}{1.318024in}}%
\pgfpathlineto{\pgfqpoint{0.881252in}{1.319564in}}%
\pgfpathlineto{\pgfqpoint{0.882927in}{1.320544in}}%
\pgfpathlineto{\pgfqpoint{0.883954in}{1.321384in}}%
\pgfpathlineto{\pgfqpoint{0.885165in}{1.322434in}}%
\pgfpathlineto{\pgfqpoint{0.886226in}{1.323624in}}%
\pgfpathlineto{\pgfqpoint{0.887121in}{1.324674in}}%
\pgfpathlineto{\pgfqpoint{0.888215in}{1.325864in}}%
\pgfpathlineto{\pgfqpoint{0.889624in}{1.326914in}}%
\pgfpathlineto{\pgfqpoint{0.890619in}{1.328104in}}%
\pgfpathlineto{\pgfqpoint{0.892044in}{1.329154in}}%
\pgfpathlineto{\pgfqpoint{0.892857in}{1.330064in}}%
\pgfpathlineto{\pgfqpoint{0.894100in}{1.331044in}}%
\pgfpathlineto{\pgfqpoint{0.894746in}{1.331814in}}%
\pgfpathlineto{\pgfqpoint{0.896089in}{1.332864in}}%
\pgfpathlineto{\pgfqpoint{0.897200in}{1.333984in}}%
\pgfpathlineto{\pgfqpoint{0.898327in}{1.335034in}}%
\pgfpathlineto{\pgfqpoint{0.899338in}{1.336714in}}%
\pgfpathlineto{\pgfqpoint{0.900267in}{1.337694in}}%
\pgfpathlineto{\pgfqpoint{0.901311in}{1.338534in}}%
\pgfpathlineto{\pgfqpoint{0.902554in}{1.339584in}}%
\pgfpathlineto{\pgfqpoint{0.903649in}{1.340494in}}%
\pgfpathlineto{\pgfqpoint{0.905240in}{1.341544in}}%
\pgfpathlineto{\pgfqpoint{0.906334in}{1.342874in}}%
\pgfpathlineto{\pgfqpoint{0.907743in}{1.343924in}}%
\pgfpathlineto{\pgfqpoint{0.908854in}{1.344764in}}%
\pgfpathlineto{\pgfqpoint{0.910048in}{1.345814in}}%
\pgfpathlineto{\pgfqpoint{0.911142in}{1.346654in}}%
\pgfpathlineto{\pgfqpoint{0.912169in}{1.347704in}}%
\pgfpathlineto{\pgfqpoint{0.913114in}{1.348754in}}%
\pgfpathlineto{\pgfqpoint{0.914242in}{1.349804in}}%
\pgfpathlineto{\pgfqpoint{0.915004in}{1.350924in}}%
\pgfpathlineto{\pgfqpoint{0.916894in}{1.351974in}}%
\pgfpathlineto{\pgfqpoint{0.917988in}{1.353234in}}%
\pgfpathlineto{\pgfqpoint{0.919646in}{1.354284in}}%
\pgfpathlineto{\pgfqpoint{0.920757in}{1.356034in}}%
\pgfpathlineto{\pgfqpoint{0.922166in}{1.357084in}}%
\pgfpathlineto{\pgfqpoint{0.923260in}{1.358204in}}%
\pgfpathlineto{\pgfqpoint{0.925332in}{1.359254in}}%
\pgfpathlineto{\pgfqpoint{0.926376in}{1.360164in}}%
\pgfpathlineto{\pgfqpoint{0.927686in}{1.361214in}}%
\pgfpathlineto{\pgfqpoint{0.928780in}{1.361914in}}%
\pgfpathlineto{\pgfqpoint{0.930886in}{1.362964in}}%
\pgfpathlineto{\pgfqpoint{0.931980in}{1.364294in}}%
\pgfpathlineto{\pgfqpoint{0.932941in}{1.365344in}}%
\pgfpathlineto{\pgfqpoint{0.934035in}{1.366604in}}%
\pgfpathlineto{\pgfqpoint{0.935925in}{1.367654in}}%
\pgfpathlineto{\pgfqpoint{0.937003in}{1.368984in}}%
\pgfpathlineto{\pgfqpoint{0.939307in}{1.370034in}}%
\pgfpathlineto{\pgfqpoint{0.940302in}{1.371014in}}%
\pgfpathlineto{\pgfqpoint{0.941876in}{1.372064in}}%
\pgfpathlineto{\pgfqpoint{0.942871in}{1.372694in}}%
\pgfpathlineto{\pgfqpoint{0.944562in}{1.373674in}}%
\pgfpathlineto{\pgfqpoint{0.945573in}{1.374934in}}%
\pgfpathlineto{\pgfqpoint{0.947314in}{1.375984in}}%
\pgfpathlineto{\pgfqpoint{0.948159in}{1.376824in}}%
\pgfpathlineto{\pgfqpoint{0.950663in}{1.377874in}}%
\pgfpathlineto{\pgfqpoint{0.951707in}{1.378504in}}%
\pgfpathlineto{\pgfqpoint{0.953083in}{1.379484in}}%
\pgfpathlineto{\pgfqpoint{0.954127in}{1.380394in}}%
\pgfpathlineto{\pgfqpoint{0.956183in}{1.381444in}}%
\pgfpathlineto{\pgfqpoint{0.957244in}{1.382914in}}%
\pgfpathlineto{\pgfqpoint{0.958736in}{1.383964in}}%
\pgfpathlineto{\pgfqpoint{0.959714in}{1.384874in}}%
\pgfpathlineto{\pgfqpoint{0.961902in}{1.385924in}}%
\pgfpathlineto{\pgfqpoint{0.962781in}{1.386414in}}%
\pgfpathlineto{\pgfqpoint{0.965599in}{1.387464in}}%
\pgfpathlineto{\pgfqpoint{0.966643in}{1.388514in}}%
\pgfpathlineto{\pgfqpoint{0.968069in}{1.389564in}}%
\pgfpathlineto{\pgfqpoint{0.968931in}{1.390404in}}%
\pgfpathlineto{\pgfqpoint{0.971733in}{1.391454in}}%
\pgfpathlineto{\pgfqpoint{0.972843in}{1.391874in}}%
\pgfpathlineto{\pgfqpoint{0.974286in}{1.392924in}}%
\pgfpathlineto{\pgfqpoint{0.975197in}{1.393554in}}%
\pgfpathlineto{\pgfqpoint{0.977336in}{1.394604in}}%
\pgfpathlineto{\pgfqpoint{0.978413in}{1.395654in}}%
\pgfpathlineto{\pgfqpoint{0.979607in}{1.396704in}}%
\pgfpathlineto{\pgfqpoint{0.980635in}{1.397614in}}%
\pgfpathlineto{\pgfqpoint{0.981729in}{1.398664in}}%
\pgfpathlineto{\pgfqpoint{0.982690in}{1.399224in}}%
\pgfpathlineto{\pgfqpoint{0.984365in}{1.400274in}}%
\pgfpathlineto{\pgfqpoint{0.985343in}{1.401044in}}%
\pgfpathlineto{\pgfqpoint{0.987050in}{1.402094in}}%
\pgfpathlineto{\pgfqpoint{0.987912in}{1.403004in}}%
\pgfpathlineto{\pgfqpoint{0.990051in}{1.404054in}}%
\pgfpathlineto{\pgfqpoint{0.990631in}{1.404474in}}%
\pgfpathlineto{\pgfqpoint{0.992935in}{1.405524in}}%
\pgfpathlineto{\pgfqpoint{0.993847in}{1.406294in}}%
\pgfpathlineto{\pgfqpoint{0.995787in}{1.407344in}}%
\pgfpathlineto{\pgfqpoint{0.996798in}{1.408534in}}%
\pgfpathlineto{\pgfqpoint{0.998456in}{1.409444in}}%
\pgfpathlineto{\pgfqpoint{0.999533in}{1.410424in}}%
\pgfpathlineto{\pgfqpoint{1.001224in}{1.411404in}}%
\pgfpathlineto{\pgfqpoint{1.002235in}{1.412314in}}%
\pgfpathlineto{\pgfqpoint{1.003545in}{1.413364in}}%
\pgfpathlineto{\pgfqpoint{1.004423in}{1.414274in}}%
\pgfpathlineto{\pgfqpoint{1.005882in}{1.415324in}}%
\pgfpathlineto{\pgfqpoint{1.006678in}{1.415954in}}%
\pgfpathlineto{\pgfqpoint{1.009165in}{1.417004in}}%
\pgfpathlineto{\pgfqpoint{1.010226in}{1.417774in}}%
\pgfpathlineto{\pgfqpoint{1.012861in}{1.418824in}}%
\pgfpathlineto{\pgfqpoint{1.013972in}{1.419524in}}%
\pgfpathlineto{\pgfqpoint{1.016442in}{1.420504in}}%
\pgfpathlineto{\pgfqpoint{1.017437in}{1.421274in}}%
\pgfpathlineto{\pgfqpoint{1.019890in}{1.422324in}}%
\pgfpathlineto{\pgfqpoint{1.020852in}{1.423444in}}%
\pgfpathlineto{\pgfqpoint{1.022626in}{1.424494in}}%
\pgfpathlineto{\pgfqpoint{1.023488in}{1.425334in}}%
\pgfpathlineto{\pgfqpoint{1.026405in}{1.426384in}}%
\pgfpathlineto{\pgfqpoint{1.027516in}{1.427224in}}%
\pgfpathlineto{\pgfqpoint{1.029787in}{1.428274in}}%
\pgfpathlineto{\pgfqpoint{1.030831in}{1.429184in}}%
\pgfpathlineto{\pgfqpoint{1.032887in}{1.430234in}}%
\pgfpathlineto{\pgfqpoint{1.033882in}{1.431074in}}%
\pgfpathlineto{\pgfqpoint{1.036667in}{1.432124in}}%
\pgfpathlineto{\pgfqpoint{1.037711in}{1.432894in}}%
\pgfpathlineto{\pgfqpoint{1.041872in}{1.433944in}}%
\pgfpathlineto{\pgfqpoint{1.042834in}{1.434504in}}%
\pgfpathlineto{\pgfqpoint{1.046530in}{1.435554in}}%
\pgfpathlineto{\pgfqpoint{1.047575in}{1.436044in}}%
\pgfpathlineto{\pgfqpoint{1.049514in}{1.437094in}}%
\pgfpathlineto{\pgfqpoint{1.050625in}{1.437794in}}%
\pgfpathlineto{\pgfqpoint{1.053095in}{1.438844in}}%
\pgfpathlineto{\pgfqpoint{1.054090in}{1.439544in}}%
\pgfpathlineto{\pgfqpoint{1.056692in}{1.440594in}}%
\pgfpathlineto{\pgfqpoint{1.057704in}{1.441364in}}%
\pgfpathlineto{\pgfqpoint{1.061301in}{1.442414in}}%
\pgfpathlineto{\pgfqpoint{1.062329in}{1.443044in}}%
\pgfpathlineto{\pgfqpoint{1.065893in}{1.444094in}}%
\pgfpathlineto{\pgfqpoint{1.066987in}{1.444584in}}%
\pgfpathlineto{\pgfqpoint{1.068976in}{1.445634in}}%
\pgfpathlineto{\pgfqpoint{1.070037in}{1.446544in}}%
\pgfpathlineto{\pgfqpoint{1.072491in}{1.447594in}}%
\pgfpathlineto{\pgfqpoint{1.073369in}{1.448224in}}%
\pgfpathlineto{\pgfqpoint{1.075840in}{1.449274in}}%
\pgfpathlineto{\pgfqpoint{1.076652in}{1.449694in}}%
\pgfpathlineto{\pgfqpoint{1.079006in}{1.450744in}}%
\pgfpathlineto{\pgfqpoint{1.079967in}{1.451234in}}%
\pgfpathlineto{\pgfqpoint{1.083001in}{1.452214in}}%
\pgfpathlineto{\pgfqpoint{1.084112in}{1.452914in}}%
\pgfpathlineto{\pgfqpoint{1.087726in}{1.453894in}}%
\pgfpathlineto{\pgfqpoint{1.088438in}{1.454384in}}%
\pgfpathlineto{\pgfqpoint{1.092119in}{1.455434in}}%
\pgfpathlineto{\pgfqpoint{1.093064in}{1.456064in}}%
\pgfpathlineto{\pgfqpoint{1.096346in}{1.457114in}}%
\pgfpathlineto{\pgfqpoint{1.097274in}{1.457674in}}%
\pgfpathlineto{\pgfqpoint{1.100092in}{1.458724in}}%
\pgfpathlineto{\pgfqpoint{1.100689in}{1.459004in}}%
\pgfpathlineto{\pgfqpoint{1.104038in}{1.460054in}}%
\pgfpathlineto{\pgfqpoint{1.105149in}{1.460404in}}%
\pgfpathlineto{\pgfqpoint{1.109989in}{1.461454in}}%
\pgfpathlineto{\pgfqpoint{1.111083in}{1.462154in}}%
\pgfpathlineto{\pgfqpoint{1.114813in}{1.463204in}}%
\pgfpathlineto{\pgfqpoint{1.115858in}{1.463624in}}%
\pgfpathlineto{\pgfqpoint{1.119389in}{1.464674in}}%
\pgfpathlineto{\pgfqpoint{1.120433in}{1.465094in}}%
\pgfpathlineto{\pgfqpoint{1.122754in}{1.466144in}}%
\pgfpathlineto{\pgfqpoint{1.123732in}{1.466494in}}%
\pgfpathlineto{\pgfqpoint{1.127910in}{1.467544in}}%
\pgfpathlineto{\pgfqpoint{1.128556in}{1.468034in}}%
\pgfpathlineto{\pgfqpoint{1.131010in}{1.469084in}}%
\pgfpathlineto{\pgfqpoint{1.131822in}{1.469924in}}%
\pgfpathlineto{\pgfqpoint{1.135187in}{1.470974in}}%
\pgfpathlineto{\pgfqpoint{1.135204in}{1.471184in}}%
\pgfpathlineto{\pgfqpoint{1.140956in}{1.472234in}}%
\pgfpathlineto{\pgfqpoint{1.141735in}{1.472584in}}%
\pgfpathlineto{\pgfqpoint{1.145598in}{1.473634in}}%
\pgfpathlineto{\pgfqpoint{1.146675in}{1.474194in}}%
\pgfpathlineto{\pgfqpoint{1.150803in}{1.475244in}}%
\pgfpathlineto{\pgfqpoint{1.151516in}{1.475874in}}%
\pgfpathlineto{\pgfqpoint{1.156058in}{1.476924in}}%
\pgfpathlineto{\pgfqpoint{1.157152in}{1.477414in}}%
\pgfpathlineto{\pgfqpoint{1.162407in}{1.478464in}}%
\pgfpathlineto{\pgfqpoint{1.163452in}{1.479094in}}%
\pgfpathlineto{\pgfqpoint{1.168060in}{1.480144in}}%
\pgfpathlineto{\pgfqpoint{1.168873in}{1.480494in}}%
\pgfpathlineto{\pgfqpoint{1.171923in}{1.481474in}}%
\pgfpathlineto{\pgfqpoint{1.172835in}{1.481754in}}%
\pgfpathlineto{\pgfqpoint{1.177161in}{1.482804in}}%
\pgfpathlineto{\pgfqpoint{1.178156in}{1.483294in}}%
\pgfpathlineto{\pgfqpoint{1.181223in}{1.484344in}}%
\pgfpathlineto{\pgfqpoint{1.181273in}{1.484554in}}%
\pgfpathlineto{\pgfqpoint{1.188633in}{1.485604in}}%
\pgfpathlineto{\pgfqpoint{1.189512in}{1.485884in}}%
\pgfpathlineto{\pgfqpoint{1.193341in}{1.486934in}}%
\pgfpathlineto{\pgfqpoint{1.194203in}{1.487354in}}%
\pgfpathlineto{\pgfqpoint{1.198928in}{1.488404in}}%
\pgfpathlineto{\pgfqpoint{1.199823in}{1.488964in}}%
\pgfpathlineto{\pgfqpoint{1.201895in}{1.489944in}}%
\pgfpathlineto{\pgfqpoint{1.202343in}{1.490224in}}%
\pgfpathlineto{\pgfqpoint{1.205343in}{1.491274in}}%
\pgfpathlineto{\pgfqpoint{1.206404in}{1.491554in}}%
\pgfpathlineto{\pgfqpoint{1.209670in}{1.492604in}}%
\pgfpathlineto{\pgfqpoint{1.210748in}{1.493024in}}%
\pgfpathlineto{\pgfqpoint{1.215887in}{1.494074in}}%
\pgfpathlineto{\pgfqpoint{1.216865in}{1.494214in}}%
\pgfpathlineto{\pgfqpoint{1.223860in}{1.495264in}}%
\pgfpathlineto{\pgfqpoint{1.224739in}{1.495684in}}%
\pgfpathlineto{\pgfqpoint{1.229994in}{1.496734in}}%
\pgfpathlineto{\pgfqpoint{1.230823in}{1.497014in}}%
\pgfpathlineto{\pgfqpoint{1.234354in}{1.498064in}}%
\pgfpathlineto{\pgfqpoint{1.235117in}{1.498344in}}%
\pgfpathlineto{\pgfqpoint{1.240256in}{1.499394in}}%
\pgfpathlineto{\pgfqpoint{1.241151in}{1.499674in}}%
\pgfpathlineto{\pgfqpoint{1.245610in}{1.500724in}}%
\pgfpathlineto{\pgfqpoint{1.246721in}{1.501144in}}%
\pgfpathlineto{\pgfqpoint{1.248942in}{1.502194in}}%
\pgfpathlineto{\pgfqpoint{1.250053in}{1.502754in}}%
\pgfpathlineto{\pgfqpoint{1.254429in}{1.503804in}}%
\pgfpathlineto{\pgfqpoint{1.254728in}{1.504084in}}%
\pgfpathlineto{\pgfqpoint{1.260845in}{1.505134in}}%
\pgfpathlineto{\pgfqpoint{1.261723in}{1.505414in}}%
\pgfpathlineto{\pgfqpoint{1.266697in}{1.506464in}}%
\pgfpathlineto{\pgfqpoint{1.267807in}{1.506744in}}%
\pgfpathlineto{\pgfqpoint{1.271338in}{1.507724in}}%
\pgfpathlineto{\pgfqpoint{1.272200in}{1.508214in}}%
\pgfpathlineto{\pgfqpoint{1.277936in}{1.509264in}}%
\pgfpathlineto{\pgfqpoint{1.278036in}{1.509404in}}%
\pgfpathlineto{\pgfqpoint{1.284203in}{1.510454in}}%
\pgfpathlineto{\pgfqpoint{1.285131in}{1.510804in}}%
\pgfpathlineto{\pgfqpoint{1.289657in}{1.511854in}}%
\pgfpathlineto{\pgfqpoint{1.290585in}{1.512274in}}%
\pgfpathlineto{\pgfqpoint{1.295940in}{1.513324in}}%
\pgfpathlineto{\pgfqpoint{1.296354in}{1.513534in}}%
\pgfpathlineto{\pgfqpoint{1.299703in}{1.514584in}}%
\pgfpathlineto{\pgfqpoint{1.300664in}{1.514934in}}%
\pgfpathlineto{\pgfqpoint{1.305389in}{1.515984in}}%
\pgfpathlineto{\pgfqpoint{1.306118in}{1.516334in}}%
\pgfpathlineto{\pgfqpoint{1.311290in}{1.517384in}}%
\pgfpathlineto{\pgfqpoint{1.312318in}{1.517734in}}%
\pgfpathlineto{\pgfqpoint{1.318104in}{1.518784in}}%
\pgfpathlineto{\pgfqpoint{1.318899in}{1.519274in}}%
\pgfpathlineto{\pgfqpoint{1.323873in}{1.520324in}}%
\pgfpathlineto{\pgfqpoint{1.324950in}{1.520674in}}%
\pgfpathlineto{\pgfqpoint{1.329575in}{1.521654in}}%
\pgfpathlineto{\pgfqpoint{1.330305in}{1.521864in}}%
\pgfpathlineto{\pgfqpoint{1.336521in}{1.522844in}}%
\pgfpathlineto{\pgfqpoint{1.336836in}{1.523194in}}%
\pgfpathlineto{\pgfqpoint{1.344180in}{1.524244in}}%
\pgfpathlineto{\pgfqpoint{1.345191in}{1.524524in}}%
\pgfpathlineto{\pgfqpoint{1.351905in}{1.525574in}}%
\pgfpathlineto{\pgfqpoint{1.352602in}{1.525994in}}%
\pgfpathlineto{\pgfqpoint{1.359050in}{1.527044in}}%
\pgfpathlineto{\pgfqpoint{1.359962in}{1.527394in}}%
\pgfpathlineto{\pgfqpoint{1.365897in}{1.528444in}}%
\pgfpathlineto{\pgfqpoint{1.366991in}{1.528654in}}%
\pgfpathlineto{\pgfqpoint{1.378463in}{1.529704in}}%
\pgfpathlineto{\pgfqpoint{1.379092in}{1.530054in}}%
\pgfpathlineto{\pgfqpoint{1.386221in}{1.531104in}}%
\pgfpathlineto{\pgfqpoint{1.387298in}{1.531314in}}%
\pgfpathlineto{\pgfqpoint{1.393366in}{1.532364in}}%
\pgfpathlineto{\pgfqpoint{1.393880in}{1.532714in}}%
\pgfpathlineto{\pgfqpoint{1.403760in}{1.533764in}}%
\pgfpathlineto{\pgfqpoint{1.404340in}{1.533904in}}%
\pgfpathlineto{\pgfqpoint{1.412529in}{1.534954in}}%
\pgfpathlineto{\pgfqpoint{1.412994in}{1.535164in}}%
\pgfpathlineto{\pgfqpoint{1.422277in}{1.536214in}}%
\pgfpathlineto{\pgfqpoint{1.422807in}{1.536354in}}%
\pgfpathlineto{\pgfqpoint{1.428460in}{1.537404in}}%
\pgfpathlineto{\pgfqpoint{1.429157in}{1.537614in}}%
\pgfpathlineto{\pgfqpoint{1.435340in}{1.538664in}}%
\pgfpathlineto{\pgfqpoint{1.436351in}{1.539014in}}%
\pgfpathlineto{\pgfqpoint{1.443861in}{1.540064in}}%
\pgfpathlineto{\pgfqpoint{1.443877in}{1.540274in}}%
\pgfpathlineto{\pgfqpoint{1.450940in}{1.541324in}}%
\pgfpathlineto{\pgfqpoint{1.451006in}{1.541464in}}%
\pgfpathlineto{\pgfqpoint{1.462776in}{1.542514in}}%
\pgfpathlineto{\pgfqpoint{1.463439in}{1.542724in}}%
\pgfpathlineto{\pgfqpoint{1.473817in}{1.543774in}}%
\pgfpathlineto{\pgfqpoint{1.474347in}{1.544124in}}%
\pgfpathlineto{\pgfqpoint{1.482304in}{1.545174in}}%
\pgfpathlineto{\pgfqpoint{1.482669in}{1.545314in}}%
\pgfpathlineto{\pgfqpoint{1.493892in}{1.546364in}}%
\pgfpathlineto{\pgfqpoint{1.494257in}{1.546504in}}%
\pgfpathlineto{\pgfqpoint{1.503756in}{1.547554in}}%
\pgfpathlineto{\pgfqpoint{1.503921in}{1.547834in}}%
\pgfpathlineto{\pgfqpoint{1.514382in}{1.548884in}}%
\pgfpathlineto{\pgfqpoint{1.515227in}{1.549234in}}%
\pgfpathlineto{\pgfqpoint{1.524046in}{1.550284in}}%
\pgfpathlineto{\pgfqpoint{1.524378in}{1.550564in}}%
\pgfpathlineto{\pgfqpoint{1.535187in}{1.551614in}}%
\pgfpathlineto{\pgfqpoint{1.535817in}{1.551754in}}%
\pgfpathlineto{\pgfqpoint{1.546012in}{1.552804in}}%
\pgfpathlineto{\pgfqpoint{1.547122in}{1.553154in}}%
\pgfpathlineto{\pgfqpoint{1.557450in}{1.554204in}}%
\pgfpathlineto{\pgfqpoint{1.557450in}{1.554274in}}%
\pgfpathlineto{\pgfqpoint{1.571690in}{1.555324in}}%
\pgfpathlineto{\pgfqpoint{1.571790in}{1.555534in}}%
\pgfpathlineto{\pgfqpoint{1.581405in}{1.556584in}}%
\pgfpathlineto{\pgfqpoint{1.581653in}{1.556794in}}%
\pgfpathlineto{\pgfqpoint{1.595396in}{1.557844in}}%
\pgfpathlineto{\pgfqpoint{1.595844in}{1.558054in}}%
\pgfpathlineto{\pgfqpoint{1.606503in}{1.559104in}}%
\pgfpathlineto{\pgfqpoint{1.607514in}{1.559384in}}%
\pgfpathlineto{\pgfqpoint{1.617610in}{1.560434in}}%
\pgfpathlineto{\pgfqpoint{1.618721in}{1.560644in}}%
\pgfpathlineto{\pgfqpoint{1.630889in}{1.561694in}}%
\pgfpathlineto{\pgfqpoint{1.631883in}{1.561834in}}%
\pgfpathlineto{\pgfqpoint{1.652605in}{1.562884in}}%
\pgfpathlineto{\pgfqpoint{1.653600in}{1.563024in}}%
\pgfpathlineto{\pgfqpoint{1.663812in}{1.564074in}}%
\pgfpathlineto{\pgfqpoint{1.663861in}{1.564214in}}%
\pgfpathlineto{\pgfqpoint{1.672283in}{1.565264in}}%
\pgfpathlineto{\pgfqpoint{1.672382in}{1.565404in}}%
\pgfpathlineto{\pgfqpoint{1.683406in}{1.566454in}}%
\pgfpathlineto{\pgfqpoint{1.684467in}{1.566664in}}%
\pgfpathlineto{\pgfqpoint{1.696586in}{1.567714in}}%
\pgfpathlineto{\pgfqpoint{1.696834in}{1.567854in}}%
\pgfpathlineto{\pgfqpoint{1.710212in}{1.568904in}}%
\pgfpathlineto{\pgfqpoint{1.710909in}{1.569114in}}%
\pgfpathlineto{\pgfqpoint{1.723706in}{1.570164in}}%
\pgfpathlineto{\pgfqpoint{1.724718in}{1.570304in}}%
\pgfpathlineto{\pgfqpoint{1.739090in}{1.571354in}}%
\pgfpathlineto{\pgfqpoint{1.739422in}{1.571494in}}%
\pgfpathlineto{\pgfqpoint{1.756828in}{1.572544in}}%
\pgfpathlineto{\pgfqpoint{1.757591in}{1.572754in}}%
\pgfpathlineto{\pgfqpoint{1.769825in}{1.573804in}}%
\pgfpathlineto{\pgfqpoint{1.770339in}{1.573944in}}%
\pgfpathlineto{\pgfqpoint{1.782242in}{1.574994in}}%
\pgfpathlineto{\pgfqpoint{1.782374in}{1.575134in}}%
\pgfpathlineto{\pgfqpoint{1.795288in}{1.576184in}}%
\pgfpathlineto{\pgfqpoint{1.795868in}{1.576394in}}%
\pgfpathlineto{\pgfqpoint{1.809910in}{1.577444in}}%
\pgfpathlineto{\pgfqpoint{1.810987in}{1.577724in}}%
\pgfpathlineto{\pgfqpoint{1.828360in}{1.578774in}}%
\pgfpathlineto{\pgfqpoint{1.829057in}{1.578984in}}%
\pgfpathlineto{\pgfqpoint{1.842236in}{1.580034in}}%
\pgfpathlineto{\pgfqpoint{1.842368in}{1.580174in}}%
\pgfpathlineto{\pgfqpoint{1.859410in}{1.581224in}}%
\pgfpathlineto{\pgfqpoint{1.859891in}{1.581364in}}%
\pgfpathlineto{\pgfqpoint{1.873683in}{1.582414in}}%
\pgfpathlineto{\pgfqpoint{1.873683in}{1.582484in}}%
\pgfpathlineto{\pgfqpoint{1.894190in}{1.583534in}}%
\pgfpathlineto{\pgfqpoint{1.894637in}{1.583674in}}%
\pgfpathlineto{\pgfqpoint{1.915542in}{1.584724in}}%
\pgfpathlineto{\pgfqpoint{1.916155in}{1.584864in}}%
\pgfpathlineto{\pgfqpoint{1.929334in}{1.585914in}}%
\pgfpathlineto{\pgfqpoint{1.930345in}{1.586054in}}%
\pgfpathlineto{\pgfqpoint{1.945680in}{1.587104in}}%
\pgfpathlineto{\pgfqpoint{1.945680in}{1.587174in}}%
\pgfpathlineto{\pgfqpoint{1.958245in}{1.588224in}}%
\pgfpathlineto{\pgfqpoint{1.959074in}{1.588434in}}%
\pgfpathlineto{\pgfqpoint{1.971541in}{1.589484in}}%
\pgfpathlineto{\pgfqpoint{1.972005in}{1.589624in}}%
\pgfpathlineto{\pgfqpoint{1.980807in}{1.590674in}}%
\pgfpathlineto{\pgfqpoint{1.981636in}{1.590954in}}%
\pgfpathlineto{\pgfqpoint{1.988947in}{1.592004in}}%
\pgfpathlineto{\pgfqpoint{1.988947in}{1.592074in}}%
\pgfpathlineto{\pgfqpoint{1.998313in}{1.593124in}}%
\pgfpathlineto{\pgfqpoint{1.999391in}{1.593334in}}%
\pgfpathlineto{\pgfqpoint{2.006536in}{1.594384in}}%
\pgfpathlineto{\pgfqpoint{2.007414in}{1.594524in}}%
\pgfpathlineto{\pgfqpoint{2.013349in}{1.595574in}}%
\pgfpathlineto{\pgfqpoint{2.033126in}{1.601944in}}%
\pgfpathlineto{\pgfqpoint{2.033126in}{1.601944in}}%
\pgfusepath{stroke}%
\end{pgfscope}%
\begin{pgfscope}%
\pgfsetrectcap%
\pgfsetmiterjoin%
\pgfsetlinewidth{0.803000pt}%
\definecolor{currentstroke}{rgb}{0.000000,0.000000,0.000000}%
\pgfsetstrokecolor{currentstroke}%
\pgfsetdash{}{0pt}%
\pgfpathmoveto{\pgfqpoint{0.553581in}{0.499444in}}%
\pgfpathlineto{\pgfqpoint{0.553581in}{1.654444in}}%
\pgfusepath{stroke}%
\end{pgfscope}%
\begin{pgfscope}%
\pgfsetrectcap%
\pgfsetmiterjoin%
\pgfsetlinewidth{0.803000pt}%
\definecolor{currentstroke}{rgb}{0.000000,0.000000,0.000000}%
\pgfsetstrokecolor{currentstroke}%
\pgfsetdash{}{0pt}%
\pgfpathmoveto{\pgfqpoint{2.103581in}{0.499444in}}%
\pgfpathlineto{\pgfqpoint{2.103581in}{1.654444in}}%
\pgfusepath{stroke}%
\end{pgfscope}%
\begin{pgfscope}%
\pgfsetrectcap%
\pgfsetmiterjoin%
\pgfsetlinewidth{0.803000pt}%
\definecolor{currentstroke}{rgb}{0.000000,0.000000,0.000000}%
\pgfsetstrokecolor{currentstroke}%
\pgfsetdash{}{0pt}%
\pgfpathmoveto{\pgfqpoint{0.553581in}{0.499444in}}%
\pgfpathlineto{\pgfqpoint{2.103581in}{0.499444in}}%
\pgfusepath{stroke}%
\end{pgfscope}%
\begin{pgfscope}%
\pgfsetrectcap%
\pgfsetmiterjoin%
\pgfsetlinewidth{0.803000pt}%
\definecolor{currentstroke}{rgb}{0.000000,0.000000,0.000000}%
\pgfsetstrokecolor{currentstroke}%
\pgfsetdash{}{0pt}%
\pgfpathmoveto{\pgfqpoint{0.553581in}{1.654444in}}%
\pgfpathlineto{\pgfqpoint{2.103581in}{1.654444in}}%
\pgfusepath{stroke}%
\end{pgfscope}%
\begin{pgfscope}%
\pgfsetbuttcap%
\pgfsetmiterjoin%
\definecolor{currentfill}{rgb}{1.000000,1.000000,1.000000}%
\pgfsetfillcolor{currentfill}%
\pgfsetfillopacity{0.800000}%
\pgfsetlinewidth{1.003750pt}%
\definecolor{currentstroke}{rgb}{0.800000,0.800000,0.800000}%
\pgfsetstrokecolor{currentstroke}%
\pgfsetstrokeopacity{0.800000}%
\pgfsetdash{}{0pt}%
\pgfpathmoveto{\pgfqpoint{0.832747in}{0.568889in}}%
\pgfpathlineto{\pgfqpoint{2.006358in}{0.568889in}}%
\pgfpathquadraticcurveto{\pgfqpoint{2.034136in}{0.568889in}}{\pgfqpoint{2.034136in}{0.596666in}}%
\pgfpathlineto{\pgfqpoint{2.034136in}{0.776388in}}%
\pgfpathquadraticcurveto{\pgfqpoint{2.034136in}{0.804166in}}{\pgfqpoint{2.006358in}{0.804166in}}%
\pgfpathlineto{\pgfqpoint{0.832747in}{0.804166in}}%
\pgfpathquadraticcurveto{\pgfqpoint{0.804970in}{0.804166in}}{\pgfqpoint{0.804970in}{0.776388in}}%
\pgfpathlineto{\pgfqpoint{0.804970in}{0.596666in}}%
\pgfpathquadraticcurveto{\pgfqpoint{0.804970in}{0.568889in}}{\pgfqpoint{0.832747in}{0.568889in}}%
\pgfpathlineto{\pgfqpoint{0.832747in}{0.568889in}}%
\pgfpathclose%
\pgfusepath{stroke,fill}%
\end{pgfscope}%
\begin{pgfscope}%
\pgfsetrectcap%
\pgfsetroundjoin%
\pgfsetlinewidth{1.505625pt}%
\definecolor{currentstroke}{rgb}{0.000000,0.000000,0.000000}%
\pgfsetstrokecolor{currentstroke}%
\pgfsetdash{}{0pt}%
\pgfpathmoveto{\pgfqpoint{0.860525in}{0.700000in}}%
\pgfpathlineto{\pgfqpoint{0.999414in}{0.700000in}}%
\pgfpathlineto{\pgfqpoint{1.138303in}{0.700000in}}%
\pgfusepath{stroke}%
\end{pgfscope}%
\begin{pgfscope}%
\definecolor{textcolor}{rgb}{0.000000,0.000000,0.000000}%
\pgfsetstrokecolor{textcolor}%
\pgfsetfillcolor{textcolor}%
\pgftext[x=1.249414in,y=0.651388in,left,base]{\color{textcolor}\rmfamily\fontsize{10.000000}{12.000000}\selectfont AUC=0.840}%
\end{pgfscope}%
\end{pgfpicture}%
\makeatother%
\endgroup%

	
&
	\vskip 0pt
	\begin{tabular}{cc|c|c|}
	&\multicolumn{1}{c}{}& \multicolumn{2}{c}{Prediction} \\[0.4em]
	&\multicolumn{1}{c}{} & \multicolumn{1}{c}{N} & \multicolumn{1}{c}{P} \cr\cline{3-4}
	\multirow{2}{*}{\rotatebox[origin=c]{90}{Actual}}&N &
66.6\% & 18.4\%
	\vrule width 0pt height 10pt depth 2pt \cr\cline{3-4}
	&P & 
3.37\% & 11.6\%
	\vrule width 0pt height 10pt depth 2pt \cr\cline{3-4}
	\end{tabular}
% 0.839832287	
%66.636	18.357	3.373	11.634	
%0.78269595	0.387906242	0.775215056	0.518728987	

\hfil\begin{tabular}{ll}
	\cr
	0.783 & Accuracy\cr
	0.388 & Precision \cr
	0.775 & Recall \cr
	0.518 & F1 \cr
	0.840 & AUC \cr
\end{tabular}
\cr
\end{tabular}
} % End parbox

\

Note that the ROC curve does not change under these transformations.  

\

{\bf Choose a Different Decision Threshold}

\

The table below shows the number of elements of the negative and positive classes in bands of values of $p$.  

\

\

\verb|Ideal_Left_20|

\

\begin{tabular}{rrrrrrrrrrrrrrr}
\toprule
p &     Neg &    Pos & mPrec &      TN &      FP &      FN &      TP &  Prec &   Rec &  FP/P & $\hat{p}$ \\
\midrule
0.00 &       0 &      0 &   nan &       0 &  85,000 &       0 &  15,000 &  0.15 &  1.00 &  5.67 &      1.00 \\
0.05 &   2,481 &    137 &  0.05 &   2,481 &  82,519 &     137 &  14,863 &  0.15 &  0.99 &  5.50 &      0.97 \\
0.10 &  13,268 &    260 &  0.02 &  15,749 &  69,251 &     397 &  14,603 &  0.17 &  0.97 &  4.62 &      0.84 \\
0.15 &  19,408 &    469 &  0.02 &  35,157 &  49,843 &     866 &  14,134 &  0.22 &  0.94 &  3.32 &      0.64 \\
0.20 &  18,054 &    941 &  0.05 &  53,211 &  31,789 &   1,807 &  13,193 &  0.29 &  0.88 &  2.12 &      0.45 \\
0.25 &  13,262 &  1,599 &  0.11 &  66,473 &  18,527 &   3,406 &  11,594 &  0.38 &  0.77 &  1.24 &      0.30 \\
0.30 &   8,302 &  2,338 &  0.22 &  74,775 &  10,225 &   5,744 &   9,256 &  0.48 &  0.62 &  0.68 &      0.19 \\
0.35 &   5,514 &  3,245 &  0.37 &  80,289 &   4,711 &   8,989 &   6,011 &  0.56 &  0.40 &  0.31 &      0.11 \\
0.40 &   2,660 &  3,354 &  0.56 &  82,949 &   2,051 &  12,343 &   2,657 &  0.56 &  0.18 &  0.14 &      0.05 \\
0.45 &   1,352 &  2,152 &  0.61 &  84,301 &     699 &  14,495 &     505 &  0.42 &  0.03 &  0.05 &      0.01 \\
0.50 &     699 &    505 &  0.42 &  85,000 &       0 &  15,000 &       0 &   nan &  0.00 &  0.00 &      0.00 \\
0.55 &       0 &      0 &   nan &  85,000 &       0 &  15,000 &       0 &   nan &  0.00 &  0.00 &      0.00 \\
\vdots & \vdots & \vdots & \vdots & \vdots & \vdots & \vdots & \vdots & \vdots & \vdots & \vdots & \vdots  \\
1.00 &       0 &      0 &   nan &  85,000 &       0 &  15,000 &       0 &   nan &  0.00 &  0.00 &      0.00 \\
\bottomrule
\end{tabular}

\vskip 12pt

Choosing decision thresholds $\theta=0.20$ and $\theta = 0.25$ give us these confusion matrices.  

\

\hfil	\begin{tabular}{cc|c|c|}
	$\theta = 0.20$ &\multicolumn{1}{c}{}& \multicolumn{2}{c}{Prediction} \\[0.4em]
	&\multicolumn{1}{c}{} & \multicolumn{1}{c}{N} & \multicolumn{1}{c}{P} \cr\cline{3-4}
	\multirow{2}{*}{\rotatebox[origin=c]{90}{Actual}}&N &
53,211 & 31,789
	\vrule width 0pt height 10pt depth 2pt \cr\cline{3-4}
	&P & 
1,807 & 13,193
	\vrule width 0pt height 10pt depth 2pt \cr\cline{3-4}
	\end{tabular}
	%
\hfil	\begin{tabular}{cc|c|c|}
	$\theta = 0.25$ &\multicolumn{1}{c}{}& \multicolumn{2}{c}{Prediction} \\[0.4em]
	&\multicolumn{1}{c}{} & \multicolumn{1}{c}{N} & \multicolumn{1}{c}{P} \cr\cline{3-4}
	\multirow{2}{*}{\rotatebox[origin=c]{90}{Actual}}&N &
66,473 & 18,527
	\vrule width 0pt height 10pt depth 2pt \cr\cline{3-4}
	&P & 
3,406 & 11,594
	\vrule width 0pt height 10pt depth 2pt \cr\cline{3-4}
	\end{tabular}

\vskip 12pt

To choose $\theta = 0.25$ instead of $\theta = 0.20$ is to send 1,599 fewer ambulances to crash persons who need them (more false negatives), but also 13,262 fewer ambulances to crash persons who did not need them (fewer false positives).  Where to make that tradeoff of lives for money is a political question, not a technical one, but given the political decision we can choose $\theta$ to calibrate the recommendation system.  

For instance, in the short term the number of available ambulances and ambulance crews is fixed.  The FP/P column gives the amount of increase in the number of ambulances going to crash persons because we are immediately dispatching some ambulances that are not needed.  If the existing infrastructure could handle only a 5\% increase in the number of ambulances sent to crash persons, then we would choose $\theta = 0.45$, immediately dispatching 1,204 ambulances, 505 of which were needed and would have been sent anyway, but 699 of which are not needed, representing an increased cost.  If we could handle a 10\% increase, we would zoom in to the data to get a value of $\theta $ between $0.40$ and $0.45$ where $\text{FP/P} = 0.10$



{\bf This section repeats a lot of stuff from the Metrics section; put in Methods}

The binary classification process actually has more nuance that can lead to more interesting metrics.  The model does not tell us directly whether it predicts that the sample belongs to the positive or negative class; instead, the model assigns to each sample a value $p \in [0,1]$.  In code using scikit-learn, the value $p$ is often called \verb|y_proba| and is returned by the \verb|model.predict_proba()| function.  This value $p$ is a proxy for the probability that the crash person needs an ambulance.  It is not exactly the probability, but increases with the probability.  If we have enough data to smooth out the effects of randomness, and if there actually is an underlying pattern to the data that will predict the target variable, then we can can get rough probabilities from $p$ and measure marginal effects of changing the decision threshold $\theta$.

To map the probability that a crash person needs an ambulance as a function of $p$, we looked at ranges of $p$ of width $0.01$, and in each band divided the number of positive samples in the band by the total number of samples in the band.  We would like to call this metric ``marginal precision,'' but that term is used in statistics for something else.  It is the marginal probability, the probability that the a crash report with $p = \theta$ needs an ambulance, where Precision is the total probability, the proportion of crashes with $p > \theta$ that need an ambulance.  

To get enough samples in each band to smooth out the randomness, we needed more than the usual 70/30 train/test split, so we went to 5-fold cross validation.  In this approach, we divide the dataset into five sets, each with the same P/N ratio.  We then train the model five times with a 80/20 train/test split, which gives us a $p$ value for each sample in the whole set.  

There are two reasons why we chose a width of $0.01$ for our $p$ bands.  First, with a much smaller width we would have far fewer samples in each band, showing more of the effects of randomness.  Second, some of our model algorithms, like the Balanced Random Forest Classifier from Imbalanced-Learn, gave most (93\%) of the $p$ values rounded to two decimal places.  Intervals like $p \in [0.501,0.509)$ would have few samples, and thus not be useful.  The Balanced Bagging model algorithm gave 99\% of the values of $p$ rounded to one decimal place, so for that model we would need to choose a band width of $0.10.$

\


%%%
\subsection{The Dataset}

Ideally, we would use a dataset of crashes with an automated notification, but we have not found such a dataset that is publicly available.  Working with such a private dataset would be an important avenue of future research.

We will use the Crash Report Sampling System (CRSS) from 2016 to 2021.  The CRSS is a curated sample of crashes in the US, weighted to more serious crashes such that 17\% of the crash persons needed an ambulance, significantly more than the proportion of all reported crashes needing an ambulance, between 2 and 3 percent.  Since most low-speed crashes would have a crash profile similar to hard braking, they would not spawn an automated notification, so it is reasonable to assume that the set of crashes with automated notifications would have a higher percentage of persons needing an ambulance.  

We will use the CRSS as a proxy for the set of crashes with automatic crash notifications, acknowledging that we do not know how good of a proxy it is.  The primary merit of CRSS for our work is that it is publicly available so that our work can be critiqued, adapted, and expanded by others.  

We merged the \verb|accident.csv|, \verb|vehicle.csv|, and \verb|person.csv| files in the six years.  We dropped many features that were irrelevant, most because they were unique to each vehicle like a VIN (Vehicle Identification Number), with no predictive power, just random noise.  We also dropped the 33,776 persons in crashes involving a pedestrian, because the deceleration profile of hitting a pedestrian or bicycle would not be different enough from hard braking to trigger an automated crash notification.  

After removing repeated and irrelevant features and pedestrian crashes, we have 118 features describing 713,566 crash persons.  Later we removed more features that have more than 20\% of the values missing or only have data for some years, and features with imputed missing values, to get 78 features.

For full details, see the \verb|Ambulance_Dispatch_01_Get_Data.ipynb| file.  

%%%
\subsection{Binning Categories}

All of the CRSS data is discrete, but some features are ordered, like \verb|HOUR| and \verb|AGE|, and others are unordered, like \verb|MAKE_MOD|.  Reducing the dimensionality of the machine learning modeling by binning the categories into less than ten per feature is ideal.  

Some features like \verb|HOUR| we binned by hand.  We looked at the proportion of crash persons hospitalized at each hour and found clear places to break it into seven contiguous but not equal blocks.  When we looked at \verb|AGE|, we considered breaking it into decades, but found that ages 15-18 have a far lower hospitalization rate than those a little younger or older, and there was a shift at about age 53 and again around age 74, so we binned it accordingly.  

Some features like \verb|MAKE_MOD|m which has 1,210 unique values, we binned by imposing an order, ordering by the proportion of crash persons hospitalized, then cutting the ordered list into five new categories plus ``Unknown.''  

For full details, see 
\verb|Ambulance_Dispatch_02_Correlation.ipynb| and 

\verb|Ambulance_Dispatch_03_Bin_Data.ipynb|.

%%%
\subsection{Imputing Missing Data}

For reasons of historical consistency going back to 1982 with the predecessors of CRSS, CRSS imputes missing values for some features but not others, using IVEware, Imputation and Variation Estimation Software from the Institute for Social Research at the University of Michigan.  Fortunately, when CRSS gives a feature with imputed values, it also retains the original feature with values signifying ``Unknown.'' CRSS has a very helpful report on its imputation methods.  We have not seen in the literature where someone has used IVEware to impute the other features and compared it to other methods.  

At this point we have 78 unimputed features, and only 250,389 out of 713,566 samples (35\%) did not have missing values in those 78 features.  We compared three methods for imputing missing values.  

\begin{itemize}
	\item IVEware
	\item Imputation to Mode
	\item Round-Robin Random Forest using Imputation to Mode as the starting point
\end{itemize}

We found that the Random Forest method was best for our purposes.

For full details and analysis, see 
\verb|Ambulance_Dispatch_04_Impute_Missing_Data.ipynb|.

%%%
\subsection{Order of Operations}

We also considered whether the order of operations made a difference:  Should we bin first, then impute, or impute first, then bin?.  We tried both methods on the {\tt Ambulance} dataset with the IVEware imputation approach.  After several runs we found that the difference between methods was about the same as the difference between runs of the same method with different random seeds.  Since IVEware can only handle up to about forty categories in each categorical field, we had had to bin some fields first either way, so we chose to bin first, then impute.  

For full details and analysis, see 
\verb|Ambulance_Dispatch_05_IVEware_Order_of_Operations.ipynb|.

%%%
\subsection{Handling Imbalanced Data}

In our dataset only about fifteen percent of the people needed an ambulance.   If a recommendation system never sent an ambulance, the model would have 85\% accuracy, but be useless.  Most algorithms for training models are designed for balanced data, with half of the samples in each of the negative and positive classes.  With an imbalanced data set we can address the imbalance in five ways:  Resampling the dataset, modifying the loss function, choosing metrics other than accuracy, using learning methods that account for the imbalance, and manually moving the decision threshold.

\subsubsection{Resampling the Dataset}

We can balance the dataset by undersampling the majority class (negative, ``No ambulance'') or oversampling the minority class (positive, ``Send Ambulance'').  To balance by undersampling would mean throwing out eighty percent of the majority class, losing valuable information.  A very popular method for oversampling is SMOTE (Synthetic Minority Oversampling TEchnique), which creates new minority samples between existing minority samples, but the ``between'' requires continuous data, and all of our data is discrete or categorical (What is between a Buick and a Volvo?).

Tomek Links is one of the few resampling methods that works for categorical data.  It is a selective undersampling method that removes majority samples that seem out of place.  We did not see a significant improvement in the model metrics from the undersampling; the difference between no undersampling, one run of Tomek, and two runs turned out to be inconsequential, by which we mean that one approach was not consistently better (measured by the area under the ROC curve) when we ran the models with different random seeds.  

We considered a related method, Condensed Nearest Neighbor, but it is impractical for data sets of this size.  


\subsubsection{Modifying the Loss Function}

A popular and well established way to modify the loss function for imbalanced data is with class weights, which can have the same effect as na{\"i}ve oversampling.  

Three of our seven models take class weights, and for those we tried three different class weights.  

\

\hfil\begin{tabular}{c|l}
	$\alpha$ & Meaning \cr\hline
	1/2 & No class weight \cr
	2/3 & $\Delta FP/\Delta TP < 2.0$ goal \cr
	$0.85$ & Balanced classes \cr 
\end{tabular}

\


A related method is with focal loss, which has a modulating hyperparameter $\gamma$ that increases the penalty for low-confidence samples. \citep{lin2017focal}  We tried five values  of $\gamma$.

\

\hfil\begin{tabular}{c|c}
	$\gamma$  & Notes \cr\hline
	0.0 & Same as binary crossentropy \cr
	0.5 & Very light modulation \cr
	1.0 & Light modulation\cr
	2.0 & Recommended by Lin \cr
	5.0 & Heavy modulation \cr
\end{tabular}	

\

We did not see significant improvement using focal loss, measured by the area under the ROC curve.  ({\bf Put in Label Reference}).

%%%
\subsubsection{Metrics for Imbalance}


In the \nameref{Methods_Metrics} subsection above we defined the metrics recall, precision, and f1.  The most common metric in machine learning, the one that most algorithms are designed to maximize, is accuracy, the proportion of samples correctly classified.  In that section's example of transformed model output, we had 150,107 out of 177,392 test samples correctly classified, giving 84.6\% accuracy.  Is that good?  The model below, the raw results of the Logistic Regression model of the easy features set, recommends sending no ambulances, and it is correct in 150,771 of 177,392 test samples, giving 84.99\% accuracy.  Is that better?



\

%%%
\parbox{\linewidth}{
%{\bf Balanced Random Forest model, Hard features, No Tomek, $\alpha = 2/3$}

\noindent\begin{tabular}{@{\hspace{-6pt}}p{2.3in} @{\hspace{-6pt}}p{2.0in} p{1.8in}}
	\vskip 0pt
	\qquad \qquad Raw Model Output
	
	%% Creator: Matplotlib, PGF backend
%%
%% To include the figure in your LaTeX document, write
%%   \input{<filename>.pgf}
%%
%% Make sure the required packages are loaded in your preamble
%%   \usepackage{pgf}
%%
%% Also ensure that all the required font packages are loaded; for instance,
%% the lmodern package is sometimes necessary when using math font.
%%   \usepackage{lmodern}
%%
%% Figures using additional raster images can only be included by \input if
%% they are in the same directory as the main LaTeX file. For loading figures
%% from other directories you can use the `import` package
%%   \usepackage{import}
%%
%% and then include the figures with
%%   \import{<path to file>}{<filename>.pgf}
%%
%% Matplotlib used the following preamble
%%   
%%   \usepackage{fontspec}
%%   \makeatletter\@ifpackageloaded{underscore}{}{\usepackage[strings]{underscore}}\makeatother
%%
\begingroup%
\makeatletter%
\begin{pgfpicture}%
\pgfpathrectangle{\pgfpointorigin}{\pgfqpoint{2.253750in}{1.754444in}}%
\pgfusepath{use as bounding box, clip}%
\begin{pgfscope}%
\pgfsetbuttcap%
\pgfsetmiterjoin%
\definecolor{currentfill}{rgb}{1.000000,1.000000,1.000000}%
\pgfsetfillcolor{currentfill}%
\pgfsetlinewidth{0.000000pt}%
\definecolor{currentstroke}{rgb}{1.000000,1.000000,1.000000}%
\pgfsetstrokecolor{currentstroke}%
\pgfsetdash{}{0pt}%
\pgfpathmoveto{\pgfqpoint{0.000000in}{0.000000in}}%
\pgfpathlineto{\pgfqpoint{2.253750in}{0.000000in}}%
\pgfpathlineto{\pgfqpoint{2.253750in}{1.754444in}}%
\pgfpathlineto{\pgfqpoint{0.000000in}{1.754444in}}%
\pgfpathlineto{\pgfqpoint{0.000000in}{0.000000in}}%
\pgfpathclose%
\pgfusepath{fill}%
\end{pgfscope}%
\begin{pgfscope}%
\pgfsetbuttcap%
\pgfsetmiterjoin%
\definecolor{currentfill}{rgb}{1.000000,1.000000,1.000000}%
\pgfsetfillcolor{currentfill}%
\pgfsetlinewidth{0.000000pt}%
\definecolor{currentstroke}{rgb}{0.000000,0.000000,0.000000}%
\pgfsetstrokecolor{currentstroke}%
\pgfsetstrokeopacity{0.000000}%
\pgfsetdash{}{0pt}%
\pgfpathmoveto{\pgfqpoint{0.515000in}{0.499444in}}%
\pgfpathlineto{\pgfqpoint{2.065000in}{0.499444in}}%
\pgfpathlineto{\pgfqpoint{2.065000in}{1.654444in}}%
\pgfpathlineto{\pgfqpoint{0.515000in}{1.654444in}}%
\pgfpathlineto{\pgfqpoint{0.515000in}{0.499444in}}%
\pgfpathclose%
\pgfusepath{fill}%
\end{pgfscope}%
\begin{pgfscope}%
\pgfpathrectangle{\pgfqpoint{0.515000in}{0.499444in}}{\pgfqpoint{1.550000in}{1.155000in}}%
\pgfusepath{clip}%
\pgfsetbuttcap%
\pgfsetmiterjoin%
\pgfsetlinewidth{1.003750pt}%
\definecolor{currentstroke}{rgb}{0.000000,0.000000,0.000000}%
\pgfsetstrokecolor{currentstroke}%
\pgfsetdash{}{0pt}%
\pgfpathmoveto{\pgfqpoint{0.505000in}{0.499444in}}%
\pgfpathlineto{\pgfqpoint{0.552805in}{0.499444in}}%
\pgfpathlineto{\pgfqpoint{0.552805in}{1.029651in}}%
\pgfpathlineto{\pgfqpoint{0.505000in}{1.029651in}}%
\pgfusepath{stroke}%
\end{pgfscope}%
\begin{pgfscope}%
\pgfpathrectangle{\pgfqpoint{0.515000in}{0.499444in}}{\pgfqpoint{1.550000in}{1.155000in}}%
\pgfusepath{clip}%
\pgfsetbuttcap%
\pgfsetmiterjoin%
\pgfsetlinewidth{1.003750pt}%
\definecolor{currentstroke}{rgb}{0.000000,0.000000,0.000000}%
\pgfsetstrokecolor{currentstroke}%
\pgfsetdash{}{0pt}%
\pgfpathmoveto{\pgfqpoint{0.643537in}{0.499444in}}%
\pgfpathlineto{\pgfqpoint{0.704025in}{0.499444in}}%
\pgfpathlineto{\pgfqpoint{0.704025in}{1.599444in}}%
\pgfpathlineto{\pgfqpoint{0.643537in}{1.599444in}}%
\pgfpathlineto{\pgfqpoint{0.643537in}{0.499444in}}%
\pgfpathclose%
\pgfusepath{stroke}%
\end{pgfscope}%
\begin{pgfscope}%
\pgfpathrectangle{\pgfqpoint{0.515000in}{0.499444in}}{\pgfqpoint{1.550000in}{1.155000in}}%
\pgfusepath{clip}%
\pgfsetbuttcap%
\pgfsetmiterjoin%
\pgfsetlinewidth{1.003750pt}%
\definecolor{currentstroke}{rgb}{0.000000,0.000000,0.000000}%
\pgfsetstrokecolor{currentstroke}%
\pgfsetdash{}{0pt}%
\pgfpathmoveto{\pgfqpoint{0.794756in}{0.499444in}}%
\pgfpathlineto{\pgfqpoint{0.855244in}{0.499444in}}%
\pgfpathlineto{\pgfqpoint{0.855244in}{0.903095in}}%
\pgfpathlineto{\pgfqpoint{0.794756in}{0.903095in}}%
\pgfpathlineto{\pgfqpoint{0.794756in}{0.499444in}}%
\pgfpathclose%
\pgfusepath{stroke}%
\end{pgfscope}%
\begin{pgfscope}%
\pgfpathrectangle{\pgfqpoint{0.515000in}{0.499444in}}{\pgfqpoint{1.550000in}{1.155000in}}%
\pgfusepath{clip}%
\pgfsetbuttcap%
\pgfsetmiterjoin%
\pgfsetlinewidth{1.003750pt}%
\definecolor{currentstroke}{rgb}{0.000000,0.000000,0.000000}%
\pgfsetstrokecolor{currentstroke}%
\pgfsetdash{}{0pt}%
\pgfpathmoveto{\pgfqpoint{0.945976in}{0.499444in}}%
\pgfpathlineto{\pgfqpoint{1.006464in}{0.499444in}}%
\pgfpathlineto{\pgfqpoint{1.006464in}{0.565819in}}%
\pgfpathlineto{\pgfqpoint{0.945976in}{0.565819in}}%
\pgfpathlineto{\pgfqpoint{0.945976in}{0.499444in}}%
\pgfpathclose%
\pgfusepath{stroke}%
\end{pgfscope}%
\begin{pgfscope}%
\pgfpathrectangle{\pgfqpoint{0.515000in}{0.499444in}}{\pgfqpoint{1.550000in}{1.155000in}}%
\pgfusepath{clip}%
\pgfsetbuttcap%
\pgfsetmiterjoin%
\pgfsetlinewidth{1.003750pt}%
\definecolor{currentstroke}{rgb}{0.000000,0.000000,0.000000}%
\pgfsetstrokecolor{currentstroke}%
\pgfsetdash{}{0pt}%
\pgfpathmoveto{\pgfqpoint{1.097195in}{0.499444in}}%
\pgfpathlineto{\pgfqpoint{1.157683in}{0.499444in}}%
\pgfpathlineto{\pgfqpoint{1.157683in}{0.505884in}}%
\pgfpathlineto{\pgfqpoint{1.097195in}{0.505884in}}%
\pgfpathlineto{\pgfqpoint{1.097195in}{0.499444in}}%
\pgfpathclose%
\pgfusepath{stroke}%
\end{pgfscope}%
\begin{pgfscope}%
\pgfpathrectangle{\pgfqpoint{0.515000in}{0.499444in}}{\pgfqpoint{1.550000in}{1.155000in}}%
\pgfusepath{clip}%
\pgfsetbuttcap%
\pgfsetmiterjoin%
\pgfsetlinewidth{1.003750pt}%
\definecolor{currentstroke}{rgb}{0.000000,0.000000,0.000000}%
\pgfsetstrokecolor{currentstroke}%
\pgfsetdash{}{0pt}%
\pgfpathmoveto{\pgfqpoint{1.248415in}{0.499444in}}%
\pgfpathlineto{\pgfqpoint{1.308903in}{0.499444in}}%
\pgfpathlineto{\pgfqpoint{1.308903in}{0.499444in}}%
\pgfpathlineto{\pgfqpoint{1.248415in}{0.499444in}}%
\pgfpathlineto{\pgfqpoint{1.248415in}{0.499444in}}%
\pgfpathclose%
\pgfusepath{stroke}%
\end{pgfscope}%
\begin{pgfscope}%
\pgfpathrectangle{\pgfqpoint{0.515000in}{0.499444in}}{\pgfqpoint{1.550000in}{1.155000in}}%
\pgfusepath{clip}%
\pgfsetbuttcap%
\pgfsetmiterjoin%
\pgfsetlinewidth{1.003750pt}%
\definecolor{currentstroke}{rgb}{0.000000,0.000000,0.000000}%
\pgfsetstrokecolor{currentstroke}%
\pgfsetdash{}{0pt}%
\pgfpathmoveto{\pgfqpoint{1.399634in}{0.499444in}}%
\pgfpathlineto{\pgfqpoint{1.460122in}{0.499444in}}%
\pgfpathlineto{\pgfqpoint{1.460122in}{0.499444in}}%
\pgfpathlineto{\pgfqpoint{1.399634in}{0.499444in}}%
\pgfpathlineto{\pgfqpoint{1.399634in}{0.499444in}}%
\pgfpathclose%
\pgfusepath{stroke}%
\end{pgfscope}%
\begin{pgfscope}%
\pgfpathrectangle{\pgfqpoint{0.515000in}{0.499444in}}{\pgfqpoint{1.550000in}{1.155000in}}%
\pgfusepath{clip}%
\pgfsetbuttcap%
\pgfsetmiterjoin%
\pgfsetlinewidth{1.003750pt}%
\definecolor{currentstroke}{rgb}{0.000000,0.000000,0.000000}%
\pgfsetstrokecolor{currentstroke}%
\pgfsetdash{}{0pt}%
\pgfpathmoveto{\pgfqpoint{1.550854in}{0.499444in}}%
\pgfpathlineto{\pgfqpoint{1.611342in}{0.499444in}}%
\pgfpathlineto{\pgfqpoint{1.611342in}{0.499444in}}%
\pgfpathlineto{\pgfqpoint{1.550854in}{0.499444in}}%
\pgfpathlineto{\pgfqpoint{1.550854in}{0.499444in}}%
\pgfpathclose%
\pgfusepath{stroke}%
\end{pgfscope}%
\begin{pgfscope}%
\pgfpathrectangle{\pgfqpoint{0.515000in}{0.499444in}}{\pgfqpoint{1.550000in}{1.155000in}}%
\pgfusepath{clip}%
\pgfsetbuttcap%
\pgfsetmiterjoin%
\pgfsetlinewidth{1.003750pt}%
\definecolor{currentstroke}{rgb}{0.000000,0.000000,0.000000}%
\pgfsetstrokecolor{currentstroke}%
\pgfsetdash{}{0pt}%
\pgfpathmoveto{\pgfqpoint{1.702073in}{0.499444in}}%
\pgfpathlineto{\pgfqpoint{1.762561in}{0.499444in}}%
\pgfpathlineto{\pgfqpoint{1.762561in}{0.499444in}}%
\pgfpathlineto{\pgfqpoint{1.702073in}{0.499444in}}%
\pgfpathlineto{\pgfqpoint{1.702073in}{0.499444in}}%
\pgfpathclose%
\pgfusepath{stroke}%
\end{pgfscope}%
\begin{pgfscope}%
\pgfpathrectangle{\pgfqpoint{0.515000in}{0.499444in}}{\pgfqpoint{1.550000in}{1.155000in}}%
\pgfusepath{clip}%
\pgfsetbuttcap%
\pgfsetmiterjoin%
\pgfsetlinewidth{1.003750pt}%
\definecolor{currentstroke}{rgb}{0.000000,0.000000,0.000000}%
\pgfsetstrokecolor{currentstroke}%
\pgfsetdash{}{0pt}%
\pgfpathmoveto{\pgfqpoint{1.853293in}{0.499444in}}%
\pgfpathlineto{\pgfqpoint{1.913781in}{0.499444in}}%
\pgfpathlineto{\pgfqpoint{1.913781in}{0.499444in}}%
\pgfpathlineto{\pgfqpoint{1.853293in}{0.499444in}}%
\pgfpathlineto{\pgfqpoint{1.853293in}{0.499444in}}%
\pgfpathclose%
\pgfusepath{stroke}%
\end{pgfscope}%
\begin{pgfscope}%
\pgfpathrectangle{\pgfqpoint{0.515000in}{0.499444in}}{\pgfqpoint{1.550000in}{1.155000in}}%
\pgfusepath{clip}%
\pgfsetbuttcap%
\pgfsetmiterjoin%
\definecolor{currentfill}{rgb}{0.000000,0.000000,0.000000}%
\pgfsetfillcolor{currentfill}%
\pgfsetlinewidth{0.000000pt}%
\definecolor{currentstroke}{rgb}{0.000000,0.000000,0.000000}%
\pgfsetstrokecolor{currentstroke}%
\pgfsetstrokeopacity{0.000000}%
\pgfsetdash{}{0pt}%
\pgfpathmoveto{\pgfqpoint{0.552805in}{0.499444in}}%
\pgfpathlineto{\pgfqpoint{0.613293in}{0.499444in}}%
\pgfpathlineto{\pgfqpoint{0.613293in}{0.539358in}}%
\pgfpathlineto{\pgfqpoint{0.552805in}{0.539358in}}%
\pgfpathlineto{\pgfqpoint{0.552805in}{0.499444in}}%
\pgfpathclose%
\pgfusepath{fill}%
\end{pgfscope}%
\begin{pgfscope}%
\pgfpathrectangle{\pgfqpoint{0.515000in}{0.499444in}}{\pgfqpoint{1.550000in}{1.155000in}}%
\pgfusepath{clip}%
\pgfsetbuttcap%
\pgfsetmiterjoin%
\definecolor{currentfill}{rgb}{0.000000,0.000000,0.000000}%
\pgfsetfillcolor{currentfill}%
\pgfsetlinewidth{0.000000pt}%
\definecolor{currentstroke}{rgb}{0.000000,0.000000,0.000000}%
\pgfsetstrokecolor{currentstroke}%
\pgfsetstrokeopacity{0.000000}%
\pgfsetdash{}{0pt}%
\pgfpathmoveto{\pgfqpoint{0.704025in}{0.499444in}}%
\pgfpathlineto{\pgfqpoint{0.764512in}{0.499444in}}%
\pgfpathlineto{\pgfqpoint{0.764512in}{0.687291in}}%
\pgfpathlineto{\pgfqpoint{0.704025in}{0.687291in}}%
\pgfpathlineto{\pgfqpoint{0.704025in}{0.499444in}}%
\pgfpathclose%
\pgfusepath{fill}%
\end{pgfscope}%
\begin{pgfscope}%
\pgfpathrectangle{\pgfqpoint{0.515000in}{0.499444in}}{\pgfqpoint{1.550000in}{1.155000in}}%
\pgfusepath{clip}%
\pgfsetbuttcap%
\pgfsetmiterjoin%
\definecolor{currentfill}{rgb}{0.000000,0.000000,0.000000}%
\pgfsetfillcolor{currentfill}%
\pgfsetlinewidth{0.000000pt}%
\definecolor{currentstroke}{rgb}{0.000000,0.000000,0.000000}%
\pgfsetstrokecolor{currentstroke}%
\pgfsetstrokeopacity{0.000000}%
\pgfsetdash{}{0pt}%
\pgfpathmoveto{\pgfqpoint{0.855244in}{0.499444in}}%
\pgfpathlineto{\pgfqpoint{0.915732in}{0.499444in}}%
\pgfpathlineto{\pgfqpoint{0.915732in}{0.632183in}}%
\pgfpathlineto{\pgfqpoint{0.855244in}{0.632183in}}%
\pgfpathlineto{\pgfqpoint{0.855244in}{0.499444in}}%
\pgfpathclose%
\pgfusepath{fill}%
\end{pgfscope}%
\begin{pgfscope}%
\pgfpathrectangle{\pgfqpoint{0.515000in}{0.499444in}}{\pgfqpoint{1.550000in}{1.155000in}}%
\pgfusepath{clip}%
\pgfsetbuttcap%
\pgfsetmiterjoin%
\definecolor{currentfill}{rgb}{0.000000,0.000000,0.000000}%
\pgfsetfillcolor{currentfill}%
\pgfsetlinewidth{0.000000pt}%
\definecolor{currentstroke}{rgb}{0.000000,0.000000,0.000000}%
\pgfsetstrokecolor{currentstroke}%
\pgfsetstrokeopacity{0.000000}%
\pgfsetdash{}{0pt}%
\pgfpathmoveto{\pgfqpoint{1.006464in}{0.499444in}}%
\pgfpathlineto{\pgfqpoint{1.066951in}{0.499444in}}%
\pgfpathlineto{\pgfqpoint{1.066951in}{0.530323in}}%
\pgfpathlineto{\pgfqpoint{1.006464in}{0.530323in}}%
\pgfpathlineto{\pgfqpoint{1.006464in}{0.499444in}}%
\pgfpathclose%
\pgfusepath{fill}%
\end{pgfscope}%
\begin{pgfscope}%
\pgfpathrectangle{\pgfqpoint{0.515000in}{0.499444in}}{\pgfqpoint{1.550000in}{1.155000in}}%
\pgfusepath{clip}%
\pgfsetbuttcap%
\pgfsetmiterjoin%
\definecolor{currentfill}{rgb}{0.000000,0.000000,0.000000}%
\pgfsetfillcolor{currentfill}%
\pgfsetlinewidth{0.000000pt}%
\definecolor{currentstroke}{rgb}{0.000000,0.000000,0.000000}%
\pgfsetstrokecolor{currentstroke}%
\pgfsetstrokeopacity{0.000000}%
\pgfsetdash{}{0pt}%
\pgfpathmoveto{\pgfqpoint{1.157683in}{0.499444in}}%
\pgfpathlineto{\pgfqpoint{1.218171in}{0.499444in}}%
\pgfpathlineto{\pgfqpoint{1.218171in}{0.503395in}}%
\pgfpathlineto{\pgfqpoint{1.157683in}{0.503395in}}%
\pgfpathlineto{\pgfqpoint{1.157683in}{0.499444in}}%
\pgfpathclose%
\pgfusepath{fill}%
\end{pgfscope}%
\begin{pgfscope}%
\pgfpathrectangle{\pgfqpoint{0.515000in}{0.499444in}}{\pgfqpoint{1.550000in}{1.155000in}}%
\pgfusepath{clip}%
\pgfsetbuttcap%
\pgfsetmiterjoin%
\definecolor{currentfill}{rgb}{0.000000,0.000000,0.000000}%
\pgfsetfillcolor{currentfill}%
\pgfsetlinewidth{0.000000pt}%
\definecolor{currentstroke}{rgb}{0.000000,0.000000,0.000000}%
\pgfsetstrokecolor{currentstroke}%
\pgfsetstrokeopacity{0.000000}%
\pgfsetdash{}{0pt}%
\pgfpathmoveto{\pgfqpoint{1.308903in}{0.499444in}}%
\pgfpathlineto{\pgfqpoint{1.369391in}{0.499444in}}%
\pgfpathlineto{\pgfqpoint{1.369391in}{0.499456in}}%
\pgfpathlineto{\pgfqpoint{1.308903in}{0.499456in}}%
\pgfpathlineto{\pgfqpoint{1.308903in}{0.499444in}}%
\pgfpathclose%
\pgfusepath{fill}%
\end{pgfscope}%
\begin{pgfscope}%
\pgfpathrectangle{\pgfqpoint{0.515000in}{0.499444in}}{\pgfqpoint{1.550000in}{1.155000in}}%
\pgfusepath{clip}%
\pgfsetbuttcap%
\pgfsetmiterjoin%
\definecolor{currentfill}{rgb}{0.000000,0.000000,0.000000}%
\pgfsetfillcolor{currentfill}%
\pgfsetlinewidth{0.000000pt}%
\definecolor{currentstroke}{rgb}{0.000000,0.000000,0.000000}%
\pgfsetstrokecolor{currentstroke}%
\pgfsetstrokeopacity{0.000000}%
\pgfsetdash{}{0pt}%
\pgfpathmoveto{\pgfqpoint{1.460122in}{0.499444in}}%
\pgfpathlineto{\pgfqpoint{1.520610in}{0.499444in}}%
\pgfpathlineto{\pgfqpoint{1.520610in}{0.499444in}}%
\pgfpathlineto{\pgfqpoint{1.460122in}{0.499444in}}%
\pgfpathlineto{\pgfqpoint{1.460122in}{0.499444in}}%
\pgfpathclose%
\pgfusepath{fill}%
\end{pgfscope}%
\begin{pgfscope}%
\pgfpathrectangle{\pgfqpoint{0.515000in}{0.499444in}}{\pgfqpoint{1.550000in}{1.155000in}}%
\pgfusepath{clip}%
\pgfsetbuttcap%
\pgfsetmiterjoin%
\definecolor{currentfill}{rgb}{0.000000,0.000000,0.000000}%
\pgfsetfillcolor{currentfill}%
\pgfsetlinewidth{0.000000pt}%
\definecolor{currentstroke}{rgb}{0.000000,0.000000,0.000000}%
\pgfsetstrokecolor{currentstroke}%
\pgfsetstrokeopacity{0.000000}%
\pgfsetdash{}{0pt}%
\pgfpathmoveto{\pgfqpoint{1.611342in}{0.499444in}}%
\pgfpathlineto{\pgfqpoint{1.671830in}{0.499444in}}%
\pgfpathlineto{\pgfqpoint{1.671830in}{0.499444in}}%
\pgfpathlineto{\pgfqpoint{1.611342in}{0.499444in}}%
\pgfpathlineto{\pgfqpoint{1.611342in}{0.499444in}}%
\pgfpathclose%
\pgfusepath{fill}%
\end{pgfscope}%
\begin{pgfscope}%
\pgfpathrectangle{\pgfqpoint{0.515000in}{0.499444in}}{\pgfqpoint{1.550000in}{1.155000in}}%
\pgfusepath{clip}%
\pgfsetbuttcap%
\pgfsetmiterjoin%
\definecolor{currentfill}{rgb}{0.000000,0.000000,0.000000}%
\pgfsetfillcolor{currentfill}%
\pgfsetlinewidth{0.000000pt}%
\definecolor{currentstroke}{rgb}{0.000000,0.000000,0.000000}%
\pgfsetstrokecolor{currentstroke}%
\pgfsetstrokeopacity{0.000000}%
\pgfsetdash{}{0pt}%
\pgfpathmoveto{\pgfqpoint{1.762561in}{0.499444in}}%
\pgfpathlineto{\pgfqpoint{1.823049in}{0.499444in}}%
\pgfpathlineto{\pgfqpoint{1.823049in}{0.499444in}}%
\pgfpathlineto{\pgfqpoint{1.762561in}{0.499444in}}%
\pgfpathlineto{\pgfqpoint{1.762561in}{0.499444in}}%
\pgfpathclose%
\pgfusepath{fill}%
\end{pgfscope}%
\begin{pgfscope}%
\pgfpathrectangle{\pgfqpoint{0.515000in}{0.499444in}}{\pgfqpoint{1.550000in}{1.155000in}}%
\pgfusepath{clip}%
\pgfsetbuttcap%
\pgfsetmiterjoin%
\definecolor{currentfill}{rgb}{0.000000,0.000000,0.000000}%
\pgfsetfillcolor{currentfill}%
\pgfsetlinewidth{0.000000pt}%
\definecolor{currentstroke}{rgb}{0.000000,0.000000,0.000000}%
\pgfsetstrokecolor{currentstroke}%
\pgfsetstrokeopacity{0.000000}%
\pgfsetdash{}{0pt}%
\pgfpathmoveto{\pgfqpoint{1.913781in}{0.499444in}}%
\pgfpathlineto{\pgfqpoint{1.974269in}{0.499444in}}%
\pgfpathlineto{\pgfqpoint{1.974269in}{0.499444in}}%
\pgfpathlineto{\pgfqpoint{1.913781in}{0.499444in}}%
\pgfpathlineto{\pgfqpoint{1.913781in}{0.499444in}}%
\pgfpathclose%
\pgfusepath{fill}%
\end{pgfscope}%
\begin{pgfscope}%
\pgfsetbuttcap%
\pgfsetroundjoin%
\definecolor{currentfill}{rgb}{0.000000,0.000000,0.000000}%
\pgfsetfillcolor{currentfill}%
\pgfsetlinewidth{0.803000pt}%
\definecolor{currentstroke}{rgb}{0.000000,0.000000,0.000000}%
\pgfsetstrokecolor{currentstroke}%
\pgfsetdash{}{0pt}%
\pgfsys@defobject{currentmarker}{\pgfqpoint{0.000000in}{-0.048611in}}{\pgfqpoint{0.000000in}{0.000000in}}{%
\pgfpathmoveto{\pgfqpoint{0.000000in}{0.000000in}}%
\pgfpathlineto{\pgfqpoint{0.000000in}{-0.048611in}}%
\pgfusepath{stroke,fill}%
}%
\begin{pgfscope}%
\pgfsys@transformshift{0.552805in}{0.499444in}%
\pgfsys@useobject{currentmarker}{}%
\end{pgfscope}%
\end{pgfscope}%
\begin{pgfscope}%
\definecolor{textcolor}{rgb}{0.000000,0.000000,0.000000}%
\pgfsetstrokecolor{textcolor}%
\pgfsetfillcolor{textcolor}%
\pgftext[x=0.552805in,y=0.402222in,,top]{\color{textcolor}\rmfamily\fontsize{10.000000}{12.000000}\selectfont 0.0}%
\end{pgfscope}%
\begin{pgfscope}%
\pgfsetbuttcap%
\pgfsetroundjoin%
\definecolor{currentfill}{rgb}{0.000000,0.000000,0.000000}%
\pgfsetfillcolor{currentfill}%
\pgfsetlinewidth{0.803000pt}%
\definecolor{currentstroke}{rgb}{0.000000,0.000000,0.000000}%
\pgfsetstrokecolor{currentstroke}%
\pgfsetdash{}{0pt}%
\pgfsys@defobject{currentmarker}{\pgfqpoint{0.000000in}{-0.048611in}}{\pgfqpoint{0.000000in}{0.000000in}}{%
\pgfpathmoveto{\pgfqpoint{0.000000in}{0.000000in}}%
\pgfpathlineto{\pgfqpoint{0.000000in}{-0.048611in}}%
\pgfusepath{stroke,fill}%
}%
\begin{pgfscope}%
\pgfsys@transformshift{0.930854in}{0.499444in}%
\pgfsys@useobject{currentmarker}{}%
\end{pgfscope}%
\end{pgfscope}%
\begin{pgfscope}%
\definecolor{textcolor}{rgb}{0.000000,0.000000,0.000000}%
\pgfsetstrokecolor{textcolor}%
\pgfsetfillcolor{textcolor}%
\pgftext[x=0.930854in,y=0.402222in,,top]{\color{textcolor}\rmfamily\fontsize{10.000000}{12.000000}\selectfont 0.25}%
\end{pgfscope}%
\begin{pgfscope}%
\pgfsetbuttcap%
\pgfsetroundjoin%
\definecolor{currentfill}{rgb}{0.000000,0.000000,0.000000}%
\pgfsetfillcolor{currentfill}%
\pgfsetlinewidth{0.803000pt}%
\definecolor{currentstroke}{rgb}{0.000000,0.000000,0.000000}%
\pgfsetstrokecolor{currentstroke}%
\pgfsetdash{}{0pt}%
\pgfsys@defobject{currentmarker}{\pgfqpoint{0.000000in}{-0.048611in}}{\pgfqpoint{0.000000in}{0.000000in}}{%
\pgfpathmoveto{\pgfqpoint{0.000000in}{0.000000in}}%
\pgfpathlineto{\pgfqpoint{0.000000in}{-0.048611in}}%
\pgfusepath{stroke,fill}%
}%
\begin{pgfscope}%
\pgfsys@transformshift{1.308903in}{0.499444in}%
\pgfsys@useobject{currentmarker}{}%
\end{pgfscope}%
\end{pgfscope}%
\begin{pgfscope}%
\definecolor{textcolor}{rgb}{0.000000,0.000000,0.000000}%
\pgfsetstrokecolor{textcolor}%
\pgfsetfillcolor{textcolor}%
\pgftext[x=1.308903in,y=0.402222in,,top]{\color{textcolor}\rmfamily\fontsize{10.000000}{12.000000}\selectfont 0.5}%
\end{pgfscope}%
\begin{pgfscope}%
\pgfsetbuttcap%
\pgfsetroundjoin%
\definecolor{currentfill}{rgb}{0.000000,0.000000,0.000000}%
\pgfsetfillcolor{currentfill}%
\pgfsetlinewidth{0.803000pt}%
\definecolor{currentstroke}{rgb}{0.000000,0.000000,0.000000}%
\pgfsetstrokecolor{currentstroke}%
\pgfsetdash{}{0pt}%
\pgfsys@defobject{currentmarker}{\pgfqpoint{0.000000in}{-0.048611in}}{\pgfqpoint{0.000000in}{0.000000in}}{%
\pgfpathmoveto{\pgfqpoint{0.000000in}{0.000000in}}%
\pgfpathlineto{\pgfqpoint{0.000000in}{-0.048611in}}%
\pgfusepath{stroke,fill}%
}%
\begin{pgfscope}%
\pgfsys@transformshift{1.686951in}{0.499444in}%
\pgfsys@useobject{currentmarker}{}%
\end{pgfscope}%
\end{pgfscope}%
\begin{pgfscope}%
\definecolor{textcolor}{rgb}{0.000000,0.000000,0.000000}%
\pgfsetstrokecolor{textcolor}%
\pgfsetfillcolor{textcolor}%
\pgftext[x=1.686951in,y=0.402222in,,top]{\color{textcolor}\rmfamily\fontsize{10.000000}{12.000000}\selectfont 0.75}%
\end{pgfscope}%
\begin{pgfscope}%
\pgfsetbuttcap%
\pgfsetroundjoin%
\definecolor{currentfill}{rgb}{0.000000,0.000000,0.000000}%
\pgfsetfillcolor{currentfill}%
\pgfsetlinewidth{0.803000pt}%
\definecolor{currentstroke}{rgb}{0.000000,0.000000,0.000000}%
\pgfsetstrokecolor{currentstroke}%
\pgfsetdash{}{0pt}%
\pgfsys@defobject{currentmarker}{\pgfqpoint{0.000000in}{-0.048611in}}{\pgfqpoint{0.000000in}{0.000000in}}{%
\pgfpathmoveto{\pgfqpoint{0.000000in}{0.000000in}}%
\pgfpathlineto{\pgfqpoint{0.000000in}{-0.048611in}}%
\pgfusepath{stroke,fill}%
}%
\begin{pgfscope}%
\pgfsys@transformshift{2.065000in}{0.499444in}%
\pgfsys@useobject{currentmarker}{}%
\end{pgfscope}%
\end{pgfscope}%
\begin{pgfscope}%
\definecolor{textcolor}{rgb}{0.000000,0.000000,0.000000}%
\pgfsetstrokecolor{textcolor}%
\pgfsetfillcolor{textcolor}%
\pgftext[x=2.065000in,y=0.402222in,,top]{\color{textcolor}\rmfamily\fontsize{10.000000}{12.000000}\selectfont 1.0}%
\end{pgfscope}%
\begin{pgfscope}%
\definecolor{textcolor}{rgb}{0.000000,0.000000,0.000000}%
\pgfsetstrokecolor{textcolor}%
\pgfsetfillcolor{textcolor}%
\pgftext[x=1.290000in,y=0.223333in,,top]{\color{textcolor}\rmfamily\fontsize{10.000000}{12.000000}\selectfont \(\displaystyle p\)}%
\end{pgfscope}%
\begin{pgfscope}%
\pgfsetbuttcap%
\pgfsetroundjoin%
\definecolor{currentfill}{rgb}{0.000000,0.000000,0.000000}%
\pgfsetfillcolor{currentfill}%
\pgfsetlinewidth{0.803000pt}%
\definecolor{currentstroke}{rgb}{0.000000,0.000000,0.000000}%
\pgfsetstrokecolor{currentstroke}%
\pgfsetdash{}{0pt}%
\pgfsys@defobject{currentmarker}{\pgfqpoint{-0.048611in}{0.000000in}}{\pgfqpoint{-0.000000in}{0.000000in}}{%
\pgfpathmoveto{\pgfqpoint{-0.000000in}{0.000000in}}%
\pgfpathlineto{\pgfqpoint{-0.048611in}{0.000000in}}%
\pgfusepath{stroke,fill}%
}%
\begin{pgfscope}%
\pgfsys@transformshift{0.515000in}{0.499444in}%
\pgfsys@useobject{currentmarker}{}%
\end{pgfscope}%
\end{pgfscope}%
\begin{pgfscope}%
\definecolor{textcolor}{rgb}{0.000000,0.000000,0.000000}%
\pgfsetstrokecolor{textcolor}%
\pgfsetfillcolor{textcolor}%
\pgftext[x=0.348333in, y=0.451250in, left, base]{\color{textcolor}\rmfamily\fontsize{10.000000}{12.000000}\selectfont \(\displaystyle {0}\)}%
\end{pgfscope}%
\begin{pgfscope}%
\pgfsetbuttcap%
\pgfsetroundjoin%
\definecolor{currentfill}{rgb}{0.000000,0.000000,0.000000}%
\pgfsetfillcolor{currentfill}%
\pgfsetlinewidth{0.803000pt}%
\definecolor{currentstroke}{rgb}{0.000000,0.000000,0.000000}%
\pgfsetstrokecolor{currentstroke}%
\pgfsetdash{}{0pt}%
\pgfsys@defobject{currentmarker}{\pgfqpoint{-0.048611in}{0.000000in}}{\pgfqpoint{-0.000000in}{0.000000in}}{%
\pgfpathmoveto{\pgfqpoint{-0.000000in}{0.000000in}}%
\pgfpathlineto{\pgfqpoint{-0.048611in}{0.000000in}}%
\pgfusepath{stroke,fill}%
}%
\begin{pgfscope}%
\pgfsys@transformshift{0.515000in}{0.999847in}%
\pgfsys@useobject{currentmarker}{}%
\end{pgfscope}%
\end{pgfscope}%
\begin{pgfscope}%
\definecolor{textcolor}{rgb}{0.000000,0.000000,0.000000}%
\pgfsetstrokecolor{textcolor}%
\pgfsetfillcolor{textcolor}%
\pgftext[x=0.278889in, y=0.951652in, left, base]{\color{textcolor}\rmfamily\fontsize{10.000000}{12.000000}\selectfont \(\displaystyle {20}\)}%
\end{pgfscope}%
\begin{pgfscope}%
\pgfsetbuttcap%
\pgfsetroundjoin%
\definecolor{currentfill}{rgb}{0.000000,0.000000,0.000000}%
\pgfsetfillcolor{currentfill}%
\pgfsetlinewidth{0.803000pt}%
\definecolor{currentstroke}{rgb}{0.000000,0.000000,0.000000}%
\pgfsetstrokecolor{currentstroke}%
\pgfsetdash{}{0pt}%
\pgfsys@defobject{currentmarker}{\pgfqpoint{-0.048611in}{0.000000in}}{\pgfqpoint{-0.000000in}{0.000000in}}{%
\pgfpathmoveto{\pgfqpoint{-0.000000in}{0.000000in}}%
\pgfpathlineto{\pgfqpoint{-0.048611in}{0.000000in}}%
\pgfusepath{stroke,fill}%
}%
\begin{pgfscope}%
\pgfsys@transformshift{0.515000in}{1.500250in}%
\pgfsys@useobject{currentmarker}{}%
\end{pgfscope}%
\end{pgfscope}%
\begin{pgfscope}%
\definecolor{textcolor}{rgb}{0.000000,0.000000,0.000000}%
\pgfsetstrokecolor{textcolor}%
\pgfsetfillcolor{textcolor}%
\pgftext[x=0.278889in, y=1.452055in, left, base]{\color{textcolor}\rmfamily\fontsize{10.000000}{12.000000}\selectfont \(\displaystyle {40}\)}%
\end{pgfscope}%
\begin{pgfscope}%
\definecolor{textcolor}{rgb}{0.000000,0.000000,0.000000}%
\pgfsetstrokecolor{textcolor}%
\pgfsetfillcolor{textcolor}%
\pgftext[x=0.223333in,y=1.076944in,,bottom,rotate=90.000000]{\color{textcolor}\rmfamily\fontsize{10.000000}{12.000000}\selectfont Percent of Data Set}%
\end{pgfscope}%
\begin{pgfscope}%
\pgfsetrectcap%
\pgfsetmiterjoin%
\pgfsetlinewidth{0.803000pt}%
\definecolor{currentstroke}{rgb}{0.000000,0.000000,0.000000}%
\pgfsetstrokecolor{currentstroke}%
\pgfsetdash{}{0pt}%
\pgfpathmoveto{\pgfqpoint{0.515000in}{0.499444in}}%
\pgfpathlineto{\pgfqpoint{0.515000in}{1.654444in}}%
\pgfusepath{stroke}%
\end{pgfscope}%
\begin{pgfscope}%
\pgfsetrectcap%
\pgfsetmiterjoin%
\pgfsetlinewidth{0.803000pt}%
\definecolor{currentstroke}{rgb}{0.000000,0.000000,0.000000}%
\pgfsetstrokecolor{currentstroke}%
\pgfsetdash{}{0pt}%
\pgfpathmoveto{\pgfqpoint{2.065000in}{0.499444in}}%
\pgfpathlineto{\pgfqpoint{2.065000in}{1.654444in}}%
\pgfusepath{stroke}%
\end{pgfscope}%
\begin{pgfscope}%
\pgfsetrectcap%
\pgfsetmiterjoin%
\pgfsetlinewidth{0.803000pt}%
\definecolor{currentstroke}{rgb}{0.000000,0.000000,0.000000}%
\pgfsetstrokecolor{currentstroke}%
\pgfsetdash{}{0pt}%
\pgfpathmoveto{\pgfqpoint{0.515000in}{0.499444in}}%
\pgfpathlineto{\pgfqpoint{2.065000in}{0.499444in}}%
\pgfusepath{stroke}%
\end{pgfscope}%
\begin{pgfscope}%
\pgfsetrectcap%
\pgfsetmiterjoin%
\pgfsetlinewidth{0.803000pt}%
\definecolor{currentstroke}{rgb}{0.000000,0.000000,0.000000}%
\pgfsetstrokecolor{currentstroke}%
\pgfsetdash{}{0pt}%
\pgfpathmoveto{\pgfqpoint{0.515000in}{1.654444in}}%
\pgfpathlineto{\pgfqpoint{2.065000in}{1.654444in}}%
\pgfusepath{stroke}%
\end{pgfscope}%
\begin{pgfscope}%
\pgfsetbuttcap%
\pgfsetmiterjoin%
\definecolor{currentfill}{rgb}{1.000000,1.000000,1.000000}%
\pgfsetfillcolor{currentfill}%
\pgfsetfillopacity{0.800000}%
\pgfsetlinewidth{1.003750pt}%
\definecolor{currentstroke}{rgb}{0.800000,0.800000,0.800000}%
\pgfsetstrokecolor{currentstroke}%
\pgfsetstrokeopacity{0.800000}%
\pgfsetdash{}{0pt}%
\pgfpathmoveto{\pgfqpoint{1.288056in}{1.154445in}}%
\pgfpathlineto{\pgfqpoint{1.967778in}{1.154445in}}%
\pgfpathquadraticcurveto{\pgfqpoint{1.995556in}{1.154445in}}{\pgfqpoint{1.995556in}{1.182222in}}%
\pgfpathlineto{\pgfqpoint{1.995556in}{1.557222in}}%
\pgfpathquadraticcurveto{\pgfqpoint{1.995556in}{1.585000in}}{\pgfqpoint{1.967778in}{1.585000in}}%
\pgfpathlineto{\pgfqpoint{1.288056in}{1.585000in}}%
\pgfpathquadraticcurveto{\pgfqpoint{1.260278in}{1.585000in}}{\pgfqpoint{1.260278in}{1.557222in}}%
\pgfpathlineto{\pgfqpoint{1.260278in}{1.182222in}}%
\pgfpathquadraticcurveto{\pgfqpoint{1.260278in}{1.154445in}}{\pgfqpoint{1.288056in}{1.154445in}}%
\pgfpathlineto{\pgfqpoint{1.288056in}{1.154445in}}%
\pgfpathclose%
\pgfusepath{stroke,fill}%
\end{pgfscope}%
\begin{pgfscope}%
\pgfsetbuttcap%
\pgfsetmiterjoin%
\pgfsetlinewidth{1.003750pt}%
\definecolor{currentstroke}{rgb}{0.000000,0.000000,0.000000}%
\pgfsetstrokecolor{currentstroke}%
\pgfsetdash{}{0pt}%
\pgfpathmoveto{\pgfqpoint{1.315834in}{1.432222in}}%
\pgfpathlineto{\pgfqpoint{1.593611in}{1.432222in}}%
\pgfpathlineto{\pgfqpoint{1.593611in}{1.529444in}}%
\pgfpathlineto{\pgfqpoint{1.315834in}{1.529444in}}%
\pgfpathlineto{\pgfqpoint{1.315834in}{1.432222in}}%
\pgfpathclose%
\pgfusepath{stroke}%
\end{pgfscope}%
\begin{pgfscope}%
\definecolor{textcolor}{rgb}{0.000000,0.000000,0.000000}%
\pgfsetstrokecolor{textcolor}%
\pgfsetfillcolor{textcolor}%
\pgftext[x=1.704722in,y=1.432222in,left,base]{\color{textcolor}\rmfamily\fontsize{10.000000}{12.000000}\selectfont Neg}%
\end{pgfscope}%
\begin{pgfscope}%
\pgfsetbuttcap%
\pgfsetmiterjoin%
\definecolor{currentfill}{rgb}{0.000000,0.000000,0.000000}%
\pgfsetfillcolor{currentfill}%
\pgfsetlinewidth{0.000000pt}%
\definecolor{currentstroke}{rgb}{0.000000,0.000000,0.000000}%
\pgfsetstrokecolor{currentstroke}%
\pgfsetstrokeopacity{0.000000}%
\pgfsetdash{}{0pt}%
\pgfpathmoveto{\pgfqpoint{1.315834in}{1.236944in}}%
\pgfpathlineto{\pgfqpoint{1.593611in}{1.236944in}}%
\pgfpathlineto{\pgfqpoint{1.593611in}{1.334167in}}%
\pgfpathlineto{\pgfqpoint{1.315834in}{1.334167in}}%
\pgfpathlineto{\pgfqpoint{1.315834in}{1.236944in}}%
\pgfpathclose%
\pgfusepath{fill}%
\end{pgfscope}%
\begin{pgfscope}%
\definecolor{textcolor}{rgb}{0.000000,0.000000,0.000000}%
\pgfsetstrokecolor{textcolor}%
\pgfsetfillcolor{textcolor}%
\pgftext[x=1.704722in,y=1.236944in,left,base]{\color{textcolor}\rmfamily\fontsize{10.000000}{12.000000}\selectfont Pos}%
\end{pgfscope}%
\end{pgfpicture}%
\makeatother%
\endgroup%

&
	\vskip 0pt
	\qquad \qquad ROC Curve
	
	%% Creator: Matplotlib, PGF backend
%%
%% To include the figure in your LaTeX document, write
%%   \input{<filename>.pgf}
%%
%% Make sure the required packages are loaded in your preamble
%%   \usepackage{pgf}
%%
%% Also ensure that all the required font packages are loaded; for instance,
%% the lmodern package is sometimes necessary when using math font.
%%   \usepackage{lmodern}
%%
%% Figures using additional raster images can only be included by \input if
%% they are in the same directory as the main LaTeX file. For loading figures
%% from other directories you can use the `import` package
%%   \usepackage{import}
%%
%% and then include the figures with
%%   \import{<path to file>}{<filename>.pgf}
%%
%% Matplotlib used the following preamble
%%   
%%   \usepackage{fontspec}
%%   \makeatletter\@ifpackageloaded{underscore}{}{\usepackage[strings]{underscore}}\makeatother
%%
\begingroup%
\makeatletter%
\begin{pgfpicture}%
\pgfpathrectangle{\pgfpointorigin}{\pgfqpoint{2.221861in}{1.754444in}}%
\pgfusepath{use as bounding box, clip}%
\begin{pgfscope}%
\pgfsetbuttcap%
\pgfsetmiterjoin%
\definecolor{currentfill}{rgb}{1.000000,1.000000,1.000000}%
\pgfsetfillcolor{currentfill}%
\pgfsetlinewidth{0.000000pt}%
\definecolor{currentstroke}{rgb}{1.000000,1.000000,1.000000}%
\pgfsetstrokecolor{currentstroke}%
\pgfsetdash{}{0pt}%
\pgfpathmoveto{\pgfqpoint{0.000000in}{0.000000in}}%
\pgfpathlineto{\pgfqpoint{2.221861in}{0.000000in}}%
\pgfpathlineto{\pgfqpoint{2.221861in}{1.754444in}}%
\pgfpathlineto{\pgfqpoint{0.000000in}{1.754444in}}%
\pgfpathlineto{\pgfqpoint{0.000000in}{0.000000in}}%
\pgfpathclose%
\pgfusepath{fill}%
\end{pgfscope}%
\begin{pgfscope}%
\pgfsetbuttcap%
\pgfsetmiterjoin%
\definecolor{currentfill}{rgb}{1.000000,1.000000,1.000000}%
\pgfsetfillcolor{currentfill}%
\pgfsetlinewidth{0.000000pt}%
\definecolor{currentstroke}{rgb}{0.000000,0.000000,0.000000}%
\pgfsetstrokecolor{currentstroke}%
\pgfsetstrokeopacity{0.000000}%
\pgfsetdash{}{0pt}%
\pgfpathmoveto{\pgfqpoint{0.553581in}{0.499444in}}%
\pgfpathlineto{\pgfqpoint{2.103581in}{0.499444in}}%
\pgfpathlineto{\pgfqpoint{2.103581in}{1.654444in}}%
\pgfpathlineto{\pgfqpoint{0.553581in}{1.654444in}}%
\pgfpathlineto{\pgfqpoint{0.553581in}{0.499444in}}%
\pgfpathclose%
\pgfusepath{fill}%
\end{pgfscope}%
\begin{pgfscope}%
\pgfsetbuttcap%
\pgfsetroundjoin%
\definecolor{currentfill}{rgb}{0.000000,0.000000,0.000000}%
\pgfsetfillcolor{currentfill}%
\pgfsetlinewidth{0.803000pt}%
\definecolor{currentstroke}{rgb}{0.000000,0.000000,0.000000}%
\pgfsetstrokecolor{currentstroke}%
\pgfsetdash{}{0pt}%
\pgfsys@defobject{currentmarker}{\pgfqpoint{0.000000in}{-0.048611in}}{\pgfqpoint{0.000000in}{0.000000in}}{%
\pgfpathmoveto{\pgfqpoint{0.000000in}{0.000000in}}%
\pgfpathlineto{\pgfqpoint{0.000000in}{-0.048611in}}%
\pgfusepath{stroke,fill}%
}%
\begin{pgfscope}%
\pgfsys@transformshift{0.624035in}{0.499444in}%
\pgfsys@useobject{currentmarker}{}%
\end{pgfscope}%
\end{pgfscope}%
\begin{pgfscope}%
\definecolor{textcolor}{rgb}{0.000000,0.000000,0.000000}%
\pgfsetstrokecolor{textcolor}%
\pgfsetfillcolor{textcolor}%
\pgftext[x=0.624035in,y=0.402222in,,top]{\color{textcolor}\rmfamily\fontsize{10.000000}{12.000000}\selectfont \(\displaystyle {0.0}\)}%
\end{pgfscope}%
\begin{pgfscope}%
\pgfsetbuttcap%
\pgfsetroundjoin%
\definecolor{currentfill}{rgb}{0.000000,0.000000,0.000000}%
\pgfsetfillcolor{currentfill}%
\pgfsetlinewidth{0.803000pt}%
\definecolor{currentstroke}{rgb}{0.000000,0.000000,0.000000}%
\pgfsetstrokecolor{currentstroke}%
\pgfsetdash{}{0pt}%
\pgfsys@defobject{currentmarker}{\pgfqpoint{0.000000in}{-0.048611in}}{\pgfqpoint{0.000000in}{0.000000in}}{%
\pgfpathmoveto{\pgfqpoint{0.000000in}{0.000000in}}%
\pgfpathlineto{\pgfqpoint{0.000000in}{-0.048611in}}%
\pgfusepath{stroke,fill}%
}%
\begin{pgfscope}%
\pgfsys@transformshift{1.328581in}{0.499444in}%
\pgfsys@useobject{currentmarker}{}%
\end{pgfscope}%
\end{pgfscope}%
\begin{pgfscope}%
\definecolor{textcolor}{rgb}{0.000000,0.000000,0.000000}%
\pgfsetstrokecolor{textcolor}%
\pgfsetfillcolor{textcolor}%
\pgftext[x=1.328581in,y=0.402222in,,top]{\color{textcolor}\rmfamily\fontsize{10.000000}{12.000000}\selectfont \(\displaystyle {0.5}\)}%
\end{pgfscope}%
\begin{pgfscope}%
\pgfsetbuttcap%
\pgfsetroundjoin%
\definecolor{currentfill}{rgb}{0.000000,0.000000,0.000000}%
\pgfsetfillcolor{currentfill}%
\pgfsetlinewidth{0.803000pt}%
\definecolor{currentstroke}{rgb}{0.000000,0.000000,0.000000}%
\pgfsetstrokecolor{currentstroke}%
\pgfsetdash{}{0pt}%
\pgfsys@defobject{currentmarker}{\pgfqpoint{0.000000in}{-0.048611in}}{\pgfqpoint{0.000000in}{0.000000in}}{%
\pgfpathmoveto{\pgfqpoint{0.000000in}{0.000000in}}%
\pgfpathlineto{\pgfqpoint{0.000000in}{-0.048611in}}%
\pgfusepath{stroke,fill}%
}%
\begin{pgfscope}%
\pgfsys@transformshift{2.033126in}{0.499444in}%
\pgfsys@useobject{currentmarker}{}%
\end{pgfscope}%
\end{pgfscope}%
\begin{pgfscope}%
\definecolor{textcolor}{rgb}{0.000000,0.000000,0.000000}%
\pgfsetstrokecolor{textcolor}%
\pgfsetfillcolor{textcolor}%
\pgftext[x=2.033126in,y=0.402222in,,top]{\color{textcolor}\rmfamily\fontsize{10.000000}{12.000000}\selectfont \(\displaystyle {1.0}\)}%
\end{pgfscope}%
\begin{pgfscope}%
\definecolor{textcolor}{rgb}{0.000000,0.000000,0.000000}%
\pgfsetstrokecolor{textcolor}%
\pgfsetfillcolor{textcolor}%
\pgftext[x=1.328581in,y=0.223333in,,top]{\color{textcolor}\rmfamily\fontsize{10.000000}{12.000000}\selectfont False positive rate}%
\end{pgfscope}%
\begin{pgfscope}%
\pgfsetbuttcap%
\pgfsetroundjoin%
\definecolor{currentfill}{rgb}{0.000000,0.000000,0.000000}%
\pgfsetfillcolor{currentfill}%
\pgfsetlinewidth{0.803000pt}%
\definecolor{currentstroke}{rgb}{0.000000,0.000000,0.000000}%
\pgfsetstrokecolor{currentstroke}%
\pgfsetdash{}{0pt}%
\pgfsys@defobject{currentmarker}{\pgfqpoint{-0.048611in}{0.000000in}}{\pgfqpoint{-0.000000in}{0.000000in}}{%
\pgfpathmoveto{\pgfqpoint{-0.000000in}{0.000000in}}%
\pgfpathlineto{\pgfqpoint{-0.048611in}{0.000000in}}%
\pgfusepath{stroke,fill}%
}%
\begin{pgfscope}%
\pgfsys@transformshift{0.553581in}{0.551944in}%
\pgfsys@useobject{currentmarker}{}%
\end{pgfscope}%
\end{pgfscope}%
\begin{pgfscope}%
\definecolor{textcolor}{rgb}{0.000000,0.000000,0.000000}%
\pgfsetstrokecolor{textcolor}%
\pgfsetfillcolor{textcolor}%
\pgftext[x=0.278889in, y=0.503750in, left, base]{\color{textcolor}\rmfamily\fontsize{10.000000}{12.000000}\selectfont \(\displaystyle {0.0}\)}%
\end{pgfscope}%
\begin{pgfscope}%
\pgfsetbuttcap%
\pgfsetroundjoin%
\definecolor{currentfill}{rgb}{0.000000,0.000000,0.000000}%
\pgfsetfillcolor{currentfill}%
\pgfsetlinewidth{0.803000pt}%
\definecolor{currentstroke}{rgb}{0.000000,0.000000,0.000000}%
\pgfsetstrokecolor{currentstroke}%
\pgfsetdash{}{0pt}%
\pgfsys@defobject{currentmarker}{\pgfqpoint{-0.048611in}{0.000000in}}{\pgfqpoint{-0.000000in}{0.000000in}}{%
\pgfpathmoveto{\pgfqpoint{-0.000000in}{0.000000in}}%
\pgfpathlineto{\pgfqpoint{-0.048611in}{0.000000in}}%
\pgfusepath{stroke,fill}%
}%
\begin{pgfscope}%
\pgfsys@transformshift{0.553581in}{1.076944in}%
\pgfsys@useobject{currentmarker}{}%
\end{pgfscope}%
\end{pgfscope}%
\begin{pgfscope}%
\definecolor{textcolor}{rgb}{0.000000,0.000000,0.000000}%
\pgfsetstrokecolor{textcolor}%
\pgfsetfillcolor{textcolor}%
\pgftext[x=0.278889in, y=1.028750in, left, base]{\color{textcolor}\rmfamily\fontsize{10.000000}{12.000000}\selectfont \(\displaystyle {0.5}\)}%
\end{pgfscope}%
\begin{pgfscope}%
\pgfsetbuttcap%
\pgfsetroundjoin%
\definecolor{currentfill}{rgb}{0.000000,0.000000,0.000000}%
\pgfsetfillcolor{currentfill}%
\pgfsetlinewidth{0.803000pt}%
\definecolor{currentstroke}{rgb}{0.000000,0.000000,0.000000}%
\pgfsetstrokecolor{currentstroke}%
\pgfsetdash{}{0pt}%
\pgfsys@defobject{currentmarker}{\pgfqpoint{-0.048611in}{0.000000in}}{\pgfqpoint{-0.000000in}{0.000000in}}{%
\pgfpathmoveto{\pgfqpoint{-0.000000in}{0.000000in}}%
\pgfpathlineto{\pgfqpoint{-0.048611in}{0.000000in}}%
\pgfusepath{stroke,fill}%
}%
\begin{pgfscope}%
\pgfsys@transformshift{0.553581in}{1.601944in}%
\pgfsys@useobject{currentmarker}{}%
\end{pgfscope}%
\end{pgfscope}%
\begin{pgfscope}%
\definecolor{textcolor}{rgb}{0.000000,0.000000,0.000000}%
\pgfsetstrokecolor{textcolor}%
\pgfsetfillcolor{textcolor}%
\pgftext[x=0.278889in, y=1.553750in, left, base]{\color{textcolor}\rmfamily\fontsize{10.000000}{12.000000}\selectfont \(\displaystyle {1.0}\)}%
\end{pgfscope}%
\begin{pgfscope}%
\definecolor{textcolor}{rgb}{0.000000,0.000000,0.000000}%
\pgfsetstrokecolor{textcolor}%
\pgfsetfillcolor{textcolor}%
\pgftext[x=0.223333in,y=1.076944in,,bottom,rotate=90.000000]{\color{textcolor}\rmfamily\fontsize{10.000000}{12.000000}\selectfont True positive rate}%
\end{pgfscope}%
\begin{pgfscope}%
\pgfpathrectangle{\pgfqpoint{0.553581in}{0.499444in}}{\pgfqpoint{1.550000in}{1.155000in}}%
\pgfusepath{clip}%
\pgfsetbuttcap%
\pgfsetroundjoin%
\pgfsetlinewidth{1.505625pt}%
\definecolor{currentstroke}{rgb}{0.000000,0.000000,0.000000}%
\pgfsetstrokecolor{currentstroke}%
\pgfsetdash{{5.550000pt}{2.400000pt}}{0.000000pt}%
\pgfpathmoveto{\pgfqpoint{0.624035in}{0.551944in}}%
\pgfpathlineto{\pgfqpoint{2.033126in}{1.601944in}}%
\pgfusepath{stroke}%
\end{pgfscope}%
\begin{pgfscope}%
\pgfpathrectangle{\pgfqpoint{0.553581in}{0.499444in}}{\pgfqpoint{1.550000in}{1.155000in}}%
\pgfusepath{clip}%
\pgfsetrectcap%
\pgfsetroundjoin%
\pgfsetlinewidth{1.505625pt}%
\definecolor{currentstroke}{rgb}{0.000000,0.000000,0.000000}%
\pgfsetstrokecolor{currentstroke}%
\pgfsetdash{}{0pt}%
\pgfpathmoveto{\pgfqpoint{0.624035in}{0.551944in}}%
\pgfpathlineto{\pgfqpoint{0.625036in}{0.554862in}}%
\pgfpathlineto{\pgfqpoint{0.625106in}{0.554862in}}%
\pgfpathlineto{\pgfqpoint{0.625724in}{0.556445in}}%
\pgfpathlineto{\pgfqpoint{0.628452in}{0.562747in}}%
\pgfpathlineto{\pgfqpoint{0.629203in}{0.563802in}}%
\pgfpathlineto{\pgfqpoint{0.630211in}{0.565789in}}%
\pgfpathlineto{\pgfqpoint{0.630266in}{0.565789in}}%
\pgfpathlineto{\pgfqpoint{0.632486in}{0.569980in}}%
\pgfpathlineto{\pgfqpoint{0.634104in}{0.572618in}}%
\pgfpathlineto{\pgfqpoint{0.634597in}{0.573581in}}%
\pgfpathlineto{\pgfqpoint{0.651514in}{0.607634in}}%
\pgfpathlineto{\pgfqpoint{0.652703in}{0.609651in}}%
\pgfpathlineto{\pgfqpoint{0.654024in}{0.612290in}}%
\pgfpathlineto{\pgfqpoint{0.656518in}{0.616419in}}%
\pgfpathlineto{\pgfqpoint{0.657245in}{0.617412in}}%
\pgfpathlineto{\pgfqpoint{0.659207in}{0.621820in}}%
\pgfpathlineto{\pgfqpoint{0.659989in}{0.622906in}}%
\pgfpathlineto{\pgfqpoint{0.663045in}{0.627966in}}%
\pgfpathlineto{\pgfqpoint{0.667001in}{0.634113in}}%
\pgfpathlineto{\pgfqpoint{0.667501in}{0.635199in}}%
\pgfpathlineto{\pgfqpoint{0.671817in}{0.642649in}}%
\pgfpathlineto{\pgfqpoint{0.674060in}{0.646467in}}%
\pgfpathlineto{\pgfqpoint{0.684153in}{0.662920in}}%
\pgfpathlineto{\pgfqpoint{0.685474in}{0.664813in}}%
\pgfpathlineto{\pgfqpoint{0.744357in}{0.756263in}}%
\pgfpathlineto{\pgfqpoint{0.745263in}{0.757350in}}%
\pgfpathlineto{\pgfqpoint{0.747077in}{0.760051in}}%
\pgfpathlineto{\pgfqpoint{0.747992in}{0.761106in}}%
\pgfpathlineto{\pgfqpoint{0.756044in}{0.772033in}}%
\pgfpathlineto{\pgfqpoint{0.758663in}{0.775199in}}%
\pgfpathlineto{\pgfqpoint{0.760914in}{0.778428in}}%
\pgfpathlineto{\pgfqpoint{0.783585in}{0.810028in}}%
\pgfpathlineto{\pgfqpoint{0.786470in}{0.814436in}}%
\pgfpathlineto{\pgfqpoint{0.787377in}{0.815430in}}%
\pgfpathlineto{\pgfqpoint{0.819672in}{0.856343in}}%
\pgfpathlineto{\pgfqpoint{0.823252in}{0.860596in}}%
\pgfpathlineto{\pgfqpoint{0.824198in}{0.861620in}}%
\pgfpathlineto{\pgfqpoint{0.855312in}{0.899368in}}%
\pgfpathlineto{\pgfqpoint{0.856196in}{0.900268in}}%
\pgfpathlineto{\pgfqpoint{0.857955in}{0.902161in}}%
\pgfpathlineto{\pgfqpoint{0.858518in}{0.902938in}}%
\pgfpathlineto{\pgfqpoint{0.864060in}{0.908153in}}%
\pgfpathlineto{\pgfqpoint{0.865647in}{0.910077in}}%
\pgfpathlineto{\pgfqpoint{0.866836in}{0.911909in}}%
\pgfpathlineto{\pgfqpoint{0.871518in}{0.917031in}}%
\pgfpathlineto{\pgfqpoint{0.874223in}{0.919949in}}%
\pgfpathlineto{\pgfqpoint{0.876592in}{0.923301in}}%
\pgfpathlineto{\pgfqpoint{0.879844in}{0.927833in}}%
\pgfpathlineto{\pgfqpoint{0.899498in}{0.950742in}}%
\pgfpathlineto{\pgfqpoint{0.900248in}{0.951612in}}%
\pgfpathlineto{\pgfqpoint{0.900248in}{0.951643in}}%
\pgfpathlineto{\pgfqpoint{0.911138in}{0.964680in}}%
\pgfpathlineto{\pgfqpoint{0.912608in}{0.965736in}}%
\pgfpathlineto{\pgfqpoint{0.914351in}{0.967598in}}%
\pgfpathlineto{\pgfqpoint{0.917564in}{0.971510in}}%
\pgfpathlineto{\pgfqpoint{0.918369in}{0.972596in}}%
\pgfpathlineto{\pgfqpoint{0.919339in}{0.973683in}}%
\pgfpathlineto{\pgfqpoint{0.923975in}{0.979115in}}%
\pgfpathlineto{\pgfqpoint{0.925515in}{0.980201in}}%
\pgfpathlineto{\pgfqpoint{0.926625in}{0.981567in}}%
\pgfpathlineto{\pgfqpoint{0.927790in}{0.982623in}}%
\pgfpathlineto{\pgfqpoint{0.930416in}{0.985106in}}%
\pgfpathlineto{\pgfqpoint{0.941228in}{0.995785in}}%
\pgfpathlineto{\pgfqpoint{0.943073in}{0.997337in}}%
\pgfpathlineto{\pgfqpoint{1.018287in}{1.067492in}}%
\pgfpathlineto{\pgfqpoint{1.022055in}{1.070938in}}%
\pgfpathlineto{\pgfqpoint{1.026573in}{1.074911in}}%
\pgfpathlineto{\pgfqpoint{1.027590in}{1.075997in}}%
\pgfpathlineto{\pgfqpoint{1.028520in}{1.076680in}}%
\pgfpathlineto{\pgfqpoint{1.028536in}{1.076680in}}%
\pgfpathlineto{\pgfqpoint{1.043467in}{1.089159in}}%
\pgfpathlineto{\pgfqpoint{1.045265in}{1.090649in}}%
\pgfpathlineto{\pgfqpoint{1.051660in}{1.094623in}}%
\pgfpathlineto{\pgfqpoint{1.053153in}{1.096361in}}%
\pgfpathlineto{\pgfqpoint{1.055530in}{1.097975in}}%
\pgfpathlineto{\pgfqpoint{1.057570in}{1.099620in}}%
\pgfpathlineto{\pgfqpoint{1.060791in}{1.101669in}}%
\pgfpathlineto{\pgfqpoint{1.063434in}{1.104122in}}%
\pgfpathlineto{\pgfqpoint{1.075285in}{1.114024in}}%
\pgfpathlineto{\pgfqpoint{1.076301in}{1.115048in}}%
\pgfpathlineto{\pgfqpoint{1.076333in}{1.115048in}}%
\pgfpathlineto{\pgfqpoint{1.083736in}{1.121226in}}%
\pgfpathlineto{\pgfqpoint{1.085159in}{1.122778in}}%
\pgfpathlineto{\pgfqpoint{1.086222in}{1.123833in}}%
\pgfpathlineto{\pgfqpoint{1.092257in}{1.128552in}}%
\pgfpathlineto{\pgfqpoint{1.094110in}{1.129918in}}%
\pgfpathlineto{\pgfqpoint{1.095056in}{1.130973in}}%
\pgfpathlineto{\pgfqpoint{1.151327in}{1.172694in}}%
\pgfpathlineto{\pgfqpoint{1.157636in}{1.177040in}}%
\pgfpathlineto{\pgfqpoint{1.159239in}{1.178002in}}%
\pgfpathlineto{\pgfqpoint{1.160490in}{1.179088in}}%
\pgfpathlineto{\pgfqpoint{1.166220in}{1.184179in}}%
\pgfpathlineto{\pgfqpoint{1.168198in}{1.185669in}}%
\pgfpathlineto{\pgfqpoint{1.172427in}{1.189022in}}%
\pgfpathlineto{\pgfqpoint{1.174835in}{1.190667in}}%
\pgfpathlineto{\pgfqpoint{1.182379in}{1.195851in}}%
\pgfpathlineto{\pgfqpoint{1.183935in}{1.196938in}}%
\pgfpathlineto{\pgfqpoint{1.185334in}{1.198210in}}%
\pgfpathlineto{\pgfqpoint{1.191604in}{1.203208in}}%
\pgfpathlineto{\pgfqpoint{1.192675in}{1.203953in}}%
\pgfpathlineto{\pgfqpoint{1.192706in}{1.203953in}}%
\pgfpathlineto{\pgfqpoint{1.194809in}{1.205722in}}%
\pgfpathlineto{\pgfqpoint{1.206067in}{1.214011in}}%
\pgfpathlineto{\pgfqpoint{1.209944in}{1.216463in}}%
\pgfpathlineto{\pgfqpoint{1.222523in}{1.226086in}}%
\pgfpathlineto{\pgfqpoint{1.223781in}{1.227483in}}%
\pgfpathlineto{\pgfqpoint{1.232639in}{1.233102in}}%
\pgfpathlineto{\pgfqpoint{1.234061in}{1.234157in}}%
\pgfpathlineto{\pgfqpoint{1.235383in}{1.235430in}}%
\pgfpathlineto{\pgfqpoint{1.243349in}{1.241762in}}%
\pgfpathlineto{\pgfqpoint{1.245585in}{1.243656in}}%
\pgfpathlineto{\pgfqpoint{1.264910in}{1.258122in}}%
\pgfpathlineto{\pgfqpoint{1.271875in}{1.262809in}}%
\pgfpathlineto{\pgfqpoint{1.282679in}{1.269980in}}%
\pgfpathlineto{\pgfqpoint{1.284102in}{1.270973in}}%
\pgfpathlineto{\pgfqpoint{1.289363in}{1.275350in}}%
\pgfpathlineto{\pgfqpoint{1.292405in}{1.277554in}}%
\pgfpathlineto{\pgfqpoint{1.296939in}{1.280844in}}%
\pgfpathlineto{\pgfqpoint{1.311136in}{1.290809in}}%
\pgfpathlineto{\pgfqpoint{1.312128in}{1.291492in}}%
\pgfpathlineto{\pgfqpoint{1.312136in}{1.291492in}}%
\pgfpathlineto{\pgfqpoint{1.314544in}{1.293044in}}%
\pgfpathlineto{\pgfqpoint{1.318078in}{1.295248in}}%
\pgfpathlineto{\pgfqpoint{1.323402in}{1.299283in}}%
\pgfpathlineto{\pgfqpoint{1.325074in}{1.300649in}}%
\pgfpathlineto{\pgfqpoint{1.326841in}{1.301736in}}%
\pgfpathlineto{\pgfqpoint{1.327834in}{1.302419in}}%
\pgfpathlineto{\pgfqpoint{1.327881in}{1.302419in}}%
\pgfpathlineto{\pgfqpoint{1.332509in}{1.304685in}}%
\pgfpathlineto{\pgfqpoint{1.334073in}{1.305678in}}%
\pgfpathlineto{\pgfqpoint{1.337606in}{1.307572in}}%
\pgfpathlineto{\pgfqpoint{1.357502in}{1.320889in}}%
\pgfpathlineto{\pgfqpoint{1.365335in}{1.324893in}}%
\pgfpathlineto{\pgfqpoint{1.368345in}{1.326476in}}%
\pgfpathlineto{\pgfqpoint{1.401594in}{1.347833in}}%
\pgfpathlineto{\pgfqpoint{1.402414in}{1.348299in}}%
\pgfpathlineto{\pgfqpoint{1.409583in}{1.353359in}}%
\pgfpathlineto{\pgfqpoint{1.412413in}{1.354787in}}%
\pgfpathlineto{\pgfqpoint{1.414008in}{1.355625in}}%
\pgfpathlineto{\pgfqpoint{1.429229in}{1.365124in}}%
\pgfpathlineto{\pgfqpoint{1.430409in}{1.365838in}}%
\pgfpathlineto{\pgfqpoint{1.445216in}{1.374871in}}%
\pgfpathlineto{\pgfqpoint{1.446928in}{1.375895in}}%
\pgfpathlineto{\pgfqpoint{1.451150in}{1.378441in}}%
\pgfpathlineto{\pgfqpoint{1.452674in}{1.379527in}}%
\pgfpathlineto{\pgfqpoint{1.457372in}{1.382507in}}%
\pgfpathlineto{\pgfqpoint{1.476635in}{1.393776in}}%
\pgfpathlineto{\pgfqpoint{1.478511in}{1.395048in}}%
\pgfpathlineto{\pgfqpoint{1.479934in}{1.396104in}}%
\pgfpathlineto{\pgfqpoint{1.484953in}{1.398898in}}%
\pgfpathlineto{\pgfqpoint{1.532242in}{1.424694in}}%
\pgfpathlineto{\pgfqpoint{1.539387in}{1.428574in}}%
\pgfpathlineto{\pgfqpoint{1.541404in}{1.429660in}}%
\pgfpathlineto{\pgfqpoint{1.549668in}{1.433820in}}%
\pgfpathlineto{\pgfqpoint{1.551786in}{1.435279in}}%
\pgfpathlineto{\pgfqpoint{1.557946in}{1.438756in}}%
\pgfpathlineto{\pgfqpoint{1.559846in}{1.439842in}}%
\pgfpathlineto{\pgfqpoint{1.561371in}{1.440773in}}%
\pgfpathlineto{\pgfqpoint{1.563333in}{1.441767in}}%
\pgfpathlineto{\pgfqpoint{1.565279in}{1.442946in}}%
\pgfpathlineto{\pgfqpoint{1.571862in}{1.446051in}}%
\pgfpathlineto{\pgfqpoint{1.576185in}{1.448130in}}%
\pgfpathlineto{\pgfqpoint{1.577334in}{1.448503in}}%
\pgfpathlineto{\pgfqpoint{1.580750in}{1.450303in}}%
\pgfpathlineto{\pgfqpoint{1.582306in}{1.451359in}}%
\pgfpathlineto{\pgfqpoint{1.584995in}{1.452414in}}%
\pgfpathlineto{\pgfqpoint{1.587223in}{1.453749in}}%
\pgfpathlineto{\pgfqpoint{1.590937in}{1.455456in}}%
\pgfpathlineto{\pgfqpoint{1.592446in}{1.456388in}}%
\pgfpathlineto{\pgfqpoint{1.592446in}{1.456512in}}%
\pgfpathlineto{\pgfqpoint{1.594885in}{1.457505in}}%
\pgfpathlineto{\pgfqpoint{1.596894in}{1.458530in}}%
\pgfpathlineto{\pgfqpoint{1.596894in}{1.458561in}}%
\pgfpathlineto{\pgfqpoint{1.611231in}{1.465048in}}%
\pgfpathlineto{\pgfqpoint{1.614921in}{1.466135in}}%
\pgfpathlineto{\pgfqpoint{1.616579in}{1.467221in}}%
\pgfpathlineto{\pgfqpoint{1.618987in}{1.468277in}}%
\pgfpathlineto{\pgfqpoint{1.621027in}{1.469022in}}%
\pgfpathlineto{\pgfqpoint{1.631213in}{1.473833in}}%
\pgfpathlineto{\pgfqpoint{1.632542in}{1.475044in}}%
\pgfpathlineto{\pgfqpoint{1.638273in}{1.477310in}}%
\pgfpathlineto{\pgfqpoint{1.641634in}{1.479110in}}%
\pgfpathlineto{\pgfqpoint{1.644284in}{1.480197in}}%
\pgfpathlineto{\pgfqpoint{1.645285in}{1.480756in}}%
\pgfpathlineto{\pgfqpoint{1.645309in}{1.480756in}}%
\pgfpathlineto{\pgfqpoint{1.659099in}{1.486436in}}%
\pgfpathlineto{\pgfqpoint{1.664884in}{1.489634in}}%
\pgfpathlineto{\pgfqpoint{1.667089in}{1.490720in}}%
\pgfpathlineto{\pgfqpoint{1.671294in}{1.492893in}}%
\pgfpathlineto{\pgfqpoint{1.675328in}{1.494756in}}%
\pgfpathlineto{\pgfqpoint{1.676290in}{1.495066in}}%
\pgfpathlineto{\pgfqpoint{1.676352in}{1.495066in}}%
\pgfpathlineto{\pgfqpoint{1.691980in}{1.501989in}}%
\pgfpathlineto{\pgfqpoint{1.699461in}{1.505031in}}%
\pgfpathlineto{\pgfqpoint{1.700470in}{1.505589in}}%
\pgfpathlineto{\pgfqpoint{1.704394in}{1.507328in}}%
\pgfpathlineto{\pgfqpoint{1.706904in}{1.508352in}}%
\pgfpathlineto{\pgfqpoint{1.715878in}{1.512201in}}%
\pgfpathlineto{\pgfqpoint{1.718185in}{1.513226in}}%
\pgfpathlineto{\pgfqpoint{1.720491in}{1.514033in}}%
\pgfpathlineto{\pgfqpoint{1.733828in}{1.519962in}}%
\pgfpathlineto{\pgfqpoint{1.734797in}{1.520272in}}%
\pgfpathlineto{\pgfqpoint{1.734883in}{1.520272in}}%
\pgfpathlineto{\pgfqpoint{1.748963in}{1.525270in}}%
\pgfpathlineto{\pgfqpoint{1.750862in}{1.526357in}}%
\pgfpathlineto{\pgfqpoint{1.757781in}{1.529212in}}%
\pgfpathlineto{\pgfqpoint{1.759016in}{1.529678in}}%
\pgfpathlineto{\pgfqpoint{1.767483in}{1.532813in}}%
\pgfpathlineto{\pgfqpoint{1.770539in}{1.533900in}}%
\pgfpathlineto{\pgfqpoint{1.773143in}{1.534986in}}%
\pgfpathlineto{\pgfqpoint{1.785792in}{1.538711in}}%
\pgfpathlineto{\pgfqpoint{1.787457in}{1.539549in}}%
\pgfpathlineto{\pgfqpoint{1.796181in}{1.541940in}}%
\pgfpathlineto{\pgfqpoint{1.800528in}{1.543026in}}%
\pgfpathlineto{\pgfqpoint{1.803295in}{1.543833in}}%
\pgfpathlineto{\pgfqpoint{1.806938in}{1.544920in}}%
\pgfpathlineto{\pgfqpoint{1.808908in}{1.545882in}}%
\pgfpathlineto{\pgfqpoint{1.886545in}{1.566773in}}%
\pgfpathlineto{\pgfqpoint{1.891033in}{1.567798in}}%
\pgfpathlineto{\pgfqpoint{1.933396in}{1.579035in}}%
\pgfpathlineto{\pgfqpoint{1.954879in}{1.583816in}}%
\pgfpathlineto{\pgfqpoint{1.960180in}{1.585212in}}%
\pgfpathlineto{\pgfqpoint{1.971992in}{1.588410in}}%
\pgfpathlineto{\pgfqpoint{1.973509in}{1.588844in}}%
\pgfpathlineto{\pgfqpoint{1.996618in}{1.593997in}}%
\pgfpathlineto{\pgfqpoint{2.014223in}{1.597909in}}%
\pgfpathlineto{\pgfqpoint{2.033126in}{1.601944in}}%
\pgfpathlineto{\pgfqpoint{2.033126in}{1.601944in}}%
\pgfusepath{stroke}%
\end{pgfscope}%
\begin{pgfscope}%
\pgfsetrectcap%
\pgfsetmiterjoin%
\pgfsetlinewidth{0.803000pt}%
\definecolor{currentstroke}{rgb}{0.000000,0.000000,0.000000}%
\pgfsetstrokecolor{currentstroke}%
\pgfsetdash{}{0pt}%
\pgfpathmoveto{\pgfqpoint{0.553581in}{0.499444in}}%
\pgfpathlineto{\pgfqpoint{0.553581in}{1.654444in}}%
\pgfusepath{stroke}%
\end{pgfscope}%
\begin{pgfscope}%
\pgfsetrectcap%
\pgfsetmiterjoin%
\pgfsetlinewidth{0.803000pt}%
\definecolor{currentstroke}{rgb}{0.000000,0.000000,0.000000}%
\pgfsetstrokecolor{currentstroke}%
\pgfsetdash{}{0pt}%
\pgfpathmoveto{\pgfqpoint{2.103581in}{0.499444in}}%
\pgfpathlineto{\pgfqpoint{2.103581in}{1.654444in}}%
\pgfusepath{stroke}%
\end{pgfscope}%
\begin{pgfscope}%
\pgfsetrectcap%
\pgfsetmiterjoin%
\pgfsetlinewidth{0.803000pt}%
\definecolor{currentstroke}{rgb}{0.000000,0.000000,0.000000}%
\pgfsetstrokecolor{currentstroke}%
\pgfsetdash{}{0pt}%
\pgfpathmoveto{\pgfqpoint{0.553581in}{0.499444in}}%
\pgfpathlineto{\pgfqpoint{2.103581in}{0.499444in}}%
\pgfusepath{stroke}%
\end{pgfscope}%
\begin{pgfscope}%
\pgfsetrectcap%
\pgfsetmiterjoin%
\pgfsetlinewidth{0.803000pt}%
\definecolor{currentstroke}{rgb}{0.000000,0.000000,0.000000}%
\pgfsetstrokecolor{currentstroke}%
\pgfsetdash{}{0pt}%
\pgfpathmoveto{\pgfqpoint{0.553581in}{1.654444in}}%
\pgfpathlineto{\pgfqpoint{2.103581in}{1.654444in}}%
\pgfusepath{stroke}%
\end{pgfscope}%
\begin{pgfscope}%
\pgfsetbuttcap%
\pgfsetmiterjoin%
\definecolor{currentfill}{rgb}{1.000000,1.000000,1.000000}%
\pgfsetfillcolor{currentfill}%
\pgfsetfillopacity{0.800000}%
\pgfsetlinewidth{1.003750pt}%
\definecolor{currentstroke}{rgb}{0.800000,0.800000,0.800000}%
\pgfsetstrokecolor{currentstroke}%
\pgfsetstrokeopacity{0.800000}%
\pgfsetdash{}{0pt}%
\pgfpathmoveto{\pgfqpoint{0.832747in}{0.568889in}}%
\pgfpathlineto{\pgfqpoint{2.006358in}{0.568889in}}%
\pgfpathquadraticcurveto{\pgfqpoint{2.034136in}{0.568889in}}{\pgfqpoint{2.034136in}{0.596666in}}%
\pgfpathlineto{\pgfqpoint{2.034136in}{0.776388in}}%
\pgfpathquadraticcurveto{\pgfqpoint{2.034136in}{0.804166in}}{\pgfqpoint{2.006358in}{0.804166in}}%
\pgfpathlineto{\pgfqpoint{0.832747in}{0.804166in}}%
\pgfpathquadraticcurveto{\pgfqpoint{0.804970in}{0.804166in}}{\pgfqpoint{0.804970in}{0.776388in}}%
\pgfpathlineto{\pgfqpoint{0.804970in}{0.596666in}}%
\pgfpathquadraticcurveto{\pgfqpoint{0.804970in}{0.568889in}}{\pgfqpoint{0.832747in}{0.568889in}}%
\pgfpathlineto{\pgfqpoint{0.832747in}{0.568889in}}%
\pgfpathclose%
\pgfusepath{stroke,fill}%
\end{pgfscope}%
\begin{pgfscope}%
\pgfsetrectcap%
\pgfsetroundjoin%
\pgfsetlinewidth{1.505625pt}%
\definecolor{currentstroke}{rgb}{0.000000,0.000000,0.000000}%
\pgfsetstrokecolor{currentstroke}%
\pgfsetdash{}{0pt}%
\pgfpathmoveto{\pgfqpoint{0.860525in}{0.700000in}}%
\pgfpathlineto{\pgfqpoint{0.999414in}{0.700000in}}%
\pgfpathlineto{\pgfqpoint{1.138303in}{0.700000in}}%
\pgfusepath{stroke}%
\end{pgfscope}%
\begin{pgfscope}%
\definecolor{textcolor}{rgb}{0.000000,0.000000,0.000000}%
\pgfsetstrokecolor{textcolor}%
\pgfsetfillcolor{textcolor}%
\pgftext[x=1.249414in,y=0.651388in,left,base]{\color{textcolor}\rmfamily\fontsize{10.000000}{12.000000}\selectfont AUC=0.654}%
\end{pgfscope}%
\end{pgfpicture}%
\makeatother%
\endgroup%

	
&
	\vskip 0pt
	\begin{tabular}{cc|c|c|}
	&\multicolumn{1}{c}{}& \multicolumn{2}{c}{Prediction} \cr
	&\multicolumn{1}{c}{} & \multicolumn{1}{c}{N} & \multicolumn{1}{c}{P} \cr\cline{3-4}
	\multirow{2}{*}{\rotatebox[origin=c]{90}{Actual}}&N &
150,771 & 0
	\vrule width 0pt height 10pt depth 2pt \cr\cline{3-4}
	&P & 
26621 & 0
	\vrule width 0pt height 10pt depth 2pt \cr\cline{3-4}
	\end{tabular}

	\hfil\begin{tabular}{ll}
	\cr
	0.8499 & Accuracy\cr
und & Precision \cr	0.0 & Recall \cr	und & F1 \cr	0.659 & AUC \cr
\end{tabular}

\cr
\end{tabular}
} % End parbox

\

In this study, we  have arbitrarily decided that we are willing to trade off up to two false positives to get one more true positive.  Once we moved our decision thresholds to the ethical tradeoff point, the accuracy only varied from 0.836 to 0.854.  The difference in accuracy tells us how many more (or fewer) false positives than true positives we have, with them being equal at 0.8499, and we get the same information from precision being less than, more than, or equal to 0.5.    Therefore, we are not going to consider accuracy in evaluating our models. 

%%%
\subsubsection{ML Algorithms for Imbalanced Data}

{\bf [Expand this subsubsection]}

\begin{itemize}
	\item Random Undersampling Composite Models
	\item Bagging
	\item Boosting
\end{itemize}

%%%%%
\subsection{Models}

We used seven binary classification algorithms.  Three of them take class weights.

\

\hfil\begin{tabular}{llc}
&& Class \cr
Model & Source & Weights \cr\hline
KerasClassifier with the Binary Focal Crossentropy loss function & Keras & Yes \cr
Balanced Random Forest Classifier & Imbalanced-Learn & Yes \cr
Balanced Bagging Classifier & Imbalanced-Learn & No \cr
RUSBoost Classifier & Imbalanced-Learn & No \cr
Easy Ensemble Classifier with AdaBoost Estimator & Imbalanced-Learn & No \cr
Logistic Regression Classifier & Scikit-Learn & Yes \cr
AdaBoost  Classifier & Scikit-Learn & No \cr
\end{tabular}

\


For the focal loss function, we tried seven different combinations of the hyperparameters $\alpha$ for class weights and $\gamma$ for penalty on badly misclassified samples.  For the random forest and bagging models we tried three values of $\alpha$.  Altogether we had seventeen model/hyperparameter combinations.  We learned each of the seven models on datasets with the easy, medium, and hard features, and on the hard features we tested with Tomek undersampling 0, 1, and 2 times, for a total of five datasets, giving eighty-five model/hyperparameter/dataset combinations.    We learned each of those sixty-five with two different random seeds, for a total of one hundred seventy results.  

\

\hfil\noindent\begin{tabular}{ccc}
	\multicolumn{3}{c}{Seventeen Models} \cr
	Model & $\alpha$ & $\gamma$ \cr\hline
	Focal & 1/2 & 0.0 \cr
	Focal & 2/3 & 0.0 \cr
	Focal & 2/3 & 0.5 \cr
	Focal & 2/3 & 1.0 \cr
	Focal & 2/3 & 2.0 \cr
	Focal & 2/3 & 5.0 \cr
	Focal & 0.85 & 0.0 \cr
	Random Forest & 1/2 & \cr
	Random Forest & 2/3 & \cr
	Random Forest & 0.85 & \cr
	Bagging && \cr
	RUSBoost && \cr
	Easy Ens && \cr
	Log Reg & 1/2 & \cr
	Log Reg & 2/3 & \cr
	Log Reg & 0.85 & \cr
	AdaBoost && \cr
\end{tabular}
\quad
$\times$
\quad
\begin{tabular}{cc}
	\multicolumn{2}{c}{Seven Datasets} \cr
	Features & Tomek \cr\hline
	Hard & None \cr
	Hard & Once \cr
	Hard & Twice \cr
	Medium & None \cr
	Medium & Once \cr
	Medium & Twice \cr
	Easy & None \cr
\end{tabular}
\quad
$\times$
\quad
\begin{tabular}{cc}
	Run twice with \cr
	different \cr
	random seeds \cr\hline
	Random seed 1 \cr
	Random seed 2
\end{tabular}
\quad 
$=$ 
\quad 
\begin{tabular}{c}
	238  \cr Sets of  \cr
	Results \cr
\end{tabular}










