%%%
\section{Conclusions and Discussion}
\label{conclusions}

\subsection{Scenario Reprise}
\label{scenario_reprise}

The (fictitious) city of Springfield is considering implementing immediate dispatch of ambulances to automated crash notifications from cell phones, and are considering an AI recommendation system to predict which crash persons are more likely to need an ambulance. 

Of all of Springfield's crash notifications, 20\% come first from automated crash notifications from cell phones.  Only 10\% percent of all crashes need an ambulance, but those that trigger automated notifications tend to be more serious, so of those, 15\% need an ambulance.  Of all of the ambulances dispatched, half go to crashes.  Scaling the numbers out of 100 ambulances dispatched, 50 go to non-crashes, 50 go to crashes, and there are 450 crashes that do not need an ambulance.  There are 100 automated crash notifications, and 15 of those need an ambulance.  Of the 15 that need an ambulance, our AI recommendation system correctly labels some (TP), and they get more prompt medical attention, but it incorrectly labels others (FN) who still get an ambulance, but only after a later 911/119 call from an eyewitness.  Of the 85 who do not need an ambulance, the AI recommendation system correctly will label most of them (TN), but incorrectly label some of them, and Springfield will send an unneeded ambulance that would not have been sent if they had waited for an eyewitness call.  


\begin{figure}[h]
\begin{minipage}{\linewidth}
{\normalfont\normalsize
\begin{tabular}{p{2in}p{3in}}
\begin{tabular}{c c  | c | c | c}
	& \multicolumn{1}{c}{} & \multicolumn{2}{c}{Prediction}  \cr
	&\multicolumn{1}{c}{} & \multicolumn{1}{c}{PN} & \multicolumn{1}{c}{PP} \cr\cline{3-4}
	\multirow{2}{*}{Actual} & N & TN & FP \vrule width 0pt height 10pt depth 2pt \cr\cline{3-4}
	 & P & FN & TP	\vrule width 0pt height 10pt depth 4pt \cr\cline{3-4}
\end{tabular}
&
\begin{tabular}{c c  | c | c | c}
	\multicolumn{2}{@{}l}{Recommendation} & \multicolumn{1}{ @{} c @{} }{Wait for Call} & \multicolumn{1}{ @{} c @{} }{Immediately}   \cr
	&\multicolumn{1}{ @{} c @{} }{} & \multicolumn{1}{ @{} c @{} }{from Eyewitness} & \multicolumn{1}{ @{} c @{} }{Dispatch} \cr\cline{3-4}
	Needs & No & Correct & Increased Cost
		\vrule width 0pt height 10pt depth 2pt \cr\cline{3-4}
	Ambulance? \ \ &Yes & 
		Normal Delay & \ Prompt Medical Help \
		\vrule width 0pt height 10pt depth 4pt \cr\cline{3-4}
\end{tabular}
\cr		
\end{tabular}
}
\end{minipage}
\caption{\normalfont\normalsize Confusion matrix for ambulance dispatch.  Figure accompanies \S\ref{intro_scenario}}
\label{intro_confusion}
\end{figure}

\FloatBarrier

If Springfield built a recommendation system based on the Balanced Random Forest model with the Hard features, depending on which of the three budgetary decision criteria they chose, Springfield crash persons would see the following values for TN, FP, FN, and TP.  With the first budgetary decision criterion, for each 50 ambulances dispatched, the model would recommend sending 1.7 sooner to crash persons who need them while sending an additional 0.7 not needed.
Note that those numbers come with the caveat that getting instantaneous access to the Hard features would likely be expensive,  if even possible.

\begin{figure}[h]
	\begin{tikzpicture}[x=0.075cm, y=0.6cm, font=\normalfont\normalsize] % 15 cm wide


	\draw [<->, color=black]  (0,3) -- (100,3) 
		node [midway, color=black, fill=white, align=center] 
		{Ambulance Needed};

	\draw [<->, color=black]  (0,4) -- (120,4) 
		node [midway, color=black, fill=white, align=center] 
		{Ambulance Sent};


	\path (207,0) circle (0pt);
	\draw [color=black] (0,0) -- (190,0);
	\draw [color=black, dashed] (190,0) -- (200,0);
	\draw [color=black] (0,2) -- (0,-1);
	\draw [color=black] (50,2) -- (50,-1);
	\draw [color=black] (100,2) -- (100,-1.5);
	\draw [color=black] (120,2) -- (120,-1);
	\node (A) at (25,0) {};
	\node (B) [above=-2pt of A, align=center, text=black] {
		Non-Crash \\ Needs Ambulance \\[0.5em] 50
	};
	\node (C) at (75,0) {};
	\node (D) [above=-2pt of C, align=center, text=black] {
		Crash \\ Needs Ambulance \\[0.5em] 50
	};
	\node (E) at (200,0) {};
	\node (F) [above left =-2pt and 0pt of E, align=right, text=black] {
		Crash Does Not Need Ambulance \\ No Ambulance Sent \\[0.5em] 450
	};
	\node (G) at (110,0) {};
	\node (H) [above=-2pt of G, align=center, text=black] {
		False \\ Alarm \\[0.5em] ?
	};

	\draw [color=black] (85,0) -- (85,-1.5);
	\draw [color=black] (185,0) -- (185,-1.5);
%	\path (85,-2) -- (100,-2) node [below, midway, color=black] {15};
%	\path (100,-1) -- (185,-1) node [below, midway, color=black] {85};
	
%	\node (G) at (110,2) [color=black, align=left] {False \\  Alarm};
	\path (100,-0.5) -- (120,-0.5) node [midway, color=black] {FP};
	\path (120,-0.5) -- (185,-0.5) node [midway, color=black] {TN};
	\draw [color=black] (92.5,0) -- (92.5,-1.5);
	\path (85,-0.5) -- (92.5,-0.5) node [midway, color=black] {F};
	\path (85,-1.2) -- (92.5,-1.0) node [midway, color=black] {N};
	\path (100,-0.5) -- (92.5,-0.5) node [midway, color=black] {T};
	\path (100,-1.2) -- (92.5,-1.0) node [midway, color=black] {P};
	
	\node () at (50,-2.2) [right, color=black] {FP/P = 0.05};
	\node () at (92.5,-2.2) [left, color=black] {13.3};
	\node () at (92.5,-2.2) [right, color=black] {1.7};
	\node () at (110,-2.2) [color=black] {0.7};
	\node () at (152.5,-2.2) [color=black] {84.3};

	\node () at (50,-3.1) [right, color=black] {Prec = 2/3};
	\node () at (92.5,-3.1) [left, color=black] {12.2};
	\node () at (92.5,-3.1) [right, color=black] {2.8};
	\node () at (110,-3.1) [color=black] {1.5};
	\node () at (152.5,-3.1) [color=black] {83.5};

	\node () at (50,-4) [right, color=black] {$m$Prob = 0.50};
	\node () at (92.5,-4) [left, color=black] {13.3};
	\node () at (92.5,-4) [right, color=black] {1.7};
	\node () at (110,-4) [color=black] {0.8};
	\node () at (152.5,-4) [color=black] {84.2};
	
	\node () at (40,-2.6) [left, color=black] {Budgetary Decision};
	\node () at (40,-3.4) [left, color=black] {Criteria Metrics};

%	\node () at (50,-3.2) [left, color=black] {$\displaystyle \Biggl\{$};


	\node () at (40,-3.2) [right, color=black] {
		$\displaystyle 
			\left\{
			\vrule width 0pt height 14pt depth 14pt
			\right.
		$
	};

	\draw [<->, color=black]  (100.5,-5) -- (185,-5) 
		node [midway, color=black, fill=white, align=center] 
		{85};

	\draw [<->, color=black]  (85,-5) -- (99.5,-5) 
		node [midway, color=black, fill=white, align=center] 
		{15};

	\draw [<->, color=black]  (50,-6) -- (92,-6) 
		node [midway, color=black, fill=white, align=center] 
		{Eyewitness};

	\draw [<->, color=black]  (93,-6) -- (120,-6) 
		node [midway, color=black, fill=white, align=center] 
		{Immediate};

	\draw [<->, color=black]  (85,-7) -- (185,-7) 
		node [midway, color=black, fill=white, align=center] 
		{Automated Notifications};

	

	
\end{tikzpicture}








\caption{\normalfont\normalsize Springfield after implementing immediate dispatch of ambulances.  Figure accompanies \S\ref{scenario_reprise}}
\label{intro_springfield_conclusion}
\end{figure}

\FloatBarrier

Should Springfield implement such an AI recommendation system for immediate dispatch of ambulances based on automated notifications from cell phones?  If the only cost were the ambulances falsely dispatched (FP), then probably yes.  Considering the fixed costs of implementing such a system, a smaller city would have other options of programs that can better improve health and save more lives with the same amount of money.  

Springfield should consider changing the values of the metrics, perhaps to allow a FP/P = 10\% increase in the number of ambulances sent to automated crash notifications.  The best model on the Hard features, however, would give TP = 2.9 and FP = 1.5, more than doubling the cost (FP) without doubling the benefit (TP) from FP/P = 5\% with TP = 1.7 and FP = 0.7.  As we saw in our model output histograms, the returns diminish very quickly.  

With better data and more appropriate model algorithms, of course, we might get models that do a better job of separating the negative and positive class, an opportunity for future research.  (See \S \ref{simplifying_assumptions})

%%%%%
\subsection{Discussion}

Choosing a tradeoff between saving lives and saving money is a fraught political question, but the decisions can be informed with models showing the likely outcomes of the possible choices.  

Many cell phones can automatically notify the emergency dispatcher when they detect the deceleration profile of a crash.  The dispatcher will immediately send police to investigate, but should they also immediately dispatch an ambulance, which has about a small chance of being needed, or wait for a call from an eyewitness?  We have explored options for an AI recommendation system that would take the known information about the crash, a model built on historical crash data, and a decision criterion chosen by the political decision makers, and return a recommendation for whether to immediately dispatch an ambulance or to wait for a call from an eyewitness.

The main problem with such a recommendation system is that it will recommend sending some ambulances that are not needed.  Budgets are limited, and priorities have to be chosen.  We showed how to incorporate three kinds of decision criterion:  Percent increase in number of ambulance calls, percent of immediately dispatched ambulances actually needed, and minimum probability that each immediately dispatched ambulance is needed.  

We considered many challenges, including the randomness inherent in machine learning models and how the numerics severely limit the precision we can claim in our results.  Another challenge is identifying with useful numerical precision the decision threshold $\theta$ that satisfies a given metric when the value of that metric is only stable after much smoothing.  We found that finding $\theta$ for the first metric, a cap on increased ambulance runs, is straightforward, but Precision requires some smoothing, and marginal probability that an ambulance is needed is very unstable on small intervals.

The most significant challenge of implementing such a system is the cost and availability of the input data required to make a useful recommendation system.  We considered three sets of input features that we called Easy, Medium, and Hard, but that can also be thought of as Free, Expensive, and Problematic.  Even through the randomness and numerics, the results show that the results of models built on the Hard features are clearly better than those built on the Medium features, which are clearly better than those built on the Easy features.  The models built on the Easy features, however, are still better than random guessing, and a recommendation system built on just those features may be worth the expense, especially since the emergency dispatcher already has that information.  




