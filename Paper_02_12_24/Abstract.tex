Some new cell phones can automatically notify an emergency dispatcher if the phone detects the deceleration profile of a vehicular crash.  Most crash notifications come from an eyewitness who can say whether an ambulance is needed, but the automated notification from the cell phone cannot provide that information directly.  Should the dispatcher immediately send an ambulance before receiving an eyewitness report?  There are three options: Always, Wait, and Sometimes.  The ``Always'' option refers to sending an ambulance to every automatically reported crash, even though most of them will not be needed.  In the ``Wait'' option, the dispatcher sends police, but always waits for a call from an eyewitness (perhaps the police) before sending an ambulance.  In the ``Sometimes'' option, the dispatcher has some system that recommends whether to immediately dispatch an ambulance, reserving the option to send one later based on an eyewitness report.


This paper explores one option for building a machine learning (ML) model for making a recommendation in the ``Sometimes'' option.    Our goal is to build a model that returns, for each feature vector (crash report, sample), a value $p \in [0,1]$ that increases with the probability that the person needs an ambulance.  Using a decision threshold $\theta$, we immediately send ambulances to those automated crash reports with $p > \theta$, and wait for eyewitness confirmation for those reports with $p < \theta$. In an actual implementation, the choice of $\theta$ is policy-based, not technical.

The costs of the false positives (FP) and false negatives (FN) in dispatching ambulances are very different.  The cost of sending an ambulance when one is not needed (FP) is measured in dollars, but the cost of not promptly sending an ambulance when one is needed (FN) is measured in lives.  Choosing the decision threshold $\theta$ is ethically problematic, but governments make such a tradeoff when they set budgets for emergency services.  

We consider and interpret three options for the decision threshold $\theta$ based on the political consideration, ``How much will it cost?''  How many automated ambulance dispatches are we willing to fund (FP + TP) for each one of them that is actually needed (TP)?  We explore three versions of that cost question.  

We show that the quality of the model depends highly on the input data available, having considered three levels of data availability.  The ``Easy'' level includes data the emergency dispatcher has before the notification, like time of day and weather.  The ``Medium'' level adds information about the location and information from the cell service provider about the user, like the age and sex.  The ``Hard'' level adds information that requires having access to records about the vehicle likely to be driven by the cell phone user and detailed and temporal information about the location, like lighting conditions and whether it is currently a work zone.  

We used the data of the Crash Report Sampling System (CRSS) to validate our approach.  We have applied new methods (for this dataset in the literature) to handle missing data, and we have investigated several methods for handling the data imbalance.  To promote discussion and future research, we have included all of the code we used in our analysis.  
