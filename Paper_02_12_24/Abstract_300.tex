Some new cell phones can detect the deceleration profile of a car crash and automatically notify emergency services who will immediately send police, but should they also immediately dispatch an ambulance?  Getting medical attention sooner could be life-saving, but most crashes do not require an ambulance.  Using historical data, we build an AI recommendation system to predict which ambulances to immediately dispatch within set budgetary constraints.  Setting those budgetary constraints is a public policy decision, and we consider three options.  We measure the success of the system by its recall, the proportion of needed ambulances that it correctly recommends for immediate dispatch.

The recall depends heavily on the features of the input data.  Does the notification come with only a location, or does the cell service provider also share what it knows about the primary user of the phone, like name, age, and sex?  Can the system instantaneously correlate the name with vehicle registration records to give the make and model of the car likely involved?   We consider three sets of features and show that the recall increases significantly with better information.  

Considering the three sets of features, three metrics, and thirteen models, the best result we achieve is that if we have the full feature set, using the budgetary constraint that we can only afford to increase the number of ambulances sent to crashes with automated notifications by 5\%, the best model correctly recommends immediately dispatching ambulances to 12\% of those who need one.  

We trained the models on the Crash Report Sampling System (CRSS) 2016-2021, a publicly available dataset of 700,000 US crashes.  We have adapted new methods (new for this dataset in the literature) to handle missing data.  To promote discussion and future research, we have posted all of the code we used in our work.
