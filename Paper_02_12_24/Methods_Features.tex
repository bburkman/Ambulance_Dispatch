\subsection{Choosing Features}
\label{features}

The \verb|Accident|, \verb|Vehicle|, and \verb|Person| files of the CRSS dataset 2016-2021 have 170 unique features.  

First we narrowed the features to those that are relevant, of good quality, and knowable at the time of the automated notification (before any eyewitness reports).  Some features, like Vehicle Identification Number (VIN), have no predictive value.  Other features have missing data for more than 20\% of samples.  Some features, like drug and alcohol test results, are unknowable at the time of the automated notification.

Having data available for instantaneous analysis when the crash notification arrives is not free, and some features are more expensive than others.  A city thinking of implementing a recommendation system for immediate dispatch would need to decide how much to spend to have the data available, and whether the more expensive features increase the quality of the models enough to be worth the cost.  We categorized the features as ``Easy,'' ``Medium,'' and ``Hard,'' which can also be called ``Free,'' ``Expensive,'' and ``Problematic.''   The ``Easy'' features are those the dispatcher already has, like day of week, time of day, weather, and urban/rural.  The ``Medium'' features add details about the location (intersection, speed limit, interstate highway) and information the cell service company probably has about the primary user of the phone (age and sex).  To have the medium features instantaneously available would require coordination of many resources and be expensive to set up.  The ``Hard'' features are much more problematic, requiring more coordination of public and private records, and having that data readily available introduces privacy and data security concerns.  Hard features include whether the location is a work zone, the likely vehicle driven by the primary user of the cell phone, and, if there are multiple automated notifications from the same location, how many crash persons are likely to be involved.   

See Table \ref{Features_EMH} for a list of features from each of the \verb|Accident|, \verb|Vehicle|, and \verb|Person| CRSS files and some features we engineered, for the Easy features, additional features in the Medium features set, and more features in the Hard features set.  

\begin{table}[h]
\label{Features_EMH}
\caption{\normalsize\normalfont Easy, Medium, and Hard Features. Table accompanies \S \ref{features}}
\centering
\normalsize\normalfont
\begin{tabular}{
	p{.7in} 
	>{\hangindent=2em \raggedright}p{1.5in}
	>{\hangindent=2em \raggedright}p{1.5in}
	>{\hangindent=2em \raggedright}p{2.0in}
	}
	\toprule
	Source & Easy Features  & Additional Features \par \qquad in Medium Set  & Additional Features in Hard Set \cr\midrule
	% Easy 
		% Accident
	\verb|Accident| 	
	\par \qquad \verb|.csv|
	& 
	Day of week \par
	Hour \par
	Month \par
	Police jurisdiction \par
\hangindent=2em
	Primary Sampling Unit (Geographical region) \par
	Region \par
	Urbanicity \par
	Weather \par
	Year 
	& 
	% Medium
		% Accident	
	Interstate highway? \par
	Relation to road \par
	Within junction? \par
	Type of Intersection \par
	& 
	% Hard
		% Accident
	Lighting conditions \par

\hangindent=2em
	Number of persons in all vehicles in accident \par
\hangindent=2em
	Number of parked or working vehicles \par
\hangindent=2em
	Number of moving motor vehicles \par
\hangindent=2em
	Total number of motor vehicles \par
	Location within junction \par
	School bus? \par
	Work zone? 
	\cr
	\midrule
	
	\verb|Vehicle.csv|
	&	
	&
	% Medium
		% Vehicle	
	Roadway alignment \par
	Total lanes in roadway \par
	Roadway grade \par
	Speed limit \par
	Traffic control device \par
	Trafficway description \par
	&
	% Hard
		% Vehicle
	Body type \par
	Being used as a bus? \par
	Emergency use? \par
	Make \par
	Number of occupants \par
	Roadway surface condition 
	\cr
	\midrule
	
	\verb|Person.csv|
	&
	&
	% Medium
		% Person
	Age \par
	Sex
	&
	% Hard
		% Person
	Driver/Passenger 
	\cr
	\midrule
	
	Engineered  Features
	&
	&
	% Medium
		% Engineered
	Age $\times$ Sex  \par
	&
	% Hard
		% Engineered
	Vehicle age (Model year and Year) \par
	Age $\times$ School bus 
	\cr
	\bottomrule
		
\end{tabular}
\end{table}

\FloatBarrier

A similar study in 2011, reprised with new data in 2019, \citep{KONONEN2011112, 13793173920190801}, asked whether emergency dispatchers can use data from Advanced Automatic Collision Notification (AACN) systems like GM OnStar to predict whether a crash had a serious injury.    See Table \ref{Features_Compare_He} for the features used in those studies with a comparison to the features we used for a system based on notifications from cell phones.

\begin{table}[h]
\label{Features_Compare_He}
\caption{\normalsize\normalfont Comparison with Features in Similar Papers on Vehicle-based Advanced Automatic Collision Notification systems like GM OnStar.  \citep{KONONEN2011112,13793173920190801} Table accompanies \S \ref{features}}
\centering
\normalsize\normalfont
\begin{tabular}{p{2.2in} | >{\hangindent=2em}p{4in}}
	\toprule
	Feature & Vehicle-based v/s Cell Phone \cr
	\midrule
	Principal direction of force (PDOF) \par
	Change in Velocity ($\delta V$)  \par
	Multiple impacts
	&
	Whether cell phone collects this data depends on sensitivity of accelerometer and amount of memory reserved for this seldom-used function.
	\cr
	\midrule
	Presence of any older occupant ($\ge$ 55) \par
	Presence of any female occupant 
	&
	Are these features knowable by a vehicle-based AACN? The cell service provider may have data about the primary user; we put Age and Sex in our Medium features
	\cr
	\midrule
	Presence of any right-sided passenger \par
	Seat belt use 
	&
	Knowable by AACN from sensors in seats and seat belts, but not knowable to cell phone
	\cr
	\midrule
	
	Vehicle type 
	& 
	Knowable to the AACN installed in the vehicle, but only knowable with a cell phone notification if the system correlates the phone user's ID with vehicle registration records; we put information about the vehicle in our Hard features
	 \cr
	
	\bottomrule
\end{tabular}
\end{table}

    



We note here our simplifying assumption (\S \ref{simplifying_assumptions}) that an actual implementation would have complete and accurate data for each automated notification.  Also, we test three combinations of features but have not done more detailed testing to see which individual features or other groupings of features are most or least useful in predicting whether a crash person needs an ambulance.  

See 
\verb|Ambulance_Dispatch_01_Get_Data.ipynb|  
for a list of the excluded features.

See \verb|Ambulance_Dispatch_03_Bin_Data.ipynb|
for a list of the features we used for imputation of missing data in CRSS.


See
\verb|Ambulance_Dispatch_07_Build_Models.ipynb|.
for the complete list of the features used in the Easy, Medium, and Hard model building with their CRSS names, like \verb|VALIGN| for roadway alignment. See the CRSS Analytical Users's Manual for complete details on all of the features. \citep{CRSS} 


