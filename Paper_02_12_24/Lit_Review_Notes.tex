%%%%%%%%%%%%%%%%%%%%%%%%
\hrulefill

%%%
\cite{13793173920190801},
{\it Crash Telemetry-Based Injury Severity Prediction is Equivalent to or Out-Performs Field Protocols in Triage of Planar Vehicle Collisions.},
{\bf Prehospital \& Disaster Medicine},

``Vehicle telemetry technology''

``Advanced Automatic Crash Notification (AACN)''

``Injury Severity Score (ISS),'' and they looked at ISS 15+ injuries

Constructed an Injury Severity Prediction (ISP) model.

Used sensitivity (recall) and specificity (TN/(TN+FP) = TN/N = TNR)

To construct the logistic regression model, used NASS-CDS, National Automotive Sampling System Crashworthiness Data System.

Used OnStar data, hospital records to validate the model.

Variables:  
\begin{tabular}{l}
	Principal direction of force (PDOF) \cr
	Change in Velocity ($\delta V$) = deceleration \cr
	Multiple impacts\cr
	Presence of any older occupant ($\ge$ 55) \cr 
	Presence of any female occupant \cr
	Presence of any right-sided passenger \cr
	Seat belt use \cr
	Vehicle type \cr
\end{tabular}

\

\cite{KONONEN2011112},
{\it Identification and validation of a logistic regression model for predicting serious injuries associated with motor vehicle crashes},
{\bf Accident Analysis \& Prevention}

2010 version of paper above, \cite{13793173920190801}

\

%%%
\subsection{AACN, not OnStar}

\cite{15857782520170401},
{\it Evaluating the Potential Benefits of Advanced Automatic Crash Notification.},
{\bf Prehospital \& Disaster Medicine}, 2017

Did not use AACN data, but if we had that data we could predict whether the emergency responders would need specialized equipment to extricate the crash person from the vehicle.

\

\cite{BOSE20111048},
{\it Computational methodology to predict injury risk for motor vehicle crash victims: A framework for improving Advanced Automatic Crash Notification systems},
{\bf Transportation Research Part C: Emerging Technologies},

Predict injury in specific body regions based on specific characteristics of the crash, occupant, and vehicle.

\

\cite{JAGTENBERG201527},
{\it An efficient heuristic for real-time ambulance redeployment},
{\bf Operations Research for Health Care},

``Advanced automatic crash notification (AACN)''

Societal benefits were assessed by comparing
correctly triaged motor vehicle crash occupants using the
AACN algorithm against real-world decisions.

\

\cite{WEAVER20221057},
{\it Advanced Automatic Crash Notification Algorithm for Children},
{\bf Academic Pediatrics},

Pediatric version of her (Weaver's) articles for adults

Algorithm inputs included: delta-v, rollover quarter-turns, belt status, multiple impacts, airbag deployment, and age.

%%%
\subsubsection{Injury Prediction Models}

\cite{KONG2023106393},
{\it Machine learning-based injury severity prediction of level 1 trauma center enrolled patients associated with car-to-car crashes in Korea},
{\bf Computers in Biology and Medicine},

To handle class-imbalanced clinical datasets, we used four data-sampling
techniques (i.e., class-weighting, resampling, synthetic minority oversampling, and adaptive synthetic
sampling). Machine-learning analytics based on logistic regression, extreme gradient boosting (XGBoost), and a
multilayer perceptron model were used for the evaluations. Each model was executed using five-fold cross-validation
to solve overfitting

Age was a significant predictor

\

\cite{CANDEFJORD2021101124},
{\it On Scene Injury Severity Prediction (OSISP) machine learning algorithms for motor vehicle crash occupants in US},
{\bf Journal of Transport \& Health},

In this study, we evaluate if methods employing machine learning and variables {\bf that can be assessed on the scene of accident} has
potential to amend field triage


%%%
\subsection{General Dispatch Algorithms, not particularly for Crashes}

\cite{ZHANG201697}
Zhang 2016
TRpC
``Performance measure for reliable travel time of emergency vehicles''

33 sources

Travel time reliability.  Apparently a big topic.  

\

\cite{BERTSIMAS2019557},
{\it Robust and stochastic formulations for ambulance deployment and dispatch},
{\bf European Journal of Operational Research},

54 Sources

Set of locations for ambulances

\

\cite{ZHEN2024103449},
{\it Traffic emergency vehicle deployment and dispatch under uncertainty},
{\bf Transportation Research Part E: Logistics and Transportation Review},

%%%
\subsubsection{Privacy}

\cite{GEUENS2010385},
{\it Mandatory implementation for in-vehicle eCall: Privacy compatible?},
{\bf Computer Law \& Security Review},

An in-vehicle
eCall would reduce road fatalities with about 2500 deaths
every year and the severity of accidents with 15\% by reducing
the response time of the emergency services in urban areas
with 40\% and in rural areas with 50\%. This response time is
reduced because the emergency services are notified instantly
in case of an accident and are given the exact location of the accident.

There is however strong resistance against in-vehicle
eCall, especially if it would be made mandatory on all new
cars in the Community. This resistance stems from fear for
invasion in private life and risks regarding personal data.

Possible text:

A legal analysis of the possibility of mandating advanced automatic crash notification (AACN) systems in new cars sold in the EU reflected the concern that the vehicle speed data in the notification could be used to prosecute the driver and void the insurance. \cite{GEUENS2010385}

%%%%
\subsubsection{Coordinating Traffic Signals}

$\times $ \quad \cite{SU2023103955}
``EMVLight: A multi-agent reinforcement learning framework for an emergency vehicle decentralized routing and traffic signal control system''

Synching emergency vehicle routing with traffic signal control.

85 sources

\

\cite{QIN20121} Qin 2012
{\it Control strategies of traffic signal timing transition for emergency vehicle preemption},

32 sources

Traffic signal preemption

\



%%%
\subsubsection{Other}

\begin{comment}

Annals of Emergency Medicine
Annals of Operations Research
Computer Operations Research
Computers and Operational Research
European Journal of Operational Research
IEEE Transactions on Intelligent Transportation Systems
INFORMS Journal on Computation
INFORMS Journal on Computing
International Journal of Operational Research
Journal of Algorithms
Journal of Operations Research, 
Journal of Transportation Engineering
Management Science
Mathematical Programming
Operations Research
Operations Research for Health Care
Transportation Research Record
Transportation Science
TRpA
TRpB
TRpC
TRpE
\end{comment}

