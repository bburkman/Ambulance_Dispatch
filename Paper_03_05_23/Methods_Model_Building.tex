%%%
\subsection{Model Building}

To build our model we primarily used
 scikit-learn \citep{scikit-learn} and imbalanced-learn \citep{Imblearn}, and also
Keras/Tensorflow \citep{chollet2015keras} for the Focal Loss model.  

We will explain our methods with examples.  

The histogram below illustrates the kind of results we can realistically hope for in a good model.  The white boxes represent the 150,771 negative samples in the test set, and the black boxes the 26,621 positive samples.  The model evaluates each sample and gives it a probability $p \in (0,1)$ that the sample is in the positive class.  Ideally we want the model to  give most of the negative samples probabilities close to zero, and the positive samples probabilities close to one, but some will be hard to classify correctly.  The ROC curve below shows the median value of $p$ for the negative (0.338) and positive (0.658) classes.  If we choose $p=0.5$ as our threshold, we get the metrics shown.  

\noindent\begin{tabular}{@{}p{0.3\textwidth}@{\hspace{24pt}} p{0.3\textwidth} @{\hspace{24pt}} p{0.3\textwidth}}
  \vspace{0pt} %% Creator: Matplotlib, PGF backend
%%
%% To include the figure in your LaTeX document, write
%%   \input{<filename>.pgf}
%%
%% Make sure the required packages are loaded in your preamble
%%   \usepackage{pgf}
%%
%% Also ensure that all the required font packages are loaded; for instance,
%% the lmodern package is sometimes necessary when using math font.
%%   \usepackage{lmodern}
%%
%% Figures using additional raster images can only be included by \input if
%% they are in the same directory as the main LaTeX file. For loading figures
%% from other directories you can use the `import` package
%%   \usepackage{import}
%%
%% and then include the figures with
%%   \import{<path to file>}{<filename>.pgf}
%%
%% Matplotlib used the following preamble
%%   
%%   \usepackage{fontspec}
%%   \makeatletter\@ifpackageloaded{underscore}{}{\usepackage[strings]{underscore}}\makeatother
%%
\begingroup%
\makeatletter%
\begin{pgfpicture}%
\pgfpathrectangle{\pgfpointorigin}{\pgfqpoint{2.153750in}{1.654444in}}%
\pgfusepath{use as bounding box, clip}%
\begin{pgfscope}%
\pgfsetbuttcap%
\pgfsetmiterjoin%
\definecolor{currentfill}{rgb}{1.000000,1.000000,1.000000}%
\pgfsetfillcolor{currentfill}%
\pgfsetlinewidth{0.000000pt}%
\definecolor{currentstroke}{rgb}{1.000000,1.000000,1.000000}%
\pgfsetstrokecolor{currentstroke}%
\pgfsetdash{}{0pt}%
\pgfpathmoveto{\pgfqpoint{0.000000in}{0.000000in}}%
\pgfpathlineto{\pgfqpoint{2.153750in}{0.000000in}}%
\pgfpathlineto{\pgfqpoint{2.153750in}{1.654444in}}%
\pgfpathlineto{\pgfqpoint{0.000000in}{1.654444in}}%
\pgfpathlineto{\pgfqpoint{0.000000in}{0.000000in}}%
\pgfpathclose%
\pgfusepath{fill}%
\end{pgfscope}%
\begin{pgfscope}%
\pgfsetbuttcap%
\pgfsetmiterjoin%
\definecolor{currentfill}{rgb}{1.000000,1.000000,1.000000}%
\pgfsetfillcolor{currentfill}%
\pgfsetlinewidth{0.000000pt}%
\definecolor{currentstroke}{rgb}{0.000000,0.000000,0.000000}%
\pgfsetstrokecolor{currentstroke}%
\pgfsetstrokeopacity{0.000000}%
\pgfsetdash{}{0pt}%
\pgfpathmoveto{\pgfqpoint{0.465000in}{0.449444in}}%
\pgfpathlineto{\pgfqpoint{2.015000in}{0.449444in}}%
\pgfpathlineto{\pgfqpoint{2.015000in}{1.604444in}}%
\pgfpathlineto{\pgfqpoint{0.465000in}{1.604444in}}%
\pgfpathlineto{\pgfqpoint{0.465000in}{0.449444in}}%
\pgfpathclose%
\pgfusepath{fill}%
\end{pgfscope}%
\begin{pgfscope}%
\pgfpathrectangle{\pgfqpoint{0.465000in}{0.449444in}}{\pgfqpoint{1.550000in}{1.155000in}}%
\pgfusepath{clip}%
\pgfsetbuttcap%
\pgfsetmiterjoin%
\pgfsetlinewidth{1.003750pt}%
\definecolor{currentstroke}{rgb}{0.000000,0.000000,0.000000}%
\pgfsetstrokecolor{currentstroke}%
\pgfsetdash{}{0pt}%
\pgfpathmoveto{\pgfqpoint{0.455000in}{0.449444in}}%
\pgfpathlineto{\pgfqpoint{0.502805in}{0.449444in}}%
\pgfpathlineto{\pgfqpoint{0.502805in}{0.590081in}}%
\pgfpathlineto{\pgfqpoint{0.455000in}{0.590081in}}%
\pgfusepath{stroke}%
\end{pgfscope}%
\begin{pgfscope}%
\pgfpathrectangle{\pgfqpoint{0.465000in}{0.449444in}}{\pgfqpoint{1.550000in}{1.155000in}}%
\pgfusepath{clip}%
\pgfsetbuttcap%
\pgfsetmiterjoin%
\pgfsetlinewidth{1.003750pt}%
\definecolor{currentstroke}{rgb}{0.000000,0.000000,0.000000}%
\pgfsetstrokecolor{currentstroke}%
\pgfsetdash{}{0pt}%
\pgfpathmoveto{\pgfqpoint{0.593537in}{0.449444in}}%
\pgfpathlineto{\pgfqpoint{0.654025in}{0.449444in}}%
\pgfpathlineto{\pgfqpoint{0.654025in}{1.204573in}}%
\pgfpathlineto{\pgfqpoint{0.593537in}{1.204573in}}%
\pgfpathlineto{\pgfqpoint{0.593537in}{0.449444in}}%
\pgfpathclose%
\pgfusepath{stroke}%
\end{pgfscope}%
\begin{pgfscope}%
\pgfpathrectangle{\pgfqpoint{0.465000in}{0.449444in}}{\pgfqpoint{1.550000in}{1.155000in}}%
\pgfusepath{clip}%
\pgfsetbuttcap%
\pgfsetmiterjoin%
\pgfsetlinewidth{1.003750pt}%
\definecolor{currentstroke}{rgb}{0.000000,0.000000,0.000000}%
\pgfsetstrokecolor{currentstroke}%
\pgfsetdash{}{0pt}%
\pgfpathmoveto{\pgfqpoint{0.744756in}{0.449444in}}%
\pgfpathlineto{\pgfqpoint{0.805244in}{0.449444in}}%
\pgfpathlineto{\pgfqpoint{0.805244in}{1.549444in}}%
\pgfpathlineto{\pgfqpoint{0.744756in}{1.549444in}}%
\pgfpathlineto{\pgfqpoint{0.744756in}{0.449444in}}%
\pgfpathclose%
\pgfusepath{stroke}%
\end{pgfscope}%
\begin{pgfscope}%
\pgfpathrectangle{\pgfqpoint{0.465000in}{0.449444in}}{\pgfqpoint{1.550000in}{1.155000in}}%
\pgfusepath{clip}%
\pgfsetbuttcap%
\pgfsetmiterjoin%
\pgfsetlinewidth{1.003750pt}%
\definecolor{currentstroke}{rgb}{0.000000,0.000000,0.000000}%
\pgfsetstrokecolor{currentstroke}%
\pgfsetdash{}{0pt}%
\pgfpathmoveto{\pgfqpoint{0.895976in}{0.449444in}}%
\pgfpathlineto{\pgfqpoint{0.956464in}{0.449444in}}%
\pgfpathlineto{\pgfqpoint{0.956464in}{1.447255in}}%
\pgfpathlineto{\pgfqpoint{0.895976in}{1.447255in}}%
\pgfpathlineto{\pgfqpoint{0.895976in}{0.449444in}}%
\pgfpathclose%
\pgfusepath{stroke}%
\end{pgfscope}%
\begin{pgfscope}%
\pgfpathrectangle{\pgfqpoint{0.465000in}{0.449444in}}{\pgfqpoint{1.550000in}{1.155000in}}%
\pgfusepath{clip}%
\pgfsetbuttcap%
\pgfsetmiterjoin%
\pgfsetlinewidth{1.003750pt}%
\definecolor{currentstroke}{rgb}{0.000000,0.000000,0.000000}%
\pgfsetstrokecolor{currentstroke}%
\pgfsetdash{}{0pt}%
\pgfpathmoveto{\pgfqpoint{1.047195in}{0.449444in}}%
\pgfpathlineto{\pgfqpoint{1.107683in}{0.449444in}}%
\pgfpathlineto{\pgfqpoint{1.107683in}{1.160058in}}%
\pgfpathlineto{\pgfqpoint{1.047195in}{1.160058in}}%
\pgfpathlineto{\pgfqpoint{1.047195in}{0.449444in}}%
\pgfpathclose%
\pgfusepath{stroke}%
\end{pgfscope}%
\begin{pgfscope}%
\pgfpathrectangle{\pgfqpoint{0.465000in}{0.449444in}}{\pgfqpoint{1.550000in}{1.155000in}}%
\pgfusepath{clip}%
\pgfsetbuttcap%
\pgfsetmiterjoin%
\pgfsetlinewidth{1.003750pt}%
\definecolor{currentstroke}{rgb}{0.000000,0.000000,0.000000}%
\pgfsetstrokecolor{currentstroke}%
\pgfsetdash{}{0pt}%
\pgfpathmoveto{\pgfqpoint{1.198415in}{0.449444in}}%
\pgfpathlineto{\pgfqpoint{1.258903in}{0.449444in}}%
\pgfpathlineto{\pgfqpoint{1.258903in}{0.885778in}}%
\pgfpathlineto{\pgfqpoint{1.198415in}{0.885778in}}%
\pgfpathlineto{\pgfqpoint{1.198415in}{0.449444in}}%
\pgfpathclose%
\pgfusepath{stroke}%
\end{pgfscope}%
\begin{pgfscope}%
\pgfpathrectangle{\pgfqpoint{0.465000in}{0.449444in}}{\pgfqpoint{1.550000in}{1.155000in}}%
\pgfusepath{clip}%
\pgfsetbuttcap%
\pgfsetmiterjoin%
\pgfsetlinewidth{1.003750pt}%
\definecolor{currentstroke}{rgb}{0.000000,0.000000,0.000000}%
\pgfsetstrokecolor{currentstroke}%
\pgfsetdash{}{0pt}%
\pgfpathmoveto{\pgfqpoint{1.349634in}{0.449444in}}%
\pgfpathlineto{\pgfqpoint{1.410122in}{0.449444in}}%
\pgfpathlineto{\pgfqpoint{1.410122in}{0.731988in}}%
\pgfpathlineto{\pgfqpoint{1.349634in}{0.731988in}}%
\pgfpathlineto{\pgfqpoint{1.349634in}{0.449444in}}%
\pgfpathclose%
\pgfusepath{stroke}%
\end{pgfscope}%
\begin{pgfscope}%
\pgfpathrectangle{\pgfqpoint{0.465000in}{0.449444in}}{\pgfqpoint{1.550000in}{1.155000in}}%
\pgfusepath{clip}%
\pgfsetbuttcap%
\pgfsetmiterjoin%
\pgfsetlinewidth{1.003750pt}%
\definecolor{currentstroke}{rgb}{0.000000,0.000000,0.000000}%
\pgfsetstrokecolor{currentstroke}%
\pgfsetdash{}{0pt}%
\pgfpathmoveto{\pgfqpoint{1.500854in}{0.449444in}}%
\pgfpathlineto{\pgfqpoint{1.561342in}{0.449444in}}%
\pgfpathlineto{\pgfqpoint{1.561342in}{0.577885in}}%
\pgfpathlineto{\pgfqpoint{1.500854in}{0.577885in}}%
\pgfpathlineto{\pgfqpoint{1.500854in}{0.449444in}}%
\pgfpathclose%
\pgfusepath{stroke}%
\end{pgfscope}%
\begin{pgfscope}%
\pgfpathrectangle{\pgfqpoint{0.465000in}{0.449444in}}{\pgfqpoint{1.550000in}{1.155000in}}%
\pgfusepath{clip}%
\pgfsetbuttcap%
\pgfsetmiterjoin%
\pgfsetlinewidth{1.003750pt}%
\definecolor{currentstroke}{rgb}{0.000000,0.000000,0.000000}%
\pgfsetstrokecolor{currentstroke}%
\pgfsetdash{}{0pt}%
\pgfpathmoveto{\pgfqpoint{1.652073in}{0.449444in}}%
\pgfpathlineto{\pgfqpoint{1.712561in}{0.449444in}}%
\pgfpathlineto{\pgfqpoint{1.712561in}{0.513416in}}%
\pgfpathlineto{\pgfqpoint{1.652073in}{0.513416in}}%
\pgfpathlineto{\pgfqpoint{1.652073in}{0.449444in}}%
\pgfpathclose%
\pgfusepath{stroke}%
\end{pgfscope}%
\begin{pgfscope}%
\pgfpathrectangle{\pgfqpoint{0.465000in}{0.449444in}}{\pgfqpoint{1.550000in}{1.155000in}}%
\pgfusepath{clip}%
\pgfsetbuttcap%
\pgfsetmiterjoin%
\pgfsetlinewidth{1.003750pt}%
\definecolor{currentstroke}{rgb}{0.000000,0.000000,0.000000}%
\pgfsetstrokecolor{currentstroke}%
\pgfsetdash{}{0pt}%
\pgfpathmoveto{\pgfqpoint{1.803293in}{0.449444in}}%
\pgfpathlineto{\pgfqpoint{1.863781in}{0.449444in}}%
\pgfpathlineto{\pgfqpoint{1.863781in}{0.467616in}}%
\pgfpathlineto{\pgfqpoint{1.803293in}{0.467616in}}%
\pgfpathlineto{\pgfqpoint{1.803293in}{0.449444in}}%
\pgfpathclose%
\pgfusepath{stroke}%
\end{pgfscope}%
\begin{pgfscope}%
\pgfpathrectangle{\pgfqpoint{0.465000in}{0.449444in}}{\pgfqpoint{1.550000in}{1.155000in}}%
\pgfusepath{clip}%
\pgfsetbuttcap%
\pgfsetmiterjoin%
\definecolor{currentfill}{rgb}{0.000000,0.000000,0.000000}%
\pgfsetfillcolor{currentfill}%
\pgfsetlinewidth{0.000000pt}%
\definecolor{currentstroke}{rgb}{0.000000,0.000000,0.000000}%
\pgfsetstrokecolor{currentstroke}%
\pgfsetstrokeopacity{0.000000}%
\pgfsetdash{}{0pt}%
\pgfpathmoveto{\pgfqpoint{0.502805in}{0.449444in}}%
\pgfpathlineto{\pgfqpoint{0.563293in}{0.449444in}}%
\pgfpathlineto{\pgfqpoint{0.563293in}{0.456162in}}%
\pgfpathlineto{\pgfqpoint{0.502805in}{0.456162in}}%
\pgfpathlineto{\pgfqpoint{0.502805in}{0.449444in}}%
\pgfpathclose%
\pgfusepath{fill}%
\end{pgfscope}%
\begin{pgfscope}%
\pgfpathrectangle{\pgfqpoint{0.465000in}{0.449444in}}{\pgfqpoint{1.550000in}{1.155000in}}%
\pgfusepath{clip}%
\pgfsetbuttcap%
\pgfsetmiterjoin%
\definecolor{currentfill}{rgb}{0.000000,0.000000,0.000000}%
\pgfsetfillcolor{currentfill}%
\pgfsetlinewidth{0.000000pt}%
\definecolor{currentstroke}{rgb}{0.000000,0.000000,0.000000}%
\pgfsetstrokecolor{currentstroke}%
\pgfsetstrokeopacity{0.000000}%
\pgfsetdash{}{0pt}%
\pgfpathmoveto{\pgfqpoint{0.654025in}{0.449444in}}%
\pgfpathlineto{\pgfqpoint{0.714512in}{0.449444in}}%
\pgfpathlineto{\pgfqpoint{0.714512in}{0.462803in}}%
\pgfpathlineto{\pgfqpoint{0.654025in}{0.462803in}}%
\pgfpathlineto{\pgfqpoint{0.654025in}{0.449444in}}%
\pgfpathclose%
\pgfusepath{fill}%
\end{pgfscope}%
\begin{pgfscope}%
\pgfpathrectangle{\pgfqpoint{0.465000in}{0.449444in}}{\pgfqpoint{1.550000in}{1.155000in}}%
\pgfusepath{clip}%
\pgfsetbuttcap%
\pgfsetmiterjoin%
\definecolor{currentfill}{rgb}{0.000000,0.000000,0.000000}%
\pgfsetfillcolor{currentfill}%
\pgfsetlinewidth{0.000000pt}%
\definecolor{currentstroke}{rgb}{0.000000,0.000000,0.000000}%
\pgfsetstrokecolor{currentstroke}%
\pgfsetstrokeopacity{0.000000}%
\pgfsetdash{}{0pt}%
\pgfpathmoveto{\pgfqpoint{0.805244in}{0.449444in}}%
\pgfpathlineto{\pgfqpoint{0.865732in}{0.449444in}}%
\pgfpathlineto{\pgfqpoint{0.865732in}{0.475374in}}%
\pgfpathlineto{\pgfqpoint{0.805244in}{0.475374in}}%
\pgfpathlineto{\pgfqpoint{0.805244in}{0.449444in}}%
\pgfpathclose%
\pgfusepath{fill}%
\end{pgfscope}%
\begin{pgfscope}%
\pgfpathrectangle{\pgfqpoint{0.465000in}{0.449444in}}{\pgfqpoint{1.550000in}{1.155000in}}%
\pgfusepath{clip}%
\pgfsetbuttcap%
\pgfsetmiterjoin%
\definecolor{currentfill}{rgb}{0.000000,0.000000,0.000000}%
\pgfsetfillcolor{currentfill}%
\pgfsetlinewidth{0.000000pt}%
\definecolor{currentstroke}{rgb}{0.000000,0.000000,0.000000}%
\pgfsetstrokecolor{currentstroke}%
\pgfsetstrokeopacity{0.000000}%
\pgfsetdash{}{0pt}%
\pgfpathmoveto{\pgfqpoint{0.956464in}{0.449444in}}%
\pgfpathlineto{\pgfqpoint{1.016951in}{0.449444in}}%
\pgfpathlineto{\pgfqpoint{1.016951in}{0.496652in}}%
\pgfpathlineto{\pgfqpoint{0.956464in}{0.496652in}}%
\pgfpathlineto{\pgfqpoint{0.956464in}{0.449444in}}%
\pgfpathclose%
\pgfusepath{fill}%
\end{pgfscope}%
\begin{pgfscope}%
\pgfpathrectangle{\pgfqpoint{0.465000in}{0.449444in}}{\pgfqpoint{1.550000in}{1.155000in}}%
\pgfusepath{clip}%
\pgfsetbuttcap%
\pgfsetmiterjoin%
\definecolor{currentfill}{rgb}{0.000000,0.000000,0.000000}%
\pgfsetfillcolor{currentfill}%
\pgfsetlinewidth{0.000000pt}%
\definecolor{currentstroke}{rgb}{0.000000,0.000000,0.000000}%
\pgfsetstrokecolor{currentstroke}%
\pgfsetstrokeopacity{0.000000}%
\pgfsetdash{}{0pt}%
\pgfpathmoveto{\pgfqpoint{1.107683in}{0.449444in}}%
\pgfpathlineto{\pgfqpoint{1.168171in}{0.449444in}}%
\pgfpathlineto{\pgfqpoint{1.168171in}{0.530884in}}%
\pgfpathlineto{\pgfqpoint{1.107683in}{0.530884in}}%
\pgfpathlineto{\pgfqpoint{1.107683in}{0.449444in}}%
\pgfpathclose%
\pgfusepath{fill}%
\end{pgfscope}%
\begin{pgfscope}%
\pgfpathrectangle{\pgfqpoint{0.465000in}{0.449444in}}{\pgfqpoint{1.550000in}{1.155000in}}%
\pgfusepath{clip}%
\pgfsetbuttcap%
\pgfsetmiterjoin%
\definecolor{currentfill}{rgb}{0.000000,0.000000,0.000000}%
\pgfsetfillcolor{currentfill}%
\pgfsetlinewidth{0.000000pt}%
\definecolor{currentstroke}{rgb}{0.000000,0.000000,0.000000}%
\pgfsetstrokecolor{currentstroke}%
\pgfsetstrokeopacity{0.000000}%
\pgfsetdash{}{0pt}%
\pgfpathmoveto{\pgfqpoint{1.258903in}{0.449444in}}%
\pgfpathlineto{\pgfqpoint{1.319391in}{0.449444in}}%
\pgfpathlineto{\pgfqpoint{1.319391in}{0.576975in}}%
\pgfpathlineto{\pgfqpoint{1.258903in}{0.576975in}}%
\pgfpathlineto{\pgfqpoint{1.258903in}{0.449444in}}%
\pgfpathclose%
\pgfusepath{fill}%
\end{pgfscope}%
\begin{pgfscope}%
\pgfpathrectangle{\pgfqpoint{0.465000in}{0.449444in}}{\pgfqpoint{1.550000in}{1.155000in}}%
\pgfusepath{clip}%
\pgfsetbuttcap%
\pgfsetmiterjoin%
\definecolor{currentfill}{rgb}{0.000000,0.000000,0.000000}%
\pgfsetfillcolor{currentfill}%
\pgfsetlinewidth{0.000000pt}%
\definecolor{currentstroke}{rgb}{0.000000,0.000000,0.000000}%
\pgfsetstrokecolor{currentstroke}%
\pgfsetstrokeopacity{0.000000}%
\pgfsetdash{}{0pt}%
\pgfpathmoveto{\pgfqpoint{1.410122in}{0.449444in}}%
\pgfpathlineto{\pgfqpoint{1.470610in}{0.449444in}}%
\pgfpathlineto{\pgfqpoint{1.470610in}{0.624175in}}%
\pgfpathlineto{\pgfqpoint{1.410122in}{0.624175in}}%
\pgfpathlineto{\pgfqpoint{1.410122in}{0.449444in}}%
\pgfpathclose%
\pgfusepath{fill}%
\end{pgfscope}%
\begin{pgfscope}%
\pgfpathrectangle{\pgfqpoint{0.465000in}{0.449444in}}{\pgfqpoint{1.550000in}{1.155000in}}%
\pgfusepath{clip}%
\pgfsetbuttcap%
\pgfsetmiterjoin%
\definecolor{currentfill}{rgb}{0.000000,0.000000,0.000000}%
\pgfsetfillcolor{currentfill}%
\pgfsetlinewidth{0.000000pt}%
\definecolor{currentstroke}{rgb}{0.000000,0.000000,0.000000}%
\pgfsetstrokecolor{currentstroke}%
\pgfsetstrokeopacity{0.000000}%
\pgfsetdash{}{0pt}%
\pgfpathmoveto{\pgfqpoint{1.561342in}{0.449444in}}%
\pgfpathlineto{\pgfqpoint{1.621830in}{0.449444in}}%
\pgfpathlineto{\pgfqpoint{1.621830in}{0.643181in}}%
\pgfpathlineto{\pgfqpoint{1.561342in}{0.643181in}}%
\pgfpathlineto{\pgfqpoint{1.561342in}{0.449444in}}%
\pgfpathclose%
\pgfusepath{fill}%
\end{pgfscope}%
\begin{pgfscope}%
\pgfpathrectangle{\pgfqpoint{0.465000in}{0.449444in}}{\pgfqpoint{1.550000in}{1.155000in}}%
\pgfusepath{clip}%
\pgfsetbuttcap%
\pgfsetmiterjoin%
\definecolor{currentfill}{rgb}{0.000000,0.000000,0.000000}%
\pgfsetfillcolor{currentfill}%
\pgfsetlinewidth{0.000000pt}%
\definecolor{currentstroke}{rgb}{0.000000,0.000000,0.000000}%
\pgfsetstrokecolor{currentstroke}%
\pgfsetstrokeopacity{0.000000}%
\pgfsetdash{}{0pt}%
\pgfpathmoveto{\pgfqpoint{1.712561in}{0.449444in}}%
\pgfpathlineto{\pgfqpoint{1.773049in}{0.449444in}}%
\pgfpathlineto{\pgfqpoint{1.773049in}{0.579591in}}%
\pgfpathlineto{\pgfqpoint{1.712561in}{0.579591in}}%
\pgfpathlineto{\pgfqpoint{1.712561in}{0.449444in}}%
\pgfpathclose%
\pgfusepath{fill}%
\end{pgfscope}%
\begin{pgfscope}%
\pgfpathrectangle{\pgfqpoint{0.465000in}{0.449444in}}{\pgfqpoint{1.550000in}{1.155000in}}%
\pgfusepath{clip}%
\pgfsetbuttcap%
\pgfsetmiterjoin%
\definecolor{currentfill}{rgb}{0.000000,0.000000,0.000000}%
\pgfsetfillcolor{currentfill}%
\pgfsetlinewidth{0.000000pt}%
\definecolor{currentstroke}{rgb}{0.000000,0.000000,0.000000}%
\pgfsetstrokecolor{currentstroke}%
\pgfsetstrokeopacity{0.000000}%
\pgfsetdash{}{0pt}%
\pgfpathmoveto{\pgfqpoint{1.863781in}{0.449444in}}%
\pgfpathlineto{\pgfqpoint{1.924269in}{0.449444in}}%
\pgfpathlineto{\pgfqpoint{1.924269in}{0.474640in}}%
\pgfpathlineto{\pgfqpoint{1.863781in}{0.474640in}}%
\pgfpathlineto{\pgfqpoint{1.863781in}{0.449444in}}%
\pgfpathclose%
\pgfusepath{fill}%
\end{pgfscope}%
\begin{pgfscope}%
\pgfsetbuttcap%
\pgfsetroundjoin%
\definecolor{currentfill}{rgb}{0.000000,0.000000,0.000000}%
\pgfsetfillcolor{currentfill}%
\pgfsetlinewidth{0.803000pt}%
\definecolor{currentstroke}{rgb}{0.000000,0.000000,0.000000}%
\pgfsetstrokecolor{currentstroke}%
\pgfsetdash{}{0pt}%
\pgfsys@defobject{currentmarker}{\pgfqpoint{0.000000in}{-0.048611in}}{\pgfqpoint{0.000000in}{0.000000in}}{%
\pgfpathmoveto{\pgfqpoint{0.000000in}{0.000000in}}%
\pgfpathlineto{\pgfqpoint{0.000000in}{-0.048611in}}%
\pgfusepath{stroke,fill}%
}%
\begin{pgfscope}%
\pgfsys@transformshift{0.502805in}{0.449444in}%
\pgfsys@useobject{currentmarker}{}%
\end{pgfscope}%
\end{pgfscope}%
\begin{pgfscope}%
\definecolor{textcolor}{rgb}{0.000000,0.000000,0.000000}%
\pgfsetstrokecolor{textcolor}%
\pgfsetfillcolor{textcolor}%
\pgftext[x=0.502805in,y=0.352222in,,top]{\color{textcolor}\rmfamily\fontsize{10.000000}{12.000000}\selectfont 0.0}%
\end{pgfscope}%
\begin{pgfscope}%
\pgfsetbuttcap%
\pgfsetroundjoin%
\definecolor{currentfill}{rgb}{0.000000,0.000000,0.000000}%
\pgfsetfillcolor{currentfill}%
\pgfsetlinewidth{0.803000pt}%
\definecolor{currentstroke}{rgb}{0.000000,0.000000,0.000000}%
\pgfsetstrokecolor{currentstroke}%
\pgfsetdash{}{0pt}%
\pgfsys@defobject{currentmarker}{\pgfqpoint{0.000000in}{-0.048611in}}{\pgfqpoint{0.000000in}{0.000000in}}{%
\pgfpathmoveto{\pgfqpoint{0.000000in}{0.000000in}}%
\pgfpathlineto{\pgfqpoint{0.000000in}{-0.048611in}}%
\pgfusepath{stroke,fill}%
}%
\begin{pgfscope}%
\pgfsys@transformshift{0.880854in}{0.449444in}%
\pgfsys@useobject{currentmarker}{}%
\end{pgfscope}%
\end{pgfscope}%
\begin{pgfscope}%
\definecolor{textcolor}{rgb}{0.000000,0.000000,0.000000}%
\pgfsetstrokecolor{textcolor}%
\pgfsetfillcolor{textcolor}%
\pgftext[x=0.880854in,y=0.352222in,,top]{\color{textcolor}\rmfamily\fontsize{10.000000}{12.000000}\selectfont 0.25}%
\end{pgfscope}%
\begin{pgfscope}%
\pgfsetbuttcap%
\pgfsetroundjoin%
\definecolor{currentfill}{rgb}{0.000000,0.000000,0.000000}%
\pgfsetfillcolor{currentfill}%
\pgfsetlinewidth{0.803000pt}%
\definecolor{currentstroke}{rgb}{0.000000,0.000000,0.000000}%
\pgfsetstrokecolor{currentstroke}%
\pgfsetdash{}{0pt}%
\pgfsys@defobject{currentmarker}{\pgfqpoint{0.000000in}{-0.048611in}}{\pgfqpoint{0.000000in}{0.000000in}}{%
\pgfpathmoveto{\pgfqpoint{0.000000in}{0.000000in}}%
\pgfpathlineto{\pgfqpoint{0.000000in}{-0.048611in}}%
\pgfusepath{stroke,fill}%
}%
\begin{pgfscope}%
\pgfsys@transformshift{1.258903in}{0.449444in}%
\pgfsys@useobject{currentmarker}{}%
\end{pgfscope}%
\end{pgfscope}%
\begin{pgfscope}%
\definecolor{textcolor}{rgb}{0.000000,0.000000,0.000000}%
\pgfsetstrokecolor{textcolor}%
\pgfsetfillcolor{textcolor}%
\pgftext[x=1.258903in,y=0.352222in,,top]{\color{textcolor}\rmfamily\fontsize{10.000000}{12.000000}\selectfont 0.5}%
\end{pgfscope}%
\begin{pgfscope}%
\pgfsetbuttcap%
\pgfsetroundjoin%
\definecolor{currentfill}{rgb}{0.000000,0.000000,0.000000}%
\pgfsetfillcolor{currentfill}%
\pgfsetlinewidth{0.803000pt}%
\definecolor{currentstroke}{rgb}{0.000000,0.000000,0.000000}%
\pgfsetstrokecolor{currentstroke}%
\pgfsetdash{}{0pt}%
\pgfsys@defobject{currentmarker}{\pgfqpoint{0.000000in}{-0.048611in}}{\pgfqpoint{0.000000in}{0.000000in}}{%
\pgfpathmoveto{\pgfqpoint{0.000000in}{0.000000in}}%
\pgfpathlineto{\pgfqpoint{0.000000in}{-0.048611in}}%
\pgfusepath{stroke,fill}%
}%
\begin{pgfscope}%
\pgfsys@transformshift{1.636951in}{0.449444in}%
\pgfsys@useobject{currentmarker}{}%
\end{pgfscope}%
\end{pgfscope}%
\begin{pgfscope}%
\definecolor{textcolor}{rgb}{0.000000,0.000000,0.000000}%
\pgfsetstrokecolor{textcolor}%
\pgfsetfillcolor{textcolor}%
\pgftext[x=1.636951in,y=0.352222in,,top]{\color{textcolor}\rmfamily\fontsize{10.000000}{12.000000}\selectfont 0.75}%
\end{pgfscope}%
\begin{pgfscope}%
\pgfsetbuttcap%
\pgfsetroundjoin%
\definecolor{currentfill}{rgb}{0.000000,0.000000,0.000000}%
\pgfsetfillcolor{currentfill}%
\pgfsetlinewidth{0.803000pt}%
\definecolor{currentstroke}{rgb}{0.000000,0.000000,0.000000}%
\pgfsetstrokecolor{currentstroke}%
\pgfsetdash{}{0pt}%
\pgfsys@defobject{currentmarker}{\pgfqpoint{0.000000in}{-0.048611in}}{\pgfqpoint{0.000000in}{0.000000in}}{%
\pgfpathmoveto{\pgfqpoint{0.000000in}{0.000000in}}%
\pgfpathlineto{\pgfqpoint{0.000000in}{-0.048611in}}%
\pgfusepath{stroke,fill}%
}%
\begin{pgfscope}%
\pgfsys@transformshift{2.015000in}{0.449444in}%
\pgfsys@useobject{currentmarker}{}%
\end{pgfscope}%
\end{pgfscope}%
\begin{pgfscope}%
\definecolor{textcolor}{rgb}{0.000000,0.000000,0.000000}%
\pgfsetstrokecolor{textcolor}%
\pgfsetfillcolor{textcolor}%
\pgftext[x=2.015000in,y=0.352222in,,top]{\color{textcolor}\rmfamily\fontsize{10.000000}{12.000000}\selectfont 1.0}%
\end{pgfscope}%
\begin{pgfscope}%
\definecolor{textcolor}{rgb}{0.000000,0.000000,0.000000}%
\pgfsetstrokecolor{textcolor}%
\pgfsetfillcolor{textcolor}%
\pgftext[x=1.240000in,y=0.173333in,,top]{\color{textcolor}\rmfamily\fontsize{10.000000}{12.000000}\selectfont \(\displaystyle p\)}%
\end{pgfscope}%
\begin{pgfscope}%
\pgfsetbuttcap%
\pgfsetroundjoin%
\definecolor{currentfill}{rgb}{0.000000,0.000000,0.000000}%
\pgfsetfillcolor{currentfill}%
\pgfsetlinewidth{0.803000pt}%
\definecolor{currentstroke}{rgb}{0.000000,0.000000,0.000000}%
\pgfsetstrokecolor{currentstroke}%
\pgfsetdash{}{0pt}%
\pgfsys@defobject{currentmarker}{\pgfqpoint{-0.048611in}{0.000000in}}{\pgfqpoint{-0.000000in}{0.000000in}}{%
\pgfpathmoveto{\pgfqpoint{-0.000000in}{0.000000in}}%
\pgfpathlineto{\pgfqpoint{-0.048611in}{0.000000in}}%
\pgfusepath{stroke,fill}%
}%
\begin{pgfscope}%
\pgfsys@transformshift{0.465000in}{0.449444in}%
\pgfsys@useobject{currentmarker}{}%
\end{pgfscope}%
\end{pgfscope}%
\begin{pgfscope}%
\definecolor{textcolor}{rgb}{0.000000,0.000000,0.000000}%
\pgfsetstrokecolor{textcolor}%
\pgfsetfillcolor{textcolor}%
\pgftext[x=0.298333in, y=0.401250in, left, base]{\color{textcolor}\rmfamily\fontsize{10.000000}{12.000000}\selectfont \(\displaystyle {0}\)}%
\end{pgfscope}%
\begin{pgfscope}%
\pgfsetbuttcap%
\pgfsetroundjoin%
\definecolor{currentfill}{rgb}{0.000000,0.000000,0.000000}%
\pgfsetfillcolor{currentfill}%
\pgfsetlinewidth{0.803000pt}%
\definecolor{currentstroke}{rgb}{0.000000,0.000000,0.000000}%
\pgfsetstrokecolor{currentstroke}%
\pgfsetdash{}{0pt}%
\pgfsys@defobject{currentmarker}{\pgfqpoint{-0.048611in}{0.000000in}}{\pgfqpoint{-0.000000in}{0.000000in}}{%
\pgfpathmoveto{\pgfqpoint{-0.000000in}{0.000000in}}%
\pgfpathlineto{\pgfqpoint{-0.048611in}{0.000000in}}%
\pgfusepath{stroke,fill}%
}%
\begin{pgfscope}%
\pgfsys@transformshift{0.465000in}{0.995409in}%
\pgfsys@useobject{currentmarker}{}%
\end{pgfscope}%
\end{pgfscope}%
\begin{pgfscope}%
\definecolor{textcolor}{rgb}{0.000000,0.000000,0.000000}%
\pgfsetstrokecolor{textcolor}%
\pgfsetfillcolor{textcolor}%
\pgftext[x=0.228889in, y=0.947214in, left, base]{\color{textcolor}\rmfamily\fontsize{10.000000}{12.000000}\selectfont \(\displaystyle {10}\)}%
\end{pgfscope}%
\begin{pgfscope}%
\pgfsetbuttcap%
\pgfsetroundjoin%
\definecolor{currentfill}{rgb}{0.000000,0.000000,0.000000}%
\pgfsetfillcolor{currentfill}%
\pgfsetlinewidth{0.803000pt}%
\definecolor{currentstroke}{rgb}{0.000000,0.000000,0.000000}%
\pgfsetstrokecolor{currentstroke}%
\pgfsetdash{}{0pt}%
\pgfsys@defobject{currentmarker}{\pgfqpoint{-0.048611in}{0.000000in}}{\pgfqpoint{-0.000000in}{0.000000in}}{%
\pgfpathmoveto{\pgfqpoint{-0.000000in}{0.000000in}}%
\pgfpathlineto{\pgfqpoint{-0.048611in}{0.000000in}}%
\pgfusepath{stroke,fill}%
}%
\begin{pgfscope}%
\pgfsys@transformshift{0.465000in}{1.541374in}%
\pgfsys@useobject{currentmarker}{}%
\end{pgfscope}%
\end{pgfscope}%
\begin{pgfscope}%
\definecolor{textcolor}{rgb}{0.000000,0.000000,0.000000}%
\pgfsetstrokecolor{textcolor}%
\pgfsetfillcolor{textcolor}%
\pgftext[x=0.228889in, y=1.493179in, left, base]{\color{textcolor}\rmfamily\fontsize{10.000000}{12.000000}\selectfont \(\displaystyle {20}\)}%
\end{pgfscope}%
\begin{pgfscope}%
\definecolor{textcolor}{rgb}{0.000000,0.000000,0.000000}%
\pgfsetstrokecolor{textcolor}%
\pgfsetfillcolor{textcolor}%
\pgftext[x=0.173333in,y=1.026944in,,bottom,rotate=90.000000]{\color{textcolor}\rmfamily\fontsize{10.000000}{12.000000}\selectfont Percent of Data Set}%
\end{pgfscope}%
\begin{pgfscope}%
\pgfsetrectcap%
\pgfsetmiterjoin%
\pgfsetlinewidth{0.803000pt}%
\definecolor{currentstroke}{rgb}{0.000000,0.000000,0.000000}%
\pgfsetstrokecolor{currentstroke}%
\pgfsetdash{}{0pt}%
\pgfpathmoveto{\pgfqpoint{0.465000in}{0.449444in}}%
\pgfpathlineto{\pgfqpoint{0.465000in}{1.604444in}}%
\pgfusepath{stroke}%
\end{pgfscope}%
\begin{pgfscope}%
\pgfsetrectcap%
\pgfsetmiterjoin%
\pgfsetlinewidth{0.803000pt}%
\definecolor{currentstroke}{rgb}{0.000000,0.000000,0.000000}%
\pgfsetstrokecolor{currentstroke}%
\pgfsetdash{}{0pt}%
\pgfpathmoveto{\pgfqpoint{2.015000in}{0.449444in}}%
\pgfpathlineto{\pgfqpoint{2.015000in}{1.604444in}}%
\pgfusepath{stroke}%
\end{pgfscope}%
\begin{pgfscope}%
\pgfsetrectcap%
\pgfsetmiterjoin%
\pgfsetlinewidth{0.803000pt}%
\definecolor{currentstroke}{rgb}{0.000000,0.000000,0.000000}%
\pgfsetstrokecolor{currentstroke}%
\pgfsetdash{}{0pt}%
\pgfpathmoveto{\pgfqpoint{0.465000in}{0.449444in}}%
\pgfpathlineto{\pgfqpoint{2.015000in}{0.449444in}}%
\pgfusepath{stroke}%
\end{pgfscope}%
\begin{pgfscope}%
\pgfsetrectcap%
\pgfsetmiterjoin%
\pgfsetlinewidth{0.803000pt}%
\definecolor{currentstroke}{rgb}{0.000000,0.000000,0.000000}%
\pgfsetstrokecolor{currentstroke}%
\pgfsetdash{}{0pt}%
\pgfpathmoveto{\pgfqpoint{0.465000in}{1.604444in}}%
\pgfpathlineto{\pgfqpoint{2.015000in}{1.604444in}}%
\pgfusepath{stroke}%
\end{pgfscope}%
\begin{pgfscope}%
\pgfsetbuttcap%
\pgfsetmiterjoin%
\definecolor{currentfill}{rgb}{1.000000,1.000000,1.000000}%
\pgfsetfillcolor{currentfill}%
\pgfsetfillopacity{0.800000}%
\pgfsetlinewidth{1.003750pt}%
\definecolor{currentstroke}{rgb}{0.800000,0.800000,0.800000}%
\pgfsetstrokecolor{currentstroke}%
\pgfsetstrokeopacity{0.800000}%
\pgfsetdash{}{0pt}%
\pgfpathmoveto{\pgfqpoint{1.238056in}{1.104445in}}%
\pgfpathlineto{\pgfqpoint{1.917778in}{1.104445in}}%
\pgfpathquadraticcurveto{\pgfqpoint{1.945556in}{1.104445in}}{\pgfqpoint{1.945556in}{1.132222in}}%
\pgfpathlineto{\pgfqpoint{1.945556in}{1.507222in}}%
\pgfpathquadraticcurveto{\pgfqpoint{1.945556in}{1.535000in}}{\pgfqpoint{1.917778in}{1.535000in}}%
\pgfpathlineto{\pgfqpoint{1.238056in}{1.535000in}}%
\pgfpathquadraticcurveto{\pgfqpoint{1.210278in}{1.535000in}}{\pgfqpoint{1.210278in}{1.507222in}}%
\pgfpathlineto{\pgfqpoint{1.210278in}{1.132222in}}%
\pgfpathquadraticcurveto{\pgfqpoint{1.210278in}{1.104445in}}{\pgfqpoint{1.238056in}{1.104445in}}%
\pgfpathlineto{\pgfqpoint{1.238056in}{1.104445in}}%
\pgfpathclose%
\pgfusepath{stroke,fill}%
\end{pgfscope}%
\begin{pgfscope}%
\pgfsetbuttcap%
\pgfsetmiterjoin%
\pgfsetlinewidth{1.003750pt}%
\definecolor{currentstroke}{rgb}{0.000000,0.000000,0.000000}%
\pgfsetstrokecolor{currentstroke}%
\pgfsetdash{}{0pt}%
\pgfpathmoveto{\pgfqpoint{1.265834in}{1.382222in}}%
\pgfpathlineto{\pgfqpoint{1.543611in}{1.382222in}}%
\pgfpathlineto{\pgfqpoint{1.543611in}{1.479444in}}%
\pgfpathlineto{\pgfqpoint{1.265834in}{1.479444in}}%
\pgfpathlineto{\pgfqpoint{1.265834in}{1.382222in}}%
\pgfpathclose%
\pgfusepath{stroke}%
\end{pgfscope}%
\begin{pgfscope}%
\definecolor{textcolor}{rgb}{0.000000,0.000000,0.000000}%
\pgfsetstrokecolor{textcolor}%
\pgfsetfillcolor{textcolor}%
\pgftext[x=1.654722in,y=1.382222in,left,base]{\color{textcolor}\rmfamily\fontsize{10.000000}{12.000000}\selectfont Neg}%
\end{pgfscope}%
\begin{pgfscope}%
\pgfsetbuttcap%
\pgfsetmiterjoin%
\definecolor{currentfill}{rgb}{0.000000,0.000000,0.000000}%
\pgfsetfillcolor{currentfill}%
\pgfsetlinewidth{0.000000pt}%
\definecolor{currentstroke}{rgb}{0.000000,0.000000,0.000000}%
\pgfsetstrokecolor{currentstroke}%
\pgfsetstrokeopacity{0.000000}%
\pgfsetdash{}{0pt}%
\pgfpathmoveto{\pgfqpoint{1.265834in}{1.186944in}}%
\pgfpathlineto{\pgfqpoint{1.543611in}{1.186944in}}%
\pgfpathlineto{\pgfqpoint{1.543611in}{1.284167in}}%
\pgfpathlineto{\pgfqpoint{1.265834in}{1.284167in}}%
\pgfpathlineto{\pgfqpoint{1.265834in}{1.186944in}}%
\pgfpathclose%
\pgfusepath{fill}%
\end{pgfscope}%
\begin{pgfscope}%
\definecolor{textcolor}{rgb}{0.000000,0.000000,0.000000}%
\pgfsetstrokecolor{textcolor}%
\pgfsetfillcolor{textcolor}%
\pgftext[x=1.654722in,y=1.186944in,left,base]{\color{textcolor}\rmfamily\fontsize{10.000000}{12.000000}\selectfont Pos}%
\end{pgfscope}%
\end{pgfpicture}%
\makeatother%
\endgroup%

  &
  \vspace{0pt} %% Creator: Matplotlib, PGF backend
%%
%% To include the figure in your LaTeX document, write
%%   \input{<filename>.pgf}
%%
%% Make sure the required packages are loaded in your preamble
%%   \usepackage{pgf}
%%
%% Also ensure that all the required font packages are loaded; for instance,
%% the lmodern package is sometimes necessary when using math font.
%%   \usepackage{lmodern}
%%
%% Figures using additional raster images can only be included by \input if
%% they are in the same directory as the main LaTeX file. For loading figures
%% from other directories you can use the `import` package
%%   \usepackage{import}
%%
%% and then include the figures with
%%   \import{<path to file>}{<filename>.pgf}
%%
%% Matplotlib used the following preamble
%%   
%%   \usepackage{fontspec}
%%   \makeatletter\@ifpackageloaded{underscore}{}{\usepackage[strings]{underscore}}\makeatother
%%
\begingroup%
\makeatletter%
\begin{pgfpicture}%
\pgfpathrectangle{\pgfpointorigin}{\pgfqpoint{2.221861in}{1.754444in}}%
\pgfusepath{use as bounding box, clip}%
\begin{pgfscope}%
\pgfsetbuttcap%
\pgfsetmiterjoin%
\definecolor{currentfill}{rgb}{1.000000,1.000000,1.000000}%
\pgfsetfillcolor{currentfill}%
\pgfsetlinewidth{0.000000pt}%
\definecolor{currentstroke}{rgb}{1.000000,1.000000,1.000000}%
\pgfsetstrokecolor{currentstroke}%
\pgfsetdash{}{0pt}%
\pgfpathmoveto{\pgfqpoint{0.000000in}{0.000000in}}%
\pgfpathlineto{\pgfqpoint{2.221861in}{0.000000in}}%
\pgfpathlineto{\pgfqpoint{2.221861in}{1.754444in}}%
\pgfpathlineto{\pgfqpoint{0.000000in}{1.754444in}}%
\pgfpathlineto{\pgfqpoint{0.000000in}{0.000000in}}%
\pgfpathclose%
\pgfusepath{fill}%
\end{pgfscope}%
\begin{pgfscope}%
\pgfsetbuttcap%
\pgfsetmiterjoin%
\definecolor{currentfill}{rgb}{1.000000,1.000000,1.000000}%
\pgfsetfillcolor{currentfill}%
\pgfsetlinewidth{0.000000pt}%
\definecolor{currentstroke}{rgb}{0.000000,0.000000,0.000000}%
\pgfsetstrokecolor{currentstroke}%
\pgfsetstrokeopacity{0.000000}%
\pgfsetdash{}{0pt}%
\pgfpathmoveto{\pgfqpoint{0.553581in}{0.499444in}}%
\pgfpathlineto{\pgfqpoint{2.103581in}{0.499444in}}%
\pgfpathlineto{\pgfqpoint{2.103581in}{1.654444in}}%
\pgfpathlineto{\pgfqpoint{0.553581in}{1.654444in}}%
\pgfpathlineto{\pgfqpoint{0.553581in}{0.499444in}}%
\pgfpathclose%
\pgfusepath{fill}%
\end{pgfscope}%
\begin{pgfscope}%
\pgfsetbuttcap%
\pgfsetroundjoin%
\definecolor{currentfill}{rgb}{0.000000,0.000000,0.000000}%
\pgfsetfillcolor{currentfill}%
\pgfsetlinewidth{0.803000pt}%
\definecolor{currentstroke}{rgb}{0.000000,0.000000,0.000000}%
\pgfsetstrokecolor{currentstroke}%
\pgfsetdash{}{0pt}%
\pgfsys@defobject{currentmarker}{\pgfqpoint{0.000000in}{-0.048611in}}{\pgfqpoint{0.000000in}{0.000000in}}{%
\pgfpathmoveto{\pgfqpoint{0.000000in}{0.000000in}}%
\pgfpathlineto{\pgfqpoint{0.000000in}{-0.048611in}}%
\pgfusepath{stroke,fill}%
}%
\begin{pgfscope}%
\pgfsys@transformshift{0.624035in}{0.499444in}%
\pgfsys@useobject{currentmarker}{}%
\end{pgfscope}%
\end{pgfscope}%
\begin{pgfscope}%
\definecolor{textcolor}{rgb}{0.000000,0.000000,0.000000}%
\pgfsetstrokecolor{textcolor}%
\pgfsetfillcolor{textcolor}%
\pgftext[x=0.624035in,y=0.402222in,,top]{\color{textcolor}\rmfamily\fontsize{10.000000}{12.000000}\selectfont \(\displaystyle {0.0}\)}%
\end{pgfscope}%
\begin{pgfscope}%
\pgfsetbuttcap%
\pgfsetroundjoin%
\definecolor{currentfill}{rgb}{0.000000,0.000000,0.000000}%
\pgfsetfillcolor{currentfill}%
\pgfsetlinewidth{0.803000pt}%
\definecolor{currentstroke}{rgb}{0.000000,0.000000,0.000000}%
\pgfsetstrokecolor{currentstroke}%
\pgfsetdash{}{0pt}%
\pgfsys@defobject{currentmarker}{\pgfqpoint{0.000000in}{-0.048611in}}{\pgfqpoint{0.000000in}{0.000000in}}{%
\pgfpathmoveto{\pgfqpoint{0.000000in}{0.000000in}}%
\pgfpathlineto{\pgfqpoint{0.000000in}{-0.048611in}}%
\pgfusepath{stroke,fill}%
}%
\begin{pgfscope}%
\pgfsys@transformshift{1.328581in}{0.499444in}%
\pgfsys@useobject{currentmarker}{}%
\end{pgfscope}%
\end{pgfscope}%
\begin{pgfscope}%
\definecolor{textcolor}{rgb}{0.000000,0.000000,0.000000}%
\pgfsetstrokecolor{textcolor}%
\pgfsetfillcolor{textcolor}%
\pgftext[x=1.328581in,y=0.402222in,,top]{\color{textcolor}\rmfamily\fontsize{10.000000}{12.000000}\selectfont \(\displaystyle {0.5}\)}%
\end{pgfscope}%
\begin{pgfscope}%
\pgfsetbuttcap%
\pgfsetroundjoin%
\definecolor{currentfill}{rgb}{0.000000,0.000000,0.000000}%
\pgfsetfillcolor{currentfill}%
\pgfsetlinewidth{0.803000pt}%
\definecolor{currentstroke}{rgb}{0.000000,0.000000,0.000000}%
\pgfsetstrokecolor{currentstroke}%
\pgfsetdash{}{0pt}%
\pgfsys@defobject{currentmarker}{\pgfqpoint{0.000000in}{-0.048611in}}{\pgfqpoint{0.000000in}{0.000000in}}{%
\pgfpathmoveto{\pgfqpoint{0.000000in}{0.000000in}}%
\pgfpathlineto{\pgfqpoint{0.000000in}{-0.048611in}}%
\pgfusepath{stroke,fill}%
}%
\begin{pgfscope}%
\pgfsys@transformshift{2.033126in}{0.499444in}%
\pgfsys@useobject{currentmarker}{}%
\end{pgfscope}%
\end{pgfscope}%
\begin{pgfscope}%
\definecolor{textcolor}{rgb}{0.000000,0.000000,0.000000}%
\pgfsetstrokecolor{textcolor}%
\pgfsetfillcolor{textcolor}%
\pgftext[x=2.033126in,y=0.402222in,,top]{\color{textcolor}\rmfamily\fontsize{10.000000}{12.000000}\selectfont \(\displaystyle {1.0}\)}%
\end{pgfscope}%
\begin{pgfscope}%
\definecolor{textcolor}{rgb}{0.000000,0.000000,0.000000}%
\pgfsetstrokecolor{textcolor}%
\pgfsetfillcolor{textcolor}%
\pgftext[x=1.328581in,y=0.223333in,,top]{\color{textcolor}\rmfamily\fontsize{10.000000}{12.000000}\selectfont False positive rate}%
\end{pgfscope}%
\begin{pgfscope}%
\pgfsetbuttcap%
\pgfsetroundjoin%
\definecolor{currentfill}{rgb}{0.000000,0.000000,0.000000}%
\pgfsetfillcolor{currentfill}%
\pgfsetlinewidth{0.803000pt}%
\definecolor{currentstroke}{rgb}{0.000000,0.000000,0.000000}%
\pgfsetstrokecolor{currentstroke}%
\pgfsetdash{}{0pt}%
\pgfsys@defobject{currentmarker}{\pgfqpoint{-0.048611in}{0.000000in}}{\pgfqpoint{-0.000000in}{0.000000in}}{%
\pgfpathmoveto{\pgfqpoint{-0.000000in}{0.000000in}}%
\pgfpathlineto{\pgfqpoint{-0.048611in}{0.000000in}}%
\pgfusepath{stroke,fill}%
}%
\begin{pgfscope}%
\pgfsys@transformshift{0.553581in}{0.551944in}%
\pgfsys@useobject{currentmarker}{}%
\end{pgfscope}%
\end{pgfscope}%
\begin{pgfscope}%
\definecolor{textcolor}{rgb}{0.000000,0.000000,0.000000}%
\pgfsetstrokecolor{textcolor}%
\pgfsetfillcolor{textcolor}%
\pgftext[x=0.278889in, y=0.503750in, left, base]{\color{textcolor}\rmfamily\fontsize{10.000000}{12.000000}\selectfont \(\displaystyle {0.0}\)}%
\end{pgfscope}%
\begin{pgfscope}%
\pgfsetbuttcap%
\pgfsetroundjoin%
\definecolor{currentfill}{rgb}{0.000000,0.000000,0.000000}%
\pgfsetfillcolor{currentfill}%
\pgfsetlinewidth{0.803000pt}%
\definecolor{currentstroke}{rgb}{0.000000,0.000000,0.000000}%
\pgfsetstrokecolor{currentstroke}%
\pgfsetdash{}{0pt}%
\pgfsys@defobject{currentmarker}{\pgfqpoint{-0.048611in}{0.000000in}}{\pgfqpoint{-0.000000in}{0.000000in}}{%
\pgfpathmoveto{\pgfqpoint{-0.000000in}{0.000000in}}%
\pgfpathlineto{\pgfqpoint{-0.048611in}{0.000000in}}%
\pgfusepath{stroke,fill}%
}%
\begin{pgfscope}%
\pgfsys@transformshift{0.553581in}{1.076944in}%
\pgfsys@useobject{currentmarker}{}%
\end{pgfscope}%
\end{pgfscope}%
\begin{pgfscope}%
\definecolor{textcolor}{rgb}{0.000000,0.000000,0.000000}%
\pgfsetstrokecolor{textcolor}%
\pgfsetfillcolor{textcolor}%
\pgftext[x=0.278889in, y=1.028750in, left, base]{\color{textcolor}\rmfamily\fontsize{10.000000}{12.000000}\selectfont \(\displaystyle {0.5}\)}%
\end{pgfscope}%
\begin{pgfscope}%
\pgfsetbuttcap%
\pgfsetroundjoin%
\definecolor{currentfill}{rgb}{0.000000,0.000000,0.000000}%
\pgfsetfillcolor{currentfill}%
\pgfsetlinewidth{0.803000pt}%
\definecolor{currentstroke}{rgb}{0.000000,0.000000,0.000000}%
\pgfsetstrokecolor{currentstroke}%
\pgfsetdash{}{0pt}%
\pgfsys@defobject{currentmarker}{\pgfqpoint{-0.048611in}{0.000000in}}{\pgfqpoint{-0.000000in}{0.000000in}}{%
\pgfpathmoveto{\pgfqpoint{-0.000000in}{0.000000in}}%
\pgfpathlineto{\pgfqpoint{-0.048611in}{0.000000in}}%
\pgfusepath{stroke,fill}%
}%
\begin{pgfscope}%
\pgfsys@transformshift{0.553581in}{1.601944in}%
\pgfsys@useobject{currentmarker}{}%
\end{pgfscope}%
\end{pgfscope}%
\begin{pgfscope}%
\definecolor{textcolor}{rgb}{0.000000,0.000000,0.000000}%
\pgfsetstrokecolor{textcolor}%
\pgfsetfillcolor{textcolor}%
\pgftext[x=0.278889in, y=1.553750in, left, base]{\color{textcolor}\rmfamily\fontsize{10.000000}{12.000000}\selectfont \(\displaystyle {1.0}\)}%
\end{pgfscope}%
\begin{pgfscope}%
\definecolor{textcolor}{rgb}{0.000000,0.000000,0.000000}%
\pgfsetstrokecolor{textcolor}%
\pgfsetfillcolor{textcolor}%
\pgftext[x=0.223333in,y=1.076944in,,bottom,rotate=90.000000]{\color{textcolor}\rmfamily\fontsize{10.000000}{12.000000}\selectfont True positive rate}%
\end{pgfscope}%
\begin{pgfscope}%
\pgfpathrectangle{\pgfqpoint{0.553581in}{0.499444in}}{\pgfqpoint{1.550000in}{1.155000in}}%
\pgfusepath{clip}%
\pgfsetbuttcap%
\pgfsetroundjoin%
\pgfsetlinewidth{1.505625pt}%
\definecolor{currentstroke}{rgb}{0.000000,0.000000,0.000000}%
\pgfsetstrokecolor{currentstroke}%
\pgfsetdash{{5.550000pt}{2.400000pt}}{0.000000pt}%
\pgfpathmoveto{\pgfqpoint{0.624035in}{0.551944in}}%
\pgfpathlineto{\pgfqpoint{2.033126in}{1.601944in}}%
\pgfusepath{stroke}%
\end{pgfscope}%
\begin{pgfscope}%
\pgfpathrectangle{\pgfqpoint{0.553581in}{0.499444in}}{\pgfqpoint{1.550000in}{1.155000in}}%
\pgfusepath{clip}%
\pgfsetrectcap%
\pgfsetroundjoin%
\pgfsetlinewidth{1.505625pt}%
\definecolor{currentstroke}{rgb}{0.000000,0.000000,0.000000}%
\pgfsetstrokecolor{currentstroke}%
\pgfsetdash{}{0pt}%
\pgfpathmoveto{\pgfqpoint{0.624035in}{0.551944in}}%
\pgfpathlineto{\pgfqpoint{0.626207in}{0.552574in}}%
\pgfpathlineto{\pgfqpoint{0.627318in}{0.561464in}}%
\pgfpathlineto{\pgfqpoint{0.628014in}{0.562514in}}%
\pgfpathlineto{\pgfqpoint{0.629125in}{0.563634in}}%
\pgfpathlineto{\pgfqpoint{0.629605in}{0.564614in}}%
\pgfpathlineto{\pgfqpoint{0.630699in}{0.567694in}}%
\pgfpathlineto{\pgfqpoint{0.631130in}{0.568604in}}%
\pgfpathlineto{\pgfqpoint{0.632225in}{0.573294in}}%
\pgfpathlineto{\pgfqpoint{0.632772in}{0.574064in}}%
\pgfpathlineto{\pgfqpoint{0.633866in}{0.579944in}}%
\pgfpathlineto{\pgfqpoint{0.634247in}{0.580854in}}%
\pgfpathlineto{\pgfqpoint{0.635358in}{0.585824in}}%
\pgfpathlineto{\pgfqpoint{0.635540in}{0.586874in}}%
\pgfpathlineto{\pgfqpoint{0.636634in}{0.593244in}}%
\pgfpathlineto{\pgfqpoint{0.636833in}{0.594154in}}%
\pgfpathlineto{\pgfqpoint{0.637944in}{0.600734in}}%
\pgfpathlineto{\pgfqpoint{0.638126in}{0.601784in}}%
\pgfpathlineto{\pgfqpoint{0.639187in}{0.608714in}}%
\pgfpathlineto{\pgfqpoint{0.639353in}{0.609694in}}%
\pgfpathlineto{\pgfqpoint{0.640464in}{0.616834in}}%
\pgfpathlineto{\pgfqpoint{0.640712in}{0.617884in}}%
\pgfpathlineto{\pgfqpoint{0.641806in}{0.625864in}}%
\pgfpathlineto{\pgfqpoint{0.642038in}{0.626844in}}%
\pgfpathlineto{\pgfqpoint{0.643149in}{0.635804in}}%
\pgfpathlineto{\pgfqpoint{0.643348in}{0.636854in}}%
\pgfpathlineto{\pgfqpoint{0.644392in}{0.645044in}}%
\pgfpathlineto{\pgfqpoint{0.644641in}{0.645884in}}%
\pgfpathlineto{\pgfqpoint{0.645752in}{0.652394in}}%
\pgfpathlineto{\pgfqpoint{0.645868in}{0.653164in}}%
\pgfpathlineto{\pgfqpoint{0.646979in}{0.661424in}}%
\pgfpathlineto{\pgfqpoint{0.647327in}{0.662264in}}%
\pgfpathlineto{\pgfqpoint{0.648437in}{0.668914in}}%
\pgfpathlineto{\pgfqpoint{0.648587in}{0.669824in}}%
\pgfpathlineto{\pgfqpoint{0.649697in}{0.678294in}}%
\pgfpathlineto{\pgfqpoint{0.649813in}{0.679274in}}%
\pgfpathlineto{\pgfqpoint{0.650924in}{0.687044in}}%
\pgfpathlineto{\pgfqpoint{0.651140in}{0.687884in}}%
\pgfpathlineto{\pgfqpoint{0.652250in}{0.696144in}}%
\pgfpathlineto{\pgfqpoint{0.652333in}{0.697194in}}%
\pgfpathlineto{\pgfqpoint{0.653444in}{0.706434in}}%
\pgfpathlineto{\pgfqpoint{0.653858in}{0.707414in}}%
\pgfpathlineto{\pgfqpoint{0.654969in}{0.716584in}}%
\pgfpathlineto{\pgfqpoint{0.655151in}{0.717564in}}%
\pgfpathlineto{\pgfqpoint{0.656262in}{0.725194in}}%
\pgfpathlineto{\pgfqpoint{0.656444in}{0.726174in}}%
\pgfpathlineto{\pgfqpoint{0.657538in}{0.733244in}}%
\pgfpathlineto{\pgfqpoint{0.657704in}{0.734294in}}%
\pgfpathlineto{\pgfqpoint{0.658815in}{0.743464in}}%
\pgfpathlineto{\pgfqpoint{0.659064in}{0.744514in}}%
\pgfpathlineto{\pgfqpoint{0.660158in}{0.751164in}}%
\pgfpathlineto{\pgfqpoint{0.660290in}{0.751794in}}%
\pgfpathlineto{\pgfqpoint{0.661401in}{0.759354in}}%
\pgfpathlineto{\pgfqpoint{0.661633in}{0.760404in}}%
\pgfpathlineto{\pgfqpoint{0.662744in}{0.769784in}}%
\pgfpathlineto{\pgfqpoint{0.662827in}{0.770344in}}%
\pgfpathlineto{\pgfqpoint{0.663937in}{0.777764in}}%
\pgfpathlineto{\pgfqpoint{0.664120in}{0.778814in}}%
\pgfpathlineto{\pgfqpoint{0.665181in}{0.783854in}}%
\pgfpathlineto{\pgfqpoint{0.665297in}{0.784834in}}%
\pgfpathlineto{\pgfqpoint{0.666407in}{0.795614in}}%
\pgfpathlineto{\pgfqpoint{0.666656in}{0.796664in}}%
\pgfpathlineto{\pgfqpoint{0.667767in}{0.801564in}}%
\pgfpathlineto{\pgfqpoint{0.667899in}{0.802404in}}%
\pgfpathlineto{\pgfqpoint{0.669010in}{0.809334in}}%
\pgfpathlineto{\pgfqpoint{0.669342in}{0.810384in}}%
\pgfpathlineto{\pgfqpoint{0.670452in}{0.818084in}}%
\pgfpathlineto{\pgfqpoint{0.670701in}{0.819134in}}%
\pgfpathlineto{\pgfqpoint{0.671812in}{0.825364in}}%
\pgfpathlineto{\pgfqpoint{0.671911in}{0.826414in}}%
\pgfpathlineto{\pgfqpoint{0.673022in}{0.833694in}}%
\pgfpathlineto{\pgfqpoint{0.673138in}{0.834744in}}%
\pgfpathlineto{\pgfqpoint{0.674215in}{0.838944in}}%
\pgfpathlineto{\pgfqpoint{0.674530in}{0.839994in}}%
\pgfpathlineto{\pgfqpoint{0.675608in}{0.846014in}}%
\pgfpathlineto{\pgfqpoint{0.675790in}{0.846994in}}%
\pgfpathlineto{\pgfqpoint{0.676901in}{0.853854in}}%
\pgfpathlineto{\pgfqpoint{0.677199in}{0.854834in}}%
\pgfpathlineto{\pgfqpoint{0.678310in}{0.859034in}}%
\pgfpathlineto{\pgfqpoint{0.678492in}{0.860084in}}%
\pgfpathlineto{\pgfqpoint{0.679587in}{0.867014in}}%
\pgfpathlineto{\pgfqpoint{0.679868in}{0.867924in}}%
\pgfpathlineto{\pgfqpoint{0.680963in}{0.874504in}}%
\pgfpathlineto{\pgfqpoint{0.681228in}{0.875484in}}%
\pgfpathlineto{\pgfqpoint{0.682338in}{0.882204in}}%
\pgfpathlineto{\pgfqpoint{0.682587in}{0.883254in}}%
\pgfpathlineto{\pgfqpoint{0.683681in}{0.889484in}}%
\pgfpathlineto{\pgfqpoint{0.683980in}{0.890394in}}%
\pgfpathlineto{\pgfqpoint{0.685074in}{0.897394in}}%
\pgfpathlineto{\pgfqpoint{0.685389in}{0.898304in}}%
\pgfpathlineto{\pgfqpoint{0.686450in}{0.904184in}}%
\pgfpathlineto{\pgfqpoint{0.686748in}{0.905094in}}%
\pgfpathlineto{\pgfqpoint{0.687859in}{0.910204in}}%
\pgfpathlineto{\pgfqpoint{0.688074in}{0.911114in}}%
\pgfpathlineto{\pgfqpoint{0.689185in}{0.915034in}}%
\pgfpathlineto{\pgfqpoint{0.689400in}{0.916084in}}%
\pgfpathlineto{\pgfqpoint{0.690511in}{0.922664in}}%
\pgfpathlineto{\pgfqpoint{0.690793in}{0.923714in}}%
\pgfpathlineto{\pgfqpoint{0.691887in}{0.928754in}}%
\pgfpathlineto{\pgfqpoint{0.692318in}{0.929734in}}%
\pgfpathlineto{\pgfqpoint{0.693412in}{0.935334in}}%
\pgfpathlineto{\pgfqpoint{0.693611in}{0.936384in}}%
\pgfpathlineto{\pgfqpoint{0.694705in}{0.940864in}}%
\pgfpathlineto{\pgfqpoint{0.695053in}{0.941914in}}%
\pgfpathlineto{\pgfqpoint{0.696164in}{0.948634in}}%
\pgfpathlineto{\pgfqpoint{0.696645in}{0.949684in}}%
\pgfpathlineto{\pgfqpoint{0.697756in}{0.954164in}}%
\pgfpathlineto{\pgfqpoint{0.697921in}{0.955144in}}%
\pgfpathlineto{\pgfqpoint{0.699015in}{0.960464in}}%
\pgfpathlineto{\pgfqpoint{0.699314in}{0.961304in}}%
\pgfpathlineto{\pgfqpoint{0.700425in}{0.965294in}}%
\pgfpathlineto{\pgfqpoint{0.700657in}{0.966344in}}%
\pgfpathlineto{\pgfqpoint{0.701767in}{0.971314in}}%
\pgfpathlineto{\pgfqpoint{0.702049in}{0.972364in}}%
\pgfpathlineto{\pgfqpoint{0.703094in}{0.977194in}}%
\pgfpathlineto{\pgfqpoint{0.703442in}{0.978104in}}%
\pgfpathlineto{\pgfqpoint{0.704536in}{0.982514in}}%
\pgfpathlineto{\pgfqpoint{0.704917in}{0.983564in}}%
\pgfpathlineto{\pgfqpoint{0.706011in}{0.988464in}}%
\pgfpathlineto{\pgfqpoint{0.706160in}{0.989444in}}%
\pgfpathlineto{\pgfqpoint{0.707271in}{0.993504in}}%
\pgfpathlineto{\pgfqpoint{0.707619in}{0.994414in}}%
\pgfpathlineto{\pgfqpoint{0.708697in}{0.998054in}}%
\pgfpathlineto{\pgfqpoint{0.709061in}{0.999104in}}%
\pgfpathlineto{\pgfqpoint{0.710172in}{1.004564in}}%
\pgfpathlineto{\pgfqpoint{0.710504in}{1.005614in}}%
\pgfpathlineto{\pgfqpoint{0.711614in}{1.009604in}}%
\pgfpathlineto{\pgfqpoint{0.711830in}{1.010584in}}%
\pgfpathlineto{\pgfqpoint{0.712874in}{1.015134in}}%
\pgfpathlineto{\pgfqpoint{0.713123in}{1.016184in}}%
\pgfpathlineto{\pgfqpoint{0.714184in}{1.020244in}}%
\pgfpathlineto{\pgfqpoint{0.714598in}{1.021294in}}%
\pgfpathlineto{\pgfqpoint{0.715709in}{1.024304in}}%
\pgfpathlineto{\pgfqpoint{0.716173in}{1.025354in}}%
\pgfpathlineto{\pgfqpoint{0.717284in}{1.029624in}}%
\pgfpathlineto{\pgfqpoint{0.717466in}{1.030604in}}%
\pgfpathlineto{\pgfqpoint{0.718560in}{1.035014in}}%
\pgfpathlineto{\pgfqpoint{0.718776in}{1.035994in}}%
\pgfpathlineto{\pgfqpoint{0.719853in}{1.041034in}}%
\pgfpathlineto{\pgfqpoint{0.720202in}{1.042084in}}%
\pgfpathlineto{\pgfqpoint{0.721312in}{1.046144in}}%
\pgfpathlineto{\pgfqpoint{0.721694in}{1.047194in}}%
\pgfpathlineto{\pgfqpoint{0.722788in}{1.050904in}}%
\pgfpathlineto{\pgfqpoint{0.723086in}{1.051954in}}%
\pgfpathlineto{\pgfqpoint{0.724180in}{1.056364in}}%
\pgfpathlineto{\pgfqpoint{0.724727in}{1.057414in}}%
\pgfpathlineto{\pgfqpoint{0.725805in}{1.061264in}}%
\pgfpathlineto{\pgfqpoint{0.726236in}{1.062314in}}%
\pgfpathlineto{\pgfqpoint{0.727280in}{1.065744in}}%
\pgfpathlineto{\pgfqpoint{0.727728in}{1.066794in}}%
\pgfpathlineto{\pgfqpoint{0.728805in}{1.069874in}}%
\pgfpathlineto{\pgfqpoint{0.729253in}{1.070924in}}%
\pgfpathlineto{\pgfqpoint{0.730364in}{1.074144in}}%
\pgfpathlineto{\pgfqpoint{0.730695in}{1.075124in}}%
\pgfpathlineto{\pgfqpoint{0.731789in}{1.079044in}}%
\pgfpathlineto{\pgfqpoint{0.732336in}{1.080024in}}%
\pgfpathlineto{\pgfqpoint{0.733447in}{1.084574in}}%
\pgfpathlineto{\pgfqpoint{0.734094in}{1.085624in}}%
\pgfpathlineto{\pgfqpoint{0.735188in}{1.089544in}}%
\pgfpathlineto{\pgfqpoint{0.735552in}{1.090594in}}%
\pgfpathlineto{\pgfqpoint{0.736597in}{1.093674in}}%
\pgfpathlineto{\pgfqpoint{0.737177in}{1.094724in}}%
\pgfpathlineto{\pgfqpoint{0.738221in}{1.097524in}}%
\pgfpathlineto{\pgfqpoint{0.738702in}{1.098574in}}%
\pgfpathlineto{\pgfqpoint{0.739780in}{1.102424in}}%
\pgfpathlineto{\pgfqpoint{0.740128in}{1.103474in}}%
\pgfpathlineto{\pgfqpoint{0.741238in}{1.106624in}}%
\pgfpathlineto{\pgfqpoint{0.741636in}{1.107674in}}%
\pgfpathlineto{\pgfqpoint{0.742697in}{1.111034in}}%
\pgfpathlineto{\pgfqpoint{0.743095in}{1.112084in}}%
\pgfpathlineto{\pgfqpoint{0.744206in}{1.115024in}}%
\pgfpathlineto{\pgfqpoint{0.744769in}{1.116074in}}%
\pgfpathlineto{\pgfqpoint{0.745051in}{1.117054in}}%
\pgfpathlineto{\pgfqpoint{0.745068in}{1.117054in}}%
\pgfpathlineto{\pgfqpoint{0.755943in}{1.118104in}}%
\pgfpathlineto{\pgfqpoint{0.757053in}{1.121814in}}%
\pgfpathlineto{\pgfqpoint{0.757600in}{1.122864in}}%
\pgfpathlineto{\pgfqpoint{0.758678in}{1.125384in}}%
\pgfpathlineto{\pgfqpoint{0.759225in}{1.126434in}}%
\pgfpathlineto{\pgfqpoint{0.760269in}{1.129094in}}%
\pgfpathlineto{\pgfqpoint{0.760817in}{1.130144in}}%
\pgfpathlineto{\pgfqpoint{0.761911in}{1.133224in}}%
\pgfpathlineto{\pgfqpoint{0.762474in}{1.134274in}}%
\pgfpathlineto{\pgfqpoint{0.763535in}{1.137844in}}%
\pgfpathlineto{\pgfqpoint{0.764165in}{1.138824in}}%
\pgfpathlineto{\pgfqpoint{0.765259in}{1.142324in}}%
\pgfpathlineto{\pgfqpoint{0.766055in}{1.143374in}}%
\pgfpathlineto{\pgfqpoint{0.767166in}{1.146314in}}%
\pgfpathlineto{\pgfqpoint{0.767696in}{1.147364in}}%
\pgfpathlineto{\pgfqpoint{0.768790in}{1.149954in}}%
\pgfpathlineto{\pgfqpoint{0.769603in}{1.151004in}}%
\pgfpathlineto{\pgfqpoint{0.770514in}{1.153314in}}%
\pgfpathlineto{\pgfqpoint{0.771161in}{1.154364in}}%
\pgfpathlineto{\pgfqpoint{0.772272in}{1.157374in}}%
\pgfpathlineto{\pgfqpoint{0.772852in}{1.158424in}}%
\pgfpathlineto{\pgfqpoint{0.773946in}{1.160664in}}%
\pgfpathlineto{\pgfqpoint{0.774609in}{1.161714in}}%
\pgfpathlineto{\pgfqpoint{0.775703in}{1.164794in}}%
\pgfpathlineto{\pgfqpoint{0.776267in}{1.165774in}}%
\pgfpathlineto{\pgfqpoint{0.777344in}{1.168714in}}%
\pgfpathlineto{\pgfqpoint{0.777759in}{1.169764in}}%
\pgfpathlineto{\pgfqpoint{0.778836in}{1.172144in}}%
\pgfpathlineto{\pgfqpoint{0.779350in}{1.173124in}}%
\pgfpathlineto{\pgfqpoint{0.780444in}{1.174524in}}%
\pgfpathlineto{\pgfqpoint{0.781008in}{1.175574in}}%
\pgfpathlineto{\pgfqpoint{0.782052in}{1.177254in}}%
\pgfpathlineto{\pgfqpoint{0.782699in}{1.178304in}}%
\pgfpathlineto{\pgfqpoint{0.783793in}{1.180894in}}%
\pgfpathlineto{\pgfqpoint{0.784290in}{1.181874in}}%
\pgfpathlineto{\pgfqpoint{0.785384in}{1.184044in}}%
\pgfpathlineto{\pgfqpoint{0.785981in}{1.185094in}}%
\pgfpathlineto{\pgfqpoint{0.787075in}{1.186914in}}%
\pgfpathlineto{\pgfqpoint{0.787755in}{1.187894in}}%
\pgfpathlineto{\pgfqpoint{0.788866in}{1.189784in}}%
\pgfpathlineto{\pgfqpoint{0.789297in}{1.190764in}}%
\pgfpathlineto{\pgfqpoint{0.790407in}{1.192794in}}%
\pgfpathlineto{\pgfqpoint{0.790689in}{1.193634in}}%
\pgfpathlineto{\pgfqpoint{0.791783in}{1.196084in}}%
\pgfpathlineto{\pgfqpoint{0.792347in}{1.197134in}}%
\pgfpathlineto{\pgfqpoint{0.793358in}{1.199724in}}%
\pgfpathlineto{\pgfqpoint{0.794054in}{1.200774in}}%
\pgfpathlineto{\pgfqpoint{0.795099in}{1.201964in}}%
\pgfpathlineto{\pgfqpoint{0.796044in}{1.203014in}}%
\pgfpathlineto{\pgfqpoint{0.797138in}{1.205534in}}%
\pgfpathlineto{\pgfqpoint{0.797718in}{1.206584in}}%
\pgfpathlineto{\pgfqpoint{0.798812in}{1.208474in}}%
\pgfpathlineto{\pgfqpoint{0.799409in}{1.209524in}}%
\pgfpathlineto{\pgfqpoint{0.800520in}{1.212114in}}%
\pgfpathlineto{\pgfqpoint{0.801382in}{1.213094in}}%
\pgfpathlineto{\pgfqpoint{0.802492in}{1.215894in}}%
\pgfpathlineto{\pgfqpoint{0.803106in}{1.216664in}}%
\pgfpathlineto{\pgfqpoint{0.804134in}{1.218554in}}%
\pgfpathlineto{\pgfqpoint{0.805128in}{1.219604in}}%
\pgfpathlineto{\pgfqpoint{0.806222in}{1.221774in}}%
\pgfpathlineto{\pgfqpoint{0.806786in}{1.222614in}}%
\pgfpathlineto{\pgfqpoint{0.807897in}{1.225484in}}%
\pgfpathlineto{\pgfqpoint{0.808659in}{1.226534in}}%
\pgfpathlineto{\pgfqpoint{0.809753in}{1.228284in}}%
\pgfpathlineto{\pgfqpoint{0.810267in}{1.229334in}}%
\pgfpathlineto{\pgfqpoint{0.811378in}{1.231224in}}%
\pgfpathlineto{\pgfqpoint{0.812654in}{1.232274in}}%
\pgfpathlineto{\pgfqpoint{0.813682in}{1.233744in}}%
\pgfpathlineto{\pgfqpoint{0.814180in}{1.234794in}}%
\pgfpathlineto{\pgfqpoint{0.815290in}{1.236614in}}%
\pgfpathlineto{\pgfqpoint{0.816053in}{1.237664in}}%
\pgfpathlineto{\pgfqpoint{0.816948in}{1.239484in}}%
\pgfpathlineto{\pgfqpoint{0.818109in}{1.240464in}}%
\pgfpathlineto{\pgfqpoint{0.819203in}{1.242424in}}%
\pgfpathlineto{\pgfqpoint{0.819949in}{1.243404in}}%
\pgfpathlineto{\pgfqpoint{0.821043in}{1.245364in}}%
\pgfpathlineto{\pgfqpoint{0.821838in}{1.246414in}}%
\pgfpathlineto{\pgfqpoint{0.822899in}{1.248374in}}%
\pgfpathlineto{\pgfqpoint{0.823828in}{1.249424in}}%
\pgfpathlineto{\pgfqpoint{0.824905in}{1.251314in}}%
\pgfpathlineto{\pgfqpoint{0.826049in}{1.252364in}}%
\pgfpathlineto{\pgfqpoint{0.827110in}{1.254044in}}%
\pgfpathlineto{\pgfqpoint{0.828204in}{1.255024in}}%
\pgfpathlineto{\pgfqpoint{0.829298in}{1.257404in}}%
\pgfpathlineto{\pgfqpoint{0.830094in}{1.258454in}}%
\pgfpathlineto{\pgfqpoint{0.831188in}{1.259784in}}%
\pgfpathlineto{\pgfqpoint{0.831901in}{1.260834in}}%
\pgfpathlineto{\pgfqpoint{0.832979in}{1.261954in}}%
\pgfpathlineto{\pgfqpoint{0.833841in}{1.263004in}}%
\pgfpathlineto{\pgfqpoint{0.834769in}{1.263774in}}%
\pgfpathlineto{\pgfqpoint{0.835697in}{1.264824in}}%
\pgfpathlineto{\pgfqpoint{0.836659in}{1.266014in}}%
\pgfpathlineto{\pgfqpoint{0.838234in}{1.267064in}}%
\pgfpathlineto{\pgfqpoint{0.839295in}{1.268814in}}%
\pgfpathlineto{\pgfqpoint{0.840438in}{1.269864in}}%
\pgfpathlineto{\pgfqpoint{0.841549in}{1.271824in}}%
\pgfpathlineto{\pgfqpoint{0.842594in}{1.272874in}}%
\pgfpathlineto{\pgfqpoint{0.843688in}{1.274414in}}%
\pgfpathlineto{\pgfqpoint{0.844881in}{1.275464in}}%
\pgfpathlineto{\pgfqpoint{0.845992in}{1.277844in}}%
\pgfpathlineto{\pgfqpoint{0.846705in}{1.278894in}}%
\pgfpathlineto{\pgfqpoint{0.847633in}{1.280294in}}%
\pgfpathlineto{\pgfqpoint{0.848644in}{1.281274in}}%
\pgfpathlineto{\pgfqpoint{0.849606in}{1.282954in}}%
\pgfpathlineto{\pgfqpoint{0.850882in}{1.284004in}}%
\pgfpathlineto{\pgfqpoint{0.851960in}{1.285194in}}%
\pgfpathlineto{\pgfqpoint{0.853120in}{1.286244in}}%
\pgfpathlineto{\pgfqpoint{0.854198in}{1.287644in}}%
\pgfpathlineto{\pgfqpoint{0.855176in}{1.288694in}}%
\pgfpathlineto{\pgfqpoint{0.856237in}{1.290234in}}%
\pgfpathlineto{\pgfqpoint{0.856917in}{1.291144in}}%
\pgfpathlineto{\pgfqpoint{0.858027in}{1.292754in}}%
\pgfpathlineto{\pgfqpoint{0.859287in}{1.293804in}}%
\pgfpathlineto{\pgfqpoint{0.860232in}{1.294714in}}%
\pgfpathlineto{\pgfqpoint{0.861160in}{1.295764in}}%
\pgfpathlineto{\pgfqpoint{0.862155in}{1.296814in}}%
\pgfpathlineto{\pgfqpoint{0.863746in}{1.297794in}}%
\pgfpathlineto{\pgfqpoint{0.864807in}{1.299054in}}%
\pgfpathlineto{\pgfqpoint{0.866084in}{1.300104in}}%
\pgfpathlineto{\pgfqpoint{0.867128in}{1.301854in}}%
\pgfpathlineto{\pgfqpoint{0.867775in}{1.302904in}}%
\pgfpathlineto{\pgfqpoint{0.868886in}{1.304164in}}%
\pgfpathlineto{\pgfqpoint{0.869897in}{1.305144in}}%
\pgfpathlineto{\pgfqpoint{0.870991in}{1.306614in}}%
\pgfpathlineto{\pgfqpoint{0.872218in}{1.307664in}}%
\pgfpathlineto{\pgfqpoint{0.873312in}{1.309624in}}%
\pgfpathlineto{\pgfqpoint{0.874572in}{1.310674in}}%
\pgfpathlineto{\pgfqpoint{0.875616in}{1.312284in}}%
\pgfpathlineto{\pgfqpoint{0.876263in}{1.313334in}}%
\pgfpathlineto{\pgfqpoint{0.877373in}{1.314804in}}%
\pgfpathlineto{\pgfqpoint{0.878036in}{1.315854in}}%
\pgfpathlineto{\pgfqpoint{0.879081in}{1.316974in}}%
\pgfpathlineto{\pgfqpoint{0.880191in}{1.318024in}}%
\pgfpathlineto{\pgfqpoint{0.881252in}{1.319564in}}%
\pgfpathlineto{\pgfqpoint{0.882927in}{1.320544in}}%
\pgfpathlineto{\pgfqpoint{0.883954in}{1.321384in}}%
\pgfpathlineto{\pgfqpoint{0.885165in}{1.322434in}}%
\pgfpathlineto{\pgfqpoint{0.886226in}{1.323624in}}%
\pgfpathlineto{\pgfqpoint{0.887121in}{1.324674in}}%
\pgfpathlineto{\pgfqpoint{0.888215in}{1.325864in}}%
\pgfpathlineto{\pgfqpoint{0.889624in}{1.326914in}}%
\pgfpathlineto{\pgfqpoint{0.890619in}{1.328104in}}%
\pgfpathlineto{\pgfqpoint{0.892044in}{1.329154in}}%
\pgfpathlineto{\pgfqpoint{0.892857in}{1.330064in}}%
\pgfpathlineto{\pgfqpoint{0.894100in}{1.331044in}}%
\pgfpathlineto{\pgfqpoint{0.894746in}{1.331814in}}%
\pgfpathlineto{\pgfqpoint{0.896089in}{1.332864in}}%
\pgfpathlineto{\pgfqpoint{0.897200in}{1.333984in}}%
\pgfpathlineto{\pgfqpoint{0.898327in}{1.335034in}}%
\pgfpathlineto{\pgfqpoint{0.899338in}{1.336714in}}%
\pgfpathlineto{\pgfqpoint{0.900267in}{1.337694in}}%
\pgfpathlineto{\pgfqpoint{0.901311in}{1.338534in}}%
\pgfpathlineto{\pgfqpoint{0.902554in}{1.339584in}}%
\pgfpathlineto{\pgfqpoint{0.903649in}{1.340494in}}%
\pgfpathlineto{\pgfqpoint{0.905240in}{1.341544in}}%
\pgfpathlineto{\pgfqpoint{0.906334in}{1.342874in}}%
\pgfpathlineto{\pgfqpoint{0.907743in}{1.343924in}}%
\pgfpathlineto{\pgfqpoint{0.908854in}{1.344764in}}%
\pgfpathlineto{\pgfqpoint{0.910048in}{1.345814in}}%
\pgfpathlineto{\pgfqpoint{0.911142in}{1.346654in}}%
\pgfpathlineto{\pgfqpoint{0.912169in}{1.347704in}}%
\pgfpathlineto{\pgfqpoint{0.913114in}{1.348754in}}%
\pgfpathlineto{\pgfqpoint{0.914242in}{1.349804in}}%
\pgfpathlineto{\pgfqpoint{0.915004in}{1.350924in}}%
\pgfpathlineto{\pgfqpoint{0.916894in}{1.351974in}}%
\pgfpathlineto{\pgfqpoint{0.917988in}{1.353234in}}%
\pgfpathlineto{\pgfqpoint{0.919646in}{1.354284in}}%
\pgfpathlineto{\pgfqpoint{0.920757in}{1.356034in}}%
\pgfpathlineto{\pgfqpoint{0.922166in}{1.357084in}}%
\pgfpathlineto{\pgfqpoint{0.923260in}{1.358204in}}%
\pgfpathlineto{\pgfqpoint{0.925332in}{1.359254in}}%
\pgfpathlineto{\pgfqpoint{0.926376in}{1.360164in}}%
\pgfpathlineto{\pgfqpoint{0.927686in}{1.361214in}}%
\pgfpathlineto{\pgfqpoint{0.928780in}{1.361914in}}%
\pgfpathlineto{\pgfqpoint{0.930886in}{1.362964in}}%
\pgfpathlineto{\pgfqpoint{0.931980in}{1.364294in}}%
\pgfpathlineto{\pgfqpoint{0.932941in}{1.365344in}}%
\pgfpathlineto{\pgfqpoint{0.934035in}{1.366604in}}%
\pgfpathlineto{\pgfqpoint{0.935925in}{1.367654in}}%
\pgfpathlineto{\pgfqpoint{0.937003in}{1.368984in}}%
\pgfpathlineto{\pgfqpoint{0.939307in}{1.370034in}}%
\pgfpathlineto{\pgfqpoint{0.940302in}{1.371014in}}%
\pgfpathlineto{\pgfqpoint{0.941876in}{1.372064in}}%
\pgfpathlineto{\pgfqpoint{0.942871in}{1.372694in}}%
\pgfpathlineto{\pgfqpoint{0.944562in}{1.373674in}}%
\pgfpathlineto{\pgfqpoint{0.945573in}{1.374934in}}%
\pgfpathlineto{\pgfqpoint{0.947314in}{1.375984in}}%
\pgfpathlineto{\pgfqpoint{0.948159in}{1.376824in}}%
\pgfpathlineto{\pgfqpoint{0.950663in}{1.377874in}}%
\pgfpathlineto{\pgfqpoint{0.951707in}{1.378504in}}%
\pgfpathlineto{\pgfqpoint{0.953083in}{1.379484in}}%
\pgfpathlineto{\pgfqpoint{0.954127in}{1.380394in}}%
\pgfpathlineto{\pgfqpoint{0.956183in}{1.381444in}}%
\pgfpathlineto{\pgfqpoint{0.957244in}{1.382914in}}%
\pgfpathlineto{\pgfqpoint{0.958736in}{1.383964in}}%
\pgfpathlineto{\pgfqpoint{0.959714in}{1.384874in}}%
\pgfpathlineto{\pgfqpoint{0.961902in}{1.385924in}}%
\pgfpathlineto{\pgfqpoint{0.962781in}{1.386414in}}%
\pgfpathlineto{\pgfqpoint{0.965599in}{1.387464in}}%
\pgfpathlineto{\pgfqpoint{0.966643in}{1.388514in}}%
\pgfpathlineto{\pgfqpoint{0.968069in}{1.389564in}}%
\pgfpathlineto{\pgfqpoint{0.968931in}{1.390404in}}%
\pgfpathlineto{\pgfqpoint{0.971733in}{1.391454in}}%
\pgfpathlineto{\pgfqpoint{0.972843in}{1.391874in}}%
\pgfpathlineto{\pgfqpoint{0.974286in}{1.392924in}}%
\pgfpathlineto{\pgfqpoint{0.975197in}{1.393554in}}%
\pgfpathlineto{\pgfqpoint{0.977336in}{1.394604in}}%
\pgfpathlineto{\pgfqpoint{0.978413in}{1.395654in}}%
\pgfpathlineto{\pgfqpoint{0.979607in}{1.396704in}}%
\pgfpathlineto{\pgfqpoint{0.980635in}{1.397614in}}%
\pgfpathlineto{\pgfqpoint{0.981729in}{1.398664in}}%
\pgfpathlineto{\pgfqpoint{0.982690in}{1.399224in}}%
\pgfpathlineto{\pgfqpoint{0.984365in}{1.400274in}}%
\pgfpathlineto{\pgfqpoint{0.985343in}{1.401044in}}%
\pgfpathlineto{\pgfqpoint{0.987050in}{1.402094in}}%
\pgfpathlineto{\pgfqpoint{0.987912in}{1.403004in}}%
\pgfpathlineto{\pgfqpoint{0.990051in}{1.404054in}}%
\pgfpathlineto{\pgfqpoint{0.990631in}{1.404474in}}%
\pgfpathlineto{\pgfqpoint{0.992935in}{1.405524in}}%
\pgfpathlineto{\pgfqpoint{0.993847in}{1.406294in}}%
\pgfpathlineto{\pgfqpoint{0.995787in}{1.407344in}}%
\pgfpathlineto{\pgfqpoint{0.996798in}{1.408534in}}%
\pgfpathlineto{\pgfqpoint{0.998456in}{1.409444in}}%
\pgfpathlineto{\pgfqpoint{0.999533in}{1.410424in}}%
\pgfpathlineto{\pgfqpoint{1.001224in}{1.411404in}}%
\pgfpathlineto{\pgfqpoint{1.002235in}{1.412314in}}%
\pgfpathlineto{\pgfqpoint{1.003545in}{1.413364in}}%
\pgfpathlineto{\pgfqpoint{1.004423in}{1.414274in}}%
\pgfpathlineto{\pgfqpoint{1.005882in}{1.415324in}}%
\pgfpathlineto{\pgfqpoint{1.006678in}{1.415954in}}%
\pgfpathlineto{\pgfqpoint{1.009165in}{1.417004in}}%
\pgfpathlineto{\pgfqpoint{1.010226in}{1.417774in}}%
\pgfpathlineto{\pgfqpoint{1.012861in}{1.418824in}}%
\pgfpathlineto{\pgfqpoint{1.013972in}{1.419524in}}%
\pgfpathlineto{\pgfqpoint{1.016442in}{1.420504in}}%
\pgfpathlineto{\pgfqpoint{1.017437in}{1.421274in}}%
\pgfpathlineto{\pgfqpoint{1.019890in}{1.422324in}}%
\pgfpathlineto{\pgfqpoint{1.020852in}{1.423444in}}%
\pgfpathlineto{\pgfqpoint{1.022626in}{1.424494in}}%
\pgfpathlineto{\pgfqpoint{1.023488in}{1.425334in}}%
\pgfpathlineto{\pgfqpoint{1.026405in}{1.426384in}}%
\pgfpathlineto{\pgfqpoint{1.027516in}{1.427224in}}%
\pgfpathlineto{\pgfqpoint{1.029787in}{1.428274in}}%
\pgfpathlineto{\pgfqpoint{1.030831in}{1.429184in}}%
\pgfpathlineto{\pgfqpoint{1.032887in}{1.430234in}}%
\pgfpathlineto{\pgfqpoint{1.033882in}{1.431074in}}%
\pgfpathlineto{\pgfqpoint{1.036667in}{1.432124in}}%
\pgfpathlineto{\pgfqpoint{1.037711in}{1.432894in}}%
\pgfpathlineto{\pgfqpoint{1.041872in}{1.433944in}}%
\pgfpathlineto{\pgfqpoint{1.042834in}{1.434504in}}%
\pgfpathlineto{\pgfqpoint{1.046530in}{1.435554in}}%
\pgfpathlineto{\pgfqpoint{1.047575in}{1.436044in}}%
\pgfpathlineto{\pgfqpoint{1.049514in}{1.437094in}}%
\pgfpathlineto{\pgfqpoint{1.050625in}{1.437794in}}%
\pgfpathlineto{\pgfqpoint{1.053095in}{1.438844in}}%
\pgfpathlineto{\pgfqpoint{1.054090in}{1.439544in}}%
\pgfpathlineto{\pgfqpoint{1.056692in}{1.440594in}}%
\pgfpathlineto{\pgfqpoint{1.057704in}{1.441364in}}%
\pgfpathlineto{\pgfqpoint{1.061301in}{1.442414in}}%
\pgfpathlineto{\pgfqpoint{1.062329in}{1.443044in}}%
\pgfpathlineto{\pgfqpoint{1.065893in}{1.444094in}}%
\pgfpathlineto{\pgfqpoint{1.066987in}{1.444584in}}%
\pgfpathlineto{\pgfqpoint{1.068976in}{1.445634in}}%
\pgfpathlineto{\pgfqpoint{1.070037in}{1.446544in}}%
\pgfpathlineto{\pgfqpoint{1.072491in}{1.447594in}}%
\pgfpathlineto{\pgfqpoint{1.073369in}{1.448224in}}%
\pgfpathlineto{\pgfqpoint{1.075840in}{1.449274in}}%
\pgfpathlineto{\pgfqpoint{1.076652in}{1.449694in}}%
\pgfpathlineto{\pgfqpoint{1.079006in}{1.450744in}}%
\pgfpathlineto{\pgfqpoint{1.079967in}{1.451234in}}%
\pgfpathlineto{\pgfqpoint{1.083001in}{1.452214in}}%
\pgfpathlineto{\pgfqpoint{1.084112in}{1.452914in}}%
\pgfpathlineto{\pgfqpoint{1.087726in}{1.453894in}}%
\pgfpathlineto{\pgfqpoint{1.088438in}{1.454384in}}%
\pgfpathlineto{\pgfqpoint{1.092119in}{1.455434in}}%
\pgfpathlineto{\pgfqpoint{1.093064in}{1.456064in}}%
\pgfpathlineto{\pgfqpoint{1.096346in}{1.457114in}}%
\pgfpathlineto{\pgfqpoint{1.097274in}{1.457674in}}%
\pgfpathlineto{\pgfqpoint{1.100092in}{1.458724in}}%
\pgfpathlineto{\pgfqpoint{1.100689in}{1.459004in}}%
\pgfpathlineto{\pgfqpoint{1.104038in}{1.460054in}}%
\pgfpathlineto{\pgfqpoint{1.105149in}{1.460404in}}%
\pgfpathlineto{\pgfqpoint{1.109989in}{1.461454in}}%
\pgfpathlineto{\pgfqpoint{1.111083in}{1.462154in}}%
\pgfpathlineto{\pgfqpoint{1.114813in}{1.463204in}}%
\pgfpathlineto{\pgfqpoint{1.115858in}{1.463624in}}%
\pgfpathlineto{\pgfqpoint{1.119389in}{1.464674in}}%
\pgfpathlineto{\pgfqpoint{1.120433in}{1.465094in}}%
\pgfpathlineto{\pgfqpoint{1.122754in}{1.466144in}}%
\pgfpathlineto{\pgfqpoint{1.123732in}{1.466494in}}%
\pgfpathlineto{\pgfqpoint{1.127910in}{1.467544in}}%
\pgfpathlineto{\pgfqpoint{1.128556in}{1.468034in}}%
\pgfpathlineto{\pgfqpoint{1.131010in}{1.469084in}}%
\pgfpathlineto{\pgfqpoint{1.131822in}{1.469924in}}%
\pgfpathlineto{\pgfqpoint{1.135187in}{1.470974in}}%
\pgfpathlineto{\pgfqpoint{1.135204in}{1.471184in}}%
\pgfpathlineto{\pgfqpoint{1.140956in}{1.472234in}}%
\pgfpathlineto{\pgfqpoint{1.141735in}{1.472584in}}%
\pgfpathlineto{\pgfqpoint{1.145598in}{1.473634in}}%
\pgfpathlineto{\pgfqpoint{1.146675in}{1.474194in}}%
\pgfpathlineto{\pgfqpoint{1.150803in}{1.475244in}}%
\pgfpathlineto{\pgfqpoint{1.151516in}{1.475874in}}%
\pgfpathlineto{\pgfqpoint{1.156058in}{1.476924in}}%
\pgfpathlineto{\pgfqpoint{1.157152in}{1.477414in}}%
\pgfpathlineto{\pgfqpoint{1.162407in}{1.478464in}}%
\pgfpathlineto{\pgfqpoint{1.163452in}{1.479094in}}%
\pgfpathlineto{\pgfqpoint{1.168060in}{1.480144in}}%
\pgfpathlineto{\pgfqpoint{1.168873in}{1.480494in}}%
\pgfpathlineto{\pgfqpoint{1.171923in}{1.481474in}}%
\pgfpathlineto{\pgfqpoint{1.172835in}{1.481754in}}%
\pgfpathlineto{\pgfqpoint{1.177161in}{1.482804in}}%
\pgfpathlineto{\pgfqpoint{1.178156in}{1.483294in}}%
\pgfpathlineto{\pgfqpoint{1.181223in}{1.484344in}}%
\pgfpathlineto{\pgfqpoint{1.181273in}{1.484554in}}%
\pgfpathlineto{\pgfqpoint{1.188633in}{1.485604in}}%
\pgfpathlineto{\pgfqpoint{1.189512in}{1.485884in}}%
\pgfpathlineto{\pgfqpoint{1.193341in}{1.486934in}}%
\pgfpathlineto{\pgfqpoint{1.194203in}{1.487354in}}%
\pgfpathlineto{\pgfqpoint{1.198928in}{1.488404in}}%
\pgfpathlineto{\pgfqpoint{1.199823in}{1.488964in}}%
\pgfpathlineto{\pgfqpoint{1.201895in}{1.489944in}}%
\pgfpathlineto{\pgfqpoint{1.202343in}{1.490224in}}%
\pgfpathlineto{\pgfqpoint{1.205343in}{1.491274in}}%
\pgfpathlineto{\pgfqpoint{1.206404in}{1.491554in}}%
\pgfpathlineto{\pgfqpoint{1.209670in}{1.492604in}}%
\pgfpathlineto{\pgfqpoint{1.210748in}{1.493024in}}%
\pgfpathlineto{\pgfqpoint{1.215887in}{1.494074in}}%
\pgfpathlineto{\pgfqpoint{1.216865in}{1.494214in}}%
\pgfpathlineto{\pgfqpoint{1.223860in}{1.495264in}}%
\pgfpathlineto{\pgfqpoint{1.224739in}{1.495684in}}%
\pgfpathlineto{\pgfqpoint{1.229994in}{1.496734in}}%
\pgfpathlineto{\pgfqpoint{1.230823in}{1.497014in}}%
\pgfpathlineto{\pgfqpoint{1.234354in}{1.498064in}}%
\pgfpathlineto{\pgfqpoint{1.235117in}{1.498344in}}%
\pgfpathlineto{\pgfqpoint{1.240256in}{1.499394in}}%
\pgfpathlineto{\pgfqpoint{1.241151in}{1.499674in}}%
\pgfpathlineto{\pgfqpoint{1.245610in}{1.500724in}}%
\pgfpathlineto{\pgfqpoint{1.246721in}{1.501144in}}%
\pgfpathlineto{\pgfqpoint{1.248942in}{1.502194in}}%
\pgfpathlineto{\pgfqpoint{1.250053in}{1.502754in}}%
\pgfpathlineto{\pgfqpoint{1.254429in}{1.503804in}}%
\pgfpathlineto{\pgfqpoint{1.254728in}{1.504084in}}%
\pgfpathlineto{\pgfqpoint{1.260845in}{1.505134in}}%
\pgfpathlineto{\pgfqpoint{1.261723in}{1.505414in}}%
\pgfpathlineto{\pgfqpoint{1.266697in}{1.506464in}}%
\pgfpathlineto{\pgfqpoint{1.267807in}{1.506744in}}%
\pgfpathlineto{\pgfqpoint{1.271338in}{1.507724in}}%
\pgfpathlineto{\pgfqpoint{1.272200in}{1.508214in}}%
\pgfpathlineto{\pgfqpoint{1.277936in}{1.509264in}}%
\pgfpathlineto{\pgfqpoint{1.278036in}{1.509404in}}%
\pgfpathlineto{\pgfqpoint{1.284203in}{1.510454in}}%
\pgfpathlineto{\pgfqpoint{1.285131in}{1.510804in}}%
\pgfpathlineto{\pgfqpoint{1.289657in}{1.511854in}}%
\pgfpathlineto{\pgfqpoint{1.290585in}{1.512274in}}%
\pgfpathlineto{\pgfqpoint{1.295940in}{1.513324in}}%
\pgfpathlineto{\pgfqpoint{1.296354in}{1.513534in}}%
\pgfpathlineto{\pgfqpoint{1.299703in}{1.514584in}}%
\pgfpathlineto{\pgfqpoint{1.300664in}{1.514934in}}%
\pgfpathlineto{\pgfqpoint{1.305389in}{1.515984in}}%
\pgfpathlineto{\pgfqpoint{1.306118in}{1.516334in}}%
\pgfpathlineto{\pgfqpoint{1.311290in}{1.517384in}}%
\pgfpathlineto{\pgfqpoint{1.312318in}{1.517734in}}%
\pgfpathlineto{\pgfqpoint{1.318104in}{1.518784in}}%
\pgfpathlineto{\pgfqpoint{1.318899in}{1.519274in}}%
\pgfpathlineto{\pgfqpoint{1.323873in}{1.520324in}}%
\pgfpathlineto{\pgfqpoint{1.324950in}{1.520674in}}%
\pgfpathlineto{\pgfqpoint{1.329575in}{1.521654in}}%
\pgfpathlineto{\pgfqpoint{1.330305in}{1.521864in}}%
\pgfpathlineto{\pgfqpoint{1.336521in}{1.522844in}}%
\pgfpathlineto{\pgfqpoint{1.336836in}{1.523194in}}%
\pgfpathlineto{\pgfqpoint{1.344180in}{1.524244in}}%
\pgfpathlineto{\pgfqpoint{1.345191in}{1.524524in}}%
\pgfpathlineto{\pgfqpoint{1.351905in}{1.525574in}}%
\pgfpathlineto{\pgfqpoint{1.352602in}{1.525994in}}%
\pgfpathlineto{\pgfqpoint{1.359050in}{1.527044in}}%
\pgfpathlineto{\pgfqpoint{1.359962in}{1.527394in}}%
\pgfpathlineto{\pgfqpoint{1.365897in}{1.528444in}}%
\pgfpathlineto{\pgfqpoint{1.366991in}{1.528654in}}%
\pgfpathlineto{\pgfqpoint{1.378463in}{1.529704in}}%
\pgfpathlineto{\pgfqpoint{1.379092in}{1.530054in}}%
\pgfpathlineto{\pgfqpoint{1.386221in}{1.531104in}}%
\pgfpathlineto{\pgfqpoint{1.387298in}{1.531314in}}%
\pgfpathlineto{\pgfqpoint{1.393366in}{1.532364in}}%
\pgfpathlineto{\pgfqpoint{1.393880in}{1.532714in}}%
\pgfpathlineto{\pgfqpoint{1.403760in}{1.533764in}}%
\pgfpathlineto{\pgfqpoint{1.404340in}{1.533904in}}%
\pgfpathlineto{\pgfqpoint{1.412529in}{1.534954in}}%
\pgfpathlineto{\pgfqpoint{1.412994in}{1.535164in}}%
\pgfpathlineto{\pgfqpoint{1.422277in}{1.536214in}}%
\pgfpathlineto{\pgfqpoint{1.422807in}{1.536354in}}%
\pgfpathlineto{\pgfqpoint{1.428460in}{1.537404in}}%
\pgfpathlineto{\pgfqpoint{1.429157in}{1.537614in}}%
\pgfpathlineto{\pgfqpoint{1.435340in}{1.538664in}}%
\pgfpathlineto{\pgfqpoint{1.436351in}{1.539014in}}%
\pgfpathlineto{\pgfqpoint{1.443861in}{1.540064in}}%
\pgfpathlineto{\pgfqpoint{1.443877in}{1.540274in}}%
\pgfpathlineto{\pgfqpoint{1.450940in}{1.541324in}}%
\pgfpathlineto{\pgfqpoint{1.451006in}{1.541464in}}%
\pgfpathlineto{\pgfqpoint{1.462776in}{1.542514in}}%
\pgfpathlineto{\pgfqpoint{1.463439in}{1.542724in}}%
\pgfpathlineto{\pgfqpoint{1.473817in}{1.543774in}}%
\pgfpathlineto{\pgfqpoint{1.474347in}{1.544124in}}%
\pgfpathlineto{\pgfqpoint{1.482304in}{1.545174in}}%
\pgfpathlineto{\pgfqpoint{1.482669in}{1.545314in}}%
\pgfpathlineto{\pgfqpoint{1.493892in}{1.546364in}}%
\pgfpathlineto{\pgfqpoint{1.494257in}{1.546504in}}%
\pgfpathlineto{\pgfqpoint{1.503756in}{1.547554in}}%
\pgfpathlineto{\pgfqpoint{1.503921in}{1.547834in}}%
\pgfpathlineto{\pgfqpoint{1.514382in}{1.548884in}}%
\pgfpathlineto{\pgfqpoint{1.515227in}{1.549234in}}%
\pgfpathlineto{\pgfqpoint{1.524046in}{1.550284in}}%
\pgfpathlineto{\pgfqpoint{1.524378in}{1.550564in}}%
\pgfpathlineto{\pgfqpoint{1.535187in}{1.551614in}}%
\pgfpathlineto{\pgfqpoint{1.535817in}{1.551754in}}%
\pgfpathlineto{\pgfqpoint{1.546012in}{1.552804in}}%
\pgfpathlineto{\pgfqpoint{1.547122in}{1.553154in}}%
\pgfpathlineto{\pgfqpoint{1.557450in}{1.554204in}}%
\pgfpathlineto{\pgfqpoint{1.557450in}{1.554274in}}%
\pgfpathlineto{\pgfqpoint{1.571690in}{1.555324in}}%
\pgfpathlineto{\pgfqpoint{1.571790in}{1.555534in}}%
\pgfpathlineto{\pgfqpoint{1.581405in}{1.556584in}}%
\pgfpathlineto{\pgfqpoint{1.581653in}{1.556794in}}%
\pgfpathlineto{\pgfqpoint{1.595396in}{1.557844in}}%
\pgfpathlineto{\pgfqpoint{1.595844in}{1.558054in}}%
\pgfpathlineto{\pgfqpoint{1.606503in}{1.559104in}}%
\pgfpathlineto{\pgfqpoint{1.607514in}{1.559384in}}%
\pgfpathlineto{\pgfqpoint{1.617610in}{1.560434in}}%
\pgfpathlineto{\pgfqpoint{1.618721in}{1.560644in}}%
\pgfpathlineto{\pgfqpoint{1.630889in}{1.561694in}}%
\pgfpathlineto{\pgfqpoint{1.631883in}{1.561834in}}%
\pgfpathlineto{\pgfqpoint{1.652605in}{1.562884in}}%
\pgfpathlineto{\pgfqpoint{1.653600in}{1.563024in}}%
\pgfpathlineto{\pgfqpoint{1.663812in}{1.564074in}}%
\pgfpathlineto{\pgfqpoint{1.663861in}{1.564214in}}%
\pgfpathlineto{\pgfqpoint{1.672283in}{1.565264in}}%
\pgfpathlineto{\pgfqpoint{1.672382in}{1.565404in}}%
\pgfpathlineto{\pgfqpoint{1.683406in}{1.566454in}}%
\pgfpathlineto{\pgfqpoint{1.684467in}{1.566664in}}%
\pgfpathlineto{\pgfqpoint{1.696586in}{1.567714in}}%
\pgfpathlineto{\pgfqpoint{1.696834in}{1.567854in}}%
\pgfpathlineto{\pgfqpoint{1.710212in}{1.568904in}}%
\pgfpathlineto{\pgfqpoint{1.710909in}{1.569114in}}%
\pgfpathlineto{\pgfqpoint{1.723706in}{1.570164in}}%
\pgfpathlineto{\pgfqpoint{1.724718in}{1.570304in}}%
\pgfpathlineto{\pgfqpoint{1.739090in}{1.571354in}}%
\pgfpathlineto{\pgfqpoint{1.739422in}{1.571494in}}%
\pgfpathlineto{\pgfqpoint{1.756828in}{1.572544in}}%
\pgfpathlineto{\pgfqpoint{1.757591in}{1.572754in}}%
\pgfpathlineto{\pgfqpoint{1.769825in}{1.573804in}}%
\pgfpathlineto{\pgfqpoint{1.770339in}{1.573944in}}%
\pgfpathlineto{\pgfqpoint{1.782242in}{1.574994in}}%
\pgfpathlineto{\pgfqpoint{1.782374in}{1.575134in}}%
\pgfpathlineto{\pgfqpoint{1.795288in}{1.576184in}}%
\pgfpathlineto{\pgfqpoint{1.795868in}{1.576394in}}%
\pgfpathlineto{\pgfqpoint{1.809910in}{1.577444in}}%
\pgfpathlineto{\pgfqpoint{1.810987in}{1.577724in}}%
\pgfpathlineto{\pgfqpoint{1.828360in}{1.578774in}}%
\pgfpathlineto{\pgfqpoint{1.829057in}{1.578984in}}%
\pgfpathlineto{\pgfqpoint{1.842236in}{1.580034in}}%
\pgfpathlineto{\pgfqpoint{1.842368in}{1.580174in}}%
\pgfpathlineto{\pgfqpoint{1.859410in}{1.581224in}}%
\pgfpathlineto{\pgfqpoint{1.859891in}{1.581364in}}%
\pgfpathlineto{\pgfqpoint{1.873683in}{1.582414in}}%
\pgfpathlineto{\pgfqpoint{1.873683in}{1.582484in}}%
\pgfpathlineto{\pgfqpoint{1.894190in}{1.583534in}}%
\pgfpathlineto{\pgfqpoint{1.894637in}{1.583674in}}%
\pgfpathlineto{\pgfqpoint{1.915542in}{1.584724in}}%
\pgfpathlineto{\pgfqpoint{1.916155in}{1.584864in}}%
\pgfpathlineto{\pgfqpoint{1.929334in}{1.585914in}}%
\pgfpathlineto{\pgfqpoint{1.930345in}{1.586054in}}%
\pgfpathlineto{\pgfqpoint{1.945680in}{1.587104in}}%
\pgfpathlineto{\pgfqpoint{1.945680in}{1.587174in}}%
\pgfpathlineto{\pgfqpoint{1.958245in}{1.588224in}}%
\pgfpathlineto{\pgfqpoint{1.959074in}{1.588434in}}%
\pgfpathlineto{\pgfqpoint{1.971541in}{1.589484in}}%
\pgfpathlineto{\pgfqpoint{1.972005in}{1.589624in}}%
\pgfpathlineto{\pgfqpoint{1.980807in}{1.590674in}}%
\pgfpathlineto{\pgfqpoint{1.981636in}{1.590954in}}%
\pgfpathlineto{\pgfqpoint{1.988947in}{1.592004in}}%
\pgfpathlineto{\pgfqpoint{1.988947in}{1.592074in}}%
\pgfpathlineto{\pgfqpoint{1.998313in}{1.593124in}}%
\pgfpathlineto{\pgfqpoint{1.999391in}{1.593334in}}%
\pgfpathlineto{\pgfqpoint{2.006536in}{1.594384in}}%
\pgfpathlineto{\pgfqpoint{2.007414in}{1.594524in}}%
\pgfpathlineto{\pgfqpoint{2.013349in}{1.595574in}}%
\pgfpathlineto{\pgfqpoint{2.014195in}{1.595854in}}%
\pgfpathlineto{\pgfqpoint{2.021936in}{1.596904in}}%
\pgfpathlineto{\pgfqpoint{2.022318in}{1.597044in}}%
\pgfpathlineto{\pgfqpoint{2.028136in}{1.598094in}}%
\pgfpathlineto{\pgfqpoint{2.029230in}{1.598584in}}%
\pgfpathlineto{\pgfqpoint{2.032015in}{1.599634in}}%
\pgfpathlineto{\pgfqpoint{2.033126in}{1.601944in}}%
\pgfpathlineto{\pgfqpoint{2.033126in}{1.601944in}}%
\pgfusepath{stroke}%
\end{pgfscope}%
\begin{pgfscope}%
\pgfsetrectcap%
\pgfsetmiterjoin%
\pgfsetlinewidth{0.803000pt}%
\definecolor{currentstroke}{rgb}{0.000000,0.000000,0.000000}%
\pgfsetstrokecolor{currentstroke}%
\pgfsetdash{}{0pt}%
\pgfpathmoveto{\pgfqpoint{0.553581in}{0.499444in}}%
\pgfpathlineto{\pgfqpoint{0.553581in}{1.654444in}}%
\pgfusepath{stroke}%
\end{pgfscope}%
\begin{pgfscope}%
\pgfsetrectcap%
\pgfsetmiterjoin%
\pgfsetlinewidth{0.803000pt}%
\definecolor{currentstroke}{rgb}{0.000000,0.000000,0.000000}%
\pgfsetstrokecolor{currentstroke}%
\pgfsetdash{}{0pt}%
\pgfpathmoveto{\pgfqpoint{2.103581in}{0.499444in}}%
\pgfpathlineto{\pgfqpoint{2.103581in}{1.654444in}}%
\pgfusepath{stroke}%
\end{pgfscope}%
\begin{pgfscope}%
\pgfsetrectcap%
\pgfsetmiterjoin%
\pgfsetlinewidth{0.803000pt}%
\definecolor{currentstroke}{rgb}{0.000000,0.000000,0.000000}%
\pgfsetstrokecolor{currentstroke}%
\pgfsetdash{}{0pt}%
\pgfpathmoveto{\pgfqpoint{0.553581in}{0.499444in}}%
\pgfpathlineto{\pgfqpoint{2.103581in}{0.499444in}}%
\pgfusepath{stroke}%
\end{pgfscope}%
\begin{pgfscope}%
\pgfsetrectcap%
\pgfsetmiterjoin%
\pgfsetlinewidth{0.803000pt}%
\definecolor{currentstroke}{rgb}{0.000000,0.000000,0.000000}%
\pgfsetstrokecolor{currentstroke}%
\pgfsetdash{}{0pt}%
\pgfpathmoveto{\pgfqpoint{0.553581in}{1.654444in}}%
\pgfpathlineto{\pgfqpoint{2.103581in}{1.654444in}}%
\pgfusepath{stroke}%
\end{pgfscope}%
\begin{pgfscope}%
\pgfsetbuttcap%
\pgfsetmiterjoin%
\definecolor{currentfill}{rgb}{1.000000,1.000000,1.000000}%
\pgfsetfillcolor{currentfill}%
\pgfsetfillopacity{0.800000}%
\pgfsetlinewidth{1.003750pt}%
\definecolor{currentstroke}{rgb}{0.800000,0.800000,0.800000}%
\pgfsetstrokecolor{currentstroke}%
\pgfsetstrokeopacity{0.800000}%
\pgfsetdash{}{0pt}%
\pgfpathmoveto{\pgfqpoint{0.832747in}{0.568889in}}%
\pgfpathlineto{\pgfqpoint{2.006358in}{0.568889in}}%
\pgfpathquadraticcurveto{\pgfqpoint{2.034136in}{0.568889in}}{\pgfqpoint{2.034136in}{0.596666in}}%
\pgfpathlineto{\pgfqpoint{2.034136in}{0.776388in}}%
\pgfpathquadraticcurveto{\pgfqpoint{2.034136in}{0.804166in}}{\pgfqpoint{2.006358in}{0.804166in}}%
\pgfpathlineto{\pgfqpoint{0.832747in}{0.804166in}}%
\pgfpathquadraticcurveto{\pgfqpoint{0.804970in}{0.804166in}}{\pgfqpoint{0.804970in}{0.776388in}}%
\pgfpathlineto{\pgfqpoint{0.804970in}{0.596666in}}%
\pgfpathquadraticcurveto{\pgfqpoint{0.804970in}{0.568889in}}{\pgfqpoint{0.832747in}{0.568889in}}%
\pgfpathlineto{\pgfqpoint{0.832747in}{0.568889in}}%
\pgfpathclose%
\pgfusepath{stroke,fill}%
\end{pgfscope}%
\begin{pgfscope}%
\pgfsetrectcap%
\pgfsetroundjoin%
\pgfsetlinewidth{1.505625pt}%
\definecolor{currentstroke}{rgb}{0.000000,0.000000,0.000000}%
\pgfsetstrokecolor{currentstroke}%
\pgfsetdash{}{0pt}%
\pgfpathmoveto{\pgfqpoint{0.860525in}{0.700000in}}%
\pgfpathlineto{\pgfqpoint{0.999414in}{0.700000in}}%
\pgfpathlineto{\pgfqpoint{1.138303in}{0.700000in}}%
\pgfusepath{stroke}%
\end{pgfscope}%
\begin{pgfscope}%
\definecolor{textcolor}{rgb}{0.000000,0.000000,0.000000}%
\pgfsetstrokecolor{textcolor}%
\pgfsetfillcolor{textcolor}%
\pgftext[x=1.249414in,y=0.651388in,left,base]{\color{textcolor}\rmfamily\fontsize{10.000000}{12.000000}\selectfont AUC=0.840}%
\end{pgfscope}%
\end{pgfpicture}%
\makeatother%
\endgroup%

  &
\vspace{0pt} 
  
\begin{tabular}{cc|c|c|}
	&\multicolumn{1}{c}{}& \multicolumn{2}{c}{Prediction} \cr
	&\multicolumn{1}{c}{} & \multicolumn{1}{c}{N} & \multicolumn{1}{c}{P} \cr\cline{3-4}
	\multirow{2}{*}{\rotatebox[origin=c]{90}{Actual}}&N & 117,929 & 32,842 \vrule width 0pt height 10pt depth 2pt \cr\cline{3-4}
	&P & 5,928 & 20,693 \vrule width 0pt height 10pt depth 2pt \cr\cline{3-4}
\end{tabular}

\begin{center}
\begin{tabular}{ll}
0.387 & Precision \cr 
0.777 & Recall \cr 
0.516 & F1 \cr 
\end{tabular}
\end{center}
  
\end{tabular}

Note that with threshold $p=0.5$ the false positives ($FP=32,842$) are less than twice as many as the true positives ($TP=20,693$), but that does not satisfy our requirement that $\Delta FP/\Delta TP < 2.0$.  The plot below shows the rate of change as a function of $p$, and that $\Delta FP/\Delta TP = 2.0$ when $p=0.635$.  

%% Creator: Matplotlib, PGF backend
%%
%% To include the figure in your LaTeX document, write
%%   \input{<filename>.pgf}
%%
%% Make sure the required packages are loaded in your preamble
%%   \usepackage{pgf}
%%
%% Also ensure that all the required font packages are loaded; for instance,
%% the lmodern package is sometimes necessary when using math font.
%%   \usepackage{lmodern}
%%
%% Figures using additional raster images can only be included by \input if
%% they are in the same directory as the main LaTeX file. For loading figures
%% from other directories you can use the `import` package
%%   \usepackage{import}
%%
%% and then include the figures with
%%   \import{<path to file>}{<filename>.pgf}
%%
%% Matplotlib used the following preamble
%%   
%%   \usepackage{fontspec}
%%   \makeatletter\@ifpackageloaded{underscore}{}{\usepackage[strings]{underscore}}\makeatother
%%
\begingroup%
\makeatletter%
\begin{pgfpicture}%
\pgfpathrectangle{\pgfpointorigin}{\pgfqpoint{2.282529in}{1.754444in}}%
\pgfusepath{use as bounding box, clip}%
\begin{pgfscope}%
\pgfsetbuttcap%
\pgfsetmiterjoin%
\definecolor{currentfill}{rgb}{1.000000,1.000000,1.000000}%
\pgfsetfillcolor{currentfill}%
\pgfsetlinewidth{0.000000pt}%
\definecolor{currentstroke}{rgb}{1.000000,1.000000,1.000000}%
\pgfsetstrokecolor{currentstroke}%
\pgfsetdash{}{0pt}%
\pgfpathmoveto{\pgfqpoint{0.000000in}{0.000000in}}%
\pgfpathlineto{\pgfqpoint{2.282529in}{0.000000in}}%
\pgfpathlineto{\pgfqpoint{2.282529in}{1.754444in}}%
\pgfpathlineto{\pgfqpoint{0.000000in}{1.754444in}}%
\pgfpathlineto{\pgfqpoint{0.000000in}{0.000000in}}%
\pgfpathclose%
\pgfusepath{fill}%
\end{pgfscope}%
\begin{pgfscope}%
\pgfsetbuttcap%
\pgfsetmiterjoin%
\definecolor{currentfill}{rgb}{1.000000,1.000000,1.000000}%
\pgfsetfillcolor{currentfill}%
\pgfsetlinewidth{0.000000pt}%
\definecolor{currentstroke}{rgb}{0.000000,0.000000,0.000000}%
\pgfsetstrokecolor{currentstroke}%
\pgfsetstrokeopacity{0.000000}%
\pgfsetdash{}{0pt}%
\pgfpathmoveto{\pgfqpoint{0.530556in}{0.499444in}}%
\pgfpathlineto{\pgfqpoint{2.080556in}{0.499444in}}%
\pgfpathlineto{\pgfqpoint{2.080556in}{1.654444in}}%
\pgfpathlineto{\pgfqpoint{0.530556in}{1.654444in}}%
\pgfpathlineto{\pgfqpoint{0.530556in}{0.499444in}}%
\pgfpathclose%
\pgfusepath{fill}%
\end{pgfscope}%
\begin{pgfscope}%
\pgfsetbuttcap%
\pgfsetroundjoin%
\definecolor{currentfill}{rgb}{0.000000,0.000000,0.000000}%
\pgfsetfillcolor{currentfill}%
\pgfsetlinewidth{0.803000pt}%
\definecolor{currentstroke}{rgb}{0.000000,0.000000,0.000000}%
\pgfsetstrokecolor{currentstroke}%
\pgfsetdash{}{0pt}%
\pgfsys@defobject{currentmarker}{\pgfqpoint{0.000000in}{-0.048611in}}{\pgfqpoint{0.000000in}{0.000000in}}{%
\pgfpathmoveto{\pgfqpoint{0.000000in}{0.000000in}}%
\pgfpathlineto{\pgfqpoint{0.000000in}{-0.048611in}}%
\pgfusepath{stroke,fill}%
}%
\begin{pgfscope}%
\pgfsys@transformshift{0.601010in}{0.499444in}%
\pgfsys@useobject{currentmarker}{}%
\end{pgfscope}%
\end{pgfscope}%
\begin{pgfscope}%
\definecolor{textcolor}{rgb}{0.000000,0.000000,0.000000}%
\pgfsetstrokecolor{textcolor}%
\pgfsetfillcolor{textcolor}%
\pgftext[x=0.601010in,y=0.402222in,,top]{\color{textcolor}\rmfamily\fontsize{10.000000}{12.000000}\selectfont 0.056}%
\end{pgfscope}%
\begin{pgfscope}%
\pgfsetbuttcap%
\pgfsetroundjoin%
\definecolor{currentfill}{rgb}{0.000000,0.000000,0.000000}%
\pgfsetfillcolor{currentfill}%
\pgfsetlinewidth{0.803000pt}%
\definecolor{currentstroke}{rgb}{0.000000,0.000000,0.000000}%
\pgfsetstrokecolor{currentstroke}%
\pgfsetdash{}{0pt}%
\pgfsys@defobject{currentmarker}{\pgfqpoint{0.000000in}{-0.048611in}}{\pgfqpoint{0.000000in}{0.000000in}}{%
\pgfpathmoveto{\pgfqpoint{0.000000in}{0.000000in}}%
\pgfpathlineto{\pgfqpoint{0.000000in}{-0.048611in}}%
\pgfusepath{stroke,fill}%
}%
\begin{pgfscope}%
\pgfsys@transformshift{2.024334in}{0.499444in}%
\pgfsys@useobject{currentmarker}{}%
\end{pgfscope}%
\end{pgfscope}%
\begin{pgfscope}%
\definecolor{textcolor}{rgb}{0.000000,0.000000,0.000000}%
\pgfsetstrokecolor{textcolor}%
\pgfsetfillcolor{textcolor}%
\pgftext[x=2.024334in,y=0.402222in,,top]{\color{textcolor}\rmfamily\fontsize{10.000000}{12.000000}\selectfont 0.975}%
\end{pgfscope}%
\begin{pgfscope}%
\definecolor{textcolor}{rgb}{0.000000,0.000000,0.000000}%
\pgfsetstrokecolor{textcolor}%
\pgfsetfillcolor{textcolor}%
\pgftext[x=1.305556in,y=0.223333in,,top]{\color{textcolor}\rmfamily\fontsize{10.000000}{12.000000}\selectfont \(\displaystyle p\)}%
\end{pgfscope}%
\begin{pgfscope}%
\pgfsetbuttcap%
\pgfsetroundjoin%
\definecolor{currentfill}{rgb}{0.000000,0.000000,0.000000}%
\pgfsetfillcolor{currentfill}%
\pgfsetlinewidth{0.803000pt}%
\definecolor{currentstroke}{rgb}{0.000000,0.000000,0.000000}%
\pgfsetstrokecolor{currentstroke}%
\pgfsetdash{}{0pt}%
\pgfsys@defobject{currentmarker}{\pgfqpoint{-0.048611in}{0.000000in}}{\pgfqpoint{-0.000000in}{0.000000in}}{%
\pgfpathmoveto{\pgfqpoint{-0.000000in}{0.000000in}}%
\pgfpathlineto{\pgfqpoint{-0.048611in}{0.000000in}}%
\pgfusepath{stroke,fill}%
}%
\begin{pgfscope}%
\pgfsys@transformshift{0.530556in}{0.537950in}%
\pgfsys@useobject{currentmarker}{}%
\end{pgfscope}%
\end{pgfscope}%
\begin{pgfscope}%
\definecolor{textcolor}{rgb}{0.000000,0.000000,0.000000}%
\pgfsetstrokecolor{textcolor}%
\pgfsetfillcolor{textcolor}%
\pgftext[x=0.363889in, y=0.489756in, left, base]{\color{textcolor}\rmfamily\fontsize{10.000000}{12.000000}\selectfont \(\displaystyle {0}\)}%
\end{pgfscope}%
\begin{pgfscope}%
\pgfsetbuttcap%
\pgfsetroundjoin%
\definecolor{currentfill}{rgb}{0.000000,0.000000,0.000000}%
\pgfsetfillcolor{currentfill}%
\pgfsetlinewidth{0.803000pt}%
\definecolor{currentstroke}{rgb}{0.000000,0.000000,0.000000}%
\pgfsetstrokecolor{currentstroke}%
\pgfsetdash{}{0pt}%
\pgfsys@defobject{currentmarker}{\pgfqpoint{-0.048611in}{0.000000in}}{\pgfqpoint{-0.000000in}{0.000000in}}{%
\pgfpathmoveto{\pgfqpoint{-0.000000in}{0.000000in}}%
\pgfpathlineto{\pgfqpoint{-0.048611in}{0.000000in}}%
\pgfusepath{stroke,fill}%
}%
\begin{pgfscope}%
\pgfsys@transformshift{0.530556in}{0.984008in}%
\pgfsys@useobject{currentmarker}{}%
\end{pgfscope}%
\end{pgfscope}%
\begin{pgfscope}%
\definecolor{textcolor}{rgb}{0.000000,0.000000,0.000000}%
\pgfsetstrokecolor{textcolor}%
\pgfsetfillcolor{textcolor}%
\pgftext[x=0.294444in, y=0.935813in, left, base]{\color{textcolor}\rmfamily\fontsize{10.000000}{12.000000}\selectfont \(\displaystyle {20}\)}%
\end{pgfscope}%
\begin{pgfscope}%
\pgfsetbuttcap%
\pgfsetroundjoin%
\definecolor{currentfill}{rgb}{0.000000,0.000000,0.000000}%
\pgfsetfillcolor{currentfill}%
\pgfsetlinewidth{0.803000pt}%
\definecolor{currentstroke}{rgb}{0.000000,0.000000,0.000000}%
\pgfsetstrokecolor{currentstroke}%
\pgfsetdash{}{0pt}%
\pgfsys@defobject{currentmarker}{\pgfqpoint{-0.048611in}{0.000000in}}{\pgfqpoint{-0.000000in}{0.000000in}}{%
\pgfpathmoveto{\pgfqpoint{-0.000000in}{0.000000in}}%
\pgfpathlineto{\pgfqpoint{-0.048611in}{0.000000in}}%
\pgfusepath{stroke,fill}%
}%
\begin{pgfscope}%
\pgfsys@transformshift{0.530556in}{1.430065in}%
\pgfsys@useobject{currentmarker}{}%
\end{pgfscope}%
\end{pgfscope}%
\begin{pgfscope}%
\definecolor{textcolor}{rgb}{0.000000,0.000000,0.000000}%
\pgfsetstrokecolor{textcolor}%
\pgfsetfillcolor{textcolor}%
\pgftext[x=0.294444in, y=1.381870in, left, base]{\color{textcolor}\rmfamily\fontsize{10.000000}{12.000000}\selectfont \(\displaystyle {40}\)}%
\end{pgfscope}%
\begin{pgfscope}%
\definecolor{textcolor}{rgb}{0.000000,0.000000,0.000000}%
\pgfsetstrokecolor{textcolor}%
\pgfsetfillcolor{textcolor}%
\pgftext[x=0.238889in,y=1.076944in,,bottom,rotate=90.000000]{\color{textcolor}\rmfamily\fontsize{10.000000}{12.000000}\selectfont \(\displaystyle \Delta\)FP/\(\displaystyle \Delta\)TP}%
\end{pgfscope}%
\begin{pgfscope}%
\pgfpathrectangle{\pgfqpoint{0.530556in}{0.499444in}}{\pgfqpoint{1.550000in}{1.155000in}}%
\pgfusepath{clip}%
\pgfsetrectcap%
\pgfsetroundjoin%
\pgfsetlinewidth{1.505625pt}%
\definecolor{currentstroke}{rgb}{0.000000,0.000000,0.000000}%
\pgfsetstrokecolor{currentstroke}%
\pgfsetdash{}{0pt}%
\pgfpathmoveto{\pgfqpoint{0.601010in}{1.309126in}}%
\pgfpathlineto{\pgfqpoint{0.615243in}{1.325979in}}%
\pgfpathlineto{\pgfqpoint{0.629477in}{1.348159in}}%
\pgfpathlineto{\pgfqpoint{0.643710in}{1.376276in}}%
\pgfpathlineto{\pgfqpoint{0.657943in}{1.396598in}}%
\pgfpathlineto{\pgfqpoint{0.672176in}{1.416080in}}%
\pgfpathlineto{\pgfqpoint{0.686410in}{1.451598in}}%
\pgfpathlineto{\pgfqpoint{0.700643in}{1.489051in}}%
\pgfpathlineto{\pgfqpoint{0.714876in}{1.523196in}}%
\pgfpathlineto{\pgfqpoint{0.729109in}{1.550619in}}%
\pgfpathlineto{\pgfqpoint{0.743343in}{1.573297in}}%
\pgfpathlineto{\pgfqpoint{0.757576in}{1.593495in}}%
\pgfpathlineto{\pgfqpoint{0.771809in}{1.601944in}}%
\pgfpathlineto{\pgfqpoint{0.786042in}{1.597356in}}%
\pgfpathlineto{\pgfqpoint{0.800276in}{1.594598in}}%
\pgfpathlineto{\pgfqpoint{0.814509in}{1.585548in}}%
\pgfpathlineto{\pgfqpoint{0.828742in}{1.572044in}}%
\pgfpathlineto{\pgfqpoint{0.842975in}{1.549833in}}%
\pgfpathlineto{\pgfqpoint{0.857209in}{1.524274in}}%
\pgfpathlineto{\pgfqpoint{0.871442in}{1.494966in}}%
\pgfpathlineto{\pgfqpoint{0.885675in}{1.460560in}}%
\pgfpathlineto{\pgfqpoint{0.899908in}{1.422997in}}%
\pgfpathlineto{\pgfqpoint{0.914142in}{1.386215in}}%
\pgfpathlineto{\pgfqpoint{0.928375in}{1.349615in}}%
\pgfpathlineto{\pgfqpoint{0.942608in}{1.307567in}}%
\pgfpathlineto{\pgfqpoint{0.956841in}{1.265269in}}%
\pgfpathlineto{\pgfqpoint{0.971075in}{1.223987in}}%
\pgfpathlineto{\pgfqpoint{0.985308in}{1.184457in}}%
\pgfpathlineto{\pgfqpoint{0.999541in}{1.144093in}}%
\pgfpathlineto{\pgfqpoint{1.013774in}{1.104850in}}%
\pgfpathlineto{\pgfqpoint{1.028007in}{1.068258in}}%
\pgfpathlineto{\pgfqpoint{1.042241in}{1.033296in}}%
\pgfpathlineto{\pgfqpoint{1.056474in}{1.000090in}}%
\pgfpathlineto{\pgfqpoint{1.070707in}{0.968203in}}%
\pgfpathlineto{\pgfqpoint{1.084940in}{0.939028in}}%
\pgfpathlineto{\pgfqpoint{1.099174in}{0.910854in}}%
\pgfpathlineto{\pgfqpoint{1.113407in}{0.884124in}}%
\pgfpathlineto{\pgfqpoint{1.127640in}{0.858671in}}%
\pgfpathlineto{\pgfqpoint{1.141873in}{0.836199in}}%
\pgfpathlineto{\pgfqpoint{1.156107in}{0.815216in}}%
\pgfpathlineto{\pgfqpoint{1.170340in}{0.795301in}}%
\pgfpathlineto{\pgfqpoint{1.184573in}{0.777165in}}%
\pgfpathlineto{\pgfqpoint{1.198806in}{0.759619in}}%
\pgfpathlineto{\pgfqpoint{1.213040in}{0.743148in}}%
\pgfpathlineto{\pgfqpoint{1.227273in}{0.727832in}}%
\pgfpathlineto{\pgfqpoint{1.241506in}{0.713713in}}%
\pgfpathlineto{\pgfqpoint{1.255739in}{0.700073in}}%
\pgfpathlineto{\pgfqpoint{1.269973in}{0.687561in}}%
\pgfpathlineto{\pgfqpoint{1.284206in}{0.675523in}}%
\pgfpathlineto{\pgfqpoint{1.298439in}{0.664347in}}%
\pgfpathlineto{\pgfqpoint{1.312672in}{0.654046in}}%
\pgfpathlineto{\pgfqpoint{1.326906in}{0.644411in}}%
\pgfpathlineto{\pgfqpoint{1.341139in}{0.635707in}}%
\pgfpathlineto{\pgfqpoint{1.355372in}{0.627761in}}%
\pgfpathlineto{\pgfqpoint{1.369605in}{0.620505in}}%
\pgfpathlineto{\pgfqpoint{1.383839in}{0.614557in}}%
\pgfpathlineto{\pgfqpoint{1.398072in}{0.609201in}}%
\pgfpathlineto{\pgfqpoint{1.412305in}{0.604257in}}%
\pgfpathlineto{\pgfqpoint{1.426538in}{0.599821in}}%
\pgfpathlineto{\pgfqpoint{1.440771in}{0.595838in}}%
\pgfpathlineto{\pgfqpoint{1.455005in}{0.592236in}}%
\pgfpathlineto{\pgfqpoint{1.469238in}{0.588808in}}%
\pgfpathlineto{\pgfqpoint{1.483471in}{0.585768in}}%
\pgfpathlineto{\pgfqpoint{1.497704in}{0.583038in}}%
\pgfpathlineto{\pgfqpoint{1.511938in}{0.580589in}}%
\pgfpathlineto{\pgfqpoint{1.526171in}{0.577323in}}%
\pgfpathlineto{\pgfqpoint{1.540404in}{0.574324in}}%
\pgfpathlineto{\pgfqpoint{1.554637in}{0.571569in}}%
\pgfpathlineto{\pgfqpoint{1.568871in}{0.569004in}}%
\pgfpathlineto{\pgfqpoint{1.583104in}{0.566653in}}%
\pgfpathlineto{\pgfqpoint{1.597337in}{0.564480in}}%
\pgfpathlineto{\pgfqpoint{1.611570in}{0.562560in}}%
\pgfpathlineto{\pgfqpoint{1.625804in}{0.560743in}}%
\pgfpathlineto{\pgfqpoint{1.640037in}{0.559049in}}%
\pgfpathlineto{\pgfqpoint{1.654270in}{0.557456in}}%
\pgfpathlineto{\pgfqpoint{1.668503in}{0.556420in}}%
\pgfpathlineto{\pgfqpoint{1.682737in}{0.555511in}}%
\pgfpathlineto{\pgfqpoint{1.696970in}{0.554704in}}%
\pgfpathlineto{\pgfqpoint{1.711203in}{0.554034in}}%
\pgfpathlineto{\pgfqpoint{1.725436in}{0.553410in}}%
\pgfpathlineto{\pgfqpoint{1.739670in}{0.552865in}}%
\pgfpathlineto{\pgfqpoint{1.753903in}{0.552438in}}%
\pgfpathlineto{\pgfqpoint{1.768136in}{0.552137in}}%
\pgfpathlineto{\pgfqpoint{1.782369in}{0.551972in}}%
\pgfpathlineto{\pgfqpoint{1.796603in}{0.551944in}}%
\pgfpathlineto{\pgfqpoint{1.810836in}{0.552041in}}%
\pgfpathlineto{\pgfqpoint{1.825069in}{0.552256in}}%
\pgfpathlineto{\pgfqpoint{1.839302in}{0.552594in}}%
\pgfpathlineto{\pgfqpoint{1.853535in}{0.553095in}}%
\pgfpathlineto{\pgfqpoint{1.867769in}{0.553860in}}%
\pgfpathlineto{\pgfqpoint{1.882002in}{0.554878in}}%
\pgfpathlineto{\pgfqpoint{1.896235in}{0.555752in}}%
\pgfpathlineto{\pgfqpoint{1.910468in}{0.556713in}}%
\pgfpathlineto{\pgfqpoint{1.924702in}{0.557773in}}%
\pgfpathlineto{\pgfqpoint{1.938935in}{0.558845in}}%
\pgfpathlineto{\pgfqpoint{1.953168in}{0.560079in}}%
\pgfpathlineto{\pgfqpoint{1.967401in}{0.560718in}}%
\pgfpathlineto{\pgfqpoint{1.981635in}{0.561401in}}%
\pgfpathlineto{\pgfqpoint{1.995868in}{0.562054in}}%
\pgfpathlineto{\pgfqpoint{2.010101in}{0.562561in}}%
\pgfusepath{stroke}%
\end{pgfscope}%
\begin{pgfscope}%
\pgfpathrectangle{\pgfqpoint{0.530556in}{0.499444in}}{\pgfqpoint{1.550000in}{1.155000in}}%
\pgfusepath{clip}%
\pgfsetbuttcap%
\pgfsetroundjoin%
\pgfsetlinewidth{1.505625pt}%
\definecolor{currentstroke}{rgb}{0.000000,0.000000,0.000000}%
\pgfsetstrokecolor{currentstroke}%
\pgfsetdash{{5.550000pt}{2.400000pt}}{0.000000pt}%
\pgfpathmoveto{\pgfqpoint{0.530556in}{0.582556in}}%
\pgfpathlineto{\pgfqpoint{2.080556in}{0.582556in}}%
\pgfusepath{stroke}%
\end{pgfscope}%
\begin{pgfscope}%
\pgfpathrectangle{\pgfqpoint{0.530556in}{0.499444in}}{\pgfqpoint{1.550000in}{1.155000in}}%
\pgfusepath{clip}%
\pgfsetrectcap%
\pgfsetroundjoin%
\pgfsetlinewidth{1.505625pt}%
\definecolor{currentstroke}{rgb}{0.121569,0.466667,0.705882}%
\pgfsetstrokecolor{currentstroke}%
\pgfsetdash{}{0pt}%
\pgfpathmoveto{\pgfqpoint{1.497704in}{0.582556in}}%
\pgfusepath{stroke}%
\end{pgfscope}%
\begin{pgfscope}%
\pgfpathrectangle{\pgfqpoint{0.530556in}{0.499444in}}{\pgfqpoint{1.550000in}{1.155000in}}%
\pgfusepath{clip}%
\pgfsetbuttcap%
\pgfsetroundjoin%
\definecolor{currentfill}{rgb}{0.000000,0.000000,0.000000}%
\pgfsetfillcolor{currentfill}%
\pgfsetlinewidth{1.003750pt}%
\definecolor{currentstroke}{rgb}{0.000000,0.000000,0.000000}%
\pgfsetstrokecolor{currentstroke}%
\pgfsetdash{}{0pt}%
\pgfsys@defobject{currentmarker}{\pgfqpoint{-0.041667in}{-0.041667in}}{\pgfqpoint{0.041667in}{0.041667in}}{%
\pgfpathmoveto{\pgfqpoint{0.000000in}{-0.041667in}}%
\pgfpathcurveto{\pgfqpoint{0.011050in}{-0.041667in}}{\pgfqpoint{0.021649in}{-0.037276in}}{\pgfqpoint{0.029463in}{-0.029463in}}%
\pgfpathcurveto{\pgfqpoint{0.037276in}{-0.021649in}}{\pgfqpoint{0.041667in}{-0.011050in}}{\pgfqpoint{0.041667in}{0.000000in}}%
\pgfpathcurveto{\pgfqpoint{0.041667in}{0.011050in}}{\pgfqpoint{0.037276in}{0.021649in}}{\pgfqpoint{0.029463in}{0.029463in}}%
\pgfpathcurveto{\pgfqpoint{0.021649in}{0.037276in}}{\pgfqpoint{0.011050in}{0.041667in}}{\pgfqpoint{0.000000in}{0.041667in}}%
\pgfpathcurveto{\pgfqpoint{-0.011050in}{0.041667in}}{\pgfqpoint{-0.021649in}{0.037276in}}{\pgfqpoint{-0.029463in}{0.029463in}}%
\pgfpathcurveto{\pgfqpoint{-0.037276in}{0.021649in}}{\pgfqpoint{-0.041667in}{0.011050in}}{\pgfqpoint{-0.041667in}{0.000000in}}%
\pgfpathcurveto{\pgfqpoint{-0.041667in}{-0.011050in}}{\pgfqpoint{-0.037276in}{-0.021649in}}{\pgfqpoint{-0.029463in}{-0.029463in}}%
\pgfpathcurveto{\pgfqpoint{-0.021649in}{-0.037276in}}{\pgfqpoint{-0.011050in}{-0.041667in}}{\pgfqpoint{0.000000in}{-0.041667in}}%
\pgfpathlineto{\pgfqpoint{0.000000in}{-0.041667in}}%
\pgfpathclose%
\pgfusepath{stroke,fill}%
}%
\begin{pgfscope}%
\pgfsys@transformshift{1.497704in}{0.582556in}%
\pgfsys@useobject{currentmarker}{}%
\end{pgfscope}%
\end{pgfscope}%
\begin{pgfscope}%
\pgfsetrectcap%
\pgfsetmiterjoin%
\pgfsetlinewidth{0.803000pt}%
\definecolor{currentstroke}{rgb}{0.000000,0.000000,0.000000}%
\pgfsetstrokecolor{currentstroke}%
\pgfsetdash{}{0pt}%
\pgfpathmoveto{\pgfqpoint{0.530556in}{0.499444in}}%
\pgfpathlineto{\pgfqpoint{0.530556in}{1.654444in}}%
\pgfusepath{stroke}%
\end{pgfscope}%
\begin{pgfscope}%
\pgfsetrectcap%
\pgfsetmiterjoin%
\pgfsetlinewidth{0.803000pt}%
\definecolor{currentstroke}{rgb}{0.000000,0.000000,0.000000}%
\pgfsetstrokecolor{currentstroke}%
\pgfsetdash{}{0pt}%
\pgfpathmoveto{\pgfqpoint{2.080556in}{0.499444in}}%
\pgfpathlineto{\pgfqpoint{2.080556in}{1.654444in}}%
\pgfusepath{stroke}%
\end{pgfscope}%
\begin{pgfscope}%
\pgfsetrectcap%
\pgfsetmiterjoin%
\pgfsetlinewidth{0.803000pt}%
\definecolor{currentstroke}{rgb}{0.000000,0.000000,0.000000}%
\pgfsetstrokecolor{currentstroke}%
\pgfsetdash{}{0pt}%
\pgfpathmoveto{\pgfqpoint{0.530556in}{0.499444in}}%
\pgfpathlineto{\pgfqpoint{2.080556in}{0.499444in}}%
\pgfusepath{stroke}%
\end{pgfscope}%
\begin{pgfscope}%
\pgfsetrectcap%
\pgfsetmiterjoin%
\pgfsetlinewidth{0.803000pt}%
\definecolor{currentstroke}{rgb}{0.000000,0.000000,0.000000}%
\pgfsetstrokecolor{currentstroke}%
\pgfsetdash{}{0pt}%
\pgfpathmoveto{\pgfqpoint{0.530556in}{1.654444in}}%
\pgfpathlineto{\pgfqpoint{2.080556in}{1.654444in}}%
\pgfusepath{stroke}%
\end{pgfscope}%
\begin{pgfscope}%
\pgfsetbuttcap%
\pgfsetmiterjoin%
\definecolor{currentfill}{rgb}{1.000000,1.000000,1.000000}%
\pgfsetfillcolor{currentfill}%
\pgfsetfillopacity{0.800000}%
\pgfsetlinewidth{1.003750pt}%
\definecolor{currentstroke}{rgb}{0.800000,0.800000,0.800000}%
\pgfsetstrokecolor{currentstroke}%
\pgfsetstrokeopacity{0.800000}%
\pgfsetdash{}{0pt}%
\pgfpathmoveto{\pgfqpoint{0.811987in}{1.126667in}}%
\pgfpathlineto{\pgfqpoint{1.983333in}{1.126667in}}%
\pgfpathquadraticcurveto{\pgfqpoint{2.011111in}{1.126667in}}{\pgfqpoint{2.011111in}{1.154444in}}%
\pgfpathlineto{\pgfqpoint{2.011111in}{1.557222in}}%
\pgfpathquadraticcurveto{\pgfqpoint{2.011111in}{1.585000in}}{\pgfqpoint{1.983333in}{1.585000in}}%
\pgfpathlineto{\pgfqpoint{0.811987in}{1.585000in}}%
\pgfpathquadraticcurveto{\pgfqpoint{0.784210in}{1.585000in}}{\pgfqpoint{0.784210in}{1.557222in}}%
\pgfpathlineto{\pgfqpoint{0.784210in}{1.154444in}}%
\pgfpathquadraticcurveto{\pgfqpoint{0.784210in}{1.126667in}}{\pgfqpoint{0.811987in}{1.126667in}}%
\pgfpathlineto{\pgfqpoint{0.811987in}{1.126667in}}%
\pgfpathclose%
\pgfusepath{stroke,fill}%
\end{pgfscope}%
\begin{pgfscope}%
\pgfsetrectcap%
\pgfsetroundjoin%
\pgfsetlinewidth{1.505625pt}%
\definecolor{currentstroke}{rgb}{0.000000,0.000000,0.000000}%
\pgfsetstrokecolor{currentstroke}%
\pgfsetdash{}{0pt}%
\pgfpathmoveto{\pgfqpoint{0.839765in}{1.473889in}}%
\pgfpathlineto{\pgfqpoint{0.978654in}{1.473889in}}%
\pgfpathlineto{\pgfqpoint{1.117543in}{1.473889in}}%
\pgfusepath{stroke}%
\end{pgfscope}%
\begin{pgfscope}%
\definecolor{textcolor}{rgb}{0.000000,0.000000,0.000000}%
\pgfsetstrokecolor{textcolor}%
\pgfsetfillcolor{textcolor}%
\pgftext[x=1.228654in,y=1.425277in,left,base]{\color{textcolor}\rmfamily\fontsize{10.000000}{12.000000}\selectfont \(\displaystyle \Delta FP/\Delta TP\)}%
\end{pgfscope}%
\begin{pgfscope}%
\pgfsetrectcap%
\pgfsetroundjoin%
\pgfsetlinewidth{1.505625pt}%
\definecolor{currentstroke}{rgb}{0.121569,0.466667,0.705882}%
\pgfsetstrokecolor{currentstroke}%
\pgfsetdash{}{0pt}%
\pgfpathmoveto{\pgfqpoint{0.839765in}{1.265555in}}%
\pgfpathlineto{\pgfqpoint{0.978654in}{1.265555in}}%
\pgfpathlineto{\pgfqpoint{1.117543in}{1.265555in}}%
\pgfusepath{stroke}%
\end{pgfscope}%
\begin{pgfscope}%
\pgfsetbuttcap%
\pgfsetroundjoin%
\definecolor{currentfill}{rgb}{0.000000,0.000000,0.000000}%
\pgfsetfillcolor{currentfill}%
\pgfsetlinewidth{1.003750pt}%
\definecolor{currentstroke}{rgb}{0.000000,0.000000,0.000000}%
\pgfsetstrokecolor{currentstroke}%
\pgfsetdash{}{0pt}%
\pgfsys@defobject{currentmarker}{\pgfqpoint{-0.041667in}{-0.041667in}}{\pgfqpoint{0.041667in}{0.041667in}}{%
\pgfpathmoveto{\pgfqpoint{0.000000in}{-0.041667in}}%
\pgfpathcurveto{\pgfqpoint{0.011050in}{-0.041667in}}{\pgfqpoint{0.021649in}{-0.037276in}}{\pgfqpoint{0.029463in}{-0.029463in}}%
\pgfpathcurveto{\pgfqpoint{0.037276in}{-0.021649in}}{\pgfqpoint{0.041667in}{-0.011050in}}{\pgfqpoint{0.041667in}{0.000000in}}%
\pgfpathcurveto{\pgfqpoint{0.041667in}{0.011050in}}{\pgfqpoint{0.037276in}{0.021649in}}{\pgfqpoint{0.029463in}{0.029463in}}%
\pgfpathcurveto{\pgfqpoint{0.021649in}{0.037276in}}{\pgfqpoint{0.011050in}{0.041667in}}{\pgfqpoint{0.000000in}{0.041667in}}%
\pgfpathcurveto{\pgfqpoint{-0.011050in}{0.041667in}}{\pgfqpoint{-0.021649in}{0.037276in}}{\pgfqpoint{-0.029463in}{0.029463in}}%
\pgfpathcurveto{\pgfqpoint{-0.037276in}{0.021649in}}{\pgfqpoint{-0.041667in}{0.011050in}}{\pgfqpoint{-0.041667in}{0.000000in}}%
\pgfpathcurveto{\pgfqpoint{-0.041667in}{-0.011050in}}{\pgfqpoint{-0.037276in}{-0.021649in}}{\pgfqpoint{-0.029463in}{-0.029463in}}%
\pgfpathcurveto{\pgfqpoint{-0.021649in}{-0.037276in}}{\pgfqpoint{-0.011050in}{-0.041667in}}{\pgfqpoint{0.000000in}{-0.041667in}}%
\pgfpathlineto{\pgfqpoint{0.000000in}{-0.041667in}}%
\pgfpathclose%
\pgfusepath{stroke,fill}%
}%
\begin{pgfscope}%
\pgfsys@transformshift{0.978654in}{1.265555in}%
\pgfsys@useobject{currentmarker}{}%
\end{pgfscope}%
\end{pgfscope}%
\begin{pgfscope}%
\definecolor{textcolor}{rgb}{0.000000,0.000000,0.000000}%
\pgfsetstrokecolor{textcolor}%
\pgfsetfillcolor{textcolor}%
\pgftext[x=1.228654in,y=1.216944in,left,base]{\color{textcolor}\rmfamily\fontsize{10.000000}{12.000000}\selectfont (0.635,2)}%
\end{pgfscope}%
\end{pgfpicture}%
\makeatother%
\endgroup%


We want $p=0.635$ to be our threshold while still using tools that assume $p=0.5$ is the threshold, so we will linearly transform the probabilities, mapping $0.635 \to 0.5$ while keeping 0.0 at 0.0, shown below in Example 2. Note that the linear transformation of the probabilities has no affect on the shape of the ROC curve nor the area under it.  The 0.243 and 0.52 are the median transformed probabilities for the negative and positive classes, respectively.  

\noindent\begin{tabular}{@{}p{0.3\textwidth}@{\hspace{24pt}} p{0.3\textwidth} @{\hspace{24pt}} p{0.3\textwidth}}
  \vspace{0pt} %% Creator: Matplotlib, PGF backend
%%
%% To include the figure in your LaTeX document, write
%%   \input{<filename>.pgf}
%%
%% Make sure the required packages are loaded in your preamble
%%   \usepackage{pgf}
%%
%% Also ensure that all the required font packages are loaded; for instance,
%% the lmodern package is sometimes necessary when using math font.
%%   \usepackage{lmodern}
%%
%% Figures using additional raster images can only be included by \input if
%% they are in the same directory as the main LaTeX file. For loading figures
%% from other directories you can use the `import` package
%%   \usepackage{import}
%%
%% and then include the figures with
%%   \import{<path to file>}{<filename>.pgf}
%%
%% Matplotlib used the following preamble
%%   
%%   \usepackage{fontspec}
%%   \makeatletter\@ifpackageloaded{underscore}{}{\usepackage[strings]{underscore}}\makeatother
%%
\begingroup%
\makeatletter%
\begin{pgfpicture}%
\pgfpathrectangle{\pgfpointorigin}{\pgfqpoint{2.253750in}{1.754444in}}%
\pgfusepath{use as bounding box, clip}%
\begin{pgfscope}%
\pgfsetbuttcap%
\pgfsetmiterjoin%
\definecolor{currentfill}{rgb}{1.000000,1.000000,1.000000}%
\pgfsetfillcolor{currentfill}%
\pgfsetlinewidth{0.000000pt}%
\definecolor{currentstroke}{rgb}{1.000000,1.000000,1.000000}%
\pgfsetstrokecolor{currentstroke}%
\pgfsetdash{}{0pt}%
\pgfpathmoveto{\pgfqpoint{0.000000in}{0.000000in}}%
\pgfpathlineto{\pgfqpoint{2.253750in}{0.000000in}}%
\pgfpathlineto{\pgfqpoint{2.253750in}{1.754444in}}%
\pgfpathlineto{\pgfqpoint{0.000000in}{1.754444in}}%
\pgfpathlineto{\pgfqpoint{0.000000in}{0.000000in}}%
\pgfpathclose%
\pgfusepath{fill}%
\end{pgfscope}%
\begin{pgfscope}%
\pgfsetbuttcap%
\pgfsetmiterjoin%
\definecolor{currentfill}{rgb}{1.000000,1.000000,1.000000}%
\pgfsetfillcolor{currentfill}%
\pgfsetlinewidth{0.000000pt}%
\definecolor{currentstroke}{rgb}{0.000000,0.000000,0.000000}%
\pgfsetstrokecolor{currentstroke}%
\pgfsetstrokeopacity{0.000000}%
\pgfsetdash{}{0pt}%
\pgfpathmoveto{\pgfqpoint{0.515000in}{0.499444in}}%
\pgfpathlineto{\pgfqpoint{2.065000in}{0.499444in}}%
\pgfpathlineto{\pgfqpoint{2.065000in}{1.654444in}}%
\pgfpathlineto{\pgfqpoint{0.515000in}{1.654444in}}%
\pgfpathlineto{\pgfqpoint{0.515000in}{0.499444in}}%
\pgfpathclose%
\pgfusepath{fill}%
\end{pgfscope}%
\begin{pgfscope}%
\pgfpathrectangle{\pgfqpoint{0.515000in}{0.499444in}}{\pgfqpoint{1.550000in}{1.155000in}}%
\pgfusepath{clip}%
\pgfsetbuttcap%
\pgfsetmiterjoin%
\pgfsetlinewidth{1.003750pt}%
\definecolor{currentstroke}{rgb}{0.000000,0.000000,0.000000}%
\pgfsetstrokecolor{currentstroke}%
\pgfsetdash{}{0pt}%
\pgfpathmoveto{\pgfqpoint{0.505000in}{0.499444in}}%
\pgfpathlineto{\pgfqpoint{0.552805in}{0.499444in}}%
\pgfpathlineto{\pgfqpoint{0.552805in}{1.039017in}}%
\pgfpathlineto{\pgfqpoint{0.505000in}{1.039017in}}%
\pgfusepath{stroke}%
\end{pgfscope}%
\begin{pgfscope}%
\pgfpathrectangle{\pgfqpoint{0.515000in}{0.499444in}}{\pgfqpoint{1.550000in}{1.155000in}}%
\pgfusepath{clip}%
\pgfsetbuttcap%
\pgfsetmiterjoin%
\pgfsetlinewidth{1.003750pt}%
\definecolor{currentstroke}{rgb}{0.000000,0.000000,0.000000}%
\pgfsetstrokecolor{currentstroke}%
\pgfsetdash{}{0pt}%
\pgfpathmoveto{\pgfqpoint{0.643537in}{0.499444in}}%
\pgfpathlineto{\pgfqpoint{0.704025in}{0.499444in}}%
\pgfpathlineto{\pgfqpoint{0.704025in}{1.599444in}}%
\pgfpathlineto{\pgfqpoint{0.643537in}{1.599444in}}%
\pgfpathlineto{\pgfqpoint{0.643537in}{0.499444in}}%
\pgfpathclose%
\pgfusepath{stroke}%
\end{pgfscope}%
\begin{pgfscope}%
\pgfpathrectangle{\pgfqpoint{0.515000in}{0.499444in}}{\pgfqpoint{1.550000in}{1.155000in}}%
\pgfusepath{clip}%
\pgfsetbuttcap%
\pgfsetmiterjoin%
\pgfsetlinewidth{1.003750pt}%
\definecolor{currentstroke}{rgb}{0.000000,0.000000,0.000000}%
\pgfsetstrokecolor{currentstroke}%
\pgfsetdash{}{0pt}%
\pgfpathmoveto{\pgfqpoint{0.794756in}{0.499444in}}%
\pgfpathlineto{\pgfqpoint{0.855244in}{0.499444in}}%
\pgfpathlineto{\pgfqpoint{0.855244in}{1.562328in}}%
\pgfpathlineto{\pgfqpoint{0.794756in}{1.562328in}}%
\pgfpathlineto{\pgfqpoint{0.794756in}{0.499444in}}%
\pgfpathclose%
\pgfusepath{stroke}%
\end{pgfscope}%
\begin{pgfscope}%
\pgfpathrectangle{\pgfqpoint{0.515000in}{0.499444in}}{\pgfqpoint{1.550000in}{1.155000in}}%
\pgfusepath{clip}%
\pgfsetbuttcap%
\pgfsetmiterjoin%
\pgfsetlinewidth{1.003750pt}%
\definecolor{currentstroke}{rgb}{0.000000,0.000000,0.000000}%
\pgfsetstrokecolor{currentstroke}%
\pgfsetdash{}{0pt}%
\pgfpathmoveto{\pgfqpoint{0.945976in}{0.499444in}}%
\pgfpathlineto{\pgfqpoint{1.006464in}{0.499444in}}%
\pgfpathlineto{\pgfqpoint{1.006464in}{1.236981in}}%
\pgfpathlineto{\pgfqpoint{0.945976in}{1.236981in}}%
\pgfpathlineto{\pgfqpoint{0.945976in}{0.499444in}}%
\pgfpathclose%
\pgfusepath{stroke}%
\end{pgfscope}%
\begin{pgfscope}%
\pgfpathrectangle{\pgfqpoint{0.515000in}{0.499444in}}{\pgfqpoint{1.550000in}{1.155000in}}%
\pgfusepath{clip}%
\pgfsetbuttcap%
\pgfsetmiterjoin%
\pgfsetlinewidth{1.003750pt}%
\definecolor{currentstroke}{rgb}{0.000000,0.000000,0.000000}%
\pgfsetstrokecolor{currentstroke}%
\pgfsetdash{}{0pt}%
\pgfpathmoveto{\pgfqpoint{1.097195in}{0.499444in}}%
\pgfpathlineto{\pgfqpoint{1.157683in}{0.499444in}}%
\pgfpathlineto{\pgfqpoint{1.157683in}{0.927639in}}%
\pgfpathlineto{\pgfqpoint{1.097195in}{0.927639in}}%
\pgfpathlineto{\pgfqpoint{1.097195in}{0.499444in}}%
\pgfpathclose%
\pgfusepath{stroke}%
\end{pgfscope}%
\begin{pgfscope}%
\pgfpathrectangle{\pgfqpoint{0.515000in}{0.499444in}}{\pgfqpoint{1.550000in}{1.155000in}}%
\pgfusepath{clip}%
\pgfsetbuttcap%
\pgfsetmiterjoin%
\pgfsetlinewidth{1.003750pt}%
\definecolor{currentstroke}{rgb}{0.000000,0.000000,0.000000}%
\pgfsetstrokecolor{currentstroke}%
\pgfsetdash{}{0pt}%
\pgfpathmoveto{\pgfqpoint{1.248415in}{0.499444in}}%
\pgfpathlineto{\pgfqpoint{1.308903in}{0.499444in}}%
\pgfpathlineto{\pgfqpoint{1.308903in}{0.751443in}}%
\pgfpathlineto{\pgfqpoint{1.248415in}{0.751443in}}%
\pgfpathlineto{\pgfqpoint{1.248415in}{0.499444in}}%
\pgfpathclose%
\pgfusepath{stroke}%
\end{pgfscope}%
\begin{pgfscope}%
\pgfpathrectangle{\pgfqpoint{0.515000in}{0.499444in}}{\pgfqpoint{1.550000in}{1.155000in}}%
\pgfusepath{clip}%
\pgfsetbuttcap%
\pgfsetmiterjoin%
\pgfsetlinewidth{1.003750pt}%
\definecolor{currentstroke}{rgb}{0.000000,0.000000,0.000000}%
\pgfsetstrokecolor{currentstroke}%
\pgfsetdash{}{0pt}%
\pgfpathmoveto{\pgfqpoint{1.399634in}{0.499444in}}%
\pgfpathlineto{\pgfqpoint{1.460122in}{0.499444in}}%
\pgfpathlineto{\pgfqpoint{1.460122in}{0.604802in}}%
\pgfpathlineto{\pgfqpoint{1.399634in}{0.604802in}}%
\pgfpathlineto{\pgfqpoint{1.399634in}{0.499444in}}%
\pgfpathclose%
\pgfusepath{stroke}%
\end{pgfscope}%
\begin{pgfscope}%
\pgfpathrectangle{\pgfqpoint{0.515000in}{0.499444in}}{\pgfqpoint{1.550000in}{1.155000in}}%
\pgfusepath{clip}%
\pgfsetbuttcap%
\pgfsetmiterjoin%
\pgfsetlinewidth{1.003750pt}%
\definecolor{currentstroke}{rgb}{0.000000,0.000000,0.000000}%
\pgfsetstrokecolor{currentstroke}%
\pgfsetdash{}{0pt}%
\pgfpathmoveto{\pgfqpoint{1.550854in}{0.499444in}}%
\pgfpathlineto{\pgfqpoint{1.611342in}{0.499444in}}%
\pgfpathlineto{\pgfqpoint{1.611342in}{0.549370in}}%
\pgfpathlineto{\pgfqpoint{1.550854in}{0.549370in}}%
\pgfpathlineto{\pgfqpoint{1.550854in}{0.499444in}}%
\pgfpathclose%
\pgfusepath{stroke}%
\end{pgfscope}%
\begin{pgfscope}%
\pgfpathrectangle{\pgfqpoint{0.515000in}{0.499444in}}{\pgfqpoint{1.550000in}{1.155000in}}%
\pgfusepath{clip}%
\pgfsetbuttcap%
\pgfsetmiterjoin%
\pgfsetlinewidth{1.003750pt}%
\definecolor{currentstroke}{rgb}{0.000000,0.000000,0.000000}%
\pgfsetstrokecolor{currentstroke}%
\pgfsetdash{}{0pt}%
\pgfpathmoveto{\pgfqpoint{1.702073in}{0.499444in}}%
\pgfpathlineto{\pgfqpoint{1.762561in}{0.499444in}}%
\pgfpathlineto{\pgfqpoint{1.762561in}{0.503638in}}%
\pgfpathlineto{\pgfqpoint{1.702073in}{0.503638in}}%
\pgfpathlineto{\pgfqpoint{1.702073in}{0.499444in}}%
\pgfpathclose%
\pgfusepath{stroke}%
\end{pgfscope}%
\begin{pgfscope}%
\pgfpathrectangle{\pgfqpoint{0.515000in}{0.499444in}}{\pgfqpoint{1.550000in}{1.155000in}}%
\pgfusepath{clip}%
\pgfsetbuttcap%
\pgfsetmiterjoin%
\pgfsetlinewidth{1.003750pt}%
\definecolor{currentstroke}{rgb}{0.000000,0.000000,0.000000}%
\pgfsetstrokecolor{currentstroke}%
\pgfsetdash{}{0pt}%
\pgfpathmoveto{\pgfqpoint{1.853293in}{0.499444in}}%
\pgfpathlineto{\pgfqpoint{1.913781in}{0.499444in}}%
\pgfpathlineto{\pgfqpoint{1.913781in}{0.499444in}}%
\pgfpathlineto{\pgfqpoint{1.853293in}{0.499444in}}%
\pgfpathlineto{\pgfqpoint{1.853293in}{0.499444in}}%
\pgfpathclose%
\pgfusepath{stroke}%
\end{pgfscope}%
\begin{pgfscope}%
\pgfpathrectangle{\pgfqpoint{0.515000in}{0.499444in}}{\pgfqpoint{1.550000in}{1.155000in}}%
\pgfusepath{clip}%
\pgfsetbuttcap%
\pgfsetmiterjoin%
\definecolor{currentfill}{rgb}{0.000000,0.000000,0.000000}%
\pgfsetfillcolor{currentfill}%
\pgfsetlinewidth{0.000000pt}%
\definecolor{currentstroke}{rgb}{0.000000,0.000000,0.000000}%
\pgfsetstrokecolor{currentstroke}%
\pgfsetstrokeopacity{0.000000}%
\pgfsetdash{}{0pt}%
\pgfpathmoveto{\pgfqpoint{0.552805in}{0.499444in}}%
\pgfpathlineto{\pgfqpoint{0.613293in}{0.499444in}}%
\pgfpathlineto{\pgfqpoint{0.613293in}{0.511940in}}%
\pgfpathlineto{\pgfqpoint{0.552805in}{0.511940in}}%
\pgfpathlineto{\pgfqpoint{0.552805in}{0.499444in}}%
\pgfpathclose%
\pgfusepath{fill}%
\end{pgfscope}%
\begin{pgfscope}%
\pgfpathrectangle{\pgfqpoint{0.515000in}{0.499444in}}{\pgfqpoint{1.550000in}{1.155000in}}%
\pgfusepath{clip}%
\pgfsetbuttcap%
\pgfsetmiterjoin%
\definecolor{currentfill}{rgb}{0.000000,0.000000,0.000000}%
\pgfsetfillcolor{currentfill}%
\pgfsetlinewidth{0.000000pt}%
\definecolor{currentstroke}{rgb}{0.000000,0.000000,0.000000}%
\pgfsetstrokecolor{currentstroke}%
\pgfsetstrokeopacity{0.000000}%
\pgfsetdash{}{0pt}%
\pgfpathmoveto{\pgfqpoint{0.704025in}{0.499444in}}%
\pgfpathlineto{\pgfqpoint{0.764512in}{0.499444in}}%
\pgfpathlineto{\pgfqpoint{0.764512in}{0.525834in}}%
\pgfpathlineto{\pgfqpoint{0.704025in}{0.525834in}}%
\pgfpathlineto{\pgfqpoint{0.704025in}{0.499444in}}%
\pgfpathclose%
\pgfusepath{fill}%
\end{pgfscope}%
\begin{pgfscope}%
\pgfpathrectangle{\pgfqpoint{0.515000in}{0.499444in}}{\pgfqpoint{1.550000in}{1.155000in}}%
\pgfusepath{clip}%
\pgfsetbuttcap%
\pgfsetmiterjoin%
\definecolor{currentfill}{rgb}{0.000000,0.000000,0.000000}%
\pgfsetfillcolor{currentfill}%
\pgfsetlinewidth{0.000000pt}%
\definecolor{currentstroke}{rgb}{0.000000,0.000000,0.000000}%
\pgfsetstrokecolor{currentstroke}%
\pgfsetstrokeopacity{0.000000}%
\pgfsetdash{}{0pt}%
\pgfpathmoveto{\pgfqpoint{0.855244in}{0.499444in}}%
\pgfpathlineto{\pgfqpoint{0.915732in}{0.499444in}}%
\pgfpathlineto{\pgfqpoint{0.915732in}{0.552052in}}%
\pgfpathlineto{\pgfqpoint{0.855244in}{0.552052in}}%
\pgfpathlineto{\pgfqpoint{0.855244in}{0.499444in}}%
\pgfpathclose%
\pgfusepath{fill}%
\end{pgfscope}%
\begin{pgfscope}%
\pgfpathrectangle{\pgfqpoint{0.515000in}{0.499444in}}{\pgfqpoint{1.550000in}{1.155000in}}%
\pgfusepath{clip}%
\pgfsetbuttcap%
\pgfsetmiterjoin%
\definecolor{currentfill}{rgb}{0.000000,0.000000,0.000000}%
\pgfsetfillcolor{currentfill}%
\pgfsetlinewidth{0.000000pt}%
\definecolor{currentstroke}{rgb}{0.000000,0.000000,0.000000}%
\pgfsetstrokecolor{currentstroke}%
\pgfsetstrokeopacity{0.000000}%
\pgfsetdash{}{0pt}%
\pgfpathmoveto{\pgfqpoint{1.006464in}{0.499444in}}%
\pgfpathlineto{\pgfqpoint{1.066951in}{0.499444in}}%
\pgfpathlineto{\pgfqpoint{1.066951in}{0.592621in}}%
\pgfpathlineto{\pgfqpoint{1.006464in}{0.592621in}}%
\pgfpathlineto{\pgfqpoint{1.006464in}{0.499444in}}%
\pgfpathclose%
\pgfusepath{fill}%
\end{pgfscope}%
\begin{pgfscope}%
\pgfpathrectangle{\pgfqpoint{0.515000in}{0.499444in}}{\pgfqpoint{1.550000in}{1.155000in}}%
\pgfusepath{clip}%
\pgfsetbuttcap%
\pgfsetmiterjoin%
\definecolor{currentfill}{rgb}{0.000000,0.000000,0.000000}%
\pgfsetfillcolor{currentfill}%
\pgfsetlinewidth{0.000000pt}%
\definecolor{currentstroke}{rgb}{0.000000,0.000000,0.000000}%
\pgfsetstrokecolor{currentstroke}%
\pgfsetstrokeopacity{0.000000}%
\pgfsetdash{}{0pt}%
\pgfpathmoveto{\pgfqpoint{1.157683in}{0.499444in}}%
\pgfpathlineto{\pgfqpoint{1.218171in}{0.499444in}}%
\pgfpathlineto{\pgfqpoint{1.218171in}{0.654015in}}%
\pgfpathlineto{\pgfqpoint{1.157683in}{0.654015in}}%
\pgfpathlineto{\pgfqpoint{1.157683in}{0.499444in}}%
\pgfpathclose%
\pgfusepath{fill}%
\end{pgfscope}%
\begin{pgfscope}%
\pgfpathrectangle{\pgfqpoint{0.515000in}{0.499444in}}{\pgfqpoint{1.550000in}{1.155000in}}%
\pgfusepath{clip}%
\pgfsetbuttcap%
\pgfsetmiterjoin%
\definecolor{currentfill}{rgb}{0.000000,0.000000,0.000000}%
\pgfsetfillcolor{currentfill}%
\pgfsetlinewidth{0.000000pt}%
\definecolor{currentstroke}{rgb}{0.000000,0.000000,0.000000}%
\pgfsetstrokecolor{currentstroke}%
\pgfsetstrokeopacity{0.000000}%
\pgfsetdash{}{0pt}%
\pgfpathmoveto{\pgfqpoint{1.308903in}{0.499444in}}%
\pgfpathlineto{\pgfqpoint{1.369391in}{0.499444in}}%
\pgfpathlineto{\pgfqpoint{1.369391in}{0.696438in}}%
\pgfpathlineto{\pgfqpoint{1.308903in}{0.696438in}}%
\pgfpathlineto{\pgfqpoint{1.308903in}{0.499444in}}%
\pgfpathclose%
\pgfusepath{fill}%
\end{pgfscope}%
\begin{pgfscope}%
\pgfpathrectangle{\pgfqpoint{0.515000in}{0.499444in}}{\pgfqpoint{1.550000in}{1.155000in}}%
\pgfusepath{clip}%
\pgfsetbuttcap%
\pgfsetmiterjoin%
\definecolor{currentfill}{rgb}{0.000000,0.000000,0.000000}%
\pgfsetfillcolor{currentfill}%
\pgfsetlinewidth{0.000000pt}%
\definecolor{currentstroke}{rgb}{0.000000,0.000000,0.000000}%
\pgfsetstrokecolor{currentstroke}%
\pgfsetstrokeopacity{0.000000}%
\pgfsetdash{}{0pt}%
\pgfpathmoveto{\pgfqpoint{1.460122in}{0.499444in}}%
\pgfpathlineto{\pgfqpoint{1.520610in}{0.499444in}}%
\pgfpathlineto{\pgfqpoint{1.520610in}{0.667738in}}%
\pgfpathlineto{\pgfqpoint{1.460122in}{0.667738in}}%
\pgfpathlineto{\pgfqpoint{1.460122in}{0.499444in}}%
\pgfpathclose%
\pgfusepath{fill}%
\end{pgfscope}%
\begin{pgfscope}%
\pgfpathrectangle{\pgfqpoint{0.515000in}{0.499444in}}{\pgfqpoint{1.550000in}{1.155000in}}%
\pgfusepath{clip}%
\pgfsetbuttcap%
\pgfsetmiterjoin%
\definecolor{currentfill}{rgb}{0.000000,0.000000,0.000000}%
\pgfsetfillcolor{currentfill}%
\pgfsetlinewidth{0.000000pt}%
\definecolor{currentstroke}{rgb}{0.000000,0.000000,0.000000}%
\pgfsetstrokecolor{currentstroke}%
\pgfsetstrokeopacity{0.000000}%
\pgfsetdash{}{0pt}%
\pgfpathmoveto{\pgfqpoint{1.611342in}{0.499444in}}%
\pgfpathlineto{\pgfqpoint{1.671830in}{0.499444in}}%
\pgfpathlineto{\pgfqpoint{1.671830in}{0.550683in}}%
\pgfpathlineto{\pgfqpoint{1.611342in}{0.550683in}}%
\pgfpathlineto{\pgfqpoint{1.611342in}{0.499444in}}%
\pgfpathclose%
\pgfusepath{fill}%
\end{pgfscope}%
\begin{pgfscope}%
\pgfpathrectangle{\pgfqpoint{0.515000in}{0.499444in}}{\pgfqpoint{1.550000in}{1.155000in}}%
\pgfusepath{clip}%
\pgfsetbuttcap%
\pgfsetmiterjoin%
\definecolor{currentfill}{rgb}{0.000000,0.000000,0.000000}%
\pgfsetfillcolor{currentfill}%
\pgfsetlinewidth{0.000000pt}%
\definecolor{currentstroke}{rgb}{0.000000,0.000000,0.000000}%
\pgfsetstrokecolor{currentstroke}%
\pgfsetstrokeopacity{0.000000}%
\pgfsetdash{}{0pt}%
\pgfpathmoveto{\pgfqpoint{1.762561in}{0.499444in}}%
\pgfpathlineto{\pgfqpoint{1.823049in}{0.499444in}}%
\pgfpathlineto{\pgfqpoint{1.823049in}{0.499558in}}%
\pgfpathlineto{\pgfqpoint{1.762561in}{0.499558in}}%
\pgfpathlineto{\pgfqpoint{1.762561in}{0.499444in}}%
\pgfpathclose%
\pgfusepath{fill}%
\end{pgfscope}%
\begin{pgfscope}%
\pgfpathrectangle{\pgfqpoint{0.515000in}{0.499444in}}{\pgfqpoint{1.550000in}{1.155000in}}%
\pgfusepath{clip}%
\pgfsetbuttcap%
\pgfsetmiterjoin%
\definecolor{currentfill}{rgb}{0.000000,0.000000,0.000000}%
\pgfsetfillcolor{currentfill}%
\pgfsetlinewidth{0.000000pt}%
\definecolor{currentstroke}{rgb}{0.000000,0.000000,0.000000}%
\pgfsetstrokecolor{currentstroke}%
\pgfsetstrokeopacity{0.000000}%
\pgfsetdash{}{0pt}%
\pgfpathmoveto{\pgfqpoint{1.913781in}{0.499444in}}%
\pgfpathlineto{\pgfqpoint{1.974269in}{0.499444in}}%
\pgfpathlineto{\pgfqpoint{1.974269in}{0.499444in}}%
\pgfpathlineto{\pgfqpoint{1.913781in}{0.499444in}}%
\pgfpathlineto{\pgfqpoint{1.913781in}{0.499444in}}%
\pgfpathclose%
\pgfusepath{fill}%
\end{pgfscope}%
\begin{pgfscope}%
\pgfsetbuttcap%
\pgfsetroundjoin%
\definecolor{currentfill}{rgb}{0.000000,0.000000,0.000000}%
\pgfsetfillcolor{currentfill}%
\pgfsetlinewidth{0.803000pt}%
\definecolor{currentstroke}{rgb}{0.000000,0.000000,0.000000}%
\pgfsetstrokecolor{currentstroke}%
\pgfsetdash{}{0pt}%
\pgfsys@defobject{currentmarker}{\pgfqpoint{0.000000in}{-0.048611in}}{\pgfqpoint{0.000000in}{0.000000in}}{%
\pgfpathmoveto{\pgfqpoint{0.000000in}{0.000000in}}%
\pgfpathlineto{\pgfqpoint{0.000000in}{-0.048611in}}%
\pgfusepath{stroke,fill}%
}%
\begin{pgfscope}%
\pgfsys@transformshift{0.552805in}{0.499444in}%
\pgfsys@useobject{currentmarker}{}%
\end{pgfscope}%
\end{pgfscope}%
\begin{pgfscope}%
\definecolor{textcolor}{rgb}{0.000000,0.000000,0.000000}%
\pgfsetstrokecolor{textcolor}%
\pgfsetfillcolor{textcolor}%
\pgftext[x=0.552805in,y=0.402222in,,top]{\color{textcolor}\rmfamily\fontsize{10.000000}{12.000000}\selectfont 0.0}%
\end{pgfscope}%
\begin{pgfscope}%
\pgfsetbuttcap%
\pgfsetroundjoin%
\definecolor{currentfill}{rgb}{0.000000,0.000000,0.000000}%
\pgfsetfillcolor{currentfill}%
\pgfsetlinewidth{0.803000pt}%
\definecolor{currentstroke}{rgb}{0.000000,0.000000,0.000000}%
\pgfsetstrokecolor{currentstroke}%
\pgfsetdash{}{0pt}%
\pgfsys@defobject{currentmarker}{\pgfqpoint{0.000000in}{-0.048611in}}{\pgfqpoint{0.000000in}{0.000000in}}{%
\pgfpathmoveto{\pgfqpoint{0.000000in}{0.000000in}}%
\pgfpathlineto{\pgfqpoint{0.000000in}{-0.048611in}}%
\pgfusepath{stroke,fill}%
}%
\begin{pgfscope}%
\pgfsys@transformshift{0.930854in}{0.499444in}%
\pgfsys@useobject{currentmarker}{}%
\end{pgfscope}%
\end{pgfscope}%
\begin{pgfscope}%
\definecolor{textcolor}{rgb}{0.000000,0.000000,0.000000}%
\pgfsetstrokecolor{textcolor}%
\pgfsetfillcolor{textcolor}%
\pgftext[x=0.930854in,y=0.402222in,,top]{\color{textcolor}\rmfamily\fontsize{10.000000}{12.000000}\selectfont 0.25}%
\end{pgfscope}%
\begin{pgfscope}%
\pgfsetbuttcap%
\pgfsetroundjoin%
\definecolor{currentfill}{rgb}{0.000000,0.000000,0.000000}%
\pgfsetfillcolor{currentfill}%
\pgfsetlinewidth{0.803000pt}%
\definecolor{currentstroke}{rgb}{0.000000,0.000000,0.000000}%
\pgfsetstrokecolor{currentstroke}%
\pgfsetdash{}{0pt}%
\pgfsys@defobject{currentmarker}{\pgfqpoint{0.000000in}{-0.048611in}}{\pgfqpoint{0.000000in}{0.000000in}}{%
\pgfpathmoveto{\pgfqpoint{0.000000in}{0.000000in}}%
\pgfpathlineto{\pgfqpoint{0.000000in}{-0.048611in}}%
\pgfusepath{stroke,fill}%
}%
\begin{pgfscope}%
\pgfsys@transformshift{1.308903in}{0.499444in}%
\pgfsys@useobject{currentmarker}{}%
\end{pgfscope}%
\end{pgfscope}%
\begin{pgfscope}%
\definecolor{textcolor}{rgb}{0.000000,0.000000,0.000000}%
\pgfsetstrokecolor{textcolor}%
\pgfsetfillcolor{textcolor}%
\pgftext[x=1.308903in,y=0.402222in,,top]{\color{textcolor}\rmfamily\fontsize{10.000000}{12.000000}\selectfont 0.5}%
\end{pgfscope}%
\begin{pgfscope}%
\pgfsetbuttcap%
\pgfsetroundjoin%
\definecolor{currentfill}{rgb}{0.000000,0.000000,0.000000}%
\pgfsetfillcolor{currentfill}%
\pgfsetlinewidth{0.803000pt}%
\definecolor{currentstroke}{rgb}{0.000000,0.000000,0.000000}%
\pgfsetstrokecolor{currentstroke}%
\pgfsetdash{}{0pt}%
\pgfsys@defobject{currentmarker}{\pgfqpoint{0.000000in}{-0.048611in}}{\pgfqpoint{0.000000in}{0.000000in}}{%
\pgfpathmoveto{\pgfqpoint{0.000000in}{0.000000in}}%
\pgfpathlineto{\pgfqpoint{0.000000in}{-0.048611in}}%
\pgfusepath{stroke,fill}%
}%
\begin{pgfscope}%
\pgfsys@transformshift{1.686951in}{0.499444in}%
\pgfsys@useobject{currentmarker}{}%
\end{pgfscope}%
\end{pgfscope}%
\begin{pgfscope}%
\definecolor{textcolor}{rgb}{0.000000,0.000000,0.000000}%
\pgfsetstrokecolor{textcolor}%
\pgfsetfillcolor{textcolor}%
\pgftext[x=1.686951in,y=0.402222in,,top]{\color{textcolor}\rmfamily\fontsize{10.000000}{12.000000}\selectfont 0.75}%
\end{pgfscope}%
\begin{pgfscope}%
\pgfsetbuttcap%
\pgfsetroundjoin%
\definecolor{currentfill}{rgb}{0.000000,0.000000,0.000000}%
\pgfsetfillcolor{currentfill}%
\pgfsetlinewidth{0.803000pt}%
\definecolor{currentstroke}{rgb}{0.000000,0.000000,0.000000}%
\pgfsetstrokecolor{currentstroke}%
\pgfsetdash{}{0pt}%
\pgfsys@defobject{currentmarker}{\pgfqpoint{0.000000in}{-0.048611in}}{\pgfqpoint{0.000000in}{0.000000in}}{%
\pgfpathmoveto{\pgfqpoint{0.000000in}{0.000000in}}%
\pgfpathlineto{\pgfqpoint{0.000000in}{-0.048611in}}%
\pgfusepath{stroke,fill}%
}%
\begin{pgfscope}%
\pgfsys@transformshift{2.065000in}{0.499444in}%
\pgfsys@useobject{currentmarker}{}%
\end{pgfscope}%
\end{pgfscope}%
\begin{pgfscope}%
\definecolor{textcolor}{rgb}{0.000000,0.000000,0.000000}%
\pgfsetstrokecolor{textcolor}%
\pgfsetfillcolor{textcolor}%
\pgftext[x=2.065000in,y=0.402222in,,top]{\color{textcolor}\rmfamily\fontsize{10.000000}{12.000000}\selectfont 1.0}%
\end{pgfscope}%
\begin{pgfscope}%
\definecolor{textcolor}{rgb}{0.000000,0.000000,0.000000}%
\pgfsetstrokecolor{textcolor}%
\pgfsetfillcolor{textcolor}%
\pgftext[x=1.290000in,y=0.223333in,,top]{\color{textcolor}\rmfamily\fontsize{10.000000}{12.000000}\selectfont \(\displaystyle p\)}%
\end{pgfscope}%
\begin{pgfscope}%
\pgfsetbuttcap%
\pgfsetroundjoin%
\definecolor{currentfill}{rgb}{0.000000,0.000000,0.000000}%
\pgfsetfillcolor{currentfill}%
\pgfsetlinewidth{0.803000pt}%
\definecolor{currentstroke}{rgb}{0.000000,0.000000,0.000000}%
\pgfsetstrokecolor{currentstroke}%
\pgfsetdash{}{0pt}%
\pgfsys@defobject{currentmarker}{\pgfqpoint{-0.048611in}{0.000000in}}{\pgfqpoint{-0.000000in}{0.000000in}}{%
\pgfpathmoveto{\pgfqpoint{-0.000000in}{0.000000in}}%
\pgfpathlineto{\pgfqpoint{-0.048611in}{0.000000in}}%
\pgfusepath{stroke,fill}%
}%
\begin{pgfscope}%
\pgfsys@transformshift{0.515000in}{0.499444in}%
\pgfsys@useobject{currentmarker}{}%
\end{pgfscope}%
\end{pgfscope}%
\begin{pgfscope}%
\definecolor{textcolor}{rgb}{0.000000,0.000000,0.000000}%
\pgfsetstrokecolor{textcolor}%
\pgfsetfillcolor{textcolor}%
\pgftext[x=0.348333in, y=0.451250in, left, base]{\color{textcolor}\rmfamily\fontsize{10.000000}{12.000000}\selectfont \(\displaystyle {0}\)}%
\end{pgfscope}%
\begin{pgfscope}%
\pgfsetbuttcap%
\pgfsetroundjoin%
\definecolor{currentfill}{rgb}{0.000000,0.000000,0.000000}%
\pgfsetfillcolor{currentfill}%
\pgfsetlinewidth{0.803000pt}%
\definecolor{currentstroke}{rgb}{0.000000,0.000000,0.000000}%
\pgfsetstrokecolor{currentstroke}%
\pgfsetdash{}{0pt}%
\pgfsys@defobject{currentmarker}{\pgfqpoint{-0.048611in}{0.000000in}}{\pgfqpoint{-0.000000in}{0.000000in}}{%
\pgfpathmoveto{\pgfqpoint{-0.000000in}{0.000000in}}%
\pgfpathlineto{\pgfqpoint{-0.048611in}{0.000000in}}%
\pgfusepath{stroke,fill}%
}%
\begin{pgfscope}%
\pgfsys@transformshift{0.515000in}{1.005529in}%
\pgfsys@useobject{currentmarker}{}%
\end{pgfscope}%
\end{pgfscope}%
\begin{pgfscope}%
\definecolor{textcolor}{rgb}{0.000000,0.000000,0.000000}%
\pgfsetstrokecolor{textcolor}%
\pgfsetfillcolor{textcolor}%
\pgftext[x=0.278889in, y=0.957335in, left, base]{\color{textcolor}\rmfamily\fontsize{10.000000}{12.000000}\selectfont \(\displaystyle {10}\)}%
\end{pgfscope}%
\begin{pgfscope}%
\pgfsetbuttcap%
\pgfsetroundjoin%
\definecolor{currentfill}{rgb}{0.000000,0.000000,0.000000}%
\pgfsetfillcolor{currentfill}%
\pgfsetlinewidth{0.803000pt}%
\definecolor{currentstroke}{rgb}{0.000000,0.000000,0.000000}%
\pgfsetstrokecolor{currentstroke}%
\pgfsetdash{}{0pt}%
\pgfsys@defobject{currentmarker}{\pgfqpoint{-0.048611in}{0.000000in}}{\pgfqpoint{-0.000000in}{0.000000in}}{%
\pgfpathmoveto{\pgfqpoint{-0.000000in}{0.000000in}}%
\pgfpathlineto{\pgfqpoint{-0.048611in}{0.000000in}}%
\pgfusepath{stroke,fill}%
}%
\begin{pgfscope}%
\pgfsys@transformshift{0.515000in}{1.511614in}%
\pgfsys@useobject{currentmarker}{}%
\end{pgfscope}%
\end{pgfscope}%
\begin{pgfscope}%
\definecolor{textcolor}{rgb}{0.000000,0.000000,0.000000}%
\pgfsetstrokecolor{textcolor}%
\pgfsetfillcolor{textcolor}%
\pgftext[x=0.278889in, y=1.463420in, left, base]{\color{textcolor}\rmfamily\fontsize{10.000000}{12.000000}\selectfont \(\displaystyle {20}\)}%
\end{pgfscope}%
\begin{pgfscope}%
\definecolor{textcolor}{rgb}{0.000000,0.000000,0.000000}%
\pgfsetstrokecolor{textcolor}%
\pgfsetfillcolor{textcolor}%
\pgftext[x=0.223333in,y=1.076944in,,bottom,rotate=90.000000]{\color{textcolor}\rmfamily\fontsize{10.000000}{12.000000}\selectfont Percent of Data Set}%
\end{pgfscope}%
\begin{pgfscope}%
\pgfsetrectcap%
\pgfsetmiterjoin%
\pgfsetlinewidth{0.803000pt}%
\definecolor{currentstroke}{rgb}{0.000000,0.000000,0.000000}%
\pgfsetstrokecolor{currentstroke}%
\pgfsetdash{}{0pt}%
\pgfpathmoveto{\pgfqpoint{0.515000in}{0.499444in}}%
\pgfpathlineto{\pgfqpoint{0.515000in}{1.654444in}}%
\pgfusepath{stroke}%
\end{pgfscope}%
\begin{pgfscope}%
\pgfsetrectcap%
\pgfsetmiterjoin%
\pgfsetlinewidth{0.803000pt}%
\definecolor{currentstroke}{rgb}{0.000000,0.000000,0.000000}%
\pgfsetstrokecolor{currentstroke}%
\pgfsetdash{}{0pt}%
\pgfpathmoveto{\pgfqpoint{2.065000in}{0.499444in}}%
\pgfpathlineto{\pgfqpoint{2.065000in}{1.654444in}}%
\pgfusepath{stroke}%
\end{pgfscope}%
\begin{pgfscope}%
\pgfsetrectcap%
\pgfsetmiterjoin%
\pgfsetlinewidth{0.803000pt}%
\definecolor{currentstroke}{rgb}{0.000000,0.000000,0.000000}%
\pgfsetstrokecolor{currentstroke}%
\pgfsetdash{}{0pt}%
\pgfpathmoveto{\pgfqpoint{0.515000in}{0.499444in}}%
\pgfpathlineto{\pgfqpoint{2.065000in}{0.499444in}}%
\pgfusepath{stroke}%
\end{pgfscope}%
\begin{pgfscope}%
\pgfsetrectcap%
\pgfsetmiterjoin%
\pgfsetlinewidth{0.803000pt}%
\definecolor{currentstroke}{rgb}{0.000000,0.000000,0.000000}%
\pgfsetstrokecolor{currentstroke}%
\pgfsetdash{}{0pt}%
\pgfpathmoveto{\pgfqpoint{0.515000in}{1.654444in}}%
\pgfpathlineto{\pgfqpoint{2.065000in}{1.654444in}}%
\pgfusepath{stroke}%
\end{pgfscope}%
\begin{pgfscope}%
\pgfsetbuttcap%
\pgfsetmiterjoin%
\definecolor{currentfill}{rgb}{1.000000,1.000000,1.000000}%
\pgfsetfillcolor{currentfill}%
\pgfsetfillopacity{0.800000}%
\pgfsetlinewidth{1.003750pt}%
\definecolor{currentstroke}{rgb}{0.800000,0.800000,0.800000}%
\pgfsetstrokecolor{currentstroke}%
\pgfsetstrokeopacity{0.800000}%
\pgfsetdash{}{0pt}%
\pgfpathmoveto{\pgfqpoint{1.288056in}{1.154445in}}%
\pgfpathlineto{\pgfqpoint{1.967778in}{1.154445in}}%
\pgfpathquadraticcurveto{\pgfqpoint{1.995556in}{1.154445in}}{\pgfqpoint{1.995556in}{1.182222in}}%
\pgfpathlineto{\pgfqpoint{1.995556in}{1.557222in}}%
\pgfpathquadraticcurveto{\pgfqpoint{1.995556in}{1.585000in}}{\pgfqpoint{1.967778in}{1.585000in}}%
\pgfpathlineto{\pgfqpoint{1.288056in}{1.585000in}}%
\pgfpathquadraticcurveto{\pgfqpoint{1.260278in}{1.585000in}}{\pgfqpoint{1.260278in}{1.557222in}}%
\pgfpathlineto{\pgfqpoint{1.260278in}{1.182222in}}%
\pgfpathquadraticcurveto{\pgfqpoint{1.260278in}{1.154445in}}{\pgfqpoint{1.288056in}{1.154445in}}%
\pgfpathlineto{\pgfqpoint{1.288056in}{1.154445in}}%
\pgfpathclose%
\pgfusepath{stroke,fill}%
\end{pgfscope}%
\begin{pgfscope}%
\pgfsetbuttcap%
\pgfsetmiterjoin%
\pgfsetlinewidth{1.003750pt}%
\definecolor{currentstroke}{rgb}{0.000000,0.000000,0.000000}%
\pgfsetstrokecolor{currentstroke}%
\pgfsetdash{}{0pt}%
\pgfpathmoveto{\pgfqpoint{1.315834in}{1.432222in}}%
\pgfpathlineto{\pgfqpoint{1.593611in}{1.432222in}}%
\pgfpathlineto{\pgfqpoint{1.593611in}{1.529444in}}%
\pgfpathlineto{\pgfqpoint{1.315834in}{1.529444in}}%
\pgfpathlineto{\pgfqpoint{1.315834in}{1.432222in}}%
\pgfpathclose%
\pgfusepath{stroke}%
\end{pgfscope}%
\begin{pgfscope}%
\definecolor{textcolor}{rgb}{0.000000,0.000000,0.000000}%
\pgfsetstrokecolor{textcolor}%
\pgfsetfillcolor{textcolor}%
\pgftext[x=1.704722in,y=1.432222in,left,base]{\color{textcolor}\rmfamily\fontsize{10.000000}{12.000000}\selectfont Neg}%
\end{pgfscope}%
\begin{pgfscope}%
\pgfsetbuttcap%
\pgfsetmiterjoin%
\definecolor{currentfill}{rgb}{0.000000,0.000000,0.000000}%
\pgfsetfillcolor{currentfill}%
\pgfsetlinewidth{0.000000pt}%
\definecolor{currentstroke}{rgb}{0.000000,0.000000,0.000000}%
\pgfsetstrokecolor{currentstroke}%
\pgfsetstrokeopacity{0.000000}%
\pgfsetdash{}{0pt}%
\pgfpathmoveto{\pgfqpoint{1.315834in}{1.236944in}}%
\pgfpathlineto{\pgfqpoint{1.593611in}{1.236944in}}%
\pgfpathlineto{\pgfqpoint{1.593611in}{1.334167in}}%
\pgfpathlineto{\pgfqpoint{1.315834in}{1.334167in}}%
\pgfpathlineto{\pgfqpoint{1.315834in}{1.236944in}}%
\pgfpathclose%
\pgfusepath{fill}%
\end{pgfscope}%
\begin{pgfscope}%
\definecolor{textcolor}{rgb}{0.000000,0.000000,0.000000}%
\pgfsetstrokecolor{textcolor}%
\pgfsetfillcolor{textcolor}%
\pgftext[x=1.704722in,y=1.236944in,left,base]{\color{textcolor}\rmfamily\fontsize{10.000000}{12.000000}\selectfont Pos}%
\end{pgfscope}%
\end{pgfpicture}%
\makeatother%
\endgroup%

  &
  \vspace{0pt} %% Creator: Matplotlib, PGF backend
%%
%% To include the figure in your LaTeX document, write
%%   \input{<filename>.pgf}
%%
%% Make sure the required packages are loaded in your preamble
%%   \usepackage{pgf}
%%
%% Also ensure that all the required font packages are loaded; for instance,
%% the lmodern package is sometimes necessary when using math font.
%%   \usepackage{lmodern}
%%
%% Figures using additional raster images can only be included by \input if
%% they are in the same directory as the main LaTeX file. For loading figures
%% from other directories you can use the `import` package
%%   \usepackage{import}
%%
%% and then include the figures with
%%   \import{<path to file>}{<filename>.pgf}
%%
%% Matplotlib used the following preamble
%%   
%%   \usepackage{fontspec}
%%   \makeatletter\@ifpackageloaded{underscore}{}{\usepackage[strings]{underscore}}\makeatother
%%
\begingroup%
\makeatletter%
\begin{pgfpicture}%
\pgfpathrectangle{\pgfpointorigin}{\pgfqpoint{2.221861in}{1.754444in}}%
\pgfusepath{use as bounding box, clip}%
\begin{pgfscope}%
\pgfsetbuttcap%
\pgfsetmiterjoin%
\definecolor{currentfill}{rgb}{1.000000,1.000000,1.000000}%
\pgfsetfillcolor{currentfill}%
\pgfsetlinewidth{0.000000pt}%
\definecolor{currentstroke}{rgb}{1.000000,1.000000,1.000000}%
\pgfsetstrokecolor{currentstroke}%
\pgfsetdash{}{0pt}%
\pgfpathmoveto{\pgfqpoint{0.000000in}{0.000000in}}%
\pgfpathlineto{\pgfqpoint{2.221861in}{0.000000in}}%
\pgfpathlineto{\pgfqpoint{2.221861in}{1.754444in}}%
\pgfpathlineto{\pgfqpoint{0.000000in}{1.754444in}}%
\pgfpathlineto{\pgfqpoint{0.000000in}{0.000000in}}%
\pgfpathclose%
\pgfusepath{fill}%
\end{pgfscope}%
\begin{pgfscope}%
\pgfsetbuttcap%
\pgfsetmiterjoin%
\definecolor{currentfill}{rgb}{1.000000,1.000000,1.000000}%
\pgfsetfillcolor{currentfill}%
\pgfsetlinewidth{0.000000pt}%
\definecolor{currentstroke}{rgb}{0.000000,0.000000,0.000000}%
\pgfsetstrokecolor{currentstroke}%
\pgfsetstrokeopacity{0.000000}%
\pgfsetdash{}{0pt}%
\pgfpathmoveto{\pgfqpoint{0.553581in}{0.499444in}}%
\pgfpathlineto{\pgfqpoint{2.103581in}{0.499444in}}%
\pgfpathlineto{\pgfqpoint{2.103581in}{1.654444in}}%
\pgfpathlineto{\pgfqpoint{0.553581in}{1.654444in}}%
\pgfpathlineto{\pgfqpoint{0.553581in}{0.499444in}}%
\pgfpathclose%
\pgfusepath{fill}%
\end{pgfscope}%
\begin{pgfscope}%
\pgfsetbuttcap%
\pgfsetroundjoin%
\definecolor{currentfill}{rgb}{0.000000,0.000000,0.000000}%
\pgfsetfillcolor{currentfill}%
\pgfsetlinewidth{0.803000pt}%
\definecolor{currentstroke}{rgb}{0.000000,0.000000,0.000000}%
\pgfsetstrokecolor{currentstroke}%
\pgfsetdash{}{0pt}%
\pgfsys@defobject{currentmarker}{\pgfqpoint{0.000000in}{-0.048611in}}{\pgfqpoint{0.000000in}{0.000000in}}{%
\pgfpathmoveto{\pgfqpoint{0.000000in}{0.000000in}}%
\pgfpathlineto{\pgfqpoint{0.000000in}{-0.048611in}}%
\pgfusepath{stroke,fill}%
}%
\begin{pgfscope}%
\pgfsys@transformshift{0.624035in}{0.499444in}%
\pgfsys@useobject{currentmarker}{}%
\end{pgfscope}%
\end{pgfscope}%
\begin{pgfscope}%
\definecolor{textcolor}{rgb}{0.000000,0.000000,0.000000}%
\pgfsetstrokecolor{textcolor}%
\pgfsetfillcolor{textcolor}%
\pgftext[x=0.624035in,y=0.402222in,,top]{\color{textcolor}\rmfamily\fontsize{10.000000}{12.000000}\selectfont \(\displaystyle {0.0}\)}%
\end{pgfscope}%
\begin{pgfscope}%
\pgfsetbuttcap%
\pgfsetroundjoin%
\definecolor{currentfill}{rgb}{0.000000,0.000000,0.000000}%
\pgfsetfillcolor{currentfill}%
\pgfsetlinewidth{0.803000pt}%
\definecolor{currentstroke}{rgb}{0.000000,0.000000,0.000000}%
\pgfsetstrokecolor{currentstroke}%
\pgfsetdash{}{0pt}%
\pgfsys@defobject{currentmarker}{\pgfqpoint{0.000000in}{-0.048611in}}{\pgfqpoint{0.000000in}{0.000000in}}{%
\pgfpathmoveto{\pgfqpoint{0.000000in}{0.000000in}}%
\pgfpathlineto{\pgfqpoint{0.000000in}{-0.048611in}}%
\pgfusepath{stroke,fill}%
}%
\begin{pgfscope}%
\pgfsys@transformshift{1.328581in}{0.499444in}%
\pgfsys@useobject{currentmarker}{}%
\end{pgfscope}%
\end{pgfscope}%
\begin{pgfscope}%
\definecolor{textcolor}{rgb}{0.000000,0.000000,0.000000}%
\pgfsetstrokecolor{textcolor}%
\pgfsetfillcolor{textcolor}%
\pgftext[x=1.328581in,y=0.402222in,,top]{\color{textcolor}\rmfamily\fontsize{10.000000}{12.000000}\selectfont \(\displaystyle {0.5}\)}%
\end{pgfscope}%
\begin{pgfscope}%
\pgfsetbuttcap%
\pgfsetroundjoin%
\definecolor{currentfill}{rgb}{0.000000,0.000000,0.000000}%
\pgfsetfillcolor{currentfill}%
\pgfsetlinewidth{0.803000pt}%
\definecolor{currentstroke}{rgb}{0.000000,0.000000,0.000000}%
\pgfsetstrokecolor{currentstroke}%
\pgfsetdash{}{0pt}%
\pgfsys@defobject{currentmarker}{\pgfqpoint{0.000000in}{-0.048611in}}{\pgfqpoint{0.000000in}{0.000000in}}{%
\pgfpathmoveto{\pgfqpoint{0.000000in}{0.000000in}}%
\pgfpathlineto{\pgfqpoint{0.000000in}{-0.048611in}}%
\pgfusepath{stroke,fill}%
}%
\begin{pgfscope}%
\pgfsys@transformshift{2.033126in}{0.499444in}%
\pgfsys@useobject{currentmarker}{}%
\end{pgfscope}%
\end{pgfscope}%
\begin{pgfscope}%
\definecolor{textcolor}{rgb}{0.000000,0.000000,0.000000}%
\pgfsetstrokecolor{textcolor}%
\pgfsetfillcolor{textcolor}%
\pgftext[x=2.033126in,y=0.402222in,,top]{\color{textcolor}\rmfamily\fontsize{10.000000}{12.000000}\selectfont \(\displaystyle {1.0}\)}%
\end{pgfscope}%
\begin{pgfscope}%
\definecolor{textcolor}{rgb}{0.000000,0.000000,0.000000}%
\pgfsetstrokecolor{textcolor}%
\pgfsetfillcolor{textcolor}%
\pgftext[x=1.328581in,y=0.223333in,,top]{\color{textcolor}\rmfamily\fontsize{10.000000}{12.000000}\selectfont False positive rate}%
\end{pgfscope}%
\begin{pgfscope}%
\pgfsetbuttcap%
\pgfsetroundjoin%
\definecolor{currentfill}{rgb}{0.000000,0.000000,0.000000}%
\pgfsetfillcolor{currentfill}%
\pgfsetlinewidth{0.803000pt}%
\definecolor{currentstroke}{rgb}{0.000000,0.000000,0.000000}%
\pgfsetstrokecolor{currentstroke}%
\pgfsetdash{}{0pt}%
\pgfsys@defobject{currentmarker}{\pgfqpoint{-0.048611in}{0.000000in}}{\pgfqpoint{-0.000000in}{0.000000in}}{%
\pgfpathmoveto{\pgfqpoint{-0.000000in}{0.000000in}}%
\pgfpathlineto{\pgfqpoint{-0.048611in}{0.000000in}}%
\pgfusepath{stroke,fill}%
}%
\begin{pgfscope}%
\pgfsys@transformshift{0.553581in}{0.551944in}%
\pgfsys@useobject{currentmarker}{}%
\end{pgfscope}%
\end{pgfscope}%
\begin{pgfscope}%
\definecolor{textcolor}{rgb}{0.000000,0.000000,0.000000}%
\pgfsetstrokecolor{textcolor}%
\pgfsetfillcolor{textcolor}%
\pgftext[x=0.278889in, y=0.503750in, left, base]{\color{textcolor}\rmfamily\fontsize{10.000000}{12.000000}\selectfont \(\displaystyle {0.0}\)}%
\end{pgfscope}%
\begin{pgfscope}%
\pgfsetbuttcap%
\pgfsetroundjoin%
\definecolor{currentfill}{rgb}{0.000000,0.000000,0.000000}%
\pgfsetfillcolor{currentfill}%
\pgfsetlinewidth{0.803000pt}%
\definecolor{currentstroke}{rgb}{0.000000,0.000000,0.000000}%
\pgfsetstrokecolor{currentstroke}%
\pgfsetdash{}{0pt}%
\pgfsys@defobject{currentmarker}{\pgfqpoint{-0.048611in}{0.000000in}}{\pgfqpoint{-0.000000in}{0.000000in}}{%
\pgfpathmoveto{\pgfqpoint{-0.000000in}{0.000000in}}%
\pgfpathlineto{\pgfqpoint{-0.048611in}{0.000000in}}%
\pgfusepath{stroke,fill}%
}%
\begin{pgfscope}%
\pgfsys@transformshift{0.553581in}{1.076944in}%
\pgfsys@useobject{currentmarker}{}%
\end{pgfscope}%
\end{pgfscope}%
\begin{pgfscope}%
\definecolor{textcolor}{rgb}{0.000000,0.000000,0.000000}%
\pgfsetstrokecolor{textcolor}%
\pgfsetfillcolor{textcolor}%
\pgftext[x=0.278889in, y=1.028750in, left, base]{\color{textcolor}\rmfamily\fontsize{10.000000}{12.000000}\selectfont \(\displaystyle {0.5}\)}%
\end{pgfscope}%
\begin{pgfscope}%
\pgfsetbuttcap%
\pgfsetroundjoin%
\definecolor{currentfill}{rgb}{0.000000,0.000000,0.000000}%
\pgfsetfillcolor{currentfill}%
\pgfsetlinewidth{0.803000pt}%
\definecolor{currentstroke}{rgb}{0.000000,0.000000,0.000000}%
\pgfsetstrokecolor{currentstroke}%
\pgfsetdash{}{0pt}%
\pgfsys@defobject{currentmarker}{\pgfqpoint{-0.048611in}{0.000000in}}{\pgfqpoint{-0.000000in}{0.000000in}}{%
\pgfpathmoveto{\pgfqpoint{-0.000000in}{0.000000in}}%
\pgfpathlineto{\pgfqpoint{-0.048611in}{0.000000in}}%
\pgfusepath{stroke,fill}%
}%
\begin{pgfscope}%
\pgfsys@transformshift{0.553581in}{1.601944in}%
\pgfsys@useobject{currentmarker}{}%
\end{pgfscope}%
\end{pgfscope}%
\begin{pgfscope}%
\definecolor{textcolor}{rgb}{0.000000,0.000000,0.000000}%
\pgfsetstrokecolor{textcolor}%
\pgfsetfillcolor{textcolor}%
\pgftext[x=0.278889in, y=1.553750in, left, base]{\color{textcolor}\rmfamily\fontsize{10.000000}{12.000000}\selectfont \(\displaystyle {1.0}\)}%
\end{pgfscope}%
\begin{pgfscope}%
\definecolor{textcolor}{rgb}{0.000000,0.000000,0.000000}%
\pgfsetstrokecolor{textcolor}%
\pgfsetfillcolor{textcolor}%
\pgftext[x=0.223333in,y=1.076944in,,bottom,rotate=90.000000]{\color{textcolor}\rmfamily\fontsize{10.000000}{12.000000}\selectfont True positive rate}%
\end{pgfscope}%
\begin{pgfscope}%
\pgfpathrectangle{\pgfqpoint{0.553581in}{0.499444in}}{\pgfqpoint{1.550000in}{1.155000in}}%
\pgfusepath{clip}%
\pgfsetbuttcap%
\pgfsetroundjoin%
\pgfsetlinewidth{1.505625pt}%
\definecolor{currentstroke}{rgb}{0.000000,0.000000,0.000000}%
\pgfsetstrokecolor{currentstroke}%
\pgfsetdash{{5.550000pt}{2.400000pt}}{0.000000pt}%
\pgfpathmoveto{\pgfqpoint{0.624035in}{0.551944in}}%
\pgfpathlineto{\pgfqpoint{2.033126in}{1.601944in}}%
\pgfusepath{stroke}%
\end{pgfscope}%
\begin{pgfscope}%
\pgfpathrectangle{\pgfqpoint{0.553581in}{0.499444in}}{\pgfqpoint{1.550000in}{1.155000in}}%
\pgfusepath{clip}%
\pgfsetrectcap%
\pgfsetroundjoin%
\pgfsetlinewidth{1.505625pt}%
\definecolor{currentstroke}{rgb}{0.000000,0.000000,0.000000}%
\pgfsetstrokecolor{currentstroke}%
\pgfsetdash{}{0pt}%
\pgfpathmoveto{\pgfqpoint{0.624035in}{0.551944in}}%
\pgfpathlineto{\pgfqpoint{0.626119in}{0.552260in}}%
\pgfpathlineto{\pgfqpoint{0.627222in}{0.560898in}}%
\pgfpathlineto{\pgfqpoint{0.627858in}{0.562002in}}%
\pgfpathlineto{\pgfqpoint{0.628904in}{0.564132in}}%
\pgfpathlineto{\pgfqpoint{0.629465in}{0.565236in}}%
\pgfpathlineto{\pgfqpoint{0.630521in}{0.567918in}}%
\pgfpathlineto{\pgfqpoint{0.631138in}{0.569023in}}%
\pgfpathlineto{\pgfqpoint{0.632241in}{0.573085in}}%
\pgfpathlineto{\pgfqpoint{0.632652in}{0.574190in}}%
\pgfpathlineto{\pgfqpoint{0.633755in}{0.580264in}}%
\pgfpathlineto{\pgfqpoint{0.634166in}{0.581171in}}%
\pgfpathlineto{\pgfqpoint{0.635269in}{0.586378in}}%
\pgfpathlineto{\pgfqpoint{0.635447in}{0.587482in}}%
\pgfpathlineto{\pgfqpoint{0.636531in}{0.593359in}}%
\pgfpathlineto{\pgfqpoint{0.636662in}{0.594384in}}%
\pgfpathlineto{\pgfqpoint{0.637764in}{0.600222in}}%
\pgfpathlineto{\pgfqpoint{0.637979in}{0.601287in}}%
\pgfpathlineto{\pgfqpoint{0.639082in}{0.607992in}}%
\pgfpathlineto{\pgfqpoint{0.639278in}{0.608978in}}%
\pgfpathlineto{\pgfqpoint{0.640372in}{0.614027in}}%
\pgfpathlineto{\pgfqpoint{0.640587in}{0.615131in}}%
\pgfpathlineto{\pgfqpoint{0.641690in}{0.622783in}}%
\pgfpathlineto{\pgfqpoint{0.641867in}{0.623887in}}%
\pgfpathlineto{\pgfqpoint{0.642970in}{0.631381in}}%
\pgfpathlineto{\pgfqpoint{0.643092in}{0.632446in}}%
\pgfpathlineto{\pgfqpoint{0.644194in}{0.641794in}}%
\pgfpathlineto{\pgfqpoint{0.644316in}{0.642741in}}%
\pgfpathlineto{\pgfqpoint{0.645419in}{0.649841in}}%
\pgfpathlineto{\pgfqpoint{0.645559in}{0.650866in}}%
\pgfpathlineto{\pgfqpoint{0.646662in}{0.658557in}}%
\pgfpathlineto{\pgfqpoint{0.646886in}{0.659662in}}%
\pgfpathlineto{\pgfqpoint{0.647979in}{0.665736in}}%
\pgfpathlineto{\pgfqpoint{0.648232in}{0.666840in}}%
\pgfpathlineto{\pgfqpoint{0.649335in}{0.674216in}}%
\pgfpathlineto{\pgfqpoint{0.649465in}{0.675281in}}%
\pgfpathlineto{\pgfqpoint{0.650568in}{0.683091in}}%
\pgfpathlineto{\pgfqpoint{0.650736in}{0.684116in}}%
\pgfpathlineto{\pgfqpoint{0.651830in}{0.692754in}}%
\pgfpathlineto{\pgfqpoint{0.652008in}{0.693819in}}%
\pgfpathlineto{\pgfqpoint{0.653073in}{0.701471in}}%
\pgfpathlineto{\pgfqpoint{0.653260in}{0.702496in}}%
\pgfpathlineto{\pgfqpoint{0.654353in}{0.710267in}}%
\pgfpathlineto{\pgfqpoint{0.654578in}{0.711331in}}%
\pgfpathlineto{\pgfqpoint{0.655671in}{0.720877in}}%
\pgfpathlineto{\pgfqpoint{0.655793in}{0.721863in}}%
\pgfpathlineto{\pgfqpoint{0.656895in}{0.730422in}}%
\pgfpathlineto{\pgfqpoint{0.657166in}{0.731526in}}%
\pgfpathlineto{\pgfqpoint{0.658269in}{0.740677in}}%
\pgfpathlineto{\pgfqpoint{0.658372in}{0.741663in}}%
\pgfpathlineto{\pgfqpoint{0.659475in}{0.749472in}}%
\pgfpathlineto{\pgfqpoint{0.659615in}{0.750577in}}%
\pgfpathlineto{\pgfqpoint{0.660671in}{0.759136in}}%
\pgfpathlineto{\pgfqpoint{0.660914in}{0.760082in}}%
\pgfpathlineto{\pgfqpoint{0.662017in}{0.767419in}}%
\pgfpathlineto{\pgfqpoint{0.662185in}{0.768365in}}%
\pgfpathlineto{\pgfqpoint{0.663279in}{0.776806in}}%
\pgfpathlineto{\pgfqpoint{0.663466in}{0.777832in}}%
\pgfpathlineto{\pgfqpoint{0.664559in}{0.785365in}}%
\pgfpathlineto{\pgfqpoint{0.664886in}{0.786430in}}%
\pgfpathlineto{\pgfqpoint{0.665989in}{0.793530in}}%
\pgfpathlineto{\pgfqpoint{0.666204in}{0.794634in}}%
\pgfpathlineto{\pgfqpoint{0.667307in}{0.802720in}}%
\pgfpathlineto{\pgfqpoint{0.667522in}{0.803785in}}%
\pgfpathlineto{\pgfqpoint{0.668625in}{0.809110in}}%
\pgfpathlineto{\pgfqpoint{0.668802in}{0.810175in}}%
\pgfpathlineto{\pgfqpoint{0.669905in}{0.815618in}}%
\pgfpathlineto{\pgfqpoint{0.670120in}{0.816683in}}%
\pgfpathlineto{\pgfqpoint{0.671223in}{0.824847in}}%
\pgfpathlineto{\pgfqpoint{0.671475in}{0.825873in}}%
\pgfpathlineto{\pgfqpoint{0.672559in}{0.834274in}}%
\pgfpathlineto{\pgfqpoint{0.672727in}{0.835378in}}%
\pgfpathlineto{\pgfqpoint{0.673830in}{0.842439in}}%
\pgfpathlineto{\pgfqpoint{0.674036in}{0.843425in}}%
\pgfpathlineto{\pgfqpoint{0.675129in}{0.849735in}}%
\pgfpathlineto{\pgfqpoint{0.675419in}{0.850840in}}%
\pgfpathlineto{\pgfqpoint{0.676503in}{0.856520in}}%
\pgfpathlineto{\pgfqpoint{0.676681in}{0.857584in}}%
\pgfpathlineto{\pgfqpoint{0.677784in}{0.863225in}}%
\pgfpathlineto{\pgfqpoint{0.678017in}{0.864329in}}%
\pgfpathlineto{\pgfqpoint{0.679120in}{0.871508in}}%
\pgfpathlineto{\pgfqpoint{0.679372in}{0.872612in}}%
\pgfpathlineto{\pgfqpoint{0.680475in}{0.878450in}}%
\pgfpathlineto{\pgfqpoint{0.680681in}{0.879554in}}%
\pgfpathlineto{\pgfqpoint{0.681784in}{0.884839in}}%
\pgfpathlineto{\pgfqpoint{0.681914in}{0.885865in}}%
\pgfpathlineto{\pgfqpoint{0.683017in}{0.892531in}}%
\pgfpathlineto{\pgfqpoint{0.683270in}{0.893595in}}%
\pgfpathlineto{\pgfqpoint{0.684344in}{0.898171in}}%
\pgfpathlineto{\pgfqpoint{0.684615in}{0.899157in}}%
\pgfpathlineto{\pgfqpoint{0.685718in}{0.905113in}}%
\pgfpathlineto{\pgfqpoint{0.686036in}{0.906217in}}%
\pgfpathlineto{\pgfqpoint{0.687129in}{0.911660in}}%
\pgfpathlineto{\pgfqpoint{0.687428in}{0.912765in}}%
\pgfpathlineto{\pgfqpoint{0.688531in}{0.917419in}}%
\pgfpathlineto{\pgfqpoint{0.688718in}{0.918523in}}%
\pgfpathlineto{\pgfqpoint{0.689821in}{0.923572in}}%
\pgfpathlineto{\pgfqpoint{0.690017in}{0.924676in}}%
\pgfpathlineto{\pgfqpoint{0.691120in}{0.930632in}}%
\pgfpathlineto{\pgfqpoint{0.691391in}{0.931736in}}%
\pgfpathlineto{\pgfqpoint{0.692494in}{0.937692in}}%
\pgfpathlineto{\pgfqpoint{0.692709in}{0.938797in}}%
\pgfpathlineto{\pgfqpoint{0.693812in}{0.943924in}}%
\pgfpathlineto{\pgfqpoint{0.694045in}{0.944910in}}%
\pgfpathlineto{\pgfqpoint{0.695148in}{0.949288in}}%
\pgfpathlineto{\pgfqpoint{0.695401in}{0.950393in}}%
\pgfpathlineto{\pgfqpoint{0.696494in}{0.955441in}}%
\pgfpathlineto{\pgfqpoint{0.696849in}{0.956467in}}%
\pgfpathlineto{\pgfqpoint{0.697924in}{0.961516in}}%
\pgfpathlineto{\pgfqpoint{0.698260in}{0.962581in}}%
\pgfpathlineto{\pgfqpoint{0.699345in}{0.967314in}}%
\pgfpathlineto{\pgfqpoint{0.699653in}{0.968379in}}%
\pgfpathlineto{\pgfqpoint{0.700746in}{0.972204in}}%
\pgfpathlineto{\pgfqpoint{0.701008in}{0.973112in}}%
\pgfpathlineto{\pgfqpoint{0.702111in}{0.978791in}}%
\pgfpathlineto{\pgfqpoint{0.702410in}{0.979896in}}%
\pgfpathlineto{\pgfqpoint{0.703503in}{0.984944in}}%
\pgfpathlineto{\pgfqpoint{0.703831in}{0.986049in}}%
\pgfpathlineto{\pgfqpoint{0.704924in}{0.990664in}}%
\pgfpathlineto{\pgfqpoint{0.705223in}{0.991610in}}%
\pgfpathlineto{\pgfqpoint{0.706326in}{0.996541in}}%
\pgfpathlineto{\pgfqpoint{0.706690in}{0.997527in}}%
\pgfpathlineto{\pgfqpoint{0.707765in}{1.002970in}}%
\pgfpathlineto{\pgfqpoint{0.708036in}{1.004074in}}%
\pgfpathlineto{\pgfqpoint{0.709130in}{1.008373in}}%
\pgfpathlineto{\pgfqpoint{0.709391in}{1.009478in}}%
\pgfpathlineto{\pgfqpoint{0.710485in}{1.012988in}}%
\pgfpathlineto{\pgfqpoint{0.710868in}{1.014092in}}%
\pgfpathlineto{\pgfqpoint{0.711961in}{1.018747in}}%
\pgfpathlineto{\pgfqpoint{0.712242in}{1.019851in}}%
\pgfpathlineto{\pgfqpoint{0.713289in}{1.023677in}}%
\pgfpathlineto{\pgfqpoint{0.713793in}{1.024742in}}%
\pgfpathlineto{\pgfqpoint{0.714877in}{1.028568in}}%
\pgfpathlineto{\pgfqpoint{0.715186in}{1.029593in}}%
\pgfpathlineto{\pgfqpoint{0.716251in}{1.033577in}}%
\pgfpathlineto{\pgfqpoint{0.716522in}{1.034681in}}%
\pgfpathlineto{\pgfqpoint{0.717606in}{1.039493in}}%
\pgfpathlineto{\pgfqpoint{0.717924in}{1.040480in}}%
\pgfpathlineto{\pgfqpoint{0.719018in}{1.043990in}}%
\pgfpathlineto{\pgfqpoint{0.719373in}{1.045055in}}%
\pgfpathlineto{\pgfqpoint{0.720457in}{1.048881in}}%
\pgfpathlineto{\pgfqpoint{0.720803in}{1.049946in}}%
\pgfpathlineto{\pgfqpoint{0.721887in}{1.053141in}}%
\pgfpathlineto{\pgfqpoint{0.722139in}{1.054245in}}%
\pgfpathlineto{\pgfqpoint{0.723233in}{1.058268in}}%
\pgfpathlineto{\pgfqpoint{0.723635in}{1.059373in}}%
\pgfpathlineto{\pgfqpoint{0.724709in}{1.063001in}}%
\pgfpathlineto{\pgfqpoint{0.725177in}{1.064106in}}%
\pgfpathlineto{\pgfqpoint{0.726186in}{1.067892in}}%
\pgfpathlineto{\pgfqpoint{0.726541in}{1.068996in}}%
\pgfpathlineto{\pgfqpoint{0.727644in}{1.072231in}}%
\pgfpathlineto{\pgfqpoint{0.728036in}{1.073335in}}%
\pgfpathlineto{\pgfqpoint{0.729139in}{1.075899in}}%
\pgfpathlineto{\pgfqpoint{0.729532in}{1.076964in}}%
\pgfpathlineto{\pgfqpoint{0.730607in}{1.080632in}}%
\pgfpathlineto{\pgfqpoint{0.731027in}{1.081618in}}%
\pgfpathlineto{\pgfqpoint{0.732111in}{1.084655in}}%
\pgfpathlineto{\pgfqpoint{0.732364in}{1.085681in}}%
\pgfpathlineto{\pgfqpoint{0.733429in}{1.089270in}}%
\pgfpathlineto{\pgfqpoint{0.733850in}{1.090374in}}%
\pgfpathlineto{\pgfqpoint{0.734924in}{1.093372in}}%
\pgfpathlineto{\pgfqpoint{0.735326in}{1.094476in}}%
\pgfpathlineto{\pgfqpoint{0.736429in}{1.097435in}}%
\pgfpathlineto{\pgfqpoint{0.736934in}{1.098500in}}%
\pgfpathlineto{\pgfqpoint{0.738009in}{1.101615in}}%
\pgfpathlineto{\pgfqpoint{0.738513in}{1.102720in}}%
\pgfpathlineto{\pgfqpoint{0.739569in}{1.105836in}}%
\pgfpathlineto{\pgfqpoint{0.739990in}{1.106901in}}%
\pgfpathlineto{\pgfqpoint{0.741093in}{1.109938in}}%
\pgfpathlineto{\pgfqpoint{0.741766in}{1.111003in}}%
\pgfpathlineto{\pgfqpoint{0.742831in}{1.114158in}}%
\pgfpathlineto{\pgfqpoint{0.743336in}{1.115263in}}%
\pgfpathlineto{\pgfqpoint{0.744382in}{1.118181in}}%
\pgfpathlineto{\pgfqpoint{0.755205in}{1.119286in}}%
\pgfpathlineto{\pgfqpoint{0.756298in}{1.121771in}}%
\pgfpathlineto{\pgfqpoint{0.756775in}{1.122875in}}%
\pgfpathlineto{\pgfqpoint{0.757878in}{1.125912in}}%
\pgfpathlineto{\pgfqpoint{0.758373in}{1.127016in}}%
\pgfpathlineto{\pgfqpoint{0.759457in}{1.129699in}}%
\pgfpathlineto{\pgfqpoint{0.759766in}{1.130803in}}%
\pgfpathlineto{\pgfqpoint{0.760841in}{1.134116in}}%
\pgfpathlineto{\pgfqpoint{0.761289in}{1.135063in}}%
\pgfpathlineto{\pgfqpoint{0.762392in}{1.137508in}}%
\pgfpathlineto{\pgfqpoint{0.762756in}{1.138613in}}%
\pgfpathlineto{\pgfqpoint{0.763859in}{1.141926in}}%
\pgfpathlineto{\pgfqpoint{0.764299in}{1.143030in}}%
\pgfpathlineto{\pgfqpoint{0.765401in}{1.145160in}}%
\pgfpathlineto{\pgfqpoint{0.765990in}{1.146264in}}%
\pgfpathlineto{\pgfqpoint{0.767046in}{1.149144in}}%
\pgfpathlineto{\pgfqpoint{0.767514in}{1.150209in}}%
\pgfpathlineto{\pgfqpoint{0.768616in}{1.152694in}}%
\pgfpathlineto{\pgfqpoint{0.769168in}{1.153798in}}%
\pgfpathlineto{\pgfqpoint{0.770243in}{1.156914in}}%
\pgfpathlineto{\pgfqpoint{0.770887in}{1.158018in}}%
\pgfpathlineto{\pgfqpoint{0.771972in}{1.160622in}}%
\pgfpathlineto{\pgfqpoint{0.772785in}{1.161686in}}%
\pgfpathlineto{\pgfqpoint{0.773869in}{1.163777in}}%
\pgfpathlineto{\pgfqpoint{0.774327in}{1.164881in}}%
\pgfpathlineto{\pgfqpoint{0.775401in}{1.167918in}}%
\pgfpathlineto{\pgfqpoint{0.775869in}{1.169023in}}%
\pgfpathlineto{\pgfqpoint{0.776906in}{1.171823in}}%
\pgfpathlineto{\pgfqpoint{0.777504in}{1.172928in}}%
\pgfpathlineto{\pgfqpoint{0.778579in}{1.175767in}}%
\pgfpathlineto{\pgfqpoint{0.779037in}{1.176872in}}%
\pgfpathlineto{\pgfqpoint{0.780130in}{1.178489in}}%
\pgfpathlineto{\pgfqpoint{0.780663in}{1.179515in}}%
\pgfpathlineto{\pgfqpoint{0.781766in}{1.182275in}}%
\pgfpathlineto{\pgfqpoint{0.782243in}{1.183380in}}%
\pgfpathlineto{\pgfqpoint{0.783317in}{1.185707in}}%
\pgfpathlineto{\pgfqpoint{0.783747in}{1.186732in}}%
\pgfpathlineto{\pgfqpoint{0.784831in}{1.188902in}}%
\pgfpathlineto{\pgfqpoint{0.785626in}{1.190006in}}%
\pgfpathlineto{\pgfqpoint{0.786691in}{1.192097in}}%
\pgfpathlineto{\pgfqpoint{0.787289in}{1.193201in}}%
\pgfpathlineto{\pgfqpoint{0.788364in}{1.194897in}}%
\pgfpathlineto{\pgfqpoint{0.789196in}{1.196001in}}%
\pgfpathlineto{\pgfqpoint{0.790299in}{1.198053in}}%
\pgfpathlineto{\pgfqpoint{0.791046in}{1.199117in}}%
\pgfpathlineto{\pgfqpoint{0.792131in}{1.201326in}}%
\pgfpathlineto{\pgfqpoint{0.792645in}{1.202391in}}%
\pgfpathlineto{\pgfqpoint{0.793738in}{1.204048in}}%
\pgfpathlineto{\pgfqpoint{0.794561in}{1.205152in}}%
\pgfpathlineto{\pgfqpoint{0.795654in}{1.206967in}}%
\pgfpathlineto{\pgfqpoint{0.796476in}{1.208071in}}%
\pgfpathlineto{\pgfqpoint{0.797579in}{1.210201in}}%
\pgfpathlineto{\pgfqpoint{0.798065in}{1.211305in}}%
\pgfpathlineto{\pgfqpoint{0.799131in}{1.213080in}}%
\pgfpathlineto{\pgfqpoint{0.800000in}{1.214106in}}%
\pgfpathlineto{\pgfqpoint{0.801084in}{1.216275in}}%
\pgfpathlineto{\pgfqpoint{0.801570in}{1.217340in}}%
\pgfpathlineto{\pgfqpoint{0.802645in}{1.219115in}}%
\pgfpathlineto{\pgfqpoint{0.803346in}{1.220219in}}%
\pgfpathlineto{\pgfqpoint{0.804383in}{1.222112in}}%
\pgfpathlineto{\pgfqpoint{0.804953in}{1.223099in}}%
\pgfpathlineto{\pgfqpoint{0.806047in}{1.225978in}}%
\pgfpathlineto{\pgfqpoint{0.806551in}{1.227082in}}%
\pgfpathlineto{\pgfqpoint{0.807589in}{1.229094in}}%
\pgfpathlineto{\pgfqpoint{0.808327in}{1.230198in}}%
\pgfpathlineto{\pgfqpoint{0.809430in}{1.232486in}}%
\pgfpathlineto{\pgfqpoint{0.810318in}{1.233590in}}%
\pgfpathlineto{\pgfqpoint{0.811411in}{1.235286in}}%
\pgfpathlineto{\pgfqpoint{0.812103in}{1.236391in}}%
\pgfpathlineto{\pgfqpoint{0.813084in}{1.237850in}}%
\pgfpathlineto{\pgfqpoint{0.813935in}{1.238954in}}%
\pgfpathlineto{\pgfqpoint{0.815009in}{1.240808in}}%
\pgfpathlineto{\pgfqpoint{0.815888in}{1.241873in}}%
\pgfpathlineto{\pgfqpoint{0.816991in}{1.243766in}}%
\pgfpathlineto{\pgfqpoint{0.817542in}{1.244871in}}%
\pgfpathlineto{\pgfqpoint{0.818645in}{1.246646in}}%
\pgfpathlineto{\pgfqpoint{0.819383in}{1.247750in}}%
\pgfpathlineto{\pgfqpoint{0.820477in}{1.249288in}}%
\pgfpathlineto{\pgfqpoint{0.821355in}{1.250393in}}%
\pgfpathlineto{\pgfqpoint{0.822458in}{1.252957in}}%
\pgfpathlineto{\pgfqpoint{0.823252in}{1.254061in}}%
\pgfpathlineto{\pgfqpoint{0.824337in}{1.255599in}}%
\pgfpathlineto{\pgfqpoint{0.825168in}{1.256704in}}%
\pgfpathlineto{\pgfqpoint{0.826271in}{1.258123in}}%
\pgfpathlineto{\pgfqpoint{0.826897in}{1.259188in}}%
\pgfpathlineto{\pgfqpoint{0.827982in}{1.260845in}}%
\pgfpathlineto{\pgfqpoint{0.829150in}{1.261949in}}%
\pgfpathlineto{\pgfqpoint{0.830196in}{1.262935in}}%
\pgfpathlineto{\pgfqpoint{0.831028in}{1.263922in}}%
\pgfpathlineto{\pgfqpoint{0.832131in}{1.265854in}}%
\pgfpathlineto{\pgfqpoint{0.832776in}{1.266959in}}%
\pgfpathlineto{\pgfqpoint{0.833860in}{1.268418in}}%
\pgfpathlineto{\pgfqpoint{0.834776in}{1.269522in}}%
\pgfpathlineto{\pgfqpoint{0.835804in}{1.270903in}}%
\pgfpathlineto{\pgfqpoint{0.836748in}{1.271928in}}%
\pgfpathlineto{\pgfqpoint{0.837804in}{1.273269in}}%
\pgfpathlineto{\pgfqpoint{0.838617in}{1.274374in}}%
\pgfpathlineto{\pgfqpoint{0.839692in}{1.276030in}}%
\pgfpathlineto{\pgfqpoint{0.840692in}{1.277135in}}%
\pgfpathlineto{\pgfqpoint{0.841655in}{1.278239in}}%
\pgfpathlineto{\pgfqpoint{0.842570in}{1.279344in}}%
\pgfpathlineto{\pgfqpoint{0.843673in}{1.280566in}}%
\pgfpathlineto{\pgfqpoint{0.844589in}{1.281671in}}%
\pgfpathlineto{\pgfqpoint{0.845692in}{1.283288in}}%
\pgfpathlineto{\pgfqpoint{0.846785in}{1.284392in}}%
\pgfpathlineto{\pgfqpoint{0.847851in}{1.285378in}}%
\pgfpathlineto{\pgfqpoint{0.848795in}{1.286483in}}%
\pgfpathlineto{\pgfqpoint{0.849888in}{1.287390in}}%
\pgfpathlineto{\pgfqpoint{0.850757in}{1.288494in}}%
\pgfpathlineto{\pgfqpoint{0.851814in}{1.290111in}}%
\pgfpathlineto{\pgfqpoint{0.852860in}{1.291176in}}%
\pgfpathlineto{\pgfqpoint{0.853832in}{1.292754in}}%
\pgfpathlineto{\pgfqpoint{0.854458in}{1.293858in}}%
\pgfpathlineto{\pgfqpoint{0.855524in}{1.295239in}}%
\pgfpathlineto{\pgfqpoint{0.856739in}{1.296343in}}%
\pgfpathlineto{\pgfqpoint{0.857804in}{1.297763in}}%
\pgfpathlineto{\pgfqpoint{0.858730in}{1.298789in}}%
\pgfpathlineto{\pgfqpoint{0.859823in}{1.299972in}}%
\pgfpathlineto{\pgfqpoint{0.860402in}{1.301076in}}%
\pgfpathlineto{\pgfqpoint{0.861430in}{1.302417in}}%
\pgfpathlineto{\pgfqpoint{0.862169in}{1.303482in}}%
\pgfpathlineto{\pgfqpoint{0.863272in}{1.304468in}}%
\pgfpathlineto{\pgfqpoint{0.864169in}{1.305573in}}%
\pgfpathlineto{\pgfqpoint{0.865272in}{1.306993in}}%
\pgfpathlineto{\pgfqpoint{0.866337in}{1.308058in}}%
\pgfpathlineto{\pgfqpoint{0.867431in}{1.309833in}}%
\pgfpathlineto{\pgfqpoint{0.868459in}{1.310898in}}%
\pgfpathlineto{\pgfqpoint{0.869561in}{1.312436in}}%
\pgfpathlineto{\pgfqpoint{0.870487in}{1.313540in}}%
\pgfpathlineto{\pgfqpoint{0.871589in}{1.314763in}}%
\pgfpathlineto{\pgfqpoint{0.872571in}{1.315867in}}%
\pgfpathlineto{\pgfqpoint{0.873636in}{1.316696in}}%
\pgfpathlineto{\pgfqpoint{0.874328in}{1.317761in}}%
\pgfpathlineto{\pgfqpoint{0.875347in}{1.318983in}}%
\pgfpathlineto{\pgfqpoint{0.876814in}{1.320088in}}%
\pgfpathlineto{\pgfqpoint{0.877907in}{1.321113in}}%
\pgfpathlineto{\pgfqpoint{0.879104in}{1.322218in}}%
\pgfpathlineto{\pgfqpoint{0.880206in}{1.323361in}}%
\pgfpathlineto{\pgfqpoint{0.881281in}{1.324387in}}%
\pgfpathlineto{\pgfqpoint{0.882347in}{1.325570in}}%
\pgfpathlineto{\pgfqpoint{0.883664in}{1.326675in}}%
\pgfpathlineto{\pgfqpoint{0.884758in}{1.328252in}}%
\pgfpathlineto{\pgfqpoint{0.886440in}{1.329357in}}%
\pgfpathlineto{\pgfqpoint{0.887496in}{1.330382in}}%
\pgfpathlineto{\pgfqpoint{0.888805in}{1.331487in}}%
\pgfpathlineto{\pgfqpoint{0.889879in}{1.332788in}}%
\pgfpathlineto{\pgfqpoint{0.890898in}{1.333893in}}%
\pgfpathlineto{\pgfqpoint{0.891954in}{1.334681in}}%
\pgfpathlineto{\pgfqpoint{0.893337in}{1.335786in}}%
\pgfpathlineto{\pgfqpoint{0.894384in}{1.337009in}}%
\pgfpathlineto{\pgfqpoint{0.895515in}{1.338074in}}%
\pgfpathlineto{\pgfqpoint{0.896562in}{1.339178in}}%
\pgfpathlineto{\pgfqpoint{0.897562in}{1.340282in}}%
\pgfpathlineto{\pgfqpoint{0.898665in}{1.341189in}}%
\pgfpathlineto{\pgfqpoint{0.899945in}{1.342294in}}%
\pgfpathlineto{\pgfqpoint{0.900992in}{1.343162in}}%
\pgfpathlineto{\pgfqpoint{0.902225in}{1.344266in}}%
\pgfpathlineto{\pgfqpoint{0.903328in}{1.345568in}}%
\pgfpathlineto{\pgfqpoint{0.904319in}{1.346672in}}%
\pgfpathlineto{\pgfqpoint{0.905272in}{1.347776in}}%
\pgfpathlineto{\pgfqpoint{0.906375in}{1.348881in}}%
\pgfpathlineto{\pgfqpoint{0.907403in}{1.349946in}}%
\pgfpathlineto{\pgfqpoint{0.909300in}{1.351050in}}%
\pgfpathlineto{\pgfqpoint{0.910347in}{1.351839in}}%
\pgfpathlineto{\pgfqpoint{0.911637in}{1.352943in}}%
\pgfpathlineto{\pgfqpoint{0.912674in}{1.353969in}}%
\pgfpathlineto{\pgfqpoint{0.914160in}{1.355073in}}%
\pgfpathlineto{\pgfqpoint{0.915235in}{1.355862in}}%
\pgfpathlineto{\pgfqpoint{0.916730in}{1.356967in}}%
\pgfpathlineto{\pgfqpoint{0.917646in}{1.357558in}}%
\pgfpathlineto{\pgfqpoint{0.919235in}{1.358623in}}%
\pgfpathlineto{\pgfqpoint{0.920216in}{1.359885in}}%
\pgfpathlineto{\pgfqpoint{0.921226in}{1.360950in}}%
\pgfpathlineto{\pgfqpoint{0.922244in}{1.361976in}}%
\pgfpathlineto{\pgfqpoint{0.924048in}{1.363080in}}%
\pgfpathlineto{\pgfqpoint{0.925085in}{1.363751in}}%
\pgfpathlineto{\pgfqpoint{0.926366in}{1.364855in}}%
\pgfpathlineto{\pgfqpoint{0.927431in}{1.365407in}}%
\pgfpathlineto{\pgfqpoint{0.929095in}{1.366512in}}%
\pgfpathlineto{\pgfqpoint{0.930114in}{1.367300in}}%
\pgfpathlineto{\pgfqpoint{0.931095in}{1.368405in}}%
\pgfpathlineto{\pgfqpoint{0.932179in}{1.369351in}}%
\pgfpathlineto{\pgfqpoint{0.933151in}{1.370456in}}%
\pgfpathlineto{\pgfqpoint{0.934160in}{1.371205in}}%
\pgfpathlineto{\pgfqpoint{0.936646in}{1.372310in}}%
\pgfpathlineto{\pgfqpoint{0.937749in}{1.373020in}}%
\pgfpathlineto{\pgfqpoint{0.939497in}{1.374124in}}%
\pgfpathlineto{\pgfqpoint{0.940431in}{1.374992in}}%
\pgfpathlineto{\pgfqpoint{0.942104in}{1.376096in}}%
\pgfpathlineto{\pgfqpoint{0.943114in}{1.376924in}}%
\pgfpathlineto{\pgfqpoint{0.945030in}{1.377989in}}%
\pgfpathlineto{\pgfqpoint{0.946123in}{1.378936in}}%
\pgfpathlineto{\pgfqpoint{0.947880in}{1.380040in}}%
\pgfpathlineto{\pgfqpoint{0.948964in}{1.381145in}}%
\pgfpathlineto{\pgfqpoint{0.950693in}{1.382249in}}%
\pgfpathlineto{\pgfqpoint{0.951740in}{1.383275in}}%
\pgfpathlineto{\pgfqpoint{0.953469in}{1.384379in}}%
\pgfpathlineto{\pgfqpoint{0.954562in}{1.385207in}}%
\pgfpathlineto{\pgfqpoint{0.956347in}{1.386312in}}%
\pgfpathlineto{\pgfqpoint{0.957404in}{1.387022in}}%
\pgfpathlineto{\pgfqpoint{0.958712in}{1.388126in}}%
\pgfpathlineto{\pgfqpoint{0.959805in}{1.388678in}}%
\pgfpathlineto{\pgfqpoint{0.961890in}{1.389783in}}%
\pgfpathlineto{\pgfqpoint{0.962964in}{1.390453in}}%
\pgfpathlineto{\pgfqpoint{0.964450in}{1.391518in}}%
\pgfpathlineto{\pgfqpoint{0.965357in}{1.391952in}}%
\pgfpathlineto{\pgfqpoint{0.967450in}{1.393056in}}%
\pgfpathlineto{\pgfqpoint{0.968535in}{1.393806in}}%
\pgfpathlineto{\pgfqpoint{0.970264in}{1.394910in}}%
\pgfpathlineto{\pgfqpoint{0.971348in}{1.395699in}}%
\pgfpathlineto{\pgfqpoint{0.972964in}{1.396764in}}%
\pgfpathlineto{\pgfqpoint{0.974011in}{1.397711in}}%
\pgfpathlineto{\pgfqpoint{0.976413in}{1.398815in}}%
\pgfpathlineto{\pgfqpoint{0.977441in}{1.399564in}}%
\pgfpathlineto{\pgfqpoint{0.979544in}{1.400669in}}%
\pgfpathlineto{\pgfqpoint{0.980628in}{1.401339in}}%
\pgfpathlineto{\pgfqpoint{0.982264in}{1.402444in}}%
\pgfpathlineto{\pgfqpoint{0.983320in}{1.403075in}}%
\pgfpathlineto{\pgfqpoint{0.985254in}{1.404179in}}%
\pgfpathlineto{\pgfqpoint{0.986357in}{1.404731in}}%
\pgfpathlineto{\pgfqpoint{0.988282in}{1.405836in}}%
\pgfpathlineto{\pgfqpoint{0.989385in}{1.406546in}}%
\pgfpathlineto{\pgfqpoint{0.992666in}{1.407650in}}%
\pgfpathlineto{\pgfqpoint{0.993759in}{1.408676in}}%
\pgfpathlineto{\pgfqpoint{0.996703in}{1.409741in}}%
\pgfpathlineto{\pgfqpoint{0.997647in}{1.410096in}}%
\pgfpathlineto{\pgfqpoint{1.000152in}{1.411161in}}%
\pgfpathlineto{\pgfqpoint{1.001255in}{1.411397in}}%
\pgfpathlineto{\pgfqpoint{1.003451in}{1.412502in}}%
\pgfpathlineto{\pgfqpoint{1.004554in}{1.413251in}}%
\pgfpathlineto{\pgfqpoint{1.007245in}{1.414355in}}%
\pgfpathlineto{\pgfqpoint{1.008283in}{1.415065in}}%
\pgfpathlineto{\pgfqpoint{1.010227in}{1.416170in}}%
\pgfpathlineto{\pgfqpoint{1.011217in}{1.416919in}}%
\pgfpathlineto{\pgfqpoint{1.013600in}{1.418024in}}%
\pgfpathlineto{\pgfqpoint{1.014638in}{1.418694in}}%
\pgfpathlineto{\pgfqpoint{1.017703in}{1.419798in}}%
\pgfpathlineto{\pgfqpoint{1.018713in}{1.420430in}}%
\pgfpathlineto{\pgfqpoint{1.021395in}{1.421495in}}%
\pgfpathlineto{\pgfqpoint{1.022367in}{1.422559in}}%
\pgfpathlineto{\pgfqpoint{1.024255in}{1.423664in}}%
\pgfpathlineto{\pgfqpoint{1.025180in}{1.423979in}}%
\pgfpathlineto{\pgfqpoint{1.027479in}{1.425084in}}%
\pgfpathlineto{\pgfqpoint{1.028554in}{1.425557in}}%
\pgfpathlineto{\pgfqpoint{1.030292in}{1.426661in}}%
\pgfpathlineto{\pgfqpoint{1.031274in}{1.427174in}}%
\pgfpathlineto{\pgfqpoint{1.033489in}{1.428279in}}%
\pgfpathlineto{\pgfqpoint{1.034573in}{1.428713in}}%
\pgfpathlineto{\pgfqpoint{1.036750in}{1.429817in}}%
\pgfpathlineto{\pgfqpoint{1.037704in}{1.430330in}}%
\pgfpathlineto{\pgfqpoint{1.040133in}{1.431434in}}%
\pgfpathlineto{\pgfqpoint{1.041218in}{1.431947in}}%
\pgfpathlineto{\pgfqpoint{1.043274in}{1.433051in}}%
\pgfpathlineto{\pgfqpoint{1.044246in}{1.433801in}}%
\pgfpathlineto{\pgfqpoint{1.047274in}{1.434905in}}%
\pgfpathlineto{\pgfqpoint{1.048040in}{1.435260in}}%
\pgfpathlineto{\pgfqpoint{1.051040in}{1.436364in}}%
\pgfpathlineto{\pgfqpoint{1.052115in}{1.436838in}}%
\pgfpathlineto{\pgfqpoint{1.054582in}{1.437942in}}%
\pgfpathlineto{\pgfqpoint{1.055648in}{1.438613in}}%
\pgfpathlineto{\pgfqpoint{1.058386in}{1.439717in}}%
\pgfpathlineto{\pgfqpoint{1.059302in}{1.440072in}}%
\pgfpathlineto{\pgfqpoint{1.061891in}{1.441176in}}%
\pgfpathlineto{\pgfqpoint{1.062975in}{1.441452in}}%
\pgfpathlineto{\pgfqpoint{1.065180in}{1.442557in}}%
\pgfpathlineto{\pgfqpoint{1.065825in}{1.442754in}}%
\pgfpathlineto{\pgfqpoint{1.069302in}{1.443858in}}%
\pgfpathlineto{\pgfqpoint{1.070358in}{1.444371in}}%
\pgfpathlineto{\pgfqpoint{1.073723in}{1.445436in}}%
\pgfpathlineto{\pgfqpoint{1.074760in}{1.445949in}}%
\pgfpathlineto{\pgfqpoint{1.078068in}{1.447053in}}%
\pgfpathlineto{\pgfqpoint{1.079087in}{1.447408in}}%
\pgfpathlineto{\pgfqpoint{1.082293in}{1.448513in}}%
\pgfpathlineto{\pgfqpoint{1.083377in}{1.448947in}}%
\pgfpathlineto{\pgfqpoint{1.085779in}{1.450051in}}%
\pgfpathlineto{\pgfqpoint{1.086835in}{1.450248in}}%
\pgfpathlineto{\pgfqpoint{1.089928in}{1.451313in}}%
\pgfpathlineto{\pgfqpoint{1.090798in}{1.451865in}}%
\pgfpathlineto{\pgfqpoint{1.093826in}{1.452970in}}%
\pgfpathlineto{\pgfqpoint{1.094872in}{1.453246in}}%
\pgfpathlineto{\pgfqpoint{1.097900in}{1.454350in}}%
\pgfpathlineto{\pgfqpoint{1.098938in}{1.454784in}}%
\pgfpathlineto{\pgfqpoint{1.101564in}{1.455888in}}%
\pgfpathlineto{\pgfqpoint{1.102630in}{1.456165in}}%
\pgfpathlineto{\pgfqpoint{1.105592in}{1.457269in}}%
\pgfpathlineto{\pgfqpoint{1.106686in}{1.457900in}}%
\pgfpathlineto{\pgfqpoint{1.109041in}{1.459004in}}%
\pgfpathlineto{\pgfqpoint{1.110144in}{1.459754in}}%
\pgfpathlineto{\pgfqpoint{1.112620in}{1.460858in}}%
\pgfpathlineto{\pgfqpoint{1.113714in}{1.461292in}}%
\pgfpathlineto{\pgfqpoint{1.116714in}{1.462396in}}%
\pgfpathlineto{\pgfqpoint{1.117536in}{1.462673in}}%
\pgfpathlineto{\pgfqpoint{1.122190in}{1.463777in}}%
\pgfpathlineto{\pgfqpoint{1.123153in}{1.464290in}}%
\pgfpathlineto{\pgfqpoint{1.127443in}{1.465394in}}%
\pgfpathlineto{\pgfqpoint{1.128508in}{1.465946in}}%
\pgfpathlineto{\pgfqpoint{1.132331in}{1.467051in}}%
\pgfpathlineto{\pgfqpoint{1.133303in}{1.467603in}}%
\pgfpathlineto{\pgfqpoint{1.135864in}{1.468707in}}%
\pgfpathlineto{\pgfqpoint{1.136807in}{1.468944in}}%
\pgfpathlineto{\pgfqpoint{1.139452in}{1.470048in}}%
\pgfpathlineto{\pgfqpoint{1.140443in}{1.470443in}}%
\pgfpathlineto{\pgfqpoint{1.143368in}{1.471547in}}%
\pgfpathlineto{\pgfqpoint{1.144116in}{1.471981in}}%
\pgfpathlineto{\pgfqpoint{1.148462in}{1.473085in}}%
\pgfpathlineto{\pgfqpoint{1.148957in}{1.473283in}}%
\pgfpathlineto{\pgfqpoint{1.153602in}{1.474387in}}%
\pgfpathlineto{\pgfqpoint{1.154462in}{1.474781in}}%
\pgfpathlineto{\pgfqpoint{1.158312in}{1.475886in}}%
\pgfpathlineto{\pgfqpoint{1.159406in}{1.476044in}}%
\pgfpathlineto{\pgfqpoint{1.163322in}{1.477109in}}%
\pgfpathlineto{\pgfqpoint{1.164322in}{1.477542in}}%
\pgfpathlineto{\pgfqpoint{1.167387in}{1.478647in}}%
\pgfpathlineto{\pgfqpoint{1.168490in}{1.479041in}}%
\pgfpathlineto{\pgfqpoint{1.171584in}{1.480146in}}%
\pgfpathlineto{\pgfqpoint{1.172640in}{1.480619in}}%
\pgfpathlineto{\pgfqpoint{1.176098in}{1.481723in}}%
\pgfpathlineto{\pgfqpoint{1.177116in}{1.482157in}}%
\pgfpathlineto{\pgfqpoint{1.180855in}{1.483262in}}%
\pgfpathlineto{\pgfqpoint{1.181864in}{1.483893in}}%
\pgfpathlineto{\pgfqpoint{1.185406in}{1.484997in}}%
\pgfpathlineto{\pgfqpoint{1.186462in}{1.485313in}}%
\pgfpathlineto{\pgfqpoint{1.191331in}{1.486417in}}%
\pgfpathlineto{\pgfqpoint{1.192201in}{1.486614in}}%
\pgfpathlineto{\pgfqpoint{1.195995in}{1.487719in}}%
\pgfpathlineto{\pgfqpoint{1.196911in}{1.488034in}}%
\pgfpathlineto{\pgfqpoint{1.200958in}{1.489138in}}%
\pgfpathlineto{\pgfqpoint{1.201986in}{1.489454in}}%
\pgfpathlineto{\pgfqpoint{1.205696in}{1.490558in}}%
\pgfpathlineto{\pgfqpoint{1.206734in}{1.490992in}}%
\pgfpathlineto{\pgfqpoint{1.212603in}{1.492097in}}%
\pgfpathlineto{\pgfqpoint{1.213706in}{1.492333in}}%
\pgfpathlineto{\pgfqpoint{1.218182in}{1.493438in}}%
\pgfpathlineto{\pgfqpoint{1.219051in}{1.493872in}}%
\pgfpathlineto{\pgfqpoint{1.223724in}{1.494976in}}%
\pgfpathlineto{\pgfqpoint{1.224771in}{1.495252in}}%
\pgfpathlineto{\pgfqpoint{1.230238in}{1.496356in}}%
\pgfpathlineto{\pgfqpoint{1.231341in}{1.496711in}}%
\pgfpathlineto{\pgfqpoint{1.236248in}{1.497816in}}%
\pgfpathlineto{\pgfqpoint{1.237257in}{1.498131in}}%
\pgfpathlineto{\pgfqpoint{1.243388in}{1.499236in}}%
\pgfpathlineto{\pgfqpoint{1.243930in}{1.499433in}}%
\pgfpathlineto{\pgfqpoint{1.249575in}{1.500537in}}%
\pgfpathlineto{\pgfqpoint{1.250304in}{1.500774in}}%
\pgfpathlineto{\pgfqpoint{1.255613in}{1.501878in}}%
\pgfpathlineto{\pgfqpoint{1.256491in}{1.502312in}}%
\pgfpathlineto{\pgfqpoint{1.261108in}{1.503417in}}%
\pgfpathlineto{\pgfqpoint{1.262080in}{1.503614in}}%
\pgfpathlineto{\pgfqpoint{1.267089in}{1.504718in}}%
\pgfpathlineto{\pgfqpoint{1.268052in}{1.504876in}}%
\pgfpathlineto{\pgfqpoint{1.274145in}{1.505980in}}%
\pgfpathlineto{\pgfqpoint{1.274912in}{1.506178in}}%
\pgfpathlineto{\pgfqpoint{1.279286in}{1.507282in}}%
\pgfpathlineto{\pgfqpoint{1.280267in}{1.507913in}}%
\pgfpathlineto{\pgfqpoint{1.285155in}{1.509018in}}%
\pgfpathlineto{\pgfqpoint{1.285856in}{1.509333in}}%
\pgfpathlineto{\pgfqpoint{1.293463in}{1.510437in}}%
\pgfpathlineto{\pgfqpoint{1.294351in}{1.510674in}}%
\pgfpathlineto{\pgfqpoint{1.298950in}{1.511779in}}%
\pgfpathlineto{\pgfqpoint{1.299744in}{1.512212in}}%
\pgfpathlineto{\pgfqpoint{1.307221in}{1.513317in}}%
\pgfpathlineto{\pgfqpoint{1.308108in}{1.513632in}}%
\pgfpathlineto{\pgfqpoint{1.314015in}{1.514737in}}%
\pgfpathlineto{\pgfqpoint{1.315109in}{1.515092in}}%
\pgfpathlineto{\pgfqpoint{1.321744in}{1.516196in}}%
\pgfpathlineto{\pgfqpoint{1.322763in}{1.516472in}}%
\pgfpathlineto{\pgfqpoint{1.328632in}{1.517577in}}%
\pgfpathlineto{\pgfqpoint{1.329669in}{1.518010in}}%
\pgfpathlineto{\pgfqpoint{1.336539in}{1.519115in}}%
\pgfpathlineto{\pgfqpoint{1.337623in}{1.519312in}}%
\pgfpathlineto{\pgfqpoint{1.343193in}{1.520416in}}%
\pgfpathlineto{\pgfqpoint{1.343885in}{1.520574in}}%
\pgfpathlineto{\pgfqpoint{1.349866in}{1.521679in}}%
\pgfpathlineto{\pgfqpoint{1.350913in}{1.521876in}}%
\pgfpathlineto{\pgfqpoint{1.358118in}{1.522980in}}%
\pgfpathlineto{\pgfqpoint{1.359203in}{1.523335in}}%
\pgfpathlineto{\pgfqpoint{1.364698in}{1.524440in}}%
\pgfpathlineto{\pgfqpoint{1.365782in}{1.524755in}}%
\pgfpathlineto{\pgfqpoint{1.370651in}{1.525859in}}%
\pgfpathlineto{\pgfqpoint{1.371530in}{1.526017in}}%
\pgfpathlineto{\pgfqpoint{1.378352in}{1.527122in}}%
\pgfpathlineto{\pgfqpoint{1.378969in}{1.527240in}}%
\pgfpathlineto{\pgfqpoint{1.385390in}{1.528344in}}%
\pgfpathlineto{\pgfqpoint{1.386212in}{1.528423in}}%
\pgfpathlineto{\pgfqpoint{1.392324in}{1.529488in}}%
\pgfpathlineto{\pgfqpoint{1.392558in}{1.529646in}}%
\pgfpathlineto{\pgfqpoint{1.400007in}{1.530750in}}%
\pgfpathlineto{\pgfqpoint{1.401044in}{1.530948in}}%
\pgfpathlineto{\pgfqpoint{1.409652in}{1.532052in}}%
\pgfpathlineto{\pgfqpoint{1.410362in}{1.532249in}}%
\pgfpathlineto{\pgfqpoint{1.417558in}{1.533354in}}%
\pgfpathlineto{\pgfqpoint{1.418353in}{1.533432in}}%
\pgfpathlineto{\pgfqpoint{1.424914in}{1.534537in}}%
\pgfpathlineto{\pgfqpoint{1.425970in}{1.534734in}}%
\pgfpathlineto{\pgfqpoint{1.434400in}{1.535838in}}%
\pgfpathlineto{\pgfqpoint{1.434400in}{1.535878in}}%
\pgfpathlineto{\pgfqpoint{1.441512in}{1.536982in}}%
\pgfpathlineto{\pgfqpoint{1.442596in}{1.537219in}}%
\pgfpathlineto{\pgfqpoint{1.451652in}{1.538323in}}%
\pgfpathlineto{\pgfqpoint{1.452549in}{1.538521in}}%
\pgfpathlineto{\pgfqpoint{1.462886in}{1.539586in}}%
\pgfpathlineto{\pgfqpoint{1.463793in}{1.539940in}}%
\pgfpathlineto{\pgfqpoint{1.472381in}{1.541045in}}%
\pgfpathlineto{\pgfqpoint{1.473381in}{1.541242in}}%
\pgfpathlineto{\pgfqpoint{1.481082in}{1.542346in}}%
\pgfpathlineto{\pgfqpoint{1.482110in}{1.542544in}}%
\pgfpathlineto{\pgfqpoint{1.492662in}{1.543648in}}%
\pgfpathlineto{\pgfqpoint{1.493709in}{1.543924in}}%
\pgfpathlineto{\pgfqpoint{1.502485in}{1.545029in}}%
\pgfpathlineto{\pgfqpoint{1.503354in}{1.545384in}}%
\pgfpathlineto{\pgfqpoint{1.511728in}{1.546488in}}%
\pgfpathlineto{\pgfqpoint{1.512587in}{1.546882in}}%
\pgfpathlineto{\pgfqpoint{1.522560in}{1.547987in}}%
\pgfpathlineto{\pgfqpoint{1.523410in}{1.548145in}}%
\pgfpathlineto{\pgfqpoint{1.534401in}{1.549209in}}%
\pgfpathlineto{\pgfqpoint{1.535391in}{1.549446in}}%
\pgfpathlineto{\pgfqpoint{1.543242in}{1.550551in}}%
\pgfpathlineto{\pgfqpoint{1.544046in}{1.550629in}}%
\pgfpathlineto{\pgfqpoint{1.553990in}{1.551734in}}%
\pgfpathlineto{\pgfqpoint{1.553990in}{1.551773in}}%
\pgfpathlineto{\pgfqpoint{1.564840in}{1.552878in}}%
\pgfpathlineto{\pgfqpoint{1.565784in}{1.553035in}}%
\pgfpathlineto{\pgfqpoint{1.578261in}{1.554140in}}%
\pgfpathlineto{\pgfqpoint{1.579037in}{1.554298in}}%
\pgfpathlineto{\pgfqpoint{1.593691in}{1.555402in}}%
\pgfpathlineto{\pgfqpoint{1.594373in}{1.555520in}}%
\pgfpathlineto{\pgfqpoint{1.608056in}{1.556625in}}%
\pgfpathlineto{\pgfqpoint{1.609028in}{1.556861in}}%
\pgfpathlineto{\pgfqpoint{1.617019in}{1.557966in}}%
\pgfpathlineto{\pgfqpoint{1.617776in}{1.558123in}}%
\pgfpathlineto{\pgfqpoint{1.628374in}{1.559228in}}%
\pgfpathlineto{\pgfqpoint{1.629421in}{1.559425in}}%
\pgfpathlineto{\pgfqpoint{1.638552in}{1.560529in}}%
\pgfpathlineto{\pgfqpoint{1.639252in}{1.560687in}}%
\pgfpathlineto{\pgfqpoint{1.649982in}{1.561792in}}%
\pgfpathlineto{\pgfqpoint{1.650626in}{1.561949in}}%
\pgfpathlineto{\pgfqpoint{1.663028in}{1.563054in}}%
\pgfpathlineto{\pgfqpoint{1.664103in}{1.563290in}}%
\pgfpathlineto{\pgfqpoint{1.675029in}{1.564395in}}%
\pgfpathlineto{\pgfqpoint{1.675029in}{1.564434in}}%
\pgfpathlineto{\pgfqpoint{1.689001in}{1.565539in}}%
\pgfpathlineto{\pgfqpoint{1.690038in}{1.565736in}}%
\pgfpathlineto{\pgfqpoint{1.704721in}{1.566840in}}%
\pgfpathlineto{\pgfqpoint{1.704721in}{1.566919in}}%
\pgfpathlineto{\pgfqpoint{1.717954in}{1.568024in}}%
\pgfpathlineto{\pgfqpoint{1.718235in}{1.568142in}}%
\pgfpathlineto{\pgfqpoint{1.732338in}{1.569246in}}%
\pgfpathlineto{\pgfqpoint{1.733095in}{1.569365in}}%
\pgfpathlineto{\pgfqpoint{1.744366in}{1.570469in}}%
\pgfpathlineto{\pgfqpoint{1.745114in}{1.570548in}}%
\pgfpathlineto{\pgfqpoint{1.758712in}{1.571652in}}%
\pgfpathlineto{\pgfqpoint{1.759618in}{1.571810in}}%
\pgfpathlineto{\pgfqpoint{1.773217in}{1.572914in}}%
\pgfpathlineto{\pgfqpoint{1.774002in}{1.572993in}}%
\pgfpathlineto{\pgfqpoint{1.787946in}{1.574098in}}%
\pgfpathlineto{\pgfqpoint{1.788974in}{1.574216in}}%
\pgfpathlineto{\pgfqpoint{1.799525in}{1.575320in}}%
\pgfpathlineto{\pgfqpoint{1.800226in}{1.575478in}}%
\pgfpathlineto{\pgfqpoint{1.813862in}{1.576583in}}%
\pgfpathlineto{\pgfqpoint{1.813862in}{1.576622in}}%
\pgfpathlineto{\pgfqpoint{1.827591in}{1.577726in}}%
\pgfpathlineto{\pgfqpoint{1.828619in}{1.577963in}}%
\pgfpathlineto{\pgfqpoint{1.841788in}{1.579067in}}%
\pgfpathlineto{\pgfqpoint{1.842890in}{1.579186in}}%
\pgfpathlineto{\pgfqpoint{1.855199in}{1.580290in}}%
\pgfpathlineto{\pgfqpoint{1.855741in}{1.580369in}}%
\pgfpathlineto{\pgfqpoint{1.869517in}{1.581473in}}%
\pgfpathlineto{\pgfqpoint{1.870246in}{1.581552in}}%
\pgfpathlineto{\pgfqpoint{1.888003in}{1.582657in}}%
\pgfpathlineto{\pgfqpoint{1.888769in}{1.582815in}}%
\pgfpathlineto{\pgfqpoint{1.904377in}{1.583919in}}%
\pgfpathlineto{\pgfqpoint{1.905433in}{1.584116in}}%
\pgfpathlineto{\pgfqpoint{1.920760in}{1.585221in}}%
\pgfpathlineto{\pgfqpoint{1.921714in}{1.585418in}}%
\pgfpathlineto{\pgfqpoint{1.932667in}{1.586522in}}%
\pgfpathlineto{\pgfqpoint{1.933321in}{1.586601in}}%
\pgfpathlineto{\pgfqpoint{1.946106in}{1.587705in}}%
\pgfpathlineto{\pgfqpoint{1.947116in}{1.587784in}}%
\pgfpathlineto{\pgfqpoint{1.961172in}{1.588889in}}%
\pgfpathlineto{\pgfqpoint{1.962266in}{1.589046in}}%
\pgfpathlineto{\pgfqpoint{1.974518in}{1.590151in}}%
\pgfpathlineto{\pgfqpoint{1.975499in}{1.590388in}}%
\pgfpathlineto{\pgfqpoint{1.986079in}{1.591492in}}%
\pgfpathlineto{\pgfqpoint{1.986845in}{1.591650in}}%
\pgfpathlineto{\pgfqpoint{1.997425in}{1.592754in}}%
\pgfpathlineto{\pgfqpoint{1.998528in}{1.592991in}}%
\pgfpathlineto{\pgfqpoint{2.006621in}{1.594095in}}%
\pgfpathlineto{\pgfqpoint{2.007294in}{1.594292in}}%
\pgfpathlineto{\pgfqpoint{2.018219in}{1.595397in}}%
\pgfpathlineto{\pgfqpoint{2.018948in}{1.595633in}}%
\pgfpathlineto{\pgfqpoint{2.025556in}{1.596738in}}%
\pgfpathlineto{\pgfqpoint{2.025948in}{1.596974in}}%
\pgfpathlineto{\pgfqpoint{2.026014in}{1.596974in}}%
\pgfpathlineto{\pgfqpoint{2.033126in}{1.601944in}}%
\pgfpathlineto{\pgfqpoint{2.033126in}{1.601944in}}%
\pgfusepath{stroke}%
\end{pgfscope}%
\begin{pgfscope}%
\pgfsetrectcap%
\pgfsetmiterjoin%
\pgfsetlinewidth{0.803000pt}%
\definecolor{currentstroke}{rgb}{0.000000,0.000000,0.000000}%
\pgfsetstrokecolor{currentstroke}%
\pgfsetdash{}{0pt}%
\pgfpathmoveto{\pgfqpoint{0.553581in}{0.499444in}}%
\pgfpathlineto{\pgfqpoint{0.553581in}{1.654444in}}%
\pgfusepath{stroke}%
\end{pgfscope}%
\begin{pgfscope}%
\pgfsetrectcap%
\pgfsetmiterjoin%
\pgfsetlinewidth{0.803000pt}%
\definecolor{currentstroke}{rgb}{0.000000,0.000000,0.000000}%
\pgfsetstrokecolor{currentstroke}%
\pgfsetdash{}{0pt}%
\pgfpathmoveto{\pgfqpoint{2.103581in}{0.499444in}}%
\pgfpathlineto{\pgfqpoint{2.103581in}{1.654444in}}%
\pgfusepath{stroke}%
\end{pgfscope}%
\begin{pgfscope}%
\pgfsetrectcap%
\pgfsetmiterjoin%
\pgfsetlinewidth{0.803000pt}%
\definecolor{currentstroke}{rgb}{0.000000,0.000000,0.000000}%
\pgfsetstrokecolor{currentstroke}%
\pgfsetdash{}{0pt}%
\pgfpathmoveto{\pgfqpoint{0.553581in}{0.499444in}}%
\pgfpathlineto{\pgfqpoint{2.103581in}{0.499444in}}%
\pgfusepath{stroke}%
\end{pgfscope}%
\begin{pgfscope}%
\pgfsetrectcap%
\pgfsetmiterjoin%
\pgfsetlinewidth{0.803000pt}%
\definecolor{currentstroke}{rgb}{0.000000,0.000000,0.000000}%
\pgfsetstrokecolor{currentstroke}%
\pgfsetdash{}{0pt}%
\pgfpathmoveto{\pgfqpoint{0.553581in}{1.654444in}}%
\pgfpathlineto{\pgfqpoint{2.103581in}{1.654444in}}%
\pgfusepath{stroke}%
\end{pgfscope}%
\begin{pgfscope}%
\pgfsetbuttcap%
\pgfsetmiterjoin%
\definecolor{currentfill}{rgb}{1.000000,1.000000,1.000000}%
\pgfsetfillcolor{currentfill}%
\pgfsetlinewidth{0.000000pt}%
\definecolor{currentstroke}{rgb}{0.000000,0.000000,0.000000}%
\pgfsetstrokecolor{currentstroke}%
\pgfsetstrokeopacity{0.000000}%
\pgfsetdash{}{0pt}%
\pgfpathmoveto{\pgfqpoint{1.286928in}{1.448926in}}%
\pgfpathlineto{\pgfqpoint{1.686650in}{1.448926in}}%
\pgfpathlineto{\pgfqpoint{1.686650in}{1.655593in}}%
\pgfpathlineto{\pgfqpoint{1.286928in}{1.655593in}}%
\pgfpathlineto{\pgfqpoint{1.286928in}{1.448926in}}%
\pgfpathclose%
\pgfusepath{fill}%
\end{pgfscope}%
\begin{pgfscope}%
\definecolor{textcolor}{rgb}{0.000000,0.000000,0.000000}%
\pgfsetstrokecolor{textcolor}%
\pgfsetfillcolor{textcolor}%
\pgftext[x=1.328595in,y=1.517537in,left,base]{\color{textcolor}\rmfamily\fontsize{10.000000}{12.000000}\selectfont 0.243}%
\end{pgfscope}%
\begin{pgfscope}%
\pgfsetbuttcap%
\pgfsetmiterjoin%
\definecolor{currentfill}{rgb}{1.000000,1.000000,1.000000}%
\pgfsetfillcolor{currentfill}%
\pgfsetlinewidth{0.000000pt}%
\definecolor{currentstroke}{rgb}{0.000000,0.000000,0.000000}%
\pgfsetstrokecolor{currentstroke}%
\pgfsetstrokeopacity{0.000000}%
\pgfsetdash{}{0pt}%
\pgfpathmoveto{\pgfqpoint{0.687846in}{1.008353in}}%
\pgfpathlineto{\pgfqpoint{1.018124in}{1.008353in}}%
\pgfpathlineto{\pgfqpoint{1.018124in}{1.215019in}}%
\pgfpathlineto{\pgfqpoint{0.687846in}{1.215019in}}%
\pgfpathlineto{\pgfqpoint{0.687846in}{1.008353in}}%
\pgfpathclose%
\pgfusepath{fill}%
\end{pgfscope}%
\begin{pgfscope}%
\definecolor{textcolor}{rgb}{0.000000,0.000000,0.000000}%
\pgfsetstrokecolor{textcolor}%
\pgfsetfillcolor{textcolor}%
\pgftext[x=0.729513in,y=1.076964in,left,base]{\color{textcolor}\rmfamily\fontsize{10.000000}{12.000000}\selectfont 0.52}%
\end{pgfscope}%
\begin{pgfscope}%
\pgfsetbuttcap%
\pgfsetmiterjoin%
\definecolor{currentfill}{rgb}{1.000000,1.000000,1.000000}%
\pgfsetfillcolor{currentfill}%
\pgfsetfillopacity{0.800000}%
\pgfsetlinewidth{1.003750pt}%
\definecolor{currentstroke}{rgb}{0.800000,0.800000,0.800000}%
\pgfsetstrokecolor{currentstroke}%
\pgfsetstrokeopacity{0.800000}%
\pgfsetdash{}{0pt}%
\pgfpathmoveto{\pgfqpoint{0.832747in}{0.568889in}}%
\pgfpathlineto{\pgfqpoint{2.006358in}{0.568889in}}%
\pgfpathquadraticcurveto{\pgfqpoint{2.034136in}{0.568889in}}{\pgfqpoint{2.034136in}{0.596666in}}%
\pgfpathlineto{\pgfqpoint{2.034136in}{0.776388in}}%
\pgfpathquadraticcurveto{\pgfqpoint{2.034136in}{0.804166in}}{\pgfqpoint{2.006358in}{0.804166in}}%
\pgfpathlineto{\pgfqpoint{0.832747in}{0.804166in}}%
\pgfpathquadraticcurveto{\pgfqpoint{0.804970in}{0.804166in}}{\pgfqpoint{0.804970in}{0.776388in}}%
\pgfpathlineto{\pgfqpoint{0.804970in}{0.596666in}}%
\pgfpathquadraticcurveto{\pgfqpoint{0.804970in}{0.568889in}}{\pgfqpoint{0.832747in}{0.568889in}}%
\pgfpathlineto{\pgfqpoint{0.832747in}{0.568889in}}%
\pgfpathclose%
\pgfusepath{stroke,fill}%
\end{pgfscope}%
\begin{pgfscope}%
\pgfsetrectcap%
\pgfsetroundjoin%
\pgfsetlinewidth{1.505625pt}%
\definecolor{currentstroke}{rgb}{0.000000,0.000000,0.000000}%
\pgfsetstrokecolor{currentstroke}%
\pgfsetdash{}{0pt}%
\pgfpathmoveto{\pgfqpoint{0.860525in}{0.700000in}}%
\pgfpathlineto{\pgfqpoint{0.999414in}{0.700000in}}%
\pgfpathlineto{\pgfqpoint{1.138303in}{0.700000in}}%
\pgfusepath{stroke}%
\end{pgfscope}%
\begin{pgfscope}%
\definecolor{textcolor}{rgb}{0.000000,0.000000,0.000000}%
\pgfsetstrokecolor{textcolor}%
\pgfsetfillcolor{textcolor}%
\pgftext[x=1.249414in,y=0.651388in,left,base]{\color{textcolor}\rmfamily\fontsize{10.000000}{12.000000}\selectfont AUC=0.840}%
\end{pgfscope}%
\end{pgfpicture}%
\makeatother%
\endgroup%

  &
\vspace{0pt} 
  
\begin{tabular}{cc|c|c|}
	&\multicolumn{1}{c}{}& \multicolumn{2}{c}{Prediction} \cr
	&\multicolumn{1}{c}{} & \multicolumn{1}{c}{N} & \multicolumn{1}{c}{P} \cr\cline{3-4}
	\multirow{2}{*}{\rotatebox[origin=c]{90}{Actual}}&N & 136,348 & 14,423 \vrule width 0pt height 10pt depth 2pt \cr\cline{3-4}
	&P & 12,017 & 14,604 \vrule width 0pt height 10pt depth 2pt \cr\cline{3-4}
\end{tabular}

\begin{center}
\begin{tabular}{ll}
0.503 & Precision \cr 
0.549 & Recall \cr 
0.525 & F1 \cr 
\end{tabular}
\end{center}
  
\end{tabular}

We have decided that we want $\Delta FP/\Delta TP < 2.0$; we had $FP/TP < 2.0$, and now have $FP/TP \approx 1$, sending fewer ambulances to crashes that need one.  Why is this better?  If we look at the change in FP and TP from changing the threshold $p = 0.5 \to 0.635$, 

$$\frac{\Delta FP}{\Delta TP} = \frac{32,842 - 14,423}{20,693-14,604} = \frac{18,419}{6,089} \approx 3$$

By changing the threshold, we have not sent some ambulances that were needed, but have not sent three times as many ambulances that were not needed.  Given our goal of $\Delta FP/\Delta TP < 2$, this tradeoff is appropriate.  

It sometimes happens that, using $p=0.5$, a model will recommend that we never send an ambulance, as in Example 3 below, which is a linear transformation of Example 1, $f(x) = 0.5x$.  (One cause of such a model is the class weight being too low.) Such a model can still be useful if it separates the negative and positive classes well, because we can move the threshold.   


\noindent\begin{tabular}{@{}p{0.3\textwidth}@{\hspace{24pt}} p{0.3\textwidth} @{\hspace{24pt}} p{0.3\textwidth}}
  \vspace{0pt} %% Creator: Matplotlib, PGF backend
%%
%% To include the figure in your LaTeX document, write
%%   \input{<filename>.pgf}
%%
%% Make sure the required packages are loaded in your preamble
%%   \usepackage{pgf}
%%
%% Also ensure that all the required font packages are loaded; for instance,
%% the lmodern package is sometimes necessary when using math font.
%%   \usepackage{lmodern}
%%
%% Figures using additional raster images can only be included by \input if
%% they are in the same directory as the main LaTeX file. For loading figures
%% from other directories you can use the `import` package
%%   \usepackage{import}
%%
%% and then include the figures with
%%   \import{<path to file>}{<filename>.pgf}
%%
%% Matplotlib used the following preamble
%%   
%%   \usepackage{fontspec}
%%   \makeatletter\@ifpackageloaded{underscore}{}{\usepackage[strings]{underscore}}\makeatother
%%
\begingroup%
\makeatletter%
\begin{pgfpicture}%
\pgfpathrectangle{\pgfpointorigin}{\pgfqpoint{2.253750in}{1.754444in}}%
\pgfusepath{use as bounding box, clip}%
\begin{pgfscope}%
\pgfsetbuttcap%
\pgfsetmiterjoin%
\definecolor{currentfill}{rgb}{1.000000,1.000000,1.000000}%
\pgfsetfillcolor{currentfill}%
\pgfsetlinewidth{0.000000pt}%
\definecolor{currentstroke}{rgb}{1.000000,1.000000,1.000000}%
\pgfsetstrokecolor{currentstroke}%
\pgfsetdash{}{0pt}%
\pgfpathmoveto{\pgfqpoint{0.000000in}{0.000000in}}%
\pgfpathlineto{\pgfqpoint{2.253750in}{0.000000in}}%
\pgfpathlineto{\pgfqpoint{2.253750in}{1.754444in}}%
\pgfpathlineto{\pgfqpoint{0.000000in}{1.754444in}}%
\pgfpathlineto{\pgfqpoint{0.000000in}{0.000000in}}%
\pgfpathclose%
\pgfusepath{fill}%
\end{pgfscope}%
\begin{pgfscope}%
\pgfsetbuttcap%
\pgfsetmiterjoin%
\definecolor{currentfill}{rgb}{1.000000,1.000000,1.000000}%
\pgfsetfillcolor{currentfill}%
\pgfsetlinewidth{0.000000pt}%
\definecolor{currentstroke}{rgb}{0.000000,0.000000,0.000000}%
\pgfsetstrokecolor{currentstroke}%
\pgfsetstrokeopacity{0.000000}%
\pgfsetdash{}{0pt}%
\pgfpathmoveto{\pgfqpoint{0.515000in}{0.499444in}}%
\pgfpathlineto{\pgfqpoint{2.065000in}{0.499444in}}%
\pgfpathlineto{\pgfqpoint{2.065000in}{1.654444in}}%
\pgfpathlineto{\pgfqpoint{0.515000in}{1.654444in}}%
\pgfpathlineto{\pgfqpoint{0.515000in}{0.499444in}}%
\pgfpathclose%
\pgfusepath{fill}%
\end{pgfscope}%
\begin{pgfscope}%
\pgfpathrectangle{\pgfqpoint{0.515000in}{0.499444in}}{\pgfqpoint{1.550000in}{1.155000in}}%
\pgfusepath{clip}%
\pgfsetbuttcap%
\pgfsetmiterjoin%
\pgfsetlinewidth{1.003750pt}%
\definecolor{currentstroke}{rgb}{0.000000,0.000000,0.000000}%
\pgfsetstrokecolor{currentstroke}%
\pgfsetdash{}{0pt}%
\pgfpathmoveto{\pgfqpoint{0.505000in}{0.499444in}}%
\pgfpathlineto{\pgfqpoint{0.552805in}{0.499444in}}%
\pgfpathlineto{\pgfqpoint{0.552805in}{0.961883in}}%
\pgfpathlineto{\pgfqpoint{0.505000in}{0.961883in}}%
\pgfusepath{stroke}%
\end{pgfscope}%
\begin{pgfscope}%
\pgfpathrectangle{\pgfqpoint{0.515000in}{0.499444in}}{\pgfqpoint{1.550000in}{1.155000in}}%
\pgfusepath{clip}%
\pgfsetbuttcap%
\pgfsetmiterjoin%
\pgfsetlinewidth{1.003750pt}%
\definecolor{currentstroke}{rgb}{0.000000,0.000000,0.000000}%
\pgfsetstrokecolor{currentstroke}%
\pgfsetdash{}{0pt}%
\pgfpathmoveto{\pgfqpoint{0.643537in}{0.499444in}}%
\pgfpathlineto{\pgfqpoint{0.704025in}{0.499444in}}%
\pgfpathlineto{\pgfqpoint{0.704025in}{1.599444in}}%
\pgfpathlineto{\pgfqpoint{0.643537in}{1.599444in}}%
\pgfpathlineto{\pgfqpoint{0.643537in}{0.499444in}}%
\pgfpathclose%
\pgfusepath{stroke}%
\end{pgfscope}%
\begin{pgfscope}%
\pgfpathrectangle{\pgfqpoint{0.515000in}{0.499444in}}{\pgfqpoint{1.550000in}{1.155000in}}%
\pgfusepath{clip}%
\pgfsetbuttcap%
\pgfsetmiterjoin%
\pgfsetlinewidth{1.003750pt}%
\definecolor{currentstroke}{rgb}{0.000000,0.000000,0.000000}%
\pgfsetstrokecolor{currentstroke}%
\pgfsetdash{}{0pt}%
\pgfpathmoveto{\pgfqpoint{0.794756in}{0.499444in}}%
\pgfpathlineto{\pgfqpoint{0.855244in}{0.499444in}}%
\pgfpathlineto{\pgfqpoint{0.855244in}{1.132630in}}%
\pgfpathlineto{\pgfqpoint{0.794756in}{1.132630in}}%
\pgfpathlineto{\pgfqpoint{0.794756in}{0.499444in}}%
\pgfpathclose%
\pgfusepath{stroke}%
\end{pgfscope}%
\begin{pgfscope}%
\pgfpathrectangle{\pgfqpoint{0.515000in}{0.499444in}}{\pgfqpoint{1.550000in}{1.155000in}}%
\pgfusepath{clip}%
\pgfsetbuttcap%
\pgfsetmiterjoin%
\pgfsetlinewidth{1.003750pt}%
\definecolor{currentstroke}{rgb}{0.000000,0.000000,0.000000}%
\pgfsetstrokecolor{currentstroke}%
\pgfsetdash{}{0pt}%
\pgfpathmoveto{\pgfqpoint{0.945976in}{0.499444in}}%
\pgfpathlineto{\pgfqpoint{1.006464in}{0.499444in}}%
\pgfpathlineto{\pgfqpoint{1.006464in}{0.739458in}}%
\pgfpathlineto{\pgfqpoint{0.945976in}{0.739458in}}%
\pgfpathlineto{\pgfqpoint{0.945976in}{0.499444in}}%
\pgfpathclose%
\pgfusepath{stroke}%
\end{pgfscope}%
\begin{pgfscope}%
\pgfpathrectangle{\pgfqpoint{0.515000in}{0.499444in}}{\pgfqpoint{1.550000in}{1.155000in}}%
\pgfusepath{clip}%
\pgfsetbuttcap%
\pgfsetmiterjoin%
\pgfsetlinewidth{1.003750pt}%
\definecolor{currentstroke}{rgb}{0.000000,0.000000,0.000000}%
\pgfsetstrokecolor{currentstroke}%
\pgfsetdash{}{0pt}%
\pgfpathmoveto{\pgfqpoint{1.097195in}{0.499444in}}%
\pgfpathlineto{\pgfqpoint{1.157683in}{0.499444in}}%
\pgfpathlineto{\pgfqpoint{1.157683in}{0.559668in}}%
\pgfpathlineto{\pgfqpoint{1.097195in}{0.559668in}}%
\pgfpathlineto{\pgfqpoint{1.097195in}{0.499444in}}%
\pgfpathclose%
\pgfusepath{stroke}%
\end{pgfscope}%
\begin{pgfscope}%
\pgfpathrectangle{\pgfqpoint{0.515000in}{0.499444in}}{\pgfqpoint{1.550000in}{1.155000in}}%
\pgfusepath{clip}%
\pgfsetbuttcap%
\pgfsetmiterjoin%
\pgfsetlinewidth{1.003750pt}%
\definecolor{currentstroke}{rgb}{0.000000,0.000000,0.000000}%
\pgfsetstrokecolor{currentstroke}%
\pgfsetdash{}{0pt}%
\pgfpathmoveto{\pgfqpoint{1.248415in}{0.499444in}}%
\pgfpathlineto{\pgfqpoint{1.308903in}{0.499444in}}%
\pgfpathlineto{\pgfqpoint{1.308903in}{0.499444in}}%
\pgfpathlineto{\pgfqpoint{1.248415in}{0.499444in}}%
\pgfpathlineto{\pgfqpoint{1.248415in}{0.499444in}}%
\pgfpathclose%
\pgfusepath{stroke}%
\end{pgfscope}%
\begin{pgfscope}%
\pgfpathrectangle{\pgfqpoint{0.515000in}{0.499444in}}{\pgfqpoint{1.550000in}{1.155000in}}%
\pgfusepath{clip}%
\pgfsetbuttcap%
\pgfsetmiterjoin%
\pgfsetlinewidth{1.003750pt}%
\definecolor{currentstroke}{rgb}{0.000000,0.000000,0.000000}%
\pgfsetstrokecolor{currentstroke}%
\pgfsetdash{}{0pt}%
\pgfpathmoveto{\pgfqpoint{1.399634in}{0.499444in}}%
\pgfpathlineto{\pgfqpoint{1.460122in}{0.499444in}}%
\pgfpathlineto{\pgfqpoint{1.460122in}{0.499444in}}%
\pgfpathlineto{\pgfqpoint{1.399634in}{0.499444in}}%
\pgfpathlineto{\pgfqpoint{1.399634in}{0.499444in}}%
\pgfpathclose%
\pgfusepath{stroke}%
\end{pgfscope}%
\begin{pgfscope}%
\pgfpathrectangle{\pgfqpoint{0.515000in}{0.499444in}}{\pgfqpoint{1.550000in}{1.155000in}}%
\pgfusepath{clip}%
\pgfsetbuttcap%
\pgfsetmiterjoin%
\pgfsetlinewidth{1.003750pt}%
\definecolor{currentstroke}{rgb}{0.000000,0.000000,0.000000}%
\pgfsetstrokecolor{currentstroke}%
\pgfsetdash{}{0pt}%
\pgfpathmoveto{\pgfqpoint{1.550854in}{0.499444in}}%
\pgfpathlineto{\pgfqpoint{1.611342in}{0.499444in}}%
\pgfpathlineto{\pgfqpoint{1.611342in}{0.499444in}}%
\pgfpathlineto{\pgfqpoint{1.550854in}{0.499444in}}%
\pgfpathlineto{\pgfqpoint{1.550854in}{0.499444in}}%
\pgfpathclose%
\pgfusepath{stroke}%
\end{pgfscope}%
\begin{pgfscope}%
\pgfpathrectangle{\pgfqpoint{0.515000in}{0.499444in}}{\pgfqpoint{1.550000in}{1.155000in}}%
\pgfusepath{clip}%
\pgfsetbuttcap%
\pgfsetmiterjoin%
\pgfsetlinewidth{1.003750pt}%
\definecolor{currentstroke}{rgb}{0.000000,0.000000,0.000000}%
\pgfsetstrokecolor{currentstroke}%
\pgfsetdash{}{0pt}%
\pgfpathmoveto{\pgfqpoint{1.702073in}{0.499444in}}%
\pgfpathlineto{\pgfqpoint{1.762561in}{0.499444in}}%
\pgfpathlineto{\pgfqpoint{1.762561in}{0.499444in}}%
\pgfpathlineto{\pgfqpoint{1.702073in}{0.499444in}}%
\pgfpathlineto{\pgfqpoint{1.702073in}{0.499444in}}%
\pgfpathclose%
\pgfusepath{stroke}%
\end{pgfscope}%
\begin{pgfscope}%
\pgfpathrectangle{\pgfqpoint{0.515000in}{0.499444in}}{\pgfqpoint{1.550000in}{1.155000in}}%
\pgfusepath{clip}%
\pgfsetbuttcap%
\pgfsetmiterjoin%
\pgfsetlinewidth{1.003750pt}%
\definecolor{currentstroke}{rgb}{0.000000,0.000000,0.000000}%
\pgfsetstrokecolor{currentstroke}%
\pgfsetdash{}{0pt}%
\pgfpathmoveto{\pgfqpoint{1.853293in}{0.499444in}}%
\pgfpathlineto{\pgfqpoint{1.913781in}{0.499444in}}%
\pgfpathlineto{\pgfqpoint{1.913781in}{0.499444in}}%
\pgfpathlineto{\pgfqpoint{1.853293in}{0.499444in}}%
\pgfpathlineto{\pgfqpoint{1.853293in}{0.499444in}}%
\pgfpathclose%
\pgfusepath{stroke}%
\end{pgfscope}%
\begin{pgfscope}%
\pgfpathrectangle{\pgfqpoint{0.515000in}{0.499444in}}{\pgfqpoint{1.550000in}{1.155000in}}%
\pgfusepath{clip}%
\pgfsetbuttcap%
\pgfsetmiterjoin%
\definecolor{currentfill}{rgb}{0.000000,0.000000,0.000000}%
\pgfsetfillcolor{currentfill}%
\pgfsetlinewidth{0.000000pt}%
\definecolor{currentstroke}{rgb}{0.000000,0.000000,0.000000}%
\pgfsetstrokecolor{currentstroke}%
\pgfsetstrokeopacity{0.000000}%
\pgfsetdash{}{0pt}%
\pgfpathmoveto{\pgfqpoint{0.552805in}{0.499444in}}%
\pgfpathlineto{\pgfqpoint{0.613293in}{0.499444in}}%
\pgfpathlineto{\pgfqpoint{0.613293in}{0.511101in}}%
\pgfpathlineto{\pgfqpoint{0.552805in}{0.511101in}}%
\pgfpathlineto{\pgfqpoint{0.552805in}{0.499444in}}%
\pgfpathclose%
\pgfusepath{fill}%
\end{pgfscope}%
\begin{pgfscope}%
\pgfpathrectangle{\pgfqpoint{0.515000in}{0.499444in}}{\pgfqpoint{1.550000in}{1.155000in}}%
\pgfusepath{clip}%
\pgfsetbuttcap%
\pgfsetmiterjoin%
\definecolor{currentfill}{rgb}{0.000000,0.000000,0.000000}%
\pgfsetfillcolor{currentfill}%
\pgfsetlinewidth{0.000000pt}%
\definecolor{currentstroke}{rgb}{0.000000,0.000000,0.000000}%
\pgfsetstrokecolor{currentstroke}%
\pgfsetstrokeopacity{0.000000}%
\pgfsetdash{}{0pt}%
\pgfpathmoveto{\pgfqpoint{0.704025in}{0.499444in}}%
\pgfpathlineto{\pgfqpoint{0.764512in}{0.499444in}}%
\pgfpathlineto{\pgfqpoint{0.764512in}{0.540846in}}%
\pgfpathlineto{\pgfqpoint{0.704025in}{0.540846in}}%
\pgfpathlineto{\pgfqpoint{0.704025in}{0.499444in}}%
\pgfpathclose%
\pgfusepath{fill}%
\end{pgfscope}%
\begin{pgfscope}%
\pgfpathrectangle{\pgfqpoint{0.515000in}{0.499444in}}{\pgfqpoint{1.550000in}{1.155000in}}%
\pgfusepath{clip}%
\pgfsetbuttcap%
\pgfsetmiterjoin%
\definecolor{currentfill}{rgb}{0.000000,0.000000,0.000000}%
\pgfsetfillcolor{currentfill}%
\pgfsetlinewidth{0.000000pt}%
\definecolor{currentstroke}{rgb}{0.000000,0.000000,0.000000}%
\pgfsetstrokecolor{currentstroke}%
\pgfsetstrokeopacity{0.000000}%
\pgfsetdash{}{0pt}%
\pgfpathmoveto{\pgfqpoint{0.855244in}{0.499444in}}%
\pgfpathlineto{\pgfqpoint{0.915732in}{0.499444in}}%
\pgfpathlineto{\pgfqpoint{0.915732in}{0.615047in}}%
\pgfpathlineto{\pgfqpoint{0.855244in}{0.615047in}}%
\pgfpathlineto{\pgfqpoint{0.855244in}{0.499444in}}%
\pgfpathclose%
\pgfusepath{fill}%
\end{pgfscope}%
\begin{pgfscope}%
\pgfpathrectangle{\pgfqpoint{0.515000in}{0.499444in}}{\pgfqpoint{1.550000in}{1.155000in}}%
\pgfusepath{clip}%
\pgfsetbuttcap%
\pgfsetmiterjoin%
\definecolor{currentfill}{rgb}{0.000000,0.000000,0.000000}%
\pgfsetfillcolor{currentfill}%
\pgfsetlinewidth{0.000000pt}%
\definecolor{currentstroke}{rgb}{0.000000,0.000000,0.000000}%
\pgfsetstrokecolor{currentstroke}%
\pgfsetstrokeopacity{0.000000}%
\pgfsetdash{}{0pt}%
\pgfpathmoveto{\pgfqpoint{1.006464in}{0.499444in}}%
\pgfpathlineto{\pgfqpoint{1.066951in}{0.499444in}}%
\pgfpathlineto{\pgfqpoint{1.066951in}{0.693211in}}%
\pgfpathlineto{\pgfqpoint{1.006464in}{0.693211in}}%
\pgfpathlineto{\pgfqpoint{1.006464in}{0.499444in}}%
\pgfpathclose%
\pgfusepath{fill}%
\end{pgfscope}%
\begin{pgfscope}%
\pgfpathrectangle{\pgfqpoint{0.515000in}{0.499444in}}{\pgfqpoint{1.550000in}{1.155000in}}%
\pgfusepath{clip}%
\pgfsetbuttcap%
\pgfsetmiterjoin%
\definecolor{currentfill}{rgb}{0.000000,0.000000,0.000000}%
\pgfsetfillcolor{currentfill}%
\pgfsetlinewidth{0.000000pt}%
\definecolor{currentstroke}{rgb}{0.000000,0.000000,0.000000}%
\pgfsetstrokecolor{currentstroke}%
\pgfsetstrokeopacity{0.000000}%
\pgfsetdash{}{0pt}%
\pgfpathmoveto{\pgfqpoint{1.157683in}{0.499444in}}%
\pgfpathlineto{\pgfqpoint{1.218171in}{0.499444in}}%
\pgfpathlineto{\pgfqpoint{1.218171in}{0.577462in}}%
\pgfpathlineto{\pgfqpoint{1.157683in}{0.577462in}}%
\pgfpathlineto{\pgfqpoint{1.157683in}{0.499444in}}%
\pgfpathclose%
\pgfusepath{fill}%
\end{pgfscope}%
\begin{pgfscope}%
\pgfpathrectangle{\pgfqpoint{0.515000in}{0.499444in}}{\pgfqpoint{1.550000in}{1.155000in}}%
\pgfusepath{clip}%
\pgfsetbuttcap%
\pgfsetmiterjoin%
\definecolor{currentfill}{rgb}{0.000000,0.000000,0.000000}%
\pgfsetfillcolor{currentfill}%
\pgfsetlinewidth{0.000000pt}%
\definecolor{currentstroke}{rgb}{0.000000,0.000000,0.000000}%
\pgfsetstrokecolor{currentstroke}%
\pgfsetstrokeopacity{0.000000}%
\pgfsetdash{}{0pt}%
\pgfpathmoveto{\pgfqpoint{1.308903in}{0.499444in}}%
\pgfpathlineto{\pgfqpoint{1.369391in}{0.499444in}}%
\pgfpathlineto{\pgfqpoint{1.369391in}{0.499444in}}%
\pgfpathlineto{\pgfqpoint{1.308903in}{0.499444in}}%
\pgfpathlineto{\pgfqpoint{1.308903in}{0.499444in}}%
\pgfpathclose%
\pgfusepath{fill}%
\end{pgfscope}%
\begin{pgfscope}%
\pgfpathrectangle{\pgfqpoint{0.515000in}{0.499444in}}{\pgfqpoint{1.550000in}{1.155000in}}%
\pgfusepath{clip}%
\pgfsetbuttcap%
\pgfsetmiterjoin%
\definecolor{currentfill}{rgb}{0.000000,0.000000,0.000000}%
\pgfsetfillcolor{currentfill}%
\pgfsetlinewidth{0.000000pt}%
\definecolor{currentstroke}{rgb}{0.000000,0.000000,0.000000}%
\pgfsetstrokecolor{currentstroke}%
\pgfsetstrokeopacity{0.000000}%
\pgfsetdash{}{0pt}%
\pgfpathmoveto{\pgfqpoint{1.460122in}{0.499444in}}%
\pgfpathlineto{\pgfqpoint{1.520610in}{0.499444in}}%
\pgfpathlineto{\pgfqpoint{1.520610in}{0.499444in}}%
\pgfpathlineto{\pgfqpoint{1.460122in}{0.499444in}}%
\pgfpathlineto{\pgfqpoint{1.460122in}{0.499444in}}%
\pgfpathclose%
\pgfusepath{fill}%
\end{pgfscope}%
\begin{pgfscope}%
\pgfpathrectangle{\pgfqpoint{0.515000in}{0.499444in}}{\pgfqpoint{1.550000in}{1.155000in}}%
\pgfusepath{clip}%
\pgfsetbuttcap%
\pgfsetmiterjoin%
\definecolor{currentfill}{rgb}{0.000000,0.000000,0.000000}%
\pgfsetfillcolor{currentfill}%
\pgfsetlinewidth{0.000000pt}%
\definecolor{currentstroke}{rgb}{0.000000,0.000000,0.000000}%
\pgfsetstrokecolor{currentstroke}%
\pgfsetstrokeopacity{0.000000}%
\pgfsetdash{}{0pt}%
\pgfpathmoveto{\pgfqpoint{1.611342in}{0.499444in}}%
\pgfpathlineto{\pgfqpoint{1.671830in}{0.499444in}}%
\pgfpathlineto{\pgfqpoint{1.671830in}{0.499444in}}%
\pgfpathlineto{\pgfqpoint{1.611342in}{0.499444in}}%
\pgfpathlineto{\pgfqpoint{1.611342in}{0.499444in}}%
\pgfpathclose%
\pgfusepath{fill}%
\end{pgfscope}%
\begin{pgfscope}%
\pgfpathrectangle{\pgfqpoint{0.515000in}{0.499444in}}{\pgfqpoint{1.550000in}{1.155000in}}%
\pgfusepath{clip}%
\pgfsetbuttcap%
\pgfsetmiterjoin%
\definecolor{currentfill}{rgb}{0.000000,0.000000,0.000000}%
\pgfsetfillcolor{currentfill}%
\pgfsetlinewidth{0.000000pt}%
\definecolor{currentstroke}{rgb}{0.000000,0.000000,0.000000}%
\pgfsetstrokecolor{currentstroke}%
\pgfsetstrokeopacity{0.000000}%
\pgfsetdash{}{0pt}%
\pgfpathmoveto{\pgfqpoint{1.762561in}{0.499444in}}%
\pgfpathlineto{\pgfqpoint{1.823049in}{0.499444in}}%
\pgfpathlineto{\pgfqpoint{1.823049in}{0.499444in}}%
\pgfpathlineto{\pgfqpoint{1.762561in}{0.499444in}}%
\pgfpathlineto{\pgfqpoint{1.762561in}{0.499444in}}%
\pgfpathclose%
\pgfusepath{fill}%
\end{pgfscope}%
\begin{pgfscope}%
\pgfpathrectangle{\pgfqpoint{0.515000in}{0.499444in}}{\pgfqpoint{1.550000in}{1.155000in}}%
\pgfusepath{clip}%
\pgfsetbuttcap%
\pgfsetmiterjoin%
\definecolor{currentfill}{rgb}{0.000000,0.000000,0.000000}%
\pgfsetfillcolor{currentfill}%
\pgfsetlinewidth{0.000000pt}%
\definecolor{currentstroke}{rgb}{0.000000,0.000000,0.000000}%
\pgfsetstrokecolor{currentstroke}%
\pgfsetstrokeopacity{0.000000}%
\pgfsetdash{}{0pt}%
\pgfpathmoveto{\pgfqpoint{1.913781in}{0.499444in}}%
\pgfpathlineto{\pgfqpoint{1.974269in}{0.499444in}}%
\pgfpathlineto{\pgfqpoint{1.974269in}{0.499444in}}%
\pgfpathlineto{\pgfqpoint{1.913781in}{0.499444in}}%
\pgfpathlineto{\pgfqpoint{1.913781in}{0.499444in}}%
\pgfpathclose%
\pgfusepath{fill}%
\end{pgfscope}%
\begin{pgfscope}%
\pgfsetbuttcap%
\pgfsetroundjoin%
\definecolor{currentfill}{rgb}{0.000000,0.000000,0.000000}%
\pgfsetfillcolor{currentfill}%
\pgfsetlinewidth{0.803000pt}%
\definecolor{currentstroke}{rgb}{0.000000,0.000000,0.000000}%
\pgfsetstrokecolor{currentstroke}%
\pgfsetdash{}{0pt}%
\pgfsys@defobject{currentmarker}{\pgfqpoint{0.000000in}{-0.048611in}}{\pgfqpoint{0.000000in}{0.000000in}}{%
\pgfpathmoveto{\pgfqpoint{0.000000in}{0.000000in}}%
\pgfpathlineto{\pgfqpoint{0.000000in}{-0.048611in}}%
\pgfusepath{stroke,fill}%
}%
\begin{pgfscope}%
\pgfsys@transformshift{0.552805in}{0.499444in}%
\pgfsys@useobject{currentmarker}{}%
\end{pgfscope}%
\end{pgfscope}%
\begin{pgfscope}%
\definecolor{textcolor}{rgb}{0.000000,0.000000,0.000000}%
\pgfsetstrokecolor{textcolor}%
\pgfsetfillcolor{textcolor}%
\pgftext[x=0.552805in,y=0.402222in,,top]{\color{textcolor}\rmfamily\fontsize{10.000000}{12.000000}\selectfont 0.0}%
\end{pgfscope}%
\begin{pgfscope}%
\pgfsetbuttcap%
\pgfsetroundjoin%
\definecolor{currentfill}{rgb}{0.000000,0.000000,0.000000}%
\pgfsetfillcolor{currentfill}%
\pgfsetlinewidth{0.803000pt}%
\definecolor{currentstroke}{rgb}{0.000000,0.000000,0.000000}%
\pgfsetstrokecolor{currentstroke}%
\pgfsetdash{}{0pt}%
\pgfsys@defobject{currentmarker}{\pgfqpoint{0.000000in}{-0.048611in}}{\pgfqpoint{0.000000in}{0.000000in}}{%
\pgfpathmoveto{\pgfqpoint{0.000000in}{0.000000in}}%
\pgfpathlineto{\pgfqpoint{0.000000in}{-0.048611in}}%
\pgfusepath{stroke,fill}%
}%
\begin{pgfscope}%
\pgfsys@transformshift{0.930854in}{0.499444in}%
\pgfsys@useobject{currentmarker}{}%
\end{pgfscope}%
\end{pgfscope}%
\begin{pgfscope}%
\definecolor{textcolor}{rgb}{0.000000,0.000000,0.000000}%
\pgfsetstrokecolor{textcolor}%
\pgfsetfillcolor{textcolor}%
\pgftext[x=0.930854in,y=0.402222in,,top]{\color{textcolor}\rmfamily\fontsize{10.000000}{12.000000}\selectfont 0.25}%
\end{pgfscope}%
\begin{pgfscope}%
\pgfsetbuttcap%
\pgfsetroundjoin%
\definecolor{currentfill}{rgb}{0.000000,0.000000,0.000000}%
\pgfsetfillcolor{currentfill}%
\pgfsetlinewidth{0.803000pt}%
\definecolor{currentstroke}{rgb}{0.000000,0.000000,0.000000}%
\pgfsetstrokecolor{currentstroke}%
\pgfsetdash{}{0pt}%
\pgfsys@defobject{currentmarker}{\pgfqpoint{0.000000in}{-0.048611in}}{\pgfqpoint{0.000000in}{0.000000in}}{%
\pgfpathmoveto{\pgfqpoint{0.000000in}{0.000000in}}%
\pgfpathlineto{\pgfqpoint{0.000000in}{-0.048611in}}%
\pgfusepath{stroke,fill}%
}%
\begin{pgfscope}%
\pgfsys@transformshift{1.308903in}{0.499444in}%
\pgfsys@useobject{currentmarker}{}%
\end{pgfscope}%
\end{pgfscope}%
\begin{pgfscope}%
\definecolor{textcolor}{rgb}{0.000000,0.000000,0.000000}%
\pgfsetstrokecolor{textcolor}%
\pgfsetfillcolor{textcolor}%
\pgftext[x=1.308903in,y=0.402222in,,top]{\color{textcolor}\rmfamily\fontsize{10.000000}{12.000000}\selectfont 0.5}%
\end{pgfscope}%
\begin{pgfscope}%
\pgfsetbuttcap%
\pgfsetroundjoin%
\definecolor{currentfill}{rgb}{0.000000,0.000000,0.000000}%
\pgfsetfillcolor{currentfill}%
\pgfsetlinewidth{0.803000pt}%
\definecolor{currentstroke}{rgb}{0.000000,0.000000,0.000000}%
\pgfsetstrokecolor{currentstroke}%
\pgfsetdash{}{0pt}%
\pgfsys@defobject{currentmarker}{\pgfqpoint{0.000000in}{-0.048611in}}{\pgfqpoint{0.000000in}{0.000000in}}{%
\pgfpathmoveto{\pgfqpoint{0.000000in}{0.000000in}}%
\pgfpathlineto{\pgfqpoint{0.000000in}{-0.048611in}}%
\pgfusepath{stroke,fill}%
}%
\begin{pgfscope}%
\pgfsys@transformshift{1.686951in}{0.499444in}%
\pgfsys@useobject{currentmarker}{}%
\end{pgfscope}%
\end{pgfscope}%
\begin{pgfscope}%
\definecolor{textcolor}{rgb}{0.000000,0.000000,0.000000}%
\pgfsetstrokecolor{textcolor}%
\pgfsetfillcolor{textcolor}%
\pgftext[x=1.686951in,y=0.402222in,,top]{\color{textcolor}\rmfamily\fontsize{10.000000}{12.000000}\selectfont 0.75}%
\end{pgfscope}%
\begin{pgfscope}%
\pgfsetbuttcap%
\pgfsetroundjoin%
\definecolor{currentfill}{rgb}{0.000000,0.000000,0.000000}%
\pgfsetfillcolor{currentfill}%
\pgfsetlinewidth{0.803000pt}%
\definecolor{currentstroke}{rgb}{0.000000,0.000000,0.000000}%
\pgfsetstrokecolor{currentstroke}%
\pgfsetdash{}{0pt}%
\pgfsys@defobject{currentmarker}{\pgfqpoint{0.000000in}{-0.048611in}}{\pgfqpoint{0.000000in}{0.000000in}}{%
\pgfpathmoveto{\pgfqpoint{0.000000in}{0.000000in}}%
\pgfpathlineto{\pgfqpoint{0.000000in}{-0.048611in}}%
\pgfusepath{stroke,fill}%
}%
\begin{pgfscope}%
\pgfsys@transformshift{2.065000in}{0.499444in}%
\pgfsys@useobject{currentmarker}{}%
\end{pgfscope}%
\end{pgfscope}%
\begin{pgfscope}%
\definecolor{textcolor}{rgb}{0.000000,0.000000,0.000000}%
\pgfsetstrokecolor{textcolor}%
\pgfsetfillcolor{textcolor}%
\pgftext[x=2.065000in,y=0.402222in,,top]{\color{textcolor}\rmfamily\fontsize{10.000000}{12.000000}\selectfont 1.0}%
\end{pgfscope}%
\begin{pgfscope}%
\definecolor{textcolor}{rgb}{0.000000,0.000000,0.000000}%
\pgfsetstrokecolor{textcolor}%
\pgfsetfillcolor{textcolor}%
\pgftext[x=1.290000in,y=0.223333in,,top]{\color{textcolor}\rmfamily\fontsize{10.000000}{12.000000}\selectfont \(\displaystyle p\)}%
\end{pgfscope}%
\begin{pgfscope}%
\pgfsetbuttcap%
\pgfsetroundjoin%
\definecolor{currentfill}{rgb}{0.000000,0.000000,0.000000}%
\pgfsetfillcolor{currentfill}%
\pgfsetlinewidth{0.803000pt}%
\definecolor{currentstroke}{rgb}{0.000000,0.000000,0.000000}%
\pgfsetstrokecolor{currentstroke}%
\pgfsetdash{}{0pt}%
\pgfsys@defobject{currentmarker}{\pgfqpoint{-0.048611in}{0.000000in}}{\pgfqpoint{-0.000000in}{0.000000in}}{%
\pgfpathmoveto{\pgfqpoint{-0.000000in}{0.000000in}}%
\pgfpathlineto{\pgfqpoint{-0.048611in}{0.000000in}}%
\pgfusepath{stroke,fill}%
}%
\begin{pgfscope}%
\pgfsys@transformshift{0.515000in}{0.499444in}%
\pgfsys@useobject{currentmarker}{}%
\end{pgfscope}%
\end{pgfscope}%
\begin{pgfscope}%
\definecolor{textcolor}{rgb}{0.000000,0.000000,0.000000}%
\pgfsetstrokecolor{textcolor}%
\pgfsetfillcolor{textcolor}%
\pgftext[x=0.348333in, y=0.451250in, left, base]{\color{textcolor}\rmfamily\fontsize{10.000000}{12.000000}\selectfont \(\displaystyle {0}\)}%
\end{pgfscope}%
\begin{pgfscope}%
\pgfsetbuttcap%
\pgfsetroundjoin%
\definecolor{currentfill}{rgb}{0.000000,0.000000,0.000000}%
\pgfsetfillcolor{currentfill}%
\pgfsetlinewidth{0.803000pt}%
\definecolor{currentstroke}{rgb}{0.000000,0.000000,0.000000}%
\pgfsetstrokecolor{currentstroke}%
\pgfsetdash{}{0pt}%
\pgfsys@defobject{currentmarker}{\pgfqpoint{-0.048611in}{0.000000in}}{\pgfqpoint{-0.000000in}{0.000000in}}{%
\pgfpathmoveto{\pgfqpoint{-0.000000in}{0.000000in}}%
\pgfpathlineto{\pgfqpoint{-0.048611in}{0.000000in}}%
\pgfusepath{stroke,fill}%
}%
\begin{pgfscope}%
\pgfsys@transformshift{0.515000in}{0.793075in}%
\pgfsys@useobject{currentmarker}{}%
\end{pgfscope}%
\end{pgfscope}%
\begin{pgfscope}%
\definecolor{textcolor}{rgb}{0.000000,0.000000,0.000000}%
\pgfsetstrokecolor{textcolor}%
\pgfsetfillcolor{textcolor}%
\pgftext[x=0.278889in, y=0.744881in, left, base]{\color{textcolor}\rmfamily\fontsize{10.000000}{12.000000}\selectfont \(\displaystyle {10}\)}%
\end{pgfscope}%
\begin{pgfscope}%
\pgfsetbuttcap%
\pgfsetroundjoin%
\definecolor{currentfill}{rgb}{0.000000,0.000000,0.000000}%
\pgfsetfillcolor{currentfill}%
\pgfsetlinewidth{0.803000pt}%
\definecolor{currentstroke}{rgb}{0.000000,0.000000,0.000000}%
\pgfsetstrokecolor{currentstroke}%
\pgfsetdash{}{0pt}%
\pgfsys@defobject{currentmarker}{\pgfqpoint{-0.048611in}{0.000000in}}{\pgfqpoint{-0.000000in}{0.000000in}}{%
\pgfpathmoveto{\pgfqpoint{-0.000000in}{0.000000in}}%
\pgfpathlineto{\pgfqpoint{-0.048611in}{0.000000in}}%
\pgfusepath{stroke,fill}%
}%
\begin{pgfscope}%
\pgfsys@transformshift{0.515000in}{1.086706in}%
\pgfsys@useobject{currentmarker}{}%
\end{pgfscope}%
\end{pgfscope}%
\begin{pgfscope}%
\definecolor{textcolor}{rgb}{0.000000,0.000000,0.000000}%
\pgfsetstrokecolor{textcolor}%
\pgfsetfillcolor{textcolor}%
\pgftext[x=0.278889in, y=1.038511in, left, base]{\color{textcolor}\rmfamily\fontsize{10.000000}{12.000000}\selectfont \(\displaystyle {20}\)}%
\end{pgfscope}%
\begin{pgfscope}%
\pgfsetbuttcap%
\pgfsetroundjoin%
\definecolor{currentfill}{rgb}{0.000000,0.000000,0.000000}%
\pgfsetfillcolor{currentfill}%
\pgfsetlinewidth{0.803000pt}%
\definecolor{currentstroke}{rgb}{0.000000,0.000000,0.000000}%
\pgfsetstrokecolor{currentstroke}%
\pgfsetdash{}{0pt}%
\pgfsys@defobject{currentmarker}{\pgfqpoint{-0.048611in}{0.000000in}}{\pgfqpoint{-0.000000in}{0.000000in}}{%
\pgfpathmoveto{\pgfqpoint{-0.000000in}{0.000000in}}%
\pgfpathlineto{\pgfqpoint{-0.048611in}{0.000000in}}%
\pgfusepath{stroke,fill}%
}%
\begin{pgfscope}%
\pgfsys@transformshift{0.515000in}{1.380337in}%
\pgfsys@useobject{currentmarker}{}%
\end{pgfscope}%
\end{pgfscope}%
\begin{pgfscope}%
\definecolor{textcolor}{rgb}{0.000000,0.000000,0.000000}%
\pgfsetstrokecolor{textcolor}%
\pgfsetfillcolor{textcolor}%
\pgftext[x=0.278889in, y=1.332142in, left, base]{\color{textcolor}\rmfamily\fontsize{10.000000}{12.000000}\selectfont \(\displaystyle {30}\)}%
\end{pgfscope}%
\begin{pgfscope}%
\definecolor{textcolor}{rgb}{0.000000,0.000000,0.000000}%
\pgfsetstrokecolor{textcolor}%
\pgfsetfillcolor{textcolor}%
\pgftext[x=0.223333in,y=1.076944in,,bottom,rotate=90.000000]{\color{textcolor}\rmfamily\fontsize{10.000000}{12.000000}\selectfont Percent of Data Set}%
\end{pgfscope}%
\begin{pgfscope}%
\pgfsetrectcap%
\pgfsetmiterjoin%
\pgfsetlinewidth{0.803000pt}%
\definecolor{currentstroke}{rgb}{0.000000,0.000000,0.000000}%
\pgfsetstrokecolor{currentstroke}%
\pgfsetdash{}{0pt}%
\pgfpathmoveto{\pgfqpoint{0.515000in}{0.499444in}}%
\pgfpathlineto{\pgfqpoint{0.515000in}{1.654444in}}%
\pgfusepath{stroke}%
\end{pgfscope}%
\begin{pgfscope}%
\pgfsetrectcap%
\pgfsetmiterjoin%
\pgfsetlinewidth{0.803000pt}%
\definecolor{currentstroke}{rgb}{0.000000,0.000000,0.000000}%
\pgfsetstrokecolor{currentstroke}%
\pgfsetdash{}{0pt}%
\pgfpathmoveto{\pgfqpoint{2.065000in}{0.499444in}}%
\pgfpathlineto{\pgfqpoint{2.065000in}{1.654444in}}%
\pgfusepath{stroke}%
\end{pgfscope}%
\begin{pgfscope}%
\pgfsetrectcap%
\pgfsetmiterjoin%
\pgfsetlinewidth{0.803000pt}%
\definecolor{currentstroke}{rgb}{0.000000,0.000000,0.000000}%
\pgfsetstrokecolor{currentstroke}%
\pgfsetdash{}{0pt}%
\pgfpathmoveto{\pgfqpoint{0.515000in}{0.499444in}}%
\pgfpathlineto{\pgfqpoint{2.065000in}{0.499444in}}%
\pgfusepath{stroke}%
\end{pgfscope}%
\begin{pgfscope}%
\pgfsetrectcap%
\pgfsetmiterjoin%
\pgfsetlinewidth{0.803000pt}%
\definecolor{currentstroke}{rgb}{0.000000,0.000000,0.000000}%
\pgfsetstrokecolor{currentstroke}%
\pgfsetdash{}{0pt}%
\pgfpathmoveto{\pgfqpoint{0.515000in}{1.654444in}}%
\pgfpathlineto{\pgfqpoint{2.065000in}{1.654444in}}%
\pgfusepath{stroke}%
\end{pgfscope}%
\begin{pgfscope}%
\pgfsetbuttcap%
\pgfsetmiterjoin%
\definecolor{currentfill}{rgb}{1.000000,1.000000,1.000000}%
\pgfsetfillcolor{currentfill}%
\pgfsetfillopacity{0.800000}%
\pgfsetlinewidth{1.003750pt}%
\definecolor{currentstroke}{rgb}{0.800000,0.800000,0.800000}%
\pgfsetstrokecolor{currentstroke}%
\pgfsetstrokeopacity{0.800000}%
\pgfsetdash{}{0pt}%
\pgfpathmoveto{\pgfqpoint{1.288056in}{1.154445in}}%
\pgfpathlineto{\pgfqpoint{1.967778in}{1.154445in}}%
\pgfpathquadraticcurveto{\pgfqpoint{1.995556in}{1.154445in}}{\pgfqpoint{1.995556in}{1.182222in}}%
\pgfpathlineto{\pgfqpoint{1.995556in}{1.557222in}}%
\pgfpathquadraticcurveto{\pgfqpoint{1.995556in}{1.585000in}}{\pgfqpoint{1.967778in}{1.585000in}}%
\pgfpathlineto{\pgfqpoint{1.288056in}{1.585000in}}%
\pgfpathquadraticcurveto{\pgfqpoint{1.260278in}{1.585000in}}{\pgfqpoint{1.260278in}{1.557222in}}%
\pgfpathlineto{\pgfqpoint{1.260278in}{1.182222in}}%
\pgfpathquadraticcurveto{\pgfqpoint{1.260278in}{1.154445in}}{\pgfqpoint{1.288056in}{1.154445in}}%
\pgfpathlineto{\pgfqpoint{1.288056in}{1.154445in}}%
\pgfpathclose%
\pgfusepath{stroke,fill}%
\end{pgfscope}%
\begin{pgfscope}%
\pgfsetbuttcap%
\pgfsetmiterjoin%
\pgfsetlinewidth{1.003750pt}%
\definecolor{currentstroke}{rgb}{0.000000,0.000000,0.000000}%
\pgfsetstrokecolor{currentstroke}%
\pgfsetdash{}{0pt}%
\pgfpathmoveto{\pgfqpoint{1.315834in}{1.432222in}}%
\pgfpathlineto{\pgfqpoint{1.593611in}{1.432222in}}%
\pgfpathlineto{\pgfqpoint{1.593611in}{1.529444in}}%
\pgfpathlineto{\pgfqpoint{1.315834in}{1.529444in}}%
\pgfpathlineto{\pgfqpoint{1.315834in}{1.432222in}}%
\pgfpathclose%
\pgfusepath{stroke}%
\end{pgfscope}%
\begin{pgfscope}%
\definecolor{textcolor}{rgb}{0.000000,0.000000,0.000000}%
\pgfsetstrokecolor{textcolor}%
\pgfsetfillcolor{textcolor}%
\pgftext[x=1.704722in,y=1.432222in,left,base]{\color{textcolor}\rmfamily\fontsize{10.000000}{12.000000}\selectfont Neg}%
\end{pgfscope}%
\begin{pgfscope}%
\pgfsetbuttcap%
\pgfsetmiterjoin%
\definecolor{currentfill}{rgb}{0.000000,0.000000,0.000000}%
\pgfsetfillcolor{currentfill}%
\pgfsetlinewidth{0.000000pt}%
\definecolor{currentstroke}{rgb}{0.000000,0.000000,0.000000}%
\pgfsetstrokecolor{currentstroke}%
\pgfsetstrokeopacity{0.000000}%
\pgfsetdash{}{0pt}%
\pgfpathmoveto{\pgfqpoint{1.315834in}{1.236944in}}%
\pgfpathlineto{\pgfqpoint{1.593611in}{1.236944in}}%
\pgfpathlineto{\pgfqpoint{1.593611in}{1.334167in}}%
\pgfpathlineto{\pgfqpoint{1.315834in}{1.334167in}}%
\pgfpathlineto{\pgfqpoint{1.315834in}{1.236944in}}%
\pgfpathclose%
\pgfusepath{fill}%
\end{pgfscope}%
\begin{pgfscope}%
\definecolor{textcolor}{rgb}{0.000000,0.000000,0.000000}%
\pgfsetstrokecolor{textcolor}%
\pgfsetfillcolor{textcolor}%
\pgftext[x=1.704722in,y=1.236944in,left,base]{\color{textcolor}\rmfamily\fontsize{10.000000}{12.000000}\selectfont Pos}%
\end{pgfscope}%
\end{pgfpicture}%
\makeatother%
\endgroup%

%  &
%  \vspace{0pt} %% Creator: Matplotlib, PGF backend
%%
%% To include the figure in your LaTeX document, write
%%   \input{<filename>.pgf}
%%
%% Make sure the required packages are loaded in your preamble
%%   \usepackage{pgf}
%%
%% Also ensure that all the required font packages are loaded; for instance,
%% the lmodern package is sometimes necessary when using math font.
%%   \usepackage{lmodern}
%%
%% Figures using additional raster images can only be included by \input if
%% they are in the same directory as the main LaTeX file. For loading figures
%% from other directories you can use the `import` package
%%   \usepackage{import}
%%
%% and then include the figures with
%%   \import{<path to file>}{<filename>.pgf}
%%
%% Matplotlib used the following preamble
%%   
%%   \usepackage{fontspec}
%%   \makeatletter\@ifpackageloaded{underscore}{}{\usepackage[strings]{underscore}}\makeatother
%%
\begingroup%
\makeatletter%
\begin{pgfpicture}%
\pgfpathrectangle{\pgfpointorigin}{\pgfqpoint{2.221861in}{1.754444in}}%
\pgfusepath{use as bounding box, clip}%
\begin{pgfscope}%
\pgfsetbuttcap%
\pgfsetmiterjoin%
\definecolor{currentfill}{rgb}{1.000000,1.000000,1.000000}%
\pgfsetfillcolor{currentfill}%
\pgfsetlinewidth{0.000000pt}%
\definecolor{currentstroke}{rgb}{1.000000,1.000000,1.000000}%
\pgfsetstrokecolor{currentstroke}%
\pgfsetdash{}{0pt}%
\pgfpathmoveto{\pgfqpoint{0.000000in}{0.000000in}}%
\pgfpathlineto{\pgfqpoint{2.221861in}{0.000000in}}%
\pgfpathlineto{\pgfqpoint{2.221861in}{1.754444in}}%
\pgfpathlineto{\pgfqpoint{0.000000in}{1.754444in}}%
\pgfpathlineto{\pgfqpoint{0.000000in}{0.000000in}}%
\pgfpathclose%
\pgfusepath{fill}%
\end{pgfscope}%
\begin{pgfscope}%
\pgfsetbuttcap%
\pgfsetmiterjoin%
\definecolor{currentfill}{rgb}{1.000000,1.000000,1.000000}%
\pgfsetfillcolor{currentfill}%
\pgfsetlinewidth{0.000000pt}%
\definecolor{currentstroke}{rgb}{0.000000,0.000000,0.000000}%
\pgfsetstrokecolor{currentstroke}%
\pgfsetstrokeopacity{0.000000}%
\pgfsetdash{}{0pt}%
\pgfpathmoveto{\pgfqpoint{0.553581in}{0.499444in}}%
\pgfpathlineto{\pgfqpoint{2.103581in}{0.499444in}}%
\pgfpathlineto{\pgfqpoint{2.103581in}{1.654444in}}%
\pgfpathlineto{\pgfqpoint{0.553581in}{1.654444in}}%
\pgfpathlineto{\pgfqpoint{0.553581in}{0.499444in}}%
\pgfpathclose%
\pgfusepath{fill}%
\end{pgfscope}%
\begin{pgfscope}%
\pgfsetbuttcap%
\pgfsetroundjoin%
\definecolor{currentfill}{rgb}{0.000000,0.000000,0.000000}%
\pgfsetfillcolor{currentfill}%
\pgfsetlinewidth{0.803000pt}%
\definecolor{currentstroke}{rgb}{0.000000,0.000000,0.000000}%
\pgfsetstrokecolor{currentstroke}%
\pgfsetdash{}{0pt}%
\pgfsys@defobject{currentmarker}{\pgfqpoint{0.000000in}{-0.048611in}}{\pgfqpoint{0.000000in}{0.000000in}}{%
\pgfpathmoveto{\pgfqpoint{0.000000in}{0.000000in}}%
\pgfpathlineto{\pgfqpoint{0.000000in}{-0.048611in}}%
\pgfusepath{stroke,fill}%
}%
\begin{pgfscope}%
\pgfsys@transformshift{0.624035in}{0.499444in}%
\pgfsys@useobject{currentmarker}{}%
\end{pgfscope}%
\end{pgfscope}%
\begin{pgfscope}%
\definecolor{textcolor}{rgb}{0.000000,0.000000,0.000000}%
\pgfsetstrokecolor{textcolor}%
\pgfsetfillcolor{textcolor}%
\pgftext[x=0.624035in,y=0.402222in,,top]{\color{textcolor}\rmfamily\fontsize{10.000000}{12.000000}\selectfont \(\displaystyle {0.0}\)}%
\end{pgfscope}%
\begin{pgfscope}%
\pgfsetbuttcap%
\pgfsetroundjoin%
\definecolor{currentfill}{rgb}{0.000000,0.000000,0.000000}%
\pgfsetfillcolor{currentfill}%
\pgfsetlinewidth{0.803000pt}%
\definecolor{currentstroke}{rgb}{0.000000,0.000000,0.000000}%
\pgfsetstrokecolor{currentstroke}%
\pgfsetdash{}{0pt}%
\pgfsys@defobject{currentmarker}{\pgfqpoint{0.000000in}{-0.048611in}}{\pgfqpoint{0.000000in}{0.000000in}}{%
\pgfpathmoveto{\pgfqpoint{0.000000in}{0.000000in}}%
\pgfpathlineto{\pgfqpoint{0.000000in}{-0.048611in}}%
\pgfusepath{stroke,fill}%
}%
\begin{pgfscope}%
\pgfsys@transformshift{1.328581in}{0.499444in}%
\pgfsys@useobject{currentmarker}{}%
\end{pgfscope}%
\end{pgfscope}%
\begin{pgfscope}%
\definecolor{textcolor}{rgb}{0.000000,0.000000,0.000000}%
\pgfsetstrokecolor{textcolor}%
\pgfsetfillcolor{textcolor}%
\pgftext[x=1.328581in,y=0.402222in,,top]{\color{textcolor}\rmfamily\fontsize{10.000000}{12.000000}\selectfont \(\displaystyle {0.5}\)}%
\end{pgfscope}%
\begin{pgfscope}%
\pgfsetbuttcap%
\pgfsetroundjoin%
\definecolor{currentfill}{rgb}{0.000000,0.000000,0.000000}%
\pgfsetfillcolor{currentfill}%
\pgfsetlinewidth{0.803000pt}%
\definecolor{currentstroke}{rgb}{0.000000,0.000000,0.000000}%
\pgfsetstrokecolor{currentstroke}%
\pgfsetdash{}{0pt}%
\pgfsys@defobject{currentmarker}{\pgfqpoint{0.000000in}{-0.048611in}}{\pgfqpoint{0.000000in}{0.000000in}}{%
\pgfpathmoveto{\pgfqpoint{0.000000in}{0.000000in}}%
\pgfpathlineto{\pgfqpoint{0.000000in}{-0.048611in}}%
\pgfusepath{stroke,fill}%
}%
\begin{pgfscope}%
\pgfsys@transformshift{2.033126in}{0.499444in}%
\pgfsys@useobject{currentmarker}{}%
\end{pgfscope}%
\end{pgfscope}%
\begin{pgfscope}%
\definecolor{textcolor}{rgb}{0.000000,0.000000,0.000000}%
\pgfsetstrokecolor{textcolor}%
\pgfsetfillcolor{textcolor}%
\pgftext[x=2.033126in,y=0.402222in,,top]{\color{textcolor}\rmfamily\fontsize{10.000000}{12.000000}\selectfont \(\displaystyle {1.0}\)}%
\end{pgfscope}%
\begin{pgfscope}%
\definecolor{textcolor}{rgb}{0.000000,0.000000,0.000000}%
\pgfsetstrokecolor{textcolor}%
\pgfsetfillcolor{textcolor}%
\pgftext[x=1.328581in,y=0.223333in,,top]{\color{textcolor}\rmfamily\fontsize{10.000000}{12.000000}\selectfont False positive rate}%
\end{pgfscope}%
\begin{pgfscope}%
\pgfsetbuttcap%
\pgfsetroundjoin%
\definecolor{currentfill}{rgb}{0.000000,0.000000,0.000000}%
\pgfsetfillcolor{currentfill}%
\pgfsetlinewidth{0.803000pt}%
\definecolor{currentstroke}{rgb}{0.000000,0.000000,0.000000}%
\pgfsetstrokecolor{currentstroke}%
\pgfsetdash{}{0pt}%
\pgfsys@defobject{currentmarker}{\pgfqpoint{-0.048611in}{0.000000in}}{\pgfqpoint{-0.000000in}{0.000000in}}{%
\pgfpathmoveto{\pgfqpoint{-0.000000in}{0.000000in}}%
\pgfpathlineto{\pgfqpoint{-0.048611in}{0.000000in}}%
\pgfusepath{stroke,fill}%
}%
\begin{pgfscope}%
\pgfsys@transformshift{0.553581in}{0.551944in}%
\pgfsys@useobject{currentmarker}{}%
\end{pgfscope}%
\end{pgfscope}%
\begin{pgfscope}%
\definecolor{textcolor}{rgb}{0.000000,0.000000,0.000000}%
\pgfsetstrokecolor{textcolor}%
\pgfsetfillcolor{textcolor}%
\pgftext[x=0.278889in, y=0.503750in, left, base]{\color{textcolor}\rmfamily\fontsize{10.000000}{12.000000}\selectfont \(\displaystyle {0.0}\)}%
\end{pgfscope}%
\begin{pgfscope}%
\pgfsetbuttcap%
\pgfsetroundjoin%
\definecolor{currentfill}{rgb}{0.000000,0.000000,0.000000}%
\pgfsetfillcolor{currentfill}%
\pgfsetlinewidth{0.803000pt}%
\definecolor{currentstroke}{rgb}{0.000000,0.000000,0.000000}%
\pgfsetstrokecolor{currentstroke}%
\pgfsetdash{}{0pt}%
\pgfsys@defobject{currentmarker}{\pgfqpoint{-0.048611in}{0.000000in}}{\pgfqpoint{-0.000000in}{0.000000in}}{%
\pgfpathmoveto{\pgfqpoint{-0.000000in}{0.000000in}}%
\pgfpathlineto{\pgfqpoint{-0.048611in}{0.000000in}}%
\pgfusepath{stroke,fill}%
}%
\begin{pgfscope}%
\pgfsys@transformshift{0.553581in}{1.076944in}%
\pgfsys@useobject{currentmarker}{}%
\end{pgfscope}%
\end{pgfscope}%
\begin{pgfscope}%
\definecolor{textcolor}{rgb}{0.000000,0.000000,0.000000}%
\pgfsetstrokecolor{textcolor}%
\pgfsetfillcolor{textcolor}%
\pgftext[x=0.278889in, y=1.028750in, left, base]{\color{textcolor}\rmfamily\fontsize{10.000000}{12.000000}\selectfont \(\displaystyle {0.5}\)}%
\end{pgfscope}%
\begin{pgfscope}%
\pgfsetbuttcap%
\pgfsetroundjoin%
\definecolor{currentfill}{rgb}{0.000000,0.000000,0.000000}%
\pgfsetfillcolor{currentfill}%
\pgfsetlinewidth{0.803000pt}%
\definecolor{currentstroke}{rgb}{0.000000,0.000000,0.000000}%
\pgfsetstrokecolor{currentstroke}%
\pgfsetdash{}{0pt}%
\pgfsys@defobject{currentmarker}{\pgfqpoint{-0.048611in}{0.000000in}}{\pgfqpoint{-0.000000in}{0.000000in}}{%
\pgfpathmoveto{\pgfqpoint{-0.000000in}{0.000000in}}%
\pgfpathlineto{\pgfqpoint{-0.048611in}{0.000000in}}%
\pgfusepath{stroke,fill}%
}%
\begin{pgfscope}%
\pgfsys@transformshift{0.553581in}{1.601944in}%
\pgfsys@useobject{currentmarker}{}%
\end{pgfscope}%
\end{pgfscope}%
\begin{pgfscope}%
\definecolor{textcolor}{rgb}{0.000000,0.000000,0.000000}%
\pgfsetstrokecolor{textcolor}%
\pgfsetfillcolor{textcolor}%
\pgftext[x=0.278889in, y=1.553750in, left, base]{\color{textcolor}\rmfamily\fontsize{10.000000}{12.000000}\selectfont \(\displaystyle {1.0}\)}%
\end{pgfscope}%
\begin{pgfscope}%
\definecolor{textcolor}{rgb}{0.000000,0.000000,0.000000}%
\pgfsetstrokecolor{textcolor}%
\pgfsetfillcolor{textcolor}%
\pgftext[x=0.223333in,y=1.076944in,,bottom,rotate=90.000000]{\color{textcolor}\rmfamily\fontsize{10.000000}{12.000000}\selectfont True positive rate}%
\end{pgfscope}%
\begin{pgfscope}%
\pgfpathrectangle{\pgfqpoint{0.553581in}{0.499444in}}{\pgfqpoint{1.550000in}{1.155000in}}%
\pgfusepath{clip}%
\pgfsetbuttcap%
\pgfsetroundjoin%
\pgfsetlinewidth{1.505625pt}%
\definecolor{currentstroke}{rgb}{0.000000,0.000000,0.000000}%
\pgfsetstrokecolor{currentstroke}%
\pgfsetdash{{5.550000pt}{2.400000pt}}{0.000000pt}%
\pgfpathmoveto{\pgfqpoint{0.624035in}{0.551944in}}%
\pgfpathlineto{\pgfqpoint{2.033126in}{1.601944in}}%
\pgfusepath{stroke}%
\end{pgfscope}%
\begin{pgfscope}%
\pgfpathrectangle{\pgfqpoint{0.553581in}{0.499444in}}{\pgfqpoint{1.550000in}{1.155000in}}%
\pgfusepath{clip}%
\pgfsetrectcap%
\pgfsetroundjoin%
\pgfsetlinewidth{1.505625pt}%
\definecolor{currentstroke}{rgb}{0.000000,0.000000,0.000000}%
\pgfsetstrokecolor{currentstroke}%
\pgfsetdash{}{0pt}%
\pgfpathmoveto{\pgfqpoint{0.624035in}{0.551944in}}%
\pgfpathlineto{\pgfqpoint{0.626207in}{0.552574in}}%
\pgfpathlineto{\pgfqpoint{0.627318in}{0.561464in}}%
\pgfpathlineto{\pgfqpoint{0.628014in}{0.562514in}}%
\pgfpathlineto{\pgfqpoint{0.629125in}{0.563634in}}%
\pgfpathlineto{\pgfqpoint{0.629605in}{0.564614in}}%
\pgfpathlineto{\pgfqpoint{0.630699in}{0.567694in}}%
\pgfpathlineto{\pgfqpoint{0.631130in}{0.568604in}}%
\pgfpathlineto{\pgfqpoint{0.632225in}{0.573294in}}%
\pgfpathlineto{\pgfqpoint{0.632772in}{0.574064in}}%
\pgfpathlineto{\pgfqpoint{0.633866in}{0.579944in}}%
\pgfpathlineto{\pgfqpoint{0.634247in}{0.580854in}}%
\pgfpathlineto{\pgfqpoint{0.635358in}{0.585824in}}%
\pgfpathlineto{\pgfqpoint{0.635540in}{0.586874in}}%
\pgfpathlineto{\pgfqpoint{0.636634in}{0.593244in}}%
\pgfpathlineto{\pgfqpoint{0.636833in}{0.594154in}}%
\pgfpathlineto{\pgfqpoint{0.637944in}{0.600734in}}%
\pgfpathlineto{\pgfqpoint{0.638126in}{0.601784in}}%
\pgfpathlineto{\pgfqpoint{0.639187in}{0.608714in}}%
\pgfpathlineto{\pgfqpoint{0.639353in}{0.609694in}}%
\pgfpathlineto{\pgfqpoint{0.640464in}{0.616834in}}%
\pgfpathlineto{\pgfqpoint{0.640712in}{0.617884in}}%
\pgfpathlineto{\pgfqpoint{0.641806in}{0.625864in}}%
\pgfpathlineto{\pgfqpoint{0.642038in}{0.626844in}}%
\pgfpathlineto{\pgfqpoint{0.643149in}{0.635804in}}%
\pgfpathlineto{\pgfqpoint{0.643348in}{0.636854in}}%
\pgfpathlineto{\pgfqpoint{0.644392in}{0.645044in}}%
\pgfpathlineto{\pgfqpoint{0.644641in}{0.645884in}}%
\pgfpathlineto{\pgfqpoint{0.645752in}{0.652394in}}%
\pgfpathlineto{\pgfqpoint{0.645868in}{0.653164in}}%
\pgfpathlineto{\pgfqpoint{0.646979in}{0.661424in}}%
\pgfpathlineto{\pgfqpoint{0.647327in}{0.662264in}}%
\pgfpathlineto{\pgfqpoint{0.648437in}{0.668914in}}%
\pgfpathlineto{\pgfqpoint{0.648587in}{0.669824in}}%
\pgfpathlineto{\pgfqpoint{0.649697in}{0.678294in}}%
\pgfpathlineto{\pgfqpoint{0.649813in}{0.679274in}}%
\pgfpathlineto{\pgfqpoint{0.650924in}{0.687044in}}%
\pgfpathlineto{\pgfqpoint{0.651140in}{0.687884in}}%
\pgfpathlineto{\pgfqpoint{0.652250in}{0.696144in}}%
\pgfpathlineto{\pgfqpoint{0.652333in}{0.697194in}}%
\pgfpathlineto{\pgfqpoint{0.653444in}{0.706434in}}%
\pgfpathlineto{\pgfqpoint{0.653858in}{0.707414in}}%
\pgfpathlineto{\pgfqpoint{0.654969in}{0.716584in}}%
\pgfpathlineto{\pgfqpoint{0.655151in}{0.717564in}}%
\pgfpathlineto{\pgfqpoint{0.656262in}{0.725194in}}%
\pgfpathlineto{\pgfqpoint{0.656444in}{0.726174in}}%
\pgfpathlineto{\pgfqpoint{0.657538in}{0.733244in}}%
\pgfpathlineto{\pgfqpoint{0.657704in}{0.734294in}}%
\pgfpathlineto{\pgfqpoint{0.658815in}{0.743464in}}%
\pgfpathlineto{\pgfqpoint{0.659064in}{0.744514in}}%
\pgfpathlineto{\pgfqpoint{0.660158in}{0.751164in}}%
\pgfpathlineto{\pgfqpoint{0.660290in}{0.751794in}}%
\pgfpathlineto{\pgfqpoint{0.661401in}{0.759354in}}%
\pgfpathlineto{\pgfqpoint{0.661633in}{0.760404in}}%
\pgfpathlineto{\pgfqpoint{0.662744in}{0.769784in}}%
\pgfpathlineto{\pgfqpoint{0.662827in}{0.770344in}}%
\pgfpathlineto{\pgfqpoint{0.663937in}{0.777764in}}%
\pgfpathlineto{\pgfqpoint{0.664120in}{0.778814in}}%
\pgfpathlineto{\pgfqpoint{0.665181in}{0.783854in}}%
\pgfpathlineto{\pgfqpoint{0.665297in}{0.784834in}}%
\pgfpathlineto{\pgfqpoint{0.666407in}{0.795614in}}%
\pgfpathlineto{\pgfqpoint{0.666656in}{0.796664in}}%
\pgfpathlineto{\pgfqpoint{0.667767in}{0.801564in}}%
\pgfpathlineto{\pgfqpoint{0.667899in}{0.802404in}}%
\pgfpathlineto{\pgfqpoint{0.669010in}{0.809334in}}%
\pgfpathlineto{\pgfqpoint{0.669342in}{0.810384in}}%
\pgfpathlineto{\pgfqpoint{0.670452in}{0.818084in}}%
\pgfpathlineto{\pgfqpoint{0.670701in}{0.819134in}}%
\pgfpathlineto{\pgfqpoint{0.671812in}{0.825364in}}%
\pgfpathlineto{\pgfqpoint{0.671911in}{0.826414in}}%
\pgfpathlineto{\pgfqpoint{0.673022in}{0.833694in}}%
\pgfpathlineto{\pgfqpoint{0.673138in}{0.834744in}}%
\pgfpathlineto{\pgfqpoint{0.674215in}{0.838944in}}%
\pgfpathlineto{\pgfqpoint{0.674530in}{0.839994in}}%
\pgfpathlineto{\pgfqpoint{0.675608in}{0.846014in}}%
\pgfpathlineto{\pgfqpoint{0.675790in}{0.846994in}}%
\pgfpathlineto{\pgfqpoint{0.676901in}{0.853854in}}%
\pgfpathlineto{\pgfqpoint{0.677199in}{0.854834in}}%
\pgfpathlineto{\pgfqpoint{0.678310in}{0.859034in}}%
\pgfpathlineto{\pgfqpoint{0.678492in}{0.860084in}}%
\pgfpathlineto{\pgfqpoint{0.679587in}{0.867014in}}%
\pgfpathlineto{\pgfqpoint{0.679868in}{0.867924in}}%
\pgfpathlineto{\pgfqpoint{0.680963in}{0.874504in}}%
\pgfpathlineto{\pgfqpoint{0.681228in}{0.875484in}}%
\pgfpathlineto{\pgfqpoint{0.682338in}{0.882204in}}%
\pgfpathlineto{\pgfqpoint{0.682587in}{0.883254in}}%
\pgfpathlineto{\pgfqpoint{0.683681in}{0.889484in}}%
\pgfpathlineto{\pgfqpoint{0.683980in}{0.890394in}}%
\pgfpathlineto{\pgfqpoint{0.685074in}{0.897394in}}%
\pgfpathlineto{\pgfqpoint{0.685389in}{0.898304in}}%
\pgfpathlineto{\pgfqpoint{0.686450in}{0.904184in}}%
\pgfpathlineto{\pgfqpoint{0.686748in}{0.905094in}}%
\pgfpathlineto{\pgfqpoint{0.687859in}{0.910204in}}%
\pgfpathlineto{\pgfqpoint{0.688074in}{0.911114in}}%
\pgfpathlineto{\pgfqpoint{0.689185in}{0.915034in}}%
\pgfpathlineto{\pgfqpoint{0.689400in}{0.916084in}}%
\pgfpathlineto{\pgfqpoint{0.690511in}{0.922664in}}%
\pgfpathlineto{\pgfqpoint{0.690793in}{0.923714in}}%
\pgfpathlineto{\pgfqpoint{0.691887in}{0.928754in}}%
\pgfpathlineto{\pgfqpoint{0.692318in}{0.929734in}}%
\pgfpathlineto{\pgfqpoint{0.693412in}{0.935334in}}%
\pgfpathlineto{\pgfqpoint{0.693611in}{0.936384in}}%
\pgfpathlineto{\pgfqpoint{0.694705in}{0.940864in}}%
\pgfpathlineto{\pgfqpoint{0.695053in}{0.941914in}}%
\pgfpathlineto{\pgfqpoint{0.696164in}{0.948634in}}%
\pgfpathlineto{\pgfqpoint{0.696645in}{0.949684in}}%
\pgfpathlineto{\pgfqpoint{0.697756in}{0.954164in}}%
\pgfpathlineto{\pgfqpoint{0.697921in}{0.955144in}}%
\pgfpathlineto{\pgfqpoint{0.699015in}{0.960464in}}%
\pgfpathlineto{\pgfqpoint{0.699314in}{0.961304in}}%
\pgfpathlineto{\pgfqpoint{0.700425in}{0.965294in}}%
\pgfpathlineto{\pgfqpoint{0.700657in}{0.966344in}}%
\pgfpathlineto{\pgfqpoint{0.701767in}{0.971314in}}%
\pgfpathlineto{\pgfqpoint{0.702049in}{0.972364in}}%
\pgfpathlineto{\pgfqpoint{0.703094in}{0.977194in}}%
\pgfpathlineto{\pgfqpoint{0.703442in}{0.978104in}}%
\pgfpathlineto{\pgfqpoint{0.704536in}{0.982514in}}%
\pgfpathlineto{\pgfqpoint{0.704917in}{0.983564in}}%
\pgfpathlineto{\pgfqpoint{0.706011in}{0.988464in}}%
\pgfpathlineto{\pgfqpoint{0.706160in}{0.989444in}}%
\pgfpathlineto{\pgfqpoint{0.707271in}{0.993504in}}%
\pgfpathlineto{\pgfqpoint{0.707619in}{0.994414in}}%
\pgfpathlineto{\pgfqpoint{0.708697in}{0.998054in}}%
\pgfpathlineto{\pgfqpoint{0.709061in}{0.999104in}}%
\pgfpathlineto{\pgfqpoint{0.710172in}{1.004564in}}%
\pgfpathlineto{\pgfqpoint{0.710504in}{1.005614in}}%
\pgfpathlineto{\pgfqpoint{0.711614in}{1.009604in}}%
\pgfpathlineto{\pgfqpoint{0.711830in}{1.010584in}}%
\pgfpathlineto{\pgfqpoint{0.712874in}{1.015134in}}%
\pgfpathlineto{\pgfqpoint{0.713123in}{1.016184in}}%
\pgfpathlineto{\pgfqpoint{0.714184in}{1.020244in}}%
\pgfpathlineto{\pgfqpoint{0.714598in}{1.021294in}}%
\pgfpathlineto{\pgfqpoint{0.715709in}{1.024304in}}%
\pgfpathlineto{\pgfqpoint{0.716173in}{1.025354in}}%
\pgfpathlineto{\pgfqpoint{0.717284in}{1.029624in}}%
\pgfpathlineto{\pgfqpoint{0.717466in}{1.030604in}}%
\pgfpathlineto{\pgfqpoint{0.718560in}{1.035014in}}%
\pgfpathlineto{\pgfqpoint{0.718776in}{1.035994in}}%
\pgfpathlineto{\pgfqpoint{0.719853in}{1.041034in}}%
\pgfpathlineto{\pgfqpoint{0.720202in}{1.042084in}}%
\pgfpathlineto{\pgfqpoint{0.721312in}{1.046144in}}%
\pgfpathlineto{\pgfqpoint{0.721694in}{1.047194in}}%
\pgfpathlineto{\pgfqpoint{0.722788in}{1.050904in}}%
\pgfpathlineto{\pgfqpoint{0.723086in}{1.051954in}}%
\pgfpathlineto{\pgfqpoint{0.724180in}{1.056364in}}%
\pgfpathlineto{\pgfqpoint{0.724727in}{1.057414in}}%
\pgfpathlineto{\pgfqpoint{0.725805in}{1.061264in}}%
\pgfpathlineto{\pgfqpoint{0.726236in}{1.062314in}}%
\pgfpathlineto{\pgfqpoint{0.727280in}{1.065744in}}%
\pgfpathlineto{\pgfqpoint{0.727728in}{1.066794in}}%
\pgfpathlineto{\pgfqpoint{0.728805in}{1.069874in}}%
\pgfpathlineto{\pgfqpoint{0.729253in}{1.070924in}}%
\pgfpathlineto{\pgfqpoint{0.730364in}{1.074144in}}%
\pgfpathlineto{\pgfqpoint{0.730695in}{1.075124in}}%
\pgfpathlineto{\pgfqpoint{0.731789in}{1.079044in}}%
\pgfpathlineto{\pgfqpoint{0.732336in}{1.080024in}}%
\pgfpathlineto{\pgfqpoint{0.733447in}{1.084574in}}%
\pgfpathlineto{\pgfqpoint{0.734094in}{1.085624in}}%
\pgfpathlineto{\pgfqpoint{0.735188in}{1.089544in}}%
\pgfpathlineto{\pgfqpoint{0.735552in}{1.090594in}}%
\pgfpathlineto{\pgfqpoint{0.736597in}{1.093674in}}%
\pgfpathlineto{\pgfqpoint{0.737177in}{1.094724in}}%
\pgfpathlineto{\pgfqpoint{0.738221in}{1.097524in}}%
\pgfpathlineto{\pgfqpoint{0.738702in}{1.098574in}}%
\pgfpathlineto{\pgfqpoint{0.739780in}{1.102424in}}%
\pgfpathlineto{\pgfqpoint{0.740128in}{1.103474in}}%
\pgfpathlineto{\pgfqpoint{0.741238in}{1.106624in}}%
\pgfpathlineto{\pgfqpoint{0.741636in}{1.107674in}}%
\pgfpathlineto{\pgfqpoint{0.742697in}{1.111034in}}%
\pgfpathlineto{\pgfqpoint{0.743095in}{1.112084in}}%
\pgfpathlineto{\pgfqpoint{0.744206in}{1.115024in}}%
\pgfpathlineto{\pgfqpoint{0.744769in}{1.116074in}}%
\pgfpathlineto{\pgfqpoint{0.745051in}{1.117054in}}%
\pgfpathlineto{\pgfqpoint{0.745068in}{1.117054in}}%
\pgfpathlineto{\pgfqpoint{0.755943in}{1.118104in}}%
\pgfpathlineto{\pgfqpoint{0.757053in}{1.121814in}}%
\pgfpathlineto{\pgfqpoint{0.757600in}{1.122864in}}%
\pgfpathlineto{\pgfqpoint{0.758678in}{1.125384in}}%
\pgfpathlineto{\pgfqpoint{0.759225in}{1.126434in}}%
\pgfpathlineto{\pgfqpoint{0.760269in}{1.129094in}}%
\pgfpathlineto{\pgfqpoint{0.760817in}{1.130144in}}%
\pgfpathlineto{\pgfqpoint{0.761911in}{1.133224in}}%
\pgfpathlineto{\pgfqpoint{0.762474in}{1.134274in}}%
\pgfpathlineto{\pgfqpoint{0.763535in}{1.137844in}}%
\pgfpathlineto{\pgfqpoint{0.764165in}{1.138824in}}%
\pgfpathlineto{\pgfqpoint{0.765259in}{1.142324in}}%
\pgfpathlineto{\pgfqpoint{0.766055in}{1.143374in}}%
\pgfpathlineto{\pgfqpoint{0.767166in}{1.146314in}}%
\pgfpathlineto{\pgfqpoint{0.767696in}{1.147364in}}%
\pgfpathlineto{\pgfqpoint{0.768790in}{1.149954in}}%
\pgfpathlineto{\pgfqpoint{0.769603in}{1.151004in}}%
\pgfpathlineto{\pgfqpoint{0.770514in}{1.153314in}}%
\pgfpathlineto{\pgfqpoint{0.771161in}{1.154364in}}%
\pgfpathlineto{\pgfqpoint{0.772272in}{1.157374in}}%
\pgfpathlineto{\pgfqpoint{0.772852in}{1.158424in}}%
\pgfpathlineto{\pgfqpoint{0.773946in}{1.160664in}}%
\pgfpathlineto{\pgfqpoint{0.774609in}{1.161714in}}%
\pgfpathlineto{\pgfqpoint{0.775703in}{1.164794in}}%
\pgfpathlineto{\pgfqpoint{0.776267in}{1.165774in}}%
\pgfpathlineto{\pgfqpoint{0.777344in}{1.168714in}}%
\pgfpathlineto{\pgfqpoint{0.777759in}{1.169764in}}%
\pgfpathlineto{\pgfqpoint{0.778836in}{1.172144in}}%
\pgfpathlineto{\pgfqpoint{0.779350in}{1.173124in}}%
\pgfpathlineto{\pgfqpoint{0.780444in}{1.174524in}}%
\pgfpathlineto{\pgfqpoint{0.781008in}{1.175574in}}%
\pgfpathlineto{\pgfqpoint{0.782052in}{1.177254in}}%
\pgfpathlineto{\pgfqpoint{0.782699in}{1.178304in}}%
\pgfpathlineto{\pgfqpoint{0.783793in}{1.180894in}}%
\pgfpathlineto{\pgfqpoint{0.784290in}{1.181874in}}%
\pgfpathlineto{\pgfqpoint{0.785384in}{1.184044in}}%
\pgfpathlineto{\pgfqpoint{0.785981in}{1.185094in}}%
\pgfpathlineto{\pgfqpoint{0.787075in}{1.186914in}}%
\pgfpathlineto{\pgfqpoint{0.787755in}{1.187894in}}%
\pgfpathlineto{\pgfqpoint{0.788866in}{1.189784in}}%
\pgfpathlineto{\pgfqpoint{0.789297in}{1.190764in}}%
\pgfpathlineto{\pgfqpoint{0.790407in}{1.192794in}}%
\pgfpathlineto{\pgfqpoint{0.790689in}{1.193634in}}%
\pgfpathlineto{\pgfqpoint{0.791783in}{1.196084in}}%
\pgfpathlineto{\pgfqpoint{0.792347in}{1.197134in}}%
\pgfpathlineto{\pgfqpoint{0.793358in}{1.199724in}}%
\pgfpathlineto{\pgfqpoint{0.794054in}{1.200774in}}%
\pgfpathlineto{\pgfqpoint{0.795099in}{1.201964in}}%
\pgfpathlineto{\pgfqpoint{0.796044in}{1.203014in}}%
\pgfpathlineto{\pgfqpoint{0.797138in}{1.205534in}}%
\pgfpathlineto{\pgfqpoint{0.797718in}{1.206584in}}%
\pgfpathlineto{\pgfqpoint{0.798812in}{1.208474in}}%
\pgfpathlineto{\pgfqpoint{0.799409in}{1.209524in}}%
\pgfpathlineto{\pgfqpoint{0.800520in}{1.212114in}}%
\pgfpathlineto{\pgfqpoint{0.801382in}{1.213094in}}%
\pgfpathlineto{\pgfqpoint{0.802492in}{1.215894in}}%
\pgfpathlineto{\pgfqpoint{0.803106in}{1.216664in}}%
\pgfpathlineto{\pgfqpoint{0.804134in}{1.218554in}}%
\pgfpathlineto{\pgfqpoint{0.805128in}{1.219604in}}%
\pgfpathlineto{\pgfqpoint{0.806222in}{1.221774in}}%
\pgfpathlineto{\pgfqpoint{0.806786in}{1.222614in}}%
\pgfpathlineto{\pgfqpoint{0.807897in}{1.225484in}}%
\pgfpathlineto{\pgfqpoint{0.808659in}{1.226534in}}%
\pgfpathlineto{\pgfqpoint{0.809753in}{1.228284in}}%
\pgfpathlineto{\pgfqpoint{0.810267in}{1.229334in}}%
\pgfpathlineto{\pgfqpoint{0.811378in}{1.231224in}}%
\pgfpathlineto{\pgfqpoint{0.812654in}{1.232274in}}%
\pgfpathlineto{\pgfqpoint{0.813682in}{1.233744in}}%
\pgfpathlineto{\pgfqpoint{0.814180in}{1.234794in}}%
\pgfpathlineto{\pgfqpoint{0.815290in}{1.236614in}}%
\pgfpathlineto{\pgfqpoint{0.816053in}{1.237664in}}%
\pgfpathlineto{\pgfqpoint{0.816948in}{1.239484in}}%
\pgfpathlineto{\pgfqpoint{0.818109in}{1.240464in}}%
\pgfpathlineto{\pgfqpoint{0.819203in}{1.242424in}}%
\pgfpathlineto{\pgfqpoint{0.819949in}{1.243404in}}%
\pgfpathlineto{\pgfqpoint{0.821043in}{1.245364in}}%
\pgfpathlineto{\pgfqpoint{0.821838in}{1.246414in}}%
\pgfpathlineto{\pgfqpoint{0.822899in}{1.248374in}}%
\pgfpathlineto{\pgfqpoint{0.823828in}{1.249424in}}%
\pgfpathlineto{\pgfqpoint{0.824905in}{1.251314in}}%
\pgfpathlineto{\pgfqpoint{0.826049in}{1.252364in}}%
\pgfpathlineto{\pgfqpoint{0.827110in}{1.254044in}}%
\pgfpathlineto{\pgfqpoint{0.828204in}{1.255024in}}%
\pgfpathlineto{\pgfqpoint{0.829298in}{1.257404in}}%
\pgfpathlineto{\pgfqpoint{0.830094in}{1.258454in}}%
\pgfpathlineto{\pgfqpoint{0.831188in}{1.259784in}}%
\pgfpathlineto{\pgfqpoint{0.831901in}{1.260834in}}%
\pgfpathlineto{\pgfqpoint{0.832979in}{1.261954in}}%
\pgfpathlineto{\pgfqpoint{0.833841in}{1.263004in}}%
\pgfpathlineto{\pgfqpoint{0.834769in}{1.263774in}}%
\pgfpathlineto{\pgfqpoint{0.835697in}{1.264824in}}%
\pgfpathlineto{\pgfqpoint{0.836659in}{1.266014in}}%
\pgfpathlineto{\pgfqpoint{0.838234in}{1.267064in}}%
\pgfpathlineto{\pgfqpoint{0.839295in}{1.268814in}}%
\pgfpathlineto{\pgfqpoint{0.840438in}{1.269864in}}%
\pgfpathlineto{\pgfqpoint{0.841549in}{1.271824in}}%
\pgfpathlineto{\pgfqpoint{0.842594in}{1.272874in}}%
\pgfpathlineto{\pgfqpoint{0.843688in}{1.274414in}}%
\pgfpathlineto{\pgfqpoint{0.844881in}{1.275464in}}%
\pgfpathlineto{\pgfqpoint{0.845992in}{1.277844in}}%
\pgfpathlineto{\pgfqpoint{0.846705in}{1.278894in}}%
\pgfpathlineto{\pgfqpoint{0.847633in}{1.280294in}}%
\pgfpathlineto{\pgfqpoint{0.848644in}{1.281274in}}%
\pgfpathlineto{\pgfqpoint{0.849606in}{1.282954in}}%
\pgfpathlineto{\pgfqpoint{0.850882in}{1.284004in}}%
\pgfpathlineto{\pgfqpoint{0.851960in}{1.285194in}}%
\pgfpathlineto{\pgfqpoint{0.853120in}{1.286244in}}%
\pgfpathlineto{\pgfqpoint{0.854198in}{1.287644in}}%
\pgfpathlineto{\pgfqpoint{0.855176in}{1.288694in}}%
\pgfpathlineto{\pgfqpoint{0.856237in}{1.290234in}}%
\pgfpathlineto{\pgfqpoint{0.856917in}{1.291144in}}%
\pgfpathlineto{\pgfqpoint{0.858027in}{1.292754in}}%
\pgfpathlineto{\pgfqpoint{0.859287in}{1.293804in}}%
\pgfpathlineto{\pgfqpoint{0.860232in}{1.294714in}}%
\pgfpathlineto{\pgfqpoint{0.861160in}{1.295764in}}%
\pgfpathlineto{\pgfqpoint{0.862155in}{1.296814in}}%
\pgfpathlineto{\pgfqpoint{0.863746in}{1.297794in}}%
\pgfpathlineto{\pgfqpoint{0.864807in}{1.299054in}}%
\pgfpathlineto{\pgfqpoint{0.866084in}{1.300104in}}%
\pgfpathlineto{\pgfqpoint{0.867128in}{1.301854in}}%
\pgfpathlineto{\pgfqpoint{0.867775in}{1.302904in}}%
\pgfpathlineto{\pgfqpoint{0.868886in}{1.304164in}}%
\pgfpathlineto{\pgfqpoint{0.869897in}{1.305144in}}%
\pgfpathlineto{\pgfqpoint{0.870991in}{1.306614in}}%
\pgfpathlineto{\pgfqpoint{0.872218in}{1.307664in}}%
\pgfpathlineto{\pgfqpoint{0.873312in}{1.309624in}}%
\pgfpathlineto{\pgfqpoint{0.874572in}{1.310674in}}%
\pgfpathlineto{\pgfqpoint{0.875616in}{1.312284in}}%
\pgfpathlineto{\pgfqpoint{0.876263in}{1.313334in}}%
\pgfpathlineto{\pgfqpoint{0.877373in}{1.314804in}}%
\pgfpathlineto{\pgfqpoint{0.878036in}{1.315854in}}%
\pgfpathlineto{\pgfqpoint{0.879081in}{1.316974in}}%
\pgfpathlineto{\pgfqpoint{0.880191in}{1.318024in}}%
\pgfpathlineto{\pgfqpoint{0.881252in}{1.319564in}}%
\pgfpathlineto{\pgfqpoint{0.882927in}{1.320544in}}%
\pgfpathlineto{\pgfqpoint{0.883954in}{1.321384in}}%
\pgfpathlineto{\pgfqpoint{0.885165in}{1.322434in}}%
\pgfpathlineto{\pgfqpoint{0.886226in}{1.323624in}}%
\pgfpathlineto{\pgfqpoint{0.887121in}{1.324674in}}%
\pgfpathlineto{\pgfqpoint{0.888215in}{1.325864in}}%
\pgfpathlineto{\pgfqpoint{0.889624in}{1.326914in}}%
\pgfpathlineto{\pgfqpoint{0.890619in}{1.328104in}}%
\pgfpathlineto{\pgfqpoint{0.892044in}{1.329154in}}%
\pgfpathlineto{\pgfqpoint{0.892857in}{1.330064in}}%
\pgfpathlineto{\pgfqpoint{0.894100in}{1.331044in}}%
\pgfpathlineto{\pgfqpoint{0.894746in}{1.331814in}}%
\pgfpathlineto{\pgfqpoint{0.896089in}{1.332864in}}%
\pgfpathlineto{\pgfqpoint{0.897200in}{1.333984in}}%
\pgfpathlineto{\pgfqpoint{0.898327in}{1.335034in}}%
\pgfpathlineto{\pgfqpoint{0.899338in}{1.336714in}}%
\pgfpathlineto{\pgfqpoint{0.900267in}{1.337694in}}%
\pgfpathlineto{\pgfqpoint{0.901311in}{1.338534in}}%
\pgfpathlineto{\pgfqpoint{0.902554in}{1.339584in}}%
\pgfpathlineto{\pgfqpoint{0.903649in}{1.340494in}}%
\pgfpathlineto{\pgfqpoint{0.905240in}{1.341544in}}%
\pgfpathlineto{\pgfqpoint{0.906334in}{1.342874in}}%
\pgfpathlineto{\pgfqpoint{0.907743in}{1.343924in}}%
\pgfpathlineto{\pgfqpoint{0.908854in}{1.344764in}}%
\pgfpathlineto{\pgfqpoint{0.910048in}{1.345814in}}%
\pgfpathlineto{\pgfqpoint{0.911142in}{1.346654in}}%
\pgfpathlineto{\pgfqpoint{0.912169in}{1.347704in}}%
\pgfpathlineto{\pgfqpoint{0.913114in}{1.348754in}}%
\pgfpathlineto{\pgfqpoint{0.914242in}{1.349804in}}%
\pgfpathlineto{\pgfqpoint{0.915004in}{1.350924in}}%
\pgfpathlineto{\pgfqpoint{0.916894in}{1.351974in}}%
\pgfpathlineto{\pgfqpoint{0.917988in}{1.353234in}}%
\pgfpathlineto{\pgfqpoint{0.919646in}{1.354284in}}%
\pgfpathlineto{\pgfqpoint{0.920757in}{1.356034in}}%
\pgfpathlineto{\pgfqpoint{0.922166in}{1.357084in}}%
\pgfpathlineto{\pgfqpoint{0.923260in}{1.358204in}}%
\pgfpathlineto{\pgfqpoint{0.925332in}{1.359254in}}%
\pgfpathlineto{\pgfqpoint{0.926376in}{1.360164in}}%
\pgfpathlineto{\pgfqpoint{0.927686in}{1.361214in}}%
\pgfpathlineto{\pgfqpoint{0.928780in}{1.361914in}}%
\pgfpathlineto{\pgfqpoint{0.930886in}{1.362964in}}%
\pgfpathlineto{\pgfqpoint{0.931980in}{1.364294in}}%
\pgfpathlineto{\pgfqpoint{0.932941in}{1.365344in}}%
\pgfpathlineto{\pgfqpoint{0.934035in}{1.366604in}}%
\pgfpathlineto{\pgfqpoint{0.935925in}{1.367654in}}%
\pgfpathlineto{\pgfqpoint{0.937003in}{1.368984in}}%
\pgfpathlineto{\pgfqpoint{0.939307in}{1.370034in}}%
\pgfpathlineto{\pgfqpoint{0.940302in}{1.371014in}}%
\pgfpathlineto{\pgfqpoint{0.941876in}{1.372064in}}%
\pgfpathlineto{\pgfqpoint{0.942871in}{1.372694in}}%
\pgfpathlineto{\pgfqpoint{0.944562in}{1.373674in}}%
\pgfpathlineto{\pgfqpoint{0.945573in}{1.374934in}}%
\pgfpathlineto{\pgfqpoint{0.947314in}{1.375984in}}%
\pgfpathlineto{\pgfqpoint{0.948159in}{1.376824in}}%
\pgfpathlineto{\pgfqpoint{0.950663in}{1.377874in}}%
\pgfpathlineto{\pgfqpoint{0.951707in}{1.378504in}}%
\pgfpathlineto{\pgfqpoint{0.953083in}{1.379484in}}%
\pgfpathlineto{\pgfqpoint{0.954127in}{1.380394in}}%
\pgfpathlineto{\pgfqpoint{0.956183in}{1.381444in}}%
\pgfpathlineto{\pgfqpoint{0.957244in}{1.382914in}}%
\pgfpathlineto{\pgfqpoint{0.958736in}{1.383964in}}%
\pgfpathlineto{\pgfqpoint{0.959714in}{1.384874in}}%
\pgfpathlineto{\pgfqpoint{0.961902in}{1.385924in}}%
\pgfpathlineto{\pgfqpoint{0.962781in}{1.386414in}}%
\pgfpathlineto{\pgfqpoint{0.965599in}{1.387464in}}%
\pgfpathlineto{\pgfqpoint{0.966643in}{1.388514in}}%
\pgfpathlineto{\pgfqpoint{0.968069in}{1.389564in}}%
\pgfpathlineto{\pgfqpoint{0.968931in}{1.390404in}}%
\pgfpathlineto{\pgfqpoint{0.971733in}{1.391454in}}%
\pgfpathlineto{\pgfqpoint{0.972843in}{1.391874in}}%
\pgfpathlineto{\pgfqpoint{0.974286in}{1.392924in}}%
\pgfpathlineto{\pgfqpoint{0.975197in}{1.393554in}}%
\pgfpathlineto{\pgfqpoint{0.977336in}{1.394604in}}%
\pgfpathlineto{\pgfqpoint{0.978413in}{1.395654in}}%
\pgfpathlineto{\pgfqpoint{0.979607in}{1.396704in}}%
\pgfpathlineto{\pgfqpoint{0.980635in}{1.397614in}}%
\pgfpathlineto{\pgfqpoint{0.981729in}{1.398664in}}%
\pgfpathlineto{\pgfqpoint{0.982690in}{1.399224in}}%
\pgfpathlineto{\pgfqpoint{0.984365in}{1.400274in}}%
\pgfpathlineto{\pgfqpoint{0.985343in}{1.401044in}}%
\pgfpathlineto{\pgfqpoint{0.987050in}{1.402094in}}%
\pgfpathlineto{\pgfqpoint{0.987912in}{1.403004in}}%
\pgfpathlineto{\pgfqpoint{0.990051in}{1.404054in}}%
\pgfpathlineto{\pgfqpoint{0.990631in}{1.404474in}}%
\pgfpathlineto{\pgfqpoint{0.992935in}{1.405524in}}%
\pgfpathlineto{\pgfqpoint{0.993847in}{1.406294in}}%
\pgfpathlineto{\pgfqpoint{0.995787in}{1.407344in}}%
\pgfpathlineto{\pgfqpoint{0.996798in}{1.408534in}}%
\pgfpathlineto{\pgfqpoint{0.998456in}{1.409444in}}%
\pgfpathlineto{\pgfqpoint{0.999533in}{1.410424in}}%
\pgfpathlineto{\pgfqpoint{1.001224in}{1.411404in}}%
\pgfpathlineto{\pgfqpoint{1.002235in}{1.412314in}}%
\pgfpathlineto{\pgfqpoint{1.003545in}{1.413364in}}%
\pgfpathlineto{\pgfqpoint{1.004423in}{1.414274in}}%
\pgfpathlineto{\pgfqpoint{1.005882in}{1.415324in}}%
\pgfpathlineto{\pgfqpoint{1.006678in}{1.415954in}}%
\pgfpathlineto{\pgfqpoint{1.009165in}{1.417004in}}%
\pgfpathlineto{\pgfqpoint{1.010226in}{1.417774in}}%
\pgfpathlineto{\pgfqpoint{1.012861in}{1.418824in}}%
\pgfpathlineto{\pgfqpoint{1.013972in}{1.419524in}}%
\pgfpathlineto{\pgfqpoint{1.016442in}{1.420504in}}%
\pgfpathlineto{\pgfqpoint{1.017437in}{1.421274in}}%
\pgfpathlineto{\pgfqpoint{1.019890in}{1.422324in}}%
\pgfpathlineto{\pgfqpoint{1.020852in}{1.423444in}}%
\pgfpathlineto{\pgfqpoint{1.022626in}{1.424494in}}%
\pgfpathlineto{\pgfqpoint{1.023488in}{1.425334in}}%
\pgfpathlineto{\pgfqpoint{1.026405in}{1.426384in}}%
\pgfpathlineto{\pgfqpoint{1.027516in}{1.427224in}}%
\pgfpathlineto{\pgfqpoint{1.029787in}{1.428274in}}%
\pgfpathlineto{\pgfqpoint{1.030831in}{1.429184in}}%
\pgfpathlineto{\pgfqpoint{1.032887in}{1.430234in}}%
\pgfpathlineto{\pgfqpoint{1.033882in}{1.431074in}}%
\pgfpathlineto{\pgfqpoint{1.036667in}{1.432124in}}%
\pgfpathlineto{\pgfqpoint{1.037711in}{1.432894in}}%
\pgfpathlineto{\pgfqpoint{1.041872in}{1.433944in}}%
\pgfpathlineto{\pgfqpoint{1.042834in}{1.434504in}}%
\pgfpathlineto{\pgfqpoint{1.046530in}{1.435554in}}%
\pgfpathlineto{\pgfqpoint{1.047575in}{1.436044in}}%
\pgfpathlineto{\pgfqpoint{1.049514in}{1.437094in}}%
\pgfpathlineto{\pgfqpoint{1.050625in}{1.437794in}}%
\pgfpathlineto{\pgfqpoint{1.053095in}{1.438844in}}%
\pgfpathlineto{\pgfqpoint{1.054090in}{1.439544in}}%
\pgfpathlineto{\pgfqpoint{1.056692in}{1.440594in}}%
\pgfpathlineto{\pgfqpoint{1.057704in}{1.441364in}}%
\pgfpathlineto{\pgfqpoint{1.061301in}{1.442414in}}%
\pgfpathlineto{\pgfqpoint{1.062329in}{1.443044in}}%
\pgfpathlineto{\pgfqpoint{1.065893in}{1.444094in}}%
\pgfpathlineto{\pgfqpoint{1.066987in}{1.444584in}}%
\pgfpathlineto{\pgfqpoint{1.068976in}{1.445634in}}%
\pgfpathlineto{\pgfqpoint{1.070037in}{1.446544in}}%
\pgfpathlineto{\pgfqpoint{1.072491in}{1.447594in}}%
\pgfpathlineto{\pgfqpoint{1.073369in}{1.448224in}}%
\pgfpathlineto{\pgfqpoint{1.075840in}{1.449274in}}%
\pgfpathlineto{\pgfqpoint{1.076652in}{1.449694in}}%
\pgfpathlineto{\pgfqpoint{1.079006in}{1.450744in}}%
\pgfpathlineto{\pgfqpoint{1.079967in}{1.451234in}}%
\pgfpathlineto{\pgfqpoint{1.083001in}{1.452214in}}%
\pgfpathlineto{\pgfqpoint{1.084112in}{1.452914in}}%
\pgfpathlineto{\pgfqpoint{1.087726in}{1.453894in}}%
\pgfpathlineto{\pgfqpoint{1.088438in}{1.454384in}}%
\pgfpathlineto{\pgfqpoint{1.092119in}{1.455434in}}%
\pgfpathlineto{\pgfqpoint{1.093064in}{1.456064in}}%
\pgfpathlineto{\pgfqpoint{1.096346in}{1.457114in}}%
\pgfpathlineto{\pgfqpoint{1.097274in}{1.457674in}}%
\pgfpathlineto{\pgfqpoint{1.100092in}{1.458724in}}%
\pgfpathlineto{\pgfqpoint{1.100689in}{1.459004in}}%
\pgfpathlineto{\pgfqpoint{1.104038in}{1.460054in}}%
\pgfpathlineto{\pgfqpoint{1.105149in}{1.460404in}}%
\pgfpathlineto{\pgfqpoint{1.109989in}{1.461454in}}%
\pgfpathlineto{\pgfqpoint{1.111083in}{1.462154in}}%
\pgfpathlineto{\pgfqpoint{1.114813in}{1.463204in}}%
\pgfpathlineto{\pgfqpoint{1.115858in}{1.463624in}}%
\pgfpathlineto{\pgfqpoint{1.119389in}{1.464674in}}%
\pgfpathlineto{\pgfqpoint{1.120433in}{1.465094in}}%
\pgfpathlineto{\pgfqpoint{1.122754in}{1.466144in}}%
\pgfpathlineto{\pgfqpoint{1.123732in}{1.466494in}}%
\pgfpathlineto{\pgfqpoint{1.127910in}{1.467544in}}%
\pgfpathlineto{\pgfqpoint{1.128556in}{1.468034in}}%
\pgfpathlineto{\pgfqpoint{1.131010in}{1.469084in}}%
\pgfpathlineto{\pgfqpoint{1.131822in}{1.469924in}}%
\pgfpathlineto{\pgfqpoint{1.135187in}{1.470974in}}%
\pgfpathlineto{\pgfqpoint{1.135204in}{1.471184in}}%
\pgfpathlineto{\pgfqpoint{1.140956in}{1.472234in}}%
\pgfpathlineto{\pgfqpoint{1.141735in}{1.472584in}}%
\pgfpathlineto{\pgfqpoint{1.145598in}{1.473634in}}%
\pgfpathlineto{\pgfqpoint{1.146675in}{1.474194in}}%
\pgfpathlineto{\pgfqpoint{1.150803in}{1.475244in}}%
\pgfpathlineto{\pgfqpoint{1.151516in}{1.475874in}}%
\pgfpathlineto{\pgfqpoint{1.156058in}{1.476924in}}%
\pgfpathlineto{\pgfqpoint{1.157152in}{1.477414in}}%
\pgfpathlineto{\pgfqpoint{1.162407in}{1.478464in}}%
\pgfpathlineto{\pgfqpoint{1.163452in}{1.479094in}}%
\pgfpathlineto{\pgfqpoint{1.168060in}{1.480144in}}%
\pgfpathlineto{\pgfqpoint{1.168873in}{1.480494in}}%
\pgfpathlineto{\pgfqpoint{1.171923in}{1.481474in}}%
\pgfpathlineto{\pgfqpoint{1.172835in}{1.481754in}}%
\pgfpathlineto{\pgfqpoint{1.177161in}{1.482804in}}%
\pgfpathlineto{\pgfqpoint{1.178156in}{1.483294in}}%
\pgfpathlineto{\pgfqpoint{1.181223in}{1.484344in}}%
\pgfpathlineto{\pgfqpoint{1.181273in}{1.484554in}}%
\pgfpathlineto{\pgfqpoint{1.188633in}{1.485604in}}%
\pgfpathlineto{\pgfqpoint{1.189512in}{1.485884in}}%
\pgfpathlineto{\pgfqpoint{1.193341in}{1.486934in}}%
\pgfpathlineto{\pgfqpoint{1.194203in}{1.487354in}}%
\pgfpathlineto{\pgfqpoint{1.198928in}{1.488404in}}%
\pgfpathlineto{\pgfqpoint{1.199823in}{1.488964in}}%
\pgfpathlineto{\pgfqpoint{1.201895in}{1.489944in}}%
\pgfpathlineto{\pgfqpoint{1.202343in}{1.490224in}}%
\pgfpathlineto{\pgfqpoint{1.205343in}{1.491274in}}%
\pgfpathlineto{\pgfqpoint{1.206404in}{1.491554in}}%
\pgfpathlineto{\pgfqpoint{1.209670in}{1.492604in}}%
\pgfpathlineto{\pgfqpoint{1.210748in}{1.493024in}}%
\pgfpathlineto{\pgfqpoint{1.215887in}{1.494074in}}%
\pgfpathlineto{\pgfqpoint{1.216865in}{1.494214in}}%
\pgfpathlineto{\pgfqpoint{1.223860in}{1.495264in}}%
\pgfpathlineto{\pgfqpoint{1.224739in}{1.495684in}}%
\pgfpathlineto{\pgfqpoint{1.229994in}{1.496734in}}%
\pgfpathlineto{\pgfqpoint{1.230823in}{1.497014in}}%
\pgfpathlineto{\pgfqpoint{1.234354in}{1.498064in}}%
\pgfpathlineto{\pgfqpoint{1.235117in}{1.498344in}}%
\pgfpathlineto{\pgfqpoint{1.240256in}{1.499394in}}%
\pgfpathlineto{\pgfqpoint{1.241151in}{1.499674in}}%
\pgfpathlineto{\pgfqpoint{1.245610in}{1.500724in}}%
\pgfpathlineto{\pgfqpoint{1.246721in}{1.501144in}}%
\pgfpathlineto{\pgfqpoint{1.248942in}{1.502194in}}%
\pgfpathlineto{\pgfqpoint{1.250053in}{1.502754in}}%
\pgfpathlineto{\pgfqpoint{1.254429in}{1.503804in}}%
\pgfpathlineto{\pgfqpoint{1.254728in}{1.504084in}}%
\pgfpathlineto{\pgfqpoint{1.260845in}{1.505134in}}%
\pgfpathlineto{\pgfqpoint{1.261723in}{1.505414in}}%
\pgfpathlineto{\pgfqpoint{1.266697in}{1.506464in}}%
\pgfpathlineto{\pgfqpoint{1.267807in}{1.506744in}}%
\pgfpathlineto{\pgfqpoint{1.271338in}{1.507724in}}%
\pgfpathlineto{\pgfqpoint{1.272200in}{1.508214in}}%
\pgfpathlineto{\pgfqpoint{1.277936in}{1.509264in}}%
\pgfpathlineto{\pgfqpoint{1.278036in}{1.509404in}}%
\pgfpathlineto{\pgfqpoint{1.284203in}{1.510454in}}%
\pgfpathlineto{\pgfqpoint{1.285131in}{1.510804in}}%
\pgfpathlineto{\pgfqpoint{1.289657in}{1.511854in}}%
\pgfpathlineto{\pgfqpoint{1.290585in}{1.512274in}}%
\pgfpathlineto{\pgfqpoint{1.295940in}{1.513324in}}%
\pgfpathlineto{\pgfqpoint{1.296354in}{1.513534in}}%
\pgfpathlineto{\pgfqpoint{1.299703in}{1.514584in}}%
\pgfpathlineto{\pgfqpoint{1.300664in}{1.514934in}}%
\pgfpathlineto{\pgfqpoint{1.305389in}{1.515984in}}%
\pgfpathlineto{\pgfqpoint{1.306118in}{1.516334in}}%
\pgfpathlineto{\pgfqpoint{1.311290in}{1.517384in}}%
\pgfpathlineto{\pgfqpoint{1.312318in}{1.517734in}}%
\pgfpathlineto{\pgfqpoint{1.318104in}{1.518784in}}%
\pgfpathlineto{\pgfqpoint{1.318899in}{1.519274in}}%
\pgfpathlineto{\pgfqpoint{1.323873in}{1.520324in}}%
\pgfpathlineto{\pgfqpoint{1.324950in}{1.520674in}}%
\pgfpathlineto{\pgfqpoint{1.329575in}{1.521654in}}%
\pgfpathlineto{\pgfqpoint{1.330305in}{1.521864in}}%
\pgfpathlineto{\pgfqpoint{1.336521in}{1.522844in}}%
\pgfpathlineto{\pgfqpoint{1.336836in}{1.523194in}}%
\pgfpathlineto{\pgfqpoint{1.344180in}{1.524244in}}%
\pgfpathlineto{\pgfqpoint{1.345191in}{1.524524in}}%
\pgfpathlineto{\pgfqpoint{1.351905in}{1.525574in}}%
\pgfpathlineto{\pgfqpoint{1.352602in}{1.525994in}}%
\pgfpathlineto{\pgfqpoint{1.359050in}{1.527044in}}%
\pgfpathlineto{\pgfqpoint{1.359962in}{1.527394in}}%
\pgfpathlineto{\pgfqpoint{1.365897in}{1.528444in}}%
\pgfpathlineto{\pgfqpoint{1.366991in}{1.528654in}}%
\pgfpathlineto{\pgfqpoint{1.378463in}{1.529704in}}%
\pgfpathlineto{\pgfqpoint{1.379092in}{1.530054in}}%
\pgfpathlineto{\pgfqpoint{1.386221in}{1.531104in}}%
\pgfpathlineto{\pgfqpoint{1.387298in}{1.531314in}}%
\pgfpathlineto{\pgfqpoint{1.393366in}{1.532364in}}%
\pgfpathlineto{\pgfqpoint{1.393880in}{1.532714in}}%
\pgfpathlineto{\pgfqpoint{1.403760in}{1.533764in}}%
\pgfpathlineto{\pgfqpoint{1.404340in}{1.533904in}}%
\pgfpathlineto{\pgfqpoint{1.412529in}{1.534954in}}%
\pgfpathlineto{\pgfqpoint{1.412994in}{1.535164in}}%
\pgfpathlineto{\pgfqpoint{1.422277in}{1.536214in}}%
\pgfpathlineto{\pgfqpoint{1.422807in}{1.536354in}}%
\pgfpathlineto{\pgfqpoint{1.428460in}{1.537404in}}%
\pgfpathlineto{\pgfqpoint{1.429157in}{1.537614in}}%
\pgfpathlineto{\pgfqpoint{1.435340in}{1.538664in}}%
\pgfpathlineto{\pgfqpoint{1.436351in}{1.539014in}}%
\pgfpathlineto{\pgfqpoint{1.443861in}{1.540064in}}%
\pgfpathlineto{\pgfqpoint{1.443877in}{1.540274in}}%
\pgfpathlineto{\pgfqpoint{1.450940in}{1.541324in}}%
\pgfpathlineto{\pgfqpoint{1.451006in}{1.541464in}}%
\pgfpathlineto{\pgfqpoint{1.462776in}{1.542514in}}%
\pgfpathlineto{\pgfqpoint{1.463439in}{1.542724in}}%
\pgfpathlineto{\pgfqpoint{1.473817in}{1.543774in}}%
\pgfpathlineto{\pgfqpoint{1.474347in}{1.544124in}}%
\pgfpathlineto{\pgfqpoint{1.482304in}{1.545174in}}%
\pgfpathlineto{\pgfqpoint{1.482669in}{1.545314in}}%
\pgfpathlineto{\pgfqpoint{1.493892in}{1.546364in}}%
\pgfpathlineto{\pgfqpoint{1.494257in}{1.546504in}}%
\pgfpathlineto{\pgfqpoint{1.503756in}{1.547554in}}%
\pgfpathlineto{\pgfqpoint{1.503921in}{1.547834in}}%
\pgfpathlineto{\pgfqpoint{1.514382in}{1.548884in}}%
\pgfpathlineto{\pgfqpoint{1.515227in}{1.549234in}}%
\pgfpathlineto{\pgfqpoint{1.524046in}{1.550284in}}%
\pgfpathlineto{\pgfqpoint{1.524378in}{1.550564in}}%
\pgfpathlineto{\pgfqpoint{1.535187in}{1.551614in}}%
\pgfpathlineto{\pgfqpoint{1.535817in}{1.551754in}}%
\pgfpathlineto{\pgfqpoint{1.546012in}{1.552804in}}%
\pgfpathlineto{\pgfqpoint{1.547122in}{1.553154in}}%
\pgfpathlineto{\pgfqpoint{1.557450in}{1.554204in}}%
\pgfpathlineto{\pgfqpoint{1.557450in}{1.554274in}}%
\pgfpathlineto{\pgfqpoint{1.571690in}{1.555324in}}%
\pgfpathlineto{\pgfqpoint{1.571790in}{1.555534in}}%
\pgfpathlineto{\pgfqpoint{1.581405in}{1.556584in}}%
\pgfpathlineto{\pgfqpoint{1.581653in}{1.556794in}}%
\pgfpathlineto{\pgfqpoint{1.595396in}{1.557844in}}%
\pgfpathlineto{\pgfqpoint{1.595844in}{1.558054in}}%
\pgfpathlineto{\pgfqpoint{1.606503in}{1.559104in}}%
\pgfpathlineto{\pgfqpoint{1.607514in}{1.559384in}}%
\pgfpathlineto{\pgfqpoint{1.617610in}{1.560434in}}%
\pgfpathlineto{\pgfqpoint{1.618721in}{1.560644in}}%
\pgfpathlineto{\pgfqpoint{1.630889in}{1.561694in}}%
\pgfpathlineto{\pgfqpoint{1.631883in}{1.561834in}}%
\pgfpathlineto{\pgfqpoint{1.652605in}{1.562884in}}%
\pgfpathlineto{\pgfqpoint{1.653600in}{1.563024in}}%
\pgfpathlineto{\pgfqpoint{1.663812in}{1.564074in}}%
\pgfpathlineto{\pgfqpoint{1.663861in}{1.564214in}}%
\pgfpathlineto{\pgfqpoint{1.672283in}{1.565264in}}%
\pgfpathlineto{\pgfqpoint{1.672382in}{1.565404in}}%
\pgfpathlineto{\pgfqpoint{1.683406in}{1.566454in}}%
\pgfpathlineto{\pgfqpoint{1.684467in}{1.566664in}}%
\pgfpathlineto{\pgfqpoint{1.696586in}{1.567714in}}%
\pgfpathlineto{\pgfqpoint{1.696834in}{1.567854in}}%
\pgfpathlineto{\pgfqpoint{1.710212in}{1.568904in}}%
\pgfpathlineto{\pgfqpoint{1.710909in}{1.569114in}}%
\pgfpathlineto{\pgfqpoint{1.723706in}{1.570164in}}%
\pgfpathlineto{\pgfqpoint{1.724718in}{1.570304in}}%
\pgfpathlineto{\pgfqpoint{1.739090in}{1.571354in}}%
\pgfpathlineto{\pgfqpoint{1.739422in}{1.571494in}}%
\pgfpathlineto{\pgfqpoint{1.756828in}{1.572544in}}%
\pgfpathlineto{\pgfqpoint{1.757591in}{1.572754in}}%
\pgfpathlineto{\pgfqpoint{1.769825in}{1.573804in}}%
\pgfpathlineto{\pgfqpoint{1.770339in}{1.573944in}}%
\pgfpathlineto{\pgfqpoint{1.782242in}{1.574994in}}%
\pgfpathlineto{\pgfqpoint{1.782374in}{1.575134in}}%
\pgfpathlineto{\pgfqpoint{1.795288in}{1.576184in}}%
\pgfpathlineto{\pgfqpoint{1.795868in}{1.576394in}}%
\pgfpathlineto{\pgfqpoint{1.809910in}{1.577444in}}%
\pgfpathlineto{\pgfqpoint{1.810987in}{1.577724in}}%
\pgfpathlineto{\pgfqpoint{1.828360in}{1.578774in}}%
\pgfpathlineto{\pgfqpoint{1.829057in}{1.578984in}}%
\pgfpathlineto{\pgfqpoint{1.842236in}{1.580034in}}%
\pgfpathlineto{\pgfqpoint{1.842368in}{1.580174in}}%
\pgfpathlineto{\pgfqpoint{1.859410in}{1.581224in}}%
\pgfpathlineto{\pgfqpoint{1.859891in}{1.581364in}}%
\pgfpathlineto{\pgfqpoint{1.873683in}{1.582414in}}%
\pgfpathlineto{\pgfqpoint{1.873683in}{1.582484in}}%
\pgfpathlineto{\pgfqpoint{1.894190in}{1.583534in}}%
\pgfpathlineto{\pgfqpoint{1.894637in}{1.583674in}}%
\pgfpathlineto{\pgfqpoint{1.915542in}{1.584724in}}%
\pgfpathlineto{\pgfqpoint{1.916155in}{1.584864in}}%
\pgfpathlineto{\pgfqpoint{1.929334in}{1.585914in}}%
\pgfpathlineto{\pgfqpoint{1.930345in}{1.586054in}}%
\pgfpathlineto{\pgfqpoint{1.945680in}{1.587104in}}%
\pgfpathlineto{\pgfqpoint{1.945680in}{1.587174in}}%
\pgfpathlineto{\pgfqpoint{1.958245in}{1.588224in}}%
\pgfpathlineto{\pgfqpoint{1.959074in}{1.588434in}}%
\pgfpathlineto{\pgfqpoint{1.971541in}{1.589484in}}%
\pgfpathlineto{\pgfqpoint{1.972005in}{1.589624in}}%
\pgfpathlineto{\pgfqpoint{1.980807in}{1.590674in}}%
\pgfpathlineto{\pgfqpoint{1.981636in}{1.590954in}}%
\pgfpathlineto{\pgfqpoint{1.988947in}{1.592004in}}%
\pgfpathlineto{\pgfqpoint{1.988947in}{1.592074in}}%
\pgfpathlineto{\pgfqpoint{1.998313in}{1.593124in}}%
\pgfpathlineto{\pgfqpoint{1.999391in}{1.593334in}}%
\pgfpathlineto{\pgfqpoint{2.006536in}{1.594384in}}%
\pgfpathlineto{\pgfqpoint{2.007414in}{1.594524in}}%
\pgfpathlineto{\pgfqpoint{2.013349in}{1.595574in}}%
\pgfpathlineto{\pgfqpoint{2.014195in}{1.595854in}}%
\pgfpathlineto{\pgfqpoint{2.021936in}{1.596904in}}%
\pgfpathlineto{\pgfqpoint{2.022318in}{1.597044in}}%
\pgfpathlineto{\pgfqpoint{2.028136in}{1.598094in}}%
\pgfpathlineto{\pgfqpoint{2.029230in}{1.598584in}}%
\pgfpathlineto{\pgfqpoint{2.032015in}{1.599634in}}%
\pgfpathlineto{\pgfqpoint{2.033126in}{1.601944in}}%
\pgfpathlineto{\pgfqpoint{2.033126in}{1.601944in}}%
\pgfusepath{stroke}%
\end{pgfscope}%
\begin{pgfscope}%
\pgfsetrectcap%
\pgfsetmiterjoin%
\pgfsetlinewidth{0.803000pt}%
\definecolor{currentstroke}{rgb}{0.000000,0.000000,0.000000}%
\pgfsetstrokecolor{currentstroke}%
\pgfsetdash{}{0pt}%
\pgfpathmoveto{\pgfqpoint{0.553581in}{0.499444in}}%
\pgfpathlineto{\pgfqpoint{0.553581in}{1.654444in}}%
\pgfusepath{stroke}%
\end{pgfscope}%
\begin{pgfscope}%
\pgfsetrectcap%
\pgfsetmiterjoin%
\pgfsetlinewidth{0.803000pt}%
\definecolor{currentstroke}{rgb}{0.000000,0.000000,0.000000}%
\pgfsetstrokecolor{currentstroke}%
\pgfsetdash{}{0pt}%
\pgfpathmoveto{\pgfqpoint{2.103581in}{0.499444in}}%
\pgfpathlineto{\pgfqpoint{2.103581in}{1.654444in}}%
\pgfusepath{stroke}%
\end{pgfscope}%
\begin{pgfscope}%
\pgfsetrectcap%
\pgfsetmiterjoin%
\pgfsetlinewidth{0.803000pt}%
\definecolor{currentstroke}{rgb}{0.000000,0.000000,0.000000}%
\pgfsetstrokecolor{currentstroke}%
\pgfsetdash{}{0pt}%
\pgfpathmoveto{\pgfqpoint{0.553581in}{0.499444in}}%
\pgfpathlineto{\pgfqpoint{2.103581in}{0.499444in}}%
\pgfusepath{stroke}%
\end{pgfscope}%
\begin{pgfscope}%
\pgfsetrectcap%
\pgfsetmiterjoin%
\pgfsetlinewidth{0.803000pt}%
\definecolor{currentstroke}{rgb}{0.000000,0.000000,0.000000}%
\pgfsetstrokecolor{currentstroke}%
\pgfsetdash{}{0pt}%
\pgfpathmoveto{\pgfqpoint{0.553581in}{1.654444in}}%
\pgfpathlineto{\pgfqpoint{2.103581in}{1.654444in}}%
\pgfusepath{stroke}%
\end{pgfscope}%
\begin{pgfscope}%
\pgfsetbuttcap%
\pgfsetmiterjoin%
\definecolor{currentfill}{rgb}{1.000000,1.000000,1.000000}%
\pgfsetfillcolor{currentfill}%
\pgfsetfillopacity{0.800000}%
\pgfsetlinewidth{1.003750pt}%
\definecolor{currentstroke}{rgb}{0.800000,0.800000,0.800000}%
\pgfsetstrokecolor{currentstroke}%
\pgfsetstrokeopacity{0.800000}%
\pgfsetdash{}{0pt}%
\pgfpathmoveto{\pgfqpoint{0.832747in}{0.568889in}}%
\pgfpathlineto{\pgfqpoint{2.006358in}{0.568889in}}%
\pgfpathquadraticcurveto{\pgfqpoint{2.034136in}{0.568889in}}{\pgfqpoint{2.034136in}{0.596666in}}%
\pgfpathlineto{\pgfqpoint{2.034136in}{0.776388in}}%
\pgfpathquadraticcurveto{\pgfqpoint{2.034136in}{0.804166in}}{\pgfqpoint{2.006358in}{0.804166in}}%
\pgfpathlineto{\pgfqpoint{0.832747in}{0.804166in}}%
\pgfpathquadraticcurveto{\pgfqpoint{0.804970in}{0.804166in}}{\pgfqpoint{0.804970in}{0.776388in}}%
\pgfpathlineto{\pgfqpoint{0.804970in}{0.596666in}}%
\pgfpathquadraticcurveto{\pgfqpoint{0.804970in}{0.568889in}}{\pgfqpoint{0.832747in}{0.568889in}}%
\pgfpathlineto{\pgfqpoint{0.832747in}{0.568889in}}%
\pgfpathclose%
\pgfusepath{stroke,fill}%
\end{pgfscope}%
\begin{pgfscope}%
\pgfsetrectcap%
\pgfsetroundjoin%
\pgfsetlinewidth{1.505625pt}%
\definecolor{currentstroke}{rgb}{0.000000,0.000000,0.000000}%
\pgfsetstrokecolor{currentstroke}%
\pgfsetdash{}{0pt}%
\pgfpathmoveto{\pgfqpoint{0.860525in}{0.700000in}}%
\pgfpathlineto{\pgfqpoint{0.999414in}{0.700000in}}%
\pgfpathlineto{\pgfqpoint{1.138303in}{0.700000in}}%
\pgfusepath{stroke}%
\end{pgfscope}%
\begin{pgfscope}%
\definecolor{textcolor}{rgb}{0.000000,0.000000,0.000000}%
\pgfsetstrokecolor{textcolor}%
\pgfsetfillcolor{textcolor}%
\pgftext[x=1.249414in,y=0.651388in,left,base]{\color{textcolor}\rmfamily\fontsize{10.000000}{12.000000}\selectfont AUC=0.840}%
\end{pgfscope}%
\end{pgfpicture}%
\makeatother%
\endgroup%

  &
\vspace{0pt} 
  
\begin{tabular}{cc|c|c|}
	&\multicolumn{1}{c}{}& \multicolumn{2}{c}{Prediction} \cr
	&\multicolumn{1}{c}{} & \multicolumn{1}{c}{N} & \multicolumn{1}{c}{P} \cr\cline{3-4}
	\multirow{2}{*}{\rotatebox[origin=c]{90}{Actual}}&N & 150,771    &  0 \vrule width 0pt height 10pt depth 2pt \cr\cline{3-4}
	&P & 22,621 & 0 \vrule width 0pt height 10pt depth 2pt \cr\cline{3-4}
\end{tabular}

\begin{center}
\begin{tabular}{ll}
und & Precision \cr 
0.000 & Recall \cr 
und & F1 \cr 
\end{tabular}
\end{center}
  
\end{tabular}

Similarly, if the class weight is too high, we may get a model like Example 4 that sends an ambulance to every reported crash.  Example 4 is Example 1 transformed with $f(x) = 0.5x + 0.5$

\noindent\begin{tabular}{@{}p{0.3\textwidth}@{\hspace{24pt}} p{0.3\textwidth} @{\hspace{24pt}} p{0.3\textwidth}}
  \vspace{0pt} %% Creator: Matplotlib, PGF backend
%%
%% To include the figure in your LaTeX document, write
%%   \input{<filename>.pgf}
%%
%% Make sure the required packages are loaded in your preamble
%%   \usepackage{pgf}
%%
%% Also ensure that all the required font packages are loaded; for instance,
%% the lmodern package is sometimes necessary when using math font.
%%   \usepackage{lmodern}
%%
%% Figures using additional raster images can only be included by \input if
%% they are in the same directory as the main LaTeX file. For loading figures
%% from other directories you can use the `import` package
%%   \usepackage{import}
%%
%% and then include the figures with
%%   \import{<path to file>}{<filename>.pgf}
%%
%% Matplotlib used the following preamble
%%   
%%   \usepackage{fontspec}
%%   \makeatletter\@ifpackageloaded{underscore}{}{\usepackage[strings]{underscore}}\makeatother
%%
\begingroup%
\makeatletter%
\begin{pgfpicture}%
\pgfpathrectangle{\pgfpointorigin}{\pgfqpoint{2.253750in}{1.754444in}}%
\pgfusepath{use as bounding box, clip}%
\begin{pgfscope}%
\pgfsetbuttcap%
\pgfsetmiterjoin%
\definecolor{currentfill}{rgb}{1.000000,1.000000,1.000000}%
\pgfsetfillcolor{currentfill}%
\pgfsetlinewidth{0.000000pt}%
\definecolor{currentstroke}{rgb}{1.000000,1.000000,1.000000}%
\pgfsetstrokecolor{currentstroke}%
\pgfsetdash{}{0pt}%
\pgfpathmoveto{\pgfqpoint{0.000000in}{0.000000in}}%
\pgfpathlineto{\pgfqpoint{2.253750in}{0.000000in}}%
\pgfpathlineto{\pgfqpoint{2.253750in}{1.754444in}}%
\pgfpathlineto{\pgfqpoint{0.000000in}{1.754444in}}%
\pgfpathlineto{\pgfqpoint{0.000000in}{0.000000in}}%
\pgfpathclose%
\pgfusepath{fill}%
\end{pgfscope}%
\begin{pgfscope}%
\pgfsetbuttcap%
\pgfsetmiterjoin%
\definecolor{currentfill}{rgb}{1.000000,1.000000,1.000000}%
\pgfsetfillcolor{currentfill}%
\pgfsetlinewidth{0.000000pt}%
\definecolor{currentstroke}{rgb}{0.000000,0.000000,0.000000}%
\pgfsetstrokecolor{currentstroke}%
\pgfsetstrokeopacity{0.000000}%
\pgfsetdash{}{0pt}%
\pgfpathmoveto{\pgfqpoint{0.515000in}{0.499444in}}%
\pgfpathlineto{\pgfqpoint{2.065000in}{0.499444in}}%
\pgfpathlineto{\pgfqpoint{2.065000in}{1.654444in}}%
\pgfpathlineto{\pgfqpoint{0.515000in}{1.654444in}}%
\pgfpathlineto{\pgfqpoint{0.515000in}{0.499444in}}%
\pgfpathclose%
\pgfusepath{fill}%
\end{pgfscope}%
\begin{pgfscope}%
\pgfpathrectangle{\pgfqpoint{0.515000in}{0.499444in}}{\pgfqpoint{1.550000in}{1.155000in}}%
\pgfusepath{clip}%
\pgfsetbuttcap%
\pgfsetmiterjoin%
\pgfsetlinewidth{1.003750pt}%
\definecolor{currentstroke}{rgb}{0.000000,0.000000,0.000000}%
\pgfsetstrokecolor{currentstroke}%
\pgfsetdash{}{0pt}%
\pgfpathmoveto{\pgfqpoint{0.505000in}{0.499444in}}%
\pgfpathlineto{\pgfqpoint{0.552805in}{0.499444in}}%
\pgfpathlineto{\pgfqpoint{0.552805in}{0.499444in}}%
\pgfpathlineto{\pgfqpoint{0.505000in}{0.499444in}}%
\pgfusepath{stroke}%
\end{pgfscope}%
\begin{pgfscope}%
\pgfpathrectangle{\pgfqpoint{0.515000in}{0.499444in}}{\pgfqpoint{1.550000in}{1.155000in}}%
\pgfusepath{clip}%
\pgfsetbuttcap%
\pgfsetmiterjoin%
\pgfsetlinewidth{1.003750pt}%
\definecolor{currentstroke}{rgb}{0.000000,0.000000,0.000000}%
\pgfsetstrokecolor{currentstroke}%
\pgfsetdash{}{0pt}%
\pgfpathmoveto{\pgfqpoint{0.643537in}{0.499444in}}%
\pgfpathlineto{\pgfqpoint{0.704025in}{0.499444in}}%
\pgfpathlineto{\pgfqpoint{0.704025in}{0.499444in}}%
\pgfpathlineto{\pgfqpoint{0.643537in}{0.499444in}}%
\pgfpathlineto{\pgfqpoint{0.643537in}{0.499444in}}%
\pgfpathclose%
\pgfusepath{stroke}%
\end{pgfscope}%
\begin{pgfscope}%
\pgfpathrectangle{\pgfqpoint{0.515000in}{0.499444in}}{\pgfqpoint{1.550000in}{1.155000in}}%
\pgfusepath{clip}%
\pgfsetbuttcap%
\pgfsetmiterjoin%
\pgfsetlinewidth{1.003750pt}%
\definecolor{currentstroke}{rgb}{0.000000,0.000000,0.000000}%
\pgfsetstrokecolor{currentstroke}%
\pgfsetdash{}{0pt}%
\pgfpathmoveto{\pgfqpoint{0.794756in}{0.499444in}}%
\pgfpathlineto{\pgfqpoint{0.855244in}{0.499444in}}%
\pgfpathlineto{\pgfqpoint{0.855244in}{0.499444in}}%
\pgfpathlineto{\pgfqpoint{0.794756in}{0.499444in}}%
\pgfpathlineto{\pgfqpoint{0.794756in}{0.499444in}}%
\pgfpathclose%
\pgfusepath{stroke}%
\end{pgfscope}%
\begin{pgfscope}%
\pgfpathrectangle{\pgfqpoint{0.515000in}{0.499444in}}{\pgfqpoint{1.550000in}{1.155000in}}%
\pgfusepath{clip}%
\pgfsetbuttcap%
\pgfsetmiterjoin%
\pgfsetlinewidth{1.003750pt}%
\definecolor{currentstroke}{rgb}{0.000000,0.000000,0.000000}%
\pgfsetstrokecolor{currentstroke}%
\pgfsetdash{}{0pt}%
\pgfpathmoveto{\pgfqpoint{0.945976in}{0.499444in}}%
\pgfpathlineto{\pgfqpoint{1.006464in}{0.499444in}}%
\pgfpathlineto{\pgfqpoint{1.006464in}{0.499444in}}%
\pgfpathlineto{\pgfqpoint{0.945976in}{0.499444in}}%
\pgfpathlineto{\pgfqpoint{0.945976in}{0.499444in}}%
\pgfpathclose%
\pgfusepath{stroke}%
\end{pgfscope}%
\begin{pgfscope}%
\pgfpathrectangle{\pgfqpoint{0.515000in}{0.499444in}}{\pgfqpoint{1.550000in}{1.155000in}}%
\pgfusepath{clip}%
\pgfsetbuttcap%
\pgfsetmiterjoin%
\pgfsetlinewidth{1.003750pt}%
\definecolor{currentstroke}{rgb}{0.000000,0.000000,0.000000}%
\pgfsetstrokecolor{currentstroke}%
\pgfsetdash{}{0pt}%
\pgfpathmoveto{\pgfqpoint{1.097195in}{0.499444in}}%
\pgfpathlineto{\pgfqpoint{1.157683in}{0.499444in}}%
\pgfpathlineto{\pgfqpoint{1.157683in}{0.499444in}}%
\pgfpathlineto{\pgfqpoint{1.097195in}{0.499444in}}%
\pgfpathlineto{\pgfqpoint{1.097195in}{0.499444in}}%
\pgfpathclose%
\pgfusepath{stroke}%
\end{pgfscope}%
\begin{pgfscope}%
\pgfpathrectangle{\pgfqpoint{0.515000in}{0.499444in}}{\pgfqpoint{1.550000in}{1.155000in}}%
\pgfusepath{clip}%
\pgfsetbuttcap%
\pgfsetmiterjoin%
\pgfsetlinewidth{1.003750pt}%
\definecolor{currentstroke}{rgb}{0.000000,0.000000,0.000000}%
\pgfsetstrokecolor{currentstroke}%
\pgfsetdash{}{0pt}%
\pgfpathmoveto{\pgfqpoint{1.248415in}{0.499444in}}%
\pgfpathlineto{\pgfqpoint{1.308903in}{0.499444in}}%
\pgfpathlineto{\pgfqpoint{1.308903in}{0.961883in}}%
\pgfpathlineto{\pgfqpoint{1.248415in}{0.961883in}}%
\pgfpathlineto{\pgfqpoint{1.248415in}{0.499444in}}%
\pgfpathclose%
\pgfusepath{stroke}%
\end{pgfscope}%
\begin{pgfscope}%
\pgfpathrectangle{\pgfqpoint{0.515000in}{0.499444in}}{\pgfqpoint{1.550000in}{1.155000in}}%
\pgfusepath{clip}%
\pgfsetbuttcap%
\pgfsetmiterjoin%
\pgfsetlinewidth{1.003750pt}%
\definecolor{currentstroke}{rgb}{0.000000,0.000000,0.000000}%
\pgfsetstrokecolor{currentstroke}%
\pgfsetdash{}{0pt}%
\pgfpathmoveto{\pgfqpoint{1.399634in}{0.499444in}}%
\pgfpathlineto{\pgfqpoint{1.460122in}{0.499444in}}%
\pgfpathlineto{\pgfqpoint{1.460122in}{1.599444in}}%
\pgfpathlineto{\pgfqpoint{1.399634in}{1.599444in}}%
\pgfpathlineto{\pgfqpoint{1.399634in}{0.499444in}}%
\pgfpathclose%
\pgfusepath{stroke}%
\end{pgfscope}%
\begin{pgfscope}%
\pgfpathrectangle{\pgfqpoint{0.515000in}{0.499444in}}{\pgfqpoint{1.550000in}{1.155000in}}%
\pgfusepath{clip}%
\pgfsetbuttcap%
\pgfsetmiterjoin%
\pgfsetlinewidth{1.003750pt}%
\definecolor{currentstroke}{rgb}{0.000000,0.000000,0.000000}%
\pgfsetstrokecolor{currentstroke}%
\pgfsetdash{}{0pt}%
\pgfpathmoveto{\pgfqpoint{1.550854in}{0.499444in}}%
\pgfpathlineto{\pgfqpoint{1.611342in}{0.499444in}}%
\pgfpathlineto{\pgfqpoint{1.611342in}{1.132630in}}%
\pgfpathlineto{\pgfqpoint{1.550854in}{1.132630in}}%
\pgfpathlineto{\pgfqpoint{1.550854in}{0.499444in}}%
\pgfpathclose%
\pgfusepath{stroke}%
\end{pgfscope}%
\begin{pgfscope}%
\pgfpathrectangle{\pgfqpoint{0.515000in}{0.499444in}}{\pgfqpoint{1.550000in}{1.155000in}}%
\pgfusepath{clip}%
\pgfsetbuttcap%
\pgfsetmiterjoin%
\pgfsetlinewidth{1.003750pt}%
\definecolor{currentstroke}{rgb}{0.000000,0.000000,0.000000}%
\pgfsetstrokecolor{currentstroke}%
\pgfsetdash{}{0pt}%
\pgfpathmoveto{\pgfqpoint{1.702073in}{0.499444in}}%
\pgfpathlineto{\pgfqpoint{1.762561in}{0.499444in}}%
\pgfpathlineto{\pgfqpoint{1.762561in}{0.739458in}}%
\pgfpathlineto{\pgfqpoint{1.702073in}{0.739458in}}%
\pgfpathlineto{\pgfqpoint{1.702073in}{0.499444in}}%
\pgfpathclose%
\pgfusepath{stroke}%
\end{pgfscope}%
\begin{pgfscope}%
\pgfpathrectangle{\pgfqpoint{0.515000in}{0.499444in}}{\pgfqpoint{1.550000in}{1.155000in}}%
\pgfusepath{clip}%
\pgfsetbuttcap%
\pgfsetmiterjoin%
\pgfsetlinewidth{1.003750pt}%
\definecolor{currentstroke}{rgb}{0.000000,0.000000,0.000000}%
\pgfsetstrokecolor{currentstroke}%
\pgfsetdash{}{0pt}%
\pgfpathmoveto{\pgfqpoint{1.853293in}{0.499444in}}%
\pgfpathlineto{\pgfqpoint{1.913781in}{0.499444in}}%
\pgfpathlineto{\pgfqpoint{1.913781in}{0.559668in}}%
\pgfpathlineto{\pgfqpoint{1.853293in}{0.559668in}}%
\pgfpathlineto{\pgfqpoint{1.853293in}{0.499444in}}%
\pgfpathclose%
\pgfusepath{stroke}%
\end{pgfscope}%
\begin{pgfscope}%
\pgfpathrectangle{\pgfqpoint{0.515000in}{0.499444in}}{\pgfqpoint{1.550000in}{1.155000in}}%
\pgfusepath{clip}%
\pgfsetbuttcap%
\pgfsetmiterjoin%
\definecolor{currentfill}{rgb}{0.000000,0.000000,0.000000}%
\pgfsetfillcolor{currentfill}%
\pgfsetlinewidth{0.000000pt}%
\definecolor{currentstroke}{rgb}{0.000000,0.000000,0.000000}%
\pgfsetstrokecolor{currentstroke}%
\pgfsetstrokeopacity{0.000000}%
\pgfsetdash{}{0pt}%
\pgfpathmoveto{\pgfqpoint{0.552805in}{0.499444in}}%
\pgfpathlineto{\pgfqpoint{0.613293in}{0.499444in}}%
\pgfpathlineto{\pgfqpoint{0.613293in}{0.499444in}}%
\pgfpathlineto{\pgfqpoint{0.552805in}{0.499444in}}%
\pgfpathlineto{\pgfqpoint{0.552805in}{0.499444in}}%
\pgfpathclose%
\pgfusepath{fill}%
\end{pgfscope}%
\begin{pgfscope}%
\pgfpathrectangle{\pgfqpoint{0.515000in}{0.499444in}}{\pgfqpoint{1.550000in}{1.155000in}}%
\pgfusepath{clip}%
\pgfsetbuttcap%
\pgfsetmiterjoin%
\definecolor{currentfill}{rgb}{0.000000,0.000000,0.000000}%
\pgfsetfillcolor{currentfill}%
\pgfsetlinewidth{0.000000pt}%
\definecolor{currentstroke}{rgb}{0.000000,0.000000,0.000000}%
\pgfsetstrokecolor{currentstroke}%
\pgfsetstrokeopacity{0.000000}%
\pgfsetdash{}{0pt}%
\pgfpathmoveto{\pgfqpoint{0.704025in}{0.499444in}}%
\pgfpathlineto{\pgfqpoint{0.764512in}{0.499444in}}%
\pgfpathlineto{\pgfqpoint{0.764512in}{0.499444in}}%
\pgfpathlineto{\pgfqpoint{0.704025in}{0.499444in}}%
\pgfpathlineto{\pgfqpoint{0.704025in}{0.499444in}}%
\pgfpathclose%
\pgfusepath{fill}%
\end{pgfscope}%
\begin{pgfscope}%
\pgfpathrectangle{\pgfqpoint{0.515000in}{0.499444in}}{\pgfqpoint{1.550000in}{1.155000in}}%
\pgfusepath{clip}%
\pgfsetbuttcap%
\pgfsetmiterjoin%
\definecolor{currentfill}{rgb}{0.000000,0.000000,0.000000}%
\pgfsetfillcolor{currentfill}%
\pgfsetlinewidth{0.000000pt}%
\definecolor{currentstroke}{rgb}{0.000000,0.000000,0.000000}%
\pgfsetstrokecolor{currentstroke}%
\pgfsetstrokeopacity{0.000000}%
\pgfsetdash{}{0pt}%
\pgfpathmoveto{\pgfqpoint{0.855244in}{0.499444in}}%
\pgfpathlineto{\pgfqpoint{0.915732in}{0.499444in}}%
\pgfpathlineto{\pgfqpoint{0.915732in}{0.499444in}}%
\pgfpathlineto{\pgfqpoint{0.855244in}{0.499444in}}%
\pgfpathlineto{\pgfqpoint{0.855244in}{0.499444in}}%
\pgfpathclose%
\pgfusepath{fill}%
\end{pgfscope}%
\begin{pgfscope}%
\pgfpathrectangle{\pgfqpoint{0.515000in}{0.499444in}}{\pgfqpoint{1.550000in}{1.155000in}}%
\pgfusepath{clip}%
\pgfsetbuttcap%
\pgfsetmiterjoin%
\definecolor{currentfill}{rgb}{0.000000,0.000000,0.000000}%
\pgfsetfillcolor{currentfill}%
\pgfsetlinewidth{0.000000pt}%
\definecolor{currentstroke}{rgb}{0.000000,0.000000,0.000000}%
\pgfsetstrokecolor{currentstroke}%
\pgfsetstrokeopacity{0.000000}%
\pgfsetdash{}{0pt}%
\pgfpathmoveto{\pgfqpoint{1.006464in}{0.499444in}}%
\pgfpathlineto{\pgfqpoint{1.066951in}{0.499444in}}%
\pgfpathlineto{\pgfqpoint{1.066951in}{0.499444in}}%
\pgfpathlineto{\pgfqpoint{1.006464in}{0.499444in}}%
\pgfpathlineto{\pgfqpoint{1.006464in}{0.499444in}}%
\pgfpathclose%
\pgfusepath{fill}%
\end{pgfscope}%
\begin{pgfscope}%
\pgfpathrectangle{\pgfqpoint{0.515000in}{0.499444in}}{\pgfqpoint{1.550000in}{1.155000in}}%
\pgfusepath{clip}%
\pgfsetbuttcap%
\pgfsetmiterjoin%
\definecolor{currentfill}{rgb}{0.000000,0.000000,0.000000}%
\pgfsetfillcolor{currentfill}%
\pgfsetlinewidth{0.000000pt}%
\definecolor{currentstroke}{rgb}{0.000000,0.000000,0.000000}%
\pgfsetstrokecolor{currentstroke}%
\pgfsetstrokeopacity{0.000000}%
\pgfsetdash{}{0pt}%
\pgfpathmoveto{\pgfqpoint{1.157683in}{0.499444in}}%
\pgfpathlineto{\pgfqpoint{1.218171in}{0.499444in}}%
\pgfpathlineto{\pgfqpoint{1.218171in}{0.499444in}}%
\pgfpathlineto{\pgfqpoint{1.157683in}{0.499444in}}%
\pgfpathlineto{\pgfqpoint{1.157683in}{0.499444in}}%
\pgfpathclose%
\pgfusepath{fill}%
\end{pgfscope}%
\begin{pgfscope}%
\pgfpathrectangle{\pgfqpoint{0.515000in}{0.499444in}}{\pgfqpoint{1.550000in}{1.155000in}}%
\pgfusepath{clip}%
\pgfsetbuttcap%
\pgfsetmiterjoin%
\definecolor{currentfill}{rgb}{0.000000,0.000000,0.000000}%
\pgfsetfillcolor{currentfill}%
\pgfsetlinewidth{0.000000pt}%
\definecolor{currentstroke}{rgb}{0.000000,0.000000,0.000000}%
\pgfsetstrokecolor{currentstroke}%
\pgfsetstrokeopacity{0.000000}%
\pgfsetdash{}{0pt}%
\pgfpathmoveto{\pgfqpoint{1.308903in}{0.499444in}}%
\pgfpathlineto{\pgfqpoint{1.369391in}{0.499444in}}%
\pgfpathlineto{\pgfqpoint{1.369391in}{0.511101in}}%
\pgfpathlineto{\pgfqpoint{1.308903in}{0.511101in}}%
\pgfpathlineto{\pgfqpoint{1.308903in}{0.499444in}}%
\pgfpathclose%
\pgfusepath{fill}%
\end{pgfscope}%
\begin{pgfscope}%
\pgfpathrectangle{\pgfqpoint{0.515000in}{0.499444in}}{\pgfqpoint{1.550000in}{1.155000in}}%
\pgfusepath{clip}%
\pgfsetbuttcap%
\pgfsetmiterjoin%
\definecolor{currentfill}{rgb}{0.000000,0.000000,0.000000}%
\pgfsetfillcolor{currentfill}%
\pgfsetlinewidth{0.000000pt}%
\definecolor{currentstroke}{rgb}{0.000000,0.000000,0.000000}%
\pgfsetstrokecolor{currentstroke}%
\pgfsetstrokeopacity{0.000000}%
\pgfsetdash{}{0pt}%
\pgfpathmoveto{\pgfqpoint{1.460122in}{0.499444in}}%
\pgfpathlineto{\pgfqpoint{1.520610in}{0.499444in}}%
\pgfpathlineto{\pgfqpoint{1.520610in}{0.540846in}}%
\pgfpathlineto{\pgfqpoint{1.460122in}{0.540846in}}%
\pgfpathlineto{\pgfqpoint{1.460122in}{0.499444in}}%
\pgfpathclose%
\pgfusepath{fill}%
\end{pgfscope}%
\begin{pgfscope}%
\pgfpathrectangle{\pgfqpoint{0.515000in}{0.499444in}}{\pgfqpoint{1.550000in}{1.155000in}}%
\pgfusepath{clip}%
\pgfsetbuttcap%
\pgfsetmiterjoin%
\definecolor{currentfill}{rgb}{0.000000,0.000000,0.000000}%
\pgfsetfillcolor{currentfill}%
\pgfsetlinewidth{0.000000pt}%
\definecolor{currentstroke}{rgb}{0.000000,0.000000,0.000000}%
\pgfsetstrokecolor{currentstroke}%
\pgfsetstrokeopacity{0.000000}%
\pgfsetdash{}{0pt}%
\pgfpathmoveto{\pgfqpoint{1.611342in}{0.499444in}}%
\pgfpathlineto{\pgfqpoint{1.671830in}{0.499444in}}%
\pgfpathlineto{\pgfqpoint{1.671830in}{0.615047in}}%
\pgfpathlineto{\pgfqpoint{1.611342in}{0.615047in}}%
\pgfpathlineto{\pgfqpoint{1.611342in}{0.499444in}}%
\pgfpathclose%
\pgfusepath{fill}%
\end{pgfscope}%
\begin{pgfscope}%
\pgfpathrectangle{\pgfqpoint{0.515000in}{0.499444in}}{\pgfqpoint{1.550000in}{1.155000in}}%
\pgfusepath{clip}%
\pgfsetbuttcap%
\pgfsetmiterjoin%
\definecolor{currentfill}{rgb}{0.000000,0.000000,0.000000}%
\pgfsetfillcolor{currentfill}%
\pgfsetlinewidth{0.000000pt}%
\definecolor{currentstroke}{rgb}{0.000000,0.000000,0.000000}%
\pgfsetstrokecolor{currentstroke}%
\pgfsetstrokeopacity{0.000000}%
\pgfsetdash{}{0pt}%
\pgfpathmoveto{\pgfqpoint{1.762561in}{0.499444in}}%
\pgfpathlineto{\pgfqpoint{1.823049in}{0.499444in}}%
\pgfpathlineto{\pgfqpoint{1.823049in}{0.693211in}}%
\pgfpathlineto{\pgfqpoint{1.762561in}{0.693211in}}%
\pgfpathlineto{\pgfqpoint{1.762561in}{0.499444in}}%
\pgfpathclose%
\pgfusepath{fill}%
\end{pgfscope}%
\begin{pgfscope}%
\pgfpathrectangle{\pgfqpoint{0.515000in}{0.499444in}}{\pgfqpoint{1.550000in}{1.155000in}}%
\pgfusepath{clip}%
\pgfsetbuttcap%
\pgfsetmiterjoin%
\definecolor{currentfill}{rgb}{0.000000,0.000000,0.000000}%
\pgfsetfillcolor{currentfill}%
\pgfsetlinewidth{0.000000pt}%
\definecolor{currentstroke}{rgb}{0.000000,0.000000,0.000000}%
\pgfsetstrokecolor{currentstroke}%
\pgfsetstrokeopacity{0.000000}%
\pgfsetdash{}{0pt}%
\pgfpathmoveto{\pgfqpoint{1.913781in}{0.499444in}}%
\pgfpathlineto{\pgfqpoint{1.974269in}{0.499444in}}%
\pgfpathlineto{\pgfqpoint{1.974269in}{0.577462in}}%
\pgfpathlineto{\pgfqpoint{1.913781in}{0.577462in}}%
\pgfpathlineto{\pgfqpoint{1.913781in}{0.499444in}}%
\pgfpathclose%
\pgfusepath{fill}%
\end{pgfscope}%
\begin{pgfscope}%
\pgfsetbuttcap%
\pgfsetroundjoin%
\definecolor{currentfill}{rgb}{0.000000,0.000000,0.000000}%
\pgfsetfillcolor{currentfill}%
\pgfsetlinewidth{0.803000pt}%
\definecolor{currentstroke}{rgb}{0.000000,0.000000,0.000000}%
\pgfsetstrokecolor{currentstroke}%
\pgfsetdash{}{0pt}%
\pgfsys@defobject{currentmarker}{\pgfqpoint{0.000000in}{-0.048611in}}{\pgfqpoint{0.000000in}{0.000000in}}{%
\pgfpathmoveto{\pgfqpoint{0.000000in}{0.000000in}}%
\pgfpathlineto{\pgfqpoint{0.000000in}{-0.048611in}}%
\pgfusepath{stroke,fill}%
}%
\begin{pgfscope}%
\pgfsys@transformshift{0.552805in}{0.499444in}%
\pgfsys@useobject{currentmarker}{}%
\end{pgfscope}%
\end{pgfscope}%
\begin{pgfscope}%
\definecolor{textcolor}{rgb}{0.000000,0.000000,0.000000}%
\pgfsetstrokecolor{textcolor}%
\pgfsetfillcolor{textcolor}%
\pgftext[x=0.552805in,y=0.402222in,,top]{\color{textcolor}\rmfamily\fontsize{10.000000}{12.000000}\selectfont 0.0}%
\end{pgfscope}%
\begin{pgfscope}%
\pgfsetbuttcap%
\pgfsetroundjoin%
\definecolor{currentfill}{rgb}{0.000000,0.000000,0.000000}%
\pgfsetfillcolor{currentfill}%
\pgfsetlinewidth{0.803000pt}%
\definecolor{currentstroke}{rgb}{0.000000,0.000000,0.000000}%
\pgfsetstrokecolor{currentstroke}%
\pgfsetdash{}{0pt}%
\pgfsys@defobject{currentmarker}{\pgfqpoint{0.000000in}{-0.048611in}}{\pgfqpoint{0.000000in}{0.000000in}}{%
\pgfpathmoveto{\pgfqpoint{0.000000in}{0.000000in}}%
\pgfpathlineto{\pgfqpoint{0.000000in}{-0.048611in}}%
\pgfusepath{stroke,fill}%
}%
\begin{pgfscope}%
\pgfsys@transformshift{0.930854in}{0.499444in}%
\pgfsys@useobject{currentmarker}{}%
\end{pgfscope}%
\end{pgfscope}%
\begin{pgfscope}%
\definecolor{textcolor}{rgb}{0.000000,0.000000,0.000000}%
\pgfsetstrokecolor{textcolor}%
\pgfsetfillcolor{textcolor}%
\pgftext[x=0.930854in,y=0.402222in,,top]{\color{textcolor}\rmfamily\fontsize{10.000000}{12.000000}\selectfont 0.25}%
\end{pgfscope}%
\begin{pgfscope}%
\pgfsetbuttcap%
\pgfsetroundjoin%
\definecolor{currentfill}{rgb}{0.000000,0.000000,0.000000}%
\pgfsetfillcolor{currentfill}%
\pgfsetlinewidth{0.803000pt}%
\definecolor{currentstroke}{rgb}{0.000000,0.000000,0.000000}%
\pgfsetstrokecolor{currentstroke}%
\pgfsetdash{}{0pt}%
\pgfsys@defobject{currentmarker}{\pgfqpoint{0.000000in}{-0.048611in}}{\pgfqpoint{0.000000in}{0.000000in}}{%
\pgfpathmoveto{\pgfqpoint{0.000000in}{0.000000in}}%
\pgfpathlineto{\pgfqpoint{0.000000in}{-0.048611in}}%
\pgfusepath{stroke,fill}%
}%
\begin{pgfscope}%
\pgfsys@transformshift{1.308903in}{0.499444in}%
\pgfsys@useobject{currentmarker}{}%
\end{pgfscope}%
\end{pgfscope}%
\begin{pgfscope}%
\definecolor{textcolor}{rgb}{0.000000,0.000000,0.000000}%
\pgfsetstrokecolor{textcolor}%
\pgfsetfillcolor{textcolor}%
\pgftext[x=1.308903in,y=0.402222in,,top]{\color{textcolor}\rmfamily\fontsize{10.000000}{12.000000}\selectfont 0.5}%
\end{pgfscope}%
\begin{pgfscope}%
\pgfsetbuttcap%
\pgfsetroundjoin%
\definecolor{currentfill}{rgb}{0.000000,0.000000,0.000000}%
\pgfsetfillcolor{currentfill}%
\pgfsetlinewidth{0.803000pt}%
\definecolor{currentstroke}{rgb}{0.000000,0.000000,0.000000}%
\pgfsetstrokecolor{currentstroke}%
\pgfsetdash{}{0pt}%
\pgfsys@defobject{currentmarker}{\pgfqpoint{0.000000in}{-0.048611in}}{\pgfqpoint{0.000000in}{0.000000in}}{%
\pgfpathmoveto{\pgfqpoint{0.000000in}{0.000000in}}%
\pgfpathlineto{\pgfqpoint{0.000000in}{-0.048611in}}%
\pgfusepath{stroke,fill}%
}%
\begin{pgfscope}%
\pgfsys@transformshift{1.686951in}{0.499444in}%
\pgfsys@useobject{currentmarker}{}%
\end{pgfscope}%
\end{pgfscope}%
\begin{pgfscope}%
\definecolor{textcolor}{rgb}{0.000000,0.000000,0.000000}%
\pgfsetstrokecolor{textcolor}%
\pgfsetfillcolor{textcolor}%
\pgftext[x=1.686951in,y=0.402222in,,top]{\color{textcolor}\rmfamily\fontsize{10.000000}{12.000000}\selectfont 0.75}%
\end{pgfscope}%
\begin{pgfscope}%
\pgfsetbuttcap%
\pgfsetroundjoin%
\definecolor{currentfill}{rgb}{0.000000,0.000000,0.000000}%
\pgfsetfillcolor{currentfill}%
\pgfsetlinewidth{0.803000pt}%
\definecolor{currentstroke}{rgb}{0.000000,0.000000,0.000000}%
\pgfsetstrokecolor{currentstroke}%
\pgfsetdash{}{0pt}%
\pgfsys@defobject{currentmarker}{\pgfqpoint{0.000000in}{-0.048611in}}{\pgfqpoint{0.000000in}{0.000000in}}{%
\pgfpathmoveto{\pgfqpoint{0.000000in}{0.000000in}}%
\pgfpathlineto{\pgfqpoint{0.000000in}{-0.048611in}}%
\pgfusepath{stroke,fill}%
}%
\begin{pgfscope}%
\pgfsys@transformshift{2.065000in}{0.499444in}%
\pgfsys@useobject{currentmarker}{}%
\end{pgfscope}%
\end{pgfscope}%
\begin{pgfscope}%
\definecolor{textcolor}{rgb}{0.000000,0.000000,0.000000}%
\pgfsetstrokecolor{textcolor}%
\pgfsetfillcolor{textcolor}%
\pgftext[x=2.065000in,y=0.402222in,,top]{\color{textcolor}\rmfamily\fontsize{10.000000}{12.000000}\selectfont 1.0}%
\end{pgfscope}%
\begin{pgfscope}%
\definecolor{textcolor}{rgb}{0.000000,0.000000,0.000000}%
\pgfsetstrokecolor{textcolor}%
\pgfsetfillcolor{textcolor}%
\pgftext[x=1.290000in,y=0.223333in,,top]{\color{textcolor}\rmfamily\fontsize{10.000000}{12.000000}\selectfont \(\displaystyle p\)}%
\end{pgfscope}%
\begin{pgfscope}%
\pgfsetbuttcap%
\pgfsetroundjoin%
\definecolor{currentfill}{rgb}{0.000000,0.000000,0.000000}%
\pgfsetfillcolor{currentfill}%
\pgfsetlinewidth{0.803000pt}%
\definecolor{currentstroke}{rgb}{0.000000,0.000000,0.000000}%
\pgfsetstrokecolor{currentstroke}%
\pgfsetdash{}{0pt}%
\pgfsys@defobject{currentmarker}{\pgfqpoint{-0.048611in}{0.000000in}}{\pgfqpoint{-0.000000in}{0.000000in}}{%
\pgfpathmoveto{\pgfqpoint{-0.000000in}{0.000000in}}%
\pgfpathlineto{\pgfqpoint{-0.048611in}{0.000000in}}%
\pgfusepath{stroke,fill}%
}%
\begin{pgfscope}%
\pgfsys@transformshift{0.515000in}{0.499444in}%
\pgfsys@useobject{currentmarker}{}%
\end{pgfscope}%
\end{pgfscope}%
\begin{pgfscope}%
\definecolor{textcolor}{rgb}{0.000000,0.000000,0.000000}%
\pgfsetstrokecolor{textcolor}%
\pgfsetfillcolor{textcolor}%
\pgftext[x=0.348333in, y=0.451250in, left, base]{\color{textcolor}\rmfamily\fontsize{10.000000}{12.000000}\selectfont \(\displaystyle {0}\)}%
\end{pgfscope}%
\begin{pgfscope}%
\pgfsetbuttcap%
\pgfsetroundjoin%
\definecolor{currentfill}{rgb}{0.000000,0.000000,0.000000}%
\pgfsetfillcolor{currentfill}%
\pgfsetlinewidth{0.803000pt}%
\definecolor{currentstroke}{rgb}{0.000000,0.000000,0.000000}%
\pgfsetstrokecolor{currentstroke}%
\pgfsetdash{}{0pt}%
\pgfsys@defobject{currentmarker}{\pgfqpoint{-0.048611in}{0.000000in}}{\pgfqpoint{-0.000000in}{0.000000in}}{%
\pgfpathmoveto{\pgfqpoint{-0.000000in}{0.000000in}}%
\pgfpathlineto{\pgfqpoint{-0.048611in}{0.000000in}}%
\pgfusepath{stroke,fill}%
}%
\begin{pgfscope}%
\pgfsys@transformshift{0.515000in}{0.793075in}%
\pgfsys@useobject{currentmarker}{}%
\end{pgfscope}%
\end{pgfscope}%
\begin{pgfscope}%
\definecolor{textcolor}{rgb}{0.000000,0.000000,0.000000}%
\pgfsetstrokecolor{textcolor}%
\pgfsetfillcolor{textcolor}%
\pgftext[x=0.278889in, y=0.744881in, left, base]{\color{textcolor}\rmfamily\fontsize{10.000000}{12.000000}\selectfont \(\displaystyle {10}\)}%
\end{pgfscope}%
\begin{pgfscope}%
\pgfsetbuttcap%
\pgfsetroundjoin%
\definecolor{currentfill}{rgb}{0.000000,0.000000,0.000000}%
\pgfsetfillcolor{currentfill}%
\pgfsetlinewidth{0.803000pt}%
\definecolor{currentstroke}{rgb}{0.000000,0.000000,0.000000}%
\pgfsetstrokecolor{currentstroke}%
\pgfsetdash{}{0pt}%
\pgfsys@defobject{currentmarker}{\pgfqpoint{-0.048611in}{0.000000in}}{\pgfqpoint{-0.000000in}{0.000000in}}{%
\pgfpathmoveto{\pgfqpoint{-0.000000in}{0.000000in}}%
\pgfpathlineto{\pgfqpoint{-0.048611in}{0.000000in}}%
\pgfusepath{stroke,fill}%
}%
\begin{pgfscope}%
\pgfsys@transformshift{0.515000in}{1.086706in}%
\pgfsys@useobject{currentmarker}{}%
\end{pgfscope}%
\end{pgfscope}%
\begin{pgfscope}%
\definecolor{textcolor}{rgb}{0.000000,0.000000,0.000000}%
\pgfsetstrokecolor{textcolor}%
\pgfsetfillcolor{textcolor}%
\pgftext[x=0.278889in, y=1.038511in, left, base]{\color{textcolor}\rmfamily\fontsize{10.000000}{12.000000}\selectfont \(\displaystyle {20}\)}%
\end{pgfscope}%
\begin{pgfscope}%
\pgfsetbuttcap%
\pgfsetroundjoin%
\definecolor{currentfill}{rgb}{0.000000,0.000000,0.000000}%
\pgfsetfillcolor{currentfill}%
\pgfsetlinewidth{0.803000pt}%
\definecolor{currentstroke}{rgb}{0.000000,0.000000,0.000000}%
\pgfsetstrokecolor{currentstroke}%
\pgfsetdash{}{0pt}%
\pgfsys@defobject{currentmarker}{\pgfqpoint{-0.048611in}{0.000000in}}{\pgfqpoint{-0.000000in}{0.000000in}}{%
\pgfpathmoveto{\pgfqpoint{-0.000000in}{0.000000in}}%
\pgfpathlineto{\pgfqpoint{-0.048611in}{0.000000in}}%
\pgfusepath{stroke,fill}%
}%
\begin{pgfscope}%
\pgfsys@transformshift{0.515000in}{1.380337in}%
\pgfsys@useobject{currentmarker}{}%
\end{pgfscope}%
\end{pgfscope}%
\begin{pgfscope}%
\definecolor{textcolor}{rgb}{0.000000,0.000000,0.000000}%
\pgfsetstrokecolor{textcolor}%
\pgfsetfillcolor{textcolor}%
\pgftext[x=0.278889in, y=1.332142in, left, base]{\color{textcolor}\rmfamily\fontsize{10.000000}{12.000000}\selectfont \(\displaystyle {30}\)}%
\end{pgfscope}%
\begin{pgfscope}%
\definecolor{textcolor}{rgb}{0.000000,0.000000,0.000000}%
\pgfsetstrokecolor{textcolor}%
\pgfsetfillcolor{textcolor}%
\pgftext[x=0.223333in,y=1.076944in,,bottom,rotate=90.000000]{\color{textcolor}\rmfamily\fontsize{10.000000}{12.000000}\selectfont Percent of Data Set}%
\end{pgfscope}%
\begin{pgfscope}%
\pgfsetrectcap%
\pgfsetmiterjoin%
\pgfsetlinewidth{0.803000pt}%
\definecolor{currentstroke}{rgb}{0.000000,0.000000,0.000000}%
\pgfsetstrokecolor{currentstroke}%
\pgfsetdash{}{0pt}%
\pgfpathmoveto{\pgfqpoint{0.515000in}{0.499444in}}%
\pgfpathlineto{\pgfqpoint{0.515000in}{1.654444in}}%
\pgfusepath{stroke}%
\end{pgfscope}%
\begin{pgfscope}%
\pgfsetrectcap%
\pgfsetmiterjoin%
\pgfsetlinewidth{0.803000pt}%
\definecolor{currentstroke}{rgb}{0.000000,0.000000,0.000000}%
\pgfsetstrokecolor{currentstroke}%
\pgfsetdash{}{0pt}%
\pgfpathmoveto{\pgfqpoint{2.065000in}{0.499444in}}%
\pgfpathlineto{\pgfqpoint{2.065000in}{1.654444in}}%
\pgfusepath{stroke}%
\end{pgfscope}%
\begin{pgfscope}%
\pgfsetrectcap%
\pgfsetmiterjoin%
\pgfsetlinewidth{0.803000pt}%
\definecolor{currentstroke}{rgb}{0.000000,0.000000,0.000000}%
\pgfsetstrokecolor{currentstroke}%
\pgfsetdash{}{0pt}%
\pgfpathmoveto{\pgfqpoint{0.515000in}{0.499444in}}%
\pgfpathlineto{\pgfqpoint{2.065000in}{0.499444in}}%
\pgfusepath{stroke}%
\end{pgfscope}%
\begin{pgfscope}%
\pgfsetrectcap%
\pgfsetmiterjoin%
\pgfsetlinewidth{0.803000pt}%
\definecolor{currentstroke}{rgb}{0.000000,0.000000,0.000000}%
\pgfsetstrokecolor{currentstroke}%
\pgfsetdash{}{0pt}%
\pgfpathmoveto{\pgfqpoint{0.515000in}{1.654444in}}%
\pgfpathlineto{\pgfqpoint{2.065000in}{1.654444in}}%
\pgfusepath{stroke}%
\end{pgfscope}%
\begin{pgfscope}%
\pgfsetbuttcap%
\pgfsetmiterjoin%
\definecolor{currentfill}{rgb}{1.000000,1.000000,1.000000}%
\pgfsetfillcolor{currentfill}%
\pgfsetfillopacity{0.800000}%
\pgfsetlinewidth{1.003750pt}%
\definecolor{currentstroke}{rgb}{0.800000,0.800000,0.800000}%
\pgfsetstrokecolor{currentstroke}%
\pgfsetstrokeopacity{0.800000}%
\pgfsetdash{}{0pt}%
\pgfpathmoveto{\pgfqpoint{0.612223in}{1.154445in}}%
\pgfpathlineto{\pgfqpoint{1.291945in}{1.154445in}}%
\pgfpathquadraticcurveto{\pgfqpoint{1.319722in}{1.154445in}}{\pgfqpoint{1.319722in}{1.182222in}}%
\pgfpathlineto{\pgfqpoint{1.319722in}{1.557222in}}%
\pgfpathquadraticcurveto{\pgfqpoint{1.319722in}{1.585000in}}{\pgfqpoint{1.291945in}{1.585000in}}%
\pgfpathlineto{\pgfqpoint{0.612223in}{1.585000in}}%
\pgfpathquadraticcurveto{\pgfqpoint{0.584445in}{1.585000in}}{\pgfqpoint{0.584445in}{1.557222in}}%
\pgfpathlineto{\pgfqpoint{0.584445in}{1.182222in}}%
\pgfpathquadraticcurveto{\pgfqpoint{0.584445in}{1.154445in}}{\pgfqpoint{0.612223in}{1.154445in}}%
\pgfpathlineto{\pgfqpoint{0.612223in}{1.154445in}}%
\pgfpathclose%
\pgfusepath{stroke,fill}%
\end{pgfscope}%
\begin{pgfscope}%
\pgfsetbuttcap%
\pgfsetmiterjoin%
\pgfsetlinewidth{1.003750pt}%
\definecolor{currentstroke}{rgb}{0.000000,0.000000,0.000000}%
\pgfsetstrokecolor{currentstroke}%
\pgfsetdash{}{0pt}%
\pgfpathmoveto{\pgfqpoint{0.640000in}{1.432222in}}%
\pgfpathlineto{\pgfqpoint{0.917778in}{1.432222in}}%
\pgfpathlineto{\pgfqpoint{0.917778in}{1.529444in}}%
\pgfpathlineto{\pgfqpoint{0.640000in}{1.529444in}}%
\pgfpathlineto{\pgfqpoint{0.640000in}{1.432222in}}%
\pgfpathclose%
\pgfusepath{stroke}%
\end{pgfscope}%
\begin{pgfscope}%
\definecolor{textcolor}{rgb}{0.000000,0.000000,0.000000}%
\pgfsetstrokecolor{textcolor}%
\pgfsetfillcolor{textcolor}%
\pgftext[x=1.028889in,y=1.432222in,left,base]{\color{textcolor}\rmfamily\fontsize{10.000000}{12.000000}\selectfont Neg}%
\end{pgfscope}%
\begin{pgfscope}%
\pgfsetbuttcap%
\pgfsetmiterjoin%
\definecolor{currentfill}{rgb}{0.000000,0.000000,0.000000}%
\pgfsetfillcolor{currentfill}%
\pgfsetlinewidth{0.000000pt}%
\definecolor{currentstroke}{rgb}{0.000000,0.000000,0.000000}%
\pgfsetstrokecolor{currentstroke}%
\pgfsetstrokeopacity{0.000000}%
\pgfsetdash{}{0pt}%
\pgfpathmoveto{\pgfqpoint{0.640000in}{1.236944in}}%
\pgfpathlineto{\pgfqpoint{0.917778in}{1.236944in}}%
\pgfpathlineto{\pgfqpoint{0.917778in}{1.334167in}}%
\pgfpathlineto{\pgfqpoint{0.640000in}{1.334167in}}%
\pgfpathlineto{\pgfqpoint{0.640000in}{1.236944in}}%
\pgfpathclose%
\pgfusepath{fill}%
\end{pgfscope}%
\begin{pgfscope}%
\definecolor{textcolor}{rgb}{0.000000,0.000000,0.000000}%
\pgfsetstrokecolor{textcolor}%
\pgfsetfillcolor{textcolor}%
\pgftext[x=1.028889in,y=1.236944in,left,base]{\color{textcolor}\rmfamily\fontsize{10.000000}{12.000000}\selectfont Pos}%
\end{pgfscope}%
\end{pgfpicture}%
\makeatother%
\endgroup%

%  &
%  \vspace{0pt} %% Creator: Matplotlib, PGF backend
%%
%% To include the figure in your LaTeX document, write
%%   \input{<filename>.pgf}
%%
%% Make sure the required packages are loaded in your preamble
%%   \usepackage{pgf}
%%
%% Also ensure that all the required font packages are loaded; for instance,
%% the lmodern package is sometimes necessary when using math font.
%%   \usepackage{lmodern}
%%
%% Figures using additional raster images can only be included by \input if
%% they are in the same directory as the main LaTeX file. For loading figures
%% from other directories you can use the `import` package
%%   \usepackage{import}
%%
%% and then include the figures with
%%   \import{<path to file>}{<filename>.pgf}
%%
%% Matplotlib used the following preamble
%%   
%%   \usepackage{fontspec}
%%   \makeatletter\@ifpackageloaded{underscore}{}{\usepackage[strings]{underscore}}\makeatother
%%
\begingroup%
\makeatletter%
\begin{pgfpicture}%
\pgfpathrectangle{\pgfpointorigin}{\pgfqpoint{2.221861in}{1.754444in}}%
\pgfusepath{use as bounding box, clip}%
\begin{pgfscope}%
\pgfsetbuttcap%
\pgfsetmiterjoin%
\definecolor{currentfill}{rgb}{1.000000,1.000000,1.000000}%
\pgfsetfillcolor{currentfill}%
\pgfsetlinewidth{0.000000pt}%
\definecolor{currentstroke}{rgb}{1.000000,1.000000,1.000000}%
\pgfsetstrokecolor{currentstroke}%
\pgfsetdash{}{0pt}%
\pgfpathmoveto{\pgfqpoint{0.000000in}{0.000000in}}%
\pgfpathlineto{\pgfqpoint{2.221861in}{0.000000in}}%
\pgfpathlineto{\pgfqpoint{2.221861in}{1.754444in}}%
\pgfpathlineto{\pgfqpoint{0.000000in}{1.754444in}}%
\pgfpathlineto{\pgfqpoint{0.000000in}{0.000000in}}%
\pgfpathclose%
\pgfusepath{fill}%
\end{pgfscope}%
\begin{pgfscope}%
\pgfsetbuttcap%
\pgfsetmiterjoin%
\definecolor{currentfill}{rgb}{1.000000,1.000000,1.000000}%
\pgfsetfillcolor{currentfill}%
\pgfsetlinewidth{0.000000pt}%
\definecolor{currentstroke}{rgb}{0.000000,0.000000,0.000000}%
\pgfsetstrokecolor{currentstroke}%
\pgfsetstrokeopacity{0.000000}%
\pgfsetdash{}{0pt}%
\pgfpathmoveto{\pgfqpoint{0.553581in}{0.499444in}}%
\pgfpathlineto{\pgfqpoint{2.103581in}{0.499444in}}%
\pgfpathlineto{\pgfqpoint{2.103581in}{1.654444in}}%
\pgfpathlineto{\pgfqpoint{0.553581in}{1.654444in}}%
\pgfpathlineto{\pgfqpoint{0.553581in}{0.499444in}}%
\pgfpathclose%
\pgfusepath{fill}%
\end{pgfscope}%
\begin{pgfscope}%
\pgfsetbuttcap%
\pgfsetroundjoin%
\definecolor{currentfill}{rgb}{0.000000,0.000000,0.000000}%
\pgfsetfillcolor{currentfill}%
\pgfsetlinewidth{0.803000pt}%
\definecolor{currentstroke}{rgb}{0.000000,0.000000,0.000000}%
\pgfsetstrokecolor{currentstroke}%
\pgfsetdash{}{0pt}%
\pgfsys@defobject{currentmarker}{\pgfqpoint{0.000000in}{-0.048611in}}{\pgfqpoint{0.000000in}{0.000000in}}{%
\pgfpathmoveto{\pgfqpoint{0.000000in}{0.000000in}}%
\pgfpathlineto{\pgfqpoint{0.000000in}{-0.048611in}}%
\pgfusepath{stroke,fill}%
}%
\begin{pgfscope}%
\pgfsys@transformshift{0.624035in}{0.499444in}%
\pgfsys@useobject{currentmarker}{}%
\end{pgfscope}%
\end{pgfscope}%
\begin{pgfscope}%
\definecolor{textcolor}{rgb}{0.000000,0.000000,0.000000}%
\pgfsetstrokecolor{textcolor}%
\pgfsetfillcolor{textcolor}%
\pgftext[x=0.624035in,y=0.402222in,,top]{\color{textcolor}\rmfamily\fontsize{10.000000}{12.000000}\selectfont \(\displaystyle {0.0}\)}%
\end{pgfscope}%
\begin{pgfscope}%
\pgfsetbuttcap%
\pgfsetroundjoin%
\definecolor{currentfill}{rgb}{0.000000,0.000000,0.000000}%
\pgfsetfillcolor{currentfill}%
\pgfsetlinewidth{0.803000pt}%
\definecolor{currentstroke}{rgb}{0.000000,0.000000,0.000000}%
\pgfsetstrokecolor{currentstroke}%
\pgfsetdash{}{0pt}%
\pgfsys@defobject{currentmarker}{\pgfqpoint{0.000000in}{-0.048611in}}{\pgfqpoint{0.000000in}{0.000000in}}{%
\pgfpathmoveto{\pgfqpoint{0.000000in}{0.000000in}}%
\pgfpathlineto{\pgfqpoint{0.000000in}{-0.048611in}}%
\pgfusepath{stroke,fill}%
}%
\begin{pgfscope}%
\pgfsys@transformshift{1.328581in}{0.499444in}%
\pgfsys@useobject{currentmarker}{}%
\end{pgfscope}%
\end{pgfscope}%
\begin{pgfscope}%
\definecolor{textcolor}{rgb}{0.000000,0.000000,0.000000}%
\pgfsetstrokecolor{textcolor}%
\pgfsetfillcolor{textcolor}%
\pgftext[x=1.328581in,y=0.402222in,,top]{\color{textcolor}\rmfamily\fontsize{10.000000}{12.000000}\selectfont \(\displaystyle {0.5}\)}%
\end{pgfscope}%
\begin{pgfscope}%
\pgfsetbuttcap%
\pgfsetroundjoin%
\definecolor{currentfill}{rgb}{0.000000,0.000000,0.000000}%
\pgfsetfillcolor{currentfill}%
\pgfsetlinewidth{0.803000pt}%
\definecolor{currentstroke}{rgb}{0.000000,0.000000,0.000000}%
\pgfsetstrokecolor{currentstroke}%
\pgfsetdash{}{0pt}%
\pgfsys@defobject{currentmarker}{\pgfqpoint{0.000000in}{-0.048611in}}{\pgfqpoint{0.000000in}{0.000000in}}{%
\pgfpathmoveto{\pgfqpoint{0.000000in}{0.000000in}}%
\pgfpathlineto{\pgfqpoint{0.000000in}{-0.048611in}}%
\pgfusepath{stroke,fill}%
}%
\begin{pgfscope}%
\pgfsys@transformshift{2.033126in}{0.499444in}%
\pgfsys@useobject{currentmarker}{}%
\end{pgfscope}%
\end{pgfscope}%
\begin{pgfscope}%
\definecolor{textcolor}{rgb}{0.000000,0.000000,0.000000}%
\pgfsetstrokecolor{textcolor}%
\pgfsetfillcolor{textcolor}%
\pgftext[x=2.033126in,y=0.402222in,,top]{\color{textcolor}\rmfamily\fontsize{10.000000}{12.000000}\selectfont \(\displaystyle {1.0}\)}%
\end{pgfscope}%
\begin{pgfscope}%
\definecolor{textcolor}{rgb}{0.000000,0.000000,0.000000}%
\pgfsetstrokecolor{textcolor}%
\pgfsetfillcolor{textcolor}%
\pgftext[x=1.328581in,y=0.223333in,,top]{\color{textcolor}\rmfamily\fontsize{10.000000}{12.000000}\selectfont False positive rate}%
\end{pgfscope}%
\begin{pgfscope}%
\pgfsetbuttcap%
\pgfsetroundjoin%
\definecolor{currentfill}{rgb}{0.000000,0.000000,0.000000}%
\pgfsetfillcolor{currentfill}%
\pgfsetlinewidth{0.803000pt}%
\definecolor{currentstroke}{rgb}{0.000000,0.000000,0.000000}%
\pgfsetstrokecolor{currentstroke}%
\pgfsetdash{}{0pt}%
\pgfsys@defobject{currentmarker}{\pgfqpoint{-0.048611in}{0.000000in}}{\pgfqpoint{-0.000000in}{0.000000in}}{%
\pgfpathmoveto{\pgfqpoint{-0.000000in}{0.000000in}}%
\pgfpathlineto{\pgfqpoint{-0.048611in}{0.000000in}}%
\pgfusepath{stroke,fill}%
}%
\begin{pgfscope}%
\pgfsys@transformshift{0.553581in}{0.551944in}%
\pgfsys@useobject{currentmarker}{}%
\end{pgfscope}%
\end{pgfscope}%
\begin{pgfscope}%
\definecolor{textcolor}{rgb}{0.000000,0.000000,0.000000}%
\pgfsetstrokecolor{textcolor}%
\pgfsetfillcolor{textcolor}%
\pgftext[x=0.278889in, y=0.503750in, left, base]{\color{textcolor}\rmfamily\fontsize{10.000000}{12.000000}\selectfont \(\displaystyle {0.0}\)}%
\end{pgfscope}%
\begin{pgfscope}%
\pgfsetbuttcap%
\pgfsetroundjoin%
\definecolor{currentfill}{rgb}{0.000000,0.000000,0.000000}%
\pgfsetfillcolor{currentfill}%
\pgfsetlinewidth{0.803000pt}%
\definecolor{currentstroke}{rgb}{0.000000,0.000000,0.000000}%
\pgfsetstrokecolor{currentstroke}%
\pgfsetdash{}{0pt}%
\pgfsys@defobject{currentmarker}{\pgfqpoint{-0.048611in}{0.000000in}}{\pgfqpoint{-0.000000in}{0.000000in}}{%
\pgfpathmoveto{\pgfqpoint{-0.000000in}{0.000000in}}%
\pgfpathlineto{\pgfqpoint{-0.048611in}{0.000000in}}%
\pgfusepath{stroke,fill}%
}%
\begin{pgfscope}%
\pgfsys@transformshift{0.553581in}{1.076944in}%
\pgfsys@useobject{currentmarker}{}%
\end{pgfscope}%
\end{pgfscope}%
\begin{pgfscope}%
\definecolor{textcolor}{rgb}{0.000000,0.000000,0.000000}%
\pgfsetstrokecolor{textcolor}%
\pgfsetfillcolor{textcolor}%
\pgftext[x=0.278889in, y=1.028750in, left, base]{\color{textcolor}\rmfamily\fontsize{10.000000}{12.000000}\selectfont \(\displaystyle {0.5}\)}%
\end{pgfscope}%
\begin{pgfscope}%
\pgfsetbuttcap%
\pgfsetroundjoin%
\definecolor{currentfill}{rgb}{0.000000,0.000000,0.000000}%
\pgfsetfillcolor{currentfill}%
\pgfsetlinewidth{0.803000pt}%
\definecolor{currentstroke}{rgb}{0.000000,0.000000,0.000000}%
\pgfsetstrokecolor{currentstroke}%
\pgfsetdash{}{0pt}%
\pgfsys@defobject{currentmarker}{\pgfqpoint{-0.048611in}{0.000000in}}{\pgfqpoint{-0.000000in}{0.000000in}}{%
\pgfpathmoveto{\pgfqpoint{-0.000000in}{0.000000in}}%
\pgfpathlineto{\pgfqpoint{-0.048611in}{0.000000in}}%
\pgfusepath{stroke,fill}%
}%
\begin{pgfscope}%
\pgfsys@transformshift{0.553581in}{1.601944in}%
\pgfsys@useobject{currentmarker}{}%
\end{pgfscope}%
\end{pgfscope}%
\begin{pgfscope}%
\definecolor{textcolor}{rgb}{0.000000,0.000000,0.000000}%
\pgfsetstrokecolor{textcolor}%
\pgfsetfillcolor{textcolor}%
\pgftext[x=0.278889in, y=1.553750in, left, base]{\color{textcolor}\rmfamily\fontsize{10.000000}{12.000000}\selectfont \(\displaystyle {1.0}\)}%
\end{pgfscope}%
\begin{pgfscope}%
\definecolor{textcolor}{rgb}{0.000000,0.000000,0.000000}%
\pgfsetstrokecolor{textcolor}%
\pgfsetfillcolor{textcolor}%
\pgftext[x=0.223333in,y=1.076944in,,bottom,rotate=90.000000]{\color{textcolor}\rmfamily\fontsize{10.000000}{12.000000}\selectfont True positive rate}%
\end{pgfscope}%
\begin{pgfscope}%
\pgfpathrectangle{\pgfqpoint{0.553581in}{0.499444in}}{\pgfqpoint{1.550000in}{1.155000in}}%
\pgfusepath{clip}%
\pgfsetbuttcap%
\pgfsetroundjoin%
\pgfsetlinewidth{1.505625pt}%
\definecolor{currentstroke}{rgb}{0.000000,0.000000,0.000000}%
\pgfsetstrokecolor{currentstroke}%
\pgfsetdash{{5.550000pt}{2.400000pt}}{0.000000pt}%
\pgfpathmoveto{\pgfqpoint{0.624035in}{0.551944in}}%
\pgfpathlineto{\pgfqpoint{2.033126in}{1.601944in}}%
\pgfusepath{stroke}%
\end{pgfscope}%
\begin{pgfscope}%
\pgfpathrectangle{\pgfqpoint{0.553581in}{0.499444in}}{\pgfqpoint{1.550000in}{1.155000in}}%
\pgfusepath{clip}%
\pgfsetrectcap%
\pgfsetroundjoin%
\pgfsetlinewidth{1.505625pt}%
\definecolor{currentstroke}{rgb}{0.000000,0.000000,0.000000}%
\pgfsetstrokecolor{currentstroke}%
\pgfsetdash{}{0pt}%
\pgfpathmoveto{\pgfqpoint{0.624035in}{0.551944in}}%
\pgfpathlineto{\pgfqpoint{0.626207in}{0.552574in}}%
\pgfpathlineto{\pgfqpoint{0.627318in}{0.561464in}}%
\pgfpathlineto{\pgfqpoint{0.628014in}{0.562514in}}%
\pgfpathlineto{\pgfqpoint{0.629125in}{0.563634in}}%
\pgfpathlineto{\pgfqpoint{0.629605in}{0.564614in}}%
\pgfpathlineto{\pgfqpoint{0.630699in}{0.567694in}}%
\pgfpathlineto{\pgfqpoint{0.631130in}{0.568604in}}%
\pgfpathlineto{\pgfqpoint{0.632225in}{0.573294in}}%
\pgfpathlineto{\pgfqpoint{0.632772in}{0.574064in}}%
\pgfpathlineto{\pgfqpoint{0.633866in}{0.579944in}}%
\pgfpathlineto{\pgfqpoint{0.634247in}{0.580854in}}%
\pgfpathlineto{\pgfqpoint{0.635358in}{0.585824in}}%
\pgfpathlineto{\pgfqpoint{0.635540in}{0.586874in}}%
\pgfpathlineto{\pgfqpoint{0.636634in}{0.593244in}}%
\pgfpathlineto{\pgfqpoint{0.636833in}{0.594154in}}%
\pgfpathlineto{\pgfqpoint{0.637944in}{0.600734in}}%
\pgfpathlineto{\pgfqpoint{0.638126in}{0.601784in}}%
\pgfpathlineto{\pgfqpoint{0.639187in}{0.608714in}}%
\pgfpathlineto{\pgfqpoint{0.639353in}{0.609694in}}%
\pgfpathlineto{\pgfqpoint{0.640464in}{0.616834in}}%
\pgfpathlineto{\pgfqpoint{0.640712in}{0.617884in}}%
\pgfpathlineto{\pgfqpoint{0.641806in}{0.625864in}}%
\pgfpathlineto{\pgfqpoint{0.642038in}{0.626844in}}%
\pgfpathlineto{\pgfqpoint{0.643149in}{0.635804in}}%
\pgfpathlineto{\pgfqpoint{0.643348in}{0.636854in}}%
\pgfpathlineto{\pgfqpoint{0.644392in}{0.645044in}}%
\pgfpathlineto{\pgfqpoint{0.644641in}{0.645884in}}%
\pgfpathlineto{\pgfqpoint{0.645752in}{0.652394in}}%
\pgfpathlineto{\pgfqpoint{0.645868in}{0.653164in}}%
\pgfpathlineto{\pgfqpoint{0.646979in}{0.661424in}}%
\pgfpathlineto{\pgfqpoint{0.647327in}{0.662264in}}%
\pgfpathlineto{\pgfqpoint{0.648437in}{0.668914in}}%
\pgfpathlineto{\pgfqpoint{0.648587in}{0.669824in}}%
\pgfpathlineto{\pgfqpoint{0.649697in}{0.678294in}}%
\pgfpathlineto{\pgfqpoint{0.649813in}{0.679274in}}%
\pgfpathlineto{\pgfqpoint{0.650924in}{0.687044in}}%
\pgfpathlineto{\pgfqpoint{0.651140in}{0.687884in}}%
\pgfpathlineto{\pgfqpoint{0.652250in}{0.696144in}}%
\pgfpathlineto{\pgfqpoint{0.652333in}{0.697194in}}%
\pgfpathlineto{\pgfqpoint{0.653444in}{0.706434in}}%
\pgfpathlineto{\pgfqpoint{0.653858in}{0.707414in}}%
\pgfpathlineto{\pgfqpoint{0.654969in}{0.716584in}}%
\pgfpathlineto{\pgfqpoint{0.655151in}{0.717564in}}%
\pgfpathlineto{\pgfqpoint{0.656262in}{0.725194in}}%
\pgfpathlineto{\pgfqpoint{0.656444in}{0.726174in}}%
\pgfpathlineto{\pgfqpoint{0.657538in}{0.733244in}}%
\pgfpathlineto{\pgfqpoint{0.657704in}{0.734294in}}%
\pgfpathlineto{\pgfqpoint{0.658815in}{0.743464in}}%
\pgfpathlineto{\pgfqpoint{0.659064in}{0.744514in}}%
\pgfpathlineto{\pgfqpoint{0.660158in}{0.751164in}}%
\pgfpathlineto{\pgfqpoint{0.660290in}{0.751794in}}%
\pgfpathlineto{\pgfqpoint{0.661401in}{0.759354in}}%
\pgfpathlineto{\pgfqpoint{0.661633in}{0.760404in}}%
\pgfpathlineto{\pgfqpoint{0.662744in}{0.769784in}}%
\pgfpathlineto{\pgfqpoint{0.662827in}{0.770344in}}%
\pgfpathlineto{\pgfqpoint{0.663937in}{0.777764in}}%
\pgfpathlineto{\pgfqpoint{0.664120in}{0.778814in}}%
\pgfpathlineto{\pgfqpoint{0.665181in}{0.783854in}}%
\pgfpathlineto{\pgfqpoint{0.665297in}{0.784834in}}%
\pgfpathlineto{\pgfqpoint{0.666407in}{0.795614in}}%
\pgfpathlineto{\pgfqpoint{0.666656in}{0.796664in}}%
\pgfpathlineto{\pgfqpoint{0.667767in}{0.801564in}}%
\pgfpathlineto{\pgfqpoint{0.667899in}{0.802404in}}%
\pgfpathlineto{\pgfqpoint{0.669010in}{0.809334in}}%
\pgfpathlineto{\pgfqpoint{0.669342in}{0.810384in}}%
\pgfpathlineto{\pgfqpoint{0.670452in}{0.818084in}}%
\pgfpathlineto{\pgfqpoint{0.670701in}{0.819134in}}%
\pgfpathlineto{\pgfqpoint{0.671812in}{0.825364in}}%
\pgfpathlineto{\pgfqpoint{0.671911in}{0.826414in}}%
\pgfpathlineto{\pgfqpoint{0.673022in}{0.833694in}}%
\pgfpathlineto{\pgfqpoint{0.673138in}{0.834744in}}%
\pgfpathlineto{\pgfqpoint{0.674215in}{0.838944in}}%
\pgfpathlineto{\pgfqpoint{0.674530in}{0.839994in}}%
\pgfpathlineto{\pgfqpoint{0.675608in}{0.846014in}}%
\pgfpathlineto{\pgfqpoint{0.675790in}{0.846994in}}%
\pgfpathlineto{\pgfqpoint{0.676901in}{0.853854in}}%
\pgfpathlineto{\pgfqpoint{0.677199in}{0.854834in}}%
\pgfpathlineto{\pgfqpoint{0.678310in}{0.859034in}}%
\pgfpathlineto{\pgfqpoint{0.678492in}{0.860084in}}%
\pgfpathlineto{\pgfqpoint{0.679587in}{0.867014in}}%
\pgfpathlineto{\pgfqpoint{0.679868in}{0.867924in}}%
\pgfpathlineto{\pgfqpoint{0.680963in}{0.874504in}}%
\pgfpathlineto{\pgfqpoint{0.681228in}{0.875484in}}%
\pgfpathlineto{\pgfqpoint{0.682338in}{0.882204in}}%
\pgfpathlineto{\pgfqpoint{0.682587in}{0.883254in}}%
\pgfpathlineto{\pgfqpoint{0.683681in}{0.889484in}}%
\pgfpathlineto{\pgfqpoint{0.683980in}{0.890394in}}%
\pgfpathlineto{\pgfqpoint{0.685074in}{0.897394in}}%
\pgfpathlineto{\pgfqpoint{0.685389in}{0.898304in}}%
\pgfpathlineto{\pgfqpoint{0.686450in}{0.904184in}}%
\pgfpathlineto{\pgfqpoint{0.686748in}{0.905094in}}%
\pgfpathlineto{\pgfqpoint{0.687859in}{0.910204in}}%
\pgfpathlineto{\pgfqpoint{0.688074in}{0.911114in}}%
\pgfpathlineto{\pgfqpoint{0.689185in}{0.915034in}}%
\pgfpathlineto{\pgfqpoint{0.689400in}{0.916084in}}%
\pgfpathlineto{\pgfqpoint{0.690511in}{0.922664in}}%
\pgfpathlineto{\pgfqpoint{0.690793in}{0.923714in}}%
\pgfpathlineto{\pgfqpoint{0.691887in}{0.928754in}}%
\pgfpathlineto{\pgfqpoint{0.692318in}{0.929734in}}%
\pgfpathlineto{\pgfqpoint{0.693412in}{0.935334in}}%
\pgfpathlineto{\pgfqpoint{0.693611in}{0.936384in}}%
\pgfpathlineto{\pgfqpoint{0.694705in}{0.940864in}}%
\pgfpathlineto{\pgfqpoint{0.695053in}{0.941914in}}%
\pgfpathlineto{\pgfqpoint{0.696164in}{0.948634in}}%
\pgfpathlineto{\pgfqpoint{0.696645in}{0.949684in}}%
\pgfpathlineto{\pgfqpoint{0.697756in}{0.954164in}}%
\pgfpathlineto{\pgfqpoint{0.697921in}{0.955144in}}%
\pgfpathlineto{\pgfqpoint{0.699015in}{0.960464in}}%
\pgfpathlineto{\pgfqpoint{0.699314in}{0.961304in}}%
\pgfpathlineto{\pgfqpoint{0.700425in}{0.965294in}}%
\pgfpathlineto{\pgfqpoint{0.700657in}{0.966344in}}%
\pgfpathlineto{\pgfqpoint{0.701767in}{0.971314in}}%
\pgfpathlineto{\pgfqpoint{0.702049in}{0.972364in}}%
\pgfpathlineto{\pgfqpoint{0.703094in}{0.977194in}}%
\pgfpathlineto{\pgfqpoint{0.703442in}{0.978104in}}%
\pgfpathlineto{\pgfqpoint{0.704536in}{0.982514in}}%
\pgfpathlineto{\pgfqpoint{0.704917in}{0.983564in}}%
\pgfpathlineto{\pgfqpoint{0.706011in}{0.988464in}}%
\pgfpathlineto{\pgfqpoint{0.706160in}{0.989444in}}%
\pgfpathlineto{\pgfqpoint{0.707271in}{0.993504in}}%
\pgfpathlineto{\pgfqpoint{0.707619in}{0.994414in}}%
\pgfpathlineto{\pgfqpoint{0.708697in}{0.998054in}}%
\pgfpathlineto{\pgfqpoint{0.709061in}{0.999104in}}%
\pgfpathlineto{\pgfqpoint{0.710172in}{1.004564in}}%
\pgfpathlineto{\pgfqpoint{0.710504in}{1.005614in}}%
\pgfpathlineto{\pgfqpoint{0.711614in}{1.009604in}}%
\pgfpathlineto{\pgfqpoint{0.711830in}{1.010584in}}%
\pgfpathlineto{\pgfqpoint{0.712874in}{1.015134in}}%
\pgfpathlineto{\pgfqpoint{0.713123in}{1.016184in}}%
\pgfpathlineto{\pgfqpoint{0.714184in}{1.020244in}}%
\pgfpathlineto{\pgfqpoint{0.714598in}{1.021294in}}%
\pgfpathlineto{\pgfqpoint{0.715709in}{1.024304in}}%
\pgfpathlineto{\pgfqpoint{0.716173in}{1.025354in}}%
\pgfpathlineto{\pgfqpoint{0.717284in}{1.029624in}}%
\pgfpathlineto{\pgfqpoint{0.717466in}{1.030604in}}%
\pgfpathlineto{\pgfqpoint{0.718560in}{1.035014in}}%
\pgfpathlineto{\pgfqpoint{0.718776in}{1.035994in}}%
\pgfpathlineto{\pgfqpoint{0.719853in}{1.041034in}}%
\pgfpathlineto{\pgfqpoint{0.720202in}{1.042084in}}%
\pgfpathlineto{\pgfqpoint{0.721312in}{1.046144in}}%
\pgfpathlineto{\pgfqpoint{0.721694in}{1.047194in}}%
\pgfpathlineto{\pgfqpoint{0.722788in}{1.050904in}}%
\pgfpathlineto{\pgfqpoint{0.723086in}{1.051954in}}%
\pgfpathlineto{\pgfqpoint{0.724180in}{1.056364in}}%
\pgfpathlineto{\pgfqpoint{0.724727in}{1.057414in}}%
\pgfpathlineto{\pgfqpoint{0.725805in}{1.061264in}}%
\pgfpathlineto{\pgfqpoint{0.726236in}{1.062314in}}%
\pgfpathlineto{\pgfqpoint{0.727280in}{1.065744in}}%
\pgfpathlineto{\pgfqpoint{0.727728in}{1.066794in}}%
\pgfpathlineto{\pgfqpoint{0.728805in}{1.069874in}}%
\pgfpathlineto{\pgfqpoint{0.729253in}{1.070924in}}%
\pgfpathlineto{\pgfqpoint{0.730364in}{1.074144in}}%
\pgfpathlineto{\pgfqpoint{0.730695in}{1.075124in}}%
\pgfpathlineto{\pgfqpoint{0.731789in}{1.079044in}}%
\pgfpathlineto{\pgfqpoint{0.732336in}{1.080024in}}%
\pgfpathlineto{\pgfqpoint{0.733447in}{1.084574in}}%
\pgfpathlineto{\pgfqpoint{0.734094in}{1.085624in}}%
\pgfpathlineto{\pgfqpoint{0.735188in}{1.089544in}}%
\pgfpathlineto{\pgfqpoint{0.735552in}{1.090594in}}%
\pgfpathlineto{\pgfqpoint{0.736597in}{1.093674in}}%
\pgfpathlineto{\pgfqpoint{0.737177in}{1.094724in}}%
\pgfpathlineto{\pgfqpoint{0.738221in}{1.097524in}}%
\pgfpathlineto{\pgfqpoint{0.738702in}{1.098574in}}%
\pgfpathlineto{\pgfqpoint{0.739780in}{1.102424in}}%
\pgfpathlineto{\pgfqpoint{0.740128in}{1.103474in}}%
\pgfpathlineto{\pgfqpoint{0.741238in}{1.106624in}}%
\pgfpathlineto{\pgfqpoint{0.741636in}{1.107674in}}%
\pgfpathlineto{\pgfqpoint{0.742697in}{1.111034in}}%
\pgfpathlineto{\pgfqpoint{0.743095in}{1.112084in}}%
\pgfpathlineto{\pgfqpoint{0.744206in}{1.115024in}}%
\pgfpathlineto{\pgfqpoint{0.744769in}{1.116074in}}%
\pgfpathlineto{\pgfqpoint{0.745051in}{1.117054in}}%
\pgfpathlineto{\pgfqpoint{0.745068in}{1.117054in}}%
\pgfpathlineto{\pgfqpoint{0.755943in}{1.118104in}}%
\pgfpathlineto{\pgfqpoint{0.757053in}{1.121814in}}%
\pgfpathlineto{\pgfqpoint{0.757600in}{1.122864in}}%
\pgfpathlineto{\pgfqpoint{0.758678in}{1.125384in}}%
\pgfpathlineto{\pgfqpoint{0.759225in}{1.126434in}}%
\pgfpathlineto{\pgfqpoint{0.760269in}{1.129094in}}%
\pgfpathlineto{\pgfqpoint{0.760817in}{1.130144in}}%
\pgfpathlineto{\pgfqpoint{0.761911in}{1.133224in}}%
\pgfpathlineto{\pgfqpoint{0.762474in}{1.134274in}}%
\pgfpathlineto{\pgfqpoint{0.763535in}{1.137844in}}%
\pgfpathlineto{\pgfqpoint{0.764165in}{1.138824in}}%
\pgfpathlineto{\pgfqpoint{0.765259in}{1.142324in}}%
\pgfpathlineto{\pgfqpoint{0.766055in}{1.143374in}}%
\pgfpathlineto{\pgfqpoint{0.767166in}{1.146314in}}%
\pgfpathlineto{\pgfqpoint{0.767696in}{1.147364in}}%
\pgfpathlineto{\pgfqpoint{0.768790in}{1.149954in}}%
\pgfpathlineto{\pgfqpoint{0.769603in}{1.151004in}}%
\pgfpathlineto{\pgfqpoint{0.770514in}{1.153314in}}%
\pgfpathlineto{\pgfqpoint{0.771161in}{1.154364in}}%
\pgfpathlineto{\pgfqpoint{0.772272in}{1.157374in}}%
\pgfpathlineto{\pgfqpoint{0.772852in}{1.158424in}}%
\pgfpathlineto{\pgfqpoint{0.773946in}{1.160664in}}%
\pgfpathlineto{\pgfqpoint{0.774609in}{1.161714in}}%
\pgfpathlineto{\pgfqpoint{0.775703in}{1.164794in}}%
\pgfpathlineto{\pgfqpoint{0.776267in}{1.165774in}}%
\pgfpathlineto{\pgfqpoint{0.777344in}{1.168714in}}%
\pgfpathlineto{\pgfqpoint{0.777759in}{1.169764in}}%
\pgfpathlineto{\pgfqpoint{0.778836in}{1.172144in}}%
\pgfpathlineto{\pgfqpoint{0.779350in}{1.173124in}}%
\pgfpathlineto{\pgfqpoint{0.780444in}{1.174524in}}%
\pgfpathlineto{\pgfqpoint{0.781008in}{1.175574in}}%
\pgfpathlineto{\pgfqpoint{0.782052in}{1.177254in}}%
\pgfpathlineto{\pgfqpoint{0.782699in}{1.178304in}}%
\pgfpathlineto{\pgfqpoint{0.783793in}{1.180894in}}%
\pgfpathlineto{\pgfqpoint{0.784290in}{1.181874in}}%
\pgfpathlineto{\pgfqpoint{0.785384in}{1.184044in}}%
\pgfpathlineto{\pgfqpoint{0.785981in}{1.185094in}}%
\pgfpathlineto{\pgfqpoint{0.787075in}{1.186914in}}%
\pgfpathlineto{\pgfqpoint{0.787755in}{1.187894in}}%
\pgfpathlineto{\pgfqpoint{0.788866in}{1.189784in}}%
\pgfpathlineto{\pgfqpoint{0.789297in}{1.190764in}}%
\pgfpathlineto{\pgfqpoint{0.790407in}{1.192794in}}%
\pgfpathlineto{\pgfqpoint{0.790689in}{1.193634in}}%
\pgfpathlineto{\pgfqpoint{0.791783in}{1.196084in}}%
\pgfpathlineto{\pgfqpoint{0.792347in}{1.197134in}}%
\pgfpathlineto{\pgfqpoint{0.793358in}{1.199724in}}%
\pgfpathlineto{\pgfqpoint{0.794054in}{1.200774in}}%
\pgfpathlineto{\pgfqpoint{0.795099in}{1.201964in}}%
\pgfpathlineto{\pgfqpoint{0.796044in}{1.203014in}}%
\pgfpathlineto{\pgfqpoint{0.797138in}{1.205534in}}%
\pgfpathlineto{\pgfqpoint{0.797718in}{1.206584in}}%
\pgfpathlineto{\pgfqpoint{0.798812in}{1.208474in}}%
\pgfpathlineto{\pgfqpoint{0.799409in}{1.209524in}}%
\pgfpathlineto{\pgfqpoint{0.800520in}{1.212114in}}%
\pgfpathlineto{\pgfqpoint{0.801382in}{1.213094in}}%
\pgfpathlineto{\pgfqpoint{0.802492in}{1.215894in}}%
\pgfpathlineto{\pgfqpoint{0.803106in}{1.216664in}}%
\pgfpathlineto{\pgfqpoint{0.804134in}{1.218554in}}%
\pgfpathlineto{\pgfqpoint{0.805128in}{1.219604in}}%
\pgfpathlineto{\pgfqpoint{0.806222in}{1.221774in}}%
\pgfpathlineto{\pgfqpoint{0.806786in}{1.222614in}}%
\pgfpathlineto{\pgfqpoint{0.807897in}{1.225484in}}%
\pgfpathlineto{\pgfqpoint{0.808659in}{1.226534in}}%
\pgfpathlineto{\pgfqpoint{0.809753in}{1.228284in}}%
\pgfpathlineto{\pgfqpoint{0.810267in}{1.229334in}}%
\pgfpathlineto{\pgfqpoint{0.811378in}{1.231224in}}%
\pgfpathlineto{\pgfqpoint{0.812654in}{1.232274in}}%
\pgfpathlineto{\pgfqpoint{0.813682in}{1.233744in}}%
\pgfpathlineto{\pgfqpoint{0.814180in}{1.234794in}}%
\pgfpathlineto{\pgfqpoint{0.815290in}{1.236614in}}%
\pgfpathlineto{\pgfqpoint{0.816053in}{1.237664in}}%
\pgfpathlineto{\pgfqpoint{0.816948in}{1.239484in}}%
\pgfpathlineto{\pgfqpoint{0.818109in}{1.240464in}}%
\pgfpathlineto{\pgfqpoint{0.819203in}{1.242424in}}%
\pgfpathlineto{\pgfqpoint{0.819949in}{1.243404in}}%
\pgfpathlineto{\pgfqpoint{0.821043in}{1.245364in}}%
\pgfpathlineto{\pgfqpoint{0.821838in}{1.246414in}}%
\pgfpathlineto{\pgfqpoint{0.822899in}{1.248374in}}%
\pgfpathlineto{\pgfqpoint{0.823828in}{1.249424in}}%
\pgfpathlineto{\pgfqpoint{0.824905in}{1.251314in}}%
\pgfpathlineto{\pgfqpoint{0.826049in}{1.252364in}}%
\pgfpathlineto{\pgfqpoint{0.827110in}{1.254044in}}%
\pgfpathlineto{\pgfqpoint{0.828204in}{1.255024in}}%
\pgfpathlineto{\pgfqpoint{0.829298in}{1.257404in}}%
\pgfpathlineto{\pgfqpoint{0.830094in}{1.258454in}}%
\pgfpathlineto{\pgfqpoint{0.831188in}{1.259784in}}%
\pgfpathlineto{\pgfqpoint{0.831901in}{1.260834in}}%
\pgfpathlineto{\pgfqpoint{0.832979in}{1.261954in}}%
\pgfpathlineto{\pgfqpoint{0.833841in}{1.263004in}}%
\pgfpathlineto{\pgfqpoint{0.834769in}{1.263774in}}%
\pgfpathlineto{\pgfqpoint{0.835697in}{1.264824in}}%
\pgfpathlineto{\pgfqpoint{0.836659in}{1.266014in}}%
\pgfpathlineto{\pgfqpoint{0.838234in}{1.267064in}}%
\pgfpathlineto{\pgfqpoint{0.839295in}{1.268814in}}%
\pgfpathlineto{\pgfqpoint{0.840438in}{1.269864in}}%
\pgfpathlineto{\pgfqpoint{0.841549in}{1.271824in}}%
\pgfpathlineto{\pgfqpoint{0.842594in}{1.272874in}}%
\pgfpathlineto{\pgfqpoint{0.843688in}{1.274414in}}%
\pgfpathlineto{\pgfqpoint{0.844881in}{1.275464in}}%
\pgfpathlineto{\pgfqpoint{0.845992in}{1.277844in}}%
\pgfpathlineto{\pgfqpoint{0.846705in}{1.278894in}}%
\pgfpathlineto{\pgfqpoint{0.847633in}{1.280294in}}%
\pgfpathlineto{\pgfqpoint{0.848644in}{1.281274in}}%
\pgfpathlineto{\pgfqpoint{0.849606in}{1.282954in}}%
\pgfpathlineto{\pgfqpoint{0.850882in}{1.284004in}}%
\pgfpathlineto{\pgfqpoint{0.851960in}{1.285194in}}%
\pgfpathlineto{\pgfqpoint{0.853120in}{1.286244in}}%
\pgfpathlineto{\pgfqpoint{0.854198in}{1.287644in}}%
\pgfpathlineto{\pgfqpoint{0.855176in}{1.288694in}}%
\pgfpathlineto{\pgfqpoint{0.856237in}{1.290234in}}%
\pgfpathlineto{\pgfqpoint{0.856917in}{1.291144in}}%
\pgfpathlineto{\pgfqpoint{0.858027in}{1.292754in}}%
\pgfpathlineto{\pgfqpoint{0.859287in}{1.293804in}}%
\pgfpathlineto{\pgfqpoint{0.860232in}{1.294714in}}%
\pgfpathlineto{\pgfqpoint{0.861160in}{1.295764in}}%
\pgfpathlineto{\pgfqpoint{0.862155in}{1.296814in}}%
\pgfpathlineto{\pgfqpoint{0.863746in}{1.297794in}}%
\pgfpathlineto{\pgfqpoint{0.864807in}{1.299054in}}%
\pgfpathlineto{\pgfqpoint{0.866084in}{1.300104in}}%
\pgfpathlineto{\pgfqpoint{0.867128in}{1.301854in}}%
\pgfpathlineto{\pgfqpoint{0.867775in}{1.302904in}}%
\pgfpathlineto{\pgfqpoint{0.868886in}{1.304164in}}%
\pgfpathlineto{\pgfqpoint{0.869897in}{1.305144in}}%
\pgfpathlineto{\pgfqpoint{0.870991in}{1.306614in}}%
\pgfpathlineto{\pgfqpoint{0.872218in}{1.307664in}}%
\pgfpathlineto{\pgfqpoint{0.873312in}{1.309624in}}%
\pgfpathlineto{\pgfqpoint{0.874572in}{1.310674in}}%
\pgfpathlineto{\pgfqpoint{0.875616in}{1.312284in}}%
\pgfpathlineto{\pgfqpoint{0.876263in}{1.313334in}}%
\pgfpathlineto{\pgfqpoint{0.877373in}{1.314804in}}%
\pgfpathlineto{\pgfqpoint{0.878036in}{1.315854in}}%
\pgfpathlineto{\pgfqpoint{0.879081in}{1.316974in}}%
\pgfpathlineto{\pgfqpoint{0.880191in}{1.318024in}}%
\pgfpathlineto{\pgfqpoint{0.881252in}{1.319564in}}%
\pgfpathlineto{\pgfqpoint{0.882927in}{1.320544in}}%
\pgfpathlineto{\pgfqpoint{0.883954in}{1.321384in}}%
\pgfpathlineto{\pgfqpoint{0.885165in}{1.322434in}}%
\pgfpathlineto{\pgfqpoint{0.886226in}{1.323624in}}%
\pgfpathlineto{\pgfqpoint{0.887121in}{1.324674in}}%
\pgfpathlineto{\pgfqpoint{0.888215in}{1.325864in}}%
\pgfpathlineto{\pgfqpoint{0.889624in}{1.326914in}}%
\pgfpathlineto{\pgfqpoint{0.890619in}{1.328104in}}%
\pgfpathlineto{\pgfqpoint{0.892044in}{1.329154in}}%
\pgfpathlineto{\pgfqpoint{0.892857in}{1.330064in}}%
\pgfpathlineto{\pgfqpoint{0.894100in}{1.331044in}}%
\pgfpathlineto{\pgfqpoint{0.894746in}{1.331814in}}%
\pgfpathlineto{\pgfqpoint{0.896089in}{1.332864in}}%
\pgfpathlineto{\pgfqpoint{0.897200in}{1.333984in}}%
\pgfpathlineto{\pgfqpoint{0.898327in}{1.335034in}}%
\pgfpathlineto{\pgfqpoint{0.899338in}{1.336714in}}%
\pgfpathlineto{\pgfqpoint{0.900267in}{1.337694in}}%
\pgfpathlineto{\pgfqpoint{0.901311in}{1.338534in}}%
\pgfpathlineto{\pgfqpoint{0.902554in}{1.339584in}}%
\pgfpathlineto{\pgfqpoint{0.903649in}{1.340494in}}%
\pgfpathlineto{\pgfqpoint{0.905240in}{1.341544in}}%
\pgfpathlineto{\pgfqpoint{0.906334in}{1.342874in}}%
\pgfpathlineto{\pgfqpoint{0.907743in}{1.343924in}}%
\pgfpathlineto{\pgfqpoint{0.908854in}{1.344764in}}%
\pgfpathlineto{\pgfqpoint{0.910048in}{1.345814in}}%
\pgfpathlineto{\pgfqpoint{0.911142in}{1.346654in}}%
\pgfpathlineto{\pgfqpoint{0.912169in}{1.347704in}}%
\pgfpathlineto{\pgfqpoint{0.913114in}{1.348754in}}%
\pgfpathlineto{\pgfqpoint{0.914242in}{1.349804in}}%
\pgfpathlineto{\pgfqpoint{0.915004in}{1.350924in}}%
\pgfpathlineto{\pgfqpoint{0.916894in}{1.351974in}}%
\pgfpathlineto{\pgfqpoint{0.917988in}{1.353234in}}%
\pgfpathlineto{\pgfqpoint{0.919646in}{1.354284in}}%
\pgfpathlineto{\pgfqpoint{0.920757in}{1.356034in}}%
\pgfpathlineto{\pgfqpoint{0.922166in}{1.357084in}}%
\pgfpathlineto{\pgfqpoint{0.923260in}{1.358204in}}%
\pgfpathlineto{\pgfqpoint{0.925332in}{1.359254in}}%
\pgfpathlineto{\pgfqpoint{0.926376in}{1.360164in}}%
\pgfpathlineto{\pgfqpoint{0.927686in}{1.361214in}}%
\pgfpathlineto{\pgfqpoint{0.928780in}{1.361914in}}%
\pgfpathlineto{\pgfqpoint{0.930886in}{1.362964in}}%
\pgfpathlineto{\pgfqpoint{0.931980in}{1.364294in}}%
\pgfpathlineto{\pgfqpoint{0.932941in}{1.365344in}}%
\pgfpathlineto{\pgfqpoint{0.934035in}{1.366604in}}%
\pgfpathlineto{\pgfqpoint{0.935925in}{1.367654in}}%
\pgfpathlineto{\pgfqpoint{0.937003in}{1.368984in}}%
\pgfpathlineto{\pgfqpoint{0.939307in}{1.370034in}}%
\pgfpathlineto{\pgfqpoint{0.940302in}{1.371014in}}%
\pgfpathlineto{\pgfqpoint{0.941876in}{1.372064in}}%
\pgfpathlineto{\pgfqpoint{0.942871in}{1.372694in}}%
\pgfpathlineto{\pgfqpoint{0.944562in}{1.373674in}}%
\pgfpathlineto{\pgfqpoint{0.945573in}{1.374934in}}%
\pgfpathlineto{\pgfqpoint{0.947314in}{1.375984in}}%
\pgfpathlineto{\pgfqpoint{0.948159in}{1.376824in}}%
\pgfpathlineto{\pgfqpoint{0.950663in}{1.377874in}}%
\pgfpathlineto{\pgfqpoint{0.951707in}{1.378504in}}%
\pgfpathlineto{\pgfqpoint{0.953083in}{1.379484in}}%
\pgfpathlineto{\pgfqpoint{0.954127in}{1.380394in}}%
\pgfpathlineto{\pgfqpoint{0.956183in}{1.381444in}}%
\pgfpathlineto{\pgfqpoint{0.957244in}{1.382914in}}%
\pgfpathlineto{\pgfqpoint{0.958736in}{1.383964in}}%
\pgfpathlineto{\pgfqpoint{0.959714in}{1.384874in}}%
\pgfpathlineto{\pgfqpoint{0.961902in}{1.385924in}}%
\pgfpathlineto{\pgfqpoint{0.962781in}{1.386414in}}%
\pgfpathlineto{\pgfqpoint{0.965599in}{1.387464in}}%
\pgfpathlineto{\pgfqpoint{0.966643in}{1.388514in}}%
\pgfpathlineto{\pgfqpoint{0.968069in}{1.389564in}}%
\pgfpathlineto{\pgfqpoint{0.968931in}{1.390404in}}%
\pgfpathlineto{\pgfqpoint{0.971733in}{1.391454in}}%
\pgfpathlineto{\pgfqpoint{0.972843in}{1.391874in}}%
\pgfpathlineto{\pgfqpoint{0.974286in}{1.392924in}}%
\pgfpathlineto{\pgfqpoint{0.975197in}{1.393554in}}%
\pgfpathlineto{\pgfqpoint{0.977336in}{1.394604in}}%
\pgfpathlineto{\pgfqpoint{0.978413in}{1.395654in}}%
\pgfpathlineto{\pgfqpoint{0.979607in}{1.396704in}}%
\pgfpathlineto{\pgfqpoint{0.980635in}{1.397614in}}%
\pgfpathlineto{\pgfqpoint{0.981729in}{1.398664in}}%
\pgfpathlineto{\pgfqpoint{0.982690in}{1.399224in}}%
\pgfpathlineto{\pgfqpoint{0.984365in}{1.400274in}}%
\pgfpathlineto{\pgfqpoint{0.985343in}{1.401044in}}%
\pgfpathlineto{\pgfqpoint{0.987050in}{1.402094in}}%
\pgfpathlineto{\pgfqpoint{0.987912in}{1.403004in}}%
\pgfpathlineto{\pgfqpoint{0.990051in}{1.404054in}}%
\pgfpathlineto{\pgfqpoint{0.990631in}{1.404474in}}%
\pgfpathlineto{\pgfqpoint{0.992935in}{1.405524in}}%
\pgfpathlineto{\pgfqpoint{0.993847in}{1.406294in}}%
\pgfpathlineto{\pgfqpoint{0.995787in}{1.407344in}}%
\pgfpathlineto{\pgfqpoint{0.996798in}{1.408534in}}%
\pgfpathlineto{\pgfqpoint{0.998456in}{1.409444in}}%
\pgfpathlineto{\pgfqpoint{0.999533in}{1.410424in}}%
\pgfpathlineto{\pgfqpoint{1.001224in}{1.411404in}}%
\pgfpathlineto{\pgfqpoint{1.002235in}{1.412314in}}%
\pgfpathlineto{\pgfqpoint{1.003545in}{1.413364in}}%
\pgfpathlineto{\pgfqpoint{1.004423in}{1.414274in}}%
\pgfpathlineto{\pgfqpoint{1.005882in}{1.415324in}}%
\pgfpathlineto{\pgfqpoint{1.006678in}{1.415954in}}%
\pgfpathlineto{\pgfqpoint{1.009165in}{1.417004in}}%
\pgfpathlineto{\pgfqpoint{1.010226in}{1.417774in}}%
\pgfpathlineto{\pgfqpoint{1.012861in}{1.418824in}}%
\pgfpathlineto{\pgfqpoint{1.013972in}{1.419524in}}%
\pgfpathlineto{\pgfqpoint{1.016442in}{1.420504in}}%
\pgfpathlineto{\pgfqpoint{1.017437in}{1.421274in}}%
\pgfpathlineto{\pgfqpoint{1.019890in}{1.422324in}}%
\pgfpathlineto{\pgfqpoint{1.020852in}{1.423444in}}%
\pgfpathlineto{\pgfqpoint{1.022626in}{1.424494in}}%
\pgfpathlineto{\pgfqpoint{1.023488in}{1.425334in}}%
\pgfpathlineto{\pgfqpoint{1.026405in}{1.426384in}}%
\pgfpathlineto{\pgfqpoint{1.027516in}{1.427224in}}%
\pgfpathlineto{\pgfqpoint{1.029787in}{1.428274in}}%
\pgfpathlineto{\pgfqpoint{1.030831in}{1.429184in}}%
\pgfpathlineto{\pgfqpoint{1.032887in}{1.430234in}}%
\pgfpathlineto{\pgfqpoint{1.033882in}{1.431074in}}%
\pgfpathlineto{\pgfqpoint{1.036667in}{1.432124in}}%
\pgfpathlineto{\pgfqpoint{1.037711in}{1.432894in}}%
\pgfpathlineto{\pgfqpoint{1.041872in}{1.433944in}}%
\pgfpathlineto{\pgfqpoint{1.042834in}{1.434504in}}%
\pgfpathlineto{\pgfqpoint{1.046530in}{1.435554in}}%
\pgfpathlineto{\pgfqpoint{1.047575in}{1.436044in}}%
\pgfpathlineto{\pgfqpoint{1.049514in}{1.437094in}}%
\pgfpathlineto{\pgfqpoint{1.050625in}{1.437794in}}%
\pgfpathlineto{\pgfqpoint{1.053095in}{1.438844in}}%
\pgfpathlineto{\pgfqpoint{1.054090in}{1.439544in}}%
\pgfpathlineto{\pgfqpoint{1.056692in}{1.440594in}}%
\pgfpathlineto{\pgfqpoint{1.057704in}{1.441364in}}%
\pgfpathlineto{\pgfqpoint{1.061301in}{1.442414in}}%
\pgfpathlineto{\pgfqpoint{1.062329in}{1.443044in}}%
\pgfpathlineto{\pgfqpoint{1.065893in}{1.444094in}}%
\pgfpathlineto{\pgfqpoint{1.066987in}{1.444584in}}%
\pgfpathlineto{\pgfqpoint{1.068976in}{1.445634in}}%
\pgfpathlineto{\pgfqpoint{1.070037in}{1.446544in}}%
\pgfpathlineto{\pgfqpoint{1.072491in}{1.447594in}}%
\pgfpathlineto{\pgfqpoint{1.073369in}{1.448224in}}%
\pgfpathlineto{\pgfqpoint{1.075840in}{1.449274in}}%
\pgfpathlineto{\pgfqpoint{1.076652in}{1.449694in}}%
\pgfpathlineto{\pgfqpoint{1.079006in}{1.450744in}}%
\pgfpathlineto{\pgfqpoint{1.079967in}{1.451234in}}%
\pgfpathlineto{\pgfqpoint{1.083001in}{1.452214in}}%
\pgfpathlineto{\pgfqpoint{1.084112in}{1.452914in}}%
\pgfpathlineto{\pgfqpoint{1.087726in}{1.453894in}}%
\pgfpathlineto{\pgfqpoint{1.088438in}{1.454384in}}%
\pgfpathlineto{\pgfqpoint{1.092119in}{1.455434in}}%
\pgfpathlineto{\pgfqpoint{1.093064in}{1.456064in}}%
\pgfpathlineto{\pgfqpoint{1.096346in}{1.457114in}}%
\pgfpathlineto{\pgfqpoint{1.097274in}{1.457674in}}%
\pgfpathlineto{\pgfqpoint{1.100092in}{1.458724in}}%
\pgfpathlineto{\pgfqpoint{1.100689in}{1.459004in}}%
\pgfpathlineto{\pgfqpoint{1.104038in}{1.460054in}}%
\pgfpathlineto{\pgfqpoint{1.105149in}{1.460404in}}%
\pgfpathlineto{\pgfqpoint{1.109989in}{1.461454in}}%
\pgfpathlineto{\pgfqpoint{1.111083in}{1.462154in}}%
\pgfpathlineto{\pgfqpoint{1.114813in}{1.463204in}}%
\pgfpathlineto{\pgfqpoint{1.115858in}{1.463624in}}%
\pgfpathlineto{\pgfqpoint{1.119389in}{1.464674in}}%
\pgfpathlineto{\pgfqpoint{1.120433in}{1.465094in}}%
\pgfpathlineto{\pgfqpoint{1.122754in}{1.466144in}}%
\pgfpathlineto{\pgfqpoint{1.123732in}{1.466494in}}%
\pgfpathlineto{\pgfqpoint{1.127910in}{1.467544in}}%
\pgfpathlineto{\pgfqpoint{1.128556in}{1.468034in}}%
\pgfpathlineto{\pgfqpoint{1.131010in}{1.469084in}}%
\pgfpathlineto{\pgfqpoint{1.131822in}{1.469924in}}%
\pgfpathlineto{\pgfqpoint{1.135187in}{1.470974in}}%
\pgfpathlineto{\pgfqpoint{1.135204in}{1.471184in}}%
\pgfpathlineto{\pgfqpoint{1.140956in}{1.472234in}}%
\pgfpathlineto{\pgfqpoint{1.141735in}{1.472584in}}%
\pgfpathlineto{\pgfqpoint{1.145598in}{1.473634in}}%
\pgfpathlineto{\pgfqpoint{1.146675in}{1.474194in}}%
\pgfpathlineto{\pgfqpoint{1.150803in}{1.475244in}}%
\pgfpathlineto{\pgfqpoint{1.151516in}{1.475874in}}%
\pgfpathlineto{\pgfqpoint{1.156058in}{1.476924in}}%
\pgfpathlineto{\pgfqpoint{1.157152in}{1.477414in}}%
\pgfpathlineto{\pgfqpoint{1.162407in}{1.478464in}}%
\pgfpathlineto{\pgfqpoint{1.163452in}{1.479094in}}%
\pgfpathlineto{\pgfqpoint{1.168060in}{1.480144in}}%
\pgfpathlineto{\pgfqpoint{1.168873in}{1.480494in}}%
\pgfpathlineto{\pgfqpoint{1.171923in}{1.481474in}}%
\pgfpathlineto{\pgfqpoint{1.172835in}{1.481754in}}%
\pgfpathlineto{\pgfqpoint{1.177161in}{1.482804in}}%
\pgfpathlineto{\pgfqpoint{1.178156in}{1.483294in}}%
\pgfpathlineto{\pgfqpoint{1.181223in}{1.484344in}}%
\pgfpathlineto{\pgfqpoint{1.181273in}{1.484554in}}%
\pgfpathlineto{\pgfqpoint{1.188633in}{1.485604in}}%
\pgfpathlineto{\pgfqpoint{1.189512in}{1.485884in}}%
\pgfpathlineto{\pgfqpoint{1.193341in}{1.486934in}}%
\pgfpathlineto{\pgfqpoint{1.194203in}{1.487354in}}%
\pgfpathlineto{\pgfqpoint{1.198928in}{1.488404in}}%
\pgfpathlineto{\pgfqpoint{1.199823in}{1.488964in}}%
\pgfpathlineto{\pgfqpoint{1.201895in}{1.489944in}}%
\pgfpathlineto{\pgfqpoint{1.202343in}{1.490224in}}%
\pgfpathlineto{\pgfqpoint{1.205343in}{1.491274in}}%
\pgfpathlineto{\pgfqpoint{1.206404in}{1.491554in}}%
\pgfpathlineto{\pgfqpoint{1.209670in}{1.492604in}}%
\pgfpathlineto{\pgfqpoint{1.210748in}{1.493024in}}%
\pgfpathlineto{\pgfqpoint{1.215887in}{1.494074in}}%
\pgfpathlineto{\pgfqpoint{1.216865in}{1.494214in}}%
\pgfpathlineto{\pgfqpoint{1.223860in}{1.495264in}}%
\pgfpathlineto{\pgfqpoint{1.224739in}{1.495684in}}%
\pgfpathlineto{\pgfqpoint{1.229994in}{1.496734in}}%
\pgfpathlineto{\pgfqpoint{1.230823in}{1.497014in}}%
\pgfpathlineto{\pgfqpoint{1.234354in}{1.498064in}}%
\pgfpathlineto{\pgfqpoint{1.235117in}{1.498344in}}%
\pgfpathlineto{\pgfqpoint{1.240256in}{1.499394in}}%
\pgfpathlineto{\pgfqpoint{1.241151in}{1.499674in}}%
\pgfpathlineto{\pgfqpoint{1.245610in}{1.500724in}}%
\pgfpathlineto{\pgfqpoint{1.246721in}{1.501144in}}%
\pgfpathlineto{\pgfqpoint{1.248942in}{1.502194in}}%
\pgfpathlineto{\pgfqpoint{1.250053in}{1.502754in}}%
\pgfpathlineto{\pgfqpoint{1.254429in}{1.503804in}}%
\pgfpathlineto{\pgfqpoint{1.254728in}{1.504084in}}%
\pgfpathlineto{\pgfqpoint{1.260845in}{1.505134in}}%
\pgfpathlineto{\pgfqpoint{1.261723in}{1.505414in}}%
\pgfpathlineto{\pgfqpoint{1.266697in}{1.506464in}}%
\pgfpathlineto{\pgfqpoint{1.267807in}{1.506744in}}%
\pgfpathlineto{\pgfqpoint{1.271338in}{1.507724in}}%
\pgfpathlineto{\pgfqpoint{1.272200in}{1.508214in}}%
\pgfpathlineto{\pgfqpoint{1.277936in}{1.509264in}}%
\pgfpathlineto{\pgfqpoint{1.278036in}{1.509404in}}%
\pgfpathlineto{\pgfqpoint{1.284203in}{1.510454in}}%
\pgfpathlineto{\pgfqpoint{1.285131in}{1.510804in}}%
\pgfpathlineto{\pgfqpoint{1.289657in}{1.511854in}}%
\pgfpathlineto{\pgfqpoint{1.290585in}{1.512274in}}%
\pgfpathlineto{\pgfqpoint{1.295940in}{1.513324in}}%
\pgfpathlineto{\pgfqpoint{1.296354in}{1.513534in}}%
\pgfpathlineto{\pgfqpoint{1.299703in}{1.514584in}}%
\pgfpathlineto{\pgfqpoint{1.300664in}{1.514934in}}%
\pgfpathlineto{\pgfqpoint{1.305389in}{1.515984in}}%
\pgfpathlineto{\pgfqpoint{1.306118in}{1.516334in}}%
\pgfpathlineto{\pgfqpoint{1.311290in}{1.517384in}}%
\pgfpathlineto{\pgfqpoint{1.312318in}{1.517734in}}%
\pgfpathlineto{\pgfqpoint{1.318104in}{1.518784in}}%
\pgfpathlineto{\pgfqpoint{1.318899in}{1.519274in}}%
\pgfpathlineto{\pgfqpoint{1.323873in}{1.520324in}}%
\pgfpathlineto{\pgfqpoint{1.324950in}{1.520674in}}%
\pgfpathlineto{\pgfqpoint{1.329575in}{1.521654in}}%
\pgfpathlineto{\pgfqpoint{1.330305in}{1.521864in}}%
\pgfpathlineto{\pgfqpoint{1.336521in}{1.522844in}}%
\pgfpathlineto{\pgfqpoint{1.336836in}{1.523194in}}%
\pgfpathlineto{\pgfqpoint{1.344180in}{1.524244in}}%
\pgfpathlineto{\pgfqpoint{1.345191in}{1.524524in}}%
\pgfpathlineto{\pgfqpoint{1.351905in}{1.525574in}}%
\pgfpathlineto{\pgfqpoint{1.352602in}{1.525994in}}%
\pgfpathlineto{\pgfqpoint{1.359050in}{1.527044in}}%
\pgfpathlineto{\pgfqpoint{1.359962in}{1.527394in}}%
\pgfpathlineto{\pgfqpoint{1.365897in}{1.528444in}}%
\pgfpathlineto{\pgfqpoint{1.366991in}{1.528654in}}%
\pgfpathlineto{\pgfqpoint{1.378463in}{1.529704in}}%
\pgfpathlineto{\pgfqpoint{1.379092in}{1.530054in}}%
\pgfpathlineto{\pgfqpoint{1.386221in}{1.531104in}}%
\pgfpathlineto{\pgfqpoint{1.387298in}{1.531314in}}%
\pgfpathlineto{\pgfqpoint{1.393366in}{1.532364in}}%
\pgfpathlineto{\pgfqpoint{1.393880in}{1.532714in}}%
\pgfpathlineto{\pgfqpoint{1.403760in}{1.533764in}}%
\pgfpathlineto{\pgfqpoint{1.404340in}{1.533904in}}%
\pgfpathlineto{\pgfqpoint{1.412529in}{1.534954in}}%
\pgfpathlineto{\pgfqpoint{1.412994in}{1.535164in}}%
\pgfpathlineto{\pgfqpoint{1.422277in}{1.536214in}}%
\pgfpathlineto{\pgfqpoint{1.422807in}{1.536354in}}%
\pgfpathlineto{\pgfqpoint{1.428460in}{1.537404in}}%
\pgfpathlineto{\pgfqpoint{1.429157in}{1.537614in}}%
\pgfpathlineto{\pgfqpoint{1.435340in}{1.538664in}}%
\pgfpathlineto{\pgfqpoint{1.436351in}{1.539014in}}%
\pgfpathlineto{\pgfqpoint{1.443861in}{1.540064in}}%
\pgfpathlineto{\pgfqpoint{1.443877in}{1.540274in}}%
\pgfpathlineto{\pgfqpoint{1.450940in}{1.541324in}}%
\pgfpathlineto{\pgfqpoint{1.451006in}{1.541464in}}%
\pgfpathlineto{\pgfqpoint{1.462776in}{1.542514in}}%
\pgfpathlineto{\pgfqpoint{1.463439in}{1.542724in}}%
\pgfpathlineto{\pgfqpoint{1.473817in}{1.543774in}}%
\pgfpathlineto{\pgfqpoint{1.474347in}{1.544124in}}%
\pgfpathlineto{\pgfqpoint{1.482304in}{1.545174in}}%
\pgfpathlineto{\pgfqpoint{1.482669in}{1.545314in}}%
\pgfpathlineto{\pgfqpoint{1.493892in}{1.546364in}}%
\pgfpathlineto{\pgfqpoint{1.494257in}{1.546504in}}%
\pgfpathlineto{\pgfqpoint{1.503756in}{1.547554in}}%
\pgfpathlineto{\pgfqpoint{1.503921in}{1.547834in}}%
\pgfpathlineto{\pgfqpoint{1.514382in}{1.548884in}}%
\pgfpathlineto{\pgfqpoint{1.515227in}{1.549234in}}%
\pgfpathlineto{\pgfqpoint{1.524046in}{1.550284in}}%
\pgfpathlineto{\pgfqpoint{1.524378in}{1.550564in}}%
\pgfpathlineto{\pgfqpoint{1.535187in}{1.551614in}}%
\pgfpathlineto{\pgfqpoint{1.535817in}{1.551754in}}%
\pgfpathlineto{\pgfqpoint{1.546012in}{1.552804in}}%
\pgfpathlineto{\pgfqpoint{1.547122in}{1.553154in}}%
\pgfpathlineto{\pgfqpoint{1.557450in}{1.554204in}}%
\pgfpathlineto{\pgfqpoint{1.557450in}{1.554274in}}%
\pgfpathlineto{\pgfqpoint{1.571690in}{1.555324in}}%
\pgfpathlineto{\pgfqpoint{1.571790in}{1.555534in}}%
\pgfpathlineto{\pgfqpoint{1.581405in}{1.556584in}}%
\pgfpathlineto{\pgfqpoint{1.581653in}{1.556794in}}%
\pgfpathlineto{\pgfqpoint{1.595396in}{1.557844in}}%
\pgfpathlineto{\pgfqpoint{1.595844in}{1.558054in}}%
\pgfpathlineto{\pgfqpoint{1.606503in}{1.559104in}}%
\pgfpathlineto{\pgfqpoint{1.607514in}{1.559384in}}%
\pgfpathlineto{\pgfqpoint{1.617610in}{1.560434in}}%
\pgfpathlineto{\pgfqpoint{1.618721in}{1.560644in}}%
\pgfpathlineto{\pgfqpoint{1.630889in}{1.561694in}}%
\pgfpathlineto{\pgfqpoint{1.631883in}{1.561834in}}%
\pgfpathlineto{\pgfqpoint{1.652605in}{1.562884in}}%
\pgfpathlineto{\pgfqpoint{1.653600in}{1.563024in}}%
\pgfpathlineto{\pgfqpoint{1.663812in}{1.564074in}}%
\pgfpathlineto{\pgfqpoint{1.663861in}{1.564214in}}%
\pgfpathlineto{\pgfqpoint{1.672283in}{1.565264in}}%
\pgfpathlineto{\pgfqpoint{1.672382in}{1.565404in}}%
\pgfpathlineto{\pgfqpoint{1.683406in}{1.566454in}}%
\pgfpathlineto{\pgfqpoint{1.684467in}{1.566664in}}%
\pgfpathlineto{\pgfqpoint{1.696586in}{1.567714in}}%
\pgfpathlineto{\pgfqpoint{1.696834in}{1.567854in}}%
\pgfpathlineto{\pgfqpoint{1.710212in}{1.568904in}}%
\pgfpathlineto{\pgfqpoint{1.710909in}{1.569114in}}%
\pgfpathlineto{\pgfqpoint{1.723706in}{1.570164in}}%
\pgfpathlineto{\pgfqpoint{1.724718in}{1.570304in}}%
\pgfpathlineto{\pgfqpoint{1.739090in}{1.571354in}}%
\pgfpathlineto{\pgfqpoint{1.739422in}{1.571494in}}%
\pgfpathlineto{\pgfqpoint{1.756828in}{1.572544in}}%
\pgfpathlineto{\pgfqpoint{1.757591in}{1.572754in}}%
\pgfpathlineto{\pgfqpoint{1.769825in}{1.573804in}}%
\pgfpathlineto{\pgfqpoint{1.770339in}{1.573944in}}%
\pgfpathlineto{\pgfqpoint{1.782242in}{1.574994in}}%
\pgfpathlineto{\pgfqpoint{1.782374in}{1.575134in}}%
\pgfpathlineto{\pgfqpoint{1.795288in}{1.576184in}}%
\pgfpathlineto{\pgfqpoint{1.795868in}{1.576394in}}%
\pgfpathlineto{\pgfqpoint{1.809910in}{1.577444in}}%
\pgfpathlineto{\pgfqpoint{1.810987in}{1.577724in}}%
\pgfpathlineto{\pgfqpoint{1.828360in}{1.578774in}}%
\pgfpathlineto{\pgfqpoint{1.829057in}{1.578984in}}%
\pgfpathlineto{\pgfqpoint{1.842236in}{1.580034in}}%
\pgfpathlineto{\pgfqpoint{1.842368in}{1.580174in}}%
\pgfpathlineto{\pgfqpoint{1.859410in}{1.581224in}}%
\pgfpathlineto{\pgfqpoint{1.859891in}{1.581364in}}%
\pgfpathlineto{\pgfqpoint{1.873683in}{1.582414in}}%
\pgfpathlineto{\pgfqpoint{1.873683in}{1.582484in}}%
\pgfpathlineto{\pgfqpoint{1.894190in}{1.583534in}}%
\pgfpathlineto{\pgfqpoint{1.894637in}{1.583674in}}%
\pgfpathlineto{\pgfqpoint{1.915542in}{1.584724in}}%
\pgfpathlineto{\pgfqpoint{1.916155in}{1.584864in}}%
\pgfpathlineto{\pgfqpoint{1.929334in}{1.585914in}}%
\pgfpathlineto{\pgfqpoint{1.930345in}{1.586054in}}%
\pgfpathlineto{\pgfqpoint{1.945680in}{1.587104in}}%
\pgfpathlineto{\pgfqpoint{1.945680in}{1.587174in}}%
\pgfpathlineto{\pgfqpoint{1.958245in}{1.588224in}}%
\pgfpathlineto{\pgfqpoint{1.959074in}{1.588434in}}%
\pgfpathlineto{\pgfqpoint{1.971541in}{1.589484in}}%
\pgfpathlineto{\pgfqpoint{1.972005in}{1.589624in}}%
\pgfpathlineto{\pgfqpoint{1.980807in}{1.590674in}}%
\pgfpathlineto{\pgfqpoint{1.981636in}{1.590954in}}%
\pgfpathlineto{\pgfqpoint{1.988947in}{1.592004in}}%
\pgfpathlineto{\pgfqpoint{1.988947in}{1.592074in}}%
\pgfpathlineto{\pgfqpoint{1.998313in}{1.593124in}}%
\pgfpathlineto{\pgfqpoint{1.999391in}{1.593334in}}%
\pgfpathlineto{\pgfqpoint{2.006536in}{1.594384in}}%
\pgfpathlineto{\pgfqpoint{2.007414in}{1.594524in}}%
\pgfpathlineto{\pgfqpoint{2.013349in}{1.595574in}}%
\pgfpathlineto{\pgfqpoint{2.014195in}{1.595854in}}%
\pgfpathlineto{\pgfqpoint{2.021936in}{1.596904in}}%
\pgfpathlineto{\pgfqpoint{2.022318in}{1.597044in}}%
\pgfpathlineto{\pgfqpoint{2.028136in}{1.598094in}}%
\pgfpathlineto{\pgfqpoint{2.029230in}{1.598584in}}%
\pgfpathlineto{\pgfqpoint{2.032015in}{1.599634in}}%
\pgfpathlineto{\pgfqpoint{2.033126in}{1.601944in}}%
\pgfpathlineto{\pgfqpoint{2.033126in}{1.601944in}}%
\pgfusepath{stroke}%
\end{pgfscope}%
\begin{pgfscope}%
\pgfsetrectcap%
\pgfsetmiterjoin%
\pgfsetlinewidth{0.803000pt}%
\definecolor{currentstroke}{rgb}{0.000000,0.000000,0.000000}%
\pgfsetstrokecolor{currentstroke}%
\pgfsetdash{}{0pt}%
\pgfpathmoveto{\pgfqpoint{0.553581in}{0.499444in}}%
\pgfpathlineto{\pgfqpoint{0.553581in}{1.654444in}}%
\pgfusepath{stroke}%
\end{pgfscope}%
\begin{pgfscope}%
\pgfsetrectcap%
\pgfsetmiterjoin%
\pgfsetlinewidth{0.803000pt}%
\definecolor{currentstroke}{rgb}{0.000000,0.000000,0.000000}%
\pgfsetstrokecolor{currentstroke}%
\pgfsetdash{}{0pt}%
\pgfpathmoveto{\pgfqpoint{2.103581in}{0.499444in}}%
\pgfpathlineto{\pgfqpoint{2.103581in}{1.654444in}}%
\pgfusepath{stroke}%
\end{pgfscope}%
\begin{pgfscope}%
\pgfsetrectcap%
\pgfsetmiterjoin%
\pgfsetlinewidth{0.803000pt}%
\definecolor{currentstroke}{rgb}{0.000000,0.000000,0.000000}%
\pgfsetstrokecolor{currentstroke}%
\pgfsetdash{}{0pt}%
\pgfpathmoveto{\pgfqpoint{0.553581in}{0.499444in}}%
\pgfpathlineto{\pgfqpoint{2.103581in}{0.499444in}}%
\pgfusepath{stroke}%
\end{pgfscope}%
\begin{pgfscope}%
\pgfsetrectcap%
\pgfsetmiterjoin%
\pgfsetlinewidth{0.803000pt}%
\definecolor{currentstroke}{rgb}{0.000000,0.000000,0.000000}%
\pgfsetstrokecolor{currentstroke}%
\pgfsetdash{}{0pt}%
\pgfpathmoveto{\pgfqpoint{0.553581in}{1.654444in}}%
\pgfpathlineto{\pgfqpoint{2.103581in}{1.654444in}}%
\pgfusepath{stroke}%
\end{pgfscope}%
\begin{pgfscope}%
\pgfsetbuttcap%
\pgfsetmiterjoin%
\definecolor{currentfill}{rgb}{1.000000,1.000000,1.000000}%
\pgfsetfillcolor{currentfill}%
\pgfsetfillopacity{0.800000}%
\pgfsetlinewidth{1.003750pt}%
\definecolor{currentstroke}{rgb}{0.800000,0.800000,0.800000}%
\pgfsetstrokecolor{currentstroke}%
\pgfsetstrokeopacity{0.800000}%
\pgfsetdash{}{0pt}%
\pgfpathmoveto{\pgfqpoint{0.832747in}{0.568889in}}%
\pgfpathlineto{\pgfqpoint{2.006358in}{0.568889in}}%
\pgfpathquadraticcurveto{\pgfqpoint{2.034136in}{0.568889in}}{\pgfqpoint{2.034136in}{0.596666in}}%
\pgfpathlineto{\pgfqpoint{2.034136in}{0.776388in}}%
\pgfpathquadraticcurveto{\pgfqpoint{2.034136in}{0.804166in}}{\pgfqpoint{2.006358in}{0.804166in}}%
\pgfpathlineto{\pgfqpoint{0.832747in}{0.804166in}}%
\pgfpathquadraticcurveto{\pgfqpoint{0.804970in}{0.804166in}}{\pgfqpoint{0.804970in}{0.776388in}}%
\pgfpathlineto{\pgfqpoint{0.804970in}{0.596666in}}%
\pgfpathquadraticcurveto{\pgfqpoint{0.804970in}{0.568889in}}{\pgfqpoint{0.832747in}{0.568889in}}%
\pgfpathlineto{\pgfqpoint{0.832747in}{0.568889in}}%
\pgfpathclose%
\pgfusepath{stroke,fill}%
\end{pgfscope}%
\begin{pgfscope}%
\pgfsetrectcap%
\pgfsetroundjoin%
\pgfsetlinewidth{1.505625pt}%
\definecolor{currentstroke}{rgb}{0.000000,0.000000,0.000000}%
\pgfsetstrokecolor{currentstroke}%
\pgfsetdash{}{0pt}%
\pgfpathmoveto{\pgfqpoint{0.860525in}{0.700000in}}%
\pgfpathlineto{\pgfqpoint{0.999414in}{0.700000in}}%
\pgfpathlineto{\pgfqpoint{1.138303in}{0.700000in}}%
\pgfusepath{stroke}%
\end{pgfscope}%
\begin{pgfscope}%
\definecolor{textcolor}{rgb}{0.000000,0.000000,0.000000}%
\pgfsetstrokecolor{textcolor}%
\pgfsetfillcolor{textcolor}%
\pgftext[x=1.249414in,y=0.651388in,left,base]{\color{textcolor}\rmfamily\fontsize{10.000000}{12.000000}\selectfont AUC=0.840}%
\end{pgfscope}%
\end{pgfpicture}%
\makeatother%
\endgroup%

  &
\vspace{0pt} 
  
\begin{tabular}{cc|c|c|}
	&\multicolumn{1}{c}{}& \multicolumn{2}{c}{Prediction} \cr
	&\multicolumn{1}{c}{} & \multicolumn{1}{c}{N} & \multicolumn{1}{c}{P} \cr\cline{3-4}
	\multirow{2}{*}{\rotatebox[origin=c]{90}{Actual}}&N & 0 & 150,771   \vrule width 0pt height 10pt depth 2pt \cr\cline{3-4}
	&P & 0 & 22,621 \vrule width 0pt height 10pt depth 2pt \cr\cline{3-4}
\end{tabular}

\begin{center}
\begin{tabular}{ll}
0.150 & Precision \cr 
1.000 & Recall \cr 
0.261 & F1 \cr 
\end{tabular}
\end{center}
  
\end{tabular}

Some models, like the Easy Ensemble Classifier and RUSBoost, give a tight range of probabilities.  Example 5 below is the probabilities from Example 1 linearly transformed with $f(x) = 0.2x + 0.4$ to have range $p \in [0,4,0.6]$.  As a model it gives the same decisions and metrics as Example 1, but is not as useful as a data visualization for comparing models, which is why we dilate the probabilities to go to $p=0.0$.

\noindent\begin{tabular}{@{}p{0.3\textwidth}@{\hspace{24pt}} p{0.3\textwidth} @{\hspace{24pt}} p{0.3\textwidth}}
  \vspace{0pt} %% Creator: Matplotlib, PGF backend
%%
%% To include the figure in your LaTeX document, write
%%   \input{<filename>.pgf}
%%
%% Make sure the required packages are loaded in your preamble
%%   \usepackage{pgf}
%%
%% Also ensure that all the required font packages are loaded; for instance,
%% the lmodern package is sometimes necessary when using math font.
%%   \usepackage{lmodern}
%%
%% Figures using additional raster images can only be included by \input if
%% they are in the same directory as the main LaTeX file. For loading figures
%% from other directories you can use the `import` package
%%   \usepackage{import}
%%
%% and then include the figures with
%%   \import{<path to file>}{<filename>.pgf}
%%
%% Matplotlib used the following preamble
%%   
%%   \usepackage{fontspec}
%%   \makeatletter\@ifpackageloaded{underscore}{}{\usepackage[strings]{underscore}}\makeatother
%%
\begingroup%
\makeatletter%
\begin{pgfpicture}%
\pgfpathrectangle{\pgfpointorigin}{\pgfqpoint{2.218304in}{1.703778in}}%
\pgfusepath{use as bounding box, clip}%
\begin{pgfscope}%
\pgfsetbuttcap%
\pgfsetmiterjoin%
\definecolor{currentfill}{rgb}{1.000000,1.000000,1.000000}%
\pgfsetfillcolor{currentfill}%
\pgfsetlinewidth{0.000000pt}%
\definecolor{currentstroke}{rgb}{1.000000,1.000000,1.000000}%
\pgfsetstrokecolor{currentstroke}%
\pgfsetdash{}{0pt}%
\pgfpathmoveto{\pgfqpoint{0.000000in}{0.000000in}}%
\pgfpathlineto{\pgfqpoint{2.218304in}{0.000000in}}%
\pgfpathlineto{\pgfqpoint{2.218304in}{1.703777in}}%
\pgfpathlineto{\pgfqpoint{0.000000in}{1.703777in}}%
\pgfpathlineto{\pgfqpoint{0.000000in}{0.000000in}}%
\pgfpathclose%
\pgfusepath{fill}%
\end{pgfscope}%
\begin{pgfscope}%
\pgfsetbuttcap%
\pgfsetmiterjoin%
\definecolor{currentfill}{rgb}{1.000000,1.000000,1.000000}%
\pgfsetfillcolor{currentfill}%
\pgfsetlinewidth{0.000000pt}%
\definecolor{currentstroke}{rgb}{0.000000,0.000000,0.000000}%
\pgfsetstrokecolor{currentstroke}%
\pgfsetstrokeopacity{0.000000}%
\pgfsetdash{}{0pt}%
\pgfpathmoveto{\pgfqpoint{0.513970in}{0.498777in}}%
\pgfpathlineto{\pgfqpoint{2.063970in}{0.498777in}}%
\pgfpathlineto{\pgfqpoint{2.063970in}{1.653777in}}%
\pgfpathlineto{\pgfqpoint{0.513970in}{1.653777in}}%
\pgfpathlineto{\pgfqpoint{0.513970in}{0.498777in}}%
\pgfpathclose%
\pgfusepath{fill}%
\end{pgfscope}%
\begin{pgfscope}%
\pgfpathrectangle{\pgfqpoint{0.513970in}{0.498777in}}{\pgfqpoint{1.550000in}{1.155000in}}%
\pgfusepath{clip}%
\pgfsetbuttcap%
\pgfsetmiterjoin%
\pgfsetlinewidth{1.003750pt}%
\definecolor{currentstroke}{rgb}{0.000000,0.000000,0.000000}%
\pgfsetstrokecolor{currentstroke}%
\pgfsetdash{}{0pt}%
\pgfpathmoveto{\pgfqpoint{0.503970in}{0.498777in}}%
\pgfpathlineto{\pgfqpoint{0.551775in}{0.498777in}}%
\pgfpathlineto{\pgfqpoint{0.551775in}{0.498777in}}%
\pgfpathlineto{\pgfqpoint{0.503970in}{0.498777in}}%
\pgfusepath{stroke}%
\end{pgfscope}%
\begin{pgfscope}%
\pgfpathrectangle{\pgfqpoint{0.513970in}{0.498777in}}{\pgfqpoint{1.550000in}{1.155000in}}%
\pgfusepath{clip}%
\pgfsetbuttcap%
\pgfsetmiterjoin%
\pgfsetlinewidth{1.003750pt}%
\definecolor{currentstroke}{rgb}{0.000000,0.000000,0.000000}%
\pgfsetstrokecolor{currentstroke}%
\pgfsetdash{}{0pt}%
\pgfpathmoveto{\pgfqpoint{0.642507in}{0.498777in}}%
\pgfpathlineto{\pgfqpoint{0.702995in}{0.498777in}}%
\pgfpathlineto{\pgfqpoint{0.702995in}{0.498777in}}%
\pgfpathlineto{\pgfqpoint{0.642507in}{0.498777in}}%
\pgfpathlineto{\pgfqpoint{0.642507in}{0.498777in}}%
\pgfpathclose%
\pgfusepath{stroke}%
\end{pgfscope}%
\begin{pgfscope}%
\pgfpathrectangle{\pgfqpoint{0.513970in}{0.498777in}}{\pgfqpoint{1.550000in}{1.155000in}}%
\pgfusepath{clip}%
\pgfsetbuttcap%
\pgfsetmiterjoin%
\pgfsetlinewidth{1.003750pt}%
\definecolor{currentstroke}{rgb}{0.000000,0.000000,0.000000}%
\pgfsetstrokecolor{currentstroke}%
\pgfsetdash{}{0pt}%
\pgfpathmoveto{\pgfqpoint{0.793727in}{0.498777in}}%
\pgfpathlineto{\pgfqpoint{0.854214in}{0.498777in}}%
\pgfpathlineto{\pgfqpoint{0.854214in}{0.498777in}}%
\pgfpathlineto{\pgfqpoint{0.793727in}{0.498777in}}%
\pgfpathlineto{\pgfqpoint{0.793727in}{0.498777in}}%
\pgfpathclose%
\pgfusepath{stroke}%
\end{pgfscope}%
\begin{pgfscope}%
\pgfpathrectangle{\pgfqpoint{0.513970in}{0.498777in}}{\pgfqpoint{1.550000in}{1.155000in}}%
\pgfusepath{clip}%
\pgfsetbuttcap%
\pgfsetmiterjoin%
\pgfsetlinewidth{1.003750pt}%
\definecolor{currentstroke}{rgb}{0.000000,0.000000,0.000000}%
\pgfsetstrokecolor{currentstroke}%
\pgfsetdash{}{0pt}%
\pgfpathmoveto{\pgfqpoint{0.944946in}{0.498777in}}%
\pgfpathlineto{\pgfqpoint{1.005434in}{0.498777in}}%
\pgfpathlineto{\pgfqpoint{1.005434in}{0.498777in}}%
\pgfpathlineto{\pgfqpoint{0.944946in}{0.498777in}}%
\pgfpathlineto{\pgfqpoint{0.944946in}{0.498777in}}%
\pgfpathclose%
\pgfusepath{stroke}%
\end{pgfscope}%
\begin{pgfscope}%
\pgfpathrectangle{\pgfqpoint{0.513970in}{0.498777in}}{\pgfqpoint{1.550000in}{1.155000in}}%
\pgfusepath{clip}%
\pgfsetbuttcap%
\pgfsetmiterjoin%
\pgfsetlinewidth{1.003750pt}%
\definecolor{currentstroke}{rgb}{0.000000,0.000000,0.000000}%
\pgfsetstrokecolor{currentstroke}%
\pgfsetdash{}{0pt}%
\pgfpathmoveto{\pgfqpoint{1.096166in}{0.498777in}}%
\pgfpathlineto{\pgfqpoint{1.156653in}{0.498777in}}%
\pgfpathlineto{\pgfqpoint{1.156653in}{1.598777in}}%
\pgfpathlineto{\pgfqpoint{1.096166in}{1.598777in}}%
\pgfpathlineto{\pgfqpoint{1.096166in}{0.498777in}}%
\pgfpathclose%
\pgfusepath{stroke}%
\end{pgfscope}%
\begin{pgfscope}%
\pgfpathrectangle{\pgfqpoint{0.513970in}{0.498777in}}{\pgfqpoint{1.550000in}{1.155000in}}%
\pgfusepath{clip}%
\pgfsetbuttcap%
\pgfsetmiterjoin%
\pgfsetlinewidth{1.003750pt}%
\definecolor{currentstroke}{rgb}{0.000000,0.000000,0.000000}%
\pgfsetstrokecolor{currentstroke}%
\pgfsetdash{}{0pt}%
\pgfpathmoveto{\pgfqpoint{1.247385in}{0.498777in}}%
\pgfpathlineto{\pgfqpoint{1.307873in}{0.498777in}}%
\pgfpathlineto{\pgfqpoint{1.307873in}{0.775681in}}%
\pgfpathlineto{\pgfqpoint{1.247385in}{0.775681in}}%
\pgfpathlineto{\pgfqpoint{1.247385in}{0.498777in}}%
\pgfpathclose%
\pgfusepath{stroke}%
\end{pgfscope}%
\begin{pgfscope}%
\pgfpathrectangle{\pgfqpoint{0.513970in}{0.498777in}}{\pgfqpoint{1.550000in}{1.155000in}}%
\pgfusepath{clip}%
\pgfsetbuttcap%
\pgfsetmiterjoin%
\pgfsetlinewidth{1.003750pt}%
\definecolor{currentstroke}{rgb}{0.000000,0.000000,0.000000}%
\pgfsetstrokecolor{currentstroke}%
\pgfsetdash{}{0pt}%
\pgfpathmoveto{\pgfqpoint{1.398605in}{0.498777in}}%
\pgfpathlineto{\pgfqpoint{1.459092in}{0.498777in}}%
\pgfpathlineto{\pgfqpoint{1.459092in}{0.498777in}}%
\pgfpathlineto{\pgfqpoint{1.398605in}{0.498777in}}%
\pgfpathlineto{\pgfqpoint{1.398605in}{0.498777in}}%
\pgfpathclose%
\pgfusepath{stroke}%
\end{pgfscope}%
\begin{pgfscope}%
\pgfpathrectangle{\pgfqpoint{0.513970in}{0.498777in}}{\pgfqpoint{1.550000in}{1.155000in}}%
\pgfusepath{clip}%
\pgfsetbuttcap%
\pgfsetmiterjoin%
\pgfsetlinewidth{1.003750pt}%
\definecolor{currentstroke}{rgb}{0.000000,0.000000,0.000000}%
\pgfsetstrokecolor{currentstroke}%
\pgfsetdash{}{0pt}%
\pgfpathmoveto{\pgfqpoint{1.549824in}{0.498777in}}%
\pgfpathlineto{\pgfqpoint{1.610312in}{0.498777in}}%
\pgfpathlineto{\pgfqpoint{1.610312in}{0.498777in}}%
\pgfpathlineto{\pgfqpoint{1.549824in}{0.498777in}}%
\pgfpathlineto{\pgfqpoint{1.549824in}{0.498777in}}%
\pgfpathclose%
\pgfusepath{stroke}%
\end{pgfscope}%
\begin{pgfscope}%
\pgfpathrectangle{\pgfqpoint{0.513970in}{0.498777in}}{\pgfqpoint{1.550000in}{1.155000in}}%
\pgfusepath{clip}%
\pgfsetbuttcap%
\pgfsetmiterjoin%
\pgfsetlinewidth{1.003750pt}%
\definecolor{currentstroke}{rgb}{0.000000,0.000000,0.000000}%
\pgfsetstrokecolor{currentstroke}%
\pgfsetdash{}{0pt}%
\pgfpathmoveto{\pgfqpoint{1.701044in}{0.498777in}}%
\pgfpathlineto{\pgfqpoint{1.761531in}{0.498777in}}%
\pgfpathlineto{\pgfqpoint{1.761531in}{0.498777in}}%
\pgfpathlineto{\pgfqpoint{1.701044in}{0.498777in}}%
\pgfpathlineto{\pgfqpoint{1.701044in}{0.498777in}}%
\pgfpathclose%
\pgfusepath{stroke}%
\end{pgfscope}%
\begin{pgfscope}%
\pgfpathrectangle{\pgfqpoint{0.513970in}{0.498777in}}{\pgfqpoint{1.550000in}{1.155000in}}%
\pgfusepath{clip}%
\pgfsetbuttcap%
\pgfsetmiterjoin%
\pgfsetlinewidth{1.003750pt}%
\definecolor{currentstroke}{rgb}{0.000000,0.000000,0.000000}%
\pgfsetstrokecolor{currentstroke}%
\pgfsetdash{}{0pt}%
\pgfpathmoveto{\pgfqpoint{1.852263in}{0.498777in}}%
\pgfpathlineto{\pgfqpoint{1.912751in}{0.498777in}}%
\pgfpathlineto{\pgfqpoint{1.912751in}{0.498777in}}%
\pgfpathlineto{\pgfqpoint{1.852263in}{0.498777in}}%
\pgfpathlineto{\pgfqpoint{1.852263in}{0.498777in}}%
\pgfpathclose%
\pgfusepath{stroke}%
\end{pgfscope}%
\begin{pgfscope}%
\pgfpathrectangle{\pgfqpoint{0.513970in}{0.498777in}}{\pgfqpoint{1.550000in}{1.155000in}}%
\pgfusepath{clip}%
\pgfsetbuttcap%
\pgfsetmiterjoin%
\definecolor{currentfill}{rgb}{0.000000,0.000000,0.000000}%
\pgfsetfillcolor{currentfill}%
\pgfsetlinewidth{0.000000pt}%
\definecolor{currentstroke}{rgb}{0.000000,0.000000,0.000000}%
\pgfsetstrokecolor{currentstroke}%
\pgfsetstrokeopacity{0.000000}%
\pgfsetdash{}{0pt}%
\pgfpathmoveto{\pgfqpoint{0.551775in}{0.498777in}}%
\pgfpathlineto{\pgfqpoint{0.612263in}{0.498777in}}%
\pgfpathlineto{\pgfqpoint{0.612263in}{0.498777in}}%
\pgfpathlineto{\pgfqpoint{0.551775in}{0.498777in}}%
\pgfpathlineto{\pgfqpoint{0.551775in}{0.498777in}}%
\pgfpathclose%
\pgfusepath{fill}%
\end{pgfscope}%
\begin{pgfscope}%
\pgfpathrectangle{\pgfqpoint{0.513970in}{0.498777in}}{\pgfqpoint{1.550000in}{1.155000in}}%
\pgfusepath{clip}%
\pgfsetbuttcap%
\pgfsetmiterjoin%
\definecolor{currentfill}{rgb}{0.000000,0.000000,0.000000}%
\pgfsetfillcolor{currentfill}%
\pgfsetlinewidth{0.000000pt}%
\definecolor{currentstroke}{rgb}{0.000000,0.000000,0.000000}%
\pgfsetstrokecolor{currentstroke}%
\pgfsetstrokeopacity{0.000000}%
\pgfsetdash{}{0pt}%
\pgfpathmoveto{\pgfqpoint{0.702995in}{0.498777in}}%
\pgfpathlineto{\pgfqpoint{0.763483in}{0.498777in}}%
\pgfpathlineto{\pgfqpoint{0.763483in}{0.498777in}}%
\pgfpathlineto{\pgfqpoint{0.702995in}{0.498777in}}%
\pgfpathlineto{\pgfqpoint{0.702995in}{0.498777in}}%
\pgfpathclose%
\pgfusepath{fill}%
\end{pgfscope}%
\begin{pgfscope}%
\pgfpathrectangle{\pgfqpoint{0.513970in}{0.498777in}}{\pgfqpoint{1.550000in}{1.155000in}}%
\pgfusepath{clip}%
\pgfsetbuttcap%
\pgfsetmiterjoin%
\definecolor{currentfill}{rgb}{0.000000,0.000000,0.000000}%
\pgfsetfillcolor{currentfill}%
\pgfsetlinewidth{0.000000pt}%
\definecolor{currentstroke}{rgb}{0.000000,0.000000,0.000000}%
\pgfsetstrokecolor{currentstroke}%
\pgfsetstrokeopacity{0.000000}%
\pgfsetdash{}{0pt}%
\pgfpathmoveto{\pgfqpoint{0.854214in}{0.498777in}}%
\pgfpathlineto{\pgfqpoint{0.914702in}{0.498777in}}%
\pgfpathlineto{\pgfqpoint{0.914702in}{0.498777in}}%
\pgfpathlineto{\pgfqpoint{0.854214in}{0.498777in}}%
\pgfpathlineto{\pgfqpoint{0.854214in}{0.498777in}}%
\pgfpathclose%
\pgfusepath{fill}%
\end{pgfscope}%
\begin{pgfscope}%
\pgfpathrectangle{\pgfqpoint{0.513970in}{0.498777in}}{\pgfqpoint{1.550000in}{1.155000in}}%
\pgfusepath{clip}%
\pgfsetbuttcap%
\pgfsetmiterjoin%
\definecolor{currentfill}{rgb}{0.000000,0.000000,0.000000}%
\pgfsetfillcolor{currentfill}%
\pgfsetlinewidth{0.000000pt}%
\definecolor{currentstroke}{rgb}{0.000000,0.000000,0.000000}%
\pgfsetstrokecolor{currentstroke}%
\pgfsetstrokeopacity{0.000000}%
\pgfsetdash{}{0pt}%
\pgfpathmoveto{\pgfqpoint{1.005434in}{0.498777in}}%
\pgfpathlineto{\pgfqpoint{1.065922in}{0.498777in}}%
\pgfpathlineto{\pgfqpoint{1.065922in}{0.498777in}}%
\pgfpathlineto{\pgfqpoint{1.005434in}{0.498777in}}%
\pgfpathlineto{\pgfqpoint{1.005434in}{0.498777in}}%
\pgfpathclose%
\pgfusepath{fill}%
\end{pgfscope}%
\begin{pgfscope}%
\pgfpathrectangle{\pgfqpoint{0.513970in}{0.498777in}}{\pgfqpoint{1.550000in}{1.155000in}}%
\pgfusepath{clip}%
\pgfsetbuttcap%
\pgfsetmiterjoin%
\definecolor{currentfill}{rgb}{0.000000,0.000000,0.000000}%
\pgfsetfillcolor{currentfill}%
\pgfsetlinewidth{0.000000pt}%
\definecolor{currentstroke}{rgb}{0.000000,0.000000,0.000000}%
\pgfsetstrokecolor{currentstroke}%
\pgfsetstrokeopacity{0.000000}%
\pgfsetdash{}{0pt}%
\pgfpathmoveto{\pgfqpoint{1.156653in}{0.498777in}}%
\pgfpathlineto{\pgfqpoint{1.217141in}{0.498777in}}%
\pgfpathlineto{\pgfqpoint{1.217141in}{0.550549in}}%
\pgfpathlineto{\pgfqpoint{1.156653in}{0.550549in}}%
\pgfpathlineto{\pgfqpoint{1.156653in}{0.498777in}}%
\pgfpathclose%
\pgfusepath{fill}%
\end{pgfscope}%
\begin{pgfscope}%
\pgfpathrectangle{\pgfqpoint{0.513970in}{0.498777in}}{\pgfqpoint{1.550000in}{1.155000in}}%
\pgfusepath{clip}%
\pgfsetbuttcap%
\pgfsetmiterjoin%
\definecolor{currentfill}{rgb}{0.000000,0.000000,0.000000}%
\pgfsetfillcolor{currentfill}%
\pgfsetlinewidth{0.000000pt}%
\definecolor{currentstroke}{rgb}{0.000000,0.000000,0.000000}%
\pgfsetstrokecolor{currentstroke}%
\pgfsetstrokeopacity{0.000000}%
\pgfsetdash{}{0pt}%
\pgfpathmoveto{\pgfqpoint{1.307873in}{0.498777in}}%
\pgfpathlineto{\pgfqpoint{1.368361in}{0.498777in}}%
\pgfpathlineto{\pgfqpoint{1.368361in}{0.689989in}}%
\pgfpathlineto{\pgfqpoint{1.307873in}{0.689989in}}%
\pgfpathlineto{\pgfqpoint{1.307873in}{0.498777in}}%
\pgfpathclose%
\pgfusepath{fill}%
\end{pgfscope}%
\begin{pgfscope}%
\pgfpathrectangle{\pgfqpoint{0.513970in}{0.498777in}}{\pgfqpoint{1.550000in}{1.155000in}}%
\pgfusepath{clip}%
\pgfsetbuttcap%
\pgfsetmiterjoin%
\definecolor{currentfill}{rgb}{0.000000,0.000000,0.000000}%
\pgfsetfillcolor{currentfill}%
\pgfsetlinewidth{0.000000pt}%
\definecolor{currentstroke}{rgb}{0.000000,0.000000,0.000000}%
\pgfsetstrokecolor{currentstroke}%
\pgfsetstrokeopacity{0.000000}%
\pgfsetdash{}{0pt}%
\pgfpathmoveto{\pgfqpoint{1.459092in}{0.498777in}}%
\pgfpathlineto{\pgfqpoint{1.519580in}{0.498777in}}%
\pgfpathlineto{\pgfqpoint{1.519580in}{0.498777in}}%
\pgfpathlineto{\pgfqpoint{1.459092in}{0.498777in}}%
\pgfpathlineto{\pgfqpoint{1.459092in}{0.498777in}}%
\pgfpathclose%
\pgfusepath{fill}%
\end{pgfscope}%
\begin{pgfscope}%
\pgfpathrectangle{\pgfqpoint{0.513970in}{0.498777in}}{\pgfqpoint{1.550000in}{1.155000in}}%
\pgfusepath{clip}%
\pgfsetbuttcap%
\pgfsetmiterjoin%
\definecolor{currentfill}{rgb}{0.000000,0.000000,0.000000}%
\pgfsetfillcolor{currentfill}%
\pgfsetlinewidth{0.000000pt}%
\definecolor{currentstroke}{rgb}{0.000000,0.000000,0.000000}%
\pgfsetstrokecolor{currentstroke}%
\pgfsetstrokeopacity{0.000000}%
\pgfsetdash{}{0pt}%
\pgfpathmoveto{\pgfqpoint{1.610312in}{0.498777in}}%
\pgfpathlineto{\pgfqpoint{1.670800in}{0.498777in}}%
\pgfpathlineto{\pgfqpoint{1.670800in}{0.498777in}}%
\pgfpathlineto{\pgfqpoint{1.610312in}{0.498777in}}%
\pgfpathlineto{\pgfqpoint{1.610312in}{0.498777in}}%
\pgfpathclose%
\pgfusepath{fill}%
\end{pgfscope}%
\begin{pgfscope}%
\pgfpathrectangle{\pgfqpoint{0.513970in}{0.498777in}}{\pgfqpoint{1.550000in}{1.155000in}}%
\pgfusepath{clip}%
\pgfsetbuttcap%
\pgfsetmiterjoin%
\definecolor{currentfill}{rgb}{0.000000,0.000000,0.000000}%
\pgfsetfillcolor{currentfill}%
\pgfsetlinewidth{0.000000pt}%
\definecolor{currentstroke}{rgb}{0.000000,0.000000,0.000000}%
\pgfsetstrokecolor{currentstroke}%
\pgfsetstrokeopacity{0.000000}%
\pgfsetdash{}{0pt}%
\pgfpathmoveto{\pgfqpoint{1.761531in}{0.498777in}}%
\pgfpathlineto{\pgfqpoint{1.822019in}{0.498777in}}%
\pgfpathlineto{\pgfqpoint{1.822019in}{0.498777in}}%
\pgfpathlineto{\pgfqpoint{1.761531in}{0.498777in}}%
\pgfpathlineto{\pgfqpoint{1.761531in}{0.498777in}}%
\pgfpathclose%
\pgfusepath{fill}%
\end{pgfscope}%
\begin{pgfscope}%
\pgfpathrectangle{\pgfqpoint{0.513970in}{0.498777in}}{\pgfqpoint{1.550000in}{1.155000in}}%
\pgfusepath{clip}%
\pgfsetbuttcap%
\pgfsetmiterjoin%
\definecolor{currentfill}{rgb}{0.000000,0.000000,0.000000}%
\pgfsetfillcolor{currentfill}%
\pgfsetlinewidth{0.000000pt}%
\definecolor{currentstroke}{rgb}{0.000000,0.000000,0.000000}%
\pgfsetstrokecolor{currentstroke}%
\pgfsetstrokeopacity{0.000000}%
\pgfsetdash{}{0pt}%
\pgfpathmoveto{\pgfqpoint{1.912751in}{0.498777in}}%
\pgfpathlineto{\pgfqpoint{1.973239in}{0.498777in}}%
\pgfpathlineto{\pgfqpoint{1.973239in}{0.498777in}}%
\pgfpathlineto{\pgfqpoint{1.912751in}{0.498777in}}%
\pgfpathlineto{\pgfqpoint{1.912751in}{0.498777in}}%
\pgfpathclose%
\pgfusepath{fill}%
\end{pgfscope}%
\begin{pgfscope}%
\pgfsetbuttcap%
\pgfsetroundjoin%
\definecolor{currentfill}{rgb}{0.000000,0.000000,0.000000}%
\pgfsetfillcolor{currentfill}%
\pgfsetlinewidth{0.803000pt}%
\definecolor{currentstroke}{rgb}{0.000000,0.000000,0.000000}%
\pgfsetstrokecolor{currentstroke}%
\pgfsetdash{}{0pt}%
\pgfsys@defobject{currentmarker}{\pgfqpoint{0.000000in}{-0.048611in}}{\pgfqpoint{0.000000in}{0.000000in}}{%
\pgfpathmoveto{\pgfqpoint{0.000000in}{0.000000in}}%
\pgfpathlineto{\pgfqpoint{0.000000in}{-0.048611in}}%
\pgfusepath{stroke,fill}%
}%
\begin{pgfscope}%
\pgfsys@transformshift{0.551775in}{0.498777in}%
\pgfsys@useobject{currentmarker}{}%
\end{pgfscope}%
\end{pgfscope}%
\begin{pgfscope}%
\definecolor{textcolor}{rgb}{0.000000,0.000000,0.000000}%
\pgfsetstrokecolor{textcolor}%
\pgfsetfillcolor{textcolor}%
\pgftext[x=0.551775in,y=0.401555in,,top]{\color{textcolor}\rmfamily\fontsize{12.000000}{14.400000}\selectfont 0.0}%
\end{pgfscope}%
\begin{pgfscope}%
\pgfsetbuttcap%
\pgfsetroundjoin%
\definecolor{currentfill}{rgb}{0.000000,0.000000,0.000000}%
\pgfsetfillcolor{currentfill}%
\pgfsetlinewidth{0.803000pt}%
\definecolor{currentstroke}{rgb}{0.000000,0.000000,0.000000}%
\pgfsetstrokecolor{currentstroke}%
\pgfsetdash{}{0pt}%
\pgfsys@defobject{currentmarker}{\pgfqpoint{0.000000in}{-0.048611in}}{\pgfqpoint{0.000000in}{0.000000in}}{%
\pgfpathmoveto{\pgfqpoint{0.000000in}{0.000000in}}%
\pgfpathlineto{\pgfqpoint{0.000000in}{-0.048611in}}%
\pgfusepath{stroke,fill}%
}%
\begin{pgfscope}%
\pgfsys@transformshift{0.929824in}{0.498777in}%
\pgfsys@useobject{currentmarker}{}%
\end{pgfscope}%
\end{pgfscope}%
\begin{pgfscope}%
\definecolor{textcolor}{rgb}{0.000000,0.000000,0.000000}%
\pgfsetstrokecolor{textcolor}%
\pgfsetfillcolor{textcolor}%
\pgftext[x=0.929824in,y=0.401555in,,top]{\color{textcolor}\rmfamily\fontsize{12.000000}{14.400000}\selectfont 0.25}%
\end{pgfscope}%
\begin{pgfscope}%
\pgfsetbuttcap%
\pgfsetroundjoin%
\definecolor{currentfill}{rgb}{0.000000,0.000000,0.000000}%
\pgfsetfillcolor{currentfill}%
\pgfsetlinewidth{0.803000pt}%
\definecolor{currentstroke}{rgb}{0.000000,0.000000,0.000000}%
\pgfsetstrokecolor{currentstroke}%
\pgfsetdash{}{0pt}%
\pgfsys@defobject{currentmarker}{\pgfqpoint{0.000000in}{-0.048611in}}{\pgfqpoint{0.000000in}{0.000000in}}{%
\pgfpathmoveto{\pgfqpoint{0.000000in}{0.000000in}}%
\pgfpathlineto{\pgfqpoint{0.000000in}{-0.048611in}}%
\pgfusepath{stroke,fill}%
}%
\begin{pgfscope}%
\pgfsys@transformshift{1.307873in}{0.498777in}%
\pgfsys@useobject{currentmarker}{}%
\end{pgfscope}%
\end{pgfscope}%
\begin{pgfscope}%
\definecolor{textcolor}{rgb}{0.000000,0.000000,0.000000}%
\pgfsetstrokecolor{textcolor}%
\pgfsetfillcolor{textcolor}%
\pgftext[x=1.307873in,y=0.401555in,,top]{\color{textcolor}\rmfamily\fontsize{12.000000}{14.400000}\selectfont 0.5}%
\end{pgfscope}%
\begin{pgfscope}%
\pgfsetbuttcap%
\pgfsetroundjoin%
\definecolor{currentfill}{rgb}{0.000000,0.000000,0.000000}%
\pgfsetfillcolor{currentfill}%
\pgfsetlinewidth{0.803000pt}%
\definecolor{currentstroke}{rgb}{0.000000,0.000000,0.000000}%
\pgfsetstrokecolor{currentstroke}%
\pgfsetdash{}{0pt}%
\pgfsys@defobject{currentmarker}{\pgfqpoint{0.000000in}{-0.048611in}}{\pgfqpoint{0.000000in}{0.000000in}}{%
\pgfpathmoveto{\pgfqpoint{0.000000in}{0.000000in}}%
\pgfpathlineto{\pgfqpoint{0.000000in}{-0.048611in}}%
\pgfusepath{stroke,fill}%
}%
\begin{pgfscope}%
\pgfsys@transformshift{1.685922in}{0.498777in}%
\pgfsys@useobject{currentmarker}{}%
\end{pgfscope}%
\end{pgfscope}%
\begin{pgfscope}%
\definecolor{textcolor}{rgb}{0.000000,0.000000,0.000000}%
\pgfsetstrokecolor{textcolor}%
\pgfsetfillcolor{textcolor}%
\pgftext[x=1.685922in,y=0.401555in,,top]{\color{textcolor}\rmfamily\fontsize{12.000000}{14.400000}\selectfont 0.75}%
\end{pgfscope}%
\begin{pgfscope}%
\pgfsetbuttcap%
\pgfsetroundjoin%
\definecolor{currentfill}{rgb}{0.000000,0.000000,0.000000}%
\pgfsetfillcolor{currentfill}%
\pgfsetlinewidth{0.803000pt}%
\definecolor{currentstroke}{rgb}{0.000000,0.000000,0.000000}%
\pgfsetstrokecolor{currentstroke}%
\pgfsetdash{}{0pt}%
\pgfsys@defobject{currentmarker}{\pgfqpoint{0.000000in}{-0.048611in}}{\pgfqpoint{0.000000in}{0.000000in}}{%
\pgfpathmoveto{\pgfqpoint{0.000000in}{0.000000in}}%
\pgfpathlineto{\pgfqpoint{0.000000in}{-0.048611in}}%
\pgfusepath{stroke,fill}%
}%
\begin{pgfscope}%
\pgfsys@transformshift{2.063970in}{0.498777in}%
\pgfsys@useobject{currentmarker}{}%
\end{pgfscope}%
\end{pgfscope}%
\begin{pgfscope}%
\definecolor{textcolor}{rgb}{0.000000,0.000000,0.000000}%
\pgfsetstrokecolor{textcolor}%
\pgfsetfillcolor{textcolor}%
\pgftext[x=2.063970in,y=0.401555in,,top]{\color{textcolor}\rmfamily\fontsize{12.000000}{14.400000}\selectfont 1.0}%
\end{pgfscope}%
\begin{pgfscope}%
\definecolor{textcolor}{rgb}{0.000000,0.000000,0.000000}%
\pgfsetstrokecolor{textcolor}%
\pgfsetfillcolor{textcolor}%
\pgftext[x=1.288970in,y=0.198000in,,top]{\color{textcolor}\rmfamily\fontsize{12.000000}{14.400000}\selectfont \(\displaystyle p\)}%
\end{pgfscope}%
\begin{pgfscope}%
\pgfsetbuttcap%
\pgfsetroundjoin%
\definecolor{currentfill}{rgb}{0.000000,0.000000,0.000000}%
\pgfsetfillcolor{currentfill}%
\pgfsetlinewidth{0.803000pt}%
\definecolor{currentstroke}{rgb}{0.000000,0.000000,0.000000}%
\pgfsetstrokecolor{currentstroke}%
\pgfsetdash{}{0pt}%
\pgfsys@defobject{currentmarker}{\pgfqpoint{-0.048611in}{0.000000in}}{\pgfqpoint{-0.000000in}{0.000000in}}{%
\pgfpathmoveto{\pgfqpoint{-0.000000in}{0.000000in}}%
\pgfpathlineto{\pgfqpoint{-0.048611in}{0.000000in}}%
\pgfusepath{stroke,fill}%
}%
\begin{pgfscope}%
\pgfsys@transformshift{0.513970in}{0.498777in}%
\pgfsys@useobject{currentmarker}{}%
\end{pgfscope}%
\end{pgfscope}%
\begin{pgfscope}%
\definecolor{textcolor}{rgb}{0.000000,0.000000,0.000000}%
\pgfsetstrokecolor{textcolor}%
\pgfsetfillcolor{textcolor}%
\pgftext[x=0.335152in, y=0.440944in, left, base]{\color{textcolor}\rmfamily\fontsize{12.000000}{14.400000}\selectfont \(\displaystyle {0}\)}%
\end{pgfscope}%
\begin{pgfscope}%
\pgfsetbuttcap%
\pgfsetroundjoin%
\definecolor{currentfill}{rgb}{0.000000,0.000000,0.000000}%
\pgfsetfillcolor{currentfill}%
\pgfsetlinewidth{0.803000pt}%
\definecolor{currentstroke}{rgb}{0.000000,0.000000,0.000000}%
\pgfsetstrokecolor{currentstroke}%
\pgfsetdash{}{0pt}%
\pgfsys@defobject{currentmarker}{\pgfqpoint{-0.048611in}{0.000000in}}{\pgfqpoint{-0.000000in}{0.000000in}}{%
\pgfpathmoveto{\pgfqpoint{-0.000000in}{0.000000in}}%
\pgfpathlineto{\pgfqpoint{-0.048611in}{0.000000in}}%
\pgfusepath{stroke,fill}%
}%
\begin{pgfscope}%
\pgfsys@transformshift{0.513970in}{0.903749in}%
\pgfsys@useobject{currentmarker}{}%
\end{pgfscope}%
\end{pgfscope}%
\begin{pgfscope}%
\definecolor{textcolor}{rgb}{0.000000,0.000000,0.000000}%
\pgfsetstrokecolor{textcolor}%
\pgfsetfillcolor{textcolor}%
\pgftext[x=0.253555in, y=0.845916in, left, base]{\color{textcolor}\rmfamily\fontsize{12.000000}{14.400000}\selectfont \(\displaystyle {25}\)}%
\end{pgfscope}%
\begin{pgfscope}%
\pgfsetbuttcap%
\pgfsetroundjoin%
\definecolor{currentfill}{rgb}{0.000000,0.000000,0.000000}%
\pgfsetfillcolor{currentfill}%
\pgfsetlinewidth{0.803000pt}%
\definecolor{currentstroke}{rgb}{0.000000,0.000000,0.000000}%
\pgfsetstrokecolor{currentstroke}%
\pgfsetdash{}{0pt}%
\pgfsys@defobject{currentmarker}{\pgfqpoint{-0.048611in}{0.000000in}}{\pgfqpoint{-0.000000in}{0.000000in}}{%
\pgfpathmoveto{\pgfqpoint{-0.000000in}{0.000000in}}%
\pgfpathlineto{\pgfqpoint{-0.048611in}{0.000000in}}%
\pgfusepath{stroke,fill}%
}%
\begin{pgfscope}%
\pgfsys@transformshift{0.513970in}{1.308721in}%
\pgfsys@useobject{currentmarker}{}%
\end{pgfscope}%
\end{pgfscope}%
\begin{pgfscope}%
\definecolor{textcolor}{rgb}{0.000000,0.000000,0.000000}%
\pgfsetstrokecolor{textcolor}%
\pgfsetfillcolor{textcolor}%
\pgftext[x=0.253555in, y=1.250887in, left, base]{\color{textcolor}\rmfamily\fontsize{12.000000}{14.400000}\selectfont \(\displaystyle {50}\)}%
\end{pgfscope}%
\begin{pgfscope}%
\definecolor{textcolor}{rgb}{0.000000,0.000000,0.000000}%
\pgfsetstrokecolor{textcolor}%
\pgfsetfillcolor{textcolor}%
\pgftext[x=0.198000in,y=1.076277in,,bottom,rotate=90.000000]{\color{textcolor}\rmfamily\fontsize{12.000000}{14.400000}\selectfont Percent of Dataset}%
\end{pgfscope}%
\begin{pgfscope}%
\pgfsetrectcap%
\pgfsetmiterjoin%
\pgfsetlinewidth{0.803000pt}%
\definecolor{currentstroke}{rgb}{0.000000,0.000000,0.000000}%
\pgfsetstrokecolor{currentstroke}%
\pgfsetdash{}{0pt}%
\pgfpathmoveto{\pgfqpoint{0.513970in}{0.498777in}}%
\pgfpathlineto{\pgfqpoint{0.513970in}{1.653777in}}%
\pgfusepath{stroke}%
\end{pgfscope}%
\begin{pgfscope}%
\pgfsetrectcap%
\pgfsetmiterjoin%
\pgfsetlinewidth{0.803000pt}%
\definecolor{currentstroke}{rgb}{0.000000,0.000000,0.000000}%
\pgfsetstrokecolor{currentstroke}%
\pgfsetdash{}{0pt}%
\pgfpathmoveto{\pgfqpoint{2.063970in}{0.498777in}}%
\pgfpathlineto{\pgfqpoint{2.063970in}{1.653777in}}%
\pgfusepath{stroke}%
\end{pgfscope}%
\begin{pgfscope}%
\pgfsetrectcap%
\pgfsetmiterjoin%
\pgfsetlinewidth{0.803000pt}%
\definecolor{currentstroke}{rgb}{0.000000,0.000000,0.000000}%
\pgfsetstrokecolor{currentstroke}%
\pgfsetdash{}{0pt}%
\pgfpathmoveto{\pgfqpoint{0.513970in}{0.498777in}}%
\pgfpathlineto{\pgfqpoint{2.063970in}{0.498777in}}%
\pgfusepath{stroke}%
\end{pgfscope}%
\begin{pgfscope}%
\pgfsetrectcap%
\pgfsetmiterjoin%
\pgfsetlinewidth{0.803000pt}%
\definecolor{currentstroke}{rgb}{0.000000,0.000000,0.000000}%
\pgfsetstrokecolor{currentstroke}%
\pgfsetdash{}{0pt}%
\pgfpathmoveto{\pgfqpoint{0.513970in}{1.653777in}}%
\pgfpathlineto{\pgfqpoint{2.063970in}{1.653777in}}%
\pgfusepath{stroke}%
\end{pgfscope}%
\begin{pgfscope}%
\pgfsetbuttcap%
\pgfsetmiterjoin%
\definecolor{currentfill}{rgb}{1.000000,1.000000,1.000000}%
\pgfsetfillcolor{currentfill}%
\pgfsetfillopacity{0.800000}%
\pgfsetlinewidth{1.003750pt}%
\definecolor{currentstroke}{rgb}{0.800000,0.800000,0.800000}%
\pgfsetstrokecolor{currentstroke}%
\pgfsetstrokeopacity{0.800000}%
\pgfsetdash{}{0pt}%
\pgfpathmoveto{\pgfqpoint{1.137470in}{1.053944in}}%
\pgfpathlineto{\pgfqpoint{1.947304in}{1.053944in}}%
\pgfpathquadraticcurveto{\pgfqpoint{1.980637in}{1.053944in}}{\pgfqpoint{1.980637in}{1.087278in}}%
\pgfpathlineto{\pgfqpoint{1.980637in}{1.537111in}}%
\pgfpathquadraticcurveto{\pgfqpoint{1.980637in}{1.570444in}}{\pgfqpoint{1.947304in}{1.570444in}}%
\pgfpathlineto{\pgfqpoint{1.137470in}{1.570444in}}%
\pgfpathquadraticcurveto{\pgfqpoint{1.104137in}{1.570444in}}{\pgfqpoint{1.104137in}{1.537111in}}%
\pgfpathlineto{\pgfqpoint{1.104137in}{1.087278in}}%
\pgfpathquadraticcurveto{\pgfqpoint{1.104137in}{1.053944in}}{\pgfqpoint{1.137470in}{1.053944in}}%
\pgfpathlineto{\pgfqpoint{1.137470in}{1.053944in}}%
\pgfpathclose%
\pgfusepath{stroke,fill}%
\end{pgfscope}%
\begin{pgfscope}%
\pgfsetbuttcap%
\pgfsetmiterjoin%
\pgfsetlinewidth{1.003750pt}%
\definecolor{currentstroke}{rgb}{0.000000,0.000000,0.000000}%
\pgfsetstrokecolor{currentstroke}%
\pgfsetdash{}{0pt}%
\pgfpathmoveto{\pgfqpoint{1.170804in}{1.387111in}}%
\pgfpathlineto{\pgfqpoint{1.504137in}{1.387111in}}%
\pgfpathlineto{\pgfqpoint{1.504137in}{1.503777in}}%
\pgfpathlineto{\pgfqpoint{1.170804in}{1.503777in}}%
\pgfpathlineto{\pgfqpoint{1.170804in}{1.387111in}}%
\pgfpathclose%
\pgfusepath{stroke}%
\end{pgfscope}%
\begin{pgfscope}%
\definecolor{textcolor}{rgb}{0.000000,0.000000,0.000000}%
\pgfsetstrokecolor{textcolor}%
\pgfsetfillcolor{textcolor}%
\pgftext[x=1.637470in,y=1.387111in,left,base]{\color{textcolor}\rmfamily\fontsize{12.000000}{14.400000}\selectfont Neg}%
\end{pgfscope}%
\begin{pgfscope}%
\pgfsetbuttcap%
\pgfsetmiterjoin%
\definecolor{currentfill}{rgb}{0.000000,0.000000,0.000000}%
\pgfsetfillcolor{currentfill}%
\pgfsetlinewidth{0.000000pt}%
\definecolor{currentstroke}{rgb}{0.000000,0.000000,0.000000}%
\pgfsetstrokecolor{currentstroke}%
\pgfsetstrokeopacity{0.000000}%
\pgfsetdash{}{0pt}%
\pgfpathmoveto{\pgfqpoint{1.170804in}{1.152944in}}%
\pgfpathlineto{\pgfqpoint{1.504137in}{1.152944in}}%
\pgfpathlineto{\pgfqpoint{1.504137in}{1.269611in}}%
\pgfpathlineto{\pgfqpoint{1.170804in}{1.269611in}}%
\pgfpathlineto{\pgfqpoint{1.170804in}{1.152944in}}%
\pgfpathclose%
\pgfusepath{fill}%
\end{pgfscope}%
\begin{pgfscope}%
\definecolor{textcolor}{rgb}{0.000000,0.000000,0.000000}%
\pgfsetstrokecolor{textcolor}%
\pgfsetfillcolor{textcolor}%
\pgftext[x=1.637470in,y=1.152944in,left,base]{\color{textcolor}\rmfamily\fontsize{12.000000}{14.400000}\selectfont Pos}%
\end{pgfscope}%
\end{pgfpicture}%
\makeatother%
\endgroup%

%  &
%  \vspace{0pt} %% Creator: Matplotlib, PGF backend
%%
%% To include the figure in your LaTeX document, write
%%   \input{<filename>.pgf}
%%
%% Make sure the required packages are loaded in your preamble
%%   \usepackage{pgf}
%%
%% Also ensure that all the required font packages are loaded; for instance,
%% the lmodern package is sometimes necessary when using math font.
%%   \usepackage{lmodern}
%%
%% Figures using additional raster images can only be included by \input if
%% they are in the same directory as the main LaTeX file. For loading figures
%% from other directories you can use the `import` package
%%   \usepackage{import}
%%
%% and then include the figures with
%%   \import{<path to file>}{<filename>.pgf}
%%
%% Matplotlib used the following preamble
%%   
%%   \usepackage{fontspec}
%%   \makeatletter\@ifpackageloaded{underscore}{}{\usepackage[strings]{underscore}}\makeatother
%%
\begingroup%
\makeatletter%
\begin{pgfpicture}%
\pgfpathrectangle{\pgfpointorigin}{\pgfqpoint{2.221861in}{1.754444in}}%
\pgfusepath{use as bounding box, clip}%
\begin{pgfscope}%
\pgfsetbuttcap%
\pgfsetmiterjoin%
\definecolor{currentfill}{rgb}{1.000000,1.000000,1.000000}%
\pgfsetfillcolor{currentfill}%
\pgfsetlinewidth{0.000000pt}%
\definecolor{currentstroke}{rgb}{1.000000,1.000000,1.000000}%
\pgfsetstrokecolor{currentstroke}%
\pgfsetdash{}{0pt}%
\pgfpathmoveto{\pgfqpoint{0.000000in}{0.000000in}}%
\pgfpathlineto{\pgfqpoint{2.221861in}{0.000000in}}%
\pgfpathlineto{\pgfqpoint{2.221861in}{1.754444in}}%
\pgfpathlineto{\pgfqpoint{0.000000in}{1.754444in}}%
\pgfpathlineto{\pgfqpoint{0.000000in}{0.000000in}}%
\pgfpathclose%
\pgfusepath{fill}%
\end{pgfscope}%
\begin{pgfscope}%
\pgfsetbuttcap%
\pgfsetmiterjoin%
\definecolor{currentfill}{rgb}{1.000000,1.000000,1.000000}%
\pgfsetfillcolor{currentfill}%
\pgfsetlinewidth{0.000000pt}%
\definecolor{currentstroke}{rgb}{0.000000,0.000000,0.000000}%
\pgfsetstrokecolor{currentstroke}%
\pgfsetstrokeopacity{0.000000}%
\pgfsetdash{}{0pt}%
\pgfpathmoveto{\pgfqpoint{0.553581in}{0.499444in}}%
\pgfpathlineto{\pgfqpoint{2.103581in}{0.499444in}}%
\pgfpathlineto{\pgfqpoint{2.103581in}{1.654444in}}%
\pgfpathlineto{\pgfqpoint{0.553581in}{1.654444in}}%
\pgfpathlineto{\pgfqpoint{0.553581in}{0.499444in}}%
\pgfpathclose%
\pgfusepath{fill}%
\end{pgfscope}%
\begin{pgfscope}%
\pgfsetbuttcap%
\pgfsetroundjoin%
\definecolor{currentfill}{rgb}{0.000000,0.000000,0.000000}%
\pgfsetfillcolor{currentfill}%
\pgfsetlinewidth{0.803000pt}%
\definecolor{currentstroke}{rgb}{0.000000,0.000000,0.000000}%
\pgfsetstrokecolor{currentstroke}%
\pgfsetdash{}{0pt}%
\pgfsys@defobject{currentmarker}{\pgfqpoint{0.000000in}{-0.048611in}}{\pgfqpoint{0.000000in}{0.000000in}}{%
\pgfpathmoveto{\pgfqpoint{0.000000in}{0.000000in}}%
\pgfpathlineto{\pgfqpoint{0.000000in}{-0.048611in}}%
\pgfusepath{stroke,fill}%
}%
\begin{pgfscope}%
\pgfsys@transformshift{0.624035in}{0.499444in}%
\pgfsys@useobject{currentmarker}{}%
\end{pgfscope}%
\end{pgfscope}%
\begin{pgfscope}%
\definecolor{textcolor}{rgb}{0.000000,0.000000,0.000000}%
\pgfsetstrokecolor{textcolor}%
\pgfsetfillcolor{textcolor}%
\pgftext[x=0.624035in,y=0.402222in,,top]{\color{textcolor}\rmfamily\fontsize{10.000000}{12.000000}\selectfont \(\displaystyle {0.0}\)}%
\end{pgfscope}%
\begin{pgfscope}%
\pgfsetbuttcap%
\pgfsetroundjoin%
\definecolor{currentfill}{rgb}{0.000000,0.000000,0.000000}%
\pgfsetfillcolor{currentfill}%
\pgfsetlinewidth{0.803000pt}%
\definecolor{currentstroke}{rgb}{0.000000,0.000000,0.000000}%
\pgfsetstrokecolor{currentstroke}%
\pgfsetdash{}{0pt}%
\pgfsys@defobject{currentmarker}{\pgfqpoint{0.000000in}{-0.048611in}}{\pgfqpoint{0.000000in}{0.000000in}}{%
\pgfpathmoveto{\pgfqpoint{0.000000in}{0.000000in}}%
\pgfpathlineto{\pgfqpoint{0.000000in}{-0.048611in}}%
\pgfusepath{stroke,fill}%
}%
\begin{pgfscope}%
\pgfsys@transformshift{1.328581in}{0.499444in}%
\pgfsys@useobject{currentmarker}{}%
\end{pgfscope}%
\end{pgfscope}%
\begin{pgfscope}%
\definecolor{textcolor}{rgb}{0.000000,0.000000,0.000000}%
\pgfsetstrokecolor{textcolor}%
\pgfsetfillcolor{textcolor}%
\pgftext[x=1.328581in,y=0.402222in,,top]{\color{textcolor}\rmfamily\fontsize{10.000000}{12.000000}\selectfont \(\displaystyle {0.5}\)}%
\end{pgfscope}%
\begin{pgfscope}%
\pgfsetbuttcap%
\pgfsetroundjoin%
\definecolor{currentfill}{rgb}{0.000000,0.000000,0.000000}%
\pgfsetfillcolor{currentfill}%
\pgfsetlinewidth{0.803000pt}%
\definecolor{currentstroke}{rgb}{0.000000,0.000000,0.000000}%
\pgfsetstrokecolor{currentstroke}%
\pgfsetdash{}{0pt}%
\pgfsys@defobject{currentmarker}{\pgfqpoint{0.000000in}{-0.048611in}}{\pgfqpoint{0.000000in}{0.000000in}}{%
\pgfpathmoveto{\pgfqpoint{0.000000in}{0.000000in}}%
\pgfpathlineto{\pgfqpoint{0.000000in}{-0.048611in}}%
\pgfusepath{stroke,fill}%
}%
\begin{pgfscope}%
\pgfsys@transformshift{2.033126in}{0.499444in}%
\pgfsys@useobject{currentmarker}{}%
\end{pgfscope}%
\end{pgfscope}%
\begin{pgfscope}%
\definecolor{textcolor}{rgb}{0.000000,0.000000,0.000000}%
\pgfsetstrokecolor{textcolor}%
\pgfsetfillcolor{textcolor}%
\pgftext[x=2.033126in,y=0.402222in,,top]{\color{textcolor}\rmfamily\fontsize{10.000000}{12.000000}\selectfont \(\displaystyle {1.0}\)}%
\end{pgfscope}%
\begin{pgfscope}%
\definecolor{textcolor}{rgb}{0.000000,0.000000,0.000000}%
\pgfsetstrokecolor{textcolor}%
\pgfsetfillcolor{textcolor}%
\pgftext[x=1.328581in,y=0.223333in,,top]{\color{textcolor}\rmfamily\fontsize{10.000000}{12.000000}\selectfont False positive rate}%
\end{pgfscope}%
\begin{pgfscope}%
\pgfsetbuttcap%
\pgfsetroundjoin%
\definecolor{currentfill}{rgb}{0.000000,0.000000,0.000000}%
\pgfsetfillcolor{currentfill}%
\pgfsetlinewidth{0.803000pt}%
\definecolor{currentstroke}{rgb}{0.000000,0.000000,0.000000}%
\pgfsetstrokecolor{currentstroke}%
\pgfsetdash{}{0pt}%
\pgfsys@defobject{currentmarker}{\pgfqpoint{-0.048611in}{0.000000in}}{\pgfqpoint{-0.000000in}{0.000000in}}{%
\pgfpathmoveto{\pgfqpoint{-0.000000in}{0.000000in}}%
\pgfpathlineto{\pgfqpoint{-0.048611in}{0.000000in}}%
\pgfusepath{stroke,fill}%
}%
\begin{pgfscope}%
\pgfsys@transformshift{0.553581in}{0.551944in}%
\pgfsys@useobject{currentmarker}{}%
\end{pgfscope}%
\end{pgfscope}%
\begin{pgfscope}%
\definecolor{textcolor}{rgb}{0.000000,0.000000,0.000000}%
\pgfsetstrokecolor{textcolor}%
\pgfsetfillcolor{textcolor}%
\pgftext[x=0.278889in, y=0.503750in, left, base]{\color{textcolor}\rmfamily\fontsize{10.000000}{12.000000}\selectfont \(\displaystyle {0.0}\)}%
\end{pgfscope}%
\begin{pgfscope}%
\pgfsetbuttcap%
\pgfsetroundjoin%
\definecolor{currentfill}{rgb}{0.000000,0.000000,0.000000}%
\pgfsetfillcolor{currentfill}%
\pgfsetlinewidth{0.803000pt}%
\definecolor{currentstroke}{rgb}{0.000000,0.000000,0.000000}%
\pgfsetstrokecolor{currentstroke}%
\pgfsetdash{}{0pt}%
\pgfsys@defobject{currentmarker}{\pgfqpoint{-0.048611in}{0.000000in}}{\pgfqpoint{-0.000000in}{0.000000in}}{%
\pgfpathmoveto{\pgfqpoint{-0.000000in}{0.000000in}}%
\pgfpathlineto{\pgfqpoint{-0.048611in}{0.000000in}}%
\pgfusepath{stroke,fill}%
}%
\begin{pgfscope}%
\pgfsys@transformshift{0.553581in}{1.076944in}%
\pgfsys@useobject{currentmarker}{}%
\end{pgfscope}%
\end{pgfscope}%
\begin{pgfscope}%
\definecolor{textcolor}{rgb}{0.000000,0.000000,0.000000}%
\pgfsetstrokecolor{textcolor}%
\pgfsetfillcolor{textcolor}%
\pgftext[x=0.278889in, y=1.028750in, left, base]{\color{textcolor}\rmfamily\fontsize{10.000000}{12.000000}\selectfont \(\displaystyle {0.5}\)}%
\end{pgfscope}%
\begin{pgfscope}%
\pgfsetbuttcap%
\pgfsetroundjoin%
\definecolor{currentfill}{rgb}{0.000000,0.000000,0.000000}%
\pgfsetfillcolor{currentfill}%
\pgfsetlinewidth{0.803000pt}%
\definecolor{currentstroke}{rgb}{0.000000,0.000000,0.000000}%
\pgfsetstrokecolor{currentstroke}%
\pgfsetdash{}{0pt}%
\pgfsys@defobject{currentmarker}{\pgfqpoint{-0.048611in}{0.000000in}}{\pgfqpoint{-0.000000in}{0.000000in}}{%
\pgfpathmoveto{\pgfqpoint{-0.000000in}{0.000000in}}%
\pgfpathlineto{\pgfqpoint{-0.048611in}{0.000000in}}%
\pgfusepath{stroke,fill}%
}%
\begin{pgfscope}%
\pgfsys@transformshift{0.553581in}{1.601944in}%
\pgfsys@useobject{currentmarker}{}%
\end{pgfscope}%
\end{pgfscope}%
\begin{pgfscope}%
\definecolor{textcolor}{rgb}{0.000000,0.000000,0.000000}%
\pgfsetstrokecolor{textcolor}%
\pgfsetfillcolor{textcolor}%
\pgftext[x=0.278889in, y=1.553750in, left, base]{\color{textcolor}\rmfamily\fontsize{10.000000}{12.000000}\selectfont \(\displaystyle {1.0}\)}%
\end{pgfscope}%
\begin{pgfscope}%
\definecolor{textcolor}{rgb}{0.000000,0.000000,0.000000}%
\pgfsetstrokecolor{textcolor}%
\pgfsetfillcolor{textcolor}%
\pgftext[x=0.223333in,y=1.076944in,,bottom,rotate=90.000000]{\color{textcolor}\rmfamily\fontsize{10.000000}{12.000000}\selectfont True positive rate}%
\end{pgfscope}%
\begin{pgfscope}%
\pgfpathrectangle{\pgfqpoint{0.553581in}{0.499444in}}{\pgfqpoint{1.550000in}{1.155000in}}%
\pgfusepath{clip}%
\pgfsetbuttcap%
\pgfsetroundjoin%
\pgfsetlinewidth{1.505625pt}%
\definecolor{currentstroke}{rgb}{0.000000,0.000000,0.000000}%
\pgfsetstrokecolor{currentstroke}%
\pgfsetdash{{5.550000pt}{2.400000pt}}{0.000000pt}%
\pgfpathmoveto{\pgfqpoint{0.624035in}{0.551944in}}%
\pgfpathlineto{\pgfqpoint{2.033126in}{1.601944in}}%
\pgfusepath{stroke}%
\end{pgfscope}%
\begin{pgfscope}%
\pgfpathrectangle{\pgfqpoint{0.553581in}{0.499444in}}{\pgfqpoint{1.550000in}{1.155000in}}%
\pgfusepath{clip}%
\pgfsetrectcap%
\pgfsetroundjoin%
\pgfsetlinewidth{1.505625pt}%
\definecolor{currentstroke}{rgb}{0.000000,0.000000,0.000000}%
\pgfsetstrokecolor{currentstroke}%
\pgfsetdash{}{0pt}%
\pgfpathmoveto{\pgfqpoint{0.624035in}{0.551944in}}%
\pgfpathlineto{\pgfqpoint{0.626207in}{0.552574in}}%
\pgfpathlineto{\pgfqpoint{0.627318in}{0.561464in}}%
\pgfpathlineto{\pgfqpoint{0.628014in}{0.562514in}}%
\pgfpathlineto{\pgfqpoint{0.629125in}{0.563634in}}%
\pgfpathlineto{\pgfqpoint{0.629605in}{0.564614in}}%
\pgfpathlineto{\pgfqpoint{0.630699in}{0.567694in}}%
\pgfpathlineto{\pgfqpoint{0.631130in}{0.568604in}}%
\pgfpathlineto{\pgfqpoint{0.632225in}{0.573294in}}%
\pgfpathlineto{\pgfqpoint{0.632772in}{0.574064in}}%
\pgfpathlineto{\pgfqpoint{0.633866in}{0.579944in}}%
\pgfpathlineto{\pgfqpoint{0.634247in}{0.580854in}}%
\pgfpathlineto{\pgfqpoint{0.635358in}{0.585824in}}%
\pgfpathlineto{\pgfqpoint{0.635540in}{0.586874in}}%
\pgfpathlineto{\pgfqpoint{0.636634in}{0.593244in}}%
\pgfpathlineto{\pgfqpoint{0.636833in}{0.594154in}}%
\pgfpathlineto{\pgfqpoint{0.637944in}{0.600734in}}%
\pgfpathlineto{\pgfqpoint{0.638126in}{0.601784in}}%
\pgfpathlineto{\pgfqpoint{0.639187in}{0.608714in}}%
\pgfpathlineto{\pgfqpoint{0.639353in}{0.609694in}}%
\pgfpathlineto{\pgfqpoint{0.640464in}{0.616834in}}%
\pgfpathlineto{\pgfqpoint{0.640712in}{0.617884in}}%
\pgfpathlineto{\pgfqpoint{0.641806in}{0.625864in}}%
\pgfpathlineto{\pgfqpoint{0.642038in}{0.626844in}}%
\pgfpathlineto{\pgfqpoint{0.643149in}{0.635804in}}%
\pgfpathlineto{\pgfqpoint{0.643348in}{0.636854in}}%
\pgfpathlineto{\pgfqpoint{0.644392in}{0.645044in}}%
\pgfpathlineto{\pgfqpoint{0.644641in}{0.645884in}}%
\pgfpathlineto{\pgfqpoint{0.645752in}{0.652394in}}%
\pgfpathlineto{\pgfqpoint{0.645868in}{0.653164in}}%
\pgfpathlineto{\pgfqpoint{0.646979in}{0.661424in}}%
\pgfpathlineto{\pgfqpoint{0.647327in}{0.662264in}}%
\pgfpathlineto{\pgfqpoint{0.648437in}{0.668914in}}%
\pgfpathlineto{\pgfqpoint{0.648587in}{0.669824in}}%
\pgfpathlineto{\pgfqpoint{0.649697in}{0.678294in}}%
\pgfpathlineto{\pgfqpoint{0.649813in}{0.679274in}}%
\pgfpathlineto{\pgfqpoint{0.650924in}{0.687044in}}%
\pgfpathlineto{\pgfqpoint{0.651140in}{0.687884in}}%
\pgfpathlineto{\pgfqpoint{0.652250in}{0.696144in}}%
\pgfpathlineto{\pgfqpoint{0.652333in}{0.697194in}}%
\pgfpathlineto{\pgfqpoint{0.653444in}{0.706434in}}%
\pgfpathlineto{\pgfqpoint{0.653858in}{0.707414in}}%
\pgfpathlineto{\pgfqpoint{0.654969in}{0.716584in}}%
\pgfpathlineto{\pgfqpoint{0.655151in}{0.717564in}}%
\pgfpathlineto{\pgfqpoint{0.656262in}{0.725194in}}%
\pgfpathlineto{\pgfqpoint{0.656444in}{0.726174in}}%
\pgfpathlineto{\pgfqpoint{0.657538in}{0.733244in}}%
\pgfpathlineto{\pgfqpoint{0.657704in}{0.734294in}}%
\pgfpathlineto{\pgfqpoint{0.658815in}{0.743464in}}%
\pgfpathlineto{\pgfqpoint{0.659064in}{0.744514in}}%
\pgfpathlineto{\pgfqpoint{0.660158in}{0.751164in}}%
\pgfpathlineto{\pgfqpoint{0.660290in}{0.751794in}}%
\pgfpathlineto{\pgfqpoint{0.661401in}{0.759354in}}%
\pgfpathlineto{\pgfqpoint{0.661633in}{0.760404in}}%
\pgfpathlineto{\pgfqpoint{0.662744in}{0.769784in}}%
\pgfpathlineto{\pgfqpoint{0.662827in}{0.770344in}}%
\pgfpathlineto{\pgfqpoint{0.663937in}{0.777764in}}%
\pgfpathlineto{\pgfqpoint{0.664120in}{0.778814in}}%
\pgfpathlineto{\pgfqpoint{0.665181in}{0.783854in}}%
\pgfpathlineto{\pgfqpoint{0.665297in}{0.784834in}}%
\pgfpathlineto{\pgfqpoint{0.666407in}{0.795614in}}%
\pgfpathlineto{\pgfqpoint{0.666656in}{0.796664in}}%
\pgfpathlineto{\pgfqpoint{0.667767in}{0.801564in}}%
\pgfpathlineto{\pgfqpoint{0.667899in}{0.802404in}}%
\pgfpathlineto{\pgfqpoint{0.669010in}{0.809334in}}%
\pgfpathlineto{\pgfqpoint{0.669342in}{0.810384in}}%
\pgfpathlineto{\pgfqpoint{0.670452in}{0.818084in}}%
\pgfpathlineto{\pgfqpoint{0.670701in}{0.819134in}}%
\pgfpathlineto{\pgfqpoint{0.671812in}{0.825364in}}%
\pgfpathlineto{\pgfqpoint{0.671911in}{0.826414in}}%
\pgfpathlineto{\pgfqpoint{0.673022in}{0.833694in}}%
\pgfpathlineto{\pgfqpoint{0.673138in}{0.834744in}}%
\pgfpathlineto{\pgfqpoint{0.674215in}{0.838944in}}%
\pgfpathlineto{\pgfqpoint{0.674530in}{0.839994in}}%
\pgfpathlineto{\pgfqpoint{0.675608in}{0.846014in}}%
\pgfpathlineto{\pgfqpoint{0.675790in}{0.846994in}}%
\pgfpathlineto{\pgfqpoint{0.676901in}{0.853854in}}%
\pgfpathlineto{\pgfqpoint{0.677199in}{0.854834in}}%
\pgfpathlineto{\pgfqpoint{0.678310in}{0.859034in}}%
\pgfpathlineto{\pgfqpoint{0.678492in}{0.860084in}}%
\pgfpathlineto{\pgfqpoint{0.679587in}{0.867014in}}%
\pgfpathlineto{\pgfqpoint{0.679868in}{0.867924in}}%
\pgfpathlineto{\pgfqpoint{0.680963in}{0.874504in}}%
\pgfpathlineto{\pgfqpoint{0.681228in}{0.875484in}}%
\pgfpathlineto{\pgfqpoint{0.682338in}{0.882204in}}%
\pgfpathlineto{\pgfqpoint{0.682587in}{0.883254in}}%
\pgfpathlineto{\pgfqpoint{0.683681in}{0.889484in}}%
\pgfpathlineto{\pgfqpoint{0.683980in}{0.890394in}}%
\pgfpathlineto{\pgfqpoint{0.685074in}{0.897394in}}%
\pgfpathlineto{\pgfqpoint{0.685389in}{0.898304in}}%
\pgfpathlineto{\pgfqpoint{0.686450in}{0.904184in}}%
\pgfpathlineto{\pgfqpoint{0.686748in}{0.905094in}}%
\pgfpathlineto{\pgfqpoint{0.687859in}{0.910204in}}%
\pgfpathlineto{\pgfqpoint{0.688074in}{0.911114in}}%
\pgfpathlineto{\pgfqpoint{0.689185in}{0.915034in}}%
\pgfpathlineto{\pgfqpoint{0.689400in}{0.916084in}}%
\pgfpathlineto{\pgfqpoint{0.690511in}{0.922664in}}%
\pgfpathlineto{\pgfqpoint{0.690793in}{0.923714in}}%
\pgfpathlineto{\pgfqpoint{0.691887in}{0.928754in}}%
\pgfpathlineto{\pgfqpoint{0.692318in}{0.929734in}}%
\pgfpathlineto{\pgfqpoint{0.693412in}{0.935334in}}%
\pgfpathlineto{\pgfqpoint{0.693611in}{0.936384in}}%
\pgfpathlineto{\pgfqpoint{0.694705in}{0.940864in}}%
\pgfpathlineto{\pgfqpoint{0.695053in}{0.941914in}}%
\pgfpathlineto{\pgfqpoint{0.696164in}{0.948634in}}%
\pgfpathlineto{\pgfqpoint{0.696645in}{0.949684in}}%
\pgfpathlineto{\pgfqpoint{0.697756in}{0.954164in}}%
\pgfpathlineto{\pgfqpoint{0.697921in}{0.955144in}}%
\pgfpathlineto{\pgfqpoint{0.699015in}{0.960464in}}%
\pgfpathlineto{\pgfqpoint{0.699314in}{0.961304in}}%
\pgfpathlineto{\pgfqpoint{0.700425in}{0.965294in}}%
\pgfpathlineto{\pgfqpoint{0.700657in}{0.966344in}}%
\pgfpathlineto{\pgfqpoint{0.701767in}{0.971314in}}%
\pgfpathlineto{\pgfqpoint{0.702049in}{0.972364in}}%
\pgfpathlineto{\pgfqpoint{0.703094in}{0.977194in}}%
\pgfpathlineto{\pgfqpoint{0.703442in}{0.978104in}}%
\pgfpathlineto{\pgfqpoint{0.704536in}{0.982514in}}%
\pgfpathlineto{\pgfqpoint{0.704917in}{0.983564in}}%
\pgfpathlineto{\pgfqpoint{0.706011in}{0.988464in}}%
\pgfpathlineto{\pgfqpoint{0.706160in}{0.989444in}}%
\pgfpathlineto{\pgfqpoint{0.707271in}{0.993504in}}%
\pgfpathlineto{\pgfqpoint{0.707619in}{0.994414in}}%
\pgfpathlineto{\pgfqpoint{0.708697in}{0.998054in}}%
\pgfpathlineto{\pgfqpoint{0.709061in}{0.999104in}}%
\pgfpathlineto{\pgfqpoint{0.710172in}{1.004564in}}%
\pgfpathlineto{\pgfqpoint{0.710504in}{1.005614in}}%
\pgfpathlineto{\pgfqpoint{0.711614in}{1.009604in}}%
\pgfpathlineto{\pgfqpoint{0.711830in}{1.010584in}}%
\pgfpathlineto{\pgfqpoint{0.712874in}{1.015134in}}%
\pgfpathlineto{\pgfqpoint{0.713123in}{1.016184in}}%
\pgfpathlineto{\pgfqpoint{0.714184in}{1.020244in}}%
\pgfpathlineto{\pgfqpoint{0.714598in}{1.021294in}}%
\pgfpathlineto{\pgfqpoint{0.715709in}{1.024304in}}%
\pgfpathlineto{\pgfqpoint{0.716173in}{1.025354in}}%
\pgfpathlineto{\pgfqpoint{0.717284in}{1.029624in}}%
\pgfpathlineto{\pgfqpoint{0.717466in}{1.030604in}}%
\pgfpathlineto{\pgfqpoint{0.718560in}{1.035014in}}%
\pgfpathlineto{\pgfqpoint{0.718776in}{1.035994in}}%
\pgfpathlineto{\pgfqpoint{0.719853in}{1.041034in}}%
\pgfpathlineto{\pgfqpoint{0.720202in}{1.042084in}}%
\pgfpathlineto{\pgfqpoint{0.721312in}{1.046144in}}%
\pgfpathlineto{\pgfqpoint{0.721694in}{1.047194in}}%
\pgfpathlineto{\pgfqpoint{0.722788in}{1.050904in}}%
\pgfpathlineto{\pgfqpoint{0.723086in}{1.051954in}}%
\pgfpathlineto{\pgfqpoint{0.724180in}{1.056364in}}%
\pgfpathlineto{\pgfqpoint{0.724727in}{1.057414in}}%
\pgfpathlineto{\pgfqpoint{0.725805in}{1.061264in}}%
\pgfpathlineto{\pgfqpoint{0.726236in}{1.062314in}}%
\pgfpathlineto{\pgfqpoint{0.727280in}{1.065744in}}%
\pgfpathlineto{\pgfqpoint{0.727728in}{1.066794in}}%
\pgfpathlineto{\pgfqpoint{0.728805in}{1.069874in}}%
\pgfpathlineto{\pgfqpoint{0.729253in}{1.070924in}}%
\pgfpathlineto{\pgfqpoint{0.730364in}{1.074144in}}%
\pgfpathlineto{\pgfqpoint{0.730695in}{1.075124in}}%
\pgfpathlineto{\pgfqpoint{0.731789in}{1.079044in}}%
\pgfpathlineto{\pgfqpoint{0.732336in}{1.080024in}}%
\pgfpathlineto{\pgfqpoint{0.733447in}{1.084574in}}%
\pgfpathlineto{\pgfqpoint{0.734094in}{1.085624in}}%
\pgfpathlineto{\pgfqpoint{0.735188in}{1.089544in}}%
\pgfpathlineto{\pgfqpoint{0.735552in}{1.090594in}}%
\pgfpathlineto{\pgfqpoint{0.736597in}{1.093674in}}%
\pgfpathlineto{\pgfqpoint{0.737177in}{1.094724in}}%
\pgfpathlineto{\pgfqpoint{0.738221in}{1.097524in}}%
\pgfpathlineto{\pgfqpoint{0.738702in}{1.098574in}}%
\pgfpathlineto{\pgfqpoint{0.739780in}{1.102424in}}%
\pgfpathlineto{\pgfqpoint{0.740128in}{1.103474in}}%
\pgfpathlineto{\pgfqpoint{0.741238in}{1.106624in}}%
\pgfpathlineto{\pgfqpoint{0.741636in}{1.107674in}}%
\pgfpathlineto{\pgfqpoint{0.742697in}{1.111034in}}%
\pgfpathlineto{\pgfqpoint{0.743095in}{1.112084in}}%
\pgfpathlineto{\pgfqpoint{0.744206in}{1.115024in}}%
\pgfpathlineto{\pgfqpoint{0.744769in}{1.116074in}}%
\pgfpathlineto{\pgfqpoint{0.745051in}{1.117054in}}%
\pgfpathlineto{\pgfqpoint{0.745068in}{1.117054in}}%
\pgfpathlineto{\pgfqpoint{0.755943in}{1.118104in}}%
\pgfpathlineto{\pgfqpoint{0.757053in}{1.121814in}}%
\pgfpathlineto{\pgfqpoint{0.757600in}{1.122864in}}%
\pgfpathlineto{\pgfqpoint{0.758678in}{1.125384in}}%
\pgfpathlineto{\pgfqpoint{0.759225in}{1.126434in}}%
\pgfpathlineto{\pgfqpoint{0.760269in}{1.129094in}}%
\pgfpathlineto{\pgfqpoint{0.760817in}{1.130144in}}%
\pgfpathlineto{\pgfqpoint{0.761911in}{1.133224in}}%
\pgfpathlineto{\pgfqpoint{0.762474in}{1.134274in}}%
\pgfpathlineto{\pgfqpoint{0.763535in}{1.137844in}}%
\pgfpathlineto{\pgfqpoint{0.764165in}{1.138824in}}%
\pgfpathlineto{\pgfqpoint{0.765259in}{1.142324in}}%
\pgfpathlineto{\pgfqpoint{0.766055in}{1.143374in}}%
\pgfpathlineto{\pgfqpoint{0.767166in}{1.146314in}}%
\pgfpathlineto{\pgfqpoint{0.767696in}{1.147364in}}%
\pgfpathlineto{\pgfqpoint{0.768790in}{1.149954in}}%
\pgfpathlineto{\pgfqpoint{0.769603in}{1.151004in}}%
\pgfpathlineto{\pgfqpoint{0.770514in}{1.153314in}}%
\pgfpathlineto{\pgfqpoint{0.771161in}{1.154364in}}%
\pgfpathlineto{\pgfqpoint{0.772272in}{1.157374in}}%
\pgfpathlineto{\pgfqpoint{0.772852in}{1.158424in}}%
\pgfpathlineto{\pgfqpoint{0.773946in}{1.160664in}}%
\pgfpathlineto{\pgfqpoint{0.774609in}{1.161714in}}%
\pgfpathlineto{\pgfqpoint{0.775703in}{1.164794in}}%
\pgfpathlineto{\pgfqpoint{0.776267in}{1.165774in}}%
\pgfpathlineto{\pgfqpoint{0.777344in}{1.168714in}}%
\pgfpathlineto{\pgfqpoint{0.777759in}{1.169764in}}%
\pgfpathlineto{\pgfqpoint{0.778836in}{1.172144in}}%
\pgfpathlineto{\pgfqpoint{0.779350in}{1.173124in}}%
\pgfpathlineto{\pgfqpoint{0.780444in}{1.174524in}}%
\pgfpathlineto{\pgfqpoint{0.781008in}{1.175574in}}%
\pgfpathlineto{\pgfqpoint{0.782052in}{1.177254in}}%
\pgfpathlineto{\pgfqpoint{0.782699in}{1.178304in}}%
\pgfpathlineto{\pgfqpoint{0.783793in}{1.180894in}}%
\pgfpathlineto{\pgfqpoint{0.784290in}{1.181874in}}%
\pgfpathlineto{\pgfqpoint{0.785384in}{1.184044in}}%
\pgfpathlineto{\pgfqpoint{0.785981in}{1.185094in}}%
\pgfpathlineto{\pgfqpoint{0.787075in}{1.186914in}}%
\pgfpathlineto{\pgfqpoint{0.787755in}{1.187894in}}%
\pgfpathlineto{\pgfqpoint{0.788866in}{1.189784in}}%
\pgfpathlineto{\pgfqpoint{0.789297in}{1.190764in}}%
\pgfpathlineto{\pgfqpoint{0.790407in}{1.192794in}}%
\pgfpathlineto{\pgfqpoint{0.790689in}{1.193634in}}%
\pgfpathlineto{\pgfqpoint{0.791783in}{1.196084in}}%
\pgfpathlineto{\pgfqpoint{0.792347in}{1.197134in}}%
\pgfpathlineto{\pgfqpoint{0.793358in}{1.199724in}}%
\pgfpathlineto{\pgfqpoint{0.794054in}{1.200774in}}%
\pgfpathlineto{\pgfqpoint{0.795099in}{1.201964in}}%
\pgfpathlineto{\pgfqpoint{0.796044in}{1.203014in}}%
\pgfpathlineto{\pgfqpoint{0.797138in}{1.205534in}}%
\pgfpathlineto{\pgfqpoint{0.797718in}{1.206584in}}%
\pgfpathlineto{\pgfqpoint{0.798812in}{1.208474in}}%
\pgfpathlineto{\pgfqpoint{0.799409in}{1.209524in}}%
\pgfpathlineto{\pgfqpoint{0.800520in}{1.212114in}}%
\pgfpathlineto{\pgfqpoint{0.801382in}{1.213094in}}%
\pgfpathlineto{\pgfqpoint{0.802492in}{1.215894in}}%
\pgfpathlineto{\pgfqpoint{0.803106in}{1.216664in}}%
\pgfpathlineto{\pgfqpoint{0.804134in}{1.218554in}}%
\pgfpathlineto{\pgfqpoint{0.805128in}{1.219604in}}%
\pgfpathlineto{\pgfqpoint{0.806222in}{1.221774in}}%
\pgfpathlineto{\pgfqpoint{0.806786in}{1.222614in}}%
\pgfpathlineto{\pgfqpoint{0.807897in}{1.225484in}}%
\pgfpathlineto{\pgfqpoint{0.808659in}{1.226534in}}%
\pgfpathlineto{\pgfqpoint{0.809753in}{1.228284in}}%
\pgfpathlineto{\pgfqpoint{0.810267in}{1.229334in}}%
\pgfpathlineto{\pgfqpoint{0.811378in}{1.231224in}}%
\pgfpathlineto{\pgfqpoint{0.812654in}{1.232274in}}%
\pgfpathlineto{\pgfqpoint{0.813682in}{1.233744in}}%
\pgfpathlineto{\pgfqpoint{0.814180in}{1.234794in}}%
\pgfpathlineto{\pgfqpoint{0.815290in}{1.236614in}}%
\pgfpathlineto{\pgfqpoint{0.816053in}{1.237664in}}%
\pgfpathlineto{\pgfqpoint{0.816948in}{1.239484in}}%
\pgfpathlineto{\pgfqpoint{0.818109in}{1.240464in}}%
\pgfpathlineto{\pgfqpoint{0.819203in}{1.242424in}}%
\pgfpathlineto{\pgfqpoint{0.819949in}{1.243404in}}%
\pgfpathlineto{\pgfqpoint{0.821043in}{1.245364in}}%
\pgfpathlineto{\pgfqpoint{0.821838in}{1.246414in}}%
\pgfpathlineto{\pgfqpoint{0.822899in}{1.248374in}}%
\pgfpathlineto{\pgfqpoint{0.823828in}{1.249424in}}%
\pgfpathlineto{\pgfqpoint{0.824905in}{1.251314in}}%
\pgfpathlineto{\pgfqpoint{0.826049in}{1.252364in}}%
\pgfpathlineto{\pgfqpoint{0.827110in}{1.254044in}}%
\pgfpathlineto{\pgfqpoint{0.828204in}{1.255024in}}%
\pgfpathlineto{\pgfqpoint{0.829298in}{1.257404in}}%
\pgfpathlineto{\pgfqpoint{0.830094in}{1.258454in}}%
\pgfpathlineto{\pgfqpoint{0.831188in}{1.259784in}}%
\pgfpathlineto{\pgfqpoint{0.831901in}{1.260834in}}%
\pgfpathlineto{\pgfqpoint{0.832979in}{1.261954in}}%
\pgfpathlineto{\pgfqpoint{0.833841in}{1.263004in}}%
\pgfpathlineto{\pgfqpoint{0.834769in}{1.263774in}}%
\pgfpathlineto{\pgfqpoint{0.835697in}{1.264824in}}%
\pgfpathlineto{\pgfqpoint{0.836659in}{1.266014in}}%
\pgfpathlineto{\pgfqpoint{0.838234in}{1.267064in}}%
\pgfpathlineto{\pgfqpoint{0.839295in}{1.268814in}}%
\pgfpathlineto{\pgfqpoint{0.840438in}{1.269864in}}%
\pgfpathlineto{\pgfqpoint{0.841549in}{1.271824in}}%
\pgfpathlineto{\pgfqpoint{0.842594in}{1.272874in}}%
\pgfpathlineto{\pgfqpoint{0.843688in}{1.274414in}}%
\pgfpathlineto{\pgfqpoint{0.844881in}{1.275464in}}%
\pgfpathlineto{\pgfqpoint{0.845992in}{1.277844in}}%
\pgfpathlineto{\pgfqpoint{0.846705in}{1.278894in}}%
\pgfpathlineto{\pgfqpoint{0.847633in}{1.280294in}}%
\pgfpathlineto{\pgfqpoint{0.848644in}{1.281274in}}%
\pgfpathlineto{\pgfqpoint{0.849606in}{1.282954in}}%
\pgfpathlineto{\pgfqpoint{0.850882in}{1.284004in}}%
\pgfpathlineto{\pgfqpoint{0.851960in}{1.285194in}}%
\pgfpathlineto{\pgfqpoint{0.853120in}{1.286244in}}%
\pgfpathlineto{\pgfqpoint{0.854198in}{1.287644in}}%
\pgfpathlineto{\pgfqpoint{0.855176in}{1.288694in}}%
\pgfpathlineto{\pgfqpoint{0.856237in}{1.290234in}}%
\pgfpathlineto{\pgfqpoint{0.856917in}{1.291144in}}%
\pgfpathlineto{\pgfqpoint{0.858027in}{1.292754in}}%
\pgfpathlineto{\pgfqpoint{0.859287in}{1.293804in}}%
\pgfpathlineto{\pgfqpoint{0.860232in}{1.294714in}}%
\pgfpathlineto{\pgfqpoint{0.861160in}{1.295764in}}%
\pgfpathlineto{\pgfqpoint{0.862155in}{1.296814in}}%
\pgfpathlineto{\pgfqpoint{0.863746in}{1.297794in}}%
\pgfpathlineto{\pgfqpoint{0.864807in}{1.299054in}}%
\pgfpathlineto{\pgfqpoint{0.866084in}{1.300104in}}%
\pgfpathlineto{\pgfqpoint{0.867128in}{1.301854in}}%
\pgfpathlineto{\pgfqpoint{0.867775in}{1.302904in}}%
\pgfpathlineto{\pgfqpoint{0.868886in}{1.304164in}}%
\pgfpathlineto{\pgfqpoint{0.869897in}{1.305144in}}%
\pgfpathlineto{\pgfqpoint{0.870991in}{1.306614in}}%
\pgfpathlineto{\pgfqpoint{0.872218in}{1.307664in}}%
\pgfpathlineto{\pgfqpoint{0.873312in}{1.309624in}}%
\pgfpathlineto{\pgfqpoint{0.874572in}{1.310674in}}%
\pgfpathlineto{\pgfqpoint{0.875616in}{1.312284in}}%
\pgfpathlineto{\pgfqpoint{0.876263in}{1.313334in}}%
\pgfpathlineto{\pgfqpoint{0.877373in}{1.314804in}}%
\pgfpathlineto{\pgfqpoint{0.878036in}{1.315854in}}%
\pgfpathlineto{\pgfqpoint{0.879081in}{1.316974in}}%
\pgfpathlineto{\pgfqpoint{0.880191in}{1.318024in}}%
\pgfpathlineto{\pgfqpoint{0.881252in}{1.319564in}}%
\pgfpathlineto{\pgfqpoint{0.882927in}{1.320544in}}%
\pgfpathlineto{\pgfqpoint{0.883954in}{1.321384in}}%
\pgfpathlineto{\pgfqpoint{0.885165in}{1.322434in}}%
\pgfpathlineto{\pgfqpoint{0.886226in}{1.323624in}}%
\pgfpathlineto{\pgfqpoint{0.887121in}{1.324674in}}%
\pgfpathlineto{\pgfqpoint{0.888215in}{1.325864in}}%
\pgfpathlineto{\pgfqpoint{0.889624in}{1.326914in}}%
\pgfpathlineto{\pgfqpoint{0.890619in}{1.328104in}}%
\pgfpathlineto{\pgfqpoint{0.892044in}{1.329154in}}%
\pgfpathlineto{\pgfqpoint{0.892857in}{1.330064in}}%
\pgfpathlineto{\pgfqpoint{0.894100in}{1.331044in}}%
\pgfpathlineto{\pgfqpoint{0.894746in}{1.331814in}}%
\pgfpathlineto{\pgfqpoint{0.896089in}{1.332864in}}%
\pgfpathlineto{\pgfqpoint{0.897200in}{1.333984in}}%
\pgfpathlineto{\pgfqpoint{0.898327in}{1.335034in}}%
\pgfpathlineto{\pgfqpoint{0.899338in}{1.336714in}}%
\pgfpathlineto{\pgfqpoint{0.900267in}{1.337694in}}%
\pgfpathlineto{\pgfqpoint{0.901311in}{1.338534in}}%
\pgfpathlineto{\pgfqpoint{0.902554in}{1.339584in}}%
\pgfpathlineto{\pgfqpoint{0.903649in}{1.340494in}}%
\pgfpathlineto{\pgfqpoint{0.905240in}{1.341544in}}%
\pgfpathlineto{\pgfqpoint{0.906334in}{1.342874in}}%
\pgfpathlineto{\pgfqpoint{0.907743in}{1.343924in}}%
\pgfpathlineto{\pgfqpoint{0.908854in}{1.344764in}}%
\pgfpathlineto{\pgfqpoint{0.910048in}{1.345814in}}%
\pgfpathlineto{\pgfqpoint{0.911142in}{1.346654in}}%
\pgfpathlineto{\pgfqpoint{0.912169in}{1.347704in}}%
\pgfpathlineto{\pgfqpoint{0.913114in}{1.348754in}}%
\pgfpathlineto{\pgfqpoint{0.914242in}{1.349804in}}%
\pgfpathlineto{\pgfqpoint{0.915004in}{1.350924in}}%
\pgfpathlineto{\pgfqpoint{0.916894in}{1.351974in}}%
\pgfpathlineto{\pgfqpoint{0.917988in}{1.353234in}}%
\pgfpathlineto{\pgfqpoint{0.919646in}{1.354284in}}%
\pgfpathlineto{\pgfqpoint{0.920757in}{1.356034in}}%
\pgfpathlineto{\pgfqpoint{0.922166in}{1.357084in}}%
\pgfpathlineto{\pgfqpoint{0.923260in}{1.358204in}}%
\pgfpathlineto{\pgfqpoint{0.925332in}{1.359254in}}%
\pgfpathlineto{\pgfqpoint{0.926376in}{1.360164in}}%
\pgfpathlineto{\pgfqpoint{0.927686in}{1.361214in}}%
\pgfpathlineto{\pgfqpoint{0.928780in}{1.361914in}}%
\pgfpathlineto{\pgfqpoint{0.930886in}{1.362964in}}%
\pgfpathlineto{\pgfqpoint{0.931980in}{1.364294in}}%
\pgfpathlineto{\pgfqpoint{0.932941in}{1.365344in}}%
\pgfpathlineto{\pgfqpoint{0.934035in}{1.366604in}}%
\pgfpathlineto{\pgfqpoint{0.935925in}{1.367654in}}%
\pgfpathlineto{\pgfqpoint{0.937003in}{1.368984in}}%
\pgfpathlineto{\pgfqpoint{0.939307in}{1.370034in}}%
\pgfpathlineto{\pgfqpoint{0.940302in}{1.371014in}}%
\pgfpathlineto{\pgfqpoint{0.941876in}{1.372064in}}%
\pgfpathlineto{\pgfqpoint{0.942871in}{1.372694in}}%
\pgfpathlineto{\pgfqpoint{0.944562in}{1.373674in}}%
\pgfpathlineto{\pgfqpoint{0.945573in}{1.374934in}}%
\pgfpathlineto{\pgfqpoint{0.947314in}{1.375984in}}%
\pgfpathlineto{\pgfqpoint{0.948159in}{1.376824in}}%
\pgfpathlineto{\pgfqpoint{0.950663in}{1.377874in}}%
\pgfpathlineto{\pgfqpoint{0.951707in}{1.378504in}}%
\pgfpathlineto{\pgfqpoint{0.953083in}{1.379484in}}%
\pgfpathlineto{\pgfqpoint{0.954127in}{1.380394in}}%
\pgfpathlineto{\pgfqpoint{0.956183in}{1.381444in}}%
\pgfpathlineto{\pgfqpoint{0.957244in}{1.382914in}}%
\pgfpathlineto{\pgfqpoint{0.958736in}{1.383964in}}%
\pgfpathlineto{\pgfqpoint{0.959714in}{1.384874in}}%
\pgfpathlineto{\pgfqpoint{0.961902in}{1.385924in}}%
\pgfpathlineto{\pgfqpoint{0.962781in}{1.386414in}}%
\pgfpathlineto{\pgfqpoint{0.965599in}{1.387464in}}%
\pgfpathlineto{\pgfqpoint{0.966643in}{1.388514in}}%
\pgfpathlineto{\pgfqpoint{0.968069in}{1.389564in}}%
\pgfpathlineto{\pgfqpoint{0.968931in}{1.390404in}}%
\pgfpathlineto{\pgfqpoint{0.971733in}{1.391454in}}%
\pgfpathlineto{\pgfqpoint{0.972843in}{1.391874in}}%
\pgfpathlineto{\pgfqpoint{0.974286in}{1.392924in}}%
\pgfpathlineto{\pgfqpoint{0.975197in}{1.393554in}}%
\pgfpathlineto{\pgfqpoint{0.977336in}{1.394604in}}%
\pgfpathlineto{\pgfqpoint{0.978413in}{1.395654in}}%
\pgfpathlineto{\pgfqpoint{0.979607in}{1.396704in}}%
\pgfpathlineto{\pgfqpoint{0.980635in}{1.397614in}}%
\pgfpathlineto{\pgfqpoint{0.981729in}{1.398664in}}%
\pgfpathlineto{\pgfqpoint{0.982690in}{1.399224in}}%
\pgfpathlineto{\pgfqpoint{0.984365in}{1.400274in}}%
\pgfpathlineto{\pgfqpoint{0.985343in}{1.401044in}}%
\pgfpathlineto{\pgfqpoint{0.987050in}{1.402094in}}%
\pgfpathlineto{\pgfqpoint{0.987912in}{1.403004in}}%
\pgfpathlineto{\pgfqpoint{0.990051in}{1.404054in}}%
\pgfpathlineto{\pgfqpoint{0.990631in}{1.404474in}}%
\pgfpathlineto{\pgfqpoint{0.992935in}{1.405524in}}%
\pgfpathlineto{\pgfqpoint{0.993847in}{1.406294in}}%
\pgfpathlineto{\pgfqpoint{0.995787in}{1.407344in}}%
\pgfpathlineto{\pgfqpoint{0.996798in}{1.408534in}}%
\pgfpathlineto{\pgfqpoint{0.998456in}{1.409444in}}%
\pgfpathlineto{\pgfqpoint{0.999533in}{1.410424in}}%
\pgfpathlineto{\pgfqpoint{1.001224in}{1.411404in}}%
\pgfpathlineto{\pgfqpoint{1.002235in}{1.412314in}}%
\pgfpathlineto{\pgfqpoint{1.003545in}{1.413364in}}%
\pgfpathlineto{\pgfqpoint{1.004423in}{1.414274in}}%
\pgfpathlineto{\pgfqpoint{1.005882in}{1.415324in}}%
\pgfpathlineto{\pgfqpoint{1.006678in}{1.415954in}}%
\pgfpathlineto{\pgfqpoint{1.009165in}{1.417004in}}%
\pgfpathlineto{\pgfqpoint{1.010226in}{1.417774in}}%
\pgfpathlineto{\pgfqpoint{1.012861in}{1.418824in}}%
\pgfpathlineto{\pgfqpoint{1.013972in}{1.419524in}}%
\pgfpathlineto{\pgfqpoint{1.016442in}{1.420504in}}%
\pgfpathlineto{\pgfqpoint{1.017437in}{1.421274in}}%
\pgfpathlineto{\pgfqpoint{1.019890in}{1.422324in}}%
\pgfpathlineto{\pgfqpoint{1.020852in}{1.423444in}}%
\pgfpathlineto{\pgfqpoint{1.022626in}{1.424494in}}%
\pgfpathlineto{\pgfqpoint{1.023488in}{1.425334in}}%
\pgfpathlineto{\pgfqpoint{1.026405in}{1.426384in}}%
\pgfpathlineto{\pgfqpoint{1.027516in}{1.427224in}}%
\pgfpathlineto{\pgfqpoint{1.029787in}{1.428274in}}%
\pgfpathlineto{\pgfqpoint{1.030831in}{1.429184in}}%
\pgfpathlineto{\pgfqpoint{1.032887in}{1.430234in}}%
\pgfpathlineto{\pgfqpoint{1.033882in}{1.431074in}}%
\pgfpathlineto{\pgfqpoint{1.036667in}{1.432124in}}%
\pgfpathlineto{\pgfqpoint{1.037711in}{1.432894in}}%
\pgfpathlineto{\pgfqpoint{1.041872in}{1.433944in}}%
\pgfpathlineto{\pgfqpoint{1.042834in}{1.434504in}}%
\pgfpathlineto{\pgfqpoint{1.046530in}{1.435554in}}%
\pgfpathlineto{\pgfqpoint{1.047575in}{1.436044in}}%
\pgfpathlineto{\pgfqpoint{1.049514in}{1.437094in}}%
\pgfpathlineto{\pgfqpoint{1.050625in}{1.437794in}}%
\pgfpathlineto{\pgfqpoint{1.053095in}{1.438844in}}%
\pgfpathlineto{\pgfqpoint{1.054090in}{1.439544in}}%
\pgfpathlineto{\pgfqpoint{1.056692in}{1.440594in}}%
\pgfpathlineto{\pgfqpoint{1.057704in}{1.441364in}}%
\pgfpathlineto{\pgfqpoint{1.061301in}{1.442414in}}%
\pgfpathlineto{\pgfqpoint{1.062329in}{1.443044in}}%
\pgfpathlineto{\pgfqpoint{1.065893in}{1.444094in}}%
\pgfpathlineto{\pgfqpoint{1.066987in}{1.444584in}}%
\pgfpathlineto{\pgfqpoint{1.068976in}{1.445634in}}%
\pgfpathlineto{\pgfqpoint{1.070037in}{1.446544in}}%
\pgfpathlineto{\pgfqpoint{1.072491in}{1.447594in}}%
\pgfpathlineto{\pgfqpoint{1.073369in}{1.448224in}}%
\pgfpathlineto{\pgfqpoint{1.075840in}{1.449274in}}%
\pgfpathlineto{\pgfqpoint{1.076652in}{1.449694in}}%
\pgfpathlineto{\pgfqpoint{1.079006in}{1.450744in}}%
\pgfpathlineto{\pgfqpoint{1.079967in}{1.451234in}}%
\pgfpathlineto{\pgfqpoint{1.083001in}{1.452214in}}%
\pgfpathlineto{\pgfqpoint{1.084112in}{1.452914in}}%
\pgfpathlineto{\pgfqpoint{1.087726in}{1.453894in}}%
\pgfpathlineto{\pgfqpoint{1.088438in}{1.454384in}}%
\pgfpathlineto{\pgfqpoint{1.092119in}{1.455434in}}%
\pgfpathlineto{\pgfqpoint{1.093064in}{1.456064in}}%
\pgfpathlineto{\pgfqpoint{1.096346in}{1.457114in}}%
\pgfpathlineto{\pgfqpoint{1.097274in}{1.457674in}}%
\pgfpathlineto{\pgfqpoint{1.100092in}{1.458724in}}%
\pgfpathlineto{\pgfqpoint{1.100689in}{1.459004in}}%
\pgfpathlineto{\pgfqpoint{1.104038in}{1.460054in}}%
\pgfpathlineto{\pgfqpoint{1.105149in}{1.460404in}}%
\pgfpathlineto{\pgfqpoint{1.109989in}{1.461454in}}%
\pgfpathlineto{\pgfqpoint{1.111083in}{1.462154in}}%
\pgfpathlineto{\pgfqpoint{1.114813in}{1.463204in}}%
\pgfpathlineto{\pgfqpoint{1.115858in}{1.463624in}}%
\pgfpathlineto{\pgfqpoint{1.119389in}{1.464674in}}%
\pgfpathlineto{\pgfqpoint{1.120433in}{1.465094in}}%
\pgfpathlineto{\pgfqpoint{1.122754in}{1.466144in}}%
\pgfpathlineto{\pgfqpoint{1.123732in}{1.466494in}}%
\pgfpathlineto{\pgfqpoint{1.127910in}{1.467544in}}%
\pgfpathlineto{\pgfqpoint{1.128556in}{1.468034in}}%
\pgfpathlineto{\pgfqpoint{1.131010in}{1.469084in}}%
\pgfpathlineto{\pgfqpoint{1.131822in}{1.469924in}}%
\pgfpathlineto{\pgfqpoint{1.135187in}{1.470974in}}%
\pgfpathlineto{\pgfqpoint{1.135204in}{1.471184in}}%
\pgfpathlineto{\pgfqpoint{1.140956in}{1.472234in}}%
\pgfpathlineto{\pgfqpoint{1.141735in}{1.472584in}}%
\pgfpathlineto{\pgfqpoint{1.145598in}{1.473634in}}%
\pgfpathlineto{\pgfqpoint{1.146675in}{1.474194in}}%
\pgfpathlineto{\pgfqpoint{1.150803in}{1.475244in}}%
\pgfpathlineto{\pgfqpoint{1.151516in}{1.475874in}}%
\pgfpathlineto{\pgfqpoint{1.156058in}{1.476924in}}%
\pgfpathlineto{\pgfqpoint{1.157152in}{1.477414in}}%
\pgfpathlineto{\pgfqpoint{1.162407in}{1.478464in}}%
\pgfpathlineto{\pgfqpoint{1.163452in}{1.479094in}}%
\pgfpathlineto{\pgfqpoint{1.168060in}{1.480144in}}%
\pgfpathlineto{\pgfqpoint{1.168873in}{1.480494in}}%
\pgfpathlineto{\pgfqpoint{1.171923in}{1.481474in}}%
\pgfpathlineto{\pgfqpoint{1.172835in}{1.481754in}}%
\pgfpathlineto{\pgfqpoint{1.177161in}{1.482804in}}%
\pgfpathlineto{\pgfqpoint{1.178156in}{1.483294in}}%
\pgfpathlineto{\pgfqpoint{1.181223in}{1.484344in}}%
\pgfpathlineto{\pgfqpoint{1.181273in}{1.484554in}}%
\pgfpathlineto{\pgfqpoint{1.188633in}{1.485604in}}%
\pgfpathlineto{\pgfqpoint{1.189512in}{1.485884in}}%
\pgfpathlineto{\pgfqpoint{1.193341in}{1.486934in}}%
\pgfpathlineto{\pgfqpoint{1.194203in}{1.487354in}}%
\pgfpathlineto{\pgfqpoint{1.198928in}{1.488404in}}%
\pgfpathlineto{\pgfqpoint{1.199823in}{1.488964in}}%
\pgfpathlineto{\pgfqpoint{1.201895in}{1.489944in}}%
\pgfpathlineto{\pgfqpoint{1.202343in}{1.490224in}}%
\pgfpathlineto{\pgfqpoint{1.205343in}{1.491274in}}%
\pgfpathlineto{\pgfqpoint{1.206404in}{1.491554in}}%
\pgfpathlineto{\pgfqpoint{1.209670in}{1.492604in}}%
\pgfpathlineto{\pgfqpoint{1.210748in}{1.493024in}}%
\pgfpathlineto{\pgfqpoint{1.215887in}{1.494074in}}%
\pgfpathlineto{\pgfqpoint{1.216865in}{1.494214in}}%
\pgfpathlineto{\pgfqpoint{1.223860in}{1.495264in}}%
\pgfpathlineto{\pgfqpoint{1.224739in}{1.495684in}}%
\pgfpathlineto{\pgfqpoint{1.229994in}{1.496734in}}%
\pgfpathlineto{\pgfqpoint{1.230823in}{1.497014in}}%
\pgfpathlineto{\pgfqpoint{1.234354in}{1.498064in}}%
\pgfpathlineto{\pgfqpoint{1.235117in}{1.498344in}}%
\pgfpathlineto{\pgfqpoint{1.240256in}{1.499394in}}%
\pgfpathlineto{\pgfqpoint{1.241151in}{1.499674in}}%
\pgfpathlineto{\pgfqpoint{1.245610in}{1.500724in}}%
\pgfpathlineto{\pgfqpoint{1.246721in}{1.501144in}}%
\pgfpathlineto{\pgfqpoint{1.248942in}{1.502194in}}%
\pgfpathlineto{\pgfqpoint{1.250053in}{1.502754in}}%
\pgfpathlineto{\pgfqpoint{1.254429in}{1.503804in}}%
\pgfpathlineto{\pgfqpoint{1.254728in}{1.504084in}}%
\pgfpathlineto{\pgfqpoint{1.260845in}{1.505134in}}%
\pgfpathlineto{\pgfqpoint{1.261723in}{1.505414in}}%
\pgfpathlineto{\pgfqpoint{1.266697in}{1.506464in}}%
\pgfpathlineto{\pgfqpoint{1.267807in}{1.506744in}}%
\pgfpathlineto{\pgfqpoint{1.271338in}{1.507724in}}%
\pgfpathlineto{\pgfqpoint{1.272200in}{1.508214in}}%
\pgfpathlineto{\pgfqpoint{1.277936in}{1.509264in}}%
\pgfpathlineto{\pgfqpoint{1.278036in}{1.509404in}}%
\pgfpathlineto{\pgfqpoint{1.284203in}{1.510454in}}%
\pgfpathlineto{\pgfqpoint{1.285131in}{1.510804in}}%
\pgfpathlineto{\pgfqpoint{1.289657in}{1.511854in}}%
\pgfpathlineto{\pgfqpoint{1.290585in}{1.512274in}}%
\pgfpathlineto{\pgfqpoint{1.295940in}{1.513324in}}%
\pgfpathlineto{\pgfqpoint{1.296354in}{1.513534in}}%
\pgfpathlineto{\pgfqpoint{1.299703in}{1.514584in}}%
\pgfpathlineto{\pgfqpoint{1.300664in}{1.514934in}}%
\pgfpathlineto{\pgfqpoint{1.305389in}{1.515984in}}%
\pgfpathlineto{\pgfqpoint{1.306118in}{1.516334in}}%
\pgfpathlineto{\pgfqpoint{1.311290in}{1.517384in}}%
\pgfpathlineto{\pgfqpoint{1.312318in}{1.517734in}}%
\pgfpathlineto{\pgfqpoint{1.318104in}{1.518784in}}%
\pgfpathlineto{\pgfqpoint{1.318899in}{1.519274in}}%
\pgfpathlineto{\pgfqpoint{1.323873in}{1.520324in}}%
\pgfpathlineto{\pgfqpoint{1.324950in}{1.520674in}}%
\pgfpathlineto{\pgfqpoint{1.329575in}{1.521654in}}%
\pgfpathlineto{\pgfqpoint{1.330305in}{1.521864in}}%
\pgfpathlineto{\pgfqpoint{1.336521in}{1.522844in}}%
\pgfpathlineto{\pgfqpoint{1.336836in}{1.523194in}}%
\pgfpathlineto{\pgfqpoint{1.344180in}{1.524244in}}%
\pgfpathlineto{\pgfqpoint{1.345191in}{1.524524in}}%
\pgfpathlineto{\pgfqpoint{1.351905in}{1.525574in}}%
\pgfpathlineto{\pgfqpoint{1.352602in}{1.525994in}}%
\pgfpathlineto{\pgfqpoint{1.359050in}{1.527044in}}%
\pgfpathlineto{\pgfqpoint{1.359962in}{1.527394in}}%
\pgfpathlineto{\pgfqpoint{1.365897in}{1.528444in}}%
\pgfpathlineto{\pgfqpoint{1.366991in}{1.528654in}}%
\pgfpathlineto{\pgfqpoint{1.378463in}{1.529704in}}%
\pgfpathlineto{\pgfqpoint{1.379092in}{1.530054in}}%
\pgfpathlineto{\pgfqpoint{1.386221in}{1.531104in}}%
\pgfpathlineto{\pgfqpoint{1.387298in}{1.531314in}}%
\pgfpathlineto{\pgfqpoint{1.393366in}{1.532364in}}%
\pgfpathlineto{\pgfqpoint{1.393880in}{1.532714in}}%
\pgfpathlineto{\pgfqpoint{1.403760in}{1.533764in}}%
\pgfpathlineto{\pgfqpoint{1.404340in}{1.533904in}}%
\pgfpathlineto{\pgfqpoint{1.412529in}{1.534954in}}%
\pgfpathlineto{\pgfqpoint{1.412994in}{1.535164in}}%
\pgfpathlineto{\pgfqpoint{1.422277in}{1.536214in}}%
\pgfpathlineto{\pgfqpoint{1.422807in}{1.536354in}}%
\pgfpathlineto{\pgfqpoint{1.428460in}{1.537404in}}%
\pgfpathlineto{\pgfqpoint{1.429157in}{1.537614in}}%
\pgfpathlineto{\pgfqpoint{1.435340in}{1.538664in}}%
\pgfpathlineto{\pgfqpoint{1.436351in}{1.539014in}}%
\pgfpathlineto{\pgfqpoint{1.443861in}{1.540064in}}%
\pgfpathlineto{\pgfqpoint{1.443877in}{1.540274in}}%
\pgfpathlineto{\pgfqpoint{1.450940in}{1.541324in}}%
\pgfpathlineto{\pgfqpoint{1.451006in}{1.541464in}}%
\pgfpathlineto{\pgfqpoint{1.462776in}{1.542514in}}%
\pgfpathlineto{\pgfqpoint{1.463439in}{1.542724in}}%
\pgfpathlineto{\pgfqpoint{1.473817in}{1.543774in}}%
\pgfpathlineto{\pgfqpoint{1.474347in}{1.544124in}}%
\pgfpathlineto{\pgfqpoint{1.482304in}{1.545174in}}%
\pgfpathlineto{\pgfqpoint{1.482669in}{1.545314in}}%
\pgfpathlineto{\pgfqpoint{1.493892in}{1.546364in}}%
\pgfpathlineto{\pgfqpoint{1.494257in}{1.546504in}}%
\pgfpathlineto{\pgfqpoint{1.503756in}{1.547554in}}%
\pgfpathlineto{\pgfqpoint{1.503921in}{1.547834in}}%
\pgfpathlineto{\pgfqpoint{1.514382in}{1.548884in}}%
\pgfpathlineto{\pgfqpoint{1.515227in}{1.549234in}}%
\pgfpathlineto{\pgfqpoint{1.524046in}{1.550284in}}%
\pgfpathlineto{\pgfqpoint{1.524378in}{1.550564in}}%
\pgfpathlineto{\pgfqpoint{1.535187in}{1.551614in}}%
\pgfpathlineto{\pgfqpoint{1.535817in}{1.551754in}}%
\pgfpathlineto{\pgfqpoint{1.546012in}{1.552804in}}%
\pgfpathlineto{\pgfqpoint{1.547122in}{1.553154in}}%
\pgfpathlineto{\pgfqpoint{1.557450in}{1.554204in}}%
\pgfpathlineto{\pgfqpoint{1.557450in}{1.554274in}}%
\pgfpathlineto{\pgfqpoint{1.571690in}{1.555324in}}%
\pgfpathlineto{\pgfqpoint{1.571790in}{1.555534in}}%
\pgfpathlineto{\pgfqpoint{1.581405in}{1.556584in}}%
\pgfpathlineto{\pgfqpoint{1.581653in}{1.556794in}}%
\pgfpathlineto{\pgfqpoint{1.595396in}{1.557844in}}%
\pgfpathlineto{\pgfqpoint{1.595844in}{1.558054in}}%
\pgfpathlineto{\pgfqpoint{1.606503in}{1.559104in}}%
\pgfpathlineto{\pgfqpoint{1.607514in}{1.559384in}}%
\pgfpathlineto{\pgfqpoint{1.617610in}{1.560434in}}%
\pgfpathlineto{\pgfqpoint{1.618721in}{1.560644in}}%
\pgfpathlineto{\pgfqpoint{1.630889in}{1.561694in}}%
\pgfpathlineto{\pgfqpoint{1.631883in}{1.561834in}}%
\pgfpathlineto{\pgfqpoint{1.652605in}{1.562884in}}%
\pgfpathlineto{\pgfqpoint{1.653600in}{1.563024in}}%
\pgfpathlineto{\pgfqpoint{1.663812in}{1.564074in}}%
\pgfpathlineto{\pgfqpoint{1.663861in}{1.564214in}}%
\pgfpathlineto{\pgfqpoint{1.672283in}{1.565264in}}%
\pgfpathlineto{\pgfqpoint{1.672382in}{1.565404in}}%
\pgfpathlineto{\pgfqpoint{1.683406in}{1.566454in}}%
\pgfpathlineto{\pgfqpoint{1.684467in}{1.566664in}}%
\pgfpathlineto{\pgfqpoint{1.696586in}{1.567714in}}%
\pgfpathlineto{\pgfqpoint{1.696834in}{1.567854in}}%
\pgfpathlineto{\pgfqpoint{1.710212in}{1.568904in}}%
\pgfpathlineto{\pgfqpoint{1.710909in}{1.569114in}}%
\pgfpathlineto{\pgfqpoint{1.723706in}{1.570164in}}%
\pgfpathlineto{\pgfqpoint{1.724718in}{1.570304in}}%
\pgfpathlineto{\pgfqpoint{1.739090in}{1.571354in}}%
\pgfpathlineto{\pgfqpoint{1.739422in}{1.571494in}}%
\pgfpathlineto{\pgfqpoint{1.756828in}{1.572544in}}%
\pgfpathlineto{\pgfqpoint{1.757591in}{1.572754in}}%
\pgfpathlineto{\pgfqpoint{1.769825in}{1.573804in}}%
\pgfpathlineto{\pgfqpoint{1.770339in}{1.573944in}}%
\pgfpathlineto{\pgfqpoint{1.782242in}{1.574994in}}%
\pgfpathlineto{\pgfqpoint{1.782374in}{1.575134in}}%
\pgfpathlineto{\pgfqpoint{1.795288in}{1.576184in}}%
\pgfpathlineto{\pgfqpoint{1.795868in}{1.576394in}}%
\pgfpathlineto{\pgfqpoint{1.809910in}{1.577444in}}%
\pgfpathlineto{\pgfqpoint{1.810987in}{1.577724in}}%
\pgfpathlineto{\pgfqpoint{1.828360in}{1.578774in}}%
\pgfpathlineto{\pgfqpoint{1.829057in}{1.578984in}}%
\pgfpathlineto{\pgfqpoint{1.842236in}{1.580034in}}%
\pgfpathlineto{\pgfqpoint{1.842368in}{1.580174in}}%
\pgfpathlineto{\pgfqpoint{1.859410in}{1.581224in}}%
\pgfpathlineto{\pgfqpoint{1.859891in}{1.581364in}}%
\pgfpathlineto{\pgfqpoint{1.873683in}{1.582414in}}%
\pgfpathlineto{\pgfqpoint{1.873683in}{1.582484in}}%
\pgfpathlineto{\pgfqpoint{1.894190in}{1.583534in}}%
\pgfpathlineto{\pgfqpoint{1.894637in}{1.583674in}}%
\pgfpathlineto{\pgfqpoint{1.915542in}{1.584724in}}%
\pgfpathlineto{\pgfqpoint{1.916155in}{1.584864in}}%
\pgfpathlineto{\pgfqpoint{1.929334in}{1.585914in}}%
\pgfpathlineto{\pgfqpoint{1.930345in}{1.586054in}}%
\pgfpathlineto{\pgfqpoint{1.945680in}{1.587104in}}%
\pgfpathlineto{\pgfqpoint{1.945680in}{1.587174in}}%
\pgfpathlineto{\pgfqpoint{1.958245in}{1.588224in}}%
\pgfpathlineto{\pgfqpoint{1.959074in}{1.588434in}}%
\pgfpathlineto{\pgfqpoint{1.971541in}{1.589484in}}%
\pgfpathlineto{\pgfqpoint{1.972005in}{1.589624in}}%
\pgfpathlineto{\pgfqpoint{1.980807in}{1.590674in}}%
\pgfpathlineto{\pgfqpoint{1.981636in}{1.590954in}}%
\pgfpathlineto{\pgfqpoint{1.988947in}{1.592004in}}%
\pgfpathlineto{\pgfqpoint{1.988947in}{1.592074in}}%
\pgfpathlineto{\pgfqpoint{1.998313in}{1.593124in}}%
\pgfpathlineto{\pgfqpoint{1.999391in}{1.593334in}}%
\pgfpathlineto{\pgfqpoint{2.006536in}{1.594384in}}%
\pgfpathlineto{\pgfqpoint{2.007414in}{1.594524in}}%
\pgfpathlineto{\pgfqpoint{2.013349in}{1.595574in}}%
\pgfpathlineto{\pgfqpoint{2.014195in}{1.595854in}}%
\pgfpathlineto{\pgfqpoint{2.021936in}{1.596904in}}%
\pgfpathlineto{\pgfqpoint{2.022318in}{1.597044in}}%
\pgfpathlineto{\pgfqpoint{2.028136in}{1.598094in}}%
\pgfpathlineto{\pgfqpoint{2.029230in}{1.598584in}}%
\pgfpathlineto{\pgfqpoint{2.032015in}{1.599634in}}%
\pgfpathlineto{\pgfqpoint{2.033126in}{1.601944in}}%
\pgfpathlineto{\pgfqpoint{2.033126in}{1.601944in}}%
\pgfusepath{stroke}%
\end{pgfscope}%
\begin{pgfscope}%
\pgfsetrectcap%
\pgfsetmiterjoin%
\pgfsetlinewidth{0.803000pt}%
\definecolor{currentstroke}{rgb}{0.000000,0.000000,0.000000}%
\pgfsetstrokecolor{currentstroke}%
\pgfsetdash{}{0pt}%
\pgfpathmoveto{\pgfqpoint{0.553581in}{0.499444in}}%
\pgfpathlineto{\pgfqpoint{0.553581in}{1.654444in}}%
\pgfusepath{stroke}%
\end{pgfscope}%
\begin{pgfscope}%
\pgfsetrectcap%
\pgfsetmiterjoin%
\pgfsetlinewidth{0.803000pt}%
\definecolor{currentstroke}{rgb}{0.000000,0.000000,0.000000}%
\pgfsetstrokecolor{currentstroke}%
\pgfsetdash{}{0pt}%
\pgfpathmoveto{\pgfqpoint{2.103581in}{0.499444in}}%
\pgfpathlineto{\pgfqpoint{2.103581in}{1.654444in}}%
\pgfusepath{stroke}%
\end{pgfscope}%
\begin{pgfscope}%
\pgfsetrectcap%
\pgfsetmiterjoin%
\pgfsetlinewidth{0.803000pt}%
\definecolor{currentstroke}{rgb}{0.000000,0.000000,0.000000}%
\pgfsetstrokecolor{currentstroke}%
\pgfsetdash{}{0pt}%
\pgfpathmoveto{\pgfqpoint{0.553581in}{0.499444in}}%
\pgfpathlineto{\pgfqpoint{2.103581in}{0.499444in}}%
\pgfusepath{stroke}%
\end{pgfscope}%
\begin{pgfscope}%
\pgfsetrectcap%
\pgfsetmiterjoin%
\pgfsetlinewidth{0.803000pt}%
\definecolor{currentstroke}{rgb}{0.000000,0.000000,0.000000}%
\pgfsetstrokecolor{currentstroke}%
\pgfsetdash{}{0pt}%
\pgfpathmoveto{\pgfqpoint{0.553581in}{1.654444in}}%
\pgfpathlineto{\pgfqpoint{2.103581in}{1.654444in}}%
\pgfusepath{stroke}%
\end{pgfscope}%
\begin{pgfscope}%
\pgfsetbuttcap%
\pgfsetmiterjoin%
\definecolor{currentfill}{rgb}{1.000000,1.000000,1.000000}%
\pgfsetfillcolor{currentfill}%
\pgfsetfillopacity{0.800000}%
\pgfsetlinewidth{1.003750pt}%
\definecolor{currentstroke}{rgb}{0.800000,0.800000,0.800000}%
\pgfsetstrokecolor{currentstroke}%
\pgfsetstrokeopacity{0.800000}%
\pgfsetdash{}{0pt}%
\pgfpathmoveto{\pgfqpoint{0.832747in}{0.568889in}}%
\pgfpathlineto{\pgfqpoint{2.006358in}{0.568889in}}%
\pgfpathquadraticcurveto{\pgfqpoint{2.034136in}{0.568889in}}{\pgfqpoint{2.034136in}{0.596666in}}%
\pgfpathlineto{\pgfqpoint{2.034136in}{0.776388in}}%
\pgfpathquadraticcurveto{\pgfqpoint{2.034136in}{0.804166in}}{\pgfqpoint{2.006358in}{0.804166in}}%
\pgfpathlineto{\pgfqpoint{0.832747in}{0.804166in}}%
\pgfpathquadraticcurveto{\pgfqpoint{0.804970in}{0.804166in}}{\pgfqpoint{0.804970in}{0.776388in}}%
\pgfpathlineto{\pgfqpoint{0.804970in}{0.596666in}}%
\pgfpathquadraticcurveto{\pgfqpoint{0.804970in}{0.568889in}}{\pgfqpoint{0.832747in}{0.568889in}}%
\pgfpathlineto{\pgfqpoint{0.832747in}{0.568889in}}%
\pgfpathclose%
\pgfusepath{stroke,fill}%
\end{pgfscope}%
\begin{pgfscope}%
\pgfsetrectcap%
\pgfsetroundjoin%
\pgfsetlinewidth{1.505625pt}%
\definecolor{currentstroke}{rgb}{0.000000,0.000000,0.000000}%
\pgfsetstrokecolor{currentstroke}%
\pgfsetdash{}{0pt}%
\pgfpathmoveto{\pgfqpoint{0.860525in}{0.700000in}}%
\pgfpathlineto{\pgfqpoint{0.999414in}{0.700000in}}%
\pgfpathlineto{\pgfqpoint{1.138303in}{0.700000in}}%
\pgfusepath{stroke}%
\end{pgfscope}%
\begin{pgfscope}%
\definecolor{textcolor}{rgb}{0.000000,0.000000,0.000000}%
\pgfsetstrokecolor{textcolor}%
\pgfsetfillcolor{textcolor}%
\pgftext[x=1.249414in,y=0.651388in,left,base]{\color{textcolor}\rmfamily\fontsize{10.000000}{12.000000}\selectfont AUC=0.840}%
\end{pgfscope}%
\end{pgfpicture}%
\makeatother%
\endgroup%

  &
\vspace{0pt} 
  
\begin{tabular}{cc|c|c|}
	&\multicolumn{1}{c}{}& \multicolumn{2}{c}{Prediction} \cr
	&\multicolumn{1}{c}{} & \multicolumn{1}{c}{N} & \multicolumn{1}{c}{P} \cr\cline{3-4}
	\multirow{2}{*}{\rotatebox[origin=c]{90}{Actual}}&N & 117,929 & 32,842 \vrule width 0pt height 10pt depth 2pt \cr\cline{3-4}
	&P & 5,928 & 20,693 \vrule width 0pt height 10pt depth 2pt \cr\cline{3-4}
\end{tabular}

\begin{center}
\begin{tabular}{ll}
0.387 & Precision \cr 
0.777 & Recall \cr 
0.516 & F1 \cr 
\end{tabular}
\end{center}
  
\end{tabular}

Because Examples 3, 4, and 5 are linear transformations of Example 1, when we find the value of decision threshold $p$ that makes $\Delta FP/\Delta TP = 2.0$ and linearly transform the probabilities, each of them becomes the distribution in Example 2.  


