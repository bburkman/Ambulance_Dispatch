%%%
\subsection{Model Evaluation:  Baselines for Comparison}


In the supervised learning method we used here, for each of the $\approx 600,000$ samples (people) in the dataset, we know the answer (the {\it label} or {\it ground truth}) to the question, whether the person needed an ambulance, $y=0$ for ``no'' and $y=1$ for ``yes.''  We are trying use historical data to build a model to predict the label for new data (incoming automated crash notifications).

The binary classification models we used return, for each sample, a continuous probability $p \in (0,1)$ that the sample belongs in the positive class.   If a sample has $p = 0.1$, the model is 90\% confident that this sample is in the negative class.   We then pick a threshold, usually but not necessarily 0.5, and make a binary prediction, that samples with $p > 0.5$ need an ambulance, and those with $p < 0.5$ do not.   (What happens if $p=0.50000000000000000$ would apply to negligibly few samples, so which way it goes does not matter.) The {\it loss function} used by the model is the sum not of how many binary predictions were incorrect, but how strongly incorrect the continuous predictions were.  If the prediction for a sample is $p = 0.3$ and the label is $y=0$, then the loss for that sample is $0.3$, but if its label were $y=1$, then the loss would be $0.7$.  

A perfect model would not only predict each sample's label correctly, but would do it with perfect certainty.  In the real world, with interesting questions about real data, we will have false positives ($y=0$ and $p>0.5$) and false negatives ($y=1$ and $p<0.5$), but we hope those are few, and that the predictions are strongly correct, meaning the predictions are close to their labels.

When we get results for our models based on crash data, we need some frame of reference for what is ``good'' and ``bad,'' so we have created some sets of entirely artificial results using a gamma distribution for ideal results and a uniform distribution for awful results.  

The histogram below of the percent of samples with predictions $p$ in each range illustrates the best results we can hope for in the real world.  The positive class is small because the data is imbalanced. about 15\% of the dataset, as in our CRSS data.  There are some false positives and negatives, but the overwhelming majority of the predictions are correct, and most with strong confidence.  

The Receiver Operating Characteristic (ROC) is a parameterized curve following the probability threshold from $p=0$ to $p=1$, plotting the true positive rate (TPR) versus the false positive rate (FPR).  The Area Under the ROC curve (AUC) is often used to compare two models, with AUC of 1 indicating perfect prediction and AUC of 0.5 indicating no discernable pattern.  

We have added to the typical ROC curve two labels, one for the positive and one for the negative class, of the median of the probabilities of the samples in that class.  T%he further apart those numbers are, the more robust the model.
%We have added to the typical ROC curve the quartiles of the probabilities $p \in (0,1)$ of all of the samples to hint at the distribution and illustrate that the curve starts at $p=0$ in the upper right and goes to $p=1$ in the lower left.  A smaller number for the first quartile would correspond to more confidence in the model's predictions of the negative class.  Interpreting the other two numbers requires considering the class imbalance.  

\begin{tabular}{p{0.5\textwidth} p{0.5\textwidth}}
  \vspace{0pt} %% Creator: Matplotlib, PGF backend
%%
%% To include the figure in your LaTeX document, write
%%   \input{<filename>.pgf}
%%
%% Make sure the required packages are loaded in your preamble
%%   \usepackage{pgf}
%%
%% Also ensure that all the required font packages are loaded; for instance,
%% the lmodern package is sometimes necessary when using math font.
%%   \usepackage{lmodern}
%%
%% Figures using additional raster images can only be included by \input if
%% they are in the same directory as the main LaTeX file. For loading figures
%% from other directories you can use the `import` package
%%   \usepackage{import}
%%
%% and then include the figures with
%%   \import{<path to file>}{<filename>.pgf}
%%
%% Matplotlib used the following preamble
%%   
%%   \usepackage{fontspec}
%%   \makeatletter\@ifpackageloaded{underscore}{}{\usepackage[strings]{underscore}}\makeatother
%%
\begingroup%
\makeatletter%
\begin{pgfpicture}%
\pgfpathrectangle{\pgfpointorigin}{\pgfqpoint{2.153750in}{1.654444in}}%
\pgfusepath{use as bounding box, clip}%
\begin{pgfscope}%
\pgfsetbuttcap%
\pgfsetmiterjoin%
\definecolor{currentfill}{rgb}{1.000000,1.000000,1.000000}%
\pgfsetfillcolor{currentfill}%
\pgfsetlinewidth{0.000000pt}%
\definecolor{currentstroke}{rgb}{1.000000,1.000000,1.000000}%
\pgfsetstrokecolor{currentstroke}%
\pgfsetdash{}{0pt}%
\pgfpathmoveto{\pgfqpoint{0.000000in}{0.000000in}}%
\pgfpathlineto{\pgfqpoint{2.153750in}{0.000000in}}%
\pgfpathlineto{\pgfqpoint{2.153750in}{1.654444in}}%
\pgfpathlineto{\pgfqpoint{0.000000in}{1.654444in}}%
\pgfpathlineto{\pgfqpoint{0.000000in}{0.000000in}}%
\pgfpathclose%
\pgfusepath{fill}%
\end{pgfscope}%
\begin{pgfscope}%
\pgfsetbuttcap%
\pgfsetmiterjoin%
\definecolor{currentfill}{rgb}{1.000000,1.000000,1.000000}%
\pgfsetfillcolor{currentfill}%
\pgfsetlinewidth{0.000000pt}%
\definecolor{currentstroke}{rgb}{0.000000,0.000000,0.000000}%
\pgfsetstrokecolor{currentstroke}%
\pgfsetstrokeopacity{0.000000}%
\pgfsetdash{}{0pt}%
\pgfpathmoveto{\pgfqpoint{0.465000in}{0.449444in}}%
\pgfpathlineto{\pgfqpoint{2.015000in}{0.449444in}}%
\pgfpathlineto{\pgfqpoint{2.015000in}{1.604444in}}%
\pgfpathlineto{\pgfqpoint{0.465000in}{1.604444in}}%
\pgfpathlineto{\pgfqpoint{0.465000in}{0.449444in}}%
\pgfpathclose%
\pgfusepath{fill}%
\end{pgfscope}%
\begin{pgfscope}%
\pgfpathrectangle{\pgfqpoint{0.465000in}{0.449444in}}{\pgfqpoint{1.550000in}{1.155000in}}%
\pgfusepath{clip}%
\pgfsetbuttcap%
\pgfsetmiterjoin%
\pgfsetlinewidth{1.003750pt}%
\definecolor{currentstroke}{rgb}{0.000000,0.000000,0.000000}%
\pgfsetstrokecolor{currentstroke}%
\pgfsetdash{}{0pt}%
\pgfpathmoveto{\pgfqpoint{0.455000in}{0.449444in}}%
\pgfpathlineto{\pgfqpoint{0.502805in}{0.449444in}}%
\pgfpathlineto{\pgfqpoint{0.502805in}{0.590081in}}%
\pgfpathlineto{\pgfqpoint{0.455000in}{0.590081in}}%
\pgfusepath{stroke}%
\end{pgfscope}%
\begin{pgfscope}%
\pgfpathrectangle{\pgfqpoint{0.465000in}{0.449444in}}{\pgfqpoint{1.550000in}{1.155000in}}%
\pgfusepath{clip}%
\pgfsetbuttcap%
\pgfsetmiterjoin%
\pgfsetlinewidth{1.003750pt}%
\definecolor{currentstroke}{rgb}{0.000000,0.000000,0.000000}%
\pgfsetstrokecolor{currentstroke}%
\pgfsetdash{}{0pt}%
\pgfpathmoveto{\pgfqpoint{0.593537in}{0.449444in}}%
\pgfpathlineto{\pgfqpoint{0.654025in}{0.449444in}}%
\pgfpathlineto{\pgfqpoint{0.654025in}{1.204573in}}%
\pgfpathlineto{\pgfqpoint{0.593537in}{1.204573in}}%
\pgfpathlineto{\pgfqpoint{0.593537in}{0.449444in}}%
\pgfpathclose%
\pgfusepath{stroke}%
\end{pgfscope}%
\begin{pgfscope}%
\pgfpathrectangle{\pgfqpoint{0.465000in}{0.449444in}}{\pgfqpoint{1.550000in}{1.155000in}}%
\pgfusepath{clip}%
\pgfsetbuttcap%
\pgfsetmiterjoin%
\pgfsetlinewidth{1.003750pt}%
\definecolor{currentstroke}{rgb}{0.000000,0.000000,0.000000}%
\pgfsetstrokecolor{currentstroke}%
\pgfsetdash{}{0pt}%
\pgfpathmoveto{\pgfqpoint{0.744756in}{0.449444in}}%
\pgfpathlineto{\pgfqpoint{0.805244in}{0.449444in}}%
\pgfpathlineto{\pgfqpoint{0.805244in}{1.549444in}}%
\pgfpathlineto{\pgfqpoint{0.744756in}{1.549444in}}%
\pgfpathlineto{\pgfqpoint{0.744756in}{0.449444in}}%
\pgfpathclose%
\pgfusepath{stroke}%
\end{pgfscope}%
\begin{pgfscope}%
\pgfpathrectangle{\pgfqpoint{0.465000in}{0.449444in}}{\pgfqpoint{1.550000in}{1.155000in}}%
\pgfusepath{clip}%
\pgfsetbuttcap%
\pgfsetmiterjoin%
\pgfsetlinewidth{1.003750pt}%
\definecolor{currentstroke}{rgb}{0.000000,0.000000,0.000000}%
\pgfsetstrokecolor{currentstroke}%
\pgfsetdash{}{0pt}%
\pgfpathmoveto{\pgfqpoint{0.895976in}{0.449444in}}%
\pgfpathlineto{\pgfqpoint{0.956464in}{0.449444in}}%
\pgfpathlineto{\pgfqpoint{0.956464in}{1.447255in}}%
\pgfpathlineto{\pgfqpoint{0.895976in}{1.447255in}}%
\pgfpathlineto{\pgfqpoint{0.895976in}{0.449444in}}%
\pgfpathclose%
\pgfusepath{stroke}%
\end{pgfscope}%
\begin{pgfscope}%
\pgfpathrectangle{\pgfqpoint{0.465000in}{0.449444in}}{\pgfqpoint{1.550000in}{1.155000in}}%
\pgfusepath{clip}%
\pgfsetbuttcap%
\pgfsetmiterjoin%
\pgfsetlinewidth{1.003750pt}%
\definecolor{currentstroke}{rgb}{0.000000,0.000000,0.000000}%
\pgfsetstrokecolor{currentstroke}%
\pgfsetdash{}{0pt}%
\pgfpathmoveto{\pgfqpoint{1.047195in}{0.449444in}}%
\pgfpathlineto{\pgfqpoint{1.107683in}{0.449444in}}%
\pgfpathlineto{\pgfqpoint{1.107683in}{1.160058in}}%
\pgfpathlineto{\pgfqpoint{1.047195in}{1.160058in}}%
\pgfpathlineto{\pgfqpoint{1.047195in}{0.449444in}}%
\pgfpathclose%
\pgfusepath{stroke}%
\end{pgfscope}%
\begin{pgfscope}%
\pgfpathrectangle{\pgfqpoint{0.465000in}{0.449444in}}{\pgfqpoint{1.550000in}{1.155000in}}%
\pgfusepath{clip}%
\pgfsetbuttcap%
\pgfsetmiterjoin%
\pgfsetlinewidth{1.003750pt}%
\definecolor{currentstroke}{rgb}{0.000000,0.000000,0.000000}%
\pgfsetstrokecolor{currentstroke}%
\pgfsetdash{}{0pt}%
\pgfpathmoveto{\pgfqpoint{1.198415in}{0.449444in}}%
\pgfpathlineto{\pgfqpoint{1.258903in}{0.449444in}}%
\pgfpathlineto{\pgfqpoint{1.258903in}{0.885778in}}%
\pgfpathlineto{\pgfqpoint{1.198415in}{0.885778in}}%
\pgfpathlineto{\pgfqpoint{1.198415in}{0.449444in}}%
\pgfpathclose%
\pgfusepath{stroke}%
\end{pgfscope}%
\begin{pgfscope}%
\pgfpathrectangle{\pgfqpoint{0.465000in}{0.449444in}}{\pgfqpoint{1.550000in}{1.155000in}}%
\pgfusepath{clip}%
\pgfsetbuttcap%
\pgfsetmiterjoin%
\pgfsetlinewidth{1.003750pt}%
\definecolor{currentstroke}{rgb}{0.000000,0.000000,0.000000}%
\pgfsetstrokecolor{currentstroke}%
\pgfsetdash{}{0pt}%
\pgfpathmoveto{\pgfqpoint{1.349634in}{0.449444in}}%
\pgfpathlineto{\pgfqpoint{1.410122in}{0.449444in}}%
\pgfpathlineto{\pgfqpoint{1.410122in}{0.731988in}}%
\pgfpathlineto{\pgfqpoint{1.349634in}{0.731988in}}%
\pgfpathlineto{\pgfqpoint{1.349634in}{0.449444in}}%
\pgfpathclose%
\pgfusepath{stroke}%
\end{pgfscope}%
\begin{pgfscope}%
\pgfpathrectangle{\pgfqpoint{0.465000in}{0.449444in}}{\pgfqpoint{1.550000in}{1.155000in}}%
\pgfusepath{clip}%
\pgfsetbuttcap%
\pgfsetmiterjoin%
\pgfsetlinewidth{1.003750pt}%
\definecolor{currentstroke}{rgb}{0.000000,0.000000,0.000000}%
\pgfsetstrokecolor{currentstroke}%
\pgfsetdash{}{0pt}%
\pgfpathmoveto{\pgfqpoint{1.500854in}{0.449444in}}%
\pgfpathlineto{\pgfqpoint{1.561342in}{0.449444in}}%
\pgfpathlineto{\pgfqpoint{1.561342in}{0.577885in}}%
\pgfpathlineto{\pgfqpoint{1.500854in}{0.577885in}}%
\pgfpathlineto{\pgfqpoint{1.500854in}{0.449444in}}%
\pgfpathclose%
\pgfusepath{stroke}%
\end{pgfscope}%
\begin{pgfscope}%
\pgfpathrectangle{\pgfqpoint{0.465000in}{0.449444in}}{\pgfqpoint{1.550000in}{1.155000in}}%
\pgfusepath{clip}%
\pgfsetbuttcap%
\pgfsetmiterjoin%
\pgfsetlinewidth{1.003750pt}%
\definecolor{currentstroke}{rgb}{0.000000,0.000000,0.000000}%
\pgfsetstrokecolor{currentstroke}%
\pgfsetdash{}{0pt}%
\pgfpathmoveto{\pgfqpoint{1.652073in}{0.449444in}}%
\pgfpathlineto{\pgfqpoint{1.712561in}{0.449444in}}%
\pgfpathlineto{\pgfqpoint{1.712561in}{0.513416in}}%
\pgfpathlineto{\pgfqpoint{1.652073in}{0.513416in}}%
\pgfpathlineto{\pgfqpoint{1.652073in}{0.449444in}}%
\pgfpathclose%
\pgfusepath{stroke}%
\end{pgfscope}%
\begin{pgfscope}%
\pgfpathrectangle{\pgfqpoint{0.465000in}{0.449444in}}{\pgfqpoint{1.550000in}{1.155000in}}%
\pgfusepath{clip}%
\pgfsetbuttcap%
\pgfsetmiterjoin%
\pgfsetlinewidth{1.003750pt}%
\definecolor{currentstroke}{rgb}{0.000000,0.000000,0.000000}%
\pgfsetstrokecolor{currentstroke}%
\pgfsetdash{}{0pt}%
\pgfpathmoveto{\pgfqpoint{1.803293in}{0.449444in}}%
\pgfpathlineto{\pgfqpoint{1.863781in}{0.449444in}}%
\pgfpathlineto{\pgfqpoint{1.863781in}{0.467616in}}%
\pgfpathlineto{\pgfqpoint{1.803293in}{0.467616in}}%
\pgfpathlineto{\pgfqpoint{1.803293in}{0.449444in}}%
\pgfpathclose%
\pgfusepath{stroke}%
\end{pgfscope}%
\begin{pgfscope}%
\pgfpathrectangle{\pgfqpoint{0.465000in}{0.449444in}}{\pgfqpoint{1.550000in}{1.155000in}}%
\pgfusepath{clip}%
\pgfsetbuttcap%
\pgfsetmiterjoin%
\definecolor{currentfill}{rgb}{0.000000,0.000000,0.000000}%
\pgfsetfillcolor{currentfill}%
\pgfsetlinewidth{0.000000pt}%
\definecolor{currentstroke}{rgb}{0.000000,0.000000,0.000000}%
\pgfsetstrokecolor{currentstroke}%
\pgfsetstrokeopacity{0.000000}%
\pgfsetdash{}{0pt}%
\pgfpathmoveto{\pgfqpoint{0.502805in}{0.449444in}}%
\pgfpathlineto{\pgfqpoint{0.563293in}{0.449444in}}%
\pgfpathlineto{\pgfqpoint{0.563293in}{0.456162in}}%
\pgfpathlineto{\pgfqpoint{0.502805in}{0.456162in}}%
\pgfpathlineto{\pgfqpoint{0.502805in}{0.449444in}}%
\pgfpathclose%
\pgfusepath{fill}%
\end{pgfscope}%
\begin{pgfscope}%
\pgfpathrectangle{\pgfqpoint{0.465000in}{0.449444in}}{\pgfqpoint{1.550000in}{1.155000in}}%
\pgfusepath{clip}%
\pgfsetbuttcap%
\pgfsetmiterjoin%
\definecolor{currentfill}{rgb}{0.000000,0.000000,0.000000}%
\pgfsetfillcolor{currentfill}%
\pgfsetlinewidth{0.000000pt}%
\definecolor{currentstroke}{rgb}{0.000000,0.000000,0.000000}%
\pgfsetstrokecolor{currentstroke}%
\pgfsetstrokeopacity{0.000000}%
\pgfsetdash{}{0pt}%
\pgfpathmoveto{\pgfqpoint{0.654025in}{0.449444in}}%
\pgfpathlineto{\pgfqpoint{0.714512in}{0.449444in}}%
\pgfpathlineto{\pgfqpoint{0.714512in}{0.462803in}}%
\pgfpathlineto{\pgfqpoint{0.654025in}{0.462803in}}%
\pgfpathlineto{\pgfqpoint{0.654025in}{0.449444in}}%
\pgfpathclose%
\pgfusepath{fill}%
\end{pgfscope}%
\begin{pgfscope}%
\pgfpathrectangle{\pgfqpoint{0.465000in}{0.449444in}}{\pgfqpoint{1.550000in}{1.155000in}}%
\pgfusepath{clip}%
\pgfsetbuttcap%
\pgfsetmiterjoin%
\definecolor{currentfill}{rgb}{0.000000,0.000000,0.000000}%
\pgfsetfillcolor{currentfill}%
\pgfsetlinewidth{0.000000pt}%
\definecolor{currentstroke}{rgb}{0.000000,0.000000,0.000000}%
\pgfsetstrokecolor{currentstroke}%
\pgfsetstrokeopacity{0.000000}%
\pgfsetdash{}{0pt}%
\pgfpathmoveto{\pgfqpoint{0.805244in}{0.449444in}}%
\pgfpathlineto{\pgfqpoint{0.865732in}{0.449444in}}%
\pgfpathlineto{\pgfqpoint{0.865732in}{0.475374in}}%
\pgfpathlineto{\pgfqpoint{0.805244in}{0.475374in}}%
\pgfpathlineto{\pgfqpoint{0.805244in}{0.449444in}}%
\pgfpathclose%
\pgfusepath{fill}%
\end{pgfscope}%
\begin{pgfscope}%
\pgfpathrectangle{\pgfqpoint{0.465000in}{0.449444in}}{\pgfqpoint{1.550000in}{1.155000in}}%
\pgfusepath{clip}%
\pgfsetbuttcap%
\pgfsetmiterjoin%
\definecolor{currentfill}{rgb}{0.000000,0.000000,0.000000}%
\pgfsetfillcolor{currentfill}%
\pgfsetlinewidth{0.000000pt}%
\definecolor{currentstroke}{rgb}{0.000000,0.000000,0.000000}%
\pgfsetstrokecolor{currentstroke}%
\pgfsetstrokeopacity{0.000000}%
\pgfsetdash{}{0pt}%
\pgfpathmoveto{\pgfqpoint{0.956464in}{0.449444in}}%
\pgfpathlineto{\pgfqpoint{1.016951in}{0.449444in}}%
\pgfpathlineto{\pgfqpoint{1.016951in}{0.496652in}}%
\pgfpathlineto{\pgfqpoint{0.956464in}{0.496652in}}%
\pgfpathlineto{\pgfqpoint{0.956464in}{0.449444in}}%
\pgfpathclose%
\pgfusepath{fill}%
\end{pgfscope}%
\begin{pgfscope}%
\pgfpathrectangle{\pgfqpoint{0.465000in}{0.449444in}}{\pgfqpoint{1.550000in}{1.155000in}}%
\pgfusepath{clip}%
\pgfsetbuttcap%
\pgfsetmiterjoin%
\definecolor{currentfill}{rgb}{0.000000,0.000000,0.000000}%
\pgfsetfillcolor{currentfill}%
\pgfsetlinewidth{0.000000pt}%
\definecolor{currentstroke}{rgb}{0.000000,0.000000,0.000000}%
\pgfsetstrokecolor{currentstroke}%
\pgfsetstrokeopacity{0.000000}%
\pgfsetdash{}{0pt}%
\pgfpathmoveto{\pgfqpoint{1.107683in}{0.449444in}}%
\pgfpathlineto{\pgfqpoint{1.168171in}{0.449444in}}%
\pgfpathlineto{\pgfqpoint{1.168171in}{0.530884in}}%
\pgfpathlineto{\pgfqpoint{1.107683in}{0.530884in}}%
\pgfpathlineto{\pgfqpoint{1.107683in}{0.449444in}}%
\pgfpathclose%
\pgfusepath{fill}%
\end{pgfscope}%
\begin{pgfscope}%
\pgfpathrectangle{\pgfqpoint{0.465000in}{0.449444in}}{\pgfqpoint{1.550000in}{1.155000in}}%
\pgfusepath{clip}%
\pgfsetbuttcap%
\pgfsetmiterjoin%
\definecolor{currentfill}{rgb}{0.000000,0.000000,0.000000}%
\pgfsetfillcolor{currentfill}%
\pgfsetlinewidth{0.000000pt}%
\definecolor{currentstroke}{rgb}{0.000000,0.000000,0.000000}%
\pgfsetstrokecolor{currentstroke}%
\pgfsetstrokeopacity{0.000000}%
\pgfsetdash{}{0pt}%
\pgfpathmoveto{\pgfqpoint{1.258903in}{0.449444in}}%
\pgfpathlineto{\pgfqpoint{1.319391in}{0.449444in}}%
\pgfpathlineto{\pgfqpoint{1.319391in}{0.576975in}}%
\pgfpathlineto{\pgfqpoint{1.258903in}{0.576975in}}%
\pgfpathlineto{\pgfqpoint{1.258903in}{0.449444in}}%
\pgfpathclose%
\pgfusepath{fill}%
\end{pgfscope}%
\begin{pgfscope}%
\pgfpathrectangle{\pgfqpoint{0.465000in}{0.449444in}}{\pgfqpoint{1.550000in}{1.155000in}}%
\pgfusepath{clip}%
\pgfsetbuttcap%
\pgfsetmiterjoin%
\definecolor{currentfill}{rgb}{0.000000,0.000000,0.000000}%
\pgfsetfillcolor{currentfill}%
\pgfsetlinewidth{0.000000pt}%
\definecolor{currentstroke}{rgb}{0.000000,0.000000,0.000000}%
\pgfsetstrokecolor{currentstroke}%
\pgfsetstrokeopacity{0.000000}%
\pgfsetdash{}{0pt}%
\pgfpathmoveto{\pgfqpoint{1.410122in}{0.449444in}}%
\pgfpathlineto{\pgfqpoint{1.470610in}{0.449444in}}%
\pgfpathlineto{\pgfqpoint{1.470610in}{0.624175in}}%
\pgfpathlineto{\pgfqpoint{1.410122in}{0.624175in}}%
\pgfpathlineto{\pgfqpoint{1.410122in}{0.449444in}}%
\pgfpathclose%
\pgfusepath{fill}%
\end{pgfscope}%
\begin{pgfscope}%
\pgfpathrectangle{\pgfqpoint{0.465000in}{0.449444in}}{\pgfqpoint{1.550000in}{1.155000in}}%
\pgfusepath{clip}%
\pgfsetbuttcap%
\pgfsetmiterjoin%
\definecolor{currentfill}{rgb}{0.000000,0.000000,0.000000}%
\pgfsetfillcolor{currentfill}%
\pgfsetlinewidth{0.000000pt}%
\definecolor{currentstroke}{rgb}{0.000000,0.000000,0.000000}%
\pgfsetstrokecolor{currentstroke}%
\pgfsetstrokeopacity{0.000000}%
\pgfsetdash{}{0pt}%
\pgfpathmoveto{\pgfqpoint{1.561342in}{0.449444in}}%
\pgfpathlineto{\pgfqpoint{1.621830in}{0.449444in}}%
\pgfpathlineto{\pgfqpoint{1.621830in}{0.643181in}}%
\pgfpathlineto{\pgfqpoint{1.561342in}{0.643181in}}%
\pgfpathlineto{\pgfqpoint{1.561342in}{0.449444in}}%
\pgfpathclose%
\pgfusepath{fill}%
\end{pgfscope}%
\begin{pgfscope}%
\pgfpathrectangle{\pgfqpoint{0.465000in}{0.449444in}}{\pgfqpoint{1.550000in}{1.155000in}}%
\pgfusepath{clip}%
\pgfsetbuttcap%
\pgfsetmiterjoin%
\definecolor{currentfill}{rgb}{0.000000,0.000000,0.000000}%
\pgfsetfillcolor{currentfill}%
\pgfsetlinewidth{0.000000pt}%
\definecolor{currentstroke}{rgb}{0.000000,0.000000,0.000000}%
\pgfsetstrokecolor{currentstroke}%
\pgfsetstrokeopacity{0.000000}%
\pgfsetdash{}{0pt}%
\pgfpathmoveto{\pgfqpoint{1.712561in}{0.449444in}}%
\pgfpathlineto{\pgfqpoint{1.773049in}{0.449444in}}%
\pgfpathlineto{\pgfqpoint{1.773049in}{0.579591in}}%
\pgfpathlineto{\pgfqpoint{1.712561in}{0.579591in}}%
\pgfpathlineto{\pgfqpoint{1.712561in}{0.449444in}}%
\pgfpathclose%
\pgfusepath{fill}%
\end{pgfscope}%
\begin{pgfscope}%
\pgfpathrectangle{\pgfqpoint{0.465000in}{0.449444in}}{\pgfqpoint{1.550000in}{1.155000in}}%
\pgfusepath{clip}%
\pgfsetbuttcap%
\pgfsetmiterjoin%
\definecolor{currentfill}{rgb}{0.000000,0.000000,0.000000}%
\pgfsetfillcolor{currentfill}%
\pgfsetlinewidth{0.000000pt}%
\definecolor{currentstroke}{rgb}{0.000000,0.000000,0.000000}%
\pgfsetstrokecolor{currentstroke}%
\pgfsetstrokeopacity{0.000000}%
\pgfsetdash{}{0pt}%
\pgfpathmoveto{\pgfqpoint{1.863781in}{0.449444in}}%
\pgfpathlineto{\pgfqpoint{1.924269in}{0.449444in}}%
\pgfpathlineto{\pgfqpoint{1.924269in}{0.474640in}}%
\pgfpathlineto{\pgfqpoint{1.863781in}{0.474640in}}%
\pgfpathlineto{\pgfqpoint{1.863781in}{0.449444in}}%
\pgfpathclose%
\pgfusepath{fill}%
\end{pgfscope}%
\begin{pgfscope}%
\pgfsetbuttcap%
\pgfsetroundjoin%
\definecolor{currentfill}{rgb}{0.000000,0.000000,0.000000}%
\pgfsetfillcolor{currentfill}%
\pgfsetlinewidth{0.803000pt}%
\definecolor{currentstroke}{rgb}{0.000000,0.000000,0.000000}%
\pgfsetstrokecolor{currentstroke}%
\pgfsetdash{}{0pt}%
\pgfsys@defobject{currentmarker}{\pgfqpoint{0.000000in}{-0.048611in}}{\pgfqpoint{0.000000in}{0.000000in}}{%
\pgfpathmoveto{\pgfqpoint{0.000000in}{0.000000in}}%
\pgfpathlineto{\pgfqpoint{0.000000in}{-0.048611in}}%
\pgfusepath{stroke,fill}%
}%
\begin{pgfscope}%
\pgfsys@transformshift{0.502805in}{0.449444in}%
\pgfsys@useobject{currentmarker}{}%
\end{pgfscope}%
\end{pgfscope}%
\begin{pgfscope}%
\definecolor{textcolor}{rgb}{0.000000,0.000000,0.000000}%
\pgfsetstrokecolor{textcolor}%
\pgfsetfillcolor{textcolor}%
\pgftext[x=0.502805in,y=0.352222in,,top]{\color{textcolor}\rmfamily\fontsize{10.000000}{12.000000}\selectfont 0.0}%
\end{pgfscope}%
\begin{pgfscope}%
\pgfsetbuttcap%
\pgfsetroundjoin%
\definecolor{currentfill}{rgb}{0.000000,0.000000,0.000000}%
\pgfsetfillcolor{currentfill}%
\pgfsetlinewidth{0.803000pt}%
\definecolor{currentstroke}{rgb}{0.000000,0.000000,0.000000}%
\pgfsetstrokecolor{currentstroke}%
\pgfsetdash{}{0pt}%
\pgfsys@defobject{currentmarker}{\pgfqpoint{0.000000in}{-0.048611in}}{\pgfqpoint{0.000000in}{0.000000in}}{%
\pgfpathmoveto{\pgfqpoint{0.000000in}{0.000000in}}%
\pgfpathlineto{\pgfqpoint{0.000000in}{-0.048611in}}%
\pgfusepath{stroke,fill}%
}%
\begin{pgfscope}%
\pgfsys@transformshift{0.880854in}{0.449444in}%
\pgfsys@useobject{currentmarker}{}%
\end{pgfscope}%
\end{pgfscope}%
\begin{pgfscope}%
\definecolor{textcolor}{rgb}{0.000000,0.000000,0.000000}%
\pgfsetstrokecolor{textcolor}%
\pgfsetfillcolor{textcolor}%
\pgftext[x=0.880854in,y=0.352222in,,top]{\color{textcolor}\rmfamily\fontsize{10.000000}{12.000000}\selectfont 0.25}%
\end{pgfscope}%
\begin{pgfscope}%
\pgfsetbuttcap%
\pgfsetroundjoin%
\definecolor{currentfill}{rgb}{0.000000,0.000000,0.000000}%
\pgfsetfillcolor{currentfill}%
\pgfsetlinewidth{0.803000pt}%
\definecolor{currentstroke}{rgb}{0.000000,0.000000,0.000000}%
\pgfsetstrokecolor{currentstroke}%
\pgfsetdash{}{0pt}%
\pgfsys@defobject{currentmarker}{\pgfqpoint{0.000000in}{-0.048611in}}{\pgfqpoint{0.000000in}{0.000000in}}{%
\pgfpathmoveto{\pgfqpoint{0.000000in}{0.000000in}}%
\pgfpathlineto{\pgfqpoint{0.000000in}{-0.048611in}}%
\pgfusepath{stroke,fill}%
}%
\begin{pgfscope}%
\pgfsys@transformshift{1.258903in}{0.449444in}%
\pgfsys@useobject{currentmarker}{}%
\end{pgfscope}%
\end{pgfscope}%
\begin{pgfscope}%
\definecolor{textcolor}{rgb}{0.000000,0.000000,0.000000}%
\pgfsetstrokecolor{textcolor}%
\pgfsetfillcolor{textcolor}%
\pgftext[x=1.258903in,y=0.352222in,,top]{\color{textcolor}\rmfamily\fontsize{10.000000}{12.000000}\selectfont 0.5}%
\end{pgfscope}%
\begin{pgfscope}%
\pgfsetbuttcap%
\pgfsetroundjoin%
\definecolor{currentfill}{rgb}{0.000000,0.000000,0.000000}%
\pgfsetfillcolor{currentfill}%
\pgfsetlinewidth{0.803000pt}%
\definecolor{currentstroke}{rgb}{0.000000,0.000000,0.000000}%
\pgfsetstrokecolor{currentstroke}%
\pgfsetdash{}{0pt}%
\pgfsys@defobject{currentmarker}{\pgfqpoint{0.000000in}{-0.048611in}}{\pgfqpoint{0.000000in}{0.000000in}}{%
\pgfpathmoveto{\pgfqpoint{0.000000in}{0.000000in}}%
\pgfpathlineto{\pgfqpoint{0.000000in}{-0.048611in}}%
\pgfusepath{stroke,fill}%
}%
\begin{pgfscope}%
\pgfsys@transformshift{1.636951in}{0.449444in}%
\pgfsys@useobject{currentmarker}{}%
\end{pgfscope}%
\end{pgfscope}%
\begin{pgfscope}%
\definecolor{textcolor}{rgb}{0.000000,0.000000,0.000000}%
\pgfsetstrokecolor{textcolor}%
\pgfsetfillcolor{textcolor}%
\pgftext[x=1.636951in,y=0.352222in,,top]{\color{textcolor}\rmfamily\fontsize{10.000000}{12.000000}\selectfont 0.75}%
\end{pgfscope}%
\begin{pgfscope}%
\pgfsetbuttcap%
\pgfsetroundjoin%
\definecolor{currentfill}{rgb}{0.000000,0.000000,0.000000}%
\pgfsetfillcolor{currentfill}%
\pgfsetlinewidth{0.803000pt}%
\definecolor{currentstroke}{rgb}{0.000000,0.000000,0.000000}%
\pgfsetstrokecolor{currentstroke}%
\pgfsetdash{}{0pt}%
\pgfsys@defobject{currentmarker}{\pgfqpoint{0.000000in}{-0.048611in}}{\pgfqpoint{0.000000in}{0.000000in}}{%
\pgfpathmoveto{\pgfqpoint{0.000000in}{0.000000in}}%
\pgfpathlineto{\pgfqpoint{0.000000in}{-0.048611in}}%
\pgfusepath{stroke,fill}%
}%
\begin{pgfscope}%
\pgfsys@transformshift{2.015000in}{0.449444in}%
\pgfsys@useobject{currentmarker}{}%
\end{pgfscope}%
\end{pgfscope}%
\begin{pgfscope}%
\definecolor{textcolor}{rgb}{0.000000,0.000000,0.000000}%
\pgfsetstrokecolor{textcolor}%
\pgfsetfillcolor{textcolor}%
\pgftext[x=2.015000in,y=0.352222in,,top]{\color{textcolor}\rmfamily\fontsize{10.000000}{12.000000}\selectfont 1.0}%
\end{pgfscope}%
\begin{pgfscope}%
\definecolor{textcolor}{rgb}{0.000000,0.000000,0.000000}%
\pgfsetstrokecolor{textcolor}%
\pgfsetfillcolor{textcolor}%
\pgftext[x=1.240000in,y=0.173333in,,top]{\color{textcolor}\rmfamily\fontsize{10.000000}{12.000000}\selectfont \(\displaystyle p\)}%
\end{pgfscope}%
\begin{pgfscope}%
\pgfsetbuttcap%
\pgfsetroundjoin%
\definecolor{currentfill}{rgb}{0.000000,0.000000,0.000000}%
\pgfsetfillcolor{currentfill}%
\pgfsetlinewidth{0.803000pt}%
\definecolor{currentstroke}{rgb}{0.000000,0.000000,0.000000}%
\pgfsetstrokecolor{currentstroke}%
\pgfsetdash{}{0pt}%
\pgfsys@defobject{currentmarker}{\pgfqpoint{-0.048611in}{0.000000in}}{\pgfqpoint{-0.000000in}{0.000000in}}{%
\pgfpathmoveto{\pgfqpoint{-0.000000in}{0.000000in}}%
\pgfpathlineto{\pgfqpoint{-0.048611in}{0.000000in}}%
\pgfusepath{stroke,fill}%
}%
\begin{pgfscope}%
\pgfsys@transformshift{0.465000in}{0.449444in}%
\pgfsys@useobject{currentmarker}{}%
\end{pgfscope}%
\end{pgfscope}%
\begin{pgfscope}%
\definecolor{textcolor}{rgb}{0.000000,0.000000,0.000000}%
\pgfsetstrokecolor{textcolor}%
\pgfsetfillcolor{textcolor}%
\pgftext[x=0.298333in, y=0.401250in, left, base]{\color{textcolor}\rmfamily\fontsize{10.000000}{12.000000}\selectfont \(\displaystyle {0}\)}%
\end{pgfscope}%
\begin{pgfscope}%
\pgfsetbuttcap%
\pgfsetroundjoin%
\definecolor{currentfill}{rgb}{0.000000,0.000000,0.000000}%
\pgfsetfillcolor{currentfill}%
\pgfsetlinewidth{0.803000pt}%
\definecolor{currentstroke}{rgb}{0.000000,0.000000,0.000000}%
\pgfsetstrokecolor{currentstroke}%
\pgfsetdash{}{0pt}%
\pgfsys@defobject{currentmarker}{\pgfqpoint{-0.048611in}{0.000000in}}{\pgfqpoint{-0.000000in}{0.000000in}}{%
\pgfpathmoveto{\pgfqpoint{-0.000000in}{0.000000in}}%
\pgfpathlineto{\pgfqpoint{-0.048611in}{0.000000in}}%
\pgfusepath{stroke,fill}%
}%
\begin{pgfscope}%
\pgfsys@transformshift{0.465000in}{0.995409in}%
\pgfsys@useobject{currentmarker}{}%
\end{pgfscope}%
\end{pgfscope}%
\begin{pgfscope}%
\definecolor{textcolor}{rgb}{0.000000,0.000000,0.000000}%
\pgfsetstrokecolor{textcolor}%
\pgfsetfillcolor{textcolor}%
\pgftext[x=0.228889in, y=0.947214in, left, base]{\color{textcolor}\rmfamily\fontsize{10.000000}{12.000000}\selectfont \(\displaystyle {10}\)}%
\end{pgfscope}%
\begin{pgfscope}%
\pgfsetbuttcap%
\pgfsetroundjoin%
\definecolor{currentfill}{rgb}{0.000000,0.000000,0.000000}%
\pgfsetfillcolor{currentfill}%
\pgfsetlinewidth{0.803000pt}%
\definecolor{currentstroke}{rgb}{0.000000,0.000000,0.000000}%
\pgfsetstrokecolor{currentstroke}%
\pgfsetdash{}{0pt}%
\pgfsys@defobject{currentmarker}{\pgfqpoint{-0.048611in}{0.000000in}}{\pgfqpoint{-0.000000in}{0.000000in}}{%
\pgfpathmoveto{\pgfqpoint{-0.000000in}{0.000000in}}%
\pgfpathlineto{\pgfqpoint{-0.048611in}{0.000000in}}%
\pgfusepath{stroke,fill}%
}%
\begin{pgfscope}%
\pgfsys@transformshift{0.465000in}{1.541374in}%
\pgfsys@useobject{currentmarker}{}%
\end{pgfscope}%
\end{pgfscope}%
\begin{pgfscope}%
\definecolor{textcolor}{rgb}{0.000000,0.000000,0.000000}%
\pgfsetstrokecolor{textcolor}%
\pgfsetfillcolor{textcolor}%
\pgftext[x=0.228889in, y=1.493179in, left, base]{\color{textcolor}\rmfamily\fontsize{10.000000}{12.000000}\selectfont \(\displaystyle {20}\)}%
\end{pgfscope}%
\begin{pgfscope}%
\definecolor{textcolor}{rgb}{0.000000,0.000000,0.000000}%
\pgfsetstrokecolor{textcolor}%
\pgfsetfillcolor{textcolor}%
\pgftext[x=0.173333in,y=1.026944in,,bottom,rotate=90.000000]{\color{textcolor}\rmfamily\fontsize{10.000000}{12.000000}\selectfont Percent of Data Set}%
\end{pgfscope}%
\begin{pgfscope}%
\pgfsetrectcap%
\pgfsetmiterjoin%
\pgfsetlinewidth{0.803000pt}%
\definecolor{currentstroke}{rgb}{0.000000,0.000000,0.000000}%
\pgfsetstrokecolor{currentstroke}%
\pgfsetdash{}{0pt}%
\pgfpathmoveto{\pgfqpoint{0.465000in}{0.449444in}}%
\pgfpathlineto{\pgfqpoint{0.465000in}{1.604444in}}%
\pgfusepath{stroke}%
\end{pgfscope}%
\begin{pgfscope}%
\pgfsetrectcap%
\pgfsetmiterjoin%
\pgfsetlinewidth{0.803000pt}%
\definecolor{currentstroke}{rgb}{0.000000,0.000000,0.000000}%
\pgfsetstrokecolor{currentstroke}%
\pgfsetdash{}{0pt}%
\pgfpathmoveto{\pgfqpoint{2.015000in}{0.449444in}}%
\pgfpathlineto{\pgfqpoint{2.015000in}{1.604444in}}%
\pgfusepath{stroke}%
\end{pgfscope}%
\begin{pgfscope}%
\pgfsetrectcap%
\pgfsetmiterjoin%
\pgfsetlinewidth{0.803000pt}%
\definecolor{currentstroke}{rgb}{0.000000,0.000000,0.000000}%
\pgfsetstrokecolor{currentstroke}%
\pgfsetdash{}{0pt}%
\pgfpathmoveto{\pgfqpoint{0.465000in}{0.449444in}}%
\pgfpathlineto{\pgfqpoint{2.015000in}{0.449444in}}%
\pgfusepath{stroke}%
\end{pgfscope}%
\begin{pgfscope}%
\pgfsetrectcap%
\pgfsetmiterjoin%
\pgfsetlinewidth{0.803000pt}%
\definecolor{currentstroke}{rgb}{0.000000,0.000000,0.000000}%
\pgfsetstrokecolor{currentstroke}%
\pgfsetdash{}{0pt}%
\pgfpathmoveto{\pgfqpoint{0.465000in}{1.604444in}}%
\pgfpathlineto{\pgfqpoint{2.015000in}{1.604444in}}%
\pgfusepath{stroke}%
\end{pgfscope}%
\begin{pgfscope}%
\pgfsetbuttcap%
\pgfsetmiterjoin%
\definecolor{currentfill}{rgb}{1.000000,1.000000,1.000000}%
\pgfsetfillcolor{currentfill}%
\pgfsetfillopacity{0.800000}%
\pgfsetlinewidth{1.003750pt}%
\definecolor{currentstroke}{rgb}{0.800000,0.800000,0.800000}%
\pgfsetstrokecolor{currentstroke}%
\pgfsetstrokeopacity{0.800000}%
\pgfsetdash{}{0pt}%
\pgfpathmoveto{\pgfqpoint{1.238056in}{1.104445in}}%
\pgfpathlineto{\pgfqpoint{1.917778in}{1.104445in}}%
\pgfpathquadraticcurveto{\pgfqpoint{1.945556in}{1.104445in}}{\pgfqpoint{1.945556in}{1.132222in}}%
\pgfpathlineto{\pgfqpoint{1.945556in}{1.507222in}}%
\pgfpathquadraticcurveto{\pgfqpoint{1.945556in}{1.535000in}}{\pgfqpoint{1.917778in}{1.535000in}}%
\pgfpathlineto{\pgfqpoint{1.238056in}{1.535000in}}%
\pgfpathquadraticcurveto{\pgfqpoint{1.210278in}{1.535000in}}{\pgfqpoint{1.210278in}{1.507222in}}%
\pgfpathlineto{\pgfqpoint{1.210278in}{1.132222in}}%
\pgfpathquadraticcurveto{\pgfqpoint{1.210278in}{1.104445in}}{\pgfqpoint{1.238056in}{1.104445in}}%
\pgfpathlineto{\pgfqpoint{1.238056in}{1.104445in}}%
\pgfpathclose%
\pgfusepath{stroke,fill}%
\end{pgfscope}%
\begin{pgfscope}%
\pgfsetbuttcap%
\pgfsetmiterjoin%
\pgfsetlinewidth{1.003750pt}%
\definecolor{currentstroke}{rgb}{0.000000,0.000000,0.000000}%
\pgfsetstrokecolor{currentstroke}%
\pgfsetdash{}{0pt}%
\pgfpathmoveto{\pgfqpoint{1.265834in}{1.382222in}}%
\pgfpathlineto{\pgfqpoint{1.543611in}{1.382222in}}%
\pgfpathlineto{\pgfqpoint{1.543611in}{1.479444in}}%
\pgfpathlineto{\pgfqpoint{1.265834in}{1.479444in}}%
\pgfpathlineto{\pgfqpoint{1.265834in}{1.382222in}}%
\pgfpathclose%
\pgfusepath{stroke}%
\end{pgfscope}%
\begin{pgfscope}%
\definecolor{textcolor}{rgb}{0.000000,0.000000,0.000000}%
\pgfsetstrokecolor{textcolor}%
\pgfsetfillcolor{textcolor}%
\pgftext[x=1.654722in,y=1.382222in,left,base]{\color{textcolor}\rmfamily\fontsize{10.000000}{12.000000}\selectfont Neg}%
\end{pgfscope}%
\begin{pgfscope}%
\pgfsetbuttcap%
\pgfsetmiterjoin%
\definecolor{currentfill}{rgb}{0.000000,0.000000,0.000000}%
\pgfsetfillcolor{currentfill}%
\pgfsetlinewidth{0.000000pt}%
\definecolor{currentstroke}{rgb}{0.000000,0.000000,0.000000}%
\pgfsetstrokecolor{currentstroke}%
\pgfsetstrokeopacity{0.000000}%
\pgfsetdash{}{0pt}%
\pgfpathmoveto{\pgfqpoint{1.265834in}{1.186944in}}%
\pgfpathlineto{\pgfqpoint{1.543611in}{1.186944in}}%
\pgfpathlineto{\pgfqpoint{1.543611in}{1.284167in}}%
\pgfpathlineto{\pgfqpoint{1.265834in}{1.284167in}}%
\pgfpathlineto{\pgfqpoint{1.265834in}{1.186944in}}%
\pgfpathclose%
\pgfusepath{fill}%
\end{pgfscope}%
\begin{pgfscope}%
\definecolor{textcolor}{rgb}{0.000000,0.000000,0.000000}%
\pgfsetstrokecolor{textcolor}%
\pgfsetfillcolor{textcolor}%
\pgftext[x=1.654722in,y=1.186944in,left,base]{\color{textcolor}\rmfamily\fontsize{10.000000}{12.000000}\selectfont Pos}%
\end{pgfscope}%
\end{pgfpicture}%
\makeatother%
\endgroup%

  &
  \vspace{0pt} %% Creator: Matplotlib, PGF backend
%%
%% To include the figure in your LaTeX document, write
%%   \input{<filename>.pgf}
%%
%% Make sure the required packages are loaded in your preamble
%%   \usepackage{pgf}
%%
%% Also ensure that all the required font packages are loaded; for instance,
%% the lmodern package is sometimes necessary when using math font.
%%   \usepackage{lmodern}
%%
%% Figures using additional raster images can only be included by \input if
%% they are in the same directory as the main LaTeX file. For loading figures
%% from other directories you can use the `import` package
%%   \usepackage{import}
%%
%% and then include the figures with
%%   \import{<path to file>}{<filename>.pgf}
%%
%% Matplotlib used the following preamble
%%   
%%   \usepackage{fontspec}
%%   \makeatletter\@ifpackageloaded{underscore}{}{\usepackage[strings]{underscore}}\makeatother
%%
\begingroup%
\makeatletter%
\begin{pgfpicture}%
\pgfpathrectangle{\pgfpointorigin}{\pgfqpoint{2.221861in}{1.754444in}}%
\pgfusepath{use as bounding box, clip}%
\begin{pgfscope}%
\pgfsetbuttcap%
\pgfsetmiterjoin%
\definecolor{currentfill}{rgb}{1.000000,1.000000,1.000000}%
\pgfsetfillcolor{currentfill}%
\pgfsetlinewidth{0.000000pt}%
\definecolor{currentstroke}{rgb}{1.000000,1.000000,1.000000}%
\pgfsetstrokecolor{currentstroke}%
\pgfsetdash{}{0pt}%
\pgfpathmoveto{\pgfqpoint{0.000000in}{0.000000in}}%
\pgfpathlineto{\pgfqpoint{2.221861in}{0.000000in}}%
\pgfpathlineto{\pgfqpoint{2.221861in}{1.754444in}}%
\pgfpathlineto{\pgfqpoint{0.000000in}{1.754444in}}%
\pgfpathlineto{\pgfqpoint{0.000000in}{0.000000in}}%
\pgfpathclose%
\pgfusepath{fill}%
\end{pgfscope}%
\begin{pgfscope}%
\pgfsetbuttcap%
\pgfsetmiterjoin%
\definecolor{currentfill}{rgb}{1.000000,1.000000,1.000000}%
\pgfsetfillcolor{currentfill}%
\pgfsetlinewidth{0.000000pt}%
\definecolor{currentstroke}{rgb}{0.000000,0.000000,0.000000}%
\pgfsetstrokecolor{currentstroke}%
\pgfsetstrokeopacity{0.000000}%
\pgfsetdash{}{0pt}%
\pgfpathmoveto{\pgfqpoint{0.553581in}{0.499444in}}%
\pgfpathlineto{\pgfqpoint{2.103581in}{0.499444in}}%
\pgfpathlineto{\pgfqpoint{2.103581in}{1.654444in}}%
\pgfpathlineto{\pgfqpoint{0.553581in}{1.654444in}}%
\pgfpathlineto{\pgfqpoint{0.553581in}{0.499444in}}%
\pgfpathclose%
\pgfusepath{fill}%
\end{pgfscope}%
\begin{pgfscope}%
\pgfsetbuttcap%
\pgfsetroundjoin%
\definecolor{currentfill}{rgb}{0.000000,0.000000,0.000000}%
\pgfsetfillcolor{currentfill}%
\pgfsetlinewidth{0.803000pt}%
\definecolor{currentstroke}{rgb}{0.000000,0.000000,0.000000}%
\pgfsetstrokecolor{currentstroke}%
\pgfsetdash{}{0pt}%
\pgfsys@defobject{currentmarker}{\pgfqpoint{0.000000in}{-0.048611in}}{\pgfqpoint{0.000000in}{0.000000in}}{%
\pgfpathmoveto{\pgfqpoint{0.000000in}{0.000000in}}%
\pgfpathlineto{\pgfqpoint{0.000000in}{-0.048611in}}%
\pgfusepath{stroke,fill}%
}%
\begin{pgfscope}%
\pgfsys@transformshift{0.624035in}{0.499444in}%
\pgfsys@useobject{currentmarker}{}%
\end{pgfscope}%
\end{pgfscope}%
\begin{pgfscope}%
\definecolor{textcolor}{rgb}{0.000000,0.000000,0.000000}%
\pgfsetstrokecolor{textcolor}%
\pgfsetfillcolor{textcolor}%
\pgftext[x=0.624035in,y=0.402222in,,top]{\color{textcolor}\rmfamily\fontsize{10.000000}{12.000000}\selectfont \(\displaystyle {0.0}\)}%
\end{pgfscope}%
\begin{pgfscope}%
\pgfsetbuttcap%
\pgfsetroundjoin%
\definecolor{currentfill}{rgb}{0.000000,0.000000,0.000000}%
\pgfsetfillcolor{currentfill}%
\pgfsetlinewidth{0.803000pt}%
\definecolor{currentstroke}{rgb}{0.000000,0.000000,0.000000}%
\pgfsetstrokecolor{currentstroke}%
\pgfsetdash{}{0pt}%
\pgfsys@defobject{currentmarker}{\pgfqpoint{0.000000in}{-0.048611in}}{\pgfqpoint{0.000000in}{0.000000in}}{%
\pgfpathmoveto{\pgfqpoint{0.000000in}{0.000000in}}%
\pgfpathlineto{\pgfqpoint{0.000000in}{-0.048611in}}%
\pgfusepath{stroke,fill}%
}%
\begin{pgfscope}%
\pgfsys@transformshift{1.328581in}{0.499444in}%
\pgfsys@useobject{currentmarker}{}%
\end{pgfscope}%
\end{pgfscope}%
\begin{pgfscope}%
\definecolor{textcolor}{rgb}{0.000000,0.000000,0.000000}%
\pgfsetstrokecolor{textcolor}%
\pgfsetfillcolor{textcolor}%
\pgftext[x=1.328581in,y=0.402222in,,top]{\color{textcolor}\rmfamily\fontsize{10.000000}{12.000000}\selectfont \(\displaystyle {0.5}\)}%
\end{pgfscope}%
\begin{pgfscope}%
\pgfsetbuttcap%
\pgfsetroundjoin%
\definecolor{currentfill}{rgb}{0.000000,0.000000,0.000000}%
\pgfsetfillcolor{currentfill}%
\pgfsetlinewidth{0.803000pt}%
\definecolor{currentstroke}{rgb}{0.000000,0.000000,0.000000}%
\pgfsetstrokecolor{currentstroke}%
\pgfsetdash{}{0pt}%
\pgfsys@defobject{currentmarker}{\pgfqpoint{0.000000in}{-0.048611in}}{\pgfqpoint{0.000000in}{0.000000in}}{%
\pgfpathmoveto{\pgfqpoint{0.000000in}{0.000000in}}%
\pgfpathlineto{\pgfqpoint{0.000000in}{-0.048611in}}%
\pgfusepath{stroke,fill}%
}%
\begin{pgfscope}%
\pgfsys@transformshift{2.033126in}{0.499444in}%
\pgfsys@useobject{currentmarker}{}%
\end{pgfscope}%
\end{pgfscope}%
\begin{pgfscope}%
\definecolor{textcolor}{rgb}{0.000000,0.000000,0.000000}%
\pgfsetstrokecolor{textcolor}%
\pgfsetfillcolor{textcolor}%
\pgftext[x=2.033126in,y=0.402222in,,top]{\color{textcolor}\rmfamily\fontsize{10.000000}{12.000000}\selectfont \(\displaystyle {1.0}\)}%
\end{pgfscope}%
\begin{pgfscope}%
\definecolor{textcolor}{rgb}{0.000000,0.000000,0.000000}%
\pgfsetstrokecolor{textcolor}%
\pgfsetfillcolor{textcolor}%
\pgftext[x=1.328581in,y=0.223333in,,top]{\color{textcolor}\rmfamily\fontsize{10.000000}{12.000000}\selectfont False positive rate}%
\end{pgfscope}%
\begin{pgfscope}%
\pgfsetbuttcap%
\pgfsetroundjoin%
\definecolor{currentfill}{rgb}{0.000000,0.000000,0.000000}%
\pgfsetfillcolor{currentfill}%
\pgfsetlinewidth{0.803000pt}%
\definecolor{currentstroke}{rgb}{0.000000,0.000000,0.000000}%
\pgfsetstrokecolor{currentstroke}%
\pgfsetdash{}{0pt}%
\pgfsys@defobject{currentmarker}{\pgfqpoint{-0.048611in}{0.000000in}}{\pgfqpoint{-0.000000in}{0.000000in}}{%
\pgfpathmoveto{\pgfqpoint{-0.000000in}{0.000000in}}%
\pgfpathlineto{\pgfqpoint{-0.048611in}{0.000000in}}%
\pgfusepath{stroke,fill}%
}%
\begin{pgfscope}%
\pgfsys@transformshift{0.553581in}{0.551944in}%
\pgfsys@useobject{currentmarker}{}%
\end{pgfscope}%
\end{pgfscope}%
\begin{pgfscope}%
\definecolor{textcolor}{rgb}{0.000000,0.000000,0.000000}%
\pgfsetstrokecolor{textcolor}%
\pgfsetfillcolor{textcolor}%
\pgftext[x=0.278889in, y=0.503750in, left, base]{\color{textcolor}\rmfamily\fontsize{10.000000}{12.000000}\selectfont \(\displaystyle {0.0}\)}%
\end{pgfscope}%
\begin{pgfscope}%
\pgfsetbuttcap%
\pgfsetroundjoin%
\definecolor{currentfill}{rgb}{0.000000,0.000000,0.000000}%
\pgfsetfillcolor{currentfill}%
\pgfsetlinewidth{0.803000pt}%
\definecolor{currentstroke}{rgb}{0.000000,0.000000,0.000000}%
\pgfsetstrokecolor{currentstroke}%
\pgfsetdash{}{0pt}%
\pgfsys@defobject{currentmarker}{\pgfqpoint{-0.048611in}{0.000000in}}{\pgfqpoint{-0.000000in}{0.000000in}}{%
\pgfpathmoveto{\pgfqpoint{-0.000000in}{0.000000in}}%
\pgfpathlineto{\pgfqpoint{-0.048611in}{0.000000in}}%
\pgfusepath{stroke,fill}%
}%
\begin{pgfscope}%
\pgfsys@transformshift{0.553581in}{1.076944in}%
\pgfsys@useobject{currentmarker}{}%
\end{pgfscope}%
\end{pgfscope}%
\begin{pgfscope}%
\definecolor{textcolor}{rgb}{0.000000,0.000000,0.000000}%
\pgfsetstrokecolor{textcolor}%
\pgfsetfillcolor{textcolor}%
\pgftext[x=0.278889in, y=1.028750in, left, base]{\color{textcolor}\rmfamily\fontsize{10.000000}{12.000000}\selectfont \(\displaystyle {0.5}\)}%
\end{pgfscope}%
\begin{pgfscope}%
\pgfsetbuttcap%
\pgfsetroundjoin%
\definecolor{currentfill}{rgb}{0.000000,0.000000,0.000000}%
\pgfsetfillcolor{currentfill}%
\pgfsetlinewidth{0.803000pt}%
\definecolor{currentstroke}{rgb}{0.000000,0.000000,0.000000}%
\pgfsetstrokecolor{currentstroke}%
\pgfsetdash{}{0pt}%
\pgfsys@defobject{currentmarker}{\pgfqpoint{-0.048611in}{0.000000in}}{\pgfqpoint{-0.000000in}{0.000000in}}{%
\pgfpathmoveto{\pgfqpoint{-0.000000in}{0.000000in}}%
\pgfpathlineto{\pgfqpoint{-0.048611in}{0.000000in}}%
\pgfusepath{stroke,fill}%
}%
\begin{pgfscope}%
\pgfsys@transformshift{0.553581in}{1.601944in}%
\pgfsys@useobject{currentmarker}{}%
\end{pgfscope}%
\end{pgfscope}%
\begin{pgfscope}%
\definecolor{textcolor}{rgb}{0.000000,0.000000,0.000000}%
\pgfsetstrokecolor{textcolor}%
\pgfsetfillcolor{textcolor}%
\pgftext[x=0.278889in, y=1.553750in, left, base]{\color{textcolor}\rmfamily\fontsize{10.000000}{12.000000}\selectfont \(\displaystyle {1.0}\)}%
\end{pgfscope}%
\begin{pgfscope}%
\definecolor{textcolor}{rgb}{0.000000,0.000000,0.000000}%
\pgfsetstrokecolor{textcolor}%
\pgfsetfillcolor{textcolor}%
\pgftext[x=0.223333in,y=1.076944in,,bottom,rotate=90.000000]{\color{textcolor}\rmfamily\fontsize{10.000000}{12.000000}\selectfont True positive rate}%
\end{pgfscope}%
\begin{pgfscope}%
\pgfpathrectangle{\pgfqpoint{0.553581in}{0.499444in}}{\pgfqpoint{1.550000in}{1.155000in}}%
\pgfusepath{clip}%
\pgfsetbuttcap%
\pgfsetroundjoin%
\pgfsetlinewidth{1.505625pt}%
\definecolor{currentstroke}{rgb}{0.000000,0.000000,0.000000}%
\pgfsetstrokecolor{currentstroke}%
\pgfsetdash{{5.550000pt}{2.400000pt}}{0.000000pt}%
\pgfpathmoveto{\pgfqpoint{0.624035in}{0.551944in}}%
\pgfpathlineto{\pgfqpoint{2.033126in}{1.601944in}}%
\pgfusepath{stroke}%
\end{pgfscope}%
\begin{pgfscope}%
\pgfpathrectangle{\pgfqpoint{0.553581in}{0.499444in}}{\pgfqpoint{1.550000in}{1.155000in}}%
\pgfusepath{clip}%
\pgfsetrectcap%
\pgfsetroundjoin%
\pgfsetlinewidth{1.505625pt}%
\definecolor{currentstroke}{rgb}{0.000000,0.000000,0.000000}%
\pgfsetstrokecolor{currentstroke}%
\pgfsetdash{}{0pt}%
\pgfpathmoveto{\pgfqpoint{0.624035in}{0.551944in}}%
\pgfpathlineto{\pgfqpoint{0.626207in}{0.552574in}}%
\pgfpathlineto{\pgfqpoint{0.627318in}{0.561464in}}%
\pgfpathlineto{\pgfqpoint{0.628014in}{0.562514in}}%
\pgfpathlineto{\pgfqpoint{0.629125in}{0.563634in}}%
\pgfpathlineto{\pgfqpoint{0.629605in}{0.564614in}}%
\pgfpathlineto{\pgfqpoint{0.630699in}{0.567694in}}%
\pgfpathlineto{\pgfqpoint{0.631130in}{0.568604in}}%
\pgfpathlineto{\pgfqpoint{0.632225in}{0.573294in}}%
\pgfpathlineto{\pgfqpoint{0.632772in}{0.574064in}}%
\pgfpathlineto{\pgfqpoint{0.633866in}{0.579944in}}%
\pgfpathlineto{\pgfqpoint{0.634247in}{0.580854in}}%
\pgfpathlineto{\pgfqpoint{0.635358in}{0.585824in}}%
\pgfpathlineto{\pgfqpoint{0.635540in}{0.586874in}}%
\pgfpathlineto{\pgfqpoint{0.636634in}{0.593244in}}%
\pgfpathlineto{\pgfqpoint{0.636833in}{0.594154in}}%
\pgfpathlineto{\pgfqpoint{0.637944in}{0.600734in}}%
\pgfpathlineto{\pgfqpoint{0.638126in}{0.601784in}}%
\pgfpathlineto{\pgfqpoint{0.639187in}{0.608714in}}%
\pgfpathlineto{\pgfqpoint{0.639353in}{0.609694in}}%
\pgfpathlineto{\pgfqpoint{0.640464in}{0.616834in}}%
\pgfpathlineto{\pgfqpoint{0.640712in}{0.617884in}}%
\pgfpathlineto{\pgfqpoint{0.641806in}{0.625864in}}%
\pgfpathlineto{\pgfqpoint{0.642038in}{0.626844in}}%
\pgfpathlineto{\pgfqpoint{0.643149in}{0.635804in}}%
\pgfpathlineto{\pgfqpoint{0.643348in}{0.636854in}}%
\pgfpathlineto{\pgfqpoint{0.644392in}{0.645044in}}%
\pgfpathlineto{\pgfqpoint{0.644641in}{0.645884in}}%
\pgfpathlineto{\pgfqpoint{0.645752in}{0.652394in}}%
\pgfpathlineto{\pgfqpoint{0.645868in}{0.653164in}}%
\pgfpathlineto{\pgfqpoint{0.646979in}{0.661424in}}%
\pgfpathlineto{\pgfqpoint{0.647327in}{0.662264in}}%
\pgfpathlineto{\pgfqpoint{0.648437in}{0.668914in}}%
\pgfpathlineto{\pgfqpoint{0.648587in}{0.669824in}}%
\pgfpathlineto{\pgfqpoint{0.649697in}{0.678294in}}%
\pgfpathlineto{\pgfqpoint{0.649813in}{0.679274in}}%
\pgfpathlineto{\pgfqpoint{0.650924in}{0.687044in}}%
\pgfpathlineto{\pgfqpoint{0.651140in}{0.687884in}}%
\pgfpathlineto{\pgfqpoint{0.652250in}{0.696144in}}%
\pgfpathlineto{\pgfqpoint{0.652333in}{0.697194in}}%
\pgfpathlineto{\pgfqpoint{0.653444in}{0.706434in}}%
\pgfpathlineto{\pgfqpoint{0.653858in}{0.707414in}}%
\pgfpathlineto{\pgfqpoint{0.654969in}{0.716584in}}%
\pgfpathlineto{\pgfqpoint{0.655151in}{0.717564in}}%
\pgfpathlineto{\pgfqpoint{0.656262in}{0.725194in}}%
\pgfpathlineto{\pgfqpoint{0.656444in}{0.726174in}}%
\pgfpathlineto{\pgfqpoint{0.657538in}{0.733244in}}%
\pgfpathlineto{\pgfqpoint{0.657704in}{0.734294in}}%
\pgfpathlineto{\pgfqpoint{0.658815in}{0.743464in}}%
\pgfpathlineto{\pgfqpoint{0.659064in}{0.744514in}}%
\pgfpathlineto{\pgfqpoint{0.660158in}{0.751164in}}%
\pgfpathlineto{\pgfqpoint{0.660290in}{0.751794in}}%
\pgfpathlineto{\pgfqpoint{0.661401in}{0.759354in}}%
\pgfpathlineto{\pgfqpoint{0.661633in}{0.760404in}}%
\pgfpathlineto{\pgfqpoint{0.662744in}{0.769784in}}%
\pgfpathlineto{\pgfqpoint{0.662827in}{0.770344in}}%
\pgfpathlineto{\pgfqpoint{0.663937in}{0.777764in}}%
\pgfpathlineto{\pgfqpoint{0.664120in}{0.778814in}}%
\pgfpathlineto{\pgfqpoint{0.665181in}{0.783854in}}%
\pgfpathlineto{\pgfqpoint{0.665297in}{0.784834in}}%
\pgfpathlineto{\pgfqpoint{0.666407in}{0.795614in}}%
\pgfpathlineto{\pgfqpoint{0.666656in}{0.796664in}}%
\pgfpathlineto{\pgfqpoint{0.667767in}{0.801564in}}%
\pgfpathlineto{\pgfqpoint{0.667899in}{0.802404in}}%
\pgfpathlineto{\pgfqpoint{0.669010in}{0.809334in}}%
\pgfpathlineto{\pgfqpoint{0.669342in}{0.810384in}}%
\pgfpathlineto{\pgfqpoint{0.670452in}{0.818084in}}%
\pgfpathlineto{\pgfqpoint{0.670701in}{0.819134in}}%
\pgfpathlineto{\pgfqpoint{0.671812in}{0.825364in}}%
\pgfpathlineto{\pgfqpoint{0.671911in}{0.826414in}}%
\pgfpathlineto{\pgfqpoint{0.673022in}{0.833694in}}%
\pgfpathlineto{\pgfqpoint{0.673138in}{0.834744in}}%
\pgfpathlineto{\pgfqpoint{0.674215in}{0.838944in}}%
\pgfpathlineto{\pgfqpoint{0.674530in}{0.839994in}}%
\pgfpathlineto{\pgfqpoint{0.675608in}{0.846014in}}%
\pgfpathlineto{\pgfqpoint{0.675790in}{0.846994in}}%
\pgfpathlineto{\pgfqpoint{0.676901in}{0.853854in}}%
\pgfpathlineto{\pgfqpoint{0.677199in}{0.854834in}}%
\pgfpathlineto{\pgfqpoint{0.678310in}{0.859034in}}%
\pgfpathlineto{\pgfqpoint{0.678492in}{0.860084in}}%
\pgfpathlineto{\pgfqpoint{0.679587in}{0.867014in}}%
\pgfpathlineto{\pgfqpoint{0.679868in}{0.867924in}}%
\pgfpathlineto{\pgfqpoint{0.680963in}{0.874504in}}%
\pgfpathlineto{\pgfqpoint{0.681228in}{0.875484in}}%
\pgfpathlineto{\pgfqpoint{0.682338in}{0.882204in}}%
\pgfpathlineto{\pgfqpoint{0.682587in}{0.883254in}}%
\pgfpathlineto{\pgfqpoint{0.683681in}{0.889484in}}%
\pgfpathlineto{\pgfqpoint{0.683980in}{0.890394in}}%
\pgfpathlineto{\pgfqpoint{0.685074in}{0.897394in}}%
\pgfpathlineto{\pgfqpoint{0.685389in}{0.898304in}}%
\pgfpathlineto{\pgfqpoint{0.686450in}{0.904184in}}%
\pgfpathlineto{\pgfqpoint{0.686748in}{0.905094in}}%
\pgfpathlineto{\pgfqpoint{0.687859in}{0.910204in}}%
\pgfpathlineto{\pgfqpoint{0.688074in}{0.911114in}}%
\pgfpathlineto{\pgfqpoint{0.689185in}{0.915034in}}%
\pgfpathlineto{\pgfqpoint{0.689400in}{0.916084in}}%
\pgfpathlineto{\pgfqpoint{0.690511in}{0.922664in}}%
\pgfpathlineto{\pgfqpoint{0.690793in}{0.923714in}}%
\pgfpathlineto{\pgfqpoint{0.691887in}{0.928754in}}%
\pgfpathlineto{\pgfqpoint{0.692318in}{0.929734in}}%
\pgfpathlineto{\pgfqpoint{0.693412in}{0.935334in}}%
\pgfpathlineto{\pgfqpoint{0.693611in}{0.936384in}}%
\pgfpathlineto{\pgfqpoint{0.694705in}{0.940864in}}%
\pgfpathlineto{\pgfqpoint{0.695053in}{0.941914in}}%
\pgfpathlineto{\pgfqpoint{0.696164in}{0.948634in}}%
\pgfpathlineto{\pgfqpoint{0.696645in}{0.949684in}}%
\pgfpathlineto{\pgfqpoint{0.697756in}{0.954164in}}%
\pgfpathlineto{\pgfqpoint{0.697921in}{0.955144in}}%
\pgfpathlineto{\pgfqpoint{0.699015in}{0.960464in}}%
\pgfpathlineto{\pgfqpoint{0.699314in}{0.961304in}}%
\pgfpathlineto{\pgfqpoint{0.700425in}{0.965294in}}%
\pgfpathlineto{\pgfqpoint{0.700657in}{0.966344in}}%
\pgfpathlineto{\pgfqpoint{0.701767in}{0.971314in}}%
\pgfpathlineto{\pgfqpoint{0.702049in}{0.972364in}}%
\pgfpathlineto{\pgfqpoint{0.703094in}{0.977194in}}%
\pgfpathlineto{\pgfqpoint{0.703442in}{0.978104in}}%
\pgfpathlineto{\pgfqpoint{0.704536in}{0.982514in}}%
\pgfpathlineto{\pgfqpoint{0.704917in}{0.983564in}}%
\pgfpathlineto{\pgfqpoint{0.706011in}{0.988464in}}%
\pgfpathlineto{\pgfqpoint{0.706160in}{0.989444in}}%
\pgfpathlineto{\pgfqpoint{0.707271in}{0.993504in}}%
\pgfpathlineto{\pgfqpoint{0.707619in}{0.994414in}}%
\pgfpathlineto{\pgfqpoint{0.708697in}{0.998054in}}%
\pgfpathlineto{\pgfqpoint{0.709061in}{0.999104in}}%
\pgfpathlineto{\pgfqpoint{0.710172in}{1.004564in}}%
\pgfpathlineto{\pgfqpoint{0.710504in}{1.005614in}}%
\pgfpathlineto{\pgfqpoint{0.711614in}{1.009604in}}%
\pgfpathlineto{\pgfqpoint{0.711830in}{1.010584in}}%
\pgfpathlineto{\pgfqpoint{0.712874in}{1.015134in}}%
\pgfpathlineto{\pgfqpoint{0.713123in}{1.016184in}}%
\pgfpathlineto{\pgfqpoint{0.714184in}{1.020244in}}%
\pgfpathlineto{\pgfqpoint{0.714598in}{1.021294in}}%
\pgfpathlineto{\pgfqpoint{0.715709in}{1.024304in}}%
\pgfpathlineto{\pgfqpoint{0.716173in}{1.025354in}}%
\pgfpathlineto{\pgfqpoint{0.717284in}{1.029624in}}%
\pgfpathlineto{\pgfqpoint{0.717466in}{1.030604in}}%
\pgfpathlineto{\pgfqpoint{0.718560in}{1.035014in}}%
\pgfpathlineto{\pgfqpoint{0.718776in}{1.035994in}}%
\pgfpathlineto{\pgfqpoint{0.719853in}{1.041034in}}%
\pgfpathlineto{\pgfqpoint{0.720202in}{1.042084in}}%
\pgfpathlineto{\pgfqpoint{0.721312in}{1.046144in}}%
\pgfpathlineto{\pgfqpoint{0.721694in}{1.047194in}}%
\pgfpathlineto{\pgfqpoint{0.722788in}{1.050904in}}%
\pgfpathlineto{\pgfqpoint{0.723086in}{1.051954in}}%
\pgfpathlineto{\pgfqpoint{0.724180in}{1.056364in}}%
\pgfpathlineto{\pgfqpoint{0.724727in}{1.057414in}}%
\pgfpathlineto{\pgfqpoint{0.725805in}{1.061264in}}%
\pgfpathlineto{\pgfqpoint{0.726236in}{1.062314in}}%
\pgfpathlineto{\pgfqpoint{0.727280in}{1.065744in}}%
\pgfpathlineto{\pgfqpoint{0.727728in}{1.066794in}}%
\pgfpathlineto{\pgfqpoint{0.728805in}{1.069874in}}%
\pgfpathlineto{\pgfqpoint{0.729253in}{1.070924in}}%
\pgfpathlineto{\pgfqpoint{0.730364in}{1.074144in}}%
\pgfpathlineto{\pgfqpoint{0.730695in}{1.075124in}}%
\pgfpathlineto{\pgfqpoint{0.731789in}{1.079044in}}%
\pgfpathlineto{\pgfqpoint{0.732336in}{1.080024in}}%
\pgfpathlineto{\pgfqpoint{0.733447in}{1.084574in}}%
\pgfpathlineto{\pgfqpoint{0.734094in}{1.085624in}}%
\pgfpathlineto{\pgfqpoint{0.735188in}{1.089544in}}%
\pgfpathlineto{\pgfqpoint{0.735552in}{1.090594in}}%
\pgfpathlineto{\pgfqpoint{0.736597in}{1.093674in}}%
\pgfpathlineto{\pgfqpoint{0.737177in}{1.094724in}}%
\pgfpathlineto{\pgfqpoint{0.738221in}{1.097524in}}%
\pgfpathlineto{\pgfqpoint{0.738702in}{1.098574in}}%
\pgfpathlineto{\pgfqpoint{0.739780in}{1.102424in}}%
\pgfpathlineto{\pgfqpoint{0.740128in}{1.103474in}}%
\pgfpathlineto{\pgfqpoint{0.741238in}{1.106624in}}%
\pgfpathlineto{\pgfqpoint{0.741636in}{1.107674in}}%
\pgfpathlineto{\pgfqpoint{0.742697in}{1.111034in}}%
\pgfpathlineto{\pgfqpoint{0.743095in}{1.112084in}}%
\pgfpathlineto{\pgfqpoint{0.744206in}{1.115024in}}%
\pgfpathlineto{\pgfqpoint{0.744769in}{1.116074in}}%
\pgfpathlineto{\pgfqpoint{0.745051in}{1.117054in}}%
\pgfpathlineto{\pgfqpoint{0.745068in}{1.117054in}}%
\pgfpathlineto{\pgfqpoint{0.755943in}{1.118104in}}%
\pgfpathlineto{\pgfqpoint{0.757053in}{1.121814in}}%
\pgfpathlineto{\pgfqpoint{0.757600in}{1.122864in}}%
\pgfpathlineto{\pgfqpoint{0.758678in}{1.125384in}}%
\pgfpathlineto{\pgfqpoint{0.759225in}{1.126434in}}%
\pgfpathlineto{\pgfqpoint{0.760269in}{1.129094in}}%
\pgfpathlineto{\pgfqpoint{0.760817in}{1.130144in}}%
\pgfpathlineto{\pgfqpoint{0.761911in}{1.133224in}}%
\pgfpathlineto{\pgfqpoint{0.762474in}{1.134274in}}%
\pgfpathlineto{\pgfqpoint{0.763535in}{1.137844in}}%
\pgfpathlineto{\pgfqpoint{0.764165in}{1.138824in}}%
\pgfpathlineto{\pgfqpoint{0.765259in}{1.142324in}}%
\pgfpathlineto{\pgfqpoint{0.766055in}{1.143374in}}%
\pgfpathlineto{\pgfqpoint{0.767166in}{1.146314in}}%
\pgfpathlineto{\pgfqpoint{0.767696in}{1.147364in}}%
\pgfpathlineto{\pgfqpoint{0.768790in}{1.149954in}}%
\pgfpathlineto{\pgfqpoint{0.769603in}{1.151004in}}%
\pgfpathlineto{\pgfqpoint{0.770514in}{1.153314in}}%
\pgfpathlineto{\pgfqpoint{0.771161in}{1.154364in}}%
\pgfpathlineto{\pgfqpoint{0.772272in}{1.157374in}}%
\pgfpathlineto{\pgfqpoint{0.772852in}{1.158424in}}%
\pgfpathlineto{\pgfqpoint{0.773946in}{1.160664in}}%
\pgfpathlineto{\pgfqpoint{0.774609in}{1.161714in}}%
\pgfpathlineto{\pgfqpoint{0.775703in}{1.164794in}}%
\pgfpathlineto{\pgfqpoint{0.776267in}{1.165774in}}%
\pgfpathlineto{\pgfqpoint{0.777344in}{1.168714in}}%
\pgfpathlineto{\pgfqpoint{0.777759in}{1.169764in}}%
\pgfpathlineto{\pgfqpoint{0.778836in}{1.172144in}}%
\pgfpathlineto{\pgfqpoint{0.779350in}{1.173124in}}%
\pgfpathlineto{\pgfqpoint{0.780444in}{1.174524in}}%
\pgfpathlineto{\pgfqpoint{0.781008in}{1.175574in}}%
\pgfpathlineto{\pgfqpoint{0.782052in}{1.177254in}}%
\pgfpathlineto{\pgfqpoint{0.782699in}{1.178304in}}%
\pgfpathlineto{\pgfqpoint{0.783793in}{1.180894in}}%
\pgfpathlineto{\pgfqpoint{0.784290in}{1.181874in}}%
\pgfpathlineto{\pgfqpoint{0.785384in}{1.184044in}}%
\pgfpathlineto{\pgfqpoint{0.785981in}{1.185094in}}%
\pgfpathlineto{\pgfqpoint{0.787075in}{1.186914in}}%
\pgfpathlineto{\pgfqpoint{0.787755in}{1.187894in}}%
\pgfpathlineto{\pgfqpoint{0.788866in}{1.189784in}}%
\pgfpathlineto{\pgfqpoint{0.789297in}{1.190764in}}%
\pgfpathlineto{\pgfqpoint{0.790407in}{1.192794in}}%
\pgfpathlineto{\pgfqpoint{0.790689in}{1.193634in}}%
\pgfpathlineto{\pgfqpoint{0.791783in}{1.196084in}}%
\pgfpathlineto{\pgfqpoint{0.792347in}{1.197134in}}%
\pgfpathlineto{\pgfqpoint{0.793358in}{1.199724in}}%
\pgfpathlineto{\pgfqpoint{0.794054in}{1.200774in}}%
\pgfpathlineto{\pgfqpoint{0.795099in}{1.201964in}}%
\pgfpathlineto{\pgfqpoint{0.796044in}{1.203014in}}%
\pgfpathlineto{\pgfqpoint{0.797138in}{1.205534in}}%
\pgfpathlineto{\pgfqpoint{0.797718in}{1.206584in}}%
\pgfpathlineto{\pgfqpoint{0.798812in}{1.208474in}}%
\pgfpathlineto{\pgfqpoint{0.799409in}{1.209524in}}%
\pgfpathlineto{\pgfqpoint{0.800520in}{1.212114in}}%
\pgfpathlineto{\pgfqpoint{0.801382in}{1.213094in}}%
\pgfpathlineto{\pgfqpoint{0.802492in}{1.215894in}}%
\pgfpathlineto{\pgfqpoint{0.803106in}{1.216664in}}%
\pgfpathlineto{\pgfqpoint{0.804134in}{1.218554in}}%
\pgfpathlineto{\pgfqpoint{0.805128in}{1.219604in}}%
\pgfpathlineto{\pgfqpoint{0.806222in}{1.221774in}}%
\pgfpathlineto{\pgfqpoint{0.806786in}{1.222614in}}%
\pgfpathlineto{\pgfqpoint{0.807897in}{1.225484in}}%
\pgfpathlineto{\pgfqpoint{0.808659in}{1.226534in}}%
\pgfpathlineto{\pgfqpoint{0.809753in}{1.228284in}}%
\pgfpathlineto{\pgfqpoint{0.810267in}{1.229334in}}%
\pgfpathlineto{\pgfqpoint{0.811378in}{1.231224in}}%
\pgfpathlineto{\pgfqpoint{0.812654in}{1.232274in}}%
\pgfpathlineto{\pgfqpoint{0.813682in}{1.233744in}}%
\pgfpathlineto{\pgfqpoint{0.814180in}{1.234794in}}%
\pgfpathlineto{\pgfqpoint{0.815290in}{1.236614in}}%
\pgfpathlineto{\pgfqpoint{0.816053in}{1.237664in}}%
\pgfpathlineto{\pgfqpoint{0.816948in}{1.239484in}}%
\pgfpathlineto{\pgfqpoint{0.818109in}{1.240464in}}%
\pgfpathlineto{\pgfqpoint{0.819203in}{1.242424in}}%
\pgfpathlineto{\pgfqpoint{0.819949in}{1.243404in}}%
\pgfpathlineto{\pgfqpoint{0.821043in}{1.245364in}}%
\pgfpathlineto{\pgfqpoint{0.821838in}{1.246414in}}%
\pgfpathlineto{\pgfqpoint{0.822899in}{1.248374in}}%
\pgfpathlineto{\pgfqpoint{0.823828in}{1.249424in}}%
\pgfpathlineto{\pgfqpoint{0.824905in}{1.251314in}}%
\pgfpathlineto{\pgfqpoint{0.826049in}{1.252364in}}%
\pgfpathlineto{\pgfqpoint{0.827110in}{1.254044in}}%
\pgfpathlineto{\pgfqpoint{0.828204in}{1.255024in}}%
\pgfpathlineto{\pgfqpoint{0.829298in}{1.257404in}}%
\pgfpathlineto{\pgfqpoint{0.830094in}{1.258454in}}%
\pgfpathlineto{\pgfqpoint{0.831188in}{1.259784in}}%
\pgfpathlineto{\pgfqpoint{0.831901in}{1.260834in}}%
\pgfpathlineto{\pgfqpoint{0.832979in}{1.261954in}}%
\pgfpathlineto{\pgfqpoint{0.833841in}{1.263004in}}%
\pgfpathlineto{\pgfqpoint{0.834769in}{1.263774in}}%
\pgfpathlineto{\pgfqpoint{0.835697in}{1.264824in}}%
\pgfpathlineto{\pgfqpoint{0.836659in}{1.266014in}}%
\pgfpathlineto{\pgfqpoint{0.838234in}{1.267064in}}%
\pgfpathlineto{\pgfqpoint{0.839295in}{1.268814in}}%
\pgfpathlineto{\pgfqpoint{0.840438in}{1.269864in}}%
\pgfpathlineto{\pgfqpoint{0.841549in}{1.271824in}}%
\pgfpathlineto{\pgfqpoint{0.842594in}{1.272874in}}%
\pgfpathlineto{\pgfqpoint{0.843688in}{1.274414in}}%
\pgfpathlineto{\pgfqpoint{0.844881in}{1.275464in}}%
\pgfpathlineto{\pgfqpoint{0.845992in}{1.277844in}}%
\pgfpathlineto{\pgfqpoint{0.846705in}{1.278894in}}%
\pgfpathlineto{\pgfqpoint{0.847633in}{1.280294in}}%
\pgfpathlineto{\pgfqpoint{0.848644in}{1.281274in}}%
\pgfpathlineto{\pgfqpoint{0.849606in}{1.282954in}}%
\pgfpathlineto{\pgfqpoint{0.850882in}{1.284004in}}%
\pgfpathlineto{\pgfqpoint{0.851960in}{1.285194in}}%
\pgfpathlineto{\pgfqpoint{0.853120in}{1.286244in}}%
\pgfpathlineto{\pgfqpoint{0.854198in}{1.287644in}}%
\pgfpathlineto{\pgfqpoint{0.855176in}{1.288694in}}%
\pgfpathlineto{\pgfqpoint{0.856237in}{1.290234in}}%
\pgfpathlineto{\pgfqpoint{0.856917in}{1.291144in}}%
\pgfpathlineto{\pgfqpoint{0.858027in}{1.292754in}}%
\pgfpathlineto{\pgfqpoint{0.859287in}{1.293804in}}%
\pgfpathlineto{\pgfqpoint{0.860232in}{1.294714in}}%
\pgfpathlineto{\pgfqpoint{0.861160in}{1.295764in}}%
\pgfpathlineto{\pgfqpoint{0.862155in}{1.296814in}}%
\pgfpathlineto{\pgfqpoint{0.863746in}{1.297794in}}%
\pgfpathlineto{\pgfqpoint{0.864807in}{1.299054in}}%
\pgfpathlineto{\pgfqpoint{0.866084in}{1.300104in}}%
\pgfpathlineto{\pgfqpoint{0.867128in}{1.301854in}}%
\pgfpathlineto{\pgfqpoint{0.867775in}{1.302904in}}%
\pgfpathlineto{\pgfqpoint{0.868886in}{1.304164in}}%
\pgfpathlineto{\pgfqpoint{0.869897in}{1.305144in}}%
\pgfpathlineto{\pgfqpoint{0.870991in}{1.306614in}}%
\pgfpathlineto{\pgfqpoint{0.872218in}{1.307664in}}%
\pgfpathlineto{\pgfqpoint{0.873312in}{1.309624in}}%
\pgfpathlineto{\pgfqpoint{0.874572in}{1.310674in}}%
\pgfpathlineto{\pgfqpoint{0.875616in}{1.312284in}}%
\pgfpathlineto{\pgfqpoint{0.876263in}{1.313334in}}%
\pgfpathlineto{\pgfqpoint{0.877373in}{1.314804in}}%
\pgfpathlineto{\pgfqpoint{0.878036in}{1.315854in}}%
\pgfpathlineto{\pgfqpoint{0.879081in}{1.316974in}}%
\pgfpathlineto{\pgfqpoint{0.880191in}{1.318024in}}%
\pgfpathlineto{\pgfqpoint{0.881252in}{1.319564in}}%
\pgfpathlineto{\pgfqpoint{0.882927in}{1.320544in}}%
\pgfpathlineto{\pgfqpoint{0.883954in}{1.321384in}}%
\pgfpathlineto{\pgfqpoint{0.885165in}{1.322434in}}%
\pgfpathlineto{\pgfqpoint{0.886226in}{1.323624in}}%
\pgfpathlineto{\pgfqpoint{0.887121in}{1.324674in}}%
\pgfpathlineto{\pgfqpoint{0.888215in}{1.325864in}}%
\pgfpathlineto{\pgfqpoint{0.889624in}{1.326914in}}%
\pgfpathlineto{\pgfqpoint{0.890619in}{1.328104in}}%
\pgfpathlineto{\pgfqpoint{0.892044in}{1.329154in}}%
\pgfpathlineto{\pgfqpoint{0.892857in}{1.330064in}}%
\pgfpathlineto{\pgfqpoint{0.894100in}{1.331044in}}%
\pgfpathlineto{\pgfqpoint{0.894746in}{1.331814in}}%
\pgfpathlineto{\pgfqpoint{0.896089in}{1.332864in}}%
\pgfpathlineto{\pgfqpoint{0.897200in}{1.333984in}}%
\pgfpathlineto{\pgfqpoint{0.898327in}{1.335034in}}%
\pgfpathlineto{\pgfqpoint{0.899338in}{1.336714in}}%
\pgfpathlineto{\pgfqpoint{0.900267in}{1.337694in}}%
\pgfpathlineto{\pgfqpoint{0.901311in}{1.338534in}}%
\pgfpathlineto{\pgfqpoint{0.902554in}{1.339584in}}%
\pgfpathlineto{\pgfqpoint{0.903649in}{1.340494in}}%
\pgfpathlineto{\pgfqpoint{0.905240in}{1.341544in}}%
\pgfpathlineto{\pgfqpoint{0.906334in}{1.342874in}}%
\pgfpathlineto{\pgfqpoint{0.907743in}{1.343924in}}%
\pgfpathlineto{\pgfqpoint{0.908854in}{1.344764in}}%
\pgfpathlineto{\pgfqpoint{0.910048in}{1.345814in}}%
\pgfpathlineto{\pgfqpoint{0.911142in}{1.346654in}}%
\pgfpathlineto{\pgfqpoint{0.912169in}{1.347704in}}%
\pgfpathlineto{\pgfqpoint{0.913114in}{1.348754in}}%
\pgfpathlineto{\pgfqpoint{0.914242in}{1.349804in}}%
\pgfpathlineto{\pgfqpoint{0.915004in}{1.350924in}}%
\pgfpathlineto{\pgfqpoint{0.916894in}{1.351974in}}%
\pgfpathlineto{\pgfqpoint{0.917988in}{1.353234in}}%
\pgfpathlineto{\pgfqpoint{0.919646in}{1.354284in}}%
\pgfpathlineto{\pgfqpoint{0.920757in}{1.356034in}}%
\pgfpathlineto{\pgfqpoint{0.922166in}{1.357084in}}%
\pgfpathlineto{\pgfqpoint{0.923260in}{1.358204in}}%
\pgfpathlineto{\pgfqpoint{0.925332in}{1.359254in}}%
\pgfpathlineto{\pgfqpoint{0.926376in}{1.360164in}}%
\pgfpathlineto{\pgfqpoint{0.927686in}{1.361214in}}%
\pgfpathlineto{\pgfqpoint{0.928780in}{1.361914in}}%
\pgfpathlineto{\pgfqpoint{0.930886in}{1.362964in}}%
\pgfpathlineto{\pgfqpoint{0.931980in}{1.364294in}}%
\pgfpathlineto{\pgfqpoint{0.932941in}{1.365344in}}%
\pgfpathlineto{\pgfqpoint{0.934035in}{1.366604in}}%
\pgfpathlineto{\pgfqpoint{0.935925in}{1.367654in}}%
\pgfpathlineto{\pgfqpoint{0.937003in}{1.368984in}}%
\pgfpathlineto{\pgfqpoint{0.939307in}{1.370034in}}%
\pgfpathlineto{\pgfqpoint{0.940302in}{1.371014in}}%
\pgfpathlineto{\pgfqpoint{0.941876in}{1.372064in}}%
\pgfpathlineto{\pgfqpoint{0.942871in}{1.372694in}}%
\pgfpathlineto{\pgfqpoint{0.944562in}{1.373674in}}%
\pgfpathlineto{\pgfqpoint{0.945573in}{1.374934in}}%
\pgfpathlineto{\pgfqpoint{0.947314in}{1.375984in}}%
\pgfpathlineto{\pgfqpoint{0.948159in}{1.376824in}}%
\pgfpathlineto{\pgfqpoint{0.950663in}{1.377874in}}%
\pgfpathlineto{\pgfqpoint{0.951707in}{1.378504in}}%
\pgfpathlineto{\pgfqpoint{0.953083in}{1.379484in}}%
\pgfpathlineto{\pgfqpoint{0.954127in}{1.380394in}}%
\pgfpathlineto{\pgfqpoint{0.956183in}{1.381444in}}%
\pgfpathlineto{\pgfqpoint{0.957244in}{1.382914in}}%
\pgfpathlineto{\pgfqpoint{0.958736in}{1.383964in}}%
\pgfpathlineto{\pgfqpoint{0.959714in}{1.384874in}}%
\pgfpathlineto{\pgfqpoint{0.961902in}{1.385924in}}%
\pgfpathlineto{\pgfqpoint{0.962781in}{1.386414in}}%
\pgfpathlineto{\pgfqpoint{0.965599in}{1.387464in}}%
\pgfpathlineto{\pgfqpoint{0.966643in}{1.388514in}}%
\pgfpathlineto{\pgfqpoint{0.968069in}{1.389564in}}%
\pgfpathlineto{\pgfqpoint{0.968931in}{1.390404in}}%
\pgfpathlineto{\pgfqpoint{0.971733in}{1.391454in}}%
\pgfpathlineto{\pgfqpoint{0.972843in}{1.391874in}}%
\pgfpathlineto{\pgfqpoint{0.974286in}{1.392924in}}%
\pgfpathlineto{\pgfqpoint{0.975197in}{1.393554in}}%
\pgfpathlineto{\pgfqpoint{0.977336in}{1.394604in}}%
\pgfpathlineto{\pgfqpoint{0.978413in}{1.395654in}}%
\pgfpathlineto{\pgfqpoint{0.979607in}{1.396704in}}%
\pgfpathlineto{\pgfqpoint{0.980635in}{1.397614in}}%
\pgfpathlineto{\pgfqpoint{0.981729in}{1.398664in}}%
\pgfpathlineto{\pgfqpoint{0.982690in}{1.399224in}}%
\pgfpathlineto{\pgfqpoint{0.984365in}{1.400274in}}%
\pgfpathlineto{\pgfqpoint{0.985343in}{1.401044in}}%
\pgfpathlineto{\pgfqpoint{0.987050in}{1.402094in}}%
\pgfpathlineto{\pgfqpoint{0.987912in}{1.403004in}}%
\pgfpathlineto{\pgfqpoint{0.990051in}{1.404054in}}%
\pgfpathlineto{\pgfqpoint{0.990631in}{1.404474in}}%
\pgfpathlineto{\pgfqpoint{0.992935in}{1.405524in}}%
\pgfpathlineto{\pgfqpoint{0.993847in}{1.406294in}}%
\pgfpathlineto{\pgfqpoint{0.995787in}{1.407344in}}%
\pgfpathlineto{\pgfqpoint{0.996798in}{1.408534in}}%
\pgfpathlineto{\pgfqpoint{0.998456in}{1.409444in}}%
\pgfpathlineto{\pgfqpoint{0.999533in}{1.410424in}}%
\pgfpathlineto{\pgfqpoint{1.001224in}{1.411404in}}%
\pgfpathlineto{\pgfqpoint{1.002235in}{1.412314in}}%
\pgfpathlineto{\pgfqpoint{1.003545in}{1.413364in}}%
\pgfpathlineto{\pgfqpoint{1.004423in}{1.414274in}}%
\pgfpathlineto{\pgfqpoint{1.005882in}{1.415324in}}%
\pgfpathlineto{\pgfqpoint{1.006678in}{1.415954in}}%
\pgfpathlineto{\pgfqpoint{1.009165in}{1.417004in}}%
\pgfpathlineto{\pgfqpoint{1.010226in}{1.417774in}}%
\pgfpathlineto{\pgfqpoint{1.012861in}{1.418824in}}%
\pgfpathlineto{\pgfqpoint{1.013972in}{1.419524in}}%
\pgfpathlineto{\pgfqpoint{1.016442in}{1.420504in}}%
\pgfpathlineto{\pgfqpoint{1.017437in}{1.421274in}}%
\pgfpathlineto{\pgfqpoint{1.019890in}{1.422324in}}%
\pgfpathlineto{\pgfqpoint{1.020852in}{1.423444in}}%
\pgfpathlineto{\pgfqpoint{1.022626in}{1.424494in}}%
\pgfpathlineto{\pgfqpoint{1.023488in}{1.425334in}}%
\pgfpathlineto{\pgfqpoint{1.026405in}{1.426384in}}%
\pgfpathlineto{\pgfqpoint{1.027516in}{1.427224in}}%
\pgfpathlineto{\pgfqpoint{1.029787in}{1.428274in}}%
\pgfpathlineto{\pgfqpoint{1.030831in}{1.429184in}}%
\pgfpathlineto{\pgfqpoint{1.032887in}{1.430234in}}%
\pgfpathlineto{\pgfqpoint{1.033882in}{1.431074in}}%
\pgfpathlineto{\pgfqpoint{1.036667in}{1.432124in}}%
\pgfpathlineto{\pgfqpoint{1.037711in}{1.432894in}}%
\pgfpathlineto{\pgfqpoint{1.041872in}{1.433944in}}%
\pgfpathlineto{\pgfqpoint{1.042834in}{1.434504in}}%
\pgfpathlineto{\pgfqpoint{1.046530in}{1.435554in}}%
\pgfpathlineto{\pgfqpoint{1.047575in}{1.436044in}}%
\pgfpathlineto{\pgfqpoint{1.049514in}{1.437094in}}%
\pgfpathlineto{\pgfqpoint{1.050625in}{1.437794in}}%
\pgfpathlineto{\pgfqpoint{1.053095in}{1.438844in}}%
\pgfpathlineto{\pgfqpoint{1.054090in}{1.439544in}}%
\pgfpathlineto{\pgfqpoint{1.056692in}{1.440594in}}%
\pgfpathlineto{\pgfqpoint{1.057704in}{1.441364in}}%
\pgfpathlineto{\pgfqpoint{1.061301in}{1.442414in}}%
\pgfpathlineto{\pgfqpoint{1.062329in}{1.443044in}}%
\pgfpathlineto{\pgfqpoint{1.065893in}{1.444094in}}%
\pgfpathlineto{\pgfqpoint{1.066987in}{1.444584in}}%
\pgfpathlineto{\pgfqpoint{1.068976in}{1.445634in}}%
\pgfpathlineto{\pgfqpoint{1.070037in}{1.446544in}}%
\pgfpathlineto{\pgfqpoint{1.072491in}{1.447594in}}%
\pgfpathlineto{\pgfqpoint{1.073369in}{1.448224in}}%
\pgfpathlineto{\pgfqpoint{1.075840in}{1.449274in}}%
\pgfpathlineto{\pgfqpoint{1.076652in}{1.449694in}}%
\pgfpathlineto{\pgfqpoint{1.079006in}{1.450744in}}%
\pgfpathlineto{\pgfqpoint{1.079967in}{1.451234in}}%
\pgfpathlineto{\pgfqpoint{1.083001in}{1.452214in}}%
\pgfpathlineto{\pgfqpoint{1.084112in}{1.452914in}}%
\pgfpathlineto{\pgfqpoint{1.087726in}{1.453894in}}%
\pgfpathlineto{\pgfqpoint{1.088438in}{1.454384in}}%
\pgfpathlineto{\pgfqpoint{1.092119in}{1.455434in}}%
\pgfpathlineto{\pgfqpoint{1.093064in}{1.456064in}}%
\pgfpathlineto{\pgfqpoint{1.096346in}{1.457114in}}%
\pgfpathlineto{\pgfqpoint{1.097274in}{1.457674in}}%
\pgfpathlineto{\pgfqpoint{1.100092in}{1.458724in}}%
\pgfpathlineto{\pgfqpoint{1.100689in}{1.459004in}}%
\pgfpathlineto{\pgfqpoint{1.104038in}{1.460054in}}%
\pgfpathlineto{\pgfqpoint{1.105149in}{1.460404in}}%
\pgfpathlineto{\pgfqpoint{1.109989in}{1.461454in}}%
\pgfpathlineto{\pgfqpoint{1.111083in}{1.462154in}}%
\pgfpathlineto{\pgfqpoint{1.114813in}{1.463204in}}%
\pgfpathlineto{\pgfqpoint{1.115858in}{1.463624in}}%
\pgfpathlineto{\pgfqpoint{1.119389in}{1.464674in}}%
\pgfpathlineto{\pgfqpoint{1.120433in}{1.465094in}}%
\pgfpathlineto{\pgfqpoint{1.122754in}{1.466144in}}%
\pgfpathlineto{\pgfqpoint{1.123732in}{1.466494in}}%
\pgfpathlineto{\pgfqpoint{1.127910in}{1.467544in}}%
\pgfpathlineto{\pgfqpoint{1.128556in}{1.468034in}}%
\pgfpathlineto{\pgfqpoint{1.131010in}{1.469084in}}%
\pgfpathlineto{\pgfqpoint{1.131822in}{1.469924in}}%
\pgfpathlineto{\pgfqpoint{1.135187in}{1.470974in}}%
\pgfpathlineto{\pgfqpoint{1.135204in}{1.471184in}}%
\pgfpathlineto{\pgfqpoint{1.140956in}{1.472234in}}%
\pgfpathlineto{\pgfqpoint{1.141735in}{1.472584in}}%
\pgfpathlineto{\pgfqpoint{1.145598in}{1.473634in}}%
\pgfpathlineto{\pgfqpoint{1.146675in}{1.474194in}}%
\pgfpathlineto{\pgfqpoint{1.150803in}{1.475244in}}%
\pgfpathlineto{\pgfqpoint{1.151516in}{1.475874in}}%
\pgfpathlineto{\pgfqpoint{1.156058in}{1.476924in}}%
\pgfpathlineto{\pgfqpoint{1.157152in}{1.477414in}}%
\pgfpathlineto{\pgfqpoint{1.162407in}{1.478464in}}%
\pgfpathlineto{\pgfqpoint{1.163452in}{1.479094in}}%
\pgfpathlineto{\pgfqpoint{1.168060in}{1.480144in}}%
\pgfpathlineto{\pgfqpoint{1.168873in}{1.480494in}}%
\pgfpathlineto{\pgfqpoint{1.171923in}{1.481474in}}%
\pgfpathlineto{\pgfqpoint{1.172835in}{1.481754in}}%
\pgfpathlineto{\pgfqpoint{1.177161in}{1.482804in}}%
\pgfpathlineto{\pgfqpoint{1.178156in}{1.483294in}}%
\pgfpathlineto{\pgfqpoint{1.181223in}{1.484344in}}%
\pgfpathlineto{\pgfqpoint{1.181273in}{1.484554in}}%
\pgfpathlineto{\pgfqpoint{1.188633in}{1.485604in}}%
\pgfpathlineto{\pgfqpoint{1.189512in}{1.485884in}}%
\pgfpathlineto{\pgfqpoint{1.193341in}{1.486934in}}%
\pgfpathlineto{\pgfqpoint{1.194203in}{1.487354in}}%
\pgfpathlineto{\pgfqpoint{1.198928in}{1.488404in}}%
\pgfpathlineto{\pgfqpoint{1.199823in}{1.488964in}}%
\pgfpathlineto{\pgfqpoint{1.201895in}{1.489944in}}%
\pgfpathlineto{\pgfqpoint{1.202343in}{1.490224in}}%
\pgfpathlineto{\pgfqpoint{1.205343in}{1.491274in}}%
\pgfpathlineto{\pgfqpoint{1.206404in}{1.491554in}}%
\pgfpathlineto{\pgfqpoint{1.209670in}{1.492604in}}%
\pgfpathlineto{\pgfqpoint{1.210748in}{1.493024in}}%
\pgfpathlineto{\pgfqpoint{1.215887in}{1.494074in}}%
\pgfpathlineto{\pgfqpoint{1.216865in}{1.494214in}}%
\pgfpathlineto{\pgfqpoint{1.223860in}{1.495264in}}%
\pgfpathlineto{\pgfqpoint{1.224739in}{1.495684in}}%
\pgfpathlineto{\pgfqpoint{1.229994in}{1.496734in}}%
\pgfpathlineto{\pgfqpoint{1.230823in}{1.497014in}}%
\pgfpathlineto{\pgfqpoint{1.234354in}{1.498064in}}%
\pgfpathlineto{\pgfqpoint{1.235117in}{1.498344in}}%
\pgfpathlineto{\pgfqpoint{1.240256in}{1.499394in}}%
\pgfpathlineto{\pgfqpoint{1.241151in}{1.499674in}}%
\pgfpathlineto{\pgfqpoint{1.245610in}{1.500724in}}%
\pgfpathlineto{\pgfqpoint{1.246721in}{1.501144in}}%
\pgfpathlineto{\pgfqpoint{1.248942in}{1.502194in}}%
\pgfpathlineto{\pgfqpoint{1.250053in}{1.502754in}}%
\pgfpathlineto{\pgfqpoint{1.254429in}{1.503804in}}%
\pgfpathlineto{\pgfqpoint{1.254728in}{1.504084in}}%
\pgfpathlineto{\pgfqpoint{1.260845in}{1.505134in}}%
\pgfpathlineto{\pgfqpoint{1.261723in}{1.505414in}}%
\pgfpathlineto{\pgfqpoint{1.266697in}{1.506464in}}%
\pgfpathlineto{\pgfqpoint{1.267807in}{1.506744in}}%
\pgfpathlineto{\pgfqpoint{1.271338in}{1.507724in}}%
\pgfpathlineto{\pgfqpoint{1.272200in}{1.508214in}}%
\pgfpathlineto{\pgfqpoint{1.277936in}{1.509264in}}%
\pgfpathlineto{\pgfqpoint{1.278036in}{1.509404in}}%
\pgfpathlineto{\pgfqpoint{1.284203in}{1.510454in}}%
\pgfpathlineto{\pgfqpoint{1.285131in}{1.510804in}}%
\pgfpathlineto{\pgfqpoint{1.289657in}{1.511854in}}%
\pgfpathlineto{\pgfqpoint{1.290585in}{1.512274in}}%
\pgfpathlineto{\pgfqpoint{1.295940in}{1.513324in}}%
\pgfpathlineto{\pgfqpoint{1.296354in}{1.513534in}}%
\pgfpathlineto{\pgfqpoint{1.299703in}{1.514584in}}%
\pgfpathlineto{\pgfqpoint{1.300664in}{1.514934in}}%
\pgfpathlineto{\pgfqpoint{1.305389in}{1.515984in}}%
\pgfpathlineto{\pgfqpoint{1.306118in}{1.516334in}}%
\pgfpathlineto{\pgfqpoint{1.311290in}{1.517384in}}%
\pgfpathlineto{\pgfqpoint{1.312318in}{1.517734in}}%
\pgfpathlineto{\pgfqpoint{1.318104in}{1.518784in}}%
\pgfpathlineto{\pgfqpoint{1.318899in}{1.519274in}}%
\pgfpathlineto{\pgfqpoint{1.323873in}{1.520324in}}%
\pgfpathlineto{\pgfqpoint{1.324950in}{1.520674in}}%
\pgfpathlineto{\pgfqpoint{1.329575in}{1.521654in}}%
\pgfpathlineto{\pgfqpoint{1.330305in}{1.521864in}}%
\pgfpathlineto{\pgfqpoint{1.336521in}{1.522844in}}%
\pgfpathlineto{\pgfqpoint{1.336836in}{1.523194in}}%
\pgfpathlineto{\pgfqpoint{1.344180in}{1.524244in}}%
\pgfpathlineto{\pgfqpoint{1.345191in}{1.524524in}}%
\pgfpathlineto{\pgfqpoint{1.351905in}{1.525574in}}%
\pgfpathlineto{\pgfqpoint{1.352602in}{1.525994in}}%
\pgfpathlineto{\pgfqpoint{1.359050in}{1.527044in}}%
\pgfpathlineto{\pgfqpoint{1.359962in}{1.527394in}}%
\pgfpathlineto{\pgfqpoint{1.365897in}{1.528444in}}%
\pgfpathlineto{\pgfqpoint{1.366991in}{1.528654in}}%
\pgfpathlineto{\pgfqpoint{1.378463in}{1.529704in}}%
\pgfpathlineto{\pgfqpoint{1.379092in}{1.530054in}}%
\pgfpathlineto{\pgfqpoint{1.386221in}{1.531104in}}%
\pgfpathlineto{\pgfqpoint{1.387298in}{1.531314in}}%
\pgfpathlineto{\pgfqpoint{1.393366in}{1.532364in}}%
\pgfpathlineto{\pgfqpoint{1.393880in}{1.532714in}}%
\pgfpathlineto{\pgfqpoint{1.403760in}{1.533764in}}%
\pgfpathlineto{\pgfqpoint{1.404340in}{1.533904in}}%
\pgfpathlineto{\pgfqpoint{1.412529in}{1.534954in}}%
\pgfpathlineto{\pgfqpoint{1.412994in}{1.535164in}}%
\pgfpathlineto{\pgfqpoint{1.422277in}{1.536214in}}%
\pgfpathlineto{\pgfqpoint{1.422807in}{1.536354in}}%
\pgfpathlineto{\pgfqpoint{1.428460in}{1.537404in}}%
\pgfpathlineto{\pgfqpoint{1.429157in}{1.537614in}}%
\pgfpathlineto{\pgfqpoint{1.435340in}{1.538664in}}%
\pgfpathlineto{\pgfqpoint{1.436351in}{1.539014in}}%
\pgfpathlineto{\pgfqpoint{1.443861in}{1.540064in}}%
\pgfpathlineto{\pgfqpoint{1.443877in}{1.540274in}}%
\pgfpathlineto{\pgfqpoint{1.450940in}{1.541324in}}%
\pgfpathlineto{\pgfqpoint{1.451006in}{1.541464in}}%
\pgfpathlineto{\pgfqpoint{1.462776in}{1.542514in}}%
\pgfpathlineto{\pgfqpoint{1.463439in}{1.542724in}}%
\pgfpathlineto{\pgfqpoint{1.473817in}{1.543774in}}%
\pgfpathlineto{\pgfqpoint{1.474347in}{1.544124in}}%
\pgfpathlineto{\pgfqpoint{1.482304in}{1.545174in}}%
\pgfpathlineto{\pgfqpoint{1.482669in}{1.545314in}}%
\pgfpathlineto{\pgfqpoint{1.493892in}{1.546364in}}%
\pgfpathlineto{\pgfqpoint{1.494257in}{1.546504in}}%
\pgfpathlineto{\pgfqpoint{1.503756in}{1.547554in}}%
\pgfpathlineto{\pgfqpoint{1.503921in}{1.547834in}}%
\pgfpathlineto{\pgfqpoint{1.514382in}{1.548884in}}%
\pgfpathlineto{\pgfqpoint{1.515227in}{1.549234in}}%
\pgfpathlineto{\pgfqpoint{1.524046in}{1.550284in}}%
\pgfpathlineto{\pgfqpoint{1.524378in}{1.550564in}}%
\pgfpathlineto{\pgfqpoint{1.535187in}{1.551614in}}%
\pgfpathlineto{\pgfqpoint{1.535817in}{1.551754in}}%
\pgfpathlineto{\pgfqpoint{1.546012in}{1.552804in}}%
\pgfpathlineto{\pgfqpoint{1.547122in}{1.553154in}}%
\pgfpathlineto{\pgfqpoint{1.557450in}{1.554204in}}%
\pgfpathlineto{\pgfqpoint{1.557450in}{1.554274in}}%
\pgfpathlineto{\pgfqpoint{1.571690in}{1.555324in}}%
\pgfpathlineto{\pgfqpoint{1.571790in}{1.555534in}}%
\pgfpathlineto{\pgfqpoint{1.581405in}{1.556584in}}%
\pgfpathlineto{\pgfqpoint{1.581653in}{1.556794in}}%
\pgfpathlineto{\pgfqpoint{1.595396in}{1.557844in}}%
\pgfpathlineto{\pgfqpoint{1.595844in}{1.558054in}}%
\pgfpathlineto{\pgfqpoint{1.606503in}{1.559104in}}%
\pgfpathlineto{\pgfqpoint{1.607514in}{1.559384in}}%
\pgfpathlineto{\pgfqpoint{1.617610in}{1.560434in}}%
\pgfpathlineto{\pgfqpoint{1.618721in}{1.560644in}}%
\pgfpathlineto{\pgfqpoint{1.630889in}{1.561694in}}%
\pgfpathlineto{\pgfqpoint{1.631883in}{1.561834in}}%
\pgfpathlineto{\pgfqpoint{1.652605in}{1.562884in}}%
\pgfpathlineto{\pgfqpoint{1.653600in}{1.563024in}}%
\pgfpathlineto{\pgfqpoint{1.663812in}{1.564074in}}%
\pgfpathlineto{\pgfqpoint{1.663861in}{1.564214in}}%
\pgfpathlineto{\pgfqpoint{1.672283in}{1.565264in}}%
\pgfpathlineto{\pgfqpoint{1.672382in}{1.565404in}}%
\pgfpathlineto{\pgfqpoint{1.683406in}{1.566454in}}%
\pgfpathlineto{\pgfqpoint{1.684467in}{1.566664in}}%
\pgfpathlineto{\pgfqpoint{1.696586in}{1.567714in}}%
\pgfpathlineto{\pgfqpoint{1.696834in}{1.567854in}}%
\pgfpathlineto{\pgfqpoint{1.710212in}{1.568904in}}%
\pgfpathlineto{\pgfqpoint{1.710909in}{1.569114in}}%
\pgfpathlineto{\pgfqpoint{1.723706in}{1.570164in}}%
\pgfpathlineto{\pgfqpoint{1.724718in}{1.570304in}}%
\pgfpathlineto{\pgfqpoint{1.739090in}{1.571354in}}%
\pgfpathlineto{\pgfqpoint{1.739422in}{1.571494in}}%
\pgfpathlineto{\pgfqpoint{1.756828in}{1.572544in}}%
\pgfpathlineto{\pgfqpoint{1.757591in}{1.572754in}}%
\pgfpathlineto{\pgfqpoint{1.769825in}{1.573804in}}%
\pgfpathlineto{\pgfqpoint{1.770339in}{1.573944in}}%
\pgfpathlineto{\pgfqpoint{1.782242in}{1.574994in}}%
\pgfpathlineto{\pgfqpoint{1.782374in}{1.575134in}}%
\pgfpathlineto{\pgfqpoint{1.795288in}{1.576184in}}%
\pgfpathlineto{\pgfqpoint{1.795868in}{1.576394in}}%
\pgfpathlineto{\pgfqpoint{1.809910in}{1.577444in}}%
\pgfpathlineto{\pgfqpoint{1.810987in}{1.577724in}}%
\pgfpathlineto{\pgfqpoint{1.828360in}{1.578774in}}%
\pgfpathlineto{\pgfqpoint{1.829057in}{1.578984in}}%
\pgfpathlineto{\pgfqpoint{1.842236in}{1.580034in}}%
\pgfpathlineto{\pgfqpoint{1.842368in}{1.580174in}}%
\pgfpathlineto{\pgfqpoint{1.859410in}{1.581224in}}%
\pgfpathlineto{\pgfqpoint{1.859891in}{1.581364in}}%
\pgfpathlineto{\pgfqpoint{1.873683in}{1.582414in}}%
\pgfpathlineto{\pgfqpoint{1.873683in}{1.582484in}}%
\pgfpathlineto{\pgfqpoint{1.894190in}{1.583534in}}%
\pgfpathlineto{\pgfqpoint{1.894637in}{1.583674in}}%
\pgfpathlineto{\pgfqpoint{1.915542in}{1.584724in}}%
\pgfpathlineto{\pgfqpoint{1.916155in}{1.584864in}}%
\pgfpathlineto{\pgfqpoint{1.929334in}{1.585914in}}%
\pgfpathlineto{\pgfqpoint{1.930345in}{1.586054in}}%
\pgfpathlineto{\pgfqpoint{1.945680in}{1.587104in}}%
\pgfpathlineto{\pgfqpoint{1.945680in}{1.587174in}}%
\pgfpathlineto{\pgfqpoint{1.958245in}{1.588224in}}%
\pgfpathlineto{\pgfqpoint{1.959074in}{1.588434in}}%
\pgfpathlineto{\pgfqpoint{1.971541in}{1.589484in}}%
\pgfpathlineto{\pgfqpoint{1.972005in}{1.589624in}}%
\pgfpathlineto{\pgfqpoint{1.980807in}{1.590674in}}%
\pgfpathlineto{\pgfqpoint{1.981636in}{1.590954in}}%
\pgfpathlineto{\pgfqpoint{1.988947in}{1.592004in}}%
\pgfpathlineto{\pgfqpoint{1.988947in}{1.592074in}}%
\pgfpathlineto{\pgfqpoint{1.998313in}{1.593124in}}%
\pgfpathlineto{\pgfqpoint{1.999391in}{1.593334in}}%
\pgfpathlineto{\pgfqpoint{2.006536in}{1.594384in}}%
\pgfpathlineto{\pgfqpoint{2.007414in}{1.594524in}}%
\pgfpathlineto{\pgfqpoint{2.013349in}{1.595574in}}%
\pgfpathlineto{\pgfqpoint{2.014195in}{1.595854in}}%
\pgfpathlineto{\pgfqpoint{2.021936in}{1.596904in}}%
\pgfpathlineto{\pgfqpoint{2.022318in}{1.597044in}}%
\pgfpathlineto{\pgfqpoint{2.028136in}{1.598094in}}%
\pgfpathlineto{\pgfqpoint{2.029230in}{1.598584in}}%
\pgfpathlineto{\pgfqpoint{2.032015in}{1.599634in}}%
\pgfpathlineto{\pgfqpoint{2.033126in}{1.601944in}}%
\pgfpathlineto{\pgfqpoint{2.033126in}{1.601944in}}%
\pgfusepath{stroke}%
\end{pgfscope}%
\begin{pgfscope}%
\pgfsetrectcap%
\pgfsetmiterjoin%
\pgfsetlinewidth{0.803000pt}%
\definecolor{currentstroke}{rgb}{0.000000,0.000000,0.000000}%
\pgfsetstrokecolor{currentstroke}%
\pgfsetdash{}{0pt}%
\pgfpathmoveto{\pgfqpoint{0.553581in}{0.499444in}}%
\pgfpathlineto{\pgfqpoint{0.553581in}{1.654444in}}%
\pgfusepath{stroke}%
\end{pgfscope}%
\begin{pgfscope}%
\pgfsetrectcap%
\pgfsetmiterjoin%
\pgfsetlinewidth{0.803000pt}%
\definecolor{currentstroke}{rgb}{0.000000,0.000000,0.000000}%
\pgfsetstrokecolor{currentstroke}%
\pgfsetdash{}{0pt}%
\pgfpathmoveto{\pgfqpoint{2.103581in}{0.499444in}}%
\pgfpathlineto{\pgfqpoint{2.103581in}{1.654444in}}%
\pgfusepath{stroke}%
\end{pgfscope}%
\begin{pgfscope}%
\pgfsetrectcap%
\pgfsetmiterjoin%
\pgfsetlinewidth{0.803000pt}%
\definecolor{currentstroke}{rgb}{0.000000,0.000000,0.000000}%
\pgfsetstrokecolor{currentstroke}%
\pgfsetdash{}{0pt}%
\pgfpathmoveto{\pgfqpoint{0.553581in}{0.499444in}}%
\pgfpathlineto{\pgfqpoint{2.103581in}{0.499444in}}%
\pgfusepath{stroke}%
\end{pgfscope}%
\begin{pgfscope}%
\pgfsetrectcap%
\pgfsetmiterjoin%
\pgfsetlinewidth{0.803000pt}%
\definecolor{currentstroke}{rgb}{0.000000,0.000000,0.000000}%
\pgfsetstrokecolor{currentstroke}%
\pgfsetdash{}{0pt}%
\pgfpathmoveto{\pgfqpoint{0.553581in}{1.654444in}}%
\pgfpathlineto{\pgfqpoint{2.103581in}{1.654444in}}%
\pgfusepath{stroke}%
\end{pgfscope}%
\begin{pgfscope}%
\pgfsetbuttcap%
\pgfsetmiterjoin%
\definecolor{currentfill}{rgb}{1.000000,1.000000,1.000000}%
\pgfsetfillcolor{currentfill}%
\pgfsetfillopacity{0.800000}%
\pgfsetlinewidth{1.003750pt}%
\definecolor{currentstroke}{rgb}{0.800000,0.800000,0.800000}%
\pgfsetstrokecolor{currentstroke}%
\pgfsetstrokeopacity{0.800000}%
\pgfsetdash{}{0pt}%
\pgfpathmoveto{\pgfqpoint{0.832747in}{0.568889in}}%
\pgfpathlineto{\pgfqpoint{2.006358in}{0.568889in}}%
\pgfpathquadraticcurveto{\pgfqpoint{2.034136in}{0.568889in}}{\pgfqpoint{2.034136in}{0.596666in}}%
\pgfpathlineto{\pgfqpoint{2.034136in}{0.776388in}}%
\pgfpathquadraticcurveto{\pgfqpoint{2.034136in}{0.804166in}}{\pgfqpoint{2.006358in}{0.804166in}}%
\pgfpathlineto{\pgfqpoint{0.832747in}{0.804166in}}%
\pgfpathquadraticcurveto{\pgfqpoint{0.804970in}{0.804166in}}{\pgfqpoint{0.804970in}{0.776388in}}%
\pgfpathlineto{\pgfqpoint{0.804970in}{0.596666in}}%
\pgfpathquadraticcurveto{\pgfqpoint{0.804970in}{0.568889in}}{\pgfqpoint{0.832747in}{0.568889in}}%
\pgfpathlineto{\pgfqpoint{0.832747in}{0.568889in}}%
\pgfpathclose%
\pgfusepath{stroke,fill}%
\end{pgfscope}%
\begin{pgfscope}%
\pgfsetrectcap%
\pgfsetroundjoin%
\pgfsetlinewidth{1.505625pt}%
\definecolor{currentstroke}{rgb}{0.000000,0.000000,0.000000}%
\pgfsetstrokecolor{currentstroke}%
\pgfsetdash{}{0pt}%
\pgfpathmoveto{\pgfqpoint{0.860525in}{0.700000in}}%
\pgfpathlineto{\pgfqpoint{0.999414in}{0.700000in}}%
\pgfpathlineto{\pgfqpoint{1.138303in}{0.700000in}}%
\pgfusepath{stroke}%
\end{pgfscope}%
\begin{pgfscope}%
\definecolor{textcolor}{rgb}{0.000000,0.000000,0.000000}%
\pgfsetstrokecolor{textcolor}%
\pgfsetfillcolor{textcolor}%
\pgftext[x=1.249414in,y=0.651388in,left,base]{\color{textcolor}\rmfamily\fontsize{10.000000}{12.000000}\selectfont AUC=0.840}%
\end{pgfscope}%
\end{pgfpicture}%
\makeatother%
\endgroup%

\end{tabular}


The {\it confusion matrix} for this ideal data set, here given as percentages of the entire dataset, shows few false positives and false negatives.  The metrics below are the ones we will watch when evaluating models.  Each of them tells a different story about what the model does well.


\begin{center}
\begin{tabular}{cc}
\begin{tabular}{cc|c|c|}
	&\multicolumn{1}{c}{}& \multicolumn{2}{c}{Prediction} \cr
	&\multicolumn{1}{c}{} & \multicolumn{1}{c}{N} & \multicolumn{1}{c}{P} \cr\cline{3-4}
	\multirow{2}{*}{Actual}&N & 67.0\% & 18.7\% \vrule width 0pt height 10pt depth 2pt \cr\cline{3-4}
	&P & 3.03\% & 11.3\% \vrule width 0pt height 10pt depth 2pt \cr\cline{3-4}
\end{tabular}
&
\begin{tabular}{ll}
0.783 & Accuracy \cr 
0.785 & Balanced Accuracy \cr 
0.376 & Precision \cr 
0.783 & Balanced Precision \cr 
0.788 & Recall \cr 
0.509 & F1 \cr 
0.785 & Balanced F1 \cr 
0.542 & Gmean \cr 	\end{tabular}
\end{tabular}
\end{center}

%%%%%
If we do not address the data imbalance, the model building algorithm will maximize accuracy by classifying most (or all) of the samples as ``No Ambulance'' with $p < 0.5$  We built the artificial results below by multiplying the probabilities in the above results by $0.5$.  Note that the Area Under the Curve (AUC) did not change.  

\begin{center}
\begin{tabular}{p{0.5\textwidth} p{0.5\textwidth}}
  \vspace{0pt} %% Creator: Matplotlib, PGF backend
%%
%% To include the figure in your LaTeX document, write
%%   \input{<filename>.pgf}
%%
%% Make sure the required packages are loaded in your preamble
%%   \usepackage{pgf}
%%
%% Also ensure that all the required font packages are loaded; for instance,
%% the lmodern package is sometimes necessary when using math font.
%%   \usepackage{lmodern}
%%
%% Figures using additional raster images can only be included by \input if
%% they are in the same directory as the main LaTeX file. For loading figures
%% from other directories you can use the `import` package
%%   \usepackage{import}
%%
%% and then include the figures with
%%   \import{<path to file>}{<filename>.pgf}
%%
%% Matplotlib used the following preamble
%%   
%%   \usepackage{fontspec}
%%   \makeatletter\@ifpackageloaded{underscore}{}{\usepackage[strings]{underscore}}\makeatother
%%
\begingroup%
\makeatletter%
\begin{pgfpicture}%
\pgfpathrectangle{\pgfpointorigin}{\pgfqpoint{3.095000in}{3.243944in}}%
\pgfusepath{use as bounding box, clip}%
\begin{pgfscope}%
\pgfsetbuttcap%
\pgfsetmiterjoin%
\definecolor{currentfill}{rgb}{1.000000,1.000000,1.000000}%
\pgfsetfillcolor{currentfill}%
\pgfsetlinewidth{0.000000pt}%
\definecolor{currentstroke}{rgb}{1.000000,1.000000,1.000000}%
\pgfsetstrokecolor{currentstroke}%
\pgfsetdash{}{0pt}%
\pgfpathmoveto{\pgfqpoint{0.000000in}{0.000000in}}%
\pgfpathlineto{\pgfqpoint{3.095000in}{0.000000in}}%
\pgfpathlineto{\pgfqpoint{3.095000in}{3.243944in}}%
\pgfpathlineto{\pgfqpoint{0.000000in}{3.243944in}}%
\pgfpathlineto{\pgfqpoint{0.000000in}{0.000000in}}%
\pgfpathclose%
\pgfusepath{fill}%
\end{pgfscope}%
\begin{pgfscope}%
\pgfsetbuttcap%
\pgfsetmiterjoin%
\definecolor{currentfill}{rgb}{1.000000,1.000000,1.000000}%
\pgfsetfillcolor{currentfill}%
\pgfsetlinewidth{0.000000pt}%
\definecolor{currentstroke}{rgb}{0.000000,0.000000,0.000000}%
\pgfsetstrokecolor{currentstroke}%
\pgfsetstrokeopacity{0.000000}%
\pgfsetdash{}{0pt}%
\pgfpathmoveto{\pgfqpoint{0.515000in}{1.096944in}}%
\pgfpathlineto{\pgfqpoint{2.995000in}{1.096944in}}%
\pgfpathlineto{\pgfqpoint{2.995000in}{2.944944in}}%
\pgfpathlineto{\pgfqpoint{0.515000in}{2.944944in}}%
\pgfpathlineto{\pgfqpoint{0.515000in}{1.096944in}}%
\pgfpathclose%
\pgfusepath{fill}%
\end{pgfscope}%
\begin{pgfscope}%
\pgfpathrectangle{\pgfqpoint{0.515000in}{1.096944in}}{\pgfqpoint{2.480000in}{1.848000in}}%
\pgfusepath{clip}%
\pgfsetbuttcap%
\pgfsetmiterjoin%
\pgfsetlinewidth{1.003750pt}%
\definecolor{currentstroke}{rgb}{0.000000,0.000000,0.000000}%
\pgfsetstrokecolor{currentstroke}%
\pgfsetdash{}{0pt}%
\pgfpathmoveto{\pgfqpoint{0.505000in}{1.096944in}}%
\pgfpathlineto{\pgfqpoint{0.577627in}{1.096944in}}%
\pgfpathlineto{\pgfqpoint{0.577627in}{1.838062in}}%
\pgfpathlineto{\pgfqpoint{0.505000in}{1.838062in}}%
\pgfusepath{stroke}%
\end{pgfscope}%
\begin{pgfscope}%
\pgfpathrectangle{\pgfqpoint{0.515000in}{1.096944in}}{\pgfqpoint{2.480000in}{1.848000in}}%
\pgfusepath{clip}%
\pgfsetbuttcap%
\pgfsetmiterjoin%
\pgfsetlinewidth{1.003750pt}%
\definecolor{currentstroke}{rgb}{0.000000,0.000000,0.000000}%
\pgfsetstrokecolor{currentstroke}%
\pgfsetdash{}{0pt}%
\pgfpathmoveto{\pgfqpoint{0.727930in}{1.096944in}}%
\pgfpathlineto{\pgfqpoint{0.828132in}{1.096944in}}%
\pgfpathlineto{\pgfqpoint{0.828132in}{2.856944in}}%
\pgfpathlineto{\pgfqpoint{0.727930in}{2.856944in}}%
\pgfpathlineto{\pgfqpoint{0.727930in}{1.096944in}}%
\pgfpathclose%
\pgfusepath{stroke}%
\end{pgfscope}%
\begin{pgfscope}%
\pgfpathrectangle{\pgfqpoint{0.515000in}{1.096944in}}{\pgfqpoint{2.480000in}{1.848000in}}%
\pgfusepath{clip}%
\pgfsetbuttcap%
\pgfsetmiterjoin%
\pgfsetlinewidth{1.003750pt}%
\definecolor{currentstroke}{rgb}{0.000000,0.000000,0.000000}%
\pgfsetstrokecolor{currentstroke}%
\pgfsetdash{}{0pt}%
\pgfpathmoveto{\pgfqpoint{0.978435in}{1.096944in}}%
\pgfpathlineto{\pgfqpoint{1.078637in}{1.096944in}}%
\pgfpathlineto{\pgfqpoint{1.078637in}{2.106449in}}%
\pgfpathlineto{\pgfqpoint{0.978435in}{2.106449in}}%
\pgfpathlineto{\pgfqpoint{0.978435in}{1.096944in}}%
\pgfpathclose%
\pgfusepath{stroke}%
\end{pgfscope}%
\begin{pgfscope}%
\pgfpathrectangle{\pgfqpoint{0.515000in}{1.096944in}}{\pgfqpoint{2.480000in}{1.848000in}}%
\pgfusepath{clip}%
\pgfsetbuttcap%
\pgfsetmiterjoin%
\pgfsetlinewidth{1.003750pt}%
\definecolor{currentstroke}{rgb}{0.000000,0.000000,0.000000}%
\pgfsetstrokecolor{currentstroke}%
\pgfsetdash{}{0pt}%
\pgfpathmoveto{\pgfqpoint{1.228940in}{1.096944in}}%
\pgfpathlineto{\pgfqpoint{1.329142in}{1.096944in}}%
\pgfpathlineto{\pgfqpoint{1.329142in}{1.480023in}}%
\pgfpathlineto{\pgfqpoint{1.228940in}{1.480023in}}%
\pgfpathlineto{\pgfqpoint{1.228940in}{1.096944in}}%
\pgfpathclose%
\pgfusepath{stroke}%
\end{pgfscope}%
\begin{pgfscope}%
\pgfpathrectangle{\pgfqpoint{0.515000in}{1.096944in}}{\pgfqpoint{2.480000in}{1.848000in}}%
\pgfusepath{clip}%
\pgfsetbuttcap%
\pgfsetmiterjoin%
\pgfsetlinewidth{1.003750pt}%
\definecolor{currentstroke}{rgb}{0.000000,0.000000,0.000000}%
\pgfsetstrokecolor{currentstroke}%
\pgfsetdash{}{0pt}%
\pgfpathmoveto{\pgfqpoint{1.479445in}{1.096944in}}%
\pgfpathlineto{\pgfqpoint{1.579647in}{1.096944in}}%
\pgfpathlineto{\pgfqpoint{1.579647in}{1.193645in}}%
\pgfpathlineto{\pgfqpoint{1.479445in}{1.193645in}}%
\pgfpathlineto{\pgfqpoint{1.479445in}{1.096944in}}%
\pgfpathclose%
\pgfusepath{stroke}%
\end{pgfscope}%
\begin{pgfscope}%
\pgfpathrectangle{\pgfqpoint{0.515000in}{1.096944in}}{\pgfqpoint{2.480000in}{1.848000in}}%
\pgfusepath{clip}%
\pgfsetbuttcap%
\pgfsetmiterjoin%
\pgfsetlinewidth{1.003750pt}%
\definecolor{currentstroke}{rgb}{0.000000,0.000000,0.000000}%
\pgfsetstrokecolor{currentstroke}%
\pgfsetdash{}{0pt}%
\pgfpathmoveto{\pgfqpoint{1.729950in}{1.096944in}}%
\pgfpathlineto{\pgfqpoint{1.830152in}{1.096944in}}%
\pgfpathlineto{\pgfqpoint{1.830152in}{1.096944in}}%
\pgfpathlineto{\pgfqpoint{1.729950in}{1.096944in}}%
\pgfpathlineto{\pgfqpoint{1.729950in}{1.096944in}}%
\pgfpathclose%
\pgfusepath{stroke}%
\end{pgfscope}%
\begin{pgfscope}%
\pgfpathrectangle{\pgfqpoint{0.515000in}{1.096944in}}{\pgfqpoint{2.480000in}{1.848000in}}%
\pgfusepath{clip}%
\pgfsetbuttcap%
\pgfsetmiterjoin%
\pgfsetlinewidth{1.003750pt}%
\definecolor{currentstroke}{rgb}{0.000000,0.000000,0.000000}%
\pgfsetstrokecolor{currentstroke}%
\pgfsetdash{}{0pt}%
\pgfpathmoveto{\pgfqpoint{1.980455in}{1.096944in}}%
\pgfpathlineto{\pgfqpoint{2.080657in}{1.096944in}}%
\pgfpathlineto{\pgfqpoint{2.080657in}{1.096944in}}%
\pgfpathlineto{\pgfqpoint{1.980455in}{1.096944in}}%
\pgfpathlineto{\pgfqpoint{1.980455in}{1.096944in}}%
\pgfpathclose%
\pgfusepath{stroke}%
\end{pgfscope}%
\begin{pgfscope}%
\pgfpathrectangle{\pgfqpoint{0.515000in}{1.096944in}}{\pgfqpoint{2.480000in}{1.848000in}}%
\pgfusepath{clip}%
\pgfsetbuttcap%
\pgfsetmiterjoin%
\pgfsetlinewidth{1.003750pt}%
\definecolor{currentstroke}{rgb}{0.000000,0.000000,0.000000}%
\pgfsetstrokecolor{currentstroke}%
\pgfsetdash{}{0pt}%
\pgfpathmoveto{\pgfqpoint{2.230960in}{1.096944in}}%
\pgfpathlineto{\pgfqpoint{2.331162in}{1.096944in}}%
\pgfpathlineto{\pgfqpoint{2.331162in}{1.096944in}}%
\pgfpathlineto{\pgfqpoint{2.230960in}{1.096944in}}%
\pgfpathlineto{\pgfqpoint{2.230960in}{1.096944in}}%
\pgfpathclose%
\pgfusepath{stroke}%
\end{pgfscope}%
\begin{pgfscope}%
\pgfpathrectangle{\pgfqpoint{0.515000in}{1.096944in}}{\pgfqpoint{2.480000in}{1.848000in}}%
\pgfusepath{clip}%
\pgfsetbuttcap%
\pgfsetmiterjoin%
\pgfsetlinewidth{1.003750pt}%
\definecolor{currentstroke}{rgb}{0.000000,0.000000,0.000000}%
\pgfsetstrokecolor{currentstroke}%
\pgfsetdash{}{0pt}%
\pgfpathmoveto{\pgfqpoint{2.481465in}{1.096944in}}%
\pgfpathlineto{\pgfqpoint{2.581667in}{1.096944in}}%
\pgfpathlineto{\pgfqpoint{2.581667in}{1.096944in}}%
\pgfpathlineto{\pgfqpoint{2.481465in}{1.096944in}}%
\pgfpathlineto{\pgfqpoint{2.481465in}{1.096944in}}%
\pgfpathclose%
\pgfusepath{stroke}%
\end{pgfscope}%
\begin{pgfscope}%
\pgfpathrectangle{\pgfqpoint{0.515000in}{1.096944in}}{\pgfqpoint{2.480000in}{1.848000in}}%
\pgfusepath{clip}%
\pgfsetbuttcap%
\pgfsetmiterjoin%
\pgfsetlinewidth{1.003750pt}%
\definecolor{currentstroke}{rgb}{0.000000,0.000000,0.000000}%
\pgfsetstrokecolor{currentstroke}%
\pgfsetdash{}{0pt}%
\pgfpathmoveto{\pgfqpoint{2.731970in}{1.096944in}}%
\pgfpathlineto{\pgfqpoint{2.832172in}{1.096944in}}%
\pgfpathlineto{\pgfqpoint{2.832172in}{1.096944in}}%
\pgfpathlineto{\pgfqpoint{2.731970in}{1.096944in}}%
\pgfpathlineto{\pgfqpoint{2.731970in}{1.096944in}}%
\pgfpathclose%
\pgfusepath{stroke}%
\end{pgfscope}%
\begin{pgfscope}%
\pgfpathrectangle{\pgfqpoint{0.515000in}{1.096944in}}{\pgfqpoint{2.480000in}{1.848000in}}%
\pgfusepath{clip}%
\pgfsetbuttcap%
\pgfsetmiterjoin%
\definecolor{currentfill}{rgb}{0.000000,0.000000,0.000000}%
\pgfsetfillcolor{currentfill}%
\pgfsetlinewidth{0.000000pt}%
\definecolor{currentstroke}{rgb}{0.000000,0.000000,0.000000}%
\pgfsetstrokecolor{currentstroke}%
\pgfsetstrokeopacity{0.000000}%
\pgfsetdash{}{0pt}%
\pgfpathmoveto{\pgfqpoint{0.577627in}{1.096944in}}%
\pgfpathlineto{\pgfqpoint{0.677829in}{1.096944in}}%
\pgfpathlineto{\pgfqpoint{0.677829in}{1.113937in}}%
\pgfpathlineto{\pgfqpoint{0.577627in}{1.113937in}}%
\pgfpathlineto{\pgfqpoint{0.577627in}{1.096944in}}%
\pgfpathclose%
\pgfusepath{fill}%
\end{pgfscope}%
\begin{pgfscope}%
\pgfpathrectangle{\pgfqpoint{0.515000in}{1.096944in}}{\pgfqpoint{2.480000in}{1.848000in}}%
\pgfusepath{clip}%
\pgfsetbuttcap%
\pgfsetmiterjoin%
\definecolor{currentfill}{rgb}{0.000000,0.000000,0.000000}%
\pgfsetfillcolor{currentfill}%
\pgfsetlinewidth{0.000000pt}%
\definecolor{currentstroke}{rgb}{0.000000,0.000000,0.000000}%
\pgfsetstrokecolor{currentstroke}%
\pgfsetstrokeopacity{0.000000}%
\pgfsetdash{}{0pt}%
\pgfpathmoveto{\pgfqpoint{0.828132in}{1.096944in}}%
\pgfpathlineto{\pgfqpoint{0.928334in}{1.096944in}}%
\pgfpathlineto{\pgfqpoint{0.928334in}{1.154439in}}%
\pgfpathlineto{\pgfqpoint{0.828132in}{1.154439in}}%
\pgfpathlineto{\pgfqpoint{0.828132in}{1.096944in}}%
\pgfpathclose%
\pgfusepath{fill}%
\end{pgfscope}%
\begin{pgfscope}%
\pgfpathrectangle{\pgfqpoint{0.515000in}{1.096944in}}{\pgfqpoint{2.480000in}{1.848000in}}%
\pgfusepath{clip}%
\pgfsetbuttcap%
\pgfsetmiterjoin%
\definecolor{currentfill}{rgb}{0.000000,0.000000,0.000000}%
\pgfsetfillcolor{currentfill}%
\pgfsetlinewidth{0.000000pt}%
\definecolor{currentstroke}{rgb}{0.000000,0.000000,0.000000}%
\pgfsetstrokecolor{currentstroke}%
\pgfsetstrokeopacity{0.000000}%
\pgfsetdash{}{0pt}%
\pgfpathmoveto{\pgfqpoint{1.078637in}{1.096944in}}%
\pgfpathlineto{\pgfqpoint{1.178839in}{1.096944in}}%
\pgfpathlineto{\pgfqpoint{1.178839in}{1.267833in}}%
\pgfpathlineto{\pgfqpoint{1.078637in}{1.267833in}}%
\pgfpathlineto{\pgfqpoint{1.078637in}{1.096944in}}%
\pgfpathclose%
\pgfusepath{fill}%
\end{pgfscope}%
\begin{pgfscope}%
\pgfpathrectangle{\pgfqpoint{0.515000in}{1.096944in}}{\pgfqpoint{2.480000in}{1.848000in}}%
\pgfusepath{clip}%
\pgfsetbuttcap%
\pgfsetmiterjoin%
\definecolor{currentfill}{rgb}{0.000000,0.000000,0.000000}%
\pgfsetfillcolor{currentfill}%
\pgfsetlinewidth{0.000000pt}%
\definecolor{currentstroke}{rgb}{0.000000,0.000000,0.000000}%
\pgfsetstrokecolor{currentstroke}%
\pgfsetstrokeopacity{0.000000}%
\pgfsetdash{}{0pt}%
\pgfpathmoveto{\pgfqpoint{1.329142in}{1.096944in}}%
\pgfpathlineto{\pgfqpoint{1.429344in}{1.096944in}}%
\pgfpathlineto{\pgfqpoint{1.429344in}{1.386581in}}%
\pgfpathlineto{\pgfqpoint{1.329142in}{1.386581in}}%
\pgfpathlineto{\pgfqpoint{1.329142in}{1.096944in}}%
\pgfpathclose%
\pgfusepath{fill}%
\end{pgfscope}%
\begin{pgfscope}%
\pgfpathrectangle{\pgfqpoint{0.515000in}{1.096944in}}{\pgfqpoint{2.480000in}{1.848000in}}%
\pgfusepath{clip}%
\pgfsetbuttcap%
\pgfsetmiterjoin%
\definecolor{currentfill}{rgb}{0.000000,0.000000,0.000000}%
\pgfsetfillcolor{currentfill}%
\pgfsetlinewidth{0.000000pt}%
\definecolor{currentstroke}{rgb}{0.000000,0.000000,0.000000}%
\pgfsetstrokecolor{currentstroke}%
\pgfsetstrokeopacity{0.000000}%
\pgfsetdash{}{0pt}%
\pgfpathmoveto{\pgfqpoint{1.579647in}{1.096944in}}%
\pgfpathlineto{\pgfqpoint{1.679849in}{1.096944in}}%
\pgfpathlineto{\pgfqpoint{1.679849in}{1.226998in}}%
\pgfpathlineto{\pgfqpoint{1.579647in}{1.226998in}}%
\pgfpathlineto{\pgfqpoint{1.579647in}{1.096944in}}%
\pgfpathclose%
\pgfusepath{fill}%
\end{pgfscope}%
\begin{pgfscope}%
\pgfpathrectangle{\pgfqpoint{0.515000in}{1.096944in}}{\pgfqpoint{2.480000in}{1.848000in}}%
\pgfusepath{clip}%
\pgfsetbuttcap%
\pgfsetmiterjoin%
\definecolor{currentfill}{rgb}{0.000000,0.000000,0.000000}%
\pgfsetfillcolor{currentfill}%
\pgfsetlinewidth{0.000000pt}%
\definecolor{currentstroke}{rgb}{0.000000,0.000000,0.000000}%
\pgfsetstrokecolor{currentstroke}%
\pgfsetstrokeopacity{0.000000}%
\pgfsetdash{}{0pt}%
\pgfpathmoveto{\pgfqpoint{1.830152in}{1.096944in}}%
\pgfpathlineto{\pgfqpoint{1.930354in}{1.096944in}}%
\pgfpathlineto{\pgfqpoint{1.930354in}{1.096944in}}%
\pgfpathlineto{\pgfqpoint{1.830152in}{1.096944in}}%
\pgfpathlineto{\pgfqpoint{1.830152in}{1.096944in}}%
\pgfpathclose%
\pgfusepath{fill}%
\end{pgfscope}%
\begin{pgfscope}%
\pgfpathrectangle{\pgfqpoint{0.515000in}{1.096944in}}{\pgfqpoint{2.480000in}{1.848000in}}%
\pgfusepath{clip}%
\pgfsetbuttcap%
\pgfsetmiterjoin%
\definecolor{currentfill}{rgb}{0.000000,0.000000,0.000000}%
\pgfsetfillcolor{currentfill}%
\pgfsetlinewidth{0.000000pt}%
\definecolor{currentstroke}{rgb}{0.000000,0.000000,0.000000}%
\pgfsetstrokecolor{currentstroke}%
\pgfsetstrokeopacity{0.000000}%
\pgfsetdash{}{0pt}%
\pgfpathmoveto{\pgfqpoint{2.080657in}{1.096944in}}%
\pgfpathlineto{\pgfqpoint{2.180859in}{1.096944in}}%
\pgfpathlineto{\pgfqpoint{2.180859in}{1.096944in}}%
\pgfpathlineto{\pgfqpoint{2.080657in}{1.096944in}}%
\pgfpathlineto{\pgfqpoint{2.080657in}{1.096944in}}%
\pgfpathclose%
\pgfusepath{fill}%
\end{pgfscope}%
\begin{pgfscope}%
\pgfpathrectangle{\pgfqpoint{0.515000in}{1.096944in}}{\pgfqpoint{2.480000in}{1.848000in}}%
\pgfusepath{clip}%
\pgfsetbuttcap%
\pgfsetmiterjoin%
\definecolor{currentfill}{rgb}{0.000000,0.000000,0.000000}%
\pgfsetfillcolor{currentfill}%
\pgfsetlinewidth{0.000000pt}%
\definecolor{currentstroke}{rgb}{0.000000,0.000000,0.000000}%
\pgfsetstrokecolor{currentstroke}%
\pgfsetstrokeopacity{0.000000}%
\pgfsetdash{}{0pt}%
\pgfpathmoveto{\pgfqpoint{2.331162in}{1.096944in}}%
\pgfpathlineto{\pgfqpoint{2.431364in}{1.096944in}}%
\pgfpathlineto{\pgfqpoint{2.431364in}{1.096944in}}%
\pgfpathlineto{\pgfqpoint{2.331162in}{1.096944in}}%
\pgfpathlineto{\pgfqpoint{2.331162in}{1.096944in}}%
\pgfpathclose%
\pgfusepath{fill}%
\end{pgfscope}%
\begin{pgfscope}%
\pgfpathrectangle{\pgfqpoint{0.515000in}{1.096944in}}{\pgfqpoint{2.480000in}{1.848000in}}%
\pgfusepath{clip}%
\pgfsetbuttcap%
\pgfsetmiterjoin%
\definecolor{currentfill}{rgb}{0.000000,0.000000,0.000000}%
\pgfsetfillcolor{currentfill}%
\pgfsetlinewidth{0.000000pt}%
\definecolor{currentstroke}{rgb}{0.000000,0.000000,0.000000}%
\pgfsetstrokecolor{currentstroke}%
\pgfsetstrokeopacity{0.000000}%
\pgfsetdash{}{0pt}%
\pgfpathmoveto{\pgfqpoint{2.581667in}{1.096944in}}%
\pgfpathlineto{\pgfqpoint{2.681869in}{1.096944in}}%
\pgfpathlineto{\pgfqpoint{2.681869in}{1.096944in}}%
\pgfpathlineto{\pgfqpoint{2.581667in}{1.096944in}}%
\pgfpathlineto{\pgfqpoint{2.581667in}{1.096944in}}%
\pgfpathclose%
\pgfusepath{fill}%
\end{pgfscope}%
\begin{pgfscope}%
\pgfpathrectangle{\pgfqpoint{0.515000in}{1.096944in}}{\pgfqpoint{2.480000in}{1.848000in}}%
\pgfusepath{clip}%
\pgfsetbuttcap%
\pgfsetmiterjoin%
\definecolor{currentfill}{rgb}{0.000000,0.000000,0.000000}%
\pgfsetfillcolor{currentfill}%
\pgfsetlinewidth{0.000000pt}%
\definecolor{currentstroke}{rgb}{0.000000,0.000000,0.000000}%
\pgfsetstrokecolor{currentstroke}%
\pgfsetstrokeopacity{0.000000}%
\pgfsetdash{}{0pt}%
\pgfpathmoveto{\pgfqpoint{2.832172in}{1.096944in}}%
\pgfpathlineto{\pgfqpoint{2.932374in}{1.096944in}}%
\pgfpathlineto{\pgfqpoint{2.932374in}{1.096944in}}%
\pgfpathlineto{\pgfqpoint{2.832172in}{1.096944in}}%
\pgfpathlineto{\pgfqpoint{2.832172in}{1.096944in}}%
\pgfpathclose%
\pgfusepath{fill}%
\end{pgfscope}%
\begin{pgfscope}%
\pgfsetbuttcap%
\pgfsetroundjoin%
\definecolor{currentfill}{rgb}{0.000000,0.000000,0.000000}%
\pgfsetfillcolor{currentfill}%
\pgfsetlinewidth{0.803000pt}%
\definecolor{currentstroke}{rgb}{0.000000,0.000000,0.000000}%
\pgfsetstrokecolor{currentstroke}%
\pgfsetdash{}{0pt}%
\pgfsys@defobject{currentmarker}{\pgfqpoint{0.000000in}{-0.048611in}}{\pgfqpoint{0.000000in}{0.000000in}}{%
\pgfpathmoveto{\pgfqpoint{0.000000in}{0.000000in}}%
\pgfpathlineto{\pgfqpoint{0.000000in}{-0.048611in}}%
\pgfusepath{stroke,fill}%
}%
\begin{pgfscope}%
\pgfsys@transformshift{0.577627in}{1.096944in}%
\pgfsys@useobject{currentmarker}{}%
\end{pgfscope}%
\end{pgfscope}%
\begin{pgfscope}%
\definecolor{textcolor}{rgb}{0.000000,0.000000,0.000000}%
\pgfsetstrokecolor{textcolor}%
\pgfsetfillcolor{textcolor}%
\pgftext[x=0.612349in, y=0.282083in, left, base,rotate=90.000000]{\color{textcolor}\rmfamily\fontsize{10.000000}{12.000000}\selectfont (-0.001, 0.1]}%
\end{pgfscope}%
\begin{pgfscope}%
\pgfsetbuttcap%
\pgfsetroundjoin%
\definecolor{currentfill}{rgb}{0.000000,0.000000,0.000000}%
\pgfsetfillcolor{currentfill}%
\pgfsetlinewidth{0.803000pt}%
\definecolor{currentstroke}{rgb}{0.000000,0.000000,0.000000}%
\pgfsetstrokecolor{currentstroke}%
\pgfsetdash{}{0pt}%
\pgfsys@defobject{currentmarker}{\pgfqpoint{0.000000in}{-0.048611in}}{\pgfqpoint{0.000000in}{0.000000in}}{%
\pgfpathmoveto{\pgfqpoint{0.000000in}{0.000000in}}%
\pgfpathlineto{\pgfqpoint{0.000000in}{-0.048611in}}%
\pgfusepath{stroke,fill}%
}%
\begin{pgfscope}%
\pgfsys@transformshift{0.828132in}{1.096944in}%
\pgfsys@useobject{currentmarker}{}%
\end{pgfscope}%
\end{pgfscope}%
\begin{pgfscope}%
\definecolor{textcolor}{rgb}{0.000000,0.000000,0.000000}%
\pgfsetstrokecolor{textcolor}%
\pgfsetfillcolor{textcolor}%
\pgftext[x=0.862854in, y=0.467222in, left, base,rotate=90.000000]{\color{textcolor}\rmfamily\fontsize{10.000000}{12.000000}\selectfont (0.1, 0.2]}%
\end{pgfscope}%
\begin{pgfscope}%
\pgfsetbuttcap%
\pgfsetroundjoin%
\definecolor{currentfill}{rgb}{0.000000,0.000000,0.000000}%
\pgfsetfillcolor{currentfill}%
\pgfsetlinewidth{0.803000pt}%
\definecolor{currentstroke}{rgb}{0.000000,0.000000,0.000000}%
\pgfsetstrokecolor{currentstroke}%
\pgfsetdash{}{0pt}%
\pgfsys@defobject{currentmarker}{\pgfqpoint{0.000000in}{-0.048611in}}{\pgfqpoint{0.000000in}{0.000000in}}{%
\pgfpathmoveto{\pgfqpoint{0.000000in}{0.000000in}}%
\pgfpathlineto{\pgfqpoint{0.000000in}{-0.048611in}}%
\pgfusepath{stroke,fill}%
}%
\begin{pgfscope}%
\pgfsys@transformshift{1.078637in}{1.096944in}%
\pgfsys@useobject{currentmarker}{}%
\end{pgfscope}%
\end{pgfscope}%
\begin{pgfscope}%
\definecolor{textcolor}{rgb}{0.000000,0.000000,0.000000}%
\pgfsetstrokecolor{textcolor}%
\pgfsetfillcolor{textcolor}%
\pgftext[x=1.113359in, y=0.467222in, left, base,rotate=90.000000]{\color{textcolor}\rmfamily\fontsize{10.000000}{12.000000}\selectfont (0.2, 0.3]}%
\end{pgfscope}%
\begin{pgfscope}%
\pgfsetbuttcap%
\pgfsetroundjoin%
\definecolor{currentfill}{rgb}{0.000000,0.000000,0.000000}%
\pgfsetfillcolor{currentfill}%
\pgfsetlinewidth{0.803000pt}%
\definecolor{currentstroke}{rgb}{0.000000,0.000000,0.000000}%
\pgfsetstrokecolor{currentstroke}%
\pgfsetdash{}{0pt}%
\pgfsys@defobject{currentmarker}{\pgfqpoint{0.000000in}{-0.048611in}}{\pgfqpoint{0.000000in}{0.000000in}}{%
\pgfpathmoveto{\pgfqpoint{0.000000in}{0.000000in}}%
\pgfpathlineto{\pgfqpoint{0.000000in}{-0.048611in}}%
\pgfusepath{stroke,fill}%
}%
\begin{pgfscope}%
\pgfsys@transformshift{1.329142in}{1.096944in}%
\pgfsys@useobject{currentmarker}{}%
\end{pgfscope}%
\end{pgfscope}%
\begin{pgfscope}%
\definecolor{textcolor}{rgb}{0.000000,0.000000,0.000000}%
\pgfsetstrokecolor{textcolor}%
\pgfsetfillcolor{textcolor}%
\pgftext[x=1.363864in, y=0.467222in, left, base,rotate=90.000000]{\color{textcolor}\rmfamily\fontsize{10.000000}{12.000000}\selectfont (0.3, 0.4]}%
\end{pgfscope}%
\begin{pgfscope}%
\pgfsetbuttcap%
\pgfsetroundjoin%
\definecolor{currentfill}{rgb}{0.000000,0.000000,0.000000}%
\pgfsetfillcolor{currentfill}%
\pgfsetlinewidth{0.803000pt}%
\definecolor{currentstroke}{rgb}{0.000000,0.000000,0.000000}%
\pgfsetstrokecolor{currentstroke}%
\pgfsetdash{}{0pt}%
\pgfsys@defobject{currentmarker}{\pgfqpoint{0.000000in}{-0.048611in}}{\pgfqpoint{0.000000in}{0.000000in}}{%
\pgfpathmoveto{\pgfqpoint{0.000000in}{0.000000in}}%
\pgfpathlineto{\pgfqpoint{0.000000in}{-0.048611in}}%
\pgfusepath{stroke,fill}%
}%
\begin{pgfscope}%
\pgfsys@transformshift{1.579647in}{1.096944in}%
\pgfsys@useobject{currentmarker}{}%
\end{pgfscope}%
\end{pgfscope}%
\begin{pgfscope}%
\definecolor{textcolor}{rgb}{0.000000,0.000000,0.000000}%
\pgfsetstrokecolor{textcolor}%
\pgfsetfillcolor{textcolor}%
\pgftext[x=1.614369in, y=0.467222in, left, base,rotate=90.000000]{\color{textcolor}\rmfamily\fontsize{10.000000}{12.000000}\selectfont (0.4, 0.5]}%
\end{pgfscope}%
\begin{pgfscope}%
\pgfsetbuttcap%
\pgfsetroundjoin%
\definecolor{currentfill}{rgb}{0.000000,0.000000,0.000000}%
\pgfsetfillcolor{currentfill}%
\pgfsetlinewidth{0.803000pt}%
\definecolor{currentstroke}{rgb}{0.000000,0.000000,0.000000}%
\pgfsetstrokecolor{currentstroke}%
\pgfsetdash{}{0pt}%
\pgfsys@defobject{currentmarker}{\pgfqpoint{0.000000in}{-0.048611in}}{\pgfqpoint{0.000000in}{0.000000in}}{%
\pgfpathmoveto{\pgfqpoint{0.000000in}{0.000000in}}%
\pgfpathlineto{\pgfqpoint{0.000000in}{-0.048611in}}%
\pgfusepath{stroke,fill}%
}%
\begin{pgfscope}%
\pgfsys@transformshift{1.830152in}{1.096944in}%
\pgfsys@useobject{currentmarker}{}%
\end{pgfscope}%
\end{pgfscope}%
\begin{pgfscope}%
\definecolor{textcolor}{rgb}{0.000000,0.000000,0.000000}%
\pgfsetstrokecolor{textcolor}%
\pgfsetfillcolor{textcolor}%
\pgftext[x=1.864874in, y=0.467222in, left, base,rotate=90.000000]{\color{textcolor}\rmfamily\fontsize{10.000000}{12.000000}\selectfont (0.5, 0.6]}%
\end{pgfscope}%
\begin{pgfscope}%
\pgfsetbuttcap%
\pgfsetroundjoin%
\definecolor{currentfill}{rgb}{0.000000,0.000000,0.000000}%
\pgfsetfillcolor{currentfill}%
\pgfsetlinewidth{0.803000pt}%
\definecolor{currentstroke}{rgb}{0.000000,0.000000,0.000000}%
\pgfsetstrokecolor{currentstroke}%
\pgfsetdash{}{0pt}%
\pgfsys@defobject{currentmarker}{\pgfqpoint{0.000000in}{-0.048611in}}{\pgfqpoint{0.000000in}{0.000000in}}{%
\pgfpathmoveto{\pgfqpoint{0.000000in}{0.000000in}}%
\pgfpathlineto{\pgfqpoint{0.000000in}{-0.048611in}}%
\pgfusepath{stroke,fill}%
}%
\begin{pgfscope}%
\pgfsys@transformshift{2.080657in}{1.096944in}%
\pgfsys@useobject{currentmarker}{}%
\end{pgfscope}%
\end{pgfscope}%
\begin{pgfscope}%
\definecolor{textcolor}{rgb}{0.000000,0.000000,0.000000}%
\pgfsetstrokecolor{textcolor}%
\pgfsetfillcolor{textcolor}%
\pgftext[x=2.115379in, y=0.467222in, left, base,rotate=90.000000]{\color{textcolor}\rmfamily\fontsize{10.000000}{12.000000}\selectfont (0.6, 0.7]}%
\end{pgfscope}%
\begin{pgfscope}%
\pgfsetbuttcap%
\pgfsetroundjoin%
\definecolor{currentfill}{rgb}{0.000000,0.000000,0.000000}%
\pgfsetfillcolor{currentfill}%
\pgfsetlinewidth{0.803000pt}%
\definecolor{currentstroke}{rgb}{0.000000,0.000000,0.000000}%
\pgfsetstrokecolor{currentstroke}%
\pgfsetdash{}{0pt}%
\pgfsys@defobject{currentmarker}{\pgfqpoint{0.000000in}{-0.048611in}}{\pgfqpoint{0.000000in}{0.000000in}}{%
\pgfpathmoveto{\pgfqpoint{0.000000in}{0.000000in}}%
\pgfpathlineto{\pgfqpoint{0.000000in}{-0.048611in}}%
\pgfusepath{stroke,fill}%
}%
\begin{pgfscope}%
\pgfsys@transformshift{2.331162in}{1.096944in}%
\pgfsys@useobject{currentmarker}{}%
\end{pgfscope}%
\end{pgfscope}%
\begin{pgfscope}%
\definecolor{textcolor}{rgb}{0.000000,0.000000,0.000000}%
\pgfsetstrokecolor{textcolor}%
\pgfsetfillcolor{textcolor}%
\pgftext[x=2.365884in, y=0.467222in, left, base,rotate=90.000000]{\color{textcolor}\rmfamily\fontsize{10.000000}{12.000000}\selectfont (0.7, 0.8]}%
\end{pgfscope}%
\begin{pgfscope}%
\pgfsetbuttcap%
\pgfsetroundjoin%
\definecolor{currentfill}{rgb}{0.000000,0.000000,0.000000}%
\pgfsetfillcolor{currentfill}%
\pgfsetlinewidth{0.803000pt}%
\definecolor{currentstroke}{rgb}{0.000000,0.000000,0.000000}%
\pgfsetstrokecolor{currentstroke}%
\pgfsetdash{}{0pt}%
\pgfsys@defobject{currentmarker}{\pgfqpoint{0.000000in}{-0.048611in}}{\pgfqpoint{0.000000in}{0.000000in}}{%
\pgfpathmoveto{\pgfqpoint{0.000000in}{0.000000in}}%
\pgfpathlineto{\pgfqpoint{0.000000in}{-0.048611in}}%
\pgfusepath{stroke,fill}%
}%
\begin{pgfscope}%
\pgfsys@transformshift{2.581667in}{1.096944in}%
\pgfsys@useobject{currentmarker}{}%
\end{pgfscope}%
\end{pgfscope}%
\begin{pgfscope}%
\definecolor{textcolor}{rgb}{0.000000,0.000000,0.000000}%
\pgfsetstrokecolor{textcolor}%
\pgfsetfillcolor{textcolor}%
\pgftext[x=2.616389in, y=0.467222in, left, base,rotate=90.000000]{\color{textcolor}\rmfamily\fontsize{10.000000}{12.000000}\selectfont (0.8, 0.9]}%
\end{pgfscope}%
\begin{pgfscope}%
\pgfsetbuttcap%
\pgfsetroundjoin%
\definecolor{currentfill}{rgb}{0.000000,0.000000,0.000000}%
\pgfsetfillcolor{currentfill}%
\pgfsetlinewidth{0.803000pt}%
\definecolor{currentstroke}{rgb}{0.000000,0.000000,0.000000}%
\pgfsetstrokecolor{currentstroke}%
\pgfsetdash{}{0pt}%
\pgfsys@defobject{currentmarker}{\pgfqpoint{0.000000in}{-0.048611in}}{\pgfqpoint{0.000000in}{0.000000in}}{%
\pgfpathmoveto{\pgfqpoint{0.000000in}{0.000000in}}%
\pgfpathlineto{\pgfqpoint{0.000000in}{-0.048611in}}%
\pgfusepath{stroke,fill}%
}%
\begin{pgfscope}%
\pgfsys@transformshift{2.832172in}{1.096944in}%
\pgfsys@useobject{currentmarker}{}%
\end{pgfscope}%
\end{pgfscope}%
\begin{pgfscope}%
\definecolor{textcolor}{rgb}{0.000000,0.000000,0.000000}%
\pgfsetstrokecolor{textcolor}%
\pgfsetfillcolor{textcolor}%
\pgftext[x=2.866894in, y=0.467222in, left, base,rotate=90.000000]{\color{textcolor}\rmfamily\fontsize{10.000000}{12.000000}\selectfont (0.9, 1.0]}%
\end{pgfscope}%
\begin{pgfscope}%
\definecolor{textcolor}{rgb}{0.000000,0.000000,0.000000}%
\pgfsetstrokecolor{textcolor}%
\pgfsetfillcolor{textcolor}%
\pgftext[x=1.755000in,y=0.226527in,,top]{\color{textcolor}\rmfamily\fontsize{10.000000}{12.000000}\selectfont Range of Prediction}%
\end{pgfscope}%
\begin{pgfscope}%
\pgfsetbuttcap%
\pgfsetroundjoin%
\definecolor{currentfill}{rgb}{0.000000,0.000000,0.000000}%
\pgfsetfillcolor{currentfill}%
\pgfsetlinewidth{0.803000pt}%
\definecolor{currentstroke}{rgb}{0.000000,0.000000,0.000000}%
\pgfsetstrokecolor{currentstroke}%
\pgfsetdash{}{0pt}%
\pgfsys@defobject{currentmarker}{\pgfqpoint{-0.048611in}{0.000000in}}{\pgfqpoint{-0.000000in}{0.000000in}}{%
\pgfpathmoveto{\pgfqpoint{-0.000000in}{0.000000in}}%
\pgfpathlineto{\pgfqpoint{-0.048611in}{0.000000in}}%
\pgfusepath{stroke,fill}%
}%
\begin{pgfscope}%
\pgfsys@transformshift{0.515000in}{1.096944in}%
\pgfsys@useobject{currentmarker}{}%
\end{pgfscope}%
\end{pgfscope}%
\begin{pgfscope}%
\definecolor{textcolor}{rgb}{0.000000,0.000000,0.000000}%
\pgfsetstrokecolor{textcolor}%
\pgfsetfillcolor{textcolor}%
\pgftext[x=0.348333in, y=1.048750in, left, base]{\color{textcolor}\rmfamily\fontsize{10.000000}{12.000000}\selectfont \(\displaystyle {0}\)}%
\end{pgfscope}%
\begin{pgfscope}%
\pgfsetbuttcap%
\pgfsetroundjoin%
\definecolor{currentfill}{rgb}{0.000000,0.000000,0.000000}%
\pgfsetfillcolor{currentfill}%
\pgfsetlinewidth{0.803000pt}%
\definecolor{currentstroke}{rgb}{0.000000,0.000000,0.000000}%
\pgfsetstrokecolor{currentstroke}%
\pgfsetdash{}{0pt}%
\pgfsys@defobject{currentmarker}{\pgfqpoint{-0.048611in}{0.000000in}}{\pgfqpoint{-0.000000in}{0.000000in}}{%
\pgfpathmoveto{\pgfqpoint{-0.000000in}{0.000000in}}%
\pgfpathlineto{\pgfqpoint{-0.048611in}{0.000000in}}%
\pgfusepath{stroke,fill}%
}%
\begin{pgfscope}%
\pgfsys@transformshift{0.515000in}{1.562491in}%
\pgfsys@useobject{currentmarker}{}%
\end{pgfscope}%
\end{pgfscope}%
\begin{pgfscope}%
\definecolor{textcolor}{rgb}{0.000000,0.000000,0.000000}%
\pgfsetstrokecolor{textcolor}%
\pgfsetfillcolor{textcolor}%
\pgftext[x=0.278889in, y=1.514297in, left, base]{\color{textcolor}\rmfamily\fontsize{10.000000}{12.000000}\selectfont \(\displaystyle {10}\)}%
\end{pgfscope}%
\begin{pgfscope}%
\pgfsetbuttcap%
\pgfsetroundjoin%
\definecolor{currentfill}{rgb}{0.000000,0.000000,0.000000}%
\pgfsetfillcolor{currentfill}%
\pgfsetlinewidth{0.803000pt}%
\definecolor{currentstroke}{rgb}{0.000000,0.000000,0.000000}%
\pgfsetstrokecolor{currentstroke}%
\pgfsetdash{}{0pt}%
\pgfsys@defobject{currentmarker}{\pgfqpoint{-0.048611in}{0.000000in}}{\pgfqpoint{-0.000000in}{0.000000in}}{%
\pgfpathmoveto{\pgfqpoint{-0.000000in}{0.000000in}}%
\pgfpathlineto{\pgfqpoint{-0.048611in}{0.000000in}}%
\pgfusepath{stroke,fill}%
}%
\begin{pgfscope}%
\pgfsys@transformshift{0.515000in}{2.028038in}%
\pgfsys@useobject{currentmarker}{}%
\end{pgfscope}%
\end{pgfscope}%
\begin{pgfscope}%
\definecolor{textcolor}{rgb}{0.000000,0.000000,0.000000}%
\pgfsetstrokecolor{textcolor}%
\pgfsetfillcolor{textcolor}%
\pgftext[x=0.278889in, y=1.979843in, left, base]{\color{textcolor}\rmfamily\fontsize{10.000000}{12.000000}\selectfont \(\displaystyle {20}\)}%
\end{pgfscope}%
\begin{pgfscope}%
\pgfsetbuttcap%
\pgfsetroundjoin%
\definecolor{currentfill}{rgb}{0.000000,0.000000,0.000000}%
\pgfsetfillcolor{currentfill}%
\pgfsetlinewidth{0.803000pt}%
\definecolor{currentstroke}{rgb}{0.000000,0.000000,0.000000}%
\pgfsetstrokecolor{currentstroke}%
\pgfsetdash{}{0pt}%
\pgfsys@defobject{currentmarker}{\pgfqpoint{-0.048611in}{0.000000in}}{\pgfqpoint{-0.000000in}{0.000000in}}{%
\pgfpathmoveto{\pgfqpoint{-0.000000in}{0.000000in}}%
\pgfpathlineto{\pgfqpoint{-0.048611in}{0.000000in}}%
\pgfusepath{stroke,fill}%
}%
\begin{pgfscope}%
\pgfsys@transformshift{0.515000in}{2.493585in}%
\pgfsys@useobject{currentmarker}{}%
\end{pgfscope}%
\end{pgfscope}%
\begin{pgfscope}%
\definecolor{textcolor}{rgb}{0.000000,0.000000,0.000000}%
\pgfsetstrokecolor{textcolor}%
\pgfsetfillcolor{textcolor}%
\pgftext[x=0.278889in, y=2.445390in, left, base]{\color{textcolor}\rmfamily\fontsize{10.000000}{12.000000}\selectfont \(\displaystyle {30}\)}%
\end{pgfscope}%
\begin{pgfscope}%
\definecolor{textcolor}{rgb}{0.000000,0.000000,0.000000}%
\pgfsetstrokecolor{textcolor}%
\pgfsetfillcolor{textcolor}%
\pgftext[x=0.223333in,y=2.020944in,,bottom,rotate=90.000000]{\color{textcolor}\rmfamily\fontsize{10.000000}{12.000000}\selectfont Percent of Data Set}%
\end{pgfscope}%
\begin{pgfscope}%
\pgfsetrectcap%
\pgfsetmiterjoin%
\pgfsetlinewidth{0.803000pt}%
\definecolor{currentstroke}{rgb}{0.000000,0.000000,0.000000}%
\pgfsetstrokecolor{currentstroke}%
\pgfsetdash{}{0pt}%
\pgfpathmoveto{\pgfqpoint{0.515000in}{1.096944in}}%
\pgfpathlineto{\pgfqpoint{0.515000in}{2.944944in}}%
\pgfusepath{stroke}%
\end{pgfscope}%
\begin{pgfscope}%
\pgfsetrectcap%
\pgfsetmiterjoin%
\pgfsetlinewidth{0.803000pt}%
\definecolor{currentstroke}{rgb}{0.000000,0.000000,0.000000}%
\pgfsetstrokecolor{currentstroke}%
\pgfsetdash{}{0pt}%
\pgfpathmoveto{\pgfqpoint{2.995000in}{1.096944in}}%
\pgfpathlineto{\pgfqpoint{2.995000in}{2.944944in}}%
\pgfusepath{stroke}%
\end{pgfscope}%
\begin{pgfscope}%
\pgfsetrectcap%
\pgfsetmiterjoin%
\pgfsetlinewidth{0.803000pt}%
\definecolor{currentstroke}{rgb}{0.000000,0.000000,0.000000}%
\pgfsetstrokecolor{currentstroke}%
\pgfsetdash{}{0pt}%
\pgfpathmoveto{\pgfqpoint{0.515000in}{1.096944in}}%
\pgfpathlineto{\pgfqpoint{2.995000in}{1.096944in}}%
\pgfusepath{stroke}%
\end{pgfscope}%
\begin{pgfscope}%
\pgfsetrectcap%
\pgfsetmiterjoin%
\pgfsetlinewidth{0.803000pt}%
\definecolor{currentstroke}{rgb}{0.000000,0.000000,0.000000}%
\pgfsetstrokecolor{currentstroke}%
\pgfsetdash{}{0pt}%
\pgfpathmoveto{\pgfqpoint{0.515000in}{2.944944in}}%
\pgfpathlineto{\pgfqpoint{2.995000in}{2.944944in}}%
\pgfusepath{stroke}%
\end{pgfscope}%
\begin{pgfscope}%
\definecolor{textcolor}{rgb}{0.000000,0.000000,0.000000}%
\pgfsetstrokecolor{textcolor}%
\pgfsetfillcolor{textcolor}%
\pgftext[x=1.755000in,y=3.028277in,,base]{\color{textcolor}\rmfamily\fontsize{12.000000}{14.400000}\selectfont Probability Distribution}%
\end{pgfscope}%
\begin{pgfscope}%
\pgfsetbuttcap%
\pgfsetmiterjoin%
\definecolor{currentfill}{rgb}{1.000000,1.000000,1.000000}%
\pgfsetfillcolor{currentfill}%
\pgfsetfillopacity{0.800000}%
\pgfsetlinewidth{1.003750pt}%
\definecolor{currentstroke}{rgb}{0.800000,0.800000,0.800000}%
\pgfsetstrokecolor{currentstroke}%
\pgfsetstrokeopacity{0.800000}%
\pgfsetdash{}{0pt}%
\pgfpathmoveto{\pgfqpoint{1.560833in}{2.444250in}}%
\pgfpathlineto{\pgfqpoint{2.897778in}{2.444250in}}%
\pgfpathquadraticcurveto{\pgfqpoint{2.925556in}{2.444250in}}{\pgfqpoint{2.925556in}{2.472028in}}%
\pgfpathlineto{\pgfqpoint{2.925556in}{2.847722in}}%
\pgfpathquadraticcurveto{\pgfqpoint{2.925556in}{2.875500in}}{\pgfqpoint{2.897778in}{2.875500in}}%
\pgfpathlineto{\pgfqpoint{1.560833in}{2.875500in}}%
\pgfpathquadraticcurveto{\pgfqpoint{1.533056in}{2.875500in}}{\pgfqpoint{1.533056in}{2.847722in}}%
\pgfpathlineto{\pgfqpoint{1.533056in}{2.472028in}}%
\pgfpathquadraticcurveto{\pgfqpoint{1.533056in}{2.444250in}}{\pgfqpoint{1.560833in}{2.444250in}}%
\pgfpathlineto{\pgfqpoint{1.560833in}{2.444250in}}%
\pgfpathclose%
\pgfusepath{stroke,fill}%
\end{pgfscope}%
\begin{pgfscope}%
\pgfsetbuttcap%
\pgfsetmiterjoin%
\pgfsetlinewidth{1.003750pt}%
\definecolor{currentstroke}{rgb}{0.000000,0.000000,0.000000}%
\pgfsetstrokecolor{currentstroke}%
\pgfsetdash{}{0pt}%
\pgfpathmoveto{\pgfqpoint{1.588611in}{2.722027in}}%
\pgfpathlineto{\pgfqpoint{1.866389in}{2.722027in}}%
\pgfpathlineto{\pgfqpoint{1.866389in}{2.819250in}}%
\pgfpathlineto{\pgfqpoint{1.588611in}{2.819250in}}%
\pgfpathlineto{\pgfqpoint{1.588611in}{2.722027in}}%
\pgfpathclose%
\pgfusepath{stroke}%
\end{pgfscope}%
\begin{pgfscope}%
\definecolor{textcolor}{rgb}{0.000000,0.000000,0.000000}%
\pgfsetstrokecolor{textcolor}%
\pgfsetfillcolor{textcolor}%
\pgftext[x=1.977500in,y=2.722027in,left,base]{\color{textcolor}\rmfamily\fontsize{10.000000}{12.000000}\selectfont Negative Class}%
\end{pgfscope}%
\begin{pgfscope}%
\pgfsetbuttcap%
\pgfsetmiterjoin%
\definecolor{currentfill}{rgb}{0.000000,0.000000,0.000000}%
\pgfsetfillcolor{currentfill}%
\pgfsetlinewidth{0.000000pt}%
\definecolor{currentstroke}{rgb}{0.000000,0.000000,0.000000}%
\pgfsetstrokecolor{currentstroke}%
\pgfsetstrokeopacity{0.000000}%
\pgfsetdash{}{0pt}%
\pgfpathmoveto{\pgfqpoint{1.588611in}{2.526750in}}%
\pgfpathlineto{\pgfqpoint{1.866389in}{2.526750in}}%
\pgfpathlineto{\pgfqpoint{1.866389in}{2.623972in}}%
\pgfpathlineto{\pgfqpoint{1.588611in}{2.623972in}}%
\pgfpathlineto{\pgfqpoint{1.588611in}{2.526750in}}%
\pgfpathclose%
\pgfusepath{fill}%
\end{pgfscope}%
\begin{pgfscope}%
\definecolor{textcolor}{rgb}{0.000000,0.000000,0.000000}%
\pgfsetstrokecolor{textcolor}%
\pgfsetfillcolor{textcolor}%
\pgftext[x=1.977500in,y=2.526750in,left,base]{\color{textcolor}\rmfamily\fontsize{10.000000}{12.000000}\selectfont Positive Class}%
\end{pgfscope}%
\end{pgfpicture}%
\makeatother%
\endgroup%

  &
  \vspace{0pt} %% Creator: Matplotlib, PGF backend
%%
%% To include the figure in your LaTeX document, write
%%   \input{<filename>.pgf}
%%
%% Make sure the required packages are loaded in your preamble
%%   \usepackage{pgf}
%%
%% Also ensure that all the required font packages are loaded; for instance,
%% the lmodern package is sometimes necessary when using math font.
%%   \usepackage{lmodern}
%%
%% Figures using additional raster images can only be included by \input if
%% they are in the same directory as the main LaTeX file. For loading figures
%% from other directories you can use the `import` package
%%   \usepackage{import}
%%
%% and then include the figures with
%%   \import{<path to file>}{<filename>.pgf}
%%
%% Matplotlib used the following preamble
%%   
%%   \usepackage{fontspec}
%%   \makeatletter\@ifpackageloaded{underscore}{}{\usepackage[strings]{underscore}}\makeatother
%%
\begingroup%
\makeatletter%
\begin{pgfpicture}%
\pgfpathrectangle{\pgfpointorigin}{\pgfqpoint{3.144311in}{2.646444in}}%
\pgfusepath{use as bounding box, clip}%
\begin{pgfscope}%
\pgfsetbuttcap%
\pgfsetmiterjoin%
\definecolor{currentfill}{rgb}{1.000000,1.000000,1.000000}%
\pgfsetfillcolor{currentfill}%
\pgfsetlinewidth{0.000000pt}%
\definecolor{currentstroke}{rgb}{1.000000,1.000000,1.000000}%
\pgfsetstrokecolor{currentstroke}%
\pgfsetdash{}{0pt}%
\pgfpathmoveto{\pgfqpoint{0.000000in}{0.000000in}}%
\pgfpathlineto{\pgfqpoint{3.144311in}{0.000000in}}%
\pgfpathlineto{\pgfqpoint{3.144311in}{2.646444in}}%
\pgfpathlineto{\pgfqpoint{0.000000in}{2.646444in}}%
\pgfpathlineto{\pgfqpoint{0.000000in}{0.000000in}}%
\pgfpathclose%
\pgfusepath{fill}%
\end{pgfscope}%
\begin{pgfscope}%
\pgfsetbuttcap%
\pgfsetmiterjoin%
\definecolor{currentfill}{rgb}{1.000000,1.000000,1.000000}%
\pgfsetfillcolor{currentfill}%
\pgfsetlinewidth{0.000000pt}%
\definecolor{currentstroke}{rgb}{0.000000,0.000000,0.000000}%
\pgfsetstrokecolor{currentstroke}%
\pgfsetstrokeopacity{0.000000}%
\pgfsetdash{}{0pt}%
\pgfpathmoveto{\pgfqpoint{0.553581in}{0.499444in}}%
\pgfpathlineto{\pgfqpoint{3.033581in}{0.499444in}}%
\pgfpathlineto{\pgfqpoint{3.033581in}{2.347444in}}%
\pgfpathlineto{\pgfqpoint{0.553581in}{2.347444in}}%
\pgfpathlineto{\pgfqpoint{0.553581in}{0.499444in}}%
\pgfpathclose%
\pgfusepath{fill}%
\end{pgfscope}%
\begin{pgfscope}%
\pgfsetbuttcap%
\pgfsetroundjoin%
\definecolor{currentfill}{rgb}{0.000000,0.000000,0.000000}%
\pgfsetfillcolor{currentfill}%
\pgfsetlinewidth{0.803000pt}%
\definecolor{currentstroke}{rgb}{0.000000,0.000000,0.000000}%
\pgfsetstrokecolor{currentstroke}%
\pgfsetdash{}{0pt}%
\pgfsys@defobject{currentmarker}{\pgfqpoint{0.000000in}{-0.048611in}}{\pgfqpoint{0.000000in}{0.000000in}}{%
\pgfpathmoveto{\pgfqpoint{0.000000in}{0.000000in}}%
\pgfpathlineto{\pgfqpoint{0.000000in}{-0.048611in}}%
\pgfusepath{stroke,fill}%
}%
\begin{pgfscope}%
\pgfsys@transformshift{0.666308in}{0.499444in}%
\pgfsys@useobject{currentmarker}{}%
\end{pgfscope}%
\end{pgfscope}%
\begin{pgfscope}%
\definecolor{textcolor}{rgb}{0.000000,0.000000,0.000000}%
\pgfsetstrokecolor{textcolor}%
\pgfsetfillcolor{textcolor}%
\pgftext[x=0.666308in,y=0.402222in,,top]{\color{textcolor}\rmfamily\fontsize{10.000000}{12.000000}\selectfont \(\displaystyle {0.00}\)}%
\end{pgfscope}%
\begin{pgfscope}%
\pgfsetbuttcap%
\pgfsetroundjoin%
\definecolor{currentfill}{rgb}{0.000000,0.000000,0.000000}%
\pgfsetfillcolor{currentfill}%
\pgfsetlinewidth{0.803000pt}%
\definecolor{currentstroke}{rgb}{0.000000,0.000000,0.000000}%
\pgfsetstrokecolor{currentstroke}%
\pgfsetdash{}{0pt}%
\pgfsys@defobject{currentmarker}{\pgfqpoint{0.000000in}{-0.048611in}}{\pgfqpoint{0.000000in}{0.000000in}}{%
\pgfpathmoveto{\pgfqpoint{0.000000in}{0.000000in}}%
\pgfpathlineto{\pgfqpoint{0.000000in}{-0.048611in}}%
\pgfusepath{stroke,fill}%
}%
\begin{pgfscope}%
\pgfsys@transformshift{1.229944in}{0.499444in}%
\pgfsys@useobject{currentmarker}{}%
\end{pgfscope}%
\end{pgfscope}%
\begin{pgfscope}%
\definecolor{textcolor}{rgb}{0.000000,0.000000,0.000000}%
\pgfsetstrokecolor{textcolor}%
\pgfsetfillcolor{textcolor}%
\pgftext[x=1.229944in,y=0.402222in,,top]{\color{textcolor}\rmfamily\fontsize{10.000000}{12.000000}\selectfont \(\displaystyle {0.25}\)}%
\end{pgfscope}%
\begin{pgfscope}%
\pgfsetbuttcap%
\pgfsetroundjoin%
\definecolor{currentfill}{rgb}{0.000000,0.000000,0.000000}%
\pgfsetfillcolor{currentfill}%
\pgfsetlinewidth{0.803000pt}%
\definecolor{currentstroke}{rgb}{0.000000,0.000000,0.000000}%
\pgfsetstrokecolor{currentstroke}%
\pgfsetdash{}{0pt}%
\pgfsys@defobject{currentmarker}{\pgfqpoint{0.000000in}{-0.048611in}}{\pgfqpoint{0.000000in}{0.000000in}}{%
\pgfpathmoveto{\pgfqpoint{0.000000in}{0.000000in}}%
\pgfpathlineto{\pgfqpoint{0.000000in}{-0.048611in}}%
\pgfusepath{stroke,fill}%
}%
\begin{pgfscope}%
\pgfsys@transformshift{1.793581in}{0.499444in}%
\pgfsys@useobject{currentmarker}{}%
\end{pgfscope}%
\end{pgfscope}%
\begin{pgfscope}%
\definecolor{textcolor}{rgb}{0.000000,0.000000,0.000000}%
\pgfsetstrokecolor{textcolor}%
\pgfsetfillcolor{textcolor}%
\pgftext[x=1.793581in,y=0.402222in,,top]{\color{textcolor}\rmfamily\fontsize{10.000000}{12.000000}\selectfont \(\displaystyle {0.50}\)}%
\end{pgfscope}%
\begin{pgfscope}%
\pgfsetbuttcap%
\pgfsetroundjoin%
\definecolor{currentfill}{rgb}{0.000000,0.000000,0.000000}%
\pgfsetfillcolor{currentfill}%
\pgfsetlinewidth{0.803000pt}%
\definecolor{currentstroke}{rgb}{0.000000,0.000000,0.000000}%
\pgfsetstrokecolor{currentstroke}%
\pgfsetdash{}{0pt}%
\pgfsys@defobject{currentmarker}{\pgfqpoint{0.000000in}{-0.048611in}}{\pgfqpoint{0.000000in}{0.000000in}}{%
\pgfpathmoveto{\pgfqpoint{0.000000in}{0.000000in}}%
\pgfpathlineto{\pgfqpoint{0.000000in}{-0.048611in}}%
\pgfusepath{stroke,fill}%
}%
\begin{pgfscope}%
\pgfsys@transformshift{2.357217in}{0.499444in}%
\pgfsys@useobject{currentmarker}{}%
\end{pgfscope}%
\end{pgfscope}%
\begin{pgfscope}%
\definecolor{textcolor}{rgb}{0.000000,0.000000,0.000000}%
\pgfsetstrokecolor{textcolor}%
\pgfsetfillcolor{textcolor}%
\pgftext[x=2.357217in,y=0.402222in,,top]{\color{textcolor}\rmfamily\fontsize{10.000000}{12.000000}\selectfont \(\displaystyle {0.75}\)}%
\end{pgfscope}%
\begin{pgfscope}%
\pgfsetbuttcap%
\pgfsetroundjoin%
\definecolor{currentfill}{rgb}{0.000000,0.000000,0.000000}%
\pgfsetfillcolor{currentfill}%
\pgfsetlinewidth{0.803000pt}%
\definecolor{currentstroke}{rgb}{0.000000,0.000000,0.000000}%
\pgfsetstrokecolor{currentstroke}%
\pgfsetdash{}{0pt}%
\pgfsys@defobject{currentmarker}{\pgfqpoint{0.000000in}{-0.048611in}}{\pgfqpoint{0.000000in}{0.000000in}}{%
\pgfpathmoveto{\pgfqpoint{0.000000in}{0.000000in}}%
\pgfpathlineto{\pgfqpoint{0.000000in}{-0.048611in}}%
\pgfusepath{stroke,fill}%
}%
\begin{pgfscope}%
\pgfsys@transformshift{2.920853in}{0.499444in}%
\pgfsys@useobject{currentmarker}{}%
\end{pgfscope}%
\end{pgfscope}%
\begin{pgfscope}%
\definecolor{textcolor}{rgb}{0.000000,0.000000,0.000000}%
\pgfsetstrokecolor{textcolor}%
\pgfsetfillcolor{textcolor}%
\pgftext[x=2.920853in,y=0.402222in,,top]{\color{textcolor}\rmfamily\fontsize{10.000000}{12.000000}\selectfont \(\displaystyle {1.00}\)}%
\end{pgfscope}%
\begin{pgfscope}%
\definecolor{textcolor}{rgb}{0.000000,0.000000,0.000000}%
\pgfsetstrokecolor{textcolor}%
\pgfsetfillcolor{textcolor}%
\pgftext[x=1.793581in,y=0.223333in,,top]{\color{textcolor}\rmfamily\fontsize{10.000000}{12.000000}\selectfont False positive rate}%
\end{pgfscope}%
\begin{pgfscope}%
\pgfsetbuttcap%
\pgfsetroundjoin%
\definecolor{currentfill}{rgb}{0.000000,0.000000,0.000000}%
\pgfsetfillcolor{currentfill}%
\pgfsetlinewidth{0.803000pt}%
\definecolor{currentstroke}{rgb}{0.000000,0.000000,0.000000}%
\pgfsetstrokecolor{currentstroke}%
\pgfsetdash{}{0pt}%
\pgfsys@defobject{currentmarker}{\pgfqpoint{-0.048611in}{0.000000in}}{\pgfqpoint{-0.000000in}{0.000000in}}{%
\pgfpathmoveto{\pgfqpoint{-0.000000in}{0.000000in}}%
\pgfpathlineto{\pgfqpoint{-0.048611in}{0.000000in}}%
\pgfusepath{stroke,fill}%
}%
\begin{pgfscope}%
\pgfsys@transformshift{0.553581in}{0.583444in}%
\pgfsys@useobject{currentmarker}{}%
\end{pgfscope}%
\end{pgfscope}%
\begin{pgfscope}%
\definecolor{textcolor}{rgb}{0.000000,0.000000,0.000000}%
\pgfsetstrokecolor{textcolor}%
\pgfsetfillcolor{textcolor}%
\pgftext[x=0.278889in, y=0.535250in, left, base]{\color{textcolor}\rmfamily\fontsize{10.000000}{12.000000}\selectfont \(\displaystyle {0.0}\)}%
\end{pgfscope}%
\begin{pgfscope}%
\pgfsetbuttcap%
\pgfsetroundjoin%
\definecolor{currentfill}{rgb}{0.000000,0.000000,0.000000}%
\pgfsetfillcolor{currentfill}%
\pgfsetlinewidth{0.803000pt}%
\definecolor{currentstroke}{rgb}{0.000000,0.000000,0.000000}%
\pgfsetstrokecolor{currentstroke}%
\pgfsetdash{}{0pt}%
\pgfsys@defobject{currentmarker}{\pgfqpoint{-0.048611in}{0.000000in}}{\pgfqpoint{-0.000000in}{0.000000in}}{%
\pgfpathmoveto{\pgfqpoint{-0.000000in}{0.000000in}}%
\pgfpathlineto{\pgfqpoint{-0.048611in}{0.000000in}}%
\pgfusepath{stroke,fill}%
}%
\begin{pgfscope}%
\pgfsys@transformshift{0.553581in}{0.919444in}%
\pgfsys@useobject{currentmarker}{}%
\end{pgfscope}%
\end{pgfscope}%
\begin{pgfscope}%
\definecolor{textcolor}{rgb}{0.000000,0.000000,0.000000}%
\pgfsetstrokecolor{textcolor}%
\pgfsetfillcolor{textcolor}%
\pgftext[x=0.278889in, y=0.871250in, left, base]{\color{textcolor}\rmfamily\fontsize{10.000000}{12.000000}\selectfont \(\displaystyle {0.2}\)}%
\end{pgfscope}%
\begin{pgfscope}%
\pgfsetbuttcap%
\pgfsetroundjoin%
\definecolor{currentfill}{rgb}{0.000000,0.000000,0.000000}%
\pgfsetfillcolor{currentfill}%
\pgfsetlinewidth{0.803000pt}%
\definecolor{currentstroke}{rgb}{0.000000,0.000000,0.000000}%
\pgfsetstrokecolor{currentstroke}%
\pgfsetdash{}{0pt}%
\pgfsys@defobject{currentmarker}{\pgfqpoint{-0.048611in}{0.000000in}}{\pgfqpoint{-0.000000in}{0.000000in}}{%
\pgfpathmoveto{\pgfqpoint{-0.000000in}{0.000000in}}%
\pgfpathlineto{\pgfqpoint{-0.048611in}{0.000000in}}%
\pgfusepath{stroke,fill}%
}%
\begin{pgfscope}%
\pgfsys@transformshift{0.553581in}{1.255444in}%
\pgfsys@useobject{currentmarker}{}%
\end{pgfscope}%
\end{pgfscope}%
\begin{pgfscope}%
\definecolor{textcolor}{rgb}{0.000000,0.000000,0.000000}%
\pgfsetstrokecolor{textcolor}%
\pgfsetfillcolor{textcolor}%
\pgftext[x=0.278889in, y=1.207250in, left, base]{\color{textcolor}\rmfamily\fontsize{10.000000}{12.000000}\selectfont \(\displaystyle {0.4}\)}%
\end{pgfscope}%
\begin{pgfscope}%
\pgfsetbuttcap%
\pgfsetroundjoin%
\definecolor{currentfill}{rgb}{0.000000,0.000000,0.000000}%
\pgfsetfillcolor{currentfill}%
\pgfsetlinewidth{0.803000pt}%
\definecolor{currentstroke}{rgb}{0.000000,0.000000,0.000000}%
\pgfsetstrokecolor{currentstroke}%
\pgfsetdash{}{0pt}%
\pgfsys@defobject{currentmarker}{\pgfqpoint{-0.048611in}{0.000000in}}{\pgfqpoint{-0.000000in}{0.000000in}}{%
\pgfpathmoveto{\pgfqpoint{-0.000000in}{0.000000in}}%
\pgfpathlineto{\pgfqpoint{-0.048611in}{0.000000in}}%
\pgfusepath{stroke,fill}%
}%
\begin{pgfscope}%
\pgfsys@transformshift{0.553581in}{1.591444in}%
\pgfsys@useobject{currentmarker}{}%
\end{pgfscope}%
\end{pgfscope}%
\begin{pgfscope}%
\definecolor{textcolor}{rgb}{0.000000,0.000000,0.000000}%
\pgfsetstrokecolor{textcolor}%
\pgfsetfillcolor{textcolor}%
\pgftext[x=0.278889in, y=1.543250in, left, base]{\color{textcolor}\rmfamily\fontsize{10.000000}{12.000000}\selectfont \(\displaystyle {0.6}\)}%
\end{pgfscope}%
\begin{pgfscope}%
\pgfsetbuttcap%
\pgfsetroundjoin%
\definecolor{currentfill}{rgb}{0.000000,0.000000,0.000000}%
\pgfsetfillcolor{currentfill}%
\pgfsetlinewidth{0.803000pt}%
\definecolor{currentstroke}{rgb}{0.000000,0.000000,0.000000}%
\pgfsetstrokecolor{currentstroke}%
\pgfsetdash{}{0pt}%
\pgfsys@defobject{currentmarker}{\pgfqpoint{-0.048611in}{0.000000in}}{\pgfqpoint{-0.000000in}{0.000000in}}{%
\pgfpathmoveto{\pgfqpoint{-0.000000in}{0.000000in}}%
\pgfpathlineto{\pgfqpoint{-0.048611in}{0.000000in}}%
\pgfusepath{stroke,fill}%
}%
\begin{pgfscope}%
\pgfsys@transformshift{0.553581in}{1.927444in}%
\pgfsys@useobject{currentmarker}{}%
\end{pgfscope}%
\end{pgfscope}%
\begin{pgfscope}%
\definecolor{textcolor}{rgb}{0.000000,0.000000,0.000000}%
\pgfsetstrokecolor{textcolor}%
\pgfsetfillcolor{textcolor}%
\pgftext[x=0.278889in, y=1.879250in, left, base]{\color{textcolor}\rmfamily\fontsize{10.000000}{12.000000}\selectfont \(\displaystyle {0.8}\)}%
\end{pgfscope}%
\begin{pgfscope}%
\pgfsetbuttcap%
\pgfsetroundjoin%
\definecolor{currentfill}{rgb}{0.000000,0.000000,0.000000}%
\pgfsetfillcolor{currentfill}%
\pgfsetlinewidth{0.803000pt}%
\definecolor{currentstroke}{rgb}{0.000000,0.000000,0.000000}%
\pgfsetstrokecolor{currentstroke}%
\pgfsetdash{}{0pt}%
\pgfsys@defobject{currentmarker}{\pgfqpoint{-0.048611in}{0.000000in}}{\pgfqpoint{-0.000000in}{0.000000in}}{%
\pgfpathmoveto{\pgfqpoint{-0.000000in}{0.000000in}}%
\pgfpathlineto{\pgfqpoint{-0.048611in}{0.000000in}}%
\pgfusepath{stroke,fill}%
}%
\begin{pgfscope}%
\pgfsys@transformshift{0.553581in}{2.263444in}%
\pgfsys@useobject{currentmarker}{}%
\end{pgfscope}%
\end{pgfscope}%
\begin{pgfscope}%
\definecolor{textcolor}{rgb}{0.000000,0.000000,0.000000}%
\pgfsetstrokecolor{textcolor}%
\pgfsetfillcolor{textcolor}%
\pgftext[x=0.278889in, y=2.215250in, left, base]{\color{textcolor}\rmfamily\fontsize{10.000000}{12.000000}\selectfont \(\displaystyle {1.0}\)}%
\end{pgfscope}%
\begin{pgfscope}%
\definecolor{textcolor}{rgb}{0.000000,0.000000,0.000000}%
\pgfsetstrokecolor{textcolor}%
\pgfsetfillcolor{textcolor}%
\pgftext[x=0.223333in,y=1.423444in,,bottom,rotate=90.000000]{\color{textcolor}\rmfamily\fontsize{10.000000}{12.000000}\selectfont True positive rate}%
\end{pgfscope}%
\begin{pgfscope}%
\pgfpathrectangle{\pgfqpoint{0.553581in}{0.499444in}}{\pgfqpoint{2.480000in}{1.848000in}}%
\pgfusepath{clip}%
\pgfsetbuttcap%
\pgfsetroundjoin%
\pgfsetlinewidth{1.505625pt}%
\definecolor{currentstroke}{rgb}{0.000000,0.000000,0.000000}%
\pgfsetstrokecolor{currentstroke}%
\pgfsetdash{{5.550000pt}{2.400000pt}}{0.000000pt}%
\pgfpathmoveto{\pgfqpoint{0.666308in}{0.583444in}}%
\pgfpathlineto{\pgfqpoint{2.920853in}{2.263444in}}%
\pgfusepath{stroke}%
\end{pgfscope}%
\begin{pgfscope}%
\pgfpathrectangle{\pgfqpoint{0.553581in}{0.499444in}}{\pgfqpoint{2.480000in}{1.848000in}}%
\pgfusepath{clip}%
\pgfsetrectcap%
\pgfsetroundjoin%
\pgfsetlinewidth{1.505625pt}%
\definecolor{currentstroke}{rgb}{0.121569,0.466667,0.705882}%
\pgfsetstrokecolor{currentstroke}%
\pgfsetdash{}{0pt}%
\pgfpathmoveto{\pgfqpoint{0.666308in}{0.583444in}}%
\pgfpathlineto{\pgfqpoint{0.669652in}{0.584032in}}%
\pgfpathlineto{\pgfqpoint{0.670742in}{0.598648in}}%
\pgfpathlineto{\pgfqpoint{0.671944in}{0.599740in}}%
\pgfpathlineto{\pgfqpoint{0.672996in}{0.602596in}}%
\pgfpathlineto{\pgfqpoint{0.673879in}{0.603688in}}%
\pgfpathlineto{\pgfqpoint{0.674988in}{0.605956in}}%
\pgfpathlineto{\pgfqpoint{0.675270in}{0.606712in}}%
\pgfpathlineto{\pgfqpoint{0.676378in}{0.610072in}}%
\pgfpathlineto{\pgfqpoint{0.676716in}{0.610912in}}%
\pgfpathlineto{\pgfqpoint{0.677825in}{0.615952in}}%
\pgfpathlineto{\pgfqpoint{0.678182in}{0.617044in}}%
\pgfpathlineto{\pgfqpoint{0.679253in}{0.620656in}}%
\pgfpathlineto{\pgfqpoint{0.679798in}{0.621748in}}%
\pgfpathlineto{\pgfqpoint{0.680906in}{0.627460in}}%
\pgfpathlineto{\pgfqpoint{0.681056in}{0.628552in}}%
\pgfpathlineto{\pgfqpoint{0.682146in}{0.634012in}}%
\pgfpathlineto{\pgfqpoint{0.682428in}{0.634936in}}%
\pgfpathlineto{\pgfqpoint{0.683461in}{0.641152in}}%
\pgfpathlineto{\pgfqpoint{0.683950in}{0.642244in}}%
\pgfpathlineto{\pgfqpoint{0.685058in}{0.648628in}}%
\pgfpathlineto{\pgfqpoint{0.685340in}{0.649636in}}%
\pgfpathlineto{\pgfqpoint{0.686430in}{0.659296in}}%
\pgfpathlineto{\pgfqpoint{0.686524in}{0.660136in}}%
\pgfpathlineto{\pgfqpoint{0.687613in}{0.664336in}}%
\pgfpathlineto{\pgfqpoint{0.687820in}{0.665092in}}%
\pgfpathlineto{\pgfqpoint{0.688929in}{0.673240in}}%
\pgfpathlineto{\pgfqpoint{0.689135in}{0.674332in}}%
\pgfpathlineto{\pgfqpoint{0.690225in}{0.681472in}}%
\pgfpathlineto{\pgfqpoint{0.690695in}{0.682564in}}%
\pgfpathlineto{\pgfqpoint{0.691784in}{0.690376in}}%
\pgfpathlineto{\pgfqpoint{0.691935in}{0.691300in}}%
\pgfpathlineto{\pgfqpoint{0.693024in}{0.698272in}}%
\pgfpathlineto{\pgfqpoint{0.693175in}{0.699196in}}%
\pgfpathlineto{\pgfqpoint{0.694264in}{0.708016in}}%
\pgfpathlineto{\pgfqpoint{0.694471in}{0.709108in}}%
\pgfpathlineto{\pgfqpoint{0.695579in}{0.716752in}}%
\pgfpathlineto{\pgfqpoint{0.695711in}{0.717844in}}%
\pgfpathlineto{\pgfqpoint{0.696819in}{0.726244in}}%
\pgfpathlineto{\pgfqpoint{0.697045in}{0.727252in}}%
\pgfpathlineto{\pgfqpoint{0.698153in}{0.736912in}}%
\pgfpathlineto{\pgfqpoint{0.698229in}{0.737752in}}%
\pgfpathlineto{\pgfqpoint{0.699318in}{0.748168in}}%
\pgfpathlineto{\pgfqpoint{0.699469in}{0.749092in}}%
\pgfpathlineto{\pgfqpoint{0.700558in}{0.755728in}}%
\pgfpathlineto{\pgfqpoint{0.700746in}{0.756484in}}%
\pgfpathlineto{\pgfqpoint{0.701855in}{0.763708in}}%
\pgfpathlineto{\pgfqpoint{0.701930in}{0.764548in}}%
\pgfpathlineto{\pgfqpoint{0.703038in}{0.772948in}}%
\pgfpathlineto{\pgfqpoint{0.703207in}{0.774040in}}%
\pgfpathlineto{\pgfqpoint{0.704241in}{0.780844in}}%
\pgfpathlineto{\pgfqpoint{0.704504in}{0.781936in}}%
\pgfpathlineto{\pgfqpoint{0.705612in}{0.790672in}}%
\pgfpathlineto{\pgfqpoint{0.705781in}{0.791680in}}%
\pgfpathlineto{\pgfqpoint{0.706890in}{0.800836in}}%
\pgfpathlineto{\pgfqpoint{0.707040in}{0.801760in}}%
\pgfpathlineto{\pgfqpoint{0.708149in}{0.810496in}}%
\pgfpathlineto{\pgfqpoint{0.708243in}{0.811588in}}%
\pgfpathlineto{\pgfqpoint{0.709351in}{0.820156in}}%
\pgfpathlineto{\pgfqpoint{0.709520in}{0.821248in}}%
\pgfpathlineto{\pgfqpoint{0.710629in}{0.829564in}}%
\pgfpathlineto{\pgfqpoint{0.710892in}{0.830656in}}%
\pgfpathlineto{\pgfqpoint{0.712000in}{0.840316in}}%
\pgfpathlineto{\pgfqpoint{0.712113in}{0.841408in}}%
\pgfpathlineto{\pgfqpoint{0.713203in}{0.847540in}}%
\pgfpathlineto{\pgfqpoint{0.713296in}{0.848632in}}%
\pgfpathlineto{\pgfqpoint{0.714292in}{0.856444in}}%
\pgfpathlineto{\pgfqpoint{0.714593in}{0.857200in}}%
\pgfpathlineto{\pgfqpoint{0.715701in}{0.866944in}}%
\pgfpathlineto{\pgfqpoint{0.715852in}{0.867868in}}%
\pgfpathlineto{\pgfqpoint{0.716960in}{0.876520in}}%
\pgfpathlineto{\pgfqpoint{0.717092in}{0.877528in}}%
\pgfpathlineto{\pgfqpoint{0.718200in}{0.886936in}}%
\pgfpathlineto{\pgfqpoint{0.718407in}{0.887860in}}%
\pgfpathlineto{\pgfqpoint{0.719515in}{0.898108in}}%
\pgfpathlineto{\pgfqpoint{0.719609in}{0.899032in}}%
\pgfpathlineto{\pgfqpoint{0.720718in}{0.908440in}}%
\pgfpathlineto{\pgfqpoint{0.720812in}{0.909448in}}%
\pgfpathlineto{\pgfqpoint{0.721901in}{0.918772in}}%
\pgfpathlineto{\pgfqpoint{0.722108in}{0.919864in}}%
\pgfpathlineto{\pgfqpoint{0.723198in}{0.927676in}}%
\pgfpathlineto{\pgfqpoint{0.723273in}{0.928600in}}%
\pgfpathlineto{\pgfqpoint{0.724381in}{0.935320in}}%
\pgfpathlineto{\pgfqpoint{0.724588in}{0.936412in}}%
\pgfpathlineto{\pgfqpoint{0.725696in}{0.944560in}}%
\pgfpathlineto{\pgfqpoint{0.725903in}{0.945652in}}%
\pgfpathlineto{\pgfqpoint{0.726974in}{0.952624in}}%
\pgfpathlineto{\pgfqpoint{0.727218in}{0.953296in}}%
\pgfpathlineto{\pgfqpoint{0.728308in}{0.963964in}}%
\pgfpathlineto{\pgfqpoint{0.728515in}{0.964972in}}%
\pgfpathlineto{\pgfqpoint{0.729623in}{0.972616in}}%
\pgfpathlineto{\pgfqpoint{0.729811in}{0.973624in}}%
\pgfpathlineto{\pgfqpoint{0.730919in}{0.980008in}}%
\pgfpathlineto{\pgfqpoint{0.731220in}{0.980932in}}%
\pgfpathlineto{\pgfqpoint{0.732291in}{0.986980in}}%
\pgfpathlineto{\pgfqpoint{0.732498in}{0.987736in}}%
\pgfpathlineto{\pgfqpoint{0.733606in}{0.995044in}}%
\pgfpathlineto{\pgfqpoint{0.733775in}{0.995968in}}%
\pgfpathlineto{\pgfqpoint{0.734865in}{1.005040in}}%
\pgfpathlineto{\pgfqpoint{0.735053in}{1.005964in}}%
\pgfpathlineto{\pgfqpoint{0.736124in}{1.012516in}}%
\pgfpathlineto{\pgfqpoint{0.736424in}{1.013608in}}%
\pgfpathlineto{\pgfqpoint{0.737514in}{1.022176in}}%
\pgfpathlineto{\pgfqpoint{0.737664in}{1.023268in}}%
\pgfpathlineto{\pgfqpoint{0.738754in}{1.029736in}}%
\pgfpathlineto{\pgfqpoint{0.738998in}{1.030660in}}%
\pgfpathlineto{\pgfqpoint{0.740107in}{1.039480in}}%
\pgfpathlineto{\pgfqpoint{0.740389in}{1.040572in}}%
\pgfpathlineto{\pgfqpoint{0.741441in}{1.046704in}}%
\pgfpathlineto{\pgfqpoint{0.741704in}{1.047796in}}%
\pgfpathlineto{\pgfqpoint{0.742793in}{1.055440in}}%
\pgfpathlineto{\pgfqpoint{0.743188in}{1.056532in}}%
\pgfpathlineto{\pgfqpoint{0.744296in}{1.064680in}}%
\pgfpathlineto{\pgfqpoint{0.744409in}{1.065772in}}%
\pgfpathlineto{\pgfqpoint{0.745518in}{1.073416in}}%
\pgfpathlineto{\pgfqpoint{0.745893in}{1.074340in}}%
\pgfpathlineto{\pgfqpoint{0.747002in}{1.081732in}}%
\pgfpathlineto{\pgfqpoint{0.747115in}{1.082572in}}%
\pgfpathlineto{\pgfqpoint{0.748129in}{1.089292in}}%
\pgfpathlineto{\pgfqpoint{0.748768in}{1.090384in}}%
\pgfpathlineto{\pgfqpoint{0.749876in}{1.098112in}}%
\pgfpathlineto{\pgfqpoint{0.750008in}{1.099120in}}%
\pgfpathlineto{\pgfqpoint{0.751116in}{1.104328in}}%
\pgfpathlineto{\pgfqpoint{0.751267in}{1.105084in}}%
\pgfpathlineto{\pgfqpoint{0.752375in}{1.110460in}}%
\pgfpathlineto{\pgfqpoint{0.752676in}{1.111468in}}%
\pgfpathlineto{\pgfqpoint{0.753766in}{1.116844in}}%
\pgfpathlineto{\pgfqpoint{0.753935in}{1.117852in}}%
\pgfpathlineto{\pgfqpoint{0.755043in}{1.123984in}}%
\pgfpathlineto{\pgfqpoint{0.755456in}{1.125076in}}%
\pgfpathlineto{\pgfqpoint{0.756565in}{1.132132in}}%
\pgfpathlineto{\pgfqpoint{0.756922in}{1.133224in}}%
\pgfpathlineto{\pgfqpoint{0.758012in}{1.139440in}}%
\pgfpathlineto{\pgfqpoint{0.758406in}{1.140532in}}%
\pgfpathlineto{\pgfqpoint{0.759515in}{1.147924in}}%
\pgfpathlineto{\pgfqpoint{0.759609in}{1.148512in}}%
\pgfpathlineto{\pgfqpoint{0.760717in}{1.155652in}}%
\pgfpathlineto{\pgfqpoint{0.760867in}{1.156744in}}%
\pgfpathlineto{\pgfqpoint{0.761976in}{1.162456in}}%
\pgfpathlineto{\pgfqpoint{0.762201in}{1.163380in}}%
\pgfpathlineto{\pgfqpoint{0.763310in}{1.168504in}}%
\pgfpathlineto{\pgfqpoint{0.763498in}{1.169512in}}%
\pgfpathlineto{\pgfqpoint{0.764606in}{1.177660in}}%
\pgfpathlineto{\pgfqpoint{0.764926in}{1.178752in}}%
\pgfpathlineto{\pgfqpoint{0.766015in}{1.184548in}}%
\pgfpathlineto{\pgfqpoint{0.766241in}{1.185556in}}%
\pgfpathlineto{\pgfqpoint{0.767349in}{1.190596in}}%
\pgfpathlineto{\pgfqpoint{0.767763in}{1.191688in}}%
\pgfpathlineto{\pgfqpoint{0.768833in}{1.197148in}}%
\pgfpathlineto{\pgfqpoint{0.769115in}{1.198156in}}%
\pgfpathlineto{\pgfqpoint{0.770186in}{1.202692in}}%
\pgfpathlineto{\pgfqpoint{0.770562in}{1.203784in}}%
\pgfpathlineto{\pgfqpoint{0.771670in}{1.208656in}}%
\pgfpathlineto{\pgfqpoint{0.771802in}{1.209748in}}%
\pgfpathlineto{\pgfqpoint{0.772892in}{1.216216in}}%
\pgfpathlineto{\pgfqpoint{0.773136in}{1.217308in}}%
\pgfpathlineto{\pgfqpoint{0.774244in}{1.222684in}}%
\pgfpathlineto{\pgfqpoint{0.774376in}{1.223692in}}%
\pgfpathlineto{\pgfqpoint{0.775484in}{1.228732in}}%
\pgfpathlineto{\pgfqpoint{0.775541in}{1.229236in}}%
\pgfpathlineto{\pgfqpoint{0.776612in}{1.233856in}}%
\pgfpathlineto{\pgfqpoint{0.776950in}{1.234864in}}%
\pgfpathlineto{\pgfqpoint{0.778039in}{1.240492in}}%
\pgfpathlineto{\pgfqpoint{0.778227in}{1.241416in}}%
\pgfpathlineto{\pgfqpoint{0.779336in}{1.246120in}}%
\pgfpathlineto{\pgfqpoint{0.779580in}{1.247212in}}%
\pgfpathlineto{\pgfqpoint{0.780670in}{1.252504in}}%
\pgfpathlineto{\pgfqpoint{0.781008in}{1.253428in}}%
\pgfpathlineto{\pgfqpoint{0.782098in}{1.259392in}}%
\pgfpathlineto{\pgfqpoint{0.782417in}{1.260400in}}%
\pgfpathlineto{\pgfqpoint{0.783526in}{1.263424in}}%
\pgfpathlineto{\pgfqpoint{0.783676in}{1.264180in}}%
\pgfpathlineto{\pgfqpoint{0.784709in}{1.269808in}}%
\pgfpathlineto{\pgfqpoint{0.784991in}{1.270564in}}%
\pgfpathlineto{\pgfqpoint{0.786081in}{1.276276in}}%
\pgfpathlineto{\pgfqpoint{0.786475in}{1.277368in}}%
\pgfpathlineto{\pgfqpoint{0.787565in}{1.280560in}}%
\pgfpathlineto{\pgfqpoint{0.787828in}{1.281484in}}%
\pgfpathlineto{\pgfqpoint{0.788936in}{1.286524in}}%
\pgfpathlineto{\pgfqpoint{0.789312in}{1.287448in}}%
\pgfpathlineto{\pgfqpoint{0.790421in}{1.294168in}}%
\pgfpathlineto{\pgfqpoint{0.790740in}{1.295176in}}%
\pgfpathlineto{\pgfqpoint{0.791849in}{1.300972in}}%
\pgfpathlineto{\pgfqpoint{0.792093in}{1.301812in}}%
\pgfpathlineto{\pgfqpoint{0.793089in}{1.305592in}}%
\pgfpathlineto{\pgfqpoint{0.793784in}{1.306600in}}%
\pgfpathlineto{\pgfqpoint{0.794892in}{1.310548in}}%
\pgfpathlineto{\pgfqpoint{0.795230in}{1.311640in}}%
\pgfpathlineto{\pgfqpoint{0.796339in}{1.315924in}}%
\pgfpathlineto{\pgfqpoint{0.796639in}{1.317016in}}%
\pgfpathlineto{\pgfqpoint{0.797748in}{1.322392in}}%
\pgfpathlineto{\pgfqpoint{0.798049in}{1.323400in}}%
\pgfpathlineto{\pgfqpoint{0.799138in}{1.327432in}}%
\pgfpathlineto{\pgfqpoint{0.799458in}{1.328524in}}%
\pgfpathlineto{\pgfqpoint{0.800529in}{1.333060in}}%
\pgfpathlineto{\pgfqpoint{0.800848in}{1.333900in}}%
\pgfpathlineto{\pgfqpoint{0.801938in}{1.338604in}}%
\pgfpathlineto{\pgfqpoint{0.802313in}{1.339696in}}%
\pgfpathlineto{\pgfqpoint{0.803384in}{1.343728in}}%
\pgfpathlineto{\pgfqpoint{0.803723in}{1.344820in}}%
\pgfpathlineto{\pgfqpoint{0.804812in}{1.347676in}}%
\pgfpathlineto{\pgfqpoint{0.805038in}{1.348600in}}%
\pgfpathlineto{\pgfqpoint{0.806127in}{1.355152in}}%
\pgfpathlineto{\pgfqpoint{0.806447in}{1.356160in}}%
\pgfpathlineto{\pgfqpoint{0.807536in}{1.361368in}}%
\pgfpathlineto{\pgfqpoint{0.807987in}{1.362376in}}%
\pgfpathlineto{\pgfqpoint{0.809077in}{1.366408in}}%
\pgfpathlineto{\pgfqpoint{0.809490in}{1.367500in}}%
\pgfpathlineto{\pgfqpoint{0.810561in}{1.370272in}}%
\pgfpathlineto{\pgfqpoint{0.810881in}{1.371280in}}%
\pgfpathlineto{\pgfqpoint{0.811952in}{1.374724in}}%
\pgfpathlineto{\pgfqpoint{0.812346in}{1.375732in}}%
\pgfpathlineto{\pgfqpoint{0.813455in}{1.378756in}}%
\pgfpathlineto{\pgfqpoint{0.813774in}{1.379764in}}%
\pgfpathlineto{\pgfqpoint{0.814807in}{1.382872in}}%
\pgfpathlineto{\pgfqpoint{0.815183in}{1.383964in}}%
\pgfpathlineto{\pgfqpoint{0.816273in}{1.388248in}}%
\pgfpathlineto{\pgfqpoint{0.816705in}{1.389340in}}%
\pgfpathlineto{\pgfqpoint{0.817795in}{1.393624in}}%
\pgfpathlineto{\pgfqpoint{0.818133in}{1.394632in}}%
\pgfpathlineto{\pgfqpoint{0.819241in}{1.398580in}}%
\pgfpathlineto{\pgfqpoint{0.819692in}{1.399672in}}%
\pgfpathlineto{\pgfqpoint{0.820763in}{1.402192in}}%
\pgfpathlineto{\pgfqpoint{0.821176in}{1.403284in}}%
\pgfpathlineto{\pgfqpoint{0.822210in}{1.407148in}}%
\pgfpathlineto{\pgfqpoint{0.822792in}{1.408240in}}%
\pgfpathlineto{\pgfqpoint{0.823901in}{1.411768in}}%
\pgfpathlineto{\pgfqpoint{0.824276in}{1.412860in}}%
\pgfpathlineto{\pgfqpoint{0.825329in}{1.417900in}}%
\pgfpathlineto{\pgfqpoint{0.825648in}{1.418908in}}%
\pgfpathlineto{\pgfqpoint{0.826663in}{1.421428in}}%
\pgfpathlineto{\pgfqpoint{0.827038in}{1.422352in}}%
\pgfpathlineto{\pgfqpoint{0.828147in}{1.424620in}}%
\pgfpathlineto{\pgfqpoint{0.828391in}{1.425712in}}%
\pgfpathlineto{\pgfqpoint{0.829499in}{1.429744in}}%
\pgfpathlineto{\pgfqpoint{0.830026in}{1.430752in}}%
\pgfpathlineto{\pgfqpoint{0.831021in}{1.434196in}}%
\pgfpathlineto{\pgfqpoint{0.831453in}{1.435288in}}%
\pgfpathlineto{\pgfqpoint{0.832562in}{1.439152in}}%
\pgfpathlineto{\pgfqpoint{0.833032in}{1.440076in}}%
\pgfpathlineto{\pgfqpoint{0.834121in}{1.443520in}}%
\pgfpathlineto{\pgfqpoint{0.834553in}{1.444612in}}%
\pgfpathlineto{\pgfqpoint{0.835662in}{1.448056in}}%
\pgfpathlineto{\pgfqpoint{0.836113in}{1.448980in}}%
\pgfpathlineto{\pgfqpoint{0.837203in}{1.452256in}}%
\pgfpathlineto{\pgfqpoint{0.837484in}{1.453264in}}%
\pgfpathlineto{\pgfqpoint{0.838593in}{1.456036in}}%
\pgfpathlineto{\pgfqpoint{0.838969in}{1.457044in}}%
\pgfpathlineto{\pgfqpoint{0.840077in}{1.461076in}}%
\pgfpathlineto{\pgfqpoint{0.840453in}{1.462168in}}%
\pgfpathlineto{\pgfqpoint{0.841505in}{1.465108in}}%
\pgfpathlineto{\pgfqpoint{0.842163in}{1.466116in}}%
\pgfpathlineto{\pgfqpoint{0.843233in}{1.469560in}}%
\pgfpathlineto{\pgfqpoint{0.843741in}{1.470484in}}%
\pgfpathlineto{\pgfqpoint{0.844849in}{1.474852in}}%
\pgfpathlineto{\pgfqpoint{0.845244in}{1.475692in}}%
\pgfpathlineto{\pgfqpoint{0.846352in}{1.479388in}}%
\pgfpathlineto{\pgfqpoint{0.846916in}{1.480396in}}%
\pgfpathlineto{\pgfqpoint{0.848024in}{1.482580in}}%
\pgfpathlineto{\pgfqpoint{0.848325in}{1.483672in}}%
\pgfpathlineto{\pgfqpoint{0.849433in}{1.488208in}}%
\pgfpathlineto{\pgfqpoint{0.849772in}{1.489300in}}%
\pgfpathlineto{\pgfqpoint{0.850843in}{1.492324in}}%
\pgfpathlineto{\pgfqpoint{0.851387in}{1.493332in}}%
\pgfpathlineto{\pgfqpoint{0.852364in}{1.496020in}}%
\pgfpathlineto{\pgfqpoint{0.853041in}{1.497112in}}%
\pgfpathlineto{\pgfqpoint{0.854055in}{1.499464in}}%
\pgfpathlineto{\pgfqpoint{0.854563in}{1.500556in}}%
\pgfpathlineto{\pgfqpoint{0.855671in}{1.504252in}}%
\pgfpathlineto{\pgfqpoint{0.856122in}{1.505344in}}%
\pgfpathlineto{\pgfqpoint{0.857230in}{1.509124in}}%
\pgfpathlineto{\pgfqpoint{0.857700in}{1.510216in}}%
\pgfpathlineto{\pgfqpoint{0.858771in}{1.511812in}}%
\pgfpathlineto{\pgfqpoint{0.859372in}{1.512904in}}%
\pgfpathlineto{\pgfqpoint{0.859466in}{1.513576in}}%
\pgfpathlineto{\pgfqpoint{0.876093in}{1.514668in}}%
\pgfpathlineto{\pgfqpoint{0.877164in}{1.517692in}}%
\pgfpathlineto{\pgfqpoint{0.877615in}{1.518784in}}%
\pgfpathlineto{\pgfqpoint{0.878724in}{1.522060in}}%
\pgfpathlineto{\pgfqpoint{0.879137in}{1.523068in}}%
\pgfpathlineto{\pgfqpoint{0.880227in}{1.527268in}}%
\pgfpathlineto{\pgfqpoint{0.880565in}{1.528276in}}%
\pgfpathlineto{\pgfqpoint{0.881579in}{1.529872in}}%
\pgfpathlineto{\pgfqpoint{0.882181in}{1.530964in}}%
\pgfpathlineto{\pgfqpoint{0.883289in}{1.533400in}}%
\pgfpathlineto{\pgfqpoint{0.883740in}{1.534408in}}%
\pgfpathlineto{\pgfqpoint{0.884849in}{1.538020in}}%
\pgfpathlineto{\pgfqpoint{0.885318in}{1.539112in}}%
\pgfpathlineto{\pgfqpoint{0.886352in}{1.541884in}}%
\pgfpathlineto{\pgfqpoint{0.886972in}{1.542976in}}%
\pgfpathlineto{\pgfqpoint{0.888024in}{1.545496in}}%
\pgfpathlineto{\pgfqpoint{0.888587in}{1.546588in}}%
\pgfpathlineto{\pgfqpoint{0.889583in}{1.549444in}}%
\pgfpathlineto{\pgfqpoint{0.890072in}{1.550536in}}%
\pgfpathlineto{\pgfqpoint{0.891180in}{1.552552in}}%
\pgfpathlineto{\pgfqpoint{0.891725in}{1.553644in}}%
\pgfpathlineto{\pgfqpoint{0.892815in}{1.555828in}}%
\pgfpathlineto{\pgfqpoint{0.893190in}{1.556920in}}%
\pgfpathlineto{\pgfqpoint{0.894299in}{1.560028in}}%
\pgfpathlineto{\pgfqpoint{0.894524in}{1.561120in}}%
\pgfpathlineto{\pgfqpoint{0.895614in}{1.563640in}}%
\pgfpathlineto{\pgfqpoint{0.896196in}{1.564732in}}%
\pgfpathlineto{\pgfqpoint{0.897230in}{1.567420in}}%
\pgfpathlineto{\pgfqpoint{0.897906in}{1.568512in}}%
\pgfpathlineto{\pgfqpoint{0.899015in}{1.570864in}}%
\pgfpathlineto{\pgfqpoint{0.900029in}{1.571872in}}%
\pgfpathlineto{\pgfqpoint{0.901138in}{1.574308in}}%
\pgfpathlineto{\pgfqpoint{0.901495in}{1.575316in}}%
\pgfpathlineto{\pgfqpoint{0.902584in}{1.578256in}}%
\pgfpathlineto{\pgfqpoint{0.903148in}{1.579264in}}%
\pgfpathlineto{\pgfqpoint{0.904256in}{1.581952in}}%
\pgfpathlineto{\pgfqpoint{0.905064in}{1.583044in}}%
\pgfpathlineto{\pgfqpoint{0.906135in}{1.585144in}}%
\pgfpathlineto{\pgfqpoint{0.906849in}{1.586152in}}%
\pgfpathlineto{\pgfqpoint{0.907883in}{1.589344in}}%
\pgfpathlineto{\pgfqpoint{0.908596in}{1.590436in}}%
\pgfpathlineto{\pgfqpoint{0.909667in}{1.593208in}}%
\pgfpathlineto{\pgfqpoint{0.910456in}{1.594216in}}%
\pgfpathlineto{\pgfqpoint{0.911565in}{1.597324in}}%
\pgfpathlineto{\pgfqpoint{0.912298in}{1.598332in}}%
\pgfpathlineto{\pgfqpoint{0.913312in}{1.600768in}}%
\pgfpathlineto{\pgfqpoint{0.913575in}{1.601776in}}%
\pgfpathlineto{\pgfqpoint{0.914684in}{1.604296in}}%
\pgfpathlineto{\pgfqpoint{0.915153in}{1.605304in}}%
\pgfpathlineto{\pgfqpoint{0.916262in}{1.607068in}}%
\pgfpathlineto{\pgfqpoint{0.916919in}{1.608160in}}%
\pgfpathlineto{\pgfqpoint{0.917972in}{1.610932in}}%
\pgfpathlineto{\pgfqpoint{0.919024in}{1.612024in}}%
\pgfpathlineto{\pgfqpoint{0.920001in}{1.614376in}}%
\pgfpathlineto{\pgfqpoint{0.921353in}{1.615468in}}%
\pgfpathlineto{\pgfqpoint{0.922274in}{1.617820in}}%
\pgfpathlineto{\pgfqpoint{0.923589in}{1.618912in}}%
\pgfpathlineto{\pgfqpoint{0.924585in}{1.620592in}}%
\pgfpathlineto{\pgfqpoint{0.925524in}{1.621684in}}%
\pgfpathlineto{\pgfqpoint{0.926576in}{1.623616in}}%
\pgfpathlineto{\pgfqpoint{0.927478in}{1.624624in}}%
\pgfpathlineto{\pgfqpoint{0.928587in}{1.627312in}}%
\pgfpathlineto{\pgfqpoint{0.929150in}{1.628320in}}%
\pgfpathlineto{\pgfqpoint{0.930240in}{1.630084in}}%
\pgfpathlineto{\pgfqpoint{0.930879in}{1.631176in}}%
\pgfpathlineto{\pgfqpoint{0.931987in}{1.633528in}}%
\pgfpathlineto{\pgfqpoint{0.932476in}{1.634368in}}%
\pgfpathlineto{\pgfqpoint{0.933547in}{1.636804in}}%
\pgfpathlineto{\pgfqpoint{0.934186in}{1.637812in}}%
\pgfpathlineto{\pgfqpoint{0.935256in}{1.639660in}}%
\pgfpathlineto{\pgfqpoint{0.935764in}{1.640668in}}%
\pgfpathlineto{\pgfqpoint{0.936722in}{1.643020in}}%
\pgfpathlineto{\pgfqpoint{0.937699in}{1.644112in}}%
\pgfpathlineto{\pgfqpoint{0.938807in}{1.645876in}}%
\pgfpathlineto{\pgfqpoint{0.939784in}{1.646968in}}%
\pgfpathlineto{\pgfqpoint{0.940818in}{1.649824in}}%
\pgfpathlineto{\pgfqpoint{0.941269in}{1.650916in}}%
\pgfpathlineto{\pgfqpoint{0.942358in}{1.653100in}}%
\pgfpathlineto{\pgfqpoint{0.943410in}{1.654192in}}%
\pgfpathlineto{\pgfqpoint{0.944463in}{1.656292in}}%
\pgfpathlineto{\pgfqpoint{0.945083in}{1.657384in}}%
\pgfpathlineto{\pgfqpoint{0.946135in}{1.659568in}}%
\pgfpathlineto{\pgfqpoint{0.946849in}{1.660660in}}%
\pgfpathlineto{\pgfqpoint{0.947919in}{1.662592in}}%
\pgfpathlineto{\pgfqpoint{0.948539in}{1.663600in}}%
\pgfpathlineto{\pgfqpoint{0.949648in}{1.665700in}}%
\pgfpathlineto{\pgfqpoint{0.950381in}{1.666792in}}%
\pgfpathlineto{\pgfqpoint{0.951470in}{1.669564in}}%
\pgfpathlineto{\pgfqpoint{0.952034in}{1.670656in}}%
\pgfpathlineto{\pgfqpoint{0.953067in}{1.672420in}}%
\pgfpathlineto{\pgfqpoint{0.953950in}{1.673428in}}%
\pgfpathlineto{\pgfqpoint{0.954890in}{1.675108in}}%
\pgfpathlineto{\pgfqpoint{0.955810in}{1.676200in}}%
\pgfpathlineto{\pgfqpoint{0.956750in}{1.677964in}}%
\pgfpathlineto{\pgfqpoint{0.957483in}{1.678888in}}%
\pgfpathlineto{\pgfqpoint{0.958591in}{1.680820in}}%
\pgfpathlineto{\pgfqpoint{0.959079in}{1.681828in}}%
\pgfpathlineto{\pgfqpoint{0.960132in}{1.684432in}}%
\pgfpathlineto{\pgfqpoint{0.961090in}{1.685524in}}%
\pgfpathlineto{\pgfqpoint{0.962086in}{1.687960in}}%
\pgfpathlineto{\pgfqpoint{0.962856in}{1.688800in}}%
\pgfpathlineto{\pgfqpoint{0.963964in}{1.690900in}}%
\pgfpathlineto{\pgfqpoint{0.964716in}{1.691908in}}%
\pgfpathlineto{\pgfqpoint{0.965618in}{1.693672in}}%
\pgfpathlineto{\pgfqpoint{0.966426in}{1.694764in}}%
\pgfpathlineto{\pgfqpoint{0.967515in}{1.696696in}}%
\pgfpathlineto{\pgfqpoint{0.968417in}{1.697788in}}%
\pgfpathlineto{\pgfqpoint{0.969507in}{1.699552in}}%
\pgfpathlineto{\pgfqpoint{0.970164in}{1.700560in}}%
\pgfpathlineto{\pgfqpoint{0.971273in}{1.703248in}}%
\pgfpathlineto{\pgfqpoint{0.971855in}{1.704340in}}%
\pgfpathlineto{\pgfqpoint{0.972964in}{1.707112in}}%
\pgfpathlineto{\pgfqpoint{0.973922in}{1.708120in}}%
\pgfpathlineto{\pgfqpoint{0.974993in}{1.710304in}}%
\pgfpathlineto{\pgfqpoint{0.975538in}{1.711312in}}%
\pgfpathlineto{\pgfqpoint{0.976609in}{1.713076in}}%
\pgfpathlineto{\pgfqpoint{0.977435in}{1.714168in}}%
\pgfpathlineto{\pgfqpoint{0.978506in}{1.716352in}}%
\pgfpathlineto{\pgfqpoint{0.979164in}{1.717444in}}%
\pgfpathlineto{\pgfqpoint{0.980272in}{1.719544in}}%
\pgfpathlineto{\pgfqpoint{0.980686in}{1.720636in}}%
\pgfpathlineto{\pgfqpoint{0.981700in}{1.721980in}}%
\pgfpathlineto{\pgfqpoint{0.982433in}{1.723072in}}%
\pgfpathlineto{\pgfqpoint{0.983541in}{1.724836in}}%
\pgfpathlineto{\pgfqpoint{0.984293in}{1.725928in}}%
\pgfpathlineto{\pgfqpoint{0.985401in}{1.728364in}}%
\pgfpathlineto{\pgfqpoint{0.986228in}{1.729372in}}%
\pgfpathlineto{\pgfqpoint{0.987336in}{1.731304in}}%
\pgfpathlineto{\pgfqpoint{0.988858in}{1.732228in}}%
\pgfpathlineto{\pgfqpoint{0.989854in}{1.734580in}}%
\pgfpathlineto{\pgfqpoint{0.990549in}{1.735672in}}%
\pgfpathlineto{\pgfqpoint{0.991451in}{1.736932in}}%
\pgfpathlineto{\pgfqpoint{0.992616in}{1.738024in}}%
\pgfpathlineto{\pgfqpoint{0.993630in}{1.739368in}}%
\pgfpathlineto{\pgfqpoint{0.994795in}{1.740460in}}%
\pgfpathlineto{\pgfqpoint{0.995885in}{1.742476in}}%
\pgfpathlineto{\pgfqpoint{0.996824in}{1.743568in}}%
\pgfpathlineto{\pgfqpoint{0.997895in}{1.746340in}}%
\pgfpathlineto{\pgfqpoint{0.998628in}{1.747264in}}%
\pgfpathlineto{\pgfqpoint{0.999473in}{1.749112in}}%
\pgfpathlineto{\pgfqpoint{1.000563in}{1.750204in}}%
\pgfpathlineto{\pgfqpoint{1.001484in}{1.751968in}}%
\pgfpathlineto{\pgfqpoint{1.002479in}{1.753060in}}%
\pgfpathlineto{\pgfqpoint{1.003569in}{1.754824in}}%
\pgfpathlineto{\pgfqpoint{1.004621in}{1.755916in}}%
\pgfpathlineto{\pgfqpoint{1.005692in}{1.757008in}}%
\pgfpathlineto{\pgfqpoint{1.006575in}{1.758100in}}%
\pgfpathlineto{\pgfqpoint{1.007477in}{1.759444in}}%
\pgfpathlineto{\pgfqpoint{1.008341in}{1.760452in}}%
\pgfpathlineto{\pgfqpoint{1.009450in}{1.761796in}}%
\pgfpathlineto{\pgfqpoint{1.010220in}{1.762888in}}%
\pgfpathlineto{\pgfqpoint{1.011272in}{1.764652in}}%
\pgfpathlineto{\pgfqpoint{1.013001in}{1.765660in}}%
\pgfpathlineto{\pgfqpoint{1.014109in}{1.767172in}}%
\pgfpathlineto{\pgfqpoint{1.015312in}{1.768264in}}%
\pgfpathlineto{\pgfqpoint{1.016420in}{1.769776in}}%
\pgfpathlineto{\pgfqpoint{1.016965in}{1.770784in}}%
\pgfpathlineto{\pgfqpoint{1.017923in}{1.772212in}}%
\pgfpathlineto{\pgfqpoint{1.018919in}{1.773304in}}%
\pgfpathlineto{\pgfqpoint{1.019746in}{1.774480in}}%
\pgfpathlineto{\pgfqpoint{1.020967in}{1.775572in}}%
\pgfpathlineto{\pgfqpoint{1.022056in}{1.776832in}}%
\pgfpathlineto{\pgfqpoint{1.022827in}{1.777756in}}%
\pgfpathlineto{\pgfqpoint{1.023916in}{1.779604in}}%
\pgfpathlineto{\pgfqpoint{1.025232in}{1.780696in}}%
\pgfpathlineto{\pgfqpoint{1.026321in}{1.782964in}}%
\pgfpathlineto{\pgfqpoint{1.027693in}{1.784056in}}%
\pgfpathlineto{\pgfqpoint{1.028783in}{1.785568in}}%
\pgfpathlineto{\pgfqpoint{1.030192in}{1.786660in}}%
\pgfpathlineto{\pgfqpoint{1.031187in}{1.787836in}}%
\pgfpathlineto{\pgfqpoint{1.032503in}{1.788928in}}%
\pgfpathlineto{\pgfqpoint{1.033536in}{1.790524in}}%
\pgfpathlineto{\pgfqpoint{1.034550in}{1.791532in}}%
\pgfpathlineto{\pgfqpoint{1.035640in}{1.792708in}}%
\pgfpathlineto{\pgfqpoint{1.036392in}{1.793800in}}%
\pgfpathlineto{\pgfqpoint{1.037181in}{1.795648in}}%
\pgfpathlineto{\pgfqpoint{1.038364in}{1.796740in}}%
\pgfpathlineto{\pgfqpoint{1.039473in}{1.798672in}}%
\pgfpathlineto{\pgfqpoint{1.040938in}{1.799680in}}%
\pgfpathlineto{\pgfqpoint{1.042028in}{1.800856in}}%
\pgfpathlineto{\pgfqpoint{1.043099in}{1.801948in}}%
\pgfpathlineto{\pgfqpoint{1.044207in}{1.803208in}}%
\pgfpathlineto{\pgfqpoint{1.045466in}{1.804300in}}%
\pgfpathlineto{\pgfqpoint{1.046575in}{1.804720in}}%
\pgfpathlineto{\pgfqpoint{1.047289in}{1.805812in}}%
\pgfpathlineto{\pgfqpoint{1.048228in}{1.806904in}}%
\pgfpathlineto{\pgfqpoint{1.049299in}{1.807996in}}%
\pgfpathlineto{\pgfqpoint{1.050389in}{1.809088in}}%
\pgfpathlineto{\pgfqpoint{1.051347in}{1.810180in}}%
\pgfpathlineto{\pgfqpoint{1.052436in}{1.811608in}}%
\pgfpathlineto{\pgfqpoint{1.052944in}{1.812700in}}%
\pgfpathlineto{\pgfqpoint{1.053770in}{1.813540in}}%
\pgfpathlineto{\pgfqpoint{1.054729in}{1.814632in}}%
\pgfpathlineto{\pgfqpoint{1.055724in}{1.815976in}}%
\pgfpathlineto{\pgfqpoint{1.056833in}{1.817068in}}%
\pgfpathlineto{\pgfqpoint{1.057923in}{1.818916in}}%
\pgfpathlineto{\pgfqpoint{1.058806in}{1.820008in}}%
\pgfpathlineto{\pgfqpoint{1.059914in}{1.821520in}}%
\pgfpathlineto{\pgfqpoint{1.060872in}{1.822612in}}%
\pgfpathlineto{\pgfqpoint{1.061943in}{1.823788in}}%
\pgfpathlineto{\pgfqpoint{1.063258in}{1.824880in}}%
\pgfpathlineto{\pgfqpoint{1.064198in}{1.825972in}}%
\pgfpathlineto{\pgfqpoint{1.065024in}{1.827064in}}%
\pgfpathlineto{\pgfqpoint{1.066001in}{1.828408in}}%
\pgfpathlineto{\pgfqpoint{1.067335in}{1.829500in}}%
\pgfpathlineto{\pgfqpoint{1.068406in}{1.830760in}}%
\pgfpathlineto{\pgfqpoint{1.069533in}{1.831852in}}%
\pgfpathlineto{\pgfqpoint{1.070360in}{1.833028in}}%
\pgfpathlineto{\pgfqpoint{1.071262in}{1.834120in}}%
\pgfpathlineto{\pgfqpoint{1.072183in}{1.834960in}}%
\pgfpathlineto{\pgfqpoint{1.073892in}{1.836052in}}%
\pgfpathlineto{\pgfqpoint{1.075001in}{1.837144in}}%
\pgfpathlineto{\pgfqpoint{1.075978in}{1.838152in}}%
\pgfpathlineto{\pgfqpoint{1.077030in}{1.839160in}}%
\pgfpathlineto{\pgfqpoint{1.078383in}{1.840168in}}%
\pgfpathlineto{\pgfqpoint{1.079040in}{1.840924in}}%
\pgfpathlineto{\pgfqpoint{1.080844in}{1.841932in}}%
\pgfpathlineto{\pgfqpoint{1.081877in}{1.842688in}}%
\pgfpathlineto{\pgfqpoint{1.082741in}{1.843696in}}%
\pgfpathlineto{\pgfqpoint{1.083850in}{1.845292in}}%
\pgfpathlineto{\pgfqpoint{1.084921in}{1.846300in}}%
\pgfpathlineto{\pgfqpoint{1.086029in}{1.847812in}}%
\pgfpathlineto{\pgfqpoint{1.087382in}{1.848904in}}%
\pgfpathlineto{\pgfqpoint{1.088265in}{1.849912in}}%
\pgfpathlineto{\pgfqpoint{1.090238in}{1.851004in}}%
\pgfpathlineto{\pgfqpoint{1.091233in}{1.851844in}}%
\pgfpathlineto{\pgfqpoint{1.092680in}{1.852768in}}%
\pgfpathlineto{\pgfqpoint{1.093582in}{1.853440in}}%
\pgfpathlineto{\pgfqpoint{1.094747in}{1.854532in}}%
\pgfpathlineto{\pgfqpoint{1.095836in}{1.855372in}}%
\pgfpathlineto{\pgfqpoint{1.097133in}{1.856464in}}%
\pgfpathlineto{\pgfqpoint{1.098241in}{1.858312in}}%
\pgfpathlineto{\pgfqpoint{1.100045in}{1.859404in}}%
\pgfpathlineto{\pgfqpoint{1.100853in}{1.860580in}}%
\pgfpathlineto{\pgfqpoint{1.102281in}{1.861588in}}%
\pgfpathlineto{\pgfqpoint{1.103295in}{1.863268in}}%
\pgfpathlineto{\pgfqpoint{1.105155in}{1.864360in}}%
\pgfpathlineto{\pgfqpoint{1.106113in}{1.865284in}}%
\pgfpathlineto{\pgfqpoint{1.107654in}{1.866376in}}%
\pgfpathlineto{\pgfqpoint{1.108744in}{1.867552in}}%
\pgfpathlineto{\pgfqpoint{1.110284in}{1.868644in}}%
\pgfpathlineto{\pgfqpoint{1.111393in}{1.870072in}}%
\pgfpathlineto{\pgfqpoint{1.112746in}{1.871164in}}%
\pgfpathlineto{\pgfqpoint{1.113835in}{1.872340in}}%
\pgfpathlineto{\pgfqpoint{1.115226in}{1.873348in}}%
\pgfpathlineto{\pgfqpoint{1.116296in}{1.874608in}}%
\pgfpathlineto{\pgfqpoint{1.118513in}{1.875700in}}%
\pgfpathlineto{\pgfqpoint{1.119622in}{1.876960in}}%
\pgfpathlineto{\pgfqpoint{1.120937in}{1.877968in}}%
\pgfpathlineto{\pgfqpoint{1.121952in}{1.878556in}}%
\pgfpathlineto{\pgfqpoint{1.123643in}{1.879648in}}%
\pgfpathlineto{\pgfqpoint{1.124695in}{1.880740in}}%
\pgfpathlineto{\pgfqpoint{1.126066in}{1.881832in}}%
\pgfpathlineto{\pgfqpoint{1.127137in}{1.883260in}}%
\pgfpathlineto{\pgfqpoint{1.128678in}{1.884268in}}%
\pgfpathlineto{\pgfqpoint{1.129786in}{1.885612in}}%
\pgfpathlineto{\pgfqpoint{1.130876in}{1.886704in}}%
\pgfpathlineto{\pgfqpoint{1.131909in}{1.887628in}}%
\pgfpathlineto{\pgfqpoint{1.133844in}{1.888720in}}%
\pgfpathlineto{\pgfqpoint{1.134878in}{1.889644in}}%
\pgfpathlineto{\pgfqpoint{1.136850in}{1.890736in}}%
\pgfpathlineto{\pgfqpoint{1.137903in}{1.891576in}}%
\pgfpathlineto{\pgfqpoint{1.139349in}{1.892668in}}%
\pgfpathlineto{\pgfqpoint{1.140458in}{1.893676in}}%
\pgfpathlineto{\pgfqpoint{1.142449in}{1.894768in}}%
\pgfpathlineto{\pgfqpoint{1.143520in}{1.896448in}}%
\pgfpathlineto{\pgfqpoint{1.145549in}{1.897540in}}%
\pgfpathlineto{\pgfqpoint{1.146601in}{1.898380in}}%
\pgfpathlineto{\pgfqpoint{1.148273in}{1.899472in}}%
\pgfpathlineto{\pgfqpoint{1.149363in}{1.900144in}}%
\pgfpathlineto{\pgfqpoint{1.150472in}{1.901236in}}%
\pgfpathlineto{\pgfqpoint{1.151505in}{1.901824in}}%
\pgfpathlineto{\pgfqpoint{1.153064in}{1.902916in}}%
\pgfpathlineto{\pgfqpoint{1.154060in}{1.903924in}}%
\pgfpathlineto{\pgfqpoint{1.155826in}{1.904932in}}%
\pgfpathlineto{\pgfqpoint{1.156916in}{1.905940in}}%
\pgfpathlineto{\pgfqpoint{1.158043in}{1.907032in}}%
\pgfpathlineto{\pgfqpoint{1.159133in}{1.908124in}}%
\pgfpathlineto{\pgfqpoint{1.161030in}{1.909216in}}%
\pgfpathlineto{\pgfqpoint{1.162007in}{1.910056in}}%
\pgfpathlineto{\pgfqpoint{1.163642in}{1.911148in}}%
\pgfpathlineto{\pgfqpoint{1.164713in}{1.911988in}}%
\pgfpathlineto{\pgfqpoint{1.166873in}{1.913080in}}%
\pgfpathlineto{\pgfqpoint{1.167926in}{1.914004in}}%
\pgfpathlineto{\pgfqpoint{1.170086in}{1.915096in}}%
\pgfpathlineto{\pgfqpoint{1.171138in}{1.916020in}}%
\pgfpathlineto{\pgfqpoint{1.173017in}{1.917112in}}%
\pgfpathlineto{\pgfqpoint{1.173881in}{1.917868in}}%
\pgfpathlineto{\pgfqpoint{1.175816in}{1.918960in}}%
\pgfpathlineto{\pgfqpoint{1.176831in}{1.919716in}}%
\pgfpathlineto{\pgfqpoint{1.178935in}{1.920808in}}%
\pgfpathlineto{\pgfqpoint{1.179912in}{1.922236in}}%
\pgfpathlineto{\pgfqpoint{1.180889in}{1.923328in}}%
\pgfpathlineto{\pgfqpoint{1.181735in}{1.924252in}}%
\pgfpathlineto{\pgfqpoint{1.184759in}{1.925344in}}%
\pgfpathlineto{\pgfqpoint{1.185718in}{1.926604in}}%
\pgfpathlineto{\pgfqpoint{1.187634in}{1.927696in}}%
\pgfpathlineto{\pgfqpoint{1.188649in}{1.928536in}}%
\pgfpathlineto{\pgfqpoint{1.190509in}{1.929628in}}%
\pgfpathlineto{\pgfqpoint{1.191429in}{1.930300in}}%
\pgfpathlineto{\pgfqpoint{1.193759in}{1.931392in}}%
\pgfpathlineto{\pgfqpoint{1.194867in}{1.932064in}}%
\pgfpathlineto{\pgfqpoint{1.196295in}{1.933156in}}%
\pgfpathlineto{\pgfqpoint{1.197404in}{1.934164in}}%
\pgfpathlineto{\pgfqpoint{1.199226in}{1.935256in}}%
\pgfpathlineto{\pgfqpoint{1.200241in}{1.935676in}}%
\pgfpathlineto{\pgfqpoint{1.202608in}{1.936768in}}%
\pgfpathlineto{\pgfqpoint{1.203435in}{1.937944in}}%
\pgfpathlineto{\pgfqpoint{1.205915in}{1.939036in}}%
\pgfpathlineto{\pgfqpoint{1.206910in}{1.939456in}}%
\pgfpathlineto{\pgfqpoint{1.209015in}{1.940548in}}%
\pgfpathlineto{\pgfqpoint{1.209484in}{1.940968in}}%
\pgfpathlineto{\pgfqpoint{1.212096in}{1.942060in}}%
\pgfpathlineto{\pgfqpoint{1.212960in}{1.942396in}}%
\pgfpathlineto{\pgfqpoint{1.215985in}{1.943488in}}%
\pgfpathlineto{\pgfqpoint{1.217075in}{1.944076in}}%
\pgfpathlineto{\pgfqpoint{1.218822in}{1.945084in}}%
\pgfpathlineto{\pgfqpoint{1.219912in}{1.946512in}}%
\pgfpathlineto{\pgfqpoint{1.222035in}{1.947604in}}%
\pgfpathlineto{\pgfqpoint{1.222655in}{1.948276in}}%
\pgfpathlineto{\pgfqpoint{1.224721in}{1.949368in}}%
\pgfpathlineto{\pgfqpoint{1.225473in}{1.949788in}}%
\pgfpathlineto{\pgfqpoint{1.227596in}{1.950880in}}%
\pgfpathlineto{\pgfqpoint{1.228704in}{1.951384in}}%
\pgfpathlineto{\pgfqpoint{1.230903in}{1.952476in}}%
\pgfpathlineto{\pgfqpoint{1.231955in}{1.953316in}}%
\pgfpathlineto{\pgfqpoint{1.233589in}{1.954240in}}%
\pgfpathlineto{\pgfqpoint{1.234585in}{1.955164in}}%
\pgfpathlineto{\pgfqpoint{1.235844in}{1.956256in}}%
\pgfpathlineto{\pgfqpoint{1.236595in}{1.956844in}}%
\pgfpathlineto{\pgfqpoint{1.237854in}{1.957852in}}%
\pgfpathlineto{\pgfqpoint{1.238963in}{1.958944in}}%
\pgfpathlineto{\pgfqpoint{1.241461in}{1.960036in}}%
\pgfpathlineto{\pgfqpoint{1.242513in}{1.960960in}}%
\pgfpathlineto{\pgfqpoint{1.244937in}{1.962052in}}%
\pgfpathlineto{\pgfqpoint{1.245482in}{1.962472in}}%
\pgfpathlineto{\pgfqpoint{1.249108in}{1.963564in}}%
\pgfpathlineto{\pgfqpoint{1.250123in}{1.964656in}}%
\pgfpathlineto{\pgfqpoint{1.251607in}{1.965748in}}%
\pgfpathlineto{\pgfqpoint{1.252415in}{1.966252in}}%
\pgfpathlineto{\pgfqpoint{1.254275in}{1.967344in}}%
\pgfpathlineto{\pgfqpoint{1.255346in}{1.968016in}}%
\pgfpathlineto{\pgfqpoint{1.256830in}{1.969024in}}%
\pgfpathlineto{\pgfqpoint{1.257938in}{1.969948in}}%
\pgfpathlineto{\pgfqpoint{1.259798in}{1.971040in}}%
\pgfpathlineto{\pgfqpoint{1.260775in}{1.971628in}}%
\pgfpathlineto{\pgfqpoint{1.263086in}{1.972720in}}%
\pgfpathlineto{\pgfqpoint{1.264063in}{1.973392in}}%
\pgfpathlineto{\pgfqpoint{1.266318in}{1.974484in}}%
\pgfpathlineto{\pgfqpoint{1.267351in}{1.975072in}}%
\pgfpathlineto{\pgfqpoint{1.270132in}{1.976164in}}%
\pgfpathlineto{\pgfqpoint{1.271127in}{1.976668in}}%
\pgfpathlineto{\pgfqpoint{1.272649in}{1.977760in}}%
\pgfpathlineto{\pgfqpoint{1.273457in}{1.978348in}}%
\pgfpathlineto{\pgfqpoint{1.276087in}{1.979440in}}%
\pgfpathlineto{\pgfqpoint{1.276970in}{1.980112in}}%
\pgfpathlineto{\pgfqpoint{1.279300in}{1.981204in}}%
\pgfpathlineto{\pgfqpoint{1.280239in}{1.982044in}}%
\pgfpathlineto{\pgfqpoint{1.282513in}{1.983136in}}%
\pgfpathlineto{\pgfqpoint{1.283302in}{1.983304in}}%
\pgfpathlineto{\pgfqpoint{1.285575in}{1.984396in}}%
\pgfpathlineto{\pgfqpoint{1.286515in}{1.985068in}}%
\pgfpathlineto{\pgfqpoint{1.288149in}{1.986160in}}%
\pgfpathlineto{\pgfqpoint{1.289220in}{1.986916in}}%
\pgfpathlineto{\pgfqpoint{1.291982in}{1.987924in}}%
\pgfpathlineto{\pgfqpoint{1.292564in}{1.988596in}}%
\pgfpathlineto{\pgfqpoint{1.294950in}{1.989604in}}%
\pgfpathlineto{\pgfqpoint{1.295646in}{1.989856in}}%
\pgfpathlineto{\pgfqpoint{1.298069in}{1.990948in}}%
\pgfpathlineto{\pgfqpoint{1.298952in}{1.991620in}}%
\pgfpathlineto{\pgfqpoint{1.301075in}{1.992628in}}%
\pgfpathlineto{\pgfqpoint{1.302033in}{1.993468in}}%
\pgfpathlineto{\pgfqpoint{1.303706in}{1.994560in}}%
\pgfpathlineto{\pgfqpoint{1.304401in}{1.994980in}}%
\pgfpathlineto{\pgfqpoint{1.306129in}{1.996072in}}%
\pgfpathlineto{\pgfqpoint{1.307238in}{1.997080in}}%
\pgfpathlineto{\pgfqpoint{1.308778in}{1.998172in}}%
\pgfpathlineto{\pgfqpoint{1.309699in}{1.998592in}}%
\pgfpathlineto{\pgfqpoint{1.311747in}{1.999600in}}%
\pgfpathlineto{\pgfqpoint{1.312592in}{2.000272in}}%
\pgfpathlineto{\pgfqpoint{1.315072in}{2.001364in}}%
\pgfpathlineto{\pgfqpoint{1.315955in}{2.001784in}}%
\pgfpathlineto{\pgfqpoint{1.318003in}{2.002876in}}%
\pgfpathlineto{\pgfqpoint{1.318623in}{2.003296in}}%
\pgfpathlineto{\pgfqpoint{1.320990in}{2.004388in}}%
\pgfpathlineto{\pgfqpoint{1.322024in}{2.004808in}}%
\pgfpathlineto{\pgfqpoint{1.323433in}{2.005900in}}%
\pgfpathlineto{\pgfqpoint{1.324391in}{2.006488in}}%
\pgfpathlineto{\pgfqpoint{1.328299in}{2.007580in}}%
\pgfpathlineto{\pgfqpoint{1.329201in}{2.007916in}}%
\pgfpathlineto{\pgfqpoint{1.331286in}{2.009008in}}%
\pgfpathlineto{\pgfqpoint{1.332338in}{2.009428in}}%
\pgfpathlineto{\pgfqpoint{1.334330in}{2.010520in}}%
\pgfpathlineto{\pgfqpoint{1.335382in}{2.011192in}}%
\pgfpathlineto{\pgfqpoint{1.336735in}{2.012284in}}%
\pgfpathlineto{\pgfqpoint{1.337599in}{2.012956in}}%
\pgfpathlineto{\pgfqpoint{1.339590in}{2.014048in}}%
\pgfpathlineto{\pgfqpoint{1.340549in}{2.014720in}}%
\pgfpathlineto{\pgfqpoint{1.343160in}{2.015728in}}%
\pgfpathlineto{\pgfqpoint{1.344062in}{2.016148in}}%
\pgfpathlineto{\pgfqpoint{1.346035in}{2.017240in}}%
\pgfpathlineto{\pgfqpoint{1.346974in}{2.017660in}}%
\pgfpathlineto{\pgfqpoint{1.350337in}{2.018752in}}%
\pgfpathlineto{\pgfqpoint{1.350901in}{2.019172in}}%
\pgfpathlineto{\pgfqpoint{1.355598in}{2.020264in}}%
\pgfpathlineto{\pgfqpoint{1.356499in}{2.020768in}}%
\pgfpathlineto{\pgfqpoint{1.358716in}{2.021860in}}%
\pgfpathlineto{\pgfqpoint{1.359618in}{2.022448in}}%
\pgfpathlineto{\pgfqpoint{1.363583in}{2.023540in}}%
\pgfpathlineto{\pgfqpoint{1.364635in}{2.024044in}}%
\pgfpathlineto{\pgfqpoint{1.366720in}{2.025136in}}%
\pgfpathlineto{\pgfqpoint{1.367810in}{2.025892in}}%
\pgfpathlineto{\pgfqpoint{1.372619in}{2.026984in}}%
\pgfpathlineto{\pgfqpoint{1.373672in}{2.027404in}}%
\pgfpathlineto{\pgfqpoint{1.376471in}{2.028412in}}%
\pgfpathlineto{\pgfqpoint{1.377579in}{2.029168in}}%
\pgfpathlineto{\pgfqpoint{1.380473in}{2.030260in}}%
\pgfpathlineto{\pgfqpoint{1.381243in}{2.030596in}}%
\pgfpathlineto{\pgfqpoint{1.383892in}{2.031688in}}%
\pgfpathlineto{\pgfqpoint{1.384080in}{2.031856in}}%
\pgfpathlineto{\pgfqpoint{1.389942in}{2.032948in}}%
\pgfpathlineto{\pgfqpoint{1.390637in}{2.033536in}}%
\pgfpathlineto{\pgfqpoint{1.395334in}{2.034628in}}%
\pgfpathlineto{\pgfqpoint{1.396349in}{2.034796in}}%
\pgfpathlineto{\pgfqpoint{1.398603in}{2.035804in}}%
\pgfpathlineto{\pgfqpoint{1.399561in}{2.036140in}}%
\pgfpathlineto{\pgfqpoint{1.402793in}{2.037232in}}%
\pgfpathlineto{\pgfqpoint{1.403901in}{2.037652in}}%
\pgfpathlineto{\pgfqpoint{1.406137in}{2.038744in}}%
\pgfpathlineto{\pgfqpoint{1.406983in}{2.039164in}}%
\pgfpathlineto{\pgfqpoint{1.409218in}{2.040256in}}%
\pgfpathlineto{\pgfqpoint{1.410233in}{2.040844in}}%
\pgfpathlineto{\pgfqpoint{1.412450in}{2.041852in}}%
\pgfpathlineto{\pgfqpoint{1.413464in}{2.042692in}}%
\pgfpathlineto{\pgfqpoint{1.418086in}{2.043784in}}%
\pgfpathlineto{\pgfqpoint{1.419119in}{2.044120in}}%
\pgfpathlineto{\pgfqpoint{1.424023in}{2.045212in}}%
\pgfpathlineto{\pgfqpoint{1.424887in}{2.045716in}}%
\pgfpathlineto{\pgfqpoint{1.428701in}{2.046808in}}%
\pgfpathlineto{\pgfqpoint{1.429810in}{2.047564in}}%
\pgfpathlineto{\pgfqpoint{1.432797in}{2.048656in}}%
\pgfpathlineto{\pgfqpoint{1.433586in}{2.049160in}}%
\pgfpathlineto{\pgfqpoint{1.435390in}{2.050252in}}%
\pgfpathlineto{\pgfqpoint{1.435859in}{2.050588in}}%
\pgfpathlineto{\pgfqpoint{1.438321in}{2.051680in}}%
\pgfpathlineto{\pgfqpoint{1.439335in}{2.052268in}}%
\pgfpathlineto{\pgfqpoint{1.443487in}{2.053360in}}%
\pgfpathlineto{\pgfqpoint{1.444558in}{2.054116in}}%
\pgfpathlineto{\pgfqpoint{1.449988in}{2.055208in}}%
\pgfpathlineto{\pgfqpoint{1.450383in}{2.055628in}}%
\pgfpathlineto{\pgfqpoint{1.454309in}{2.056720in}}%
\pgfpathlineto{\pgfqpoint{1.455418in}{2.057308in}}%
\pgfpathlineto{\pgfqpoint{1.458236in}{2.058400in}}%
\pgfpathlineto{\pgfqpoint{1.459307in}{2.058904in}}%
\pgfpathlineto{\pgfqpoint{1.462764in}{2.059996in}}%
\pgfpathlineto{\pgfqpoint{1.463872in}{2.060920in}}%
\pgfpathlineto{\pgfqpoint{1.467630in}{2.062012in}}%
\pgfpathlineto{\pgfqpoint{1.468569in}{2.062432in}}%
\pgfpathlineto{\pgfqpoint{1.471406in}{2.063524in}}%
\pgfpathlineto{\pgfqpoint{1.472458in}{2.064028in}}%
\pgfpathlineto{\pgfqpoint{1.476460in}{2.065120in}}%
\pgfpathlineto{\pgfqpoint{1.477362in}{2.065456in}}%
\pgfpathlineto{\pgfqpoint{1.479729in}{2.066548in}}%
\pgfpathlineto{\pgfqpoint{1.480800in}{2.067388in}}%
\pgfpathlineto{\pgfqpoint{1.484144in}{2.068480in}}%
\pgfpathlineto{\pgfqpoint{1.485009in}{2.068816in}}%
\pgfpathlineto{\pgfqpoint{1.489649in}{2.069908in}}%
\pgfpathlineto{\pgfqpoint{1.490701in}{2.070328in}}%
\pgfpathlineto{\pgfqpoint{1.494666in}{2.071420in}}%
\pgfpathlineto{\pgfqpoint{1.495455in}{2.071588in}}%
\pgfpathlineto{\pgfqpoint{1.501467in}{2.072680in}}%
\pgfpathlineto{\pgfqpoint{1.502387in}{2.073268in}}%
\pgfpathlineto{\pgfqpoint{1.505168in}{2.074360in}}%
\pgfpathlineto{\pgfqpoint{1.505844in}{2.074780in}}%
\pgfpathlineto{\pgfqpoint{1.511330in}{2.075872in}}%
\pgfpathlineto{\pgfqpoint{1.512383in}{2.076292in}}%
\pgfpathlineto{\pgfqpoint{1.515821in}{2.077384in}}%
\pgfpathlineto{\pgfqpoint{1.516723in}{2.077636in}}%
\pgfpathlineto{\pgfqpoint{1.521250in}{2.078728in}}%
\pgfpathlineto{\pgfqpoint{1.521250in}{2.078812in}}%
\pgfpathlineto{\pgfqpoint{1.526210in}{2.079904in}}%
\pgfpathlineto{\pgfqpoint{1.527281in}{2.080408in}}%
\pgfpathlineto{\pgfqpoint{1.530626in}{2.081500in}}%
\pgfpathlineto{\pgfqpoint{1.531640in}{2.081920in}}%
\pgfpathlineto{\pgfqpoint{1.534984in}{2.083012in}}%
\pgfpathlineto{\pgfqpoint{1.535999in}{2.083432in}}%
\pgfpathlineto{\pgfqpoint{1.538967in}{2.084524in}}%
\pgfpathlineto{\pgfqpoint{1.539888in}{2.084944in}}%
\pgfpathlineto{\pgfqpoint{1.544435in}{2.086036in}}%
\pgfpathlineto{\pgfqpoint{1.545130in}{2.086456in}}%
\pgfpathlineto{\pgfqpoint{1.548399in}{2.087548in}}%
\pgfpathlineto{\pgfqpoint{1.548399in}{2.087632in}}%
\pgfpathlineto{\pgfqpoint{1.553378in}{2.088724in}}%
\pgfpathlineto{\pgfqpoint{1.554467in}{2.089228in}}%
\pgfpathlineto{\pgfqpoint{1.559503in}{2.090320in}}%
\pgfpathlineto{\pgfqpoint{1.560461in}{2.090488in}}%
\pgfpathlineto{\pgfqpoint{1.563598in}{2.091580in}}%
\pgfpathlineto{\pgfqpoint{1.564275in}{2.091916in}}%
\pgfpathlineto{\pgfqpoint{1.569817in}{2.093008in}}%
\pgfpathlineto{\pgfqpoint{1.570756in}{2.093764in}}%
\pgfpathlineto{\pgfqpoint{1.574552in}{2.094856in}}%
\pgfpathlineto{\pgfqpoint{1.575040in}{2.095192in}}%
\pgfpathlineto{\pgfqpoint{1.580958in}{2.096284in}}%
\pgfpathlineto{\pgfqpoint{1.580958in}{2.096368in}}%
\pgfpathlineto{\pgfqpoint{1.588924in}{2.097460in}}%
\pgfpathlineto{\pgfqpoint{1.589394in}{2.097796in}}%
\pgfpathlineto{\pgfqpoint{1.594354in}{2.098888in}}%
\pgfpathlineto{\pgfqpoint{1.594673in}{2.099056in}}%
\pgfpathlineto{\pgfqpoint{1.601212in}{2.100148in}}%
\pgfpathlineto{\pgfqpoint{1.602019in}{2.100484in}}%
\pgfpathlineto{\pgfqpoint{1.606735in}{2.101576in}}%
\pgfpathlineto{\pgfqpoint{1.607581in}{2.102080in}}%
\pgfpathlineto{\pgfqpoint{1.613405in}{2.103172in}}%
\pgfpathlineto{\pgfqpoint{1.613856in}{2.103508in}}%
\pgfpathlineto{\pgfqpoint{1.618797in}{2.104600in}}%
\pgfpathlineto{\pgfqpoint{1.619473in}{2.105020in}}%
\pgfpathlineto{\pgfqpoint{1.629299in}{2.106112in}}%
\pgfpathlineto{\pgfqpoint{1.630164in}{2.106364in}}%
\pgfpathlineto{\pgfqpoint{1.633376in}{2.107456in}}%
\pgfpathlineto{\pgfqpoint{1.633921in}{2.107708in}}%
\pgfpathlineto{\pgfqpoint{1.639032in}{2.108800in}}%
\pgfpathlineto{\pgfqpoint{1.639520in}{2.108968in}}%
\pgfpathlineto{\pgfqpoint{1.648444in}{2.109976in}}%
\pgfpathlineto{\pgfqpoint{1.649403in}{2.110228in}}%
\pgfpathlineto{\pgfqpoint{1.655959in}{2.111320in}}%
\pgfpathlineto{\pgfqpoint{1.657049in}{2.111824in}}%
\pgfpathlineto{\pgfqpoint{1.662159in}{2.112916in}}%
\pgfpathlineto{\pgfqpoint{1.663155in}{2.113504in}}%
\pgfpathlineto{\pgfqpoint{1.669750in}{2.114596in}}%
\pgfpathlineto{\pgfqpoint{1.670802in}{2.114932in}}%
\pgfpathlineto{\pgfqpoint{1.678167in}{2.116024in}}%
\pgfpathlineto{\pgfqpoint{1.679106in}{2.116360in}}%
\pgfpathlineto{\pgfqpoint{1.686114in}{2.117452in}}%
\pgfpathlineto{\pgfqpoint{1.686959in}{2.117620in}}%
\pgfpathlineto{\pgfqpoint{1.690041in}{2.118712in}}%
\pgfpathlineto{\pgfqpoint{1.690999in}{2.118964in}}%
\pgfpathlineto{\pgfqpoint{1.697462in}{2.120056in}}%
\pgfpathlineto{\pgfqpoint{1.698007in}{2.120392in}}%
\pgfpathlineto{\pgfqpoint{1.706987in}{2.121484in}}%
\pgfpathlineto{\pgfqpoint{1.707927in}{2.121736in}}%
\pgfpathlineto{\pgfqpoint{1.712830in}{2.122828in}}%
\pgfpathlineto{\pgfqpoint{1.713338in}{2.123164in}}%
\pgfpathlineto{\pgfqpoint{1.718129in}{2.124256in}}%
\pgfpathlineto{\pgfqpoint{1.718129in}{2.124340in}}%
\pgfpathlineto{\pgfqpoint{1.724084in}{2.125432in}}%
\pgfpathlineto{\pgfqpoint{1.725024in}{2.125600in}}%
\pgfpathlineto{\pgfqpoint{1.731675in}{2.126692in}}%
\pgfpathlineto{\pgfqpoint{1.732614in}{2.126944in}}%
\pgfpathlineto{\pgfqpoint{1.738363in}{2.128036in}}%
\pgfpathlineto{\pgfqpoint{1.739453in}{2.128372in}}%
\pgfpathlineto{\pgfqpoint{1.744544in}{2.129464in}}%
\pgfpathlineto{\pgfqpoint{1.744544in}{2.129548in}}%
\pgfpathlineto{\pgfqpoint{1.754840in}{2.130640in}}%
\pgfpathlineto{\pgfqpoint{1.755159in}{2.130808in}}%
\pgfpathlineto{\pgfqpoint{1.761979in}{2.131900in}}%
\pgfpathlineto{\pgfqpoint{1.762938in}{2.132152in}}%
\pgfpathlineto{\pgfqpoint{1.769006in}{2.133244in}}%
\pgfpathlineto{\pgfqpoint{1.769006in}{2.133328in}}%
\pgfpathlineto{\pgfqpoint{1.777310in}{2.134420in}}%
\pgfpathlineto{\pgfqpoint{1.778306in}{2.134588in}}%
\pgfpathlineto{\pgfqpoint{1.783679in}{2.135680in}}%
\pgfpathlineto{\pgfqpoint{1.784356in}{2.135848in}}%
\pgfpathlineto{\pgfqpoint{1.791646in}{2.136940in}}%
\pgfpathlineto{\pgfqpoint{1.792604in}{2.137276in}}%
\pgfpathlineto{\pgfqpoint{1.800025in}{2.138368in}}%
\pgfpathlineto{\pgfqpoint{1.800983in}{2.138788in}}%
\pgfpathlineto{\pgfqpoint{1.807484in}{2.139880in}}%
\pgfpathlineto{\pgfqpoint{1.808066in}{2.140300in}}%
\pgfpathlineto{\pgfqpoint{1.813515in}{2.141392in}}%
\pgfpathlineto{\pgfqpoint{1.814097in}{2.141644in}}%
\pgfpathlineto{\pgfqpoint{1.821011in}{2.142736in}}%
\pgfpathlineto{\pgfqpoint{1.822101in}{2.143072in}}%
\pgfpathlineto{\pgfqpoint{1.828338in}{2.144164in}}%
\pgfpathlineto{\pgfqpoint{1.829296in}{2.144836in}}%
\pgfpathlineto{\pgfqpoint{1.832321in}{2.145928in}}%
\pgfpathlineto{\pgfqpoint{1.832979in}{2.146264in}}%
\pgfpathlineto{\pgfqpoint{1.842730in}{2.147356in}}%
\pgfpathlineto{\pgfqpoint{1.842730in}{2.147440in}}%
\pgfpathlineto{\pgfqpoint{1.850639in}{2.148532in}}%
\pgfpathlineto{\pgfqpoint{1.851184in}{2.148952in}}%
\pgfpathlineto{\pgfqpoint{1.858718in}{2.150044in}}%
\pgfpathlineto{\pgfqpoint{1.858718in}{2.150128in}}%
\pgfpathlineto{\pgfqpoint{1.866572in}{2.151220in}}%
\pgfpathlineto{\pgfqpoint{1.867304in}{2.151472in}}%
\pgfpathlineto{\pgfqpoint{1.874068in}{2.152564in}}%
\pgfpathlineto{\pgfqpoint{1.874632in}{2.152900in}}%
\pgfpathlineto{\pgfqpoint{1.881301in}{2.153992in}}%
\pgfpathlineto{\pgfqpoint{1.882203in}{2.154412in}}%
\pgfpathlineto{\pgfqpoint{1.889887in}{2.155504in}}%
\pgfpathlineto{\pgfqpoint{1.890432in}{2.155924in}}%
\pgfpathlineto{\pgfqpoint{1.902419in}{2.157016in}}%
\pgfpathlineto{\pgfqpoint{1.903358in}{2.157268in}}%
\pgfpathlineto{\pgfqpoint{1.911268in}{2.158360in}}%
\pgfpathlineto{\pgfqpoint{1.912038in}{2.158696in}}%
\pgfpathlineto{\pgfqpoint{1.916641in}{2.159788in}}%
\pgfpathlineto{\pgfqpoint{1.917618in}{2.160124in}}%
\pgfpathlineto{\pgfqpoint{1.927970in}{2.161216in}}%
\pgfpathlineto{\pgfqpoint{1.929079in}{2.161468in}}%
\pgfpathlineto{\pgfqpoint{1.939206in}{2.162560in}}%
\pgfpathlineto{\pgfqpoint{1.940239in}{2.162980in}}%
\pgfpathlineto{\pgfqpoint{1.957843in}{2.164072in}}%
\pgfpathlineto{\pgfqpoint{1.958519in}{2.164324in}}%
\pgfpathlineto{\pgfqpoint{1.967575in}{2.165416in}}%
\pgfpathlineto{\pgfqpoint{1.967895in}{2.165584in}}%
\pgfpathlineto{\pgfqpoint{1.981103in}{2.166676in}}%
\pgfpathlineto{\pgfqpoint{1.981422in}{2.166844in}}%
\pgfpathlineto{\pgfqpoint{1.989069in}{2.167936in}}%
\pgfpathlineto{\pgfqpoint{1.990027in}{2.168272in}}%
\pgfpathlineto{\pgfqpoint{1.999045in}{2.169364in}}%
\pgfpathlineto{\pgfqpoint{2.000135in}{2.169784in}}%
\pgfpathlineto{\pgfqpoint{2.009886in}{2.170876in}}%
\pgfpathlineto{\pgfqpoint{2.010581in}{2.171212in}}%
\pgfpathlineto{\pgfqpoint{2.020332in}{2.172304in}}%
\pgfpathlineto{\pgfqpoint{2.020332in}{2.172388in}}%
\pgfpathlineto{\pgfqpoint{2.027959in}{2.173480in}}%
\pgfpathlineto{\pgfqpoint{2.028561in}{2.173648in}}%
\pgfpathlineto{\pgfqpoint{2.037466in}{2.174740in}}%
\pgfpathlineto{\pgfqpoint{2.037466in}{2.174824in}}%
\pgfpathlineto{\pgfqpoint{2.044887in}{2.175916in}}%
\pgfpathlineto{\pgfqpoint{2.045226in}{2.176168in}}%
\pgfpathlineto{\pgfqpoint{2.053962in}{2.177260in}}%
\pgfpathlineto{\pgfqpoint{2.054281in}{2.177428in}}%
\pgfpathlineto{\pgfqpoint{2.064972in}{2.178520in}}%
\pgfpathlineto{\pgfqpoint{2.065592in}{2.178688in}}%
\pgfpathlineto{\pgfqpoint{2.071040in}{2.179780in}}%
\pgfpathlineto{\pgfqpoint{2.072130in}{2.180200in}}%
\pgfpathlineto{\pgfqpoint{2.082294in}{2.181292in}}%
\pgfpathlineto{\pgfqpoint{2.082613in}{2.181544in}}%
\pgfpathlineto{\pgfqpoint{2.093247in}{2.182636in}}%
\pgfpathlineto{\pgfqpoint{2.093247in}{2.182720in}}%
\pgfpathlineto{\pgfqpoint{2.103618in}{2.183812in}}%
\pgfpathlineto{\pgfqpoint{2.104483in}{2.184064in}}%
\pgfpathlineto{\pgfqpoint{2.118536in}{2.185156in}}%
\pgfpathlineto{\pgfqpoint{2.119269in}{2.185324in}}%
\pgfpathlineto{\pgfqpoint{2.127253in}{2.186416in}}%
\pgfpathlineto{\pgfqpoint{2.128212in}{2.186752in}}%
\pgfpathlineto{\pgfqpoint{2.140536in}{2.187844in}}%
\pgfpathlineto{\pgfqpoint{2.141419in}{2.188012in}}%
\pgfpathlineto{\pgfqpoint{2.153012in}{2.189104in}}%
\pgfpathlineto{\pgfqpoint{2.153632in}{2.189440in}}%
\pgfpathlineto{\pgfqpoint{2.163307in}{2.190532in}}%
\pgfpathlineto{\pgfqpoint{2.164266in}{2.190700in}}%
\pgfpathlineto{\pgfqpoint{2.176496in}{2.191792in}}%
\pgfpathlineto{\pgfqpoint{2.176496in}{2.191876in}}%
\pgfpathlineto{\pgfqpoint{2.186980in}{2.192968in}}%
\pgfpathlineto{\pgfqpoint{2.187619in}{2.193136in}}%
\pgfpathlineto{\pgfqpoint{2.198441in}{2.194228in}}%
\pgfpathlineto{\pgfqpoint{2.199249in}{2.194396in}}%
\pgfpathlineto{\pgfqpoint{2.210540in}{2.195488in}}%
\pgfpathlineto{\pgfqpoint{2.211573in}{2.195740in}}%
\pgfpathlineto{\pgfqpoint{2.222715in}{2.196832in}}%
\pgfpathlineto{\pgfqpoint{2.222715in}{2.196916in}}%
\pgfpathlineto{\pgfqpoint{2.237275in}{2.198008in}}%
\pgfpathlineto{\pgfqpoint{2.238365in}{2.198176in}}%
\pgfpathlineto{\pgfqpoint{2.249055in}{2.199268in}}%
\pgfpathlineto{\pgfqpoint{2.249675in}{2.199520in}}%
\pgfpathlineto{\pgfqpoint{2.261944in}{2.200612in}}%
\pgfpathlineto{\pgfqpoint{2.262564in}{2.200864in}}%
\pgfpathlineto{\pgfqpoint{2.274475in}{2.201956in}}%
\pgfpathlineto{\pgfqpoint{2.274475in}{2.202040in}}%
\pgfpathlineto{\pgfqpoint{2.293752in}{2.203132in}}%
\pgfpathlineto{\pgfqpoint{2.293752in}{2.203216in}}%
\pgfpathlineto{\pgfqpoint{2.311337in}{2.204308in}}%
\pgfpathlineto{\pgfqpoint{2.311337in}{2.204392in}}%
\pgfpathlineto{\pgfqpoint{2.328979in}{2.205484in}}%
\pgfpathlineto{\pgfqpoint{2.329881in}{2.205652in}}%
\pgfpathlineto{\pgfqpoint{2.345212in}{2.206744in}}%
\pgfpathlineto{\pgfqpoint{2.345212in}{2.206828in}}%
\pgfpathlineto{\pgfqpoint{2.355939in}{2.207920in}}%
\pgfpathlineto{\pgfqpoint{2.355939in}{2.208004in}}%
\pgfpathlineto{\pgfqpoint{2.366198in}{2.209096in}}%
\pgfpathlineto{\pgfqpoint{2.367193in}{2.209432in}}%
\pgfpathlineto{\pgfqpoint{2.378259in}{2.210524in}}%
\pgfpathlineto{\pgfqpoint{2.378259in}{2.210608in}}%
\pgfpathlineto{\pgfqpoint{2.397367in}{2.211700in}}%
\pgfpathlineto{\pgfqpoint{2.397761in}{2.211868in}}%
\pgfpathlineto{\pgfqpoint{2.412529in}{2.212960in}}%
\pgfpathlineto{\pgfqpoint{2.413224in}{2.213380in}}%
\pgfpathlineto{\pgfqpoint{2.428216in}{2.214472in}}%
\pgfpathlineto{\pgfqpoint{2.428874in}{2.214640in}}%
\pgfpathlineto{\pgfqpoint{2.437779in}{2.215732in}}%
\pgfpathlineto{\pgfqpoint{2.437779in}{2.215816in}}%
\pgfpathlineto{\pgfqpoint{2.453505in}{2.216908in}}%
\pgfpathlineto{\pgfqpoint{2.453505in}{2.216992in}}%
\pgfpathlineto{\pgfqpoint{2.470771in}{2.218084in}}%
\pgfpathlineto{\pgfqpoint{2.471429in}{2.218336in}}%
\pgfpathlineto{\pgfqpoint{2.485557in}{2.219428in}}%
\pgfpathlineto{\pgfqpoint{2.485970in}{2.219596in}}%
\pgfpathlineto{\pgfqpoint{2.506055in}{2.220688in}}%
\pgfpathlineto{\pgfqpoint{2.506055in}{2.220772in}}%
\pgfpathlineto{\pgfqpoint{2.519413in}{2.221864in}}%
\pgfpathlineto{\pgfqpoint{2.519789in}{2.222032in}}%
\pgfpathlineto{\pgfqpoint{2.529464in}{2.223124in}}%
\pgfpathlineto{\pgfqpoint{2.529558in}{2.223292in}}%
\pgfpathlineto{\pgfqpoint{2.547557in}{2.224384in}}%
\pgfpathlineto{\pgfqpoint{2.548177in}{2.224552in}}%
\pgfpathlineto{\pgfqpoint{2.567472in}{2.225644in}}%
\pgfpathlineto{\pgfqpoint{2.567886in}{2.225812in}}%
\pgfpathlineto{\pgfqpoint{2.584532in}{2.226904in}}%
\pgfpathlineto{\pgfqpoint{2.585452in}{2.227072in}}%
\pgfpathlineto{\pgfqpoint{2.595616in}{2.228164in}}%
\pgfpathlineto{\pgfqpoint{2.596105in}{2.228332in}}%
\pgfpathlineto{\pgfqpoint{2.612018in}{2.229424in}}%
\pgfpathlineto{\pgfqpoint{2.612300in}{2.229592in}}%
\pgfpathlineto{\pgfqpoint{2.626767in}{2.230684in}}%
\pgfpathlineto{\pgfqpoint{2.626767in}{2.230768in}}%
\pgfpathlineto{\pgfqpoint{2.639881in}{2.231860in}}%
\pgfpathlineto{\pgfqpoint{2.640933in}{2.232028in}}%
\pgfpathlineto{\pgfqpoint{2.653821in}{2.233120in}}%
\pgfpathlineto{\pgfqpoint{2.654836in}{2.233288in}}%
\pgfpathlineto{\pgfqpoint{2.667292in}{2.234380in}}%
\pgfpathlineto{\pgfqpoint{2.668269in}{2.234716in}}%
\pgfpathlineto{\pgfqpoint{2.684934in}{2.235808in}}%
\pgfpathlineto{\pgfqpoint{2.685404in}{2.235976in}}%
\pgfpathlineto{\pgfqpoint{2.702163in}{2.237068in}}%
\pgfpathlineto{\pgfqpoint{2.702670in}{2.237320in}}%
\pgfpathlineto{\pgfqpoint{2.719278in}{2.238412in}}%
\pgfpathlineto{\pgfqpoint{2.720067in}{2.238580in}}%
\pgfpathlineto{\pgfqpoint{2.736169in}{2.239672in}}%
\pgfpathlineto{\pgfqpoint{2.737221in}{2.240092in}}%
\pgfpathlineto{\pgfqpoint{2.756685in}{2.241184in}}%
\pgfpathlineto{\pgfqpoint{2.756685in}{2.241352in}}%
\pgfpathlineto{\pgfqpoint{2.772166in}{2.242444in}}%
\pgfpathlineto{\pgfqpoint{2.773275in}{2.242780in}}%
\pgfpathlineto{\pgfqpoint{2.792551in}{2.243872in}}%
\pgfpathlineto{\pgfqpoint{2.792683in}{2.244040in}}%
\pgfpathlineto{\pgfqpoint{2.803655in}{2.245048in}}%
\pgfpathlineto{\pgfqpoint{2.803655in}{2.245216in}}%
\pgfpathlineto{\pgfqpoint{2.818741in}{2.246308in}}%
\pgfpathlineto{\pgfqpoint{2.818741in}{2.246392in}}%
\pgfpathlineto{\pgfqpoint{2.834730in}{2.247484in}}%
\pgfpathlineto{\pgfqpoint{2.835331in}{2.247820in}}%
\pgfpathlineto{\pgfqpoint{2.844875in}{2.248912in}}%
\pgfpathlineto{\pgfqpoint{2.844875in}{2.248996in}}%
\pgfpathlineto{\pgfqpoint{2.859868in}{2.250088in}}%
\pgfpathlineto{\pgfqpoint{2.860789in}{2.250508in}}%
\pgfpathlineto{\pgfqpoint{2.873996in}{2.251600in}}%
\pgfpathlineto{\pgfqpoint{2.875049in}{2.251768in}}%
\pgfpathlineto{\pgfqpoint{2.886021in}{2.252860in}}%
\pgfpathlineto{\pgfqpoint{2.886472in}{2.253028in}}%
\pgfpathlineto{\pgfqpoint{2.894118in}{2.254120in}}%
\pgfpathlineto{\pgfqpoint{2.894118in}{2.254204in}}%
\pgfpathlineto{\pgfqpoint{2.901126in}{2.255296in}}%
\pgfpathlineto{\pgfqpoint{2.902178in}{2.255800in}}%
\pgfpathlineto{\pgfqpoint{2.908547in}{2.256892in}}%
\pgfpathlineto{\pgfqpoint{2.909449in}{2.257396in}}%
\pgfpathlineto{\pgfqpoint{2.914992in}{2.258488in}}%
\pgfpathlineto{\pgfqpoint{2.916100in}{2.258824in}}%
\pgfpathlineto{\pgfqpoint{2.918806in}{2.259916in}}%
\pgfpathlineto{\pgfqpoint{2.919876in}{2.260840in}}%
\pgfpathlineto{\pgfqpoint{2.920628in}{2.261848in}}%
\pgfpathlineto{\pgfqpoint{2.920853in}{2.263444in}}%
\pgfpathlineto{\pgfqpoint{2.920853in}{2.263444in}}%
\pgfusepath{stroke}%
\end{pgfscope}%
\begin{pgfscope}%
\pgfsetrectcap%
\pgfsetmiterjoin%
\pgfsetlinewidth{0.803000pt}%
\definecolor{currentstroke}{rgb}{0.000000,0.000000,0.000000}%
\pgfsetstrokecolor{currentstroke}%
\pgfsetdash{}{0pt}%
\pgfpathmoveto{\pgfqpoint{0.553581in}{0.499444in}}%
\pgfpathlineto{\pgfqpoint{0.553581in}{2.347444in}}%
\pgfusepath{stroke}%
\end{pgfscope}%
\begin{pgfscope}%
\pgfsetrectcap%
\pgfsetmiterjoin%
\pgfsetlinewidth{0.803000pt}%
\definecolor{currentstroke}{rgb}{0.000000,0.000000,0.000000}%
\pgfsetstrokecolor{currentstroke}%
\pgfsetdash{}{0pt}%
\pgfpathmoveto{\pgfqpoint{3.033581in}{0.499444in}}%
\pgfpathlineto{\pgfqpoint{3.033581in}{2.347444in}}%
\pgfusepath{stroke}%
\end{pgfscope}%
\begin{pgfscope}%
\pgfsetrectcap%
\pgfsetmiterjoin%
\pgfsetlinewidth{0.803000pt}%
\definecolor{currentstroke}{rgb}{0.000000,0.000000,0.000000}%
\pgfsetstrokecolor{currentstroke}%
\pgfsetdash{}{0pt}%
\pgfpathmoveto{\pgfqpoint{0.553581in}{0.499444in}}%
\pgfpathlineto{\pgfqpoint{3.033581in}{0.499444in}}%
\pgfusepath{stroke}%
\end{pgfscope}%
\begin{pgfscope}%
\pgfsetrectcap%
\pgfsetmiterjoin%
\pgfsetlinewidth{0.803000pt}%
\definecolor{currentstroke}{rgb}{0.000000,0.000000,0.000000}%
\pgfsetstrokecolor{currentstroke}%
\pgfsetdash{}{0pt}%
\pgfpathmoveto{\pgfqpoint{0.553581in}{2.347444in}}%
\pgfpathlineto{\pgfqpoint{3.033581in}{2.347444in}}%
\pgfusepath{stroke}%
\end{pgfscope}%
\begin{pgfscope}%
\pgfsetbuttcap%
\pgfsetmiterjoin%
\definecolor{currentfill}{rgb}{1.000000,1.000000,1.000000}%
\pgfsetfillcolor{currentfill}%
\pgfsetlinewidth{1.003750pt}%
\definecolor{currentstroke}{rgb}{1.000000,1.000000,1.000000}%
\pgfsetstrokecolor{currentstroke}%
\pgfsetdash{}{0pt}%
\pgfpathmoveto{\pgfqpoint{1.738420in}{2.054860in}}%
\pgfpathlineto{\pgfqpoint{2.165920in}{2.054860in}}%
\pgfpathlineto{\pgfqpoint{2.165920in}{2.289305in}}%
\pgfpathlineto{\pgfqpoint{1.738420in}{2.289305in}}%
\pgfpathlineto{\pgfqpoint{1.738420in}{2.054860in}}%
\pgfpathclose%
\pgfusepath{stroke,fill}%
\end{pgfscope}%
\begin{pgfscope}%
\definecolor{textcolor}{rgb}{0.000000,0.000000,0.000000}%
\pgfsetstrokecolor{textcolor}%
\pgfsetfillcolor{textcolor}%
\pgftext[x=1.793975in,y=2.137360in,left,base]{\color{textcolor}\rmfamily\fontsize{10.000000}{12.000000}\selectfont 0.169}%
\end{pgfscope}%
\begin{pgfscope}%
\pgfsetbuttcap%
\pgfsetmiterjoin%
\definecolor{currentfill}{rgb}{1.000000,1.000000,1.000000}%
\pgfsetfillcolor{currentfill}%
\pgfsetlinewidth{1.003750pt}%
\definecolor{currentstroke}{rgb}{1.000000,1.000000,1.000000}%
\pgfsetstrokecolor{currentstroke}%
\pgfsetdash{}{0pt}%
\pgfpathmoveto{\pgfqpoint{0.771915in}{1.340944in}}%
\pgfpathlineto{\pgfqpoint{1.199415in}{1.340944in}}%
\pgfpathlineto{\pgfqpoint{1.199415in}{1.575389in}}%
\pgfpathlineto{\pgfqpoint{0.771915in}{1.575389in}}%
\pgfpathlineto{\pgfqpoint{0.771915in}{1.340944in}}%
\pgfpathclose%
\pgfusepath{stroke,fill}%
\end{pgfscope}%
\begin{pgfscope}%
\definecolor{textcolor}{rgb}{0.000000,0.000000,0.000000}%
\pgfsetstrokecolor{textcolor}%
\pgfsetfillcolor{textcolor}%
\pgftext[x=0.827470in,y=1.423444in,left,base]{\color{textcolor}\rmfamily\fontsize{10.000000}{12.000000}\selectfont 0.332}%
\end{pgfscope}%
\begin{pgfscope}%
\definecolor{textcolor}{rgb}{0.000000,0.000000,0.000000}%
\pgfsetstrokecolor{textcolor}%
\pgfsetfillcolor{textcolor}%
\pgftext[x=1.793581in,y=2.430778in,,base]{\color{textcolor}\rmfamily\fontsize{12.000000}{14.400000}\selectfont ROC Curve}%
\end{pgfscope}%
\begin{pgfscope}%
\pgfsetbuttcap%
\pgfsetmiterjoin%
\definecolor{currentfill}{rgb}{1.000000,1.000000,1.000000}%
\pgfsetfillcolor{currentfill}%
\pgfsetfillopacity{0.800000}%
\pgfsetlinewidth{1.003750pt}%
\definecolor{currentstroke}{rgb}{0.800000,0.800000,0.800000}%
\pgfsetstrokecolor{currentstroke}%
\pgfsetstrokeopacity{0.800000}%
\pgfsetdash{}{0pt}%
\pgfpathmoveto{\pgfqpoint{0.800942in}{0.568889in}}%
\pgfpathlineto{\pgfqpoint{2.936358in}{0.568889in}}%
\pgfpathquadraticcurveto{\pgfqpoint{2.964136in}{0.568889in}}{\pgfqpoint{2.964136in}{0.596666in}}%
\pgfpathlineto{\pgfqpoint{2.964136in}{0.791111in}}%
\pgfpathquadraticcurveto{\pgfqpoint{2.964136in}{0.818888in}}{\pgfqpoint{2.936358in}{0.818888in}}%
\pgfpathlineto{\pgfqpoint{0.800942in}{0.818888in}}%
\pgfpathquadraticcurveto{\pgfqpoint{0.773164in}{0.818888in}}{\pgfqpoint{0.773164in}{0.791111in}}%
\pgfpathlineto{\pgfqpoint{0.773164in}{0.596666in}}%
\pgfpathquadraticcurveto{\pgfqpoint{0.773164in}{0.568889in}}{\pgfqpoint{0.800942in}{0.568889in}}%
\pgfpathlineto{\pgfqpoint{0.800942in}{0.568889in}}%
\pgfpathclose%
\pgfusepath{stroke,fill}%
\end{pgfscope}%
\begin{pgfscope}%
\pgfsetrectcap%
\pgfsetroundjoin%
\pgfsetlinewidth{1.505625pt}%
\definecolor{currentstroke}{rgb}{0.121569,0.466667,0.705882}%
\pgfsetstrokecolor{currentstroke}%
\pgfsetdash{}{0pt}%
\pgfpathmoveto{\pgfqpoint{0.828720in}{0.707777in}}%
\pgfpathlineto{\pgfqpoint{0.967608in}{0.707777in}}%
\pgfpathlineto{\pgfqpoint{1.106497in}{0.707777in}}%
\pgfusepath{stroke}%
\end{pgfscope}%
\begin{pgfscope}%
\definecolor{textcolor}{rgb}{0.000000,0.000000,0.000000}%
\pgfsetstrokecolor{textcolor}%
\pgfsetfillcolor{textcolor}%
\pgftext[x=1.217608in,y=0.659166in,left,base]{\color{textcolor}\rmfamily\fontsize{10.000000}{12.000000}\selectfont Area Under Curve = 0.847)}%
\end{pgfscope}%
\end{pgfpicture}%
\makeatother%
\endgroup%

\end{tabular}
\end{center}

\begin{center}
\begin{tabular}{cc}
\begin{tabular}{cc|c|c|}
	&\multicolumn{1}{c}{}& \multicolumn{2}{c}{Prediction} \cr
	&\multicolumn{1}{c}{} & \multicolumn{1}{c}{N} & \multicolumn{1}{c}{P} \cr\cline{3-4}
	\multirow{2}{*}{Actual}&N & 85.7\% & 0.0\% \vrule width 0pt height 10pt depth 2pt \cr\cline{3-4}
	&P & 14.3\% & 0.0\% \vrule width 0pt height 10pt depth 2pt \cr\cline{3-4}
\end{tabular}
&
\begin{tabular}{ll}
0.857 & Accuracy \cr 
0.500 & Balanced Accuracy \cr 
0.000 & Precision \cr 
0.000 & Balanced Precision \cr 
0.000 & Recall \cr 
0.000 & F1 \cr 
0.000 & Balanced F1 \cr 
0.000 & Gmean \cr 
	\end{tabular}
\end{tabular}
\end{center}

%%%%%
Such a recommendation system (``Never send an ambulance'') would be useless, but note that the distribution still separates the negative and positive classes, just not at $p=0.5$.  We can fix that in two ways; the first is to shift the distribution to be centered at $p=0.5$.  By ``centered,'' we mean that the average of the medians of the negative and positive classes (the 0.107 and 0.293 on the ROC curve above) will now be 0.5.  Further research can explore whether centering the distribution at the $p=0.5$ threshold or another value of $p$ is most useful.  

\begin{center}
\begin{tabular}{p{0.5\textwidth} p{0.5\textwidth}}
  \vspace{0pt} %% Creator: Matplotlib, PGF backend
%%
%% To include the figure in your LaTeX document, write
%%   \input{<filename>.pgf}
%%
%% Make sure the required packages are loaded in your preamble
%%   \usepackage{pgf}
%%
%% Also ensure that all the required font packages are loaded; for instance,
%% the lmodern package is sometimes necessary when using math font.
%%   \usepackage{lmodern}
%%
%% Figures using additional raster images can only be included by \input if
%% they are in the same directory as the main LaTeX file. For loading figures
%% from other directories you can use the `import` package
%%   \usepackage{import}
%%
%% and then include the figures with
%%   \import{<path to file>}{<filename>.pgf}
%%
%% Matplotlib used the following preamble
%%   
%%   \usepackage{fontspec}
%%   \makeatletter\@ifpackageloaded{underscore}{}{\usepackage[strings]{underscore}}\makeatother
%%
\begingroup%
\makeatletter%
\begin{pgfpicture}%
\pgfpathrectangle{\pgfpointorigin}{\pgfqpoint{3.095000in}{3.243944in}}%
\pgfusepath{use as bounding box, clip}%
\begin{pgfscope}%
\pgfsetbuttcap%
\pgfsetmiterjoin%
\definecolor{currentfill}{rgb}{1.000000,1.000000,1.000000}%
\pgfsetfillcolor{currentfill}%
\pgfsetlinewidth{0.000000pt}%
\definecolor{currentstroke}{rgb}{1.000000,1.000000,1.000000}%
\pgfsetstrokecolor{currentstroke}%
\pgfsetdash{}{0pt}%
\pgfpathmoveto{\pgfqpoint{0.000000in}{0.000000in}}%
\pgfpathlineto{\pgfqpoint{3.095000in}{0.000000in}}%
\pgfpathlineto{\pgfqpoint{3.095000in}{3.243944in}}%
\pgfpathlineto{\pgfqpoint{0.000000in}{3.243944in}}%
\pgfpathlineto{\pgfqpoint{0.000000in}{0.000000in}}%
\pgfpathclose%
\pgfusepath{fill}%
\end{pgfscope}%
\begin{pgfscope}%
\pgfsetbuttcap%
\pgfsetmiterjoin%
\definecolor{currentfill}{rgb}{1.000000,1.000000,1.000000}%
\pgfsetfillcolor{currentfill}%
\pgfsetlinewidth{0.000000pt}%
\definecolor{currentstroke}{rgb}{0.000000,0.000000,0.000000}%
\pgfsetstrokecolor{currentstroke}%
\pgfsetstrokeopacity{0.000000}%
\pgfsetdash{}{0pt}%
\pgfpathmoveto{\pgfqpoint{0.515000in}{1.096944in}}%
\pgfpathlineto{\pgfqpoint{2.995000in}{1.096944in}}%
\pgfpathlineto{\pgfqpoint{2.995000in}{2.944944in}}%
\pgfpathlineto{\pgfqpoint{0.515000in}{2.944944in}}%
\pgfpathlineto{\pgfqpoint{0.515000in}{1.096944in}}%
\pgfpathclose%
\pgfusepath{fill}%
\end{pgfscope}%
\begin{pgfscope}%
\pgfpathrectangle{\pgfqpoint{0.515000in}{1.096944in}}{\pgfqpoint{2.480000in}{1.848000in}}%
\pgfusepath{clip}%
\pgfsetbuttcap%
\pgfsetmiterjoin%
\pgfsetlinewidth{1.003750pt}%
\definecolor{currentstroke}{rgb}{0.000000,0.000000,0.000000}%
\pgfsetstrokecolor{currentstroke}%
\pgfsetdash{}{0pt}%
\pgfpathmoveto{\pgfqpoint{0.505000in}{1.096944in}}%
\pgfpathlineto{\pgfqpoint{0.577627in}{1.096944in}}%
\pgfpathlineto{\pgfqpoint{0.577627in}{1.096944in}}%
\pgfpathlineto{\pgfqpoint{0.505000in}{1.096944in}}%
\pgfusepath{stroke}%
\end{pgfscope}%
\begin{pgfscope}%
\pgfpathrectangle{\pgfqpoint{0.515000in}{1.096944in}}{\pgfqpoint{2.480000in}{1.848000in}}%
\pgfusepath{clip}%
\pgfsetbuttcap%
\pgfsetmiterjoin%
\pgfsetlinewidth{1.003750pt}%
\definecolor{currentstroke}{rgb}{0.000000,0.000000,0.000000}%
\pgfsetstrokecolor{currentstroke}%
\pgfsetdash{}{0pt}%
\pgfpathmoveto{\pgfqpoint{0.727930in}{1.096944in}}%
\pgfpathlineto{\pgfqpoint{0.828132in}{1.096944in}}%
\pgfpathlineto{\pgfqpoint{0.828132in}{1.096944in}}%
\pgfpathlineto{\pgfqpoint{0.727930in}{1.096944in}}%
\pgfpathlineto{\pgfqpoint{0.727930in}{1.096944in}}%
\pgfpathclose%
\pgfusepath{stroke}%
\end{pgfscope}%
\begin{pgfscope}%
\pgfpathrectangle{\pgfqpoint{0.515000in}{1.096944in}}{\pgfqpoint{2.480000in}{1.848000in}}%
\pgfusepath{clip}%
\pgfsetbuttcap%
\pgfsetmiterjoin%
\pgfsetlinewidth{1.003750pt}%
\definecolor{currentstroke}{rgb}{0.000000,0.000000,0.000000}%
\pgfsetstrokecolor{currentstroke}%
\pgfsetdash{}{0pt}%
\pgfpathmoveto{\pgfqpoint{0.978435in}{1.096944in}}%
\pgfpathlineto{\pgfqpoint{1.078637in}{1.096944in}}%
\pgfpathlineto{\pgfqpoint{1.078637in}{1.237592in}}%
\pgfpathlineto{\pgfqpoint{0.978435in}{1.237592in}}%
\pgfpathlineto{\pgfqpoint{0.978435in}{1.096944in}}%
\pgfpathclose%
\pgfusepath{stroke}%
\end{pgfscope}%
\begin{pgfscope}%
\pgfpathrectangle{\pgfqpoint{0.515000in}{1.096944in}}{\pgfqpoint{2.480000in}{1.848000in}}%
\pgfusepath{clip}%
\pgfsetbuttcap%
\pgfsetmiterjoin%
\pgfsetlinewidth{1.003750pt}%
\definecolor{currentstroke}{rgb}{0.000000,0.000000,0.000000}%
\pgfsetstrokecolor{currentstroke}%
\pgfsetdash{}{0pt}%
\pgfpathmoveto{\pgfqpoint{1.228940in}{1.096944in}}%
\pgfpathlineto{\pgfqpoint{1.329142in}{1.096944in}}%
\pgfpathlineto{\pgfqpoint{1.329142in}{2.856944in}}%
\pgfpathlineto{\pgfqpoint{1.228940in}{2.856944in}}%
\pgfpathlineto{\pgfqpoint{1.228940in}{1.096944in}}%
\pgfpathclose%
\pgfusepath{stroke}%
\end{pgfscope}%
\begin{pgfscope}%
\pgfpathrectangle{\pgfqpoint{0.515000in}{1.096944in}}{\pgfqpoint{2.480000in}{1.848000in}}%
\pgfusepath{clip}%
\pgfsetbuttcap%
\pgfsetmiterjoin%
\pgfsetlinewidth{1.003750pt}%
\definecolor{currentstroke}{rgb}{0.000000,0.000000,0.000000}%
\pgfsetstrokecolor{currentstroke}%
\pgfsetdash{}{0pt}%
\pgfpathmoveto{\pgfqpoint{1.479445in}{1.096944in}}%
\pgfpathlineto{\pgfqpoint{1.579647in}{1.096944in}}%
\pgfpathlineto{\pgfqpoint{1.579647in}{2.772168in}}%
\pgfpathlineto{\pgfqpoint{1.479445in}{2.772168in}}%
\pgfpathlineto{\pgfqpoint{1.479445in}{1.096944in}}%
\pgfpathclose%
\pgfusepath{stroke}%
\end{pgfscope}%
\begin{pgfscope}%
\pgfpathrectangle{\pgfqpoint{0.515000in}{1.096944in}}{\pgfqpoint{2.480000in}{1.848000in}}%
\pgfusepath{clip}%
\pgfsetbuttcap%
\pgfsetmiterjoin%
\pgfsetlinewidth{1.003750pt}%
\definecolor{currentstroke}{rgb}{0.000000,0.000000,0.000000}%
\pgfsetstrokecolor{currentstroke}%
\pgfsetdash{}{0pt}%
\pgfpathmoveto{\pgfqpoint{1.729950in}{1.096944in}}%
\pgfpathlineto{\pgfqpoint{1.830152in}{1.096944in}}%
\pgfpathlineto{\pgfqpoint{1.830152in}{1.835669in}}%
\pgfpathlineto{\pgfqpoint{1.729950in}{1.835669in}}%
\pgfpathlineto{\pgfqpoint{1.729950in}{1.096944in}}%
\pgfpathclose%
\pgfusepath{stroke}%
\end{pgfscope}%
\begin{pgfscope}%
\pgfpathrectangle{\pgfqpoint{0.515000in}{1.096944in}}{\pgfqpoint{2.480000in}{1.848000in}}%
\pgfusepath{clip}%
\pgfsetbuttcap%
\pgfsetmiterjoin%
\pgfsetlinewidth{1.003750pt}%
\definecolor{currentstroke}{rgb}{0.000000,0.000000,0.000000}%
\pgfsetstrokecolor{currentstroke}%
\pgfsetdash{}{0pt}%
\pgfpathmoveto{\pgfqpoint{1.980455in}{1.096944in}}%
\pgfpathlineto{\pgfqpoint{2.080657in}{1.096944in}}%
\pgfpathlineto{\pgfqpoint{2.080657in}{1.309666in}}%
\pgfpathlineto{\pgfqpoint{1.980455in}{1.309666in}}%
\pgfpathlineto{\pgfqpoint{1.980455in}{1.096944in}}%
\pgfpathclose%
\pgfusepath{stroke}%
\end{pgfscope}%
\begin{pgfscope}%
\pgfpathrectangle{\pgfqpoint{0.515000in}{1.096944in}}{\pgfqpoint{2.480000in}{1.848000in}}%
\pgfusepath{clip}%
\pgfsetbuttcap%
\pgfsetmiterjoin%
\pgfsetlinewidth{1.003750pt}%
\definecolor{currentstroke}{rgb}{0.000000,0.000000,0.000000}%
\pgfsetstrokecolor{currentstroke}%
\pgfsetdash{}{0pt}%
\pgfpathmoveto{\pgfqpoint{2.230960in}{1.096944in}}%
\pgfpathlineto{\pgfqpoint{2.331162in}{1.096944in}}%
\pgfpathlineto{\pgfqpoint{2.331162in}{1.133646in}}%
\pgfpathlineto{\pgfqpoint{2.230960in}{1.133646in}}%
\pgfpathlineto{\pgfqpoint{2.230960in}{1.096944in}}%
\pgfpathclose%
\pgfusepath{stroke}%
\end{pgfscope}%
\begin{pgfscope}%
\pgfpathrectangle{\pgfqpoint{0.515000in}{1.096944in}}{\pgfqpoint{2.480000in}{1.848000in}}%
\pgfusepath{clip}%
\pgfsetbuttcap%
\pgfsetmiterjoin%
\pgfsetlinewidth{1.003750pt}%
\definecolor{currentstroke}{rgb}{0.000000,0.000000,0.000000}%
\pgfsetstrokecolor{currentstroke}%
\pgfsetdash{}{0pt}%
\pgfpathmoveto{\pgfqpoint{2.481465in}{1.096944in}}%
\pgfpathlineto{\pgfqpoint{2.581667in}{1.096944in}}%
\pgfpathlineto{\pgfqpoint{2.581667in}{1.096944in}}%
\pgfpathlineto{\pgfqpoint{2.481465in}{1.096944in}}%
\pgfpathlineto{\pgfqpoint{2.481465in}{1.096944in}}%
\pgfpathclose%
\pgfusepath{stroke}%
\end{pgfscope}%
\begin{pgfscope}%
\pgfpathrectangle{\pgfqpoint{0.515000in}{1.096944in}}{\pgfqpoint{2.480000in}{1.848000in}}%
\pgfusepath{clip}%
\pgfsetbuttcap%
\pgfsetmiterjoin%
\pgfsetlinewidth{1.003750pt}%
\definecolor{currentstroke}{rgb}{0.000000,0.000000,0.000000}%
\pgfsetstrokecolor{currentstroke}%
\pgfsetdash{}{0pt}%
\pgfpathmoveto{\pgfqpoint{2.731970in}{1.096944in}}%
\pgfpathlineto{\pgfqpoint{2.832172in}{1.096944in}}%
\pgfpathlineto{\pgfqpoint{2.832172in}{1.096944in}}%
\pgfpathlineto{\pgfqpoint{2.731970in}{1.096944in}}%
\pgfpathlineto{\pgfqpoint{2.731970in}{1.096944in}}%
\pgfpathclose%
\pgfusepath{stroke}%
\end{pgfscope}%
\begin{pgfscope}%
\pgfpathrectangle{\pgfqpoint{0.515000in}{1.096944in}}{\pgfqpoint{2.480000in}{1.848000in}}%
\pgfusepath{clip}%
\pgfsetbuttcap%
\pgfsetmiterjoin%
\definecolor{currentfill}{rgb}{0.000000,0.000000,0.000000}%
\pgfsetfillcolor{currentfill}%
\pgfsetlinewidth{0.000000pt}%
\definecolor{currentstroke}{rgb}{0.000000,0.000000,0.000000}%
\pgfsetstrokecolor{currentstroke}%
\pgfsetstrokeopacity{0.000000}%
\pgfsetdash{}{0pt}%
\pgfpathmoveto{\pgfqpoint{0.577627in}{1.096944in}}%
\pgfpathlineto{\pgfqpoint{0.677829in}{1.096944in}}%
\pgfpathlineto{\pgfqpoint{0.677829in}{1.096944in}}%
\pgfpathlineto{\pgfqpoint{0.577627in}{1.096944in}}%
\pgfpathlineto{\pgfqpoint{0.577627in}{1.096944in}}%
\pgfpathclose%
\pgfusepath{fill}%
\end{pgfscope}%
\begin{pgfscope}%
\pgfpathrectangle{\pgfqpoint{0.515000in}{1.096944in}}{\pgfqpoint{2.480000in}{1.848000in}}%
\pgfusepath{clip}%
\pgfsetbuttcap%
\pgfsetmiterjoin%
\definecolor{currentfill}{rgb}{0.000000,0.000000,0.000000}%
\pgfsetfillcolor{currentfill}%
\pgfsetlinewidth{0.000000pt}%
\definecolor{currentstroke}{rgb}{0.000000,0.000000,0.000000}%
\pgfsetstrokecolor{currentstroke}%
\pgfsetstrokeopacity{0.000000}%
\pgfsetdash{}{0pt}%
\pgfpathmoveto{\pgfqpoint{0.828132in}{1.096944in}}%
\pgfpathlineto{\pgfqpoint{0.928334in}{1.096944in}}%
\pgfpathlineto{\pgfqpoint{0.928334in}{1.096944in}}%
\pgfpathlineto{\pgfqpoint{0.828132in}{1.096944in}}%
\pgfpathlineto{\pgfqpoint{0.828132in}{1.096944in}}%
\pgfpathclose%
\pgfusepath{fill}%
\end{pgfscope}%
\begin{pgfscope}%
\pgfpathrectangle{\pgfqpoint{0.515000in}{1.096944in}}{\pgfqpoint{2.480000in}{1.848000in}}%
\pgfusepath{clip}%
\pgfsetbuttcap%
\pgfsetmiterjoin%
\definecolor{currentfill}{rgb}{0.000000,0.000000,0.000000}%
\pgfsetfillcolor{currentfill}%
\pgfsetlinewidth{0.000000pt}%
\definecolor{currentstroke}{rgb}{0.000000,0.000000,0.000000}%
\pgfsetstrokecolor{currentstroke}%
\pgfsetstrokeopacity{0.000000}%
\pgfsetdash{}{0pt}%
\pgfpathmoveto{\pgfqpoint{1.078637in}{1.096944in}}%
\pgfpathlineto{\pgfqpoint{1.178839in}{1.096944in}}%
\pgfpathlineto{\pgfqpoint{1.178839in}{1.103220in}}%
\pgfpathlineto{\pgfqpoint{1.078637in}{1.103220in}}%
\pgfpathlineto{\pgfqpoint{1.078637in}{1.096944in}}%
\pgfpathclose%
\pgfusepath{fill}%
\end{pgfscope}%
\begin{pgfscope}%
\pgfpathrectangle{\pgfqpoint{0.515000in}{1.096944in}}{\pgfqpoint{2.480000in}{1.848000in}}%
\pgfusepath{clip}%
\pgfsetbuttcap%
\pgfsetmiterjoin%
\definecolor{currentfill}{rgb}{0.000000,0.000000,0.000000}%
\pgfsetfillcolor{currentfill}%
\pgfsetlinewidth{0.000000pt}%
\definecolor{currentstroke}{rgb}{0.000000,0.000000,0.000000}%
\pgfsetstrokecolor{currentstroke}%
\pgfsetstrokeopacity{0.000000}%
\pgfsetdash{}{0pt}%
\pgfpathmoveto{\pgfqpoint{1.329142in}{1.096944in}}%
\pgfpathlineto{\pgfqpoint{1.429344in}{1.096944in}}%
\pgfpathlineto{\pgfqpoint{1.429344in}{1.134407in}}%
\pgfpathlineto{\pgfqpoint{1.329142in}{1.134407in}}%
\pgfpathlineto{\pgfqpoint{1.329142in}{1.096944in}}%
\pgfpathclose%
\pgfusepath{fill}%
\end{pgfscope}%
\begin{pgfscope}%
\pgfpathrectangle{\pgfqpoint{0.515000in}{1.096944in}}{\pgfqpoint{2.480000in}{1.848000in}}%
\pgfusepath{clip}%
\pgfsetbuttcap%
\pgfsetmiterjoin%
\definecolor{currentfill}{rgb}{0.000000,0.000000,0.000000}%
\pgfsetfillcolor{currentfill}%
\pgfsetlinewidth{0.000000pt}%
\definecolor{currentstroke}{rgb}{0.000000,0.000000,0.000000}%
\pgfsetstrokecolor{currentstroke}%
\pgfsetstrokeopacity{0.000000}%
\pgfsetdash{}{0pt}%
\pgfpathmoveto{\pgfqpoint{1.579647in}{1.096944in}}%
\pgfpathlineto{\pgfqpoint{1.679849in}{1.096944in}}%
\pgfpathlineto{\pgfqpoint{1.679849in}{1.215951in}}%
\pgfpathlineto{\pgfqpoint{1.579647in}{1.215951in}}%
\pgfpathlineto{\pgfqpoint{1.579647in}{1.096944in}}%
\pgfpathclose%
\pgfusepath{fill}%
\end{pgfscope}%
\begin{pgfscope}%
\pgfpathrectangle{\pgfqpoint{0.515000in}{1.096944in}}{\pgfqpoint{2.480000in}{1.848000in}}%
\pgfusepath{clip}%
\pgfsetbuttcap%
\pgfsetmiterjoin%
\definecolor{currentfill}{rgb}{0.000000,0.000000,0.000000}%
\pgfsetfillcolor{currentfill}%
\pgfsetlinewidth{0.000000pt}%
\definecolor{currentstroke}{rgb}{0.000000,0.000000,0.000000}%
\pgfsetstrokecolor{currentstroke}%
\pgfsetstrokeopacity{0.000000}%
\pgfsetdash{}{0pt}%
\pgfpathmoveto{\pgfqpoint{1.830152in}{1.096944in}}%
\pgfpathlineto{\pgfqpoint{1.930354in}{1.096944in}}%
\pgfpathlineto{\pgfqpoint{1.930354in}{1.375273in}}%
\pgfpathlineto{\pgfqpoint{1.830152in}{1.375273in}}%
\pgfpathlineto{\pgfqpoint{1.830152in}{1.096944in}}%
\pgfpathclose%
\pgfusepath{fill}%
\end{pgfscope}%
\begin{pgfscope}%
\pgfpathrectangle{\pgfqpoint{0.515000in}{1.096944in}}{\pgfqpoint{2.480000in}{1.848000in}}%
\pgfusepath{clip}%
\pgfsetbuttcap%
\pgfsetmiterjoin%
\definecolor{currentfill}{rgb}{0.000000,0.000000,0.000000}%
\pgfsetfillcolor{currentfill}%
\pgfsetlinewidth{0.000000pt}%
\definecolor{currentstroke}{rgb}{0.000000,0.000000,0.000000}%
\pgfsetstrokecolor{currentstroke}%
\pgfsetstrokeopacity{0.000000}%
\pgfsetdash{}{0pt}%
\pgfpathmoveto{\pgfqpoint{2.080657in}{1.096944in}}%
\pgfpathlineto{\pgfqpoint{2.180859in}{1.096944in}}%
\pgfpathlineto{\pgfqpoint{2.180859in}{1.388471in}}%
\pgfpathlineto{\pgfqpoint{2.080657in}{1.388471in}}%
\pgfpathlineto{\pgfqpoint{2.080657in}{1.096944in}}%
\pgfpathclose%
\pgfusepath{fill}%
\end{pgfscope}%
\begin{pgfscope}%
\pgfpathrectangle{\pgfqpoint{0.515000in}{1.096944in}}{\pgfqpoint{2.480000in}{1.848000in}}%
\pgfusepath{clip}%
\pgfsetbuttcap%
\pgfsetmiterjoin%
\definecolor{currentfill}{rgb}{0.000000,0.000000,0.000000}%
\pgfsetfillcolor{currentfill}%
\pgfsetlinewidth{0.000000pt}%
\definecolor{currentstroke}{rgb}{0.000000,0.000000,0.000000}%
\pgfsetstrokecolor{currentstroke}%
\pgfsetstrokeopacity{0.000000}%
\pgfsetdash{}{0pt}%
\pgfpathmoveto{\pgfqpoint{2.331162in}{1.096944in}}%
\pgfpathlineto{\pgfqpoint{2.431364in}{1.096944in}}%
\pgfpathlineto{\pgfqpoint{2.431364in}{1.125013in}}%
\pgfpathlineto{\pgfqpoint{2.331162in}{1.125013in}}%
\pgfpathlineto{\pgfqpoint{2.331162in}{1.096944in}}%
\pgfpathclose%
\pgfusepath{fill}%
\end{pgfscope}%
\begin{pgfscope}%
\pgfpathrectangle{\pgfqpoint{0.515000in}{1.096944in}}{\pgfqpoint{2.480000in}{1.848000in}}%
\pgfusepath{clip}%
\pgfsetbuttcap%
\pgfsetmiterjoin%
\definecolor{currentfill}{rgb}{0.000000,0.000000,0.000000}%
\pgfsetfillcolor{currentfill}%
\pgfsetlinewidth{0.000000pt}%
\definecolor{currentstroke}{rgb}{0.000000,0.000000,0.000000}%
\pgfsetstrokecolor{currentstroke}%
\pgfsetstrokeopacity{0.000000}%
\pgfsetdash{}{0pt}%
\pgfpathmoveto{\pgfqpoint{2.581667in}{1.096944in}}%
\pgfpathlineto{\pgfqpoint{2.681869in}{1.096944in}}%
\pgfpathlineto{\pgfqpoint{2.681869in}{1.096944in}}%
\pgfpathlineto{\pgfqpoint{2.581667in}{1.096944in}}%
\pgfpathlineto{\pgfqpoint{2.581667in}{1.096944in}}%
\pgfpathclose%
\pgfusepath{fill}%
\end{pgfscope}%
\begin{pgfscope}%
\pgfpathrectangle{\pgfqpoint{0.515000in}{1.096944in}}{\pgfqpoint{2.480000in}{1.848000in}}%
\pgfusepath{clip}%
\pgfsetbuttcap%
\pgfsetmiterjoin%
\definecolor{currentfill}{rgb}{0.000000,0.000000,0.000000}%
\pgfsetfillcolor{currentfill}%
\pgfsetlinewidth{0.000000pt}%
\definecolor{currentstroke}{rgb}{0.000000,0.000000,0.000000}%
\pgfsetstrokecolor{currentstroke}%
\pgfsetstrokeopacity{0.000000}%
\pgfsetdash{}{0pt}%
\pgfpathmoveto{\pgfqpoint{2.832172in}{1.096944in}}%
\pgfpathlineto{\pgfqpoint{2.932374in}{1.096944in}}%
\pgfpathlineto{\pgfqpoint{2.932374in}{1.096944in}}%
\pgfpathlineto{\pgfqpoint{2.832172in}{1.096944in}}%
\pgfpathlineto{\pgfqpoint{2.832172in}{1.096944in}}%
\pgfpathclose%
\pgfusepath{fill}%
\end{pgfscope}%
\begin{pgfscope}%
\pgfsetbuttcap%
\pgfsetroundjoin%
\definecolor{currentfill}{rgb}{0.000000,0.000000,0.000000}%
\pgfsetfillcolor{currentfill}%
\pgfsetlinewidth{0.803000pt}%
\definecolor{currentstroke}{rgb}{0.000000,0.000000,0.000000}%
\pgfsetstrokecolor{currentstroke}%
\pgfsetdash{}{0pt}%
\pgfsys@defobject{currentmarker}{\pgfqpoint{0.000000in}{-0.048611in}}{\pgfqpoint{0.000000in}{0.000000in}}{%
\pgfpathmoveto{\pgfqpoint{0.000000in}{0.000000in}}%
\pgfpathlineto{\pgfqpoint{0.000000in}{-0.048611in}}%
\pgfusepath{stroke,fill}%
}%
\begin{pgfscope}%
\pgfsys@transformshift{0.577627in}{1.096944in}%
\pgfsys@useobject{currentmarker}{}%
\end{pgfscope}%
\end{pgfscope}%
\begin{pgfscope}%
\definecolor{textcolor}{rgb}{0.000000,0.000000,0.000000}%
\pgfsetstrokecolor{textcolor}%
\pgfsetfillcolor{textcolor}%
\pgftext[x=0.612349in, y=0.282083in, left, base,rotate=90.000000]{\color{textcolor}\rmfamily\fontsize{10.000000}{12.000000}\selectfont (-0.001, 0.1]}%
\end{pgfscope}%
\begin{pgfscope}%
\pgfsetbuttcap%
\pgfsetroundjoin%
\definecolor{currentfill}{rgb}{0.000000,0.000000,0.000000}%
\pgfsetfillcolor{currentfill}%
\pgfsetlinewidth{0.803000pt}%
\definecolor{currentstroke}{rgb}{0.000000,0.000000,0.000000}%
\pgfsetstrokecolor{currentstroke}%
\pgfsetdash{}{0pt}%
\pgfsys@defobject{currentmarker}{\pgfqpoint{0.000000in}{-0.048611in}}{\pgfqpoint{0.000000in}{0.000000in}}{%
\pgfpathmoveto{\pgfqpoint{0.000000in}{0.000000in}}%
\pgfpathlineto{\pgfqpoint{0.000000in}{-0.048611in}}%
\pgfusepath{stroke,fill}%
}%
\begin{pgfscope}%
\pgfsys@transformshift{0.828132in}{1.096944in}%
\pgfsys@useobject{currentmarker}{}%
\end{pgfscope}%
\end{pgfscope}%
\begin{pgfscope}%
\definecolor{textcolor}{rgb}{0.000000,0.000000,0.000000}%
\pgfsetstrokecolor{textcolor}%
\pgfsetfillcolor{textcolor}%
\pgftext[x=0.862854in, y=0.467222in, left, base,rotate=90.000000]{\color{textcolor}\rmfamily\fontsize{10.000000}{12.000000}\selectfont (0.1, 0.2]}%
\end{pgfscope}%
\begin{pgfscope}%
\pgfsetbuttcap%
\pgfsetroundjoin%
\definecolor{currentfill}{rgb}{0.000000,0.000000,0.000000}%
\pgfsetfillcolor{currentfill}%
\pgfsetlinewidth{0.803000pt}%
\definecolor{currentstroke}{rgb}{0.000000,0.000000,0.000000}%
\pgfsetstrokecolor{currentstroke}%
\pgfsetdash{}{0pt}%
\pgfsys@defobject{currentmarker}{\pgfqpoint{0.000000in}{-0.048611in}}{\pgfqpoint{0.000000in}{0.000000in}}{%
\pgfpathmoveto{\pgfqpoint{0.000000in}{0.000000in}}%
\pgfpathlineto{\pgfqpoint{0.000000in}{-0.048611in}}%
\pgfusepath{stroke,fill}%
}%
\begin{pgfscope}%
\pgfsys@transformshift{1.078637in}{1.096944in}%
\pgfsys@useobject{currentmarker}{}%
\end{pgfscope}%
\end{pgfscope}%
\begin{pgfscope}%
\definecolor{textcolor}{rgb}{0.000000,0.000000,0.000000}%
\pgfsetstrokecolor{textcolor}%
\pgfsetfillcolor{textcolor}%
\pgftext[x=1.113359in, y=0.467222in, left, base,rotate=90.000000]{\color{textcolor}\rmfamily\fontsize{10.000000}{12.000000}\selectfont (0.2, 0.3]}%
\end{pgfscope}%
\begin{pgfscope}%
\pgfsetbuttcap%
\pgfsetroundjoin%
\definecolor{currentfill}{rgb}{0.000000,0.000000,0.000000}%
\pgfsetfillcolor{currentfill}%
\pgfsetlinewidth{0.803000pt}%
\definecolor{currentstroke}{rgb}{0.000000,0.000000,0.000000}%
\pgfsetstrokecolor{currentstroke}%
\pgfsetdash{}{0pt}%
\pgfsys@defobject{currentmarker}{\pgfqpoint{0.000000in}{-0.048611in}}{\pgfqpoint{0.000000in}{0.000000in}}{%
\pgfpathmoveto{\pgfqpoint{0.000000in}{0.000000in}}%
\pgfpathlineto{\pgfqpoint{0.000000in}{-0.048611in}}%
\pgfusepath{stroke,fill}%
}%
\begin{pgfscope}%
\pgfsys@transformshift{1.329142in}{1.096944in}%
\pgfsys@useobject{currentmarker}{}%
\end{pgfscope}%
\end{pgfscope}%
\begin{pgfscope}%
\definecolor{textcolor}{rgb}{0.000000,0.000000,0.000000}%
\pgfsetstrokecolor{textcolor}%
\pgfsetfillcolor{textcolor}%
\pgftext[x=1.363864in, y=0.467222in, left, base,rotate=90.000000]{\color{textcolor}\rmfamily\fontsize{10.000000}{12.000000}\selectfont (0.3, 0.4]}%
\end{pgfscope}%
\begin{pgfscope}%
\pgfsetbuttcap%
\pgfsetroundjoin%
\definecolor{currentfill}{rgb}{0.000000,0.000000,0.000000}%
\pgfsetfillcolor{currentfill}%
\pgfsetlinewidth{0.803000pt}%
\definecolor{currentstroke}{rgb}{0.000000,0.000000,0.000000}%
\pgfsetstrokecolor{currentstroke}%
\pgfsetdash{}{0pt}%
\pgfsys@defobject{currentmarker}{\pgfqpoint{0.000000in}{-0.048611in}}{\pgfqpoint{0.000000in}{0.000000in}}{%
\pgfpathmoveto{\pgfqpoint{0.000000in}{0.000000in}}%
\pgfpathlineto{\pgfqpoint{0.000000in}{-0.048611in}}%
\pgfusepath{stroke,fill}%
}%
\begin{pgfscope}%
\pgfsys@transformshift{1.579647in}{1.096944in}%
\pgfsys@useobject{currentmarker}{}%
\end{pgfscope}%
\end{pgfscope}%
\begin{pgfscope}%
\definecolor{textcolor}{rgb}{0.000000,0.000000,0.000000}%
\pgfsetstrokecolor{textcolor}%
\pgfsetfillcolor{textcolor}%
\pgftext[x=1.614369in, y=0.467222in, left, base,rotate=90.000000]{\color{textcolor}\rmfamily\fontsize{10.000000}{12.000000}\selectfont (0.4, 0.5]}%
\end{pgfscope}%
\begin{pgfscope}%
\pgfsetbuttcap%
\pgfsetroundjoin%
\definecolor{currentfill}{rgb}{0.000000,0.000000,0.000000}%
\pgfsetfillcolor{currentfill}%
\pgfsetlinewidth{0.803000pt}%
\definecolor{currentstroke}{rgb}{0.000000,0.000000,0.000000}%
\pgfsetstrokecolor{currentstroke}%
\pgfsetdash{}{0pt}%
\pgfsys@defobject{currentmarker}{\pgfqpoint{0.000000in}{-0.048611in}}{\pgfqpoint{0.000000in}{0.000000in}}{%
\pgfpathmoveto{\pgfqpoint{0.000000in}{0.000000in}}%
\pgfpathlineto{\pgfqpoint{0.000000in}{-0.048611in}}%
\pgfusepath{stroke,fill}%
}%
\begin{pgfscope}%
\pgfsys@transformshift{1.830152in}{1.096944in}%
\pgfsys@useobject{currentmarker}{}%
\end{pgfscope}%
\end{pgfscope}%
\begin{pgfscope}%
\definecolor{textcolor}{rgb}{0.000000,0.000000,0.000000}%
\pgfsetstrokecolor{textcolor}%
\pgfsetfillcolor{textcolor}%
\pgftext[x=1.864874in, y=0.467222in, left, base,rotate=90.000000]{\color{textcolor}\rmfamily\fontsize{10.000000}{12.000000}\selectfont (0.5, 0.6]}%
\end{pgfscope}%
\begin{pgfscope}%
\pgfsetbuttcap%
\pgfsetroundjoin%
\definecolor{currentfill}{rgb}{0.000000,0.000000,0.000000}%
\pgfsetfillcolor{currentfill}%
\pgfsetlinewidth{0.803000pt}%
\definecolor{currentstroke}{rgb}{0.000000,0.000000,0.000000}%
\pgfsetstrokecolor{currentstroke}%
\pgfsetdash{}{0pt}%
\pgfsys@defobject{currentmarker}{\pgfqpoint{0.000000in}{-0.048611in}}{\pgfqpoint{0.000000in}{0.000000in}}{%
\pgfpathmoveto{\pgfqpoint{0.000000in}{0.000000in}}%
\pgfpathlineto{\pgfqpoint{0.000000in}{-0.048611in}}%
\pgfusepath{stroke,fill}%
}%
\begin{pgfscope}%
\pgfsys@transformshift{2.080657in}{1.096944in}%
\pgfsys@useobject{currentmarker}{}%
\end{pgfscope}%
\end{pgfscope}%
\begin{pgfscope}%
\definecolor{textcolor}{rgb}{0.000000,0.000000,0.000000}%
\pgfsetstrokecolor{textcolor}%
\pgfsetfillcolor{textcolor}%
\pgftext[x=2.115379in, y=0.467222in, left, base,rotate=90.000000]{\color{textcolor}\rmfamily\fontsize{10.000000}{12.000000}\selectfont (0.6, 0.7]}%
\end{pgfscope}%
\begin{pgfscope}%
\pgfsetbuttcap%
\pgfsetroundjoin%
\definecolor{currentfill}{rgb}{0.000000,0.000000,0.000000}%
\pgfsetfillcolor{currentfill}%
\pgfsetlinewidth{0.803000pt}%
\definecolor{currentstroke}{rgb}{0.000000,0.000000,0.000000}%
\pgfsetstrokecolor{currentstroke}%
\pgfsetdash{}{0pt}%
\pgfsys@defobject{currentmarker}{\pgfqpoint{0.000000in}{-0.048611in}}{\pgfqpoint{0.000000in}{0.000000in}}{%
\pgfpathmoveto{\pgfqpoint{0.000000in}{0.000000in}}%
\pgfpathlineto{\pgfqpoint{0.000000in}{-0.048611in}}%
\pgfusepath{stroke,fill}%
}%
\begin{pgfscope}%
\pgfsys@transformshift{2.331162in}{1.096944in}%
\pgfsys@useobject{currentmarker}{}%
\end{pgfscope}%
\end{pgfscope}%
\begin{pgfscope}%
\definecolor{textcolor}{rgb}{0.000000,0.000000,0.000000}%
\pgfsetstrokecolor{textcolor}%
\pgfsetfillcolor{textcolor}%
\pgftext[x=2.365884in, y=0.467222in, left, base,rotate=90.000000]{\color{textcolor}\rmfamily\fontsize{10.000000}{12.000000}\selectfont (0.7, 0.8]}%
\end{pgfscope}%
\begin{pgfscope}%
\pgfsetbuttcap%
\pgfsetroundjoin%
\definecolor{currentfill}{rgb}{0.000000,0.000000,0.000000}%
\pgfsetfillcolor{currentfill}%
\pgfsetlinewidth{0.803000pt}%
\definecolor{currentstroke}{rgb}{0.000000,0.000000,0.000000}%
\pgfsetstrokecolor{currentstroke}%
\pgfsetdash{}{0pt}%
\pgfsys@defobject{currentmarker}{\pgfqpoint{0.000000in}{-0.048611in}}{\pgfqpoint{0.000000in}{0.000000in}}{%
\pgfpathmoveto{\pgfqpoint{0.000000in}{0.000000in}}%
\pgfpathlineto{\pgfqpoint{0.000000in}{-0.048611in}}%
\pgfusepath{stroke,fill}%
}%
\begin{pgfscope}%
\pgfsys@transformshift{2.581667in}{1.096944in}%
\pgfsys@useobject{currentmarker}{}%
\end{pgfscope}%
\end{pgfscope}%
\begin{pgfscope}%
\definecolor{textcolor}{rgb}{0.000000,0.000000,0.000000}%
\pgfsetstrokecolor{textcolor}%
\pgfsetfillcolor{textcolor}%
\pgftext[x=2.616389in, y=0.467222in, left, base,rotate=90.000000]{\color{textcolor}\rmfamily\fontsize{10.000000}{12.000000}\selectfont (0.8, 0.9]}%
\end{pgfscope}%
\begin{pgfscope}%
\pgfsetbuttcap%
\pgfsetroundjoin%
\definecolor{currentfill}{rgb}{0.000000,0.000000,0.000000}%
\pgfsetfillcolor{currentfill}%
\pgfsetlinewidth{0.803000pt}%
\definecolor{currentstroke}{rgb}{0.000000,0.000000,0.000000}%
\pgfsetstrokecolor{currentstroke}%
\pgfsetdash{}{0pt}%
\pgfsys@defobject{currentmarker}{\pgfqpoint{0.000000in}{-0.048611in}}{\pgfqpoint{0.000000in}{0.000000in}}{%
\pgfpathmoveto{\pgfqpoint{0.000000in}{0.000000in}}%
\pgfpathlineto{\pgfqpoint{0.000000in}{-0.048611in}}%
\pgfusepath{stroke,fill}%
}%
\begin{pgfscope}%
\pgfsys@transformshift{2.832172in}{1.096944in}%
\pgfsys@useobject{currentmarker}{}%
\end{pgfscope}%
\end{pgfscope}%
\begin{pgfscope}%
\definecolor{textcolor}{rgb}{0.000000,0.000000,0.000000}%
\pgfsetstrokecolor{textcolor}%
\pgfsetfillcolor{textcolor}%
\pgftext[x=2.866894in, y=0.467222in, left, base,rotate=90.000000]{\color{textcolor}\rmfamily\fontsize{10.000000}{12.000000}\selectfont (0.9, 1.0]}%
\end{pgfscope}%
\begin{pgfscope}%
\definecolor{textcolor}{rgb}{0.000000,0.000000,0.000000}%
\pgfsetstrokecolor{textcolor}%
\pgfsetfillcolor{textcolor}%
\pgftext[x=1.755000in,y=0.226527in,,top]{\color{textcolor}\rmfamily\fontsize{10.000000}{12.000000}\selectfont Range of Prediction}%
\end{pgfscope}%
\begin{pgfscope}%
\pgfsetbuttcap%
\pgfsetroundjoin%
\definecolor{currentfill}{rgb}{0.000000,0.000000,0.000000}%
\pgfsetfillcolor{currentfill}%
\pgfsetlinewidth{0.803000pt}%
\definecolor{currentstroke}{rgb}{0.000000,0.000000,0.000000}%
\pgfsetstrokecolor{currentstroke}%
\pgfsetdash{}{0pt}%
\pgfsys@defobject{currentmarker}{\pgfqpoint{-0.048611in}{0.000000in}}{\pgfqpoint{-0.000000in}{0.000000in}}{%
\pgfpathmoveto{\pgfqpoint{-0.000000in}{0.000000in}}%
\pgfpathlineto{\pgfqpoint{-0.048611in}{0.000000in}}%
\pgfusepath{stroke,fill}%
}%
\begin{pgfscope}%
\pgfsys@transformshift{0.515000in}{1.096944in}%
\pgfsys@useobject{currentmarker}{}%
\end{pgfscope}%
\end{pgfscope}%
\begin{pgfscope}%
\definecolor{textcolor}{rgb}{0.000000,0.000000,0.000000}%
\pgfsetstrokecolor{textcolor}%
\pgfsetfillcolor{textcolor}%
\pgftext[x=0.348333in, y=1.048750in, left, base]{\color{textcolor}\rmfamily\fontsize{10.000000}{12.000000}\selectfont \(\displaystyle {0}\)}%
\end{pgfscope}%
\begin{pgfscope}%
\pgfsetbuttcap%
\pgfsetroundjoin%
\definecolor{currentfill}{rgb}{0.000000,0.000000,0.000000}%
\pgfsetfillcolor{currentfill}%
\pgfsetlinewidth{0.803000pt}%
\definecolor{currentstroke}{rgb}{0.000000,0.000000,0.000000}%
\pgfsetstrokecolor{currentstroke}%
\pgfsetdash{}{0pt}%
\pgfsys@defobject{currentmarker}{\pgfqpoint{-0.048611in}{0.000000in}}{\pgfqpoint{-0.000000in}{0.000000in}}{%
\pgfpathmoveto{\pgfqpoint{-0.000000in}{0.000000in}}%
\pgfpathlineto{\pgfqpoint{-0.048611in}{0.000000in}}%
\pgfusepath{stroke,fill}%
}%
\begin{pgfscope}%
\pgfsys@transformshift{0.515000in}{1.629413in}%
\pgfsys@useobject{currentmarker}{}%
\end{pgfscope}%
\end{pgfscope}%
\begin{pgfscope}%
\definecolor{textcolor}{rgb}{0.000000,0.000000,0.000000}%
\pgfsetstrokecolor{textcolor}%
\pgfsetfillcolor{textcolor}%
\pgftext[x=0.278889in, y=1.581219in, left, base]{\color{textcolor}\rmfamily\fontsize{10.000000}{12.000000}\selectfont \(\displaystyle {10}\)}%
\end{pgfscope}%
\begin{pgfscope}%
\pgfsetbuttcap%
\pgfsetroundjoin%
\definecolor{currentfill}{rgb}{0.000000,0.000000,0.000000}%
\pgfsetfillcolor{currentfill}%
\pgfsetlinewidth{0.803000pt}%
\definecolor{currentstroke}{rgb}{0.000000,0.000000,0.000000}%
\pgfsetstrokecolor{currentstroke}%
\pgfsetdash{}{0pt}%
\pgfsys@defobject{currentmarker}{\pgfqpoint{-0.048611in}{0.000000in}}{\pgfqpoint{-0.000000in}{0.000000in}}{%
\pgfpathmoveto{\pgfqpoint{-0.000000in}{0.000000in}}%
\pgfpathlineto{\pgfqpoint{-0.048611in}{0.000000in}}%
\pgfusepath{stroke,fill}%
}%
\begin{pgfscope}%
\pgfsys@transformshift{0.515000in}{2.161882in}%
\pgfsys@useobject{currentmarker}{}%
\end{pgfscope}%
\end{pgfscope}%
\begin{pgfscope}%
\definecolor{textcolor}{rgb}{0.000000,0.000000,0.000000}%
\pgfsetstrokecolor{textcolor}%
\pgfsetfillcolor{textcolor}%
\pgftext[x=0.278889in, y=2.113688in, left, base]{\color{textcolor}\rmfamily\fontsize{10.000000}{12.000000}\selectfont \(\displaystyle {20}\)}%
\end{pgfscope}%
\begin{pgfscope}%
\pgfsetbuttcap%
\pgfsetroundjoin%
\definecolor{currentfill}{rgb}{0.000000,0.000000,0.000000}%
\pgfsetfillcolor{currentfill}%
\pgfsetlinewidth{0.803000pt}%
\definecolor{currentstroke}{rgb}{0.000000,0.000000,0.000000}%
\pgfsetstrokecolor{currentstroke}%
\pgfsetdash{}{0pt}%
\pgfsys@defobject{currentmarker}{\pgfqpoint{-0.048611in}{0.000000in}}{\pgfqpoint{-0.000000in}{0.000000in}}{%
\pgfpathmoveto{\pgfqpoint{-0.000000in}{0.000000in}}%
\pgfpathlineto{\pgfqpoint{-0.048611in}{0.000000in}}%
\pgfusepath{stroke,fill}%
}%
\begin{pgfscope}%
\pgfsys@transformshift{0.515000in}{2.694351in}%
\pgfsys@useobject{currentmarker}{}%
\end{pgfscope}%
\end{pgfscope}%
\begin{pgfscope}%
\definecolor{textcolor}{rgb}{0.000000,0.000000,0.000000}%
\pgfsetstrokecolor{textcolor}%
\pgfsetfillcolor{textcolor}%
\pgftext[x=0.278889in, y=2.646157in, left, base]{\color{textcolor}\rmfamily\fontsize{10.000000}{12.000000}\selectfont \(\displaystyle {30}\)}%
\end{pgfscope}%
\begin{pgfscope}%
\definecolor{textcolor}{rgb}{0.000000,0.000000,0.000000}%
\pgfsetstrokecolor{textcolor}%
\pgfsetfillcolor{textcolor}%
\pgftext[x=0.223333in,y=2.020944in,,bottom,rotate=90.000000]{\color{textcolor}\rmfamily\fontsize{10.000000}{12.000000}\selectfont Percent of Data Set}%
\end{pgfscope}%
\begin{pgfscope}%
\pgfsetrectcap%
\pgfsetmiterjoin%
\pgfsetlinewidth{0.803000pt}%
\definecolor{currentstroke}{rgb}{0.000000,0.000000,0.000000}%
\pgfsetstrokecolor{currentstroke}%
\pgfsetdash{}{0pt}%
\pgfpathmoveto{\pgfqpoint{0.515000in}{1.096944in}}%
\pgfpathlineto{\pgfqpoint{0.515000in}{2.944944in}}%
\pgfusepath{stroke}%
\end{pgfscope}%
\begin{pgfscope}%
\pgfsetrectcap%
\pgfsetmiterjoin%
\pgfsetlinewidth{0.803000pt}%
\definecolor{currentstroke}{rgb}{0.000000,0.000000,0.000000}%
\pgfsetstrokecolor{currentstroke}%
\pgfsetdash{}{0pt}%
\pgfpathmoveto{\pgfqpoint{2.995000in}{1.096944in}}%
\pgfpathlineto{\pgfqpoint{2.995000in}{2.944944in}}%
\pgfusepath{stroke}%
\end{pgfscope}%
\begin{pgfscope}%
\pgfsetrectcap%
\pgfsetmiterjoin%
\pgfsetlinewidth{0.803000pt}%
\definecolor{currentstroke}{rgb}{0.000000,0.000000,0.000000}%
\pgfsetstrokecolor{currentstroke}%
\pgfsetdash{}{0pt}%
\pgfpathmoveto{\pgfqpoint{0.515000in}{1.096944in}}%
\pgfpathlineto{\pgfqpoint{2.995000in}{1.096944in}}%
\pgfusepath{stroke}%
\end{pgfscope}%
\begin{pgfscope}%
\pgfsetrectcap%
\pgfsetmiterjoin%
\pgfsetlinewidth{0.803000pt}%
\definecolor{currentstroke}{rgb}{0.000000,0.000000,0.000000}%
\pgfsetstrokecolor{currentstroke}%
\pgfsetdash{}{0pt}%
\pgfpathmoveto{\pgfqpoint{0.515000in}{2.944944in}}%
\pgfpathlineto{\pgfqpoint{2.995000in}{2.944944in}}%
\pgfusepath{stroke}%
\end{pgfscope}%
\begin{pgfscope}%
\definecolor{textcolor}{rgb}{0.000000,0.000000,0.000000}%
\pgfsetstrokecolor{textcolor}%
\pgfsetfillcolor{textcolor}%
\pgftext[x=1.755000in,y=3.028277in,,base]{\color{textcolor}\rmfamily\fontsize{12.000000}{14.400000}\selectfont Probability Distribution}%
\end{pgfscope}%
\begin{pgfscope}%
\pgfsetbuttcap%
\pgfsetmiterjoin%
\definecolor{currentfill}{rgb}{1.000000,1.000000,1.000000}%
\pgfsetfillcolor{currentfill}%
\pgfsetfillopacity{0.800000}%
\pgfsetlinewidth{1.003750pt}%
\definecolor{currentstroke}{rgb}{0.800000,0.800000,0.800000}%
\pgfsetstrokecolor{currentstroke}%
\pgfsetstrokeopacity{0.800000}%
\pgfsetdash{}{0pt}%
\pgfpathmoveto{\pgfqpoint{1.560833in}{2.444250in}}%
\pgfpathlineto{\pgfqpoint{2.897778in}{2.444250in}}%
\pgfpathquadraticcurveto{\pgfqpoint{2.925556in}{2.444250in}}{\pgfqpoint{2.925556in}{2.472028in}}%
\pgfpathlineto{\pgfqpoint{2.925556in}{2.847722in}}%
\pgfpathquadraticcurveto{\pgfqpoint{2.925556in}{2.875500in}}{\pgfqpoint{2.897778in}{2.875500in}}%
\pgfpathlineto{\pgfqpoint{1.560833in}{2.875500in}}%
\pgfpathquadraticcurveto{\pgfqpoint{1.533056in}{2.875500in}}{\pgfqpoint{1.533056in}{2.847722in}}%
\pgfpathlineto{\pgfqpoint{1.533056in}{2.472028in}}%
\pgfpathquadraticcurveto{\pgfqpoint{1.533056in}{2.444250in}}{\pgfqpoint{1.560833in}{2.444250in}}%
\pgfpathlineto{\pgfqpoint{1.560833in}{2.444250in}}%
\pgfpathclose%
\pgfusepath{stroke,fill}%
\end{pgfscope}%
\begin{pgfscope}%
\pgfsetbuttcap%
\pgfsetmiterjoin%
\pgfsetlinewidth{1.003750pt}%
\definecolor{currentstroke}{rgb}{0.000000,0.000000,0.000000}%
\pgfsetstrokecolor{currentstroke}%
\pgfsetdash{}{0pt}%
\pgfpathmoveto{\pgfqpoint{1.588611in}{2.722027in}}%
\pgfpathlineto{\pgfqpoint{1.866389in}{2.722027in}}%
\pgfpathlineto{\pgfqpoint{1.866389in}{2.819250in}}%
\pgfpathlineto{\pgfqpoint{1.588611in}{2.819250in}}%
\pgfpathlineto{\pgfqpoint{1.588611in}{2.722027in}}%
\pgfpathclose%
\pgfusepath{stroke}%
\end{pgfscope}%
\begin{pgfscope}%
\definecolor{textcolor}{rgb}{0.000000,0.000000,0.000000}%
\pgfsetstrokecolor{textcolor}%
\pgfsetfillcolor{textcolor}%
\pgftext[x=1.977500in,y=2.722027in,left,base]{\color{textcolor}\rmfamily\fontsize{10.000000}{12.000000}\selectfont Negative Class}%
\end{pgfscope}%
\begin{pgfscope}%
\pgfsetbuttcap%
\pgfsetmiterjoin%
\definecolor{currentfill}{rgb}{0.000000,0.000000,0.000000}%
\pgfsetfillcolor{currentfill}%
\pgfsetlinewidth{0.000000pt}%
\definecolor{currentstroke}{rgb}{0.000000,0.000000,0.000000}%
\pgfsetstrokecolor{currentstroke}%
\pgfsetstrokeopacity{0.000000}%
\pgfsetdash{}{0pt}%
\pgfpathmoveto{\pgfqpoint{1.588611in}{2.526750in}}%
\pgfpathlineto{\pgfqpoint{1.866389in}{2.526750in}}%
\pgfpathlineto{\pgfqpoint{1.866389in}{2.623972in}}%
\pgfpathlineto{\pgfqpoint{1.588611in}{2.623972in}}%
\pgfpathlineto{\pgfqpoint{1.588611in}{2.526750in}}%
\pgfpathclose%
\pgfusepath{fill}%
\end{pgfscope}%
\begin{pgfscope}%
\definecolor{textcolor}{rgb}{0.000000,0.000000,0.000000}%
\pgfsetstrokecolor{textcolor}%
\pgfsetfillcolor{textcolor}%
\pgftext[x=1.977500in,y=2.526750in,left,base]{\color{textcolor}\rmfamily\fontsize{10.000000}{12.000000}\selectfont Positive Class}%
\end{pgfscope}%
\end{pgfpicture}%
\makeatother%
\endgroup%

  &
  \vspace{0pt} %% Creator: Matplotlib, PGF backend
%%
%% To include the figure in your LaTeX document, write
%%   \input{<filename>.pgf}
%%
%% Make sure the required packages are loaded in your preamble
%%   \usepackage{pgf}
%%
%% Also ensure that all the required font packages are loaded; for instance,
%% the lmodern package is sometimes necessary when using math font.
%%   \usepackage{lmodern}
%%
%% Figures using additional raster images can only be included by \input if
%% they are in the same directory as the main LaTeX file. For loading figures
%% from other directories you can use the `import` package
%%   \usepackage{import}
%%
%% and then include the figures with
%%   \import{<path to file>}{<filename>.pgf}
%%
%% Matplotlib used the following preamble
%%   
%%   \usepackage{fontspec}
%%   \makeatletter\@ifpackageloaded{underscore}{}{\usepackage[strings]{underscore}}\makeatother
%%
\begingroup%
\makeatletter%
\begin{pgfpicture}%
\pgfpathrectangle{\pgfpointorigin}{\pgfqpoint{3.144311in}{2.646444in}}%
\pgfusepath{use as bounding box, clip}%
\begin{pgfscope}%
\pgfsetbuttcap%
\pgfsetmiterjoin%
\definecolor{currentfill}{rgb}{1.000000,1.000000,1.000000}%
\pgfsetfillcolor{currentfill}%
\pgfsetlinewidth{0.000000pt}%
\definecolor{currentstroke}{rgb}{1.000000,1.000000,1.000000}%
\pgfsetstrokecolor{currentstroke}%
\pgfsetdash{}{0pt}%
\pgfpathmoveto{\pgfqpoint{0.000000in}{0.000000in}}%
\pgfpathlineto{\pgfqpoint{3.144311in}{0.000000in}}%
\pgfpathlineto{\pgfqpoint{3.144311in}{2.646444in}}%
\pgfpathlineto{\pgfqpoint{0.000000in}{2.646444in}}%
\pgfpathlineto{\pgfqpoint{0.000000in}{0.000000in}}%
\pgfpathclose%
\pgfusepath{fill}%
\end{pgfscope}%
\begin{pgfscope}%
\pgfsetbuttcap%
\pgfsetmiterjoin%
\definecolor{currentfill}{rgb}{1.000000,1.000000,1.000000}%
\pgfsetfillcolor{currentfill}%
\pgfsetlinewidth{0.000000pt}%
\definecolor{currentstroke}{rgb}{0.000000,0.000000,0.000000}%
\pgfsetstrokecolor{currentstroke}%
\pgfsetstrokeopacity{0.000000}%
\pgfsetdash{}{0pt}%
\pgfpathmoveto{\pgfqpoint{0.553581in}{0.499444in}}%
\pgfpathlineto{\pgfqpoint{3.033581in}{0.499444in}}%
\pgfpathlineto{\pgfqpoint{3.033581in}{2.347444in}}%
\pgfpathlineto{\pgfqpoint{0.553581in}{2.347444in}}%
\pgfpathlineto{\pgfqpoint{0.553581in}{0.499444in}}%
\pgfpathclose%
\pgfusepath{fill}%
\end{pgfscope}%
\begin{pgfscope}%
\pgfsetbuttcap%
\pgfsetroundjoin%
\definecolor{currentfill}{rgb}{0.000000,0.000000,0.000000}%
\pgfsetfillcolor{currentfill}%
\pgfsetlinewidth{0.803000pt}%
\definecolor{currentstroke}{rgb}{0.000000,0.000000,0.000000}%
\pgfsetstrokecolor{currentstroke}%
\pgfsetdash{}{0pt}%
\pgfsys@defobject{currentmarker}{\pgfqpoint{0.000000in}{-0.048611in}}{\pgfqpoint{0.000000in}{0.000000in}}{%
\pgfpathmoveto{\pgfqpoint{0.000000in}{0.000000in}}%
\pgfpathlineto{\pgfqpoint{0.000000in}{-0.048611in}}%
\pgfusepath{stroke,fill}%
}%
\begin{pgfscope}%
\pgfsys@transformshift{0.666308in}{0.499444in}%
\pgfsys@useobject{currentmarker}{}%
\end{pgfscope}%
\end{pgfscope}%
\begin{pgfscope}%
\definecolor{textcolor}{rgb}{0.000000,0.000000,0.000000}%
\pgfsetstrokecolor{textcolor}%
\pgfsetfillcolor{textcolor}%
\pgftext[x=0.666308in,y=0.402222in,,top]{\color{textcolor}\rmfamily\fontsize{10.000000}{12.000000}\selectfont \(\displaystyle {0.00}\)}%
\end{pgfscope}%
\begin{pgfscope}%
\pgfsetbuttcap%
\pgfsetroundjoin%
\definecolor{currentfill}{rgb}{0.000000,0.000000,0.000000}%
\pgfsetfillcolor{currentfill}%
\pgfsetlinewidth{0.803000pt}%
\definecolor{currentstroke}{rgb}{0.000000,0.000000,0.000000}%
\pgfsetstrokecolor{currentstroke}%
\pgfsetdash{}{0pt}%
\pgfsys@defobject{currentmarker}{\pgfqpoint{0.000000in}{-0.048611in}}{\pgfqpoint{0.000000in}{0.000000in}}{%
\pgfpathmoveto{\pgfqpoint{0.000000in}{0.000000in}}%
\pgfpathlineto{\pgfqpoint{0.000000in}{-0.048611in}}%
\pgfusepath{stroke,fill}%
}%
\begin{pgfscope}%
\pgfsys@transformshift{1.229944in}{0.499444in}%
\pgfsys@useobject{currentmarker}{}%
\end{pgfscope}%
\end{pgfscope}%
\begin{pgfscope}%
\definecolor{textcolor}{rgb}{0.000000,0.000000,0.000000}%
\pgfsetstrokecolor{textcolor}%
\pgfsetfillcolor{textcolor}%
\pgftext[x=1.229944in,y=0.402222in,,top]{\color{textcolor}\rmfamily\fontsize{10.000000}{12.000000}\selectfont \(\displaystyle {0.25}\)}%
\end{pgfscope}%
\begin{pgfscope}%
\pgfsetbuttcap%
\pgfsetroundjoin%
\definecolor{currentfill}{rgb}{0.000000,0.000000,0.000000}%
\pgfsetfillcolor{currentfill}%
\pgfsetlinewidth{0.803000pt}%
\definecolor{currentstroke}{rgb}{0.000000,0.000000,0.000000}%
\pgfsetstrokecolor{currentstroke}%
\pgfsetdash{}{0pt}%
\pgfsys@defobject{currentmarker}{\pgfqpoint{0.000000in}{-0.048611in}}{\pgfqpoint{0.000000in}{0.000000in}}{%
\pgfpathmoveto{\pgfqpoint{0.000000in}{0.000000in}}%
\pgfpathlineto{\pgfqpoint{0.000000in}{-0.048611in}}%
\pgfusepath{stroke,fill}%
}%
\begin{pgfscope}%
\pgfsys@transformshift{1.793581in}{0.499444in}%
\pgfsys@useobject{currentmarker}{}%
\end{pgfscope}%
\end{pgfscope}%
\begin{pgfscope}%
\definecolor{textcolor}{rgb}{0.000000,0.000000,0.000000}%
\pgfsetstrokecolor{textcolor}%
\pgfsetfillcolor{textcolor}%
\pgftext[x=1.793581in,y=0.402222in,,top]{\color{textcolor}\rmfamily\fontsize{10.000000}{12.000000}\selectfont \(\displaystyle {0.50}\)}%
\end{pgfscope}%
\begin{pgfscope}%
\pgfsetbuttcap%
\pgfsetroundjoin%
\definecolor{currentfill}{rgb}{0.000000,0.000000,0.000000}%
\pgfsetfillcolor{currentfill}%
\pgfsetlinewidth{0.803000pt}%
\definecolor{currentstroke}{rgb}{0.000000,0.000000,0.000000}%
\pgfsetstrokecolor{currentstroke}%
\pgfsetdash{}{0pt}%
\pgfsys@defobject{currentmarker}{\pgfqpoint{0.000000in}{-0.048611in}}{\pgfqpoint{0.000000in}{0.000000in}}{%
\pgfpathmoveto{\pgfqpoint{0.000000in}{0.000000in}}%
\pgfpathlineto{\pgfqpoint{0.000000in}{-0.048611in}}%
\pgfusepath{stroke,fill}%
}%
\begin{pgfscope}%
\pgfsys@transformshift{2.357217in}{0.499444in}%
\pgfsys@useobject{currentmarker}{}%
\end{pgfscope}%
\end{pgfscope}%
\begin{pgfscope}%
\definecolor{textcolor}{rgb}{0.000000,0.000000,0.000000}%
\pgfsetstrokecolor{textcolor}%
\pgfsetfillcolor{textcolor}%
\pgftext[x=2.357217in,y=0.402222in,,top]{\color{textcolor}\rmfamily\fontsize{10.000000}{12.000000}\selectfont \(\displaystyle {0.75}\)}%
\end{pgfscope}%
\begin{pgfscope}%
\pgfsetbuttcap%
\pgfsetroundjoin%
\definecolor{currentfill}{rgb}{0.000000,0.000000,0.000000}%
\pgfsetfillcolor{currentfill}%
\pgfsetlinewidth{0.803000pt}%
\definecolor{currentstroke}{rgb}{0.000000,0.000000,0.000000}%
\pgfsetstrokecolor{currentstroke}%
\pgfsetdash{}{0pt}%
\pgfsys@defobject{currentmarker}{\pgfqpoint{0.000000in}{-0.048611in}}{\pgfqpoint{0.000000in}{0.000000in}}{%
\pgfpathmoveto{\pgfqpoint{0.000000in}{0.000000in}}%
\pgfpathlineto{\pgfqpoint{0.000000in}{-0.048611in}}%
\pgfusepath{stroke,fill}%
}%
\begin{pgfscope}%
\pgfsys@transformshift{2.920853in}{0.499444in}%
\pgfsys@useobject{currentmarker}{}%
\end{pgfscope}%
\end{pgfscope}%
\begin{pgfscope}%
\definecolor{textcolor}{rgb}{0.000000,0.000000,0.000000}%
\pgfsetstrokecolor{textcolor}%
\pgfsetfillcolor{textcolor}%
\pgftext[x=2.920853in,y=0.402222in,,top]{\color{textcolor}\rmfamily\fontsize{10.000000}{12.000000}\selectfont \(\displaystyle {1.00}\)}%
\end{pgfscope}%
\begin{pgfscope}%
\definecolor{textcolor}{rgb}{0.000000,0.000000,0.000000}%
\pgfsetstrokecolor{textcolor}%
\pgfsetfillcolor{textcolor}%
\pgftext[x=1.793581in,y=0.223333in,,top]{\color{textcolor}\rmfamily\fontsize{10.000000}{12.000000}\selectfont False positive rate}%
\end{pgfscope}%
\begin{pgfscope}%
\pgfsetbuttcap%
\pgfsetroundjoin%
\definecolor{currentfill}{rgb}{0.000000,0.000000,0.000000}%
\pgfsetfillcolor{currentfill}%
\pgfsetlinewidth{0.803000pt}%
\definecolor{currentstroke}{rgb}{0.000000,0.000000,0.000000}%
\pgfsetstrokecolor{currentstroke}%
\pgfsetdash{}{0pt}%
\pgfsys@defobject{currentmarker}{\pgfqpoint{-0.048611in}{0.000000in}}{\pgfqpoint{-0.000000in}{0.000000in}}{%
\pgfpathmoveto{\pgfqpoint{-0.000000in}{0.000000in}}%
\pgfpathlineto{\pgfqpoint{-0.048611in}{0.000000in}}%
\pgfusepath{stroke,fill}%
}%
\begin{pgfscope}%
\pgfsys@transformshift{0.553581in}{0.583444in}%
\pgfsys@useobject{currentmarker}{}%
\end{pgfscope}%
\end{pgfscope}%
\begin{pgfscope}%
\definecolor{textcolor}{rgb}{0.000000,0.000000,0.000000}%
\pgfsetstrokecolor{textcolor}%
\pgfsetfillcolor{textcolor}%
\pgftext[x=0.278889in, y=0.535250in, left, base]{\color{textcolor}\rmfamily\fontsize{10.000000}{12.000000}\selectfont \(\displaystyle {0.0}\)}%
\end{pgfscope}%
\begin{pgfscope}%
\pgfsetbuttcap%
\pgfsetroundjoin%
\definecolor{currentfill}{rgb}{0.000000,0.000000,0.000000}%
\pgfsetfillcolor{currentfill}%
\pgfsetlinewidth{0.803000pt}%
\definecolor{currentstroke}{rgb}{0.000000,0.000000,0.000000}%
\pgfsetstrokecolor{currentstroke}%
\pgfsetdash{}{0pt}%
\pgfsys@defobject{currentmarker}{\pgfqpoint{-0.048611in}{0.000000in}}{\pgfqpoint{-0.000000in}{0.000000in}}{%
\pgfpathmoveto{\pgfqpoint{-0.000000in}{0.000000in}}%
\pgfpathlineto{\pgfqpoint{-0.048611in}{0.000000in}}%
\pgfusepath{stroke,fill}%
}%
\begin{pgfscope}%
\pgfsys@transformshift{0.553581in}{0.919444in}%
\pgfsys@useobject{currentmarker}{}%
\end{pgfscope}%
\end{pgfscope}%
\begin{pgfscope}%
\definecolor{textcolor}{rgb}{0.000000,0.000000,0.000000}%
\pgfsetstrokecolor{textcolor}%
\pgfsetfillcolor{textcolor}%
\pgftext[x=0.278889in, y=0.871250in, left, base]{\color{textcolor}\rmfamily\fontsize{10.000000}{12.000000}\selectfont \(\displaystyle {0.2}\)}%
\end{pgfscope}%
\begin{pgfscope}%
\pgfsetbuttcap%
\pgfsetroundjoin%
\definecolor{currentfill}{rgb}{0.000000,0.000000,0.000000}%
\pgfsetfillcolor{currentfill}%
\pgfsetlinewidth{0.803000pt}%
\definecolor{currentstroke}{rgb}{0.000000,0.000000,0.000000}%
\pgfsetstrokecolor{currentstroke}%
\pgfsetdash{}{0pt}%
\pgfsys@defobject{currentmarker}{\pgfqpoint{-0.048611in}{0.000000in}}{\pgfqpoint{-0.000000in}{0.000000in}}{%
\pgfpathmoveto{\pgfqpoint{-0.000000in}{0.000000in}}%
\pgfpathlineto{\pgfqpoint{-0.048611in}{0.000000in}}%
\pgfusepath{stroke,fill}%
}%
\begin{pgfscope}%
\pgfsys@transformshift{0.553581in}{1.255444in}%
\pgfsys@useobject{currentmarker}{}%
\end{pgfscope}%
\end{pgfscope}%
\begin{pgfscope}%
\definecolor{textcolor}{rgb}{0.000000,0.000000,0.000000}%
\pgfsetstrokecolor{textcolor}%
\pgfsetfillcolor{textcolor}%
\pgftext[x=0.278889in, y=1.207250in, left, base]{\color{textcolor}\rmfamily\fontsize{10.000000}{12.000000}\selectfont \(\displaystyle {0.4}\)}%
\end{pgfscope}%
\begin{pgfscope}%
\pgfsetbuttcap%
\pgfsetroundjoin%
\definecolor{currentfill}{rgb}{0.000000,0.000000,0.000000}%
\pgfsetfillcolor{currentfill}%
\pgfsetlinewidth{0.803000pt}%
\definecolor{currentstroke}{rgb}{0.000000,0.000000,0.000000}%
\pgfsetstrokecolor{currentstroke}%
\pgfsetdash{}{0pt}%
\pgfsys@defobject{currentmarker}{\pgfqpoint{-0.048611in}{0.000000in}}{\pgfqpoint{-0.000000in}{0.000000in}}{%
\pgfpathmoveto{\pgfqpoint{-0.000000in}{0.000000in}}%
\pgfpathlineto{\pgfqpoint{-0.048611in}{0.000000in}}%
\pgfusepath{stroke,fill}%
}%
\begin{pgfscope}%
\pgfsys@transformshift{0.553581in}{1.591444in}%
\pgfsys@useobject{currentmarker}{}%
\end{pgfscope}%
\end{pgfscope}%
\begin{pgfscope}%
\definecolor{textcolor}{rgb}{0.000000,0.000000,0.000000}%
\pgfsetstrokecolor{textcolor}%
\pgfsetfillcolor{textcolor}%
\pgftext[x=0.278889in, y=1.543250in, left, base]{\color{textcolor}\rmfamily\fontsize{10.000000}{12.000000}\selectfont \(\displaystyle {0.6}\)}%
\end{pgfscope}%
\begin{pgfscope}%
\pgfsetbuttcap%
\pgfsetroundjoin%
\definecolor{currentfill}{rgb}{0.000000,0.000000,0.000000}%
\pgfsetfillcolor{currentfill}%
\pgfsetlinewidth{0.803000pt}%
\definecolor{currentstroke}{rgb}{0.000000,0.000000,0.000000}%
\pgfsetstrokecolor{currentstroke}%
\pgfsetdash{}{0pt}%
\pgfsys@defobject{currentmarker}{\pgfqpoint{-0.048611in}{0.000000in}}{\pgfqpoint{-0.000000in}{0.000000in}}{%
\pgfpathmoveto{\pgfqpoint{-0.000000in}{0.000000in}}%
\pgfpathlineto{\pgfqpoint{-0.048611in}{0.000000in}}%
\pgfusepath{stroke,fill}%
}%
\begin{pgfscope}%
\pgfsys@transformshift{0.553581in}{1.927444in}%
\pgfsys@useobject{currentmarker}{}%
\end{pgfscope}%
\end{pgfscope}%
\begin{pgfscope}%
\definecolor{textcolor}{rgb}{0.000000,0.000000,0.000000}%
\pgfsetstrokecolor{textcolor}%
\pgfsetfillcolor{textcolor}%
\pgftext[x=0.278889in, y=1.879250in, left, base]{\color{textcolor}\rmfamily\fontsize{10.000000}{12.000000}\selectfont \(\displaystyle {0.8}\)}%
\end{pgfscope}%
\begin{pgfscope}%
\pgfsetbuttcap%
\pgfsetroundjoin%
\definecolor{currentfill}{rgb}{0.000000,0.000000,0.000000}%
\pgfsetfillcolor{currentfill}%
\pgfsetlinewidth{0.803000pt}%
\definecolor{currentstroke}{rgb}{0.000000,0.000000,0.000000}%
\pgfsetstrokecolor{currentstroke}%
\pgfsetdash{}{0pt}%
\pgfsys@defobject{currentmarker}{\pgfqpoint{-0.048611in}{0.000000in}}{\pgfqpoint{-0.000000in}{0.000000in}}{%
\pgfpathmoveto{\pgfqpoint{-0.000000in}{0.000000in}}%
\pgfpathlineto{\pgfqpoint{-0.048611in}{0.000000in}}%
\pgfusepath{stroke,fill}%
}%
\begin{pgfscope}%
\pgfsys@transformshift{0.553581in}{2.263444in}%
\pgfsys@useobject{currentmarker}{}%
\end{pgfscope}%
\end{pgfscope}%
\begin{pgfscope}%
\definecolor{textcolor}{rgb}{0.000000,0.000000,0.000000}%
\pgfsetstrokecolor{textcolor}%
\pgfsetfillcolor{textcolor}%
\pgftext[x=0.278889in, y=2.215250in, left, base]{\color{textcolor}\rmfamily\fontsize{10.000000}{12.000000}\selectfont \(\displaystyle {1.0}\)}%
\end{pgfscope}%
\begin{pgfscope}%
\definecolor{textcolor}{rgb}{0.000000,0.000000,0.000000}%
\pgfsetstrokecolor{textcolor}%
\pgfsetfillcolor{textcolor}%
\pgftext[x=0.223333in,y=1.423444in,,bottom,rotate=90.000000]{\color{textcolor}\rmfamily\fontsize{10.000000}{12.000000}\selectfont True positive rate}%
\end{pgfscope}%
\begin{pgfscope}%
\pgfpathrectangle{\pgfqpoint{0.553581in}{0.499444in}}{\pgfqpoint{2.480000in}{1.848000in}}%
\pgfusepath{clip}%
\pgfsetbuttcap%
\pgfsetroundjoin%
\pgfsetlinewidth{1.505625pt}%
\definecolor{currentstroke}{rgb}{0.000000,0.000000,0.000000}%
\pgfsetstrokecolor{currentstroke}%
\pgfsetdash{{5.550000pt}{2.400000pt}}{0.000000pt}%
\pgfpathmoveto{\pgfqpoint{0.666308in}{0.583444in}}%
\pgfpathlineto{\pgfqpoint{2.920853in}{2.263444in}}%
\pgfusepath{stroke}%
\end{pgfscope}%
\begin{pgfscope}%
\pgfpathrectangle{\pgfqpoint{0.553581in}{0.499444in}}{\pgfqpoint{2.480000in}{1.848000in}}%
\pgfusepath{clip}%
\pgfsetrectcap%
\pgfsetroundjoin%
\pgfsetlinewidth{1.505625pt}%
\definecolor{currentstroke}{rgb}{0.121569,0.466667,0.705882}%
\pgfsetstrokecolor{currentstroke}%
\pgfsetdash{}{0pt}%
\pgfpathmoveto{\pgfqpoint{0.666308in}{0.583444in}}%
\pgfpathlineto{\pgfqpoint{0.669652in}{0.584032in}}%
\pgfpathlineto{\pgfqpoint{0.670742in}{0.598648in}}%
\pgfpathlineto{\pgfqpoint{0.671944in}{0.599740in}}%
\pgfpathlineto{\pgfqpoint{0.672996in}{0.602596in}}%
\pgfpathlineto{\pgfqpoint{0.673879in}{0.603688in}}%
\pgfpathlineto{\pgfqpoint{0.674988in}{0.605956in}}%
\pgfpathlineto{\pgfqpoint{0.675270in}{0.606712in}}%
\pgfpathlineto{\pgfqpoint{0.676378in}{0.610072in}}%
\pgfpathlineto{\pgfqpoint{0.676716in}{0.610912in}}%
\pgfpathlineto{\pgfqpoint{0.677825in}{0.615952in}}%
\pgfpathlineto{\pgfqpoint{0.678182in}{0.617044in}}%
\pgfpathlineto{\pgfqpoint{0.679253in}{0.620656in}}%
\pgfpathlineto{\pgfqpoint{0.679798in}{0.621748in}}%
\pgfpathlineto{\pgfqpoint{0.680906in}{0.627460in}}%
\pgfpathlineto{\pgfqpoint{0.681056in}{0.628552in}}%
\pgfpathlineto{\pgfqpoint{0.682146in}{0.634012in}}%
\pgfpathlineto{\pgfqpoint{0.682428in}{0.634936in}}%
\pgfpathlineto{\pgfqpoint{0.683461in}{0.641152in}}%
\pgfpathlineto{\pgfqpoint{0.683950in}{0.642244in}}%
\pgfpathlineto{\pgfqpoint{0.685058in}{0.648628in}}%
\pgfpathlineto{\pgfqpoint{0.685340in}{0.649636in}}%
\pgfpathlineto{\pgfqpoint{0.686430in}{0.659296in}}%
\pgfpathlineto{\pgfqpoint{0.686524in}{0.660136in}}%
\pgfpathlineto{\pgfqpoint{0.687613in}{0.664336in}}%
\pgfpathlineto{\pgfqpoint{0.687820in}{0.665092in}}%
\pgfpathlineto{\pgfqpoint{0.688929in}{0.673240in}}%
\pgfpathlineto{\pgfqpoint{0.689135in}{0.674332in}}%
\pgfpathlineto{\pgfqpoint{0.690225in}{0.681472in}}%
\pgfpathlineto{\pgfqpoint{0.690695in}{0.682564in}}%
\pgfpathlineto{\pgfqpoint{0.691784in}{0.690376in}}%
\pgfpathlineto{\pgfqpoint{0.691935in}{0.691300in}}%
\pgfpathlineto{\pgfqpoint{0.693024in}{0.698272in}}%
\pgfpathlineto{\pgfqpoint{0.693175in}{0.699196in}}%
\pgfpathlineto{\pgfqpoint{0.694264in}{0.708016in}}%
\pgfpathlineto{\pgfqpoint{0.694471in}{0.709108in}}%
\pgfpathlineto{\pgfqpoint{0.695579in}{0.716752in}}%
\pgfpathlineto{\pgfqpoint{0.695711in}{0.717844in}}%
\pgfpathlineto{\pgfqpoint{0.696819in}{0.726244in}}%
\pgfpathlineto{\pgfqpoint{0.697045in}{0.727252in}}%
\pgfpathlineto{\pgfqpoint{0.698153in}{0.736912in}}%
\pgfpathlineto{\pgfqpoint{0.698229in}{0.737752in}}%
\pgfpathlineto{\pgfqpoint{0.699318in}{0.748168in}}%
\pgfpathlineto{\pgfqpoint{0.699469in}{0.749092in}}%
\pgfpathlineto{\pgfqpoint{0.700558in}{0.755728in}}%
\pgfpathlineto{\pgfqpoint{0.700746in}{0.756484in}}%
\pgfpathlineto{\pgfqpoint{0.701855in}{0.763708in}}%
\pgfpathlineto{\pgfqpoint{0.701930in}{0.764548in}}%
\pgfpathlineto{\pgfqpoint{0.703038in}{0.772948in}}%
\pgfpathlineto{\pgfqpoint{0.703207in}{0.774040in}}%
\pgfpathlineto{\pgfqpoint{0.704241in}{0.780844in}}%
\pgfpathlineto{\pgfqpoint{0.704504in}{0.781936in}}%
\pgfpathlineto{\pgfqpoint{0.705612in}{0.790672in}}%
\pgfpathlineto{\pgfqpoint{0.705781in}{0.791680in}}%
\pgfpathlineto{\pgfqpoint{0.706890in}{0.800836in}}%
\pgfpathlineto{\pgfqpoint{0.707040in}{0.801760in}}%
\pgfpathlineto{\pgfqpoint{0.708149in}{0.810496in}}%
\pgfpathlineto{\pgfqpoint{0.708243in}{0.811588in}}%
\pgfpathlineto{\pgfqpoint{0.709351in}{0.820156in}}%
\pgfpathlineto{\pgfqpoint{0.709520in}{0.821248in}}%
\pgfpathlineto{\pgfqpoint{0.710629in}{0.829564in}}%
\pgfpathlineto{\pgfqpoint{0.710892in}{0.830656in}}%
\pgfpathlineto{\pgfqpoint{0.712000in}{0.840316in}}%
\pgfpathlineto{\pgfqpoint{0.712113in}{0.841408in}}%
\pgfpathlineto{\pgfqpoint{0.713203in}{0.847540in}}%
\pgfpathlineto{\pgfqpoint{0.713296in}{0.848632in}}%
\pgfpathlineto{\pgfqpoint{0.714292in}{0.856444in}}%
\pgfpathlineto{\pgfqpoint{0.714593in}{0.857200in}}%
\pgfpathlineto{\pgfqpoint{0.715701in}{0.866944in}}%
\pgfpathlineto{\pgfqpoint{0.715852in}{0.867868in}}%
\pgfpathlineto{\pgfqpoint{0.716960in}{0.876520in}}%
\pgfpathlineto{\pgfqpoint{0.717092in}{0.877528in}}%
\pgfpathlineto{\pgfqpoint{0.718200in}{0.886936in}}%
\pgfpathlineto{\pgfqpoint{0.718407in}{0.887860in}}%
\pgfpathlineto{\pgfqpoint{0.719515in}{0.898108in}}%
\pgfpathlineto{\pgfqpoint{0.719609in}{0.899032in}}%
\pgfpathlineto{\pgfqpoint{0.720718in}{0.908440in}}%
\pgfpathlineto{\pgfqpoint{0.720812in}{0.909448in}}%
\pgfpathlineto{\pgfqpoint{0.721901in}{0.918772in}}%
\pgfpathlineto{\pgfqpoint{0.722108in}{0.919864in}}%
\pgfpathlineto{\pgfqpoint{0.723198in}{0.927676in}}%
\pgfpathlineto{\pgfqpoint{0.723273in}{0.928600in}}%
\pgfpathlineto{\pgfqpoint{0.724381in}{0.935320in}}%
\pgfpathlineto{\pgfqpoint{0.724588in}{0.936412in}}%
\pgfpathlineto{\pgfqpoint{0.725696in}{0.944560in}}%
\pgfpathlineto{\pgfqpoint{0.725903in}{0.945652in}}%
\pgfpathlineto{\pgfqpoint{0.726974in}{0.952624in}}%
\pgfpathlineto{\pgfqpoint{0.727218in}{0.953296in}}%
\pgfpathlineto{\pgfqpoint{0.728308in}{0.963964in}}%
\pgfpathlineto{\pgfqpoint{0.728515in}{0.964972in}}%
\pgfpathlineto{\pgfqpoint{0.729623in}{0.972616in}}%
\pgfpathlineto{\pgfqpoint{0.729811in}{0.973624in}}%
\pgfpathlineto{\pgfqpoint{0.730919in}{0.980008in}}%
\pgfpathlineto{\pgfqpoint{0.731220in}{0.980932in}}%
\pgfpathlineto{\pgfqpoint{0.732291in}{0.986980in}}%
\pgfpathlineto{\pgfqpoint{0.732498in}{0.987736in}}%
\pgfpathlineto{\pgfqpoint{0.733606in}{0.995044in}}%
\pgfpathlineto{\pgfqpoint{0.733775in}{0.995968in}}%
\pgfpathlineto{\pgfqpoint{0.734865in}{1.005040in}}%
\pgfpathlineto{\pgfqpoint{0.735053in}{1.005964in}}%
\pgfpathlineto{\pgfqpoint{0.736124in}{1.012516in}}%
\pgfpathlineto{\pgfqpoint{0.736424in}{1.013608in}}%
\pgfpathlineto{\pgfqpoint{0.737514in}{1.022176in}}%
\pgfpathlineto{\pgfqpoint{0.737664in}{1.023268in}}%
\pgfpathlineto{\pgfqpoint{0.738754in}{1.029736in}}%
\pgfpathlineto{\pgfqpoint{0.738998in}{1.030660in}}%
\pgfpathlineto{\pgfqpoint{0.740107in}{1.039480in}}%
\pgfpathlineto{\pgfqpoint{0.740389in}{1.040572in}}%
\pgfpathlineto{\pgfqpoint{0.741441in}{1.046704in}}%
\pgfpathlineto{\pgfqpoint{0.741704in}{1.047796in}}%
\pgfpathlineto{\pgfqpoint{0.742793in}{1.055440in}}%
\pgfpathlineto{\pgfqpoint{0.743188in}{1.056532in}}%
\pgfpathlineto{\pgfqpoint{0.744296in}{1.064680in}}%
\pgfpathlineto{\pgfqpoint{0.744409in}{1.065772in}}%
\pgfpathlineto{\pgfqpoint{0.745518in}{1.073416in}}%
\pgfpathlineto{\pgfqpoint{0.745893in}{1.074340in}}%
\pgfpathlineto{\pgfqpoint{0.747002in}{1.081732in}}%
\pgfpathlineto{\pgfqpoint{0.747115in}{1.082572in}}%
\pgfpathlineto{\pgfqpoint{0.748129in}{1.089292in}}%
\pgfpathlineto{\pgfqpoint{0.748768in}{1.090384in}}%
\pgfpathlineto{\pgfqpoint{0.749876in}{1.098112in}}%
\pgfpathlineto{\pgfqpoint{0.750008in}{1.099120in}}%
\pgfpathlineto{\pgfqpoint{0.751116in}{1.104328in}}%
\pgfpathlineto{\pgfqpoint{0.751267in}{1.105084in}}%
\pgfpathlineto{\pgfqpoint{0.752375in}{1.110460in}}%
\pgfpathlineto{\pgfqpoint{0.752676in}{1.111468in}}%
\pgfpathlineto{\pgfqpoint{0.753766in}{1.116844in}}%
\pgfpathlineto{\pgfqpoint{0.753935in}{1.117852in}}%
\pgfpathlineto{\pgfqpoint{0.755043in}{1.123984in}}%
\pgfpathlineto{\pgfqpoint{0.755456in}{1.125076in}}%
\pgfpathlineto{\pgfqpoint{0.756565in}{1.132132in}}%
\pgfpathlineto{\pgfqpoint{0.756922in}{1.133224in}}%
\pgfpathlineto{\pgfqpoint{0.758012in}{1.139440in}}%
\pgfpathlineto{\pgfqpoint{0.758406in}{1.140532in}}%
\pgfpathlineto{\pgfqpoint{0.759515in}{1.147924in}}%
\pgfpathlineto{\pgfqpoint{0.759609in}{1.148512in}}%
\pgfpathlineto{\pgfqpoint{0.760717in}{1.155652in}}%
\pgfpathlineto{\pgfqpoint{0.760867in}{1.156744in}}%
\pgfpathlineto{\pgfqpoint{0.761976in}{1.162456in}}%
\pgfpathlineto{\pgfqpoint{0.762201in}{1.163380in}}%
\pgfpathlineto{\pgfqpoint{0.763310in}{1.168504in}}%
\pgfpathlineto{\pgfqpoint{0.763498in}{1.169512in}}%
\pgfpathlineto{\pgfqpoint{0.764606in}{1.177660in}}%
\pgfpathlineto{\pgfqpoint{0.764926in}{1.178752in}}%
\pgfpathlineto{\pgfqpoint{0.766015in}{1.184548in}}%
\pgfpathlineto{\pgfqpoint{0.766241in}{1.185556in}}%
\pgfpathlineto{\pgfqpoint{0.767349in}{1.190596in}}%
\pgfpathlineto{\pgfqpoint{0.767763in}{1.191688in}}%
\pgfpathlineto{\pgfqpoint{0.768833in}{1.197148in}}%
\pgfpathlineto{\pgfqpoint{0.769115in}{1.198156in}}%
\pgfpathlineto{\pgfqpoint{0.770186in}{1.202692in}}%
\pgfpathlineto{\pgfqpoint{0.770562in}{1.203784in}}%
\pgfpathlineto{\pgfqpoint{0.771670in}{1.208656in}}%
\pgfpathlineto{\pgfqpoint{0.771802in}{1.209748in}}%
\pgfpathlineto{\pgfqpoint{0.772892in}{1.216216in}}%
\pgfpathlineto{\pgfqpoint{0.773136in}{1.217308in}}%
\pgfpathlineto{\pgfqpoint{0.774244in}{1.222684in}}%
\pgfpathlineto{\pgfqpoint{0.774376in}{1.223692in}}%
\pgfpathlineto{\pgfqpoint{0.775484in}{1.228732in}}%
\pgfpathlineto{\pgfqpoint{0.775541in}{1.229236in}}%
\pgfpathlineto{\pgfqpoint{0.776612in}{1.233856in}}%
\pgfpathlineto{\pgfqpoint{0.776950in}{1.234864in}}%
\pgfpathlineto{\pgfqpoint{0.778039in}{1.240492in}}%
\pgfpathlineto{\pgfqpoint{0.778227in}{1.241416in}}%
\pgfpathlineto{\pgfqpoint{0.779336in}{1.246120in}}%
\pgfpathlineto{\pgfqpoint{0.779580in}{1.247212in}}%
\pgfpathlineto{\pgfqpoint{0.780670in}{1.252504in}}%
\pgfpathlineto{\pgfqpoint{0.781008in}{1.253428in}}%
\pgfpathlineto{\pgfqpoint{0.782098in}{1.259392in}}%
\pgfpathlineto{\pgfqpoint{0.782417in}{1.260400in}}%
\pgfpathlineto{\pgfqpoint{0.783526in}{1.263424in}}%
\pgfpathlineto{\pgfqpoint{0.783676in}{1.264180in}}%
\pgfpathlineto{\pgfqpoint{0.784709in}{1.269808in}}%
\pgfpathlineto{\pgfqpoint{0.784991in}{1.270564in}}%
\pgfpathlineto{\pgfqpoint{0.786081in}{1.276276in}}%
\pgfpathlineto{\pgfqpoint{0.786475in}{1.277368in}}%
\pgfpathlineto{\pgfqpoint{0.787565in}{1.280560in}}%
\pgfpathlineto{\pgfqpoint{0.787828in}{1.281484in}}%
\pgfpathlineto{\pgfqpoint{0.788936in}{1.286524in}}%
\pgfpathlineto{\pgfqpoint{0.789312in}{1.287448in}}%
\pgfpathlineto{\pgfqpoint{0.790421in}{1.294168in}}%
\pgfpathlineto{\pgfqpoint{0.790740in}{1.295176in}}%
\pgfpathlineto{\pgfqpoint{0.791849in}{1.300972in}}%
\pgfpathlineto{\pgfqpoint{0.792093in}{1.301812in}}%
\pgfpathlineto{\pgfqpoint{0.793089in}{1.305592in}}%
\pgfpathlineto{\pgfqpoint{0.793784in}{1.306600in}}%
\pgfpathlineto{\pgfqpoint{0.794892in}{1.310548in}}%
\pgfpathlineto{\pgfqpoint{0.795230in}{1.311640in}}%
\pgfpathlineto{\pgfqpoint{0.796339in}{1.315924in}}%
\pgfpathlineto{\pgfqpoint{0.796639in}{1.317016in}}%
\pgfpathlineto{\pgfqpoint{0.797748in}{1.322392in}}%
\pgfpathlineto{\pgfqpoint{0.798049in}{1.323400in}}%
\pgfpathlineto{\pgfqpoint{0.799138in}{1.327432in}}%
\pgfpathlineto{\pgfqpoint{0.799458in}{1.328524in}}%
\pgfpathlineto{\pgfqpoint{0.800529in}{1.333060in}}%
\pgfpathlineto{\pgfqpoint{0.800848in}{1.333900in}}%
\pgfpathlineto{\pgfqpoint{0.801938in}{1.338604in}}%
\pgfpathlineto{\pgfqpoint{0.802313in}{1.339696in}}%
\pgfpathlineto{\pgfqpoint{0.803384in}{1.343728in}}%
\pgfpathlineto{\pgfqpoint{0.803723in}{1.344820in}}%
\pgfpathlineto{\pgfqpoint{0.804812in}{1.347676in}}%
\pgfpathlineto{\pgfqpoint{0.805038in}{1.348600in}}%
\pgfpathlineto{\pgfqpoint{0.806127in}{1.355152in}}%
\pgfpathlineto{\pgfqpoint{0.806447in}{1.356160in}}%
\pgfpathlineto{\pgfqpoint{0.807536in}{1.361368in}}%
\pgfpathlineto{\pgfqpoint{0.807987in}{1.362376in}}%
\pgfpathlineto{\pgfqpoint{0.809077in}{1.366408in}}%
\pgfpathlineto{\pgfqpoint{0.809490in}{1.367500in}}%
\pgfpathlineto{\pgfqpoint{0.810561in}{1.370272in}}%
\pgfpathlineto{\pgfqpoint{0.810881in}{1.371280in}}%
\pgfpathlineto{\pgfqpoint{0.811952in}{1.374724in}}%
\pgfpathlineto{\pgfqpoint{0.812346in}{1.375732in}}%
\pgfpathlineto{\pgfqpoint{0.813455in}{1.378756in}}%
\pgfpathlineto{\pgfqpoint{0.813774in}{1.379764in}}%
\pgfpathlineto{\pgfqpoint{0.814807in}{1.382872in}}%
\pgfpathlineto{\pgfqpoint{0.815183in}{1.383964in}}%
\pgfpathlineto{\pgfqpoint{0.816273in}{1.388248in}}%
\pgfpathlineto{\pgfqpoint{0.816705in}{1.389340in}}%
\pgfpathlineto{\pgfqpoint{0.817795in}{1.393624in}}%
\pgfpathlineto{\pgfqpoint{0.818133in}{1.394632in}}%
\pgfpathlineto{\pgfqpoint{0.819241in}{1.398580in}}%
\pgfpathlineto{\pgfqpoint{0.819692in}{1.399672in}}%
\pgfpathlineto{\pgfqpoint{0.820763in}{1.402192in}}%
\pgfpathlineto{\pgfqpoint{0.821176in}{1.403284in}}%
\pgfpathlineto{\pgfqpoint{0.822210in}{1.407148in}}%
\pgfpathlineto{\pgfqpoint{0.822792in}{1.408240in}}%
\pgfpathlineto{\pgfqpoint{0.823901in}{1.411768in}}%
\pgfpathlineto{\pgfqpoint{0.824276in}{1.412860in}}%
\pgfpathlineto{\pgfqpoint{0.825329in}{1.417900in}}%
\pgfpathlineto{\pgfqpoint{0.825648in}{1.418908in}}%
\pgfpathlineto{\pgfqpoint{0.826663in}{1.421428in}}%
\pgfpathlineto{\pgfqpoint{0.827038in}{1.422352in}}%
\pgfpathlineto{\pgfqpoint{0.828147in}{1.424620in}}%
\pgfpathlineto{\pgfqpoint{0.828391in}{1.425712in}}%
\pgfpathlineto{\pgfqpoint{0.829499in}{1.429744in}}%
\pgfpathlineto{\pgfqpoint{0.830026in}{1.430752in}}%
\pgfpathlineto{\pgfqpoint{0.831021in}{1.434196in}}%
\pgfpathlineto{\pgfqpoint{0.831453in}{1.435288in}}%
\pgfpathlineto{\pgfqpoint{0.832562in}{1.439152in}}%
\pgfpathlineto{\pgfqpoint{0.833032in}{1.440076in}}%
\pgfpathlineto{\pgfqpoint{0.834121in}{1.443520in}}%
\pgfpathlineto{\pgfqpoint{0.834553in}{1.444612in}}%
\pgfpathlineto{\pgfqpoint{0.835662in}{1.448056in}}%
\pgfpathlineto{\pgfqpoint{0.836113in}{1.448980in}}%
\pgfpathlineto{\pgfqpoint{0.837203in}{1.452256in}}%
\pgfpathlineto{\pgfqpoint{0.837484in}{1.453264in}}%
\pgfpathlineto{\pgfqpoint{0.838593in}{1.456036in}}%
\pgfpathlineto{\pgfqpoint{0.838969in}{1.457044in}}%
\pgfpathlineto{\pgfqpoint{0.840077in}{1.461076in}}%
\pgfpathlineto{\pgfqpoint{0.840453in}{1.462168in}}%
\pgfpathlineto{\pgfqpoint{0.841505in}{1.465108in}}%
\pgfpathlineto{\pgfqpoint{0.842163in}{1.466116in}}%
\pgfpathlineto{\pgfqpoint{0.843233in}{1.469560in}}%
\pgfpathlineto{\pgfqpoint{0.843741in}{1.470484in}}%
\pgfpathlineto{\pgfqpoint{0.844849in}{1.474852in}}%
\pgfpathlineto{\pgfqpoint{0.845244in}{1.475692in}}%
\pgfpathlineto{\pgfqpoint{0.846352in}{1.479388in}}%
\pgfpathlineto{\pgfqpoint{0.846916in}{1.480396in}}%
\pgfpathlineto{\pgfqpoint{0.848024in}{1.482580in}}%
\pgfpathlineto{\pgfqpoint{0.848325in}{1.483672in}}%
\pgfpathlineto{\pgfqpoint{0.849433in}{1.488208in}}%
\pgfpathlineto{\pgfqpoint{0.849772in}{1.489300in}}%
\pgfpathlineto{\pgfqpoint{0.850843in}{1.492324in}}%
\pgfpathlineto{\pgfqpoint{0.851387in}{1.493332in}}%
\pgfpathlineto{\pgfqpoint{0.852364in}{1.496020in}}%
\pgfpathlineto{\pgfqpoint{0.853041in}{1.497112in}}%
\pgfpathlineto{\pgfqpoint{0.854055in}{1.499464in}}%
\pgfpathlineto{\pgfqpoint{0.854563in}{1.500556in}}%
\pgfpathlineto{\pgfqpoint{0.855671in}{1.504252in}}%
\pgfpathlineto{\pgfqpoint{0.856122in}{1.505344in}}%
\pgfpathlineto{\pgfqpoint{0.857230in}{1.509124in}}%
\pgfpathlineto{\pgfqpoint{0.857700in}{1.510216in}}%
\pgfpathlineto{\pgfqpoint{0.858771in}{1.511812in}}%
\pgfpathlineto{\pgfqpoint{0.859372in}{1.512904in}}%
\pgfpathlineto{\pgfqpoint{0.859466in}{1.513576in}}%
\pgfpathlineto{\pgfqpoint{0.876093in}{1.514668in}}%
\pgfpathlineto{\pgfqpoint{0.877164in}{1.517692in}}%
\pgfpathlineto{\pgfqpoint{0.877615in}{1.518784in}}%
\pgfpathlineto{\pgfqpoint{0.878724in}{1.522060in}}%
\pgfpathlineto{\pgfqpoint{0.879137in}{1.523068in}}%
\pgfpathlineto{\pgfqpoint{0.880227in}{1.527268in}}%
\pgfpathlineto{\pgfqpoint{0.880565in}{1.528276in}}%
\pgfpathlineto{\pgfqpoint{0.881579in}{1.529872in}}%
\pgfpathlineto{\pgfqpoint{0.882181in}{1.530964in}}%
\pgfpathlineto{\pgfqpoint{0.883289in}{1.533400in}}%
\pgfpathlineto{\pgfqpoint{0.883740in}{1.534408in}}%
\pgfpathlineto{\pgfqpoint{0.884849in}{1.538020in}}%
\pgfpathlineto{\pgfqpoint{0.885318in}{1.539112in}}%
\pgfpathlineto{\pgfqpoint{0.886352in}{1.541884in}}%
\pgfpathlineto{\pgfqpoint{0.886972in}{1.542976in}}%
\pgfpathlineto{\pgfqpoint{0.888024in}{1.545496in}}%
\pgfpathlineto{\pgfqpoint{0.888587in}{1.546588in}}%
\pgfpathlineto{\pgfqpoint{0.889583in}{1.549444in}}%
\pgfpathlineto{\pgfqpoint{0.890072in}{1.550536in}}%
\pgfpathlineto{\pgfqpoint{0.891180in}{1.552552in}}%
\pgfpathlineto{\pgfqpoint{0.891725in}{1.553644in}}%
\pgfpathlineto{\pgfqpoint{0.892815in}{1.555828in}}%
\pgfpathlineto{\pgfqpoint{0.893190in}{1.556920in}}%
\pgfpathlineto{\pgfqpoint{0.894299in}{1.560028in}}%
\pgfpathlineto{\pgfqpoint{0.894524in}{1.561120in}}%
\pgfpathlineto{\pgfqpoint{0.895614in}{1.563640in}}%
\pgfpathlineto{\pgfqpoint{0.896196in}{1.564732in}}%
\pgfpathlineto{\pgfqpoint{0.897230in}{1.567420in}}%
\pgfpathlineto{\pgfqpoint{0.897906in}{1.568512in}}%
\pgfpathlineto{\pgfqpoint{0.899015in}{1.570864in}}%
\pgfpathlineto{\pgfqpoint{0.900029in}{1.571872in}}%
\pgfpathlineto{\pgfqpoint{0.901138in}{1.574308in}}%
\pgfpathlineto{\pgfqpoint{0.901495in}{1.575316in}}%
\pgfpathlineto{\pgfqpoint{0.902584in}{1.578256in}}%
\pgfpathlineto{\pgfqpoint{0.903148in}{1.579264in}}%
\pgfpathlineto{\pgfqpoint{0.904256in}{1.581952in}}%
\pgfpathlineto{\pgfqpoint{0.905064in}{1.583044in}}%
\pgfpathlineto{\pgfqpoint{0.906135in}{1.585144in}}%
\pgfpathlineto{\pgfqpoint{0.906849in}{1.586152in}}%
\pgfpathlineto{\pgfqpoint{0.907883in}{1.589344in}}%
\pgfpathlineto{\pgfqpoint{0.908596in}{1.590436in}}%
\pgfpathlineto{\pgfqpoint{0.909667in}{1.593208in}}%
\pgfpathlineto{\pgfqpoint{0.910456in}{1.594216in}}%
\pgfpathlineto{\pgfqpoint{0.911565in}{1.597324in}}%
\pgfpathlineto{\pgfqpoint{0.912298in}{1.598332in}}%
\pgfpathlineto{\pgfqpoint{0.913312in}{1.600768in}}%
\pgfpathlineto{\pgfqpoint{0.913575in}{1.601776in}}%
\pgfpathlineto{\pgfqpoint{0.914684in}{1.604296in}}%
\pgfpathlineto{\pgfqpoint{0.915153in}{1.605304in}}%
\pgfpathlineto{\pgfqpoint{0.916262in}{1.607068in}}%
\pgfpathlineto{\pgfqpoint{0.916919in}{1.608160in}}%
\pgfpathlineto{\pgfqpoint{0.917972in}{1.610932in}}%
\pgfpathlineto{\pgfqpoint{0.919024in}{1.612024in}}%
\pgfpathlineto{\pgfqpoint{0.920001in}{1.614376in}}%
\pgfpathlineto{\pgfqpoint{0.921353in}{1.615468in}}%
\pgfpathlineto{\pgfqpoint{0.922274in}{1.617820in}}%
\pgfpathlineto{\pgfqpoint{0.923589in}{1.618912in}}%
\pgfpathlineto{\pgfqpoint{0.924585in}{1.620592in}}%
\pgfpathlineto{\pgfqpoint{0.925524in}{1.621684in}}%
\pgfpathlineto{\pgfqpoint{0.926576in}{1.623616in}}%
\pgfpathlineto{\pgfqpoint{0.927478in}{1.624624in}}%
\pgfpathlineto{\pgfqpoint{0.928587in}{1.627312in}}%
\pgfpathlineto{\pgfqpoint{0.929150in}{1.628320in}}%
\pgfpathlineto{\pgfqpoint{0.930240in}{1.630084in}}%
\pgfpathlineto{\pgfqpoint{0.930879in}{1.631176in}}%
\pgfpathlineto{\pgfqpoint{0.931987in}{1.633528in}}%
\pgfpathlineto{\pgfqpoint{0.932476in}{1.634368in}}%
\pgfpathlineto{\pgfqpoint{0.933547in}{1.636804in}}%
\pgfpathlineto{\pgfqpoint{0.934186in}{1.637812in}}%
\pgfpathlineto{\pgfqpoint{0.935256in}{1.639660in}}%
\pgfpathlineto{\pgfqpoint{0.935764in}{1.640668in}}%
\pgfpathlineto{\pgfqpoint{0.936722in}{1.643020in}}%
\pgfpathlineto{\pgfqpoint{0.937699in}{1.644112in}}%
\pgfpathlineto{\pgfqpoint{0.938807in}{1.645876in}}%
\pgfpathlineto{\pgfqpoint{0.939784in}{1.646968in}}%
\pgfpathlineto{\pgfqpoint{0.940818in}{1.649824in}}%
\pgfpathlineto{\pgfqpoint{0.941269in}{1.650916in}}%
\pgfpathlineto{\pgfqpoint{0.942358in}{1.653100in}}%
\pgfpathlineto{\pgfqpoint{0.943410in}{1.654192in}}%
\pgfpathlineto{\pgfqpoint{0.944463in}{1.656292in}}%
\pgfpathlineto{\pgfqpoint{0.945083in}{1.657384in}}%
\pgfpathlineto{\pgfqpoint{0.946135in}{1.659568in}}%
\pgfpathlineto{\pgfqpoint{0.946849in}{1.660660in}}%
\pgfpathlineto{\pgfqpoint{0.947919in}{1.662592in}}%
\pgfpathlineto{\pgfqpoint{0.948539in}{1.663600in}}%
\pgfpathlineto{\pgfqpoint{0.949648in}{1.665700in}}%
\pgfpathlineto{\pgfqpoint{0.950381in}{1.666792in}}%
\pgfpathlineto{\pgfqpoint{0.951470in}{1.669564in}}%
\pgfpathlineto{\pgfqpoint{0.952034in}{1.670656in}}%
\pgfpathlineto{\pgfqpoint{0.953067in}{1.672420in}}%
\pgfpathlineto{\pgfqpoint{0.953950in}{1.673428in}}%
\pgfpathlineto{\pgfqpoint{0.954890in}{1.675108in}}%
\pgfpathlineto{\pgfqpoint{0.955810in}{1.676200in}}%
\pgfpathlineto{\pgfqpoint{0.956750in}{1.677964in}}%
\pgfpathlineto{\pgfqpoint{0.957483in}{1.678888in}}%
\pgfpathlineto{\pgfqpoint{0.958591in}{1.680820in}}%
\pgfpathlineto{\pgfqpoint{0.959079in}{1.681828in}}%
\pgfpathlineto{\pgfqpoint{0.960132in}{1.684432in}}%
\pgfpathlineto{\pgfqpoint{0.961090in}{1.685524in}}%
\pgfpathlineto{\pgfqpoint{0.962086in}{1.687960in}}%
\pgfpathlineto{\pgfqpoint{0.962856in}{1.688800in}}%
\pgfpathlineto{\pgfqpoint{0.963964in}{1.690900in}}%
\pgfpathlineto{\pgfqpoint{0.964716in}{1.691908in}}%
\pgfpathlineto{\pgfqpoint{0.965618in}{1.693672in}}%
\pgfpathlineto{\pgfqpoint{0.966426in}{1.694764in}}%
\pgfpathlineto{\pgfqpoint{0.967515in}{1.696696in}}%
\pgfpathlineto{\pgfqpoint{0.968417in}{1.697788in}}%
\pgfpathlineto{\pgfqpoint{0.969507in}{1.699552in}}%
\pgfpathlineto{\pgfqpoint{0.970164in}{1.700560in}}%
\pgfpathlineto{\pgfqpoint{0.971273in}{1.703248in}}%
\pgfpathlineto{\pgfqpoint{0.971855in}{1.704340in}}%
\pgfpathlineto{\pgfqpoint{0.972964in}{1.707112in}}%
\pgfpathlineto{\pgfqpoint{0.973922in}{1.708120in}}%
\pgfpathlineto{\pgfqpoint{0.974993in}{1.710304in}}%
\pgfpathlineto{\pgfqpoint{0.975538in}{1.711312in}}%
\pgfpathlineto{\pgfqpoint{0.976609in}{1.713076in}}%
\pgfpathlineto{\pgfqpoint{0.977435in}{1.714168in}}%
\pgfpathlineto{\pgfqpoint{0.978506in}{1.716352in}}%
\pgfpathlineto{\pgfqpoint{0.979164in}{1.717444in}}%
\pgfpathlineto{\pgfqpoint{0.980272in}{1.719544in}}%
\pgfpathlineto{\pgfqpoint{0.980686in}{1.720636in}}%
\pgfpathlineto{\pgfqpoint{0.981700in}{1.721980in}}%
\pgfpathlineto{\pgfqpoint{0.982433in}{1.723072in}}%
\pgfpathlineto{\pgfqpoint{0.983541in}{1.724836in}}%
\pgfpathlineto{\pgfqpoint{0.984293in}{1.725928in}}%
\pgfpathlineto{\pgfqpoint{0.985401in}{1.728364in}}%
\pgfpathlineto{\pgfqpoint{0.986228in}{1.729372in}}%
\pgfpathlineto{\pgfqpoint{0.987336in}{1.731304in}}%
\pgfpathlineto{\pgfqpoint{0.988858in}{1.732228in}}%
\pgfpathlineto{\pgfqpoint{0.989854in}{1.734580in}}%
\pgfpathlineto{\pgfqpoint{0.990549in}{1.735672in}}%
\pgfpathlineto{\pgfqpoint{0.991451in}{1.736932in}}%
\pgfpathlineto{\pgfqpoint{0.992616in}{1.738024in}}%
\pgfpathlineto{\pgfqpoint{0.993630in}{1.739368in}}%
\pgfpathlineto{\pgfqpoint{0.994795in}{1.740460in}}%
\pgfpathlineto{\pgfqpoint{0.995885in}{1.742476in}}%
\pgfpathlineto{\pgfqpoint{0.996824in}{1.743568in}}%
\pgfpathlineto{\pgfqpoint{0.997895in}{1.746340in}}%
\pgfpathlineto{\pgfqpoint{0.998628in}{1.747264in}}%
\pgfpathlineto{\pgfqpoint{0.999473in}{1.749112in}}%
\pgfpathlineto{\pgfqpoint{1.000563in}{1.750204in}}%
\pgfpathlineto{\pgfqpoint{1.001484in}{1.751968in}}%
\pgfpathlineto{\pgfqpoint{1.002479in}{1.753060in}}%
\pgfpathlineto{\pgfqpoint{1.003569in}{1.754824in}}%
\pgfpathlineto{\pgfqpoint{1.004621in}{1.755916in}}%
\pgfpathlineto{\pgfqpoint{1.005692in}{1.757008in}}%
\pgfpathlineto{\pgfqpoint{1.006575in}{1.758100in}}%
\pgfpathlineto{\pgfqpoint{1.007477in}{1.759444in}}%
\pgfpathlineto{\pgfqpoint{1.008341in}{1.760452in}}%
\pgfpathlineto{\pgfqpoint{1.009450in}{1.761796in}}%
\pgfpathlineto{\pgfqpoint{1.010220in}{1.762888in}}%
\pgfpathlineto{\pgfqpoint{1.011272in}{1.764652in}}%
\pgfpathlineto{\pgfqpoint{1.013001in}{1.765660in}}%
\pgfpathlineto{\pgfqpoint{1.014109in}{1.767172in}}%
\pgfpathlineto{\pgfqpoint{1.015312in}{1.768264in}}%
\pgfpathlineto{\pgfqpoint{1.016420in}{1.769776in}}%
\pgfpathlineto{\pgfqpoint{1.016965in}{1.770784in}}%
\pgfpathlineto{\pgfqpoint{1.017923in}{1.772212in}}%
\pgfpathlineto{\pgfqpoint{1.018919in}{1.773304in}}%
\pgfpathlineto{\pgfqpoint{1.019746in}{1.774480in}}%
\pgfpathlineto{\pgfqpoint{1.020967in}{1.775572in}}%
\pgfpathlineto{\pgfqpoint{1.022056in}{1.776832in}}%
\pgfpathlineto{\pgfqpoint{1.022827in}{1.777756in}}%
\pgfpathlineto{\pgfqpoint{1.023916in}{1.779604in}}%
\pgfpathlineto{\pgfqpoint{1.025232in}{1.780696in}}%
\pgfpathlineto{\pgfqpoint{1.026321in}{1.782964in}}%
\pgfpathlineto{\pgfqpoint{1.027693in}{1.784056in}}%
\pgfpathlineto{\pgfqpoint{1.028783in}{1.785568in}}%
\pgfpathlineto{\pgfqpoint{1.030192in}{1.786660in}}%
\pgfpathlineto{\pgfqpoint{1.031187in}{1.787836in}}%
\pgfpathlineto{\pgfqpoint{1.032503in}{1.788928in}}%
\pgfpathlineto{\pgfqpoint{1.033536in}{1.790524in}}%
\pgfpathlineto{\pgfqpoint{1.034550in}{1.791532in}}%
\pgfpathlineto{\pgfqpoint{1.035640in}{1.792708in}}%
\pgfpathlineto{\pgfqpoint{1.036392in}{1.793800in}}%
\pgfpathlineto{\pgfqpoint{1.037181in}{1.795648in}}%
\pgfpathlineto{\pgfqpoint{1.038364in}{1.796740in}}%
\pgfpathlineto{\pgfqpoint{1.039473in}{1.798672in}}%
\pgfpathlineto{\pgfqpoint{1.040938in}{1.799680in}}%
\pgfpathlineto{\pgfqpoint{1.042028in}{1.800856in}}%
\pgfpathlineto{\pgfqpoint{1.043099in}{1.801948in}}%
\pgfpathlineto{\pgfqpoint{1.044207in}{1.803208in}}%
\pgfpathlineto{\pgfqpoint{1.045466in}{1.804300in}}%
\pgfpathlineto{\pgfqpoint{1.046575in}{1.804720in}}%
\pgfpathlineto{\pgfqpoint{1.047289in}{1.805812in}}%
\pgfpathlineto{\pgfqpoint{1.048228in}{1.806904in}}%
\pgfpathlineto{\pgfqpoint{1.049299in}{1.807996in}}%
\pgfpathlineto{\pgfqpoint{1.050389in}{1.809088in}}%
\pgfpathlineto{\pgfqpoint{1.051347in}{1.810180in}}%
\pgfpathlineto{\pgfqpoint{1.052436in}{1.811608in}}%
\pgfpathlineto{\pgfqpoint{1.052944in}{1.812700in}}%
\pgfpathlineto{\pgfqpoint{1.053770in}{1.813540in}}%
\pgfpathlineto{\pgfqpoint{1.054729in}{1.814632in}}%
\pgfpathlineto{\pgfqpoint{1.055724in}{1.815976in}}%
\pgfpathlineto{\pgfqpoint{1.056833in}{1.817068in}}%
\pgfpathlineto{\pgfqpoint{1.057923in}{1.818916in}}%
\pgfpathlineto{\pgfqpoint{1.058806in}{1.820008in}}%
\pgfpathlineto{\pgfqpoint{1.059914in}{1.821520in}}%
\pgfpathlineto{\pgfqpoint{1.060872in}{1.822612in}}%
\pgfpathlineto{\pgfqpoint{1.061943in}{1.823788in}}%
\pgfpathlineto{\pgfqpoint{1.063258in}{1.824880in}}%
\pgfpathlineto{\pgfqpoint{1.064198in}{1.825972in}}%
\pgfpathlineto{\pgfqpoint{1.065024in}{1.827064in}}%
\pgfpathlineto{\pgfqpoint{1.066001in}{1.828408in}}%
\pgfpathlineto{\pgfqpoint{1.067335in}{1.829500in}}%
\pgfpathlineto{\pgfqpoint{1.068406in}{1.830760in}}%
\pgfpathlineto{\pgfqpoint{1.069533in}{1.831852in}}%
\pgfpathlineto{\pgfqpoint{1.070360in}{1.833028in}}%
\pgfpathlineto{\pgfqpoint{1.071262in}{1.834120in}}%
\pgfpathlineto{\pgfqpoint{1.072183in}{1.834960in}}%
\pgfpathlineto{\pgfqpoint{1.073892in}{1.836052in}}%
\pgfpathlineto{\pgfqpoint{1.075001in}{1.837144in}}%
\pgfpathlineto{\pgfqpoint{1.075978in}{1.838152in}}%
\pgfpathlineto{\pgfqpoint{1.077030in}{1.839160in}}%
\pgfpathlineto{\pgfqpoint{1.078383in}{1.840168in}}%
\pgfpathlineto{\pgfqpoint{1.079040in}{1.840924in}}%
\pgfpathlineto{\pgfqpoint{1.080844in}{1.841932in}}%
\pgfpathlineto{\pgfqpoint{1.081877in}{1.842688in}}%
\pgfpathlineto{\pgfqpoint{1.082741in}{1.843696in}}%
\pgfpathlineto{\pgfqpoint{1.083850in}{1.845292in}}%
\pgfpathlineto{\pgfqpoint{1.084921in}{1.846300in}}%
\pgfpathlineto{\pgfqpoint{1.086029in}{1.847812in}}%
\pgfpathlineto{\pgfqpoint{1.087382in}{1.848904in}}%
\pgfpathlineto{\pgfqpoint{1.088265in}{1.849912in}}%
\pgfpathlineto{\pgfqpoint{1.090238in}{1.851004in}}%
\pgfpathlineto{\pgfqpoint{1.091233in}{1.851844in}}%
\pgfpathlineto{\pgfqpoint{1.092680in}{1.852768in}}%
\pgfpathlineto{\pgfqpoint{1.093582in}{1.853440in}}%
\pgfpathlineto{\pgfqpoint{1.094747in}{1.854532in}}%
\pgfpathlineto{\pgfqpoint{1.095836in}{1.855372in}}%
\pgfpathlineto{\pgfqpoint{1.097133in}{1.856464in}}%
\pgfpathlineto{\pgfqpoint{1.098241in}{1.858312in}}%
\pgfpathlineto{\pgfqpoint{1.100045in}{1.859404in}}%
\pgfpathlineto{\pgfqpoint{1.100853in}{1.860580in}}%
\pgfpathlineto{\pgfqpoint{1.102281in}{1.861588in}}%
\pgfpathlineto{\pgfqpoint{1.103295in}{1.863268in}}%
\pgfpathlineto{\pgfqpoint{1.105155in}{1.864360in}}%
\pgfpathlineto{\pgfqpoint{1.106113in}{1.865284in}}%
\pgfpathlineto{\pgfqpoint{1.107654in}{1.866376in}}%
\pgfpathlineto{\pgfqpoint{1.108744in}{1.867552in}}%
\pgfpathlineto{\pgfqpoint{1.110284in}{1.868644in}}%
\pgfpathlineto{\pgfqpoint{1.111393in}{1.870072in}}%
\pgfpathlineto{\pgfqpoint{1.112746in}{1.871164in}}%
\pgfpathlineto{\pgfqpoint{1.113835in}{1.872340in}}%
\pgfpathlineto{\pgfqpoint{1.115226in}{1.873348in}}%
\pgfpathlineto{\pgfqpoint{1.116296in}{1.874608in}}%
\pgfpathlineto{\pgfqpoint{1.118513in}{1.875700in}}%
\pgfpathlineto{\pgfqpoint{1.119622in}{1.876960in}}%
\pgfpathlineto{\pgfqpoint{1.120937in}{1.877968in}}%
\pgfpathlineto{\pgfqpoint{1.121952in}{1.878556in}}%
\pgfpathlineto{\pgfqpoint{1.123643in}{1.879648in}}%
\pgfpathlineto{\pgfqpoint{1.124695in}{1.880740in}}%
\pgfpathlineto{\pgfqpoint{1.126066in}{1.881832in}}%
\pgfpathlineto{\pgfqpoint{1.127137in}{1.883260in}}%
\pgfpathlineto{\pgfqpoint{1.128678in}{1.884268in}}%
\pgfpathlineto{\pgfqpoint{1.129786in}{1.885612in}}%
\pgfpathlineto{\pgfqpoint{1.130876in}{1.886704in}}%
\pgfpathlineto{\pgfqpoint{1.131909in}{1.887628in}}%
\pgfpathlineto{\pgfqpoint{1.133844in}{1.888720in}}%
\pgfpathlineto{\pgfqpoint{1.134878in}{1.889644in}}%
\pgfpathlineto{\pgfqpoint{1.136850in}{1.890736in}}%
\pgfpathlineto{\pgfqpoint{1.137903in}{1.891576in}}%
\pgfpathlineto{\pgfqpoint{1.139349in}{1.892668in}}%
\pgfpathlineto{\pgfqpoint{1.140458in}{1.893676in}}%
\pgfpathlineto{\pgfqpoint{1.142449in}{1.894768in}}%
\pgfpathlineto{\pgfqpoint{1.143520in}{1.896448in}}%
\pgfpathlineto{\pgfqpoint{1.145549in}{1.897540in}}%
\pgfpathlineto{\pgfqpoint{1.146601in}{1.898380in}}%
\pgfpathlineto{\pgfqpoint{1.148273in}{1.899472in}}%
\pgfpathlineto{\pgfqpoint{1.149363in}{1.900144in}}%
\pgfpathlineto{\pgfqpoint{1.150472in}{1.901236in}}%
\pgfpathlineto{\pgfqpoint{1.151505in}{1.901824in}}%
\pgfpathlineto{\pgfqpoint{1.153064in}{1.902916in}}%
\pgfpathlineto{\pgfqpoint{1.154060in}{1.903924in}}%
\pgfpathlineto{\pgfqpoint{1.155826in}{1.904932in}}%
\pgfpathlineto{\pgfqpoint{1.156916in}{1.905940in}}%
\pgfpathlineto{\pgfqpoint{1.158043in}{1.907032in}}%
\pgfpathlineto{\pgfqpoint{1.159133in}{1.908124in}}%
\pgfpathlineto{\pgfqpoint{1.161030in}{1.909216in}}%
\pgfpathlineto{\pgfqpoint{1.162007in}{1.910056in}}%
\pgfpathlineto{\pgfqpoint{1.163642in}{1.911148in}}%
\pgfpathlineto{\pgfqpoint{1.164713in}{1.911988in}}%
\pgfpathlineto{\pgfqpoint{1.166873in}{1.913080in}}%
\pgfpathlineto{\pgfqpoint{1.167926in}{1.914004in}}%
\pgfpathlineto{\pgfqpoint{1.170086in}{1.915096in}}%
\pgfpathlineto{\pgfqpoint{1.171138in}{1.916020in}}%
\pgfpathlineto{\pgfqpoint{1.173017in}{1.917112in}}%
\pgfpathlineto{\pgfqpoint{1.173881in}{1.917868in}}%
\pgfpathlineto{\pgfqpoint{1.175816in}{1.918960in}}%
\pgfpathlineto{\pgfqpoint{1.176831in}{1.919716in}}%
\pgfpathlineto{\pgfqpoint{1.178935in}{1.920808in}}%
\pgfpathlineto{\pgfqpoint{1.179912in}{1.922236in}}%
\pgfpathlineto{\pgfqpoint{1.180889in}{1.923328in}}%
\pgfpathlineto{\pgfqpoint{1.181735in}{1.924252in}}%
\pgfpathlineto{\pgfqpoint{1.184759in}{1.925344in}}%
\pgfpathlineto{\pgfqpoint{1.185718in}{1.926604in}}%
\pgfpathlineto{\pgfqpoint{1.187634in}{1.927696in}}%
\pgfpathlineto{\pgfqpoint{1.188649in}{1.928536in}}%
\pgfpathlineto{\pgfqpoint{1.190509in}{1.929628in}}%
\pgfpathlineto{\pgfqpoint{1.191429in}{1.930300in}}%
\pgfpathlineto{\pgfqpoint{1.193759in}{1.931392in}}%
\pgfpathlineto{\pgfqpoint{1.194867in}{1.932064in}}%
\pgfpathlineto{\pgfqpoint{1.196295in}{1.933156in}}%
\pgfpathlineto{\pgfqpoint{1.197404in}{1.934164in}}%
\pgfpathlineto{\pgfqpoint{1.199226in}{1.935256in}}%
\pgfpathlineto{\pgfqpoint{1.200241in}{1.935676in}}%
\pgfpathlineto{\pgfqpoint{1.202608in}{1.936768in}}%
\pgfpathlineto{\pgfqpoint{1.203435in}{1.937944in}}%
\pgfpathlineto{\pgfqpoint{1.205915in}{1.939036in}}%
\pgfpathlineto{\pgfqpoint{1.206910in}{1.939456in}}%
\pgfpathlineto{\pgfqpoint{1.209015in}{1.940548in}}%
\pgfpathlineto{\pgfqpoint{1.209484in}{1.940968in}}%
\pgfpathlineto{\pgfqpoint{1.212096in}{1.942060in}}%
\pgfpathlineto{\pgfqpoint{1.212960in}{1.942396in}}%
\pgfpathlineto{\pgfqpoint{1.215985in}{1.943488in}}%
\pgfpathlineto{\pgfqpoint{1.217075in}{1.944076in}}%
\pgfpathlineto{\pgfqpoint{1.218822in}{1.945084in}}%
\pgfpathlineto{\pgfqpoint{1.219912in}{1.946512in}}%
\pgfpathlineto{\pgfqpoint{1.222035in}{1.947604in}}%
\pgfpathlineto{\pgfqpoint{1.222655in}{1.948276in}}%
\pgfpathlineto{\pgfqpoint{1.224721in}{1.949368in}}%
\pgfpathlineto{\pgfqpoint{1.225473in}{1.949788in}}%
\pgfpathlineto{\pgfqpoint{1.227596in}{1.950880in}}%
\pgfpathlineto{\pgfqpoint{1.228704in}{1.951384in}}%
\pgfpathlineto{\pgfqpoint{1.230903in}{1.952476in}}%
\pgfpathlineto{\pgfqpoint{1.231955in}{1.953316in}}%
\pgfpathlineto{\pgfqpoint{1.233589in}{1.954240in}}%
\pgfpathlineto{\pgfqpoint{1.234585in}{1.955164in}}%
\pgfpathlineto{\pgfqpoint{1.235844in}{1.956256in}}%
\pgfpathlineto{\pgfqpoint{1.236595in}{1.956844in}}%
\pgfpathlineto{\pgfqpoint{1.237854in}{1.957852in}}%
\pgfpathlineto{\pgfqpoint{1.238963in}{1.958944in}}%
\pgfpathlineto{\pgfqpoint{1.241461in}{1.960036in}}%
\pgfpathlineto{\pgfqpoint{1.242513in}{1.960960in}}%
\pgfpathlineto{\pgfqpoint{1.244937in}{1.962052in}}%
\pgfpathlineto{\pgfqpoint{1.245482in}{1.962472in}}%
\pgfpathlineto{\pgfqpoint{1.249108in}{1.963564in}}%
\pgfpathlineto{\pgfqpoint{1.250123in}{1.964656in}}%
\pgfpathlineto{\pgfqpoint{1.251607in}{1.965748in}}%
\pgfpathlineto{\pgfqpoint{1.252415in}{1.966252in}}%
\pgfpathlineto{\pgfqpoint{1.254275in}{1.967344in}}%
\pgfpathlineto{\pgfqpoint{1.255346in}{1.968016in}}%
\pgfpathlineto{\pgfqpoint{1.256830in}{1.969024in}}%
\pgfpathlineto{\pgfqpoint{1.257938in}{1.969948in}}%
\pgfpathlineto{\pgfqpoint{1.259798in}{1.971040in}}%
\pgfpathlineto{\pgfqpoint{1.260775in}{1.971628in}}%
\pgfpathlineto{\pgfqpoint{1.263086in}{1.972720in}}%
\pgfpathlineto{\pgfqpoint{1.264063in}{1.973392in}}%
\pgfpathlineto{\pgfqpoint{1.266318in}{1.974484in}}%
\pgfpathlineto{\pgfqpoint{1.267351in}{1.975072in}}%
\pgfpathlineto{\pgfqpoint{1.270132in}{1.976164in}}%
\pgfpathlineto{\pgfqpoint{1.271127in}{1.976668in}}%
\pgfpathlineto{\pgfqpoint{1.272649in}{1.977760in}}%
\pgfpathlineto{\pgfqpoint{1.273457in}{1.978348in}}%
\pgfpathlineto{\pgfqpoint{1.276087in}{1.979440in}}%
\pgfpathlineto{\pgfqpoint{1.276970in}{1.980112in}}%
\pgfpathlineto{\pgfqpoint{1.279300in}{1.981204in}}%
\pgfpathlineto{\pgfqpoint{1.280239in}{1.982044in}}%
\pgfpathlineto{\pgfqpoint{1.282513in}{1.983136in}}%
\pgfpathlineto{\pgfqpoint{1.283302in}{1.983304in}}%
\pgfpathlineto{\pgfqpoint{1.285575in}{1.984396in}}%
\pgfpathlineto{\pgfqpoint{1.286515in}{1.985068in}}%
\pgfpathlineto{\pgfqpoint{1.288149in}{1.986160in}}%
\pgfpathlineto{\pgfqpoint{1.289220in}{1.986916in}}%
\pgfpathlineto{\pgfqpoint{1.291982in}{1.987924in}}%
\pgfpathlineto{\pgfqpoint{1.292564in}{1.988596in}}%
\pgfpathlineto{\pgfqpoint{1.294950in}{1.989604in}}%
\pgfpathlineto{\pgfqpoint{1.295646in}{1.989856in}}%
\pgfpathlineto{\pgfqpoint{1.298069in}{1.990948in}}%
\pgfpathlineto{\pgfqpoint{1.298952in}{1.991620in}}%
\pgfpathlineto{\pgfqpoint{1.301075in}{1.992628in}}%
\pgfpathlineto{\pgfqpoint{1.302033in}{1.993468in}}%
\pgfpathlineto{\pgfqpoint{1.303706in}{1.994560in}}%
\pgfpathlineto{\pgfqpoint{1.304401in}{1.994980in}}%
\pgfpathlineto{\pgfqpoint{1.306129in}{1.996072in}}%
\pgfpathlineto{\pgfqpoint{1.307238in}{1.997080in}}%
\pgfpathlineto{\pgfqpoint{1.308778in}{1.998172in}}%
\pgfpathlineto{\pgfqpoint{1.309699in}{1.998592in}}%
\pgfpathlineto{\pgfqpoint{1.311747in}{1.999600in}}%
\pgfpathlineto{\pgfqpoint{1.312592in}{2.000272in}}%
\pgfpathlineto{\pgfqpoint{1.315072in}{2.001364in}}%
\pgfpathlineto{\pgfqpoint{1.315955in}{2.001784in}}%
\pgfpathlineto{\pgfqpoint{1.318003in}{2.002876in}}%
\pgfpathlineto{\pgfqpoint{1.318623in}{2.003296in}}%
\pgfpathlineto{\pgfqpoint{1.320990in}{2.004388in}}%
\pgfpathlineto{\pgfqpoint{1.322024in}{2.004808in}}%
\pgfpathlineto{\pgfqpoint{1.323433in}{2.005900in}}%
\pgfpathlineto{\pgfqpoint{1.324391in}{2.006488in}}%
\pgfpathlineto{\pgfqpoint{1.328299in}{2.007580in}}%
\pgfpathlineto{\pgfqpoint{1.329201in}{2.007916in}}%
\pgfpathlineto{\pgfqpoint{1.331286in}{2.009008in}}%
\pgfpathlineto{\pgfqpoint{1.332338in}{2.009428in}}%
\pgfpathlineto{\pgfqpoint{1.334330in}{2.010520in}}%
\pgfpathlineto{\pgfqpoint{1.335382in}{2.011192in}}%
\pgfpathlineto{\pgfqpoint{1.336735in}{2.012284in}}%
\pgfpathlineto{\pgfqpoint{1.337599in}{2.012956in}}%
\pgfpathlineto{\pgfqpoint{1.339590in}{2.014048in}}%
\pgfpathlineto{\pgfqpoint{1.340549in}{2.014720in}}%
\pgfpathlineto{\pgfqpoint{1.343160in}{2.015728in}}%
\pgfpathlineto{\pgfqpoint{1.344062in}{2.016148in}}%
\pgfpathlineto{\pgfqpoint{1.346035in}{2.017240in}}%
\pgfpathlineto{\pgfqpoint{1.346974in}{2.017660in}}%
\pgfpathlineto{\pgfqpoint{1.350337in}{2.018752in}}%
\pgfpathlineto{\pgfqpoint{1.350901in}{2.019172in}}%
\pgfpathlineto{\pgfqpoint{1.355598in}{2.020264in}}%
\pgfpathlineto{\pgfqpoint{1.356499in}{2.020768in}}%
\pgfpathlineto{\pgfqpoint{1.358716in}{2.021860in}}%
\pgfpathlineto{\pgfqpoint{1.359618in}{2.022448in}}%
\pgfpathlineto{\pgfqpoint{1.363583in}{2.023540in}}%
\pgfpathlineto{\pgfqpoint{1.364635in}{2.024044in}}%
\pgfpathlineto{\pgfqpoint{1.366720in}{2.025136in}}%
\pgfpathlineto{\pgfqpoint{1.367810in}{2.025892in}}%
\pgfpathlineto{\pgfqpoint{1.372619in}{2.026984in}}%
\pgfpathlineto{\pgfqpoint{1.373672in}{2.027404in}}%
\pgfpathlineto{\pgfqpoint{1.376471in}{2.028412in}}%
\pgfpathlineto{\pgfqpoint{1.377579in}{2.029168in}}%
\pgfpathlineto{\pgfqpoint{1.380473in}{2.030260in}}%
\pgfpathlineto{\pgfqpoint{1.381243in}{2.030596in}}%
\pgfpathlineto{\pgfqpoint{1.383892in}{2.031688in}}%
\pgfpathlineto{\pgfqpoint{1.384080in}{2.031856in}}%
\pgfpathlineto{\pgfqpoint{1.389942in}{2.032948in}}%
\pgfpathlineto{\pgfqpoint{1.390637in}{2.033536in}}%
\pgfpathlineto{\pgfqpoint{1.395334in}{2.034628in}}%
\pgfpathlineto{\pgfqpoint{1.396349in}{2.034796in}}%
\pgfpathlineto{\pgfqpoint{1.398603in}{2.035804in}}%
\pgfpathlineto{\pgfqpoint{1.399561in}{2.036140in}}%
\pgfpathlineto{\pgfqpoint{1.402793in}{2.037232in}}%
\pgfpathlineto{\pgfqpoint{1.403901in}{2.037652in}}%
\pgfpathlineto{\pgfqpoint{1.406137in}{2.038744in}}%
\pgfpathlineto{\pgfqpoint{1.406983in}{2.039164in}}%
\pgfpathlineto{\pgfqpoint{1.409218in}{2.040256in}}%
\pgfpathlineto{\pgfqpoint{1.410233in}{2.040844in}}%
\pgfpathlineto{\pgfqpoint{1.412450in}{2.041852in}}%
\pgfpathlineto{\pgfqpoint{1.413464in}{2.042692in}}%
\pgfpathlineto{\pgfqpoint{1.418086in}{2.043784in}}%
\pgfpathlineto{\pgfqpoint{1.419119in}{2.044120in}}%
\pgfpathlineto{\pgfqpoint{1.424023in}{2.045212in}}%
\pgfpathlineto{\pgfqpoint{1.424887in}{2.045716in}}%
\pgfpathlineto{\pgfqpoint{1.428701in}{2.046808in}}%
\pgfpathlineto{\pgfqpoint{1.429810in}{2.047564in}}%
\pgfpathlineto{\pgfqpoint{1.432797in}{2.048656in}}%
\pgfpathlineto{\pgfqpoint{1.433586in}{2.049160in}}%
\pgfpathlineto{\pgfqpoint{1.435390in}{2.050252in}}%
\pgfpathlineto{\pgfqpoint{1.435859in}{2.050588in}}%
\pgfpathlineto{\pgfqpoint{1.438321in}{2.051680in}}%
\pgfpathlineto{\pgfqpoint{1.439335in}{2.052268in}}%
\pgfpathlineto{\pgfqpoint{1.443487in}{2.053360in}}%
\pgfpathlineto{\pgfqpoint{1.444558in}{2.054116in}}%
\pgfpathlineto{\pgfqpoint{1.449988in}{2.055208in}}%
\pgfpathlineto{\pgfqpoint{1.450383in}{2.055628in}}%
\pgfpathlineto{\pgfqpoint{1.454309in}{2.056720in}}%
\pgfpathlineto{\pgfqpoint{1.455418in}{2.057308in}}%
\pgfpathlineto{\pgfqpoint{1.458236in}{2.058400in}}%
\pgfpathlineto{\pgfqpoint{1.459307in}{2.058904in}}%
\pgfpathlineto{\pgfqpoint{1.462764in}{2.059996in}}%
\pgfpathlineto{\pgfqpoint{1.463872in}{2.060920in}}%
\pgfpathlineto{\pgfqpoint{1.467630in}{2.062012in}}%
\pgfpathlineto{\pgfqpoint{1.468569in}{2.062432in}}%
\pgfpathlineto{\pgfqpoint{1.471406in}{2.063524in}}%
\pgfpathlineto{\pgfqpoint{1.472458in}{2.064028in}}%
\pgfpathlineto{\pgfqpoint{1.476460in}{2.065120in}}%
\pgfpathlineto{\pgfqpoint{1.477362in}{2.065456in}}%
\pgfpathlineto{\pgfqpoint{1.479729in}{2.066548in}}%
\pgfpathlineto{\pgfqpoint{1.480800in}{2.067388in}}%
\pgfpathlineto{\pgfqpoint{1.484144in}{2.068480in}}%
\pgfpathlineto{\pgfqpoint{1.485009in}{2.068816in}}%
\pgfpathlineto{\pgfqpoint{1.489649in}{2.069908in}}%
\pgfpathlineto{\pgfqpoint{1.490701in}{2.070328in}}%
\pgfpathlineto{\pgfqpoint{1.494666in}{2.071420in}}%
\pgfpathlineto{\pgfqpoint{1.495455in}{2.071588in}}%
\pgfpathlineto{\pgfqpoint{1.501467in}{2.072680in}}%
\pgfpathlineto{\pgfqpoint{1.502387in}{2.073268in}}%
\pgfpathlineto{\pgfqpoint{1.505168in}{2.074360in}}%
\pgfpathlineto{\pgfqpoint{1.505844in}{2.074780in}}%
\pgfpathlineto{\pgfqpoint{1.511330in}{2.075872in}}%
\pgfpathlineto{\pgfqpoint{1.512383in}{2.076292in}}%
\pgfpathlineto{\pgfqpoint{1.515821in}{2.077384in}}%
\pgfpathlineto{\pgfqpoint{1.516723in}{2.077636in}}%
\pgfpathlineto{\pgfqpoint{1.521250in}{2.078728in}}%
\pgfpathlineto{\pgfqpoint{1.521250in}{2.078812in}}%
\pgfpathlineto{\pgfqpoint{1.526210in}{2.079904in}}%
\pgfpathlineto{\pgfqpoint{1.527281in}{2.080408in}}%
\pgfpathlineto{\pgfqpoint{1.530626in}{2.081500in}}%
\pgfpathlineto{\pgfqpoint{1.531640in}{2.081920in}}%
\pgfpathlineto{\pgfqpoint{1.534984in}{2.083012in}}%
\pgfpathlineto{\pgfqpoint{1.535999in}{2.083432in}}%
\pgfpathlineto{\pgfqpoint{1.538967in}{2.084524in}}%
\pgfpathlineto{\pgfqpoint{1.539888in}{2.084944in}}%
\pgfpathlineto{\pgfqpoint{1.544435in}{2.086036in}}%
\pgfpathlineto{\pgfqpoint{1.545130in}{2.086456in}}%
\pgfpathlineto{\pgfqpoint{1.548399in}{2.087548in}}%
\pgfpathlineto{\pgfqpoint{1.548399in}{2.087632in}}%
\pgfpathlineto{\pgfqpoint{1.553378in}{2.088724in}}%
\pgfpathlineto{\pgfqpoint{1.554467in}{2.089228in}}%
\pgfpathlineto{\pgfqpoint{1.559503in}{2.090320in}}%
\pgfpathlineto{\pgfqpoint{1.560461in}{2.090488in}}%
\pgfpathlineto{\pgfqpoint{1.563598in}{2.091580in}}%
\pgfpathlineto{\pgfqpoint{1.564275in}{2.091916in}}%
\pgfpathlineto{\pgfqpoint{1.569817in}{2.093008in}}%
\pgfpathlineto{\pgfqpoint{1.570756in}{2.093764in}}%
\pgfpathlineto{\pgfqpoint{1.574552in}{2.094856in}}%
\pgfpathlineto{\pgfqpoint{1.575040in}{2.095192in}}%
\pgfpathlineto{\pgfqpoint{1.580958in}{2.096284in}}%
\pgfpathlineto{\pgfqpoint{1.580958in}{2.096368in}}%
\pgfpathlineto{\pgfqpoint{1.588924in}{2.097460in}}%
\pgfpathlineto{\pgfqpoint{1.589394in}{2.097796in}}%
\pgfpathlineto{\pgfqpoint{1.594354in}{2.098888in}}%
\pgfpathlineto{\pgfqpoint{1.594673in}{2.099056in}}%
\pgfpathlineto{\pgfqpoint{1.601212in}{2.100148in}}%
\pgfpathlineto{\pgfqpoint{1.602019in}{2.100484in}}%
\pgfpathlineto{\pgfqpoint{1.606735in}{2.101576in}}%
\pgfpathlineto{\pgfqpoint{1.607581in}{2.102080in}}%
\pgfpathlineto{\pgfqpoint{1.613405in}{2.103172in}}%
\pgfpathlineto{\pgfqpoint{1.613856in}{2.103508in}}%
\pgfpathlineto{\pgfqpoint{1.618797in}{2.104600in}}%
\pgfpathlineto{\pgfqpoint{1.619473in}{2.105020in}}%
\pgfpathlineto{\pgfqpoint{1.629299in}{2.106112in}}%
\pgfpathlineto{\pgfqpoint{1.630164in}{2.106364in}}%
\pgfpathlineto{\pgfqpoint{1.633376in}{2.107456in}}%
\pgfpathlineto{\pgfqpoint{1.633921in}{2.107708in}}%
\pgfpathlineto{\pgfqpoint{1.639032in}{2.108800in}}%
\pgfpathlineto{\pgfqpoint{1.639520in}{2.108968in}}%
\pgfpathlineto{\pgfqpoint{1.648444in}{2.109976in}}%
\pgfpathlineto{\pgfqpoint{1.649403in}{2.110228in}}%
\pgfpathlineto{\pgfqpoint{1.655959in}{2.111320in}}%
\pgfpathlineto{\pgfqpoint{1.657049in}{2.111824in}}%
\pgfpathlineto{\pgfqpoint{1.662159in}{2.112916in}}%
\pgfpathlineto{\pgfqpoint{1.663155in}{2.113504in}}%
\pgfpathlineto{\pgfqpoint{1.669750in}{2.114596in}}%
\pgfpathlineto{\pgfqpoint{1.670802in}{2.114932in}}%
\pgfpathlineto{\pgfqpoint{1.678167in}{2.116024in}}%
\pgfpathlineto{\pgfqpoint{1.679106in}{2.116360in}}%
\pgfpathlineto{\pgfqpoint{1.686114in}{2.117452in}}%
\pgfpathlineto{\pgfqpoint{1.686959in}{2.117620in}}%
\pgfpathlineto{\pgfqpoint{1.690041in}{2.118712in}}%
\pgfpathlineto{\pgfqpoint{1.690999in}{2.118964in}}%
\pgfpathlineto{\pgfqpoint{1.697462in}{2.120056in}}%
\pgfpathlineto{\pgfqpoint{1.698007in}{2.120392in}}%
\pgfpathlineto{\pgfqpoint{1.706987in}{2.121484in}}%
\pgfpathlineto{\pgfqpoint{1.707927in}{2.121736in}}%
\pgfpathlineto{\pgfqpoint{1.712830in}{2.122828in}}%
\pgfpathlineto{\pgfqpoint{1.713338in}{2.123164in}}%
\pgfpathlineto{\pgfqpoint{1.718129in}{2.124256in}}%
\pgfpathlineto{\pgfqpoint{1.718129in}{2.124340in}}%
\pgfpathlineto{\pgfqpoint{1.724084in}{2.125432in}}%
\pgfpathlineto{\pgfqpoint{1.725024in}{2.125600in}}%
\pgfpathlineto{\pgfqpoint{1.731675in}{2.126692in}}%
\pgfpathlineto{\pgfqpoint{1.732614in}{2.126944in}}%
\pgfpathlineto{\pgfqpoint{1.738363in}{2.128036in}}%
\pgfpathlineto{\pgfqpoint{1.739453in}{2.128372in}}%
\pgfpathlineto{\pgfqpoint{1.744544in}{2.129464in}}%
\pgfpathlineto{\pgfqpoint{1.744544in}{2.129548in}}%
\pgfpathlineto{\pgfqpoint{1.754840in}{2.130640in}}%
\pgfpathlineto{\pgfqpoint{1.755159in}{2.130808in}}%
\pgfpathlineto{\pgfqpoint{1.761979in}{2.131900in}}%
\pgfpathlineto{\pgfqpoint{1.762938in}{2.132152in}}%
\pgfpathlineto{\pgfqpoint{1.769006in}{2.133244in}}%
\pgfpathlineto{\pgfqpoint{1.769006in}{2.133328in}}%
\pgfpathlineto{\pgfqpoint{1.777310in}{2.134420in}}%
\pgfpathlineto{\pgfqpoint{1.778306in}{2.134588in}}%
\pgfpathlineto{\pgfqpoint{1.783679in}{2.135680in}}%
\pgfpathlineto{\pgfqpoint{1.784356in}{2.135848in}}%
\pgfpathlineto{\pgfqpoint{1.791646in}{2.136940in}}%
\pgfpathlineto{\pgfqpoint{1.792604in}{2.137276in}}%
\pgfpathlineto{\pgfqpoint{1.800025in}{2.138368in}}%
\pgfpathlineto{\pgfqpoint{1.800983in}{2.138788in}}%
\pgfpathlineto{\pgfqpoint{1.807484in}{2.139880in}}%
\pgfpathlineto{\pgfqpoint{1.808066in}{2.140300in}}%
\pgfpathlineto{\pgfqpoint{1.813515in}{2.141392in}}%
\pgfpathlineto{\pgfqpoint{1.814097in}{2.141644in}}%
\pgfpathlineto{\pgfqpoint{1.821011in}{2.142736in}}%
\pgfpathlineto{\pgfqpoint{1.822101in}{2.143072in}}%
\pgfpathlineto{\pgfqpoint{1.828338in}{2.144164in}}%
\pgfpathlineto{\pgfqpoint{1.829296in}{2.144836in}}%
\pgfpathlineto{\pgfqpoint{1.832321in}{2.145928in}}%
\pgfpathlineto{\pgfqpoint{1.832979in}{2.146264in}}%
\pgfpathlineto{\pgfqpoint{1.842730in}{2.147356in}}%
\pgfpathlineto{\pgfqpoint{1.842730in}{2.147440in}}%
\pgfpathlineto{\pgfqpoint{1.850639in}{2.148532in}}%
\pgfpathlineto{\pgfqpoint{1.851184in}{2.148952in}}%
\pgfpathlineto{\pgfqpoint{1.858718in}{2.150044in}}%
\pgfpathlineto{\pgfqpoint{1.858718in}{2.150128in}}%
\pgfpathlineto{\pgfqpoint{1.866572in}{2.151220in}}%
\pgfpathlineto{\pgfqpoint{1.867304in}{2.151472in}}%
\pgfpathlineto{\pgfqpoint{1.874068in}{2.152564in}}%
\pgfpathlineto{\pgfqpoint{1.874632in}{2.152900in}}%
\pgfpathlineto{\pgfqpoint{1.881301in}{2.153992in}}%
\pgfpathlineto{\pgfqpoint{1.882203in}{2.154412in}}%
\pgfpathlineto{\pgfqpoint{1.889887in}{2.155504in}}%
\pgfpathlineto{\pgfqpoint{1.890432in}{2.155924in}}%
\pgfpathlineto{\pgfqpoint{1.902419in}{2.157016in}}%
\pgfpathlineto{\pgfqpoint{1.903358in}{2.157268in}}%
\pgfpathlineto{\pgfqpoint{1.911268in}{2.158360in}}%
\pgfpathlineto{\pgfqpoint{1.912038in}{2.158696in}}%
\pgfpathlineto{\pgfqpoint{1.916641in}{2.159788in}}%
\pgfpathlineto{\pgfqpoint{1.917618in}{2.160124in}}%
\pgfpathlineto{\pgfqpoint{1.927970in}{2.161216in}}%
\pgfpathlineto{\pgfqpoint{1.929079in}{2.161468in}}%
\pgfpathlineto{\pgfqpoint{1.939206in}{2.162560in}}%
\pgfpathlineto{\pgfqpoint{1.940239in}{2.162980in}}%
\pgfpathlineto{\pgfqpoint{1.957843in}{2.164072in}}%
\pgfpathlineto{\pgfqpoint{1.958519in}{2.164324in}}%
\pgfpathlineto{\pgfqpoint{1.967575in}{2.165416in}}%
\pgfpathlineto{\pgfqpoint{1.967895in}{2.165584in}}%
\pgfpathlineto{\pgfqpoint{1.981103in}{2.166676in}}%
\pgfpathlineto{\pgfqpoint{1.981422in}{2.166844in}}%
\pgfpathlineto{\pgfqpoint{1.989069in}{2.167936in}}%
\pgfpathlineto{\pgfqpoint{1.990027in}{2.168272in}}%
\pgfpathlineto{\pgfqpoint{1.999045in}{2.169364in}}%
\pgfpathlineto{\pgfqpoint{2.000135in}{2.169784in}}%
\pgfpathlineto{\pgfqpoint{2.009886in}{2.170876in}}%
\pgfpathlineto{\pgfqpoint{2.010581in}{2.171212in}}%
\pgfpathlineto{\pgfqpoint{2.020332in}{2.172304in}}%
\pgfpathlineto{\pgfqpoint{2.020332in}{2.172388in}}%
\pgfpathlineto{\pgfqpoint{2.027959in}{2.173480in}}%
\pgfpathlineto{\pgfqpoint{2.028561in}{2.173648in}}%
\pgfpathlineto{\pgfqpoint{2.037466in}{2.174740in}}%
\pgfpathlineto{\pgfqpoint{2.037466in}{2.174824in}}%
\pgfpathlineto{\pgfqpoint{2.044887in}{2.175916in}}%
\pgfpathlineto{\pgfqpoint{2.045226in}{2.176168in}}%
\pgfpathlineto{\pgfqpoint{2.053962in}{2.177260in}}%
\pgfpathlineto{\pgfqpoint{2.054281in}{2.177428in}}%
\pgfpathlineto{\pgfqpoint{2.064972in}{2.178520in}}%
\pgfpathlineto{\pgfqpoint{2.065592in}{2.178688in}}%
\pgfpathlineto{\pgfqpoint{2.071040in}{2.179780in}}%
\pgfpathlineto{\pgfqpoint{2.072130in}{2.180200in}}%
\pgfpathlineto{\pgfqpoint{2.082294in}{2.181292in}}%
\pgfpathlineto{\pgfqpoint{2.082613in}{2.181544in}}%
\pgfpathlineto{\pgfqpoint{2.093247in}{2.182636in}}%
\pgfpathlineto{\pgfqpoint{2.093247in}{2.182720in}}%
\pgfpathlineto{\pgfqpoint{2.103618in}{2.183812in}}%
\pgfpathlineto{\pgfqpoint{2.104483in}{2.184064in}}%
\pgfpathlineto{\pgfqpoint{2.118536in}{2.185156in}}%
\pgfpathlineto{\pgfqpoint{2.119269in}{2.185324in}}%
\pgfpathlineto{\pgfqpoint{2.127253in}{2.186416in}}%
\pgfpathlineto{\pgfqpoint{2.128212in}{2.186752in}}%
\pgfpathlineto{\pgfqpoint{2.140536in}{2.187844in}}%
\pgfpathlineto{\pgfqpoint{2.141419in}{2.188012in}}%
\pgfpathlineto{\pgfqpoint{2.153012in}{2.189104in}}%
\pgfpathlineto{\pgfqpoint{2.153632in}{2.189440in}}%
\pgfpathlineto{\pgfqpoint{2.163307in}{2.190532in}}%
\pgfpathlineto{\pgfqpoint{2.164266in}{2.190700in}}%
\pgfpathlineto{\pgfqpoint{2.176496in}{2.191792in}}%
\pgfpathlineto{\pgfqpoint{2.176496in}{2.191876in}}%
\pgfpathlineto{\pgfqpoint{2.186980in}{2.192968in}}%
\pgfpathlineto{\pgfqpoint{2.187619in}{2.193136in}}%
\pgfpathlineto{\pgfqpoint{2.198441in}{2.194228in}}%
\pgfpathlineto{\pgfqpoint{2.199249in}{2.194396in}}%
\pgfpathlineto{\pgfqpoint{2.210540in}{2.195488in}}%
\pgfpathlineto{\pgfqpoint{2.211573in}{2.195740in}}%
\pgfpathlineto{\pgfqpoint{2.222715in}{2.196832in}}%
\pgfpathlineto{\pgfqpoint{2.222715in}{2.196916in}}%
\pgfpathlineto{\pgfqpoint{2.237275in}{2.198008in}}%
\pgfpathlineto{\pgfqpoint{2.238365in}{2.198176in}}%
\pgfpathlineto{\pgfqpoint{2.249055in}{2.199268in}}%
\pgfpathlineto{\pgfqpoint{2.249675in}{2.199520in}}%
\pgfpathlineto{\pgfqpoint{2.261944in}{2.200612in}}%
\pgfpathlineto{\pgfqpoint{2.262564in}{2.200864in}}%
\pgfpathlineto{\pgfqpoint{2.274475in}{2.201956in}}%
\pgfpathlineto{\pgfqpoint{2.274475in}{2.202040in}}%
\pgfpathlineto{\pgfqpoint{2.293752in}{2.203132in}}%
\pgfpathlineto{\pgfqpoint{2.293752in}{2.203216in}}%
\pgfpathlineto{\pgfqpoint{2.311337in}{2.204308in}}%
\pgfpathlineto{\pgfqpoint{2.311337in}{2.204392in}}%
\pgfpathlineto{\pgfqpoint{2.328979in}{2.205484in}}%
\pgfpathlineto{\pgfqpoint{2.329881in}{2.205652in}}%
\pgfpathlineto{\pgfqpoint{2.345212in}{2.206744in}}%
\pgfpathlineto{\pgfqpoint{2.345212in}{2.206828in}}%
\pgfpathlineto{\pgfqpoint{2.355939in}{2.207920in}}%
\pgfpathlineto{\pgfqpoint{2.355939in}{2.208004in}}%
\pgfpathlineto{\pgfqpoint{2.366198in}{2.209096in}}%
\pgfpathlineto{\pgfqpoint{2.367193in}{2.209432in}}%
\pgfpathlineto{\pgfqpoint{2.378259in}{2.210524in}}%
\pgfpathlineto{\pgfqpoint{2.378259in}{2.210608in}}%
\pgfpathlineto{\pgfqpoint{2.397367in}{2.211700in}}%
\pgfpathlineto{\pgfqpoint{2.397761in}{2.211868in}}%
\pgfpathlineto{\pgfqpoint{2.412529in}{2.212960in}}%
\pgfpathlineto{\pgfqpoint{2.413224in}{2.213380in}}%
\pgfpathlineto{\pgfqpoint{2.428216in}{2.214472in}}%
\pgfpathlineto{\pgfqpoint{2.428874in}{2.214640in}}%
\pgfpathlineto{\pgfqpoint{2.437779in}{2.215732in}}%
\pgfpathlineto{\pgfqpoint{2.437779in}{2.215816in}}%
\pgfpathlineto{\pgfqpoint{2.453505in}{2.216908in}}%
\pgfpathlineto{\pgfqpoint{2.453505in}{2.216992in}}%
\pgfpathlineto{\pgfqpoint{2.470771in}{2.218084in}}%
\pgfpathlineto{\pgfqpoint{2.471429in}{2.218336in}}%
\pgfpathlineto{\pgfqpoint{2.485557in}{2.219428in}}%
\pgfpathlineto{\pgfqpoint{2.485970in}{2.219596in}}%
\pgfpathlineto{\pgfqpoint{2.506055in}{2.220688in}}%
\pgfpathlineto{\pgfqpoint{2.506055in}{2.220772in}}%
\pgfpathlineto{\pgfqpoint{2.519413in}{2.221864in}}%
\pgfpathlineto{\pgfqpoint{2.519789in}{2.222032in}}%
\pgfpathlineto{\pgfqpoint{2.529464in}{2.223124in}}%
\pgfpathlineto{\pgfqpoint{2.529558in}{2.223292in}}%
\pgfpathlineto{\pgfqpoint{2.547557in}{2.224384in}}%
\pgfpathlineto{\pgfqpoint{2.548177in}{2.224552in}}%
\pgfpathlineto{\pgfqpoint{2.567472in}{2.225644in}}%
\pgfpathlineto{\pgfqpoint{2.567886in}{2.225812in}}%
\pgfpathlineto{\pgfqpoint{2.584532in}{2.226904in}}%
\pgfpathlineto{\pgfqpoint{2.585452in}{2.227072in}}%
\pgfpathlineto{\pgfqpoint{2.595616in}{2.228164in}}%
\pgfpathlineto{\pgfqpoint{2.596105in}{2.228332in}}%
\pgfpathlineto{\pgfqpoint{2.612018in}{2.229424in}}%
\pgfpathlineto{\pgfqpoint{2.612300in}{2.229592in}}%
\pgfpathlineto{\pgfqpoint{2.626767in}{2.230684in}}%
\pgfpathlineto{\pgfqpoint{2.626767in}{2.230768in}}%
\pgfpathlineto{\pgfqpoint{2.639881in}{2.231860in}}%
\pgfpathlineto{\pgfqpoint{2.640933in}{2.232028in}}%
\pgfpathlineto{\pgfqpoint{2.653821in}{2.233120in}}%
\pgfpathlineto{\pgfqpoint{2.654836in}{2.233288in}}%
\pgfpathlineto{\pgfqpoint{2.667292in}{2.234380in}}%
\pgfpathlineto{\pgfqpoint{2.668269in}{2.234716in}}%
\pgfpathlineto{\pgfqpoint{2.684934in}{2.235808in}}%
\pgfpathlineto{\pgfqpoint{2.685404in}{2.235976in}}%
\pgfpathlineto{\pgfqpoint{2.702163in}{2.237068in}}%
\pgfpathlineto{\pgfqpoint{2.702670in}{2.237320in}}%
\pgfpathlineto{\pgfqpoint{2.719278in}{2.238412in}}%
\pgfpathlineto{\pgfqpoint{2.720067in}{2.238580in}}%
\pgfpathlineto{\pgfqpoint{2.736169in}{2.239672in}}%
\pgfpathlineto{\pgfqpoint{2.737221in}{2.240092in}}%
\pgfpathlineto{\pgfqpoint{2.756685in}{2.241184in}}%
\pgfpathlineto{\pgfqpoint{2.756685in}{2.241352in}}%
\pgfpathlineto{\pgfqpoint{2.772166in}{2.242444in}}%
\pgfpathlineto{\pgfqpoint{2.773275in}{2.242780in}}%
\pgfpathlineto{\pgfqpoint{2.792551in}{2.243872in}}%
\pgfpathlineto{\pgfqpoint{2.792683in}{2.244040in}}%
\pgfpathlineto{\pgfqpoint{2.803655in}{2.245048in}}%
\pgfpathlineto{\pgfqpoint{2.803655in}{2.245216in}}%
\pgfpathlineto{\pgfqpoint{2.818741in}{2.246308in}}%
\pgfpathlineto{\pgfqpoint{2.818741in}{2.246392in}}%
\pgfpathlineto{\pgfqpoint{2.834730in}{2.247484in}}%
\pgfpathlineto{\pgfqpoint{2.835331in}{2.247820in}}%
\pgfpathlineto{\pgfqpoint{2.844875in}{2.248912in}}%
\pgfpathlineto{\pgfqpoint{2.844875in}{2.248996in}}%
\pgfpathlineto{\pgfqpoint{2.859868in}{2.250088in}}%
\pgfpathlineto{\pgfqpoint{2.860789in}{2.250508in}}%
\pgfpathlineto{\pgfqpoint{2.873996in}{2.251600in}}%
\pgfpathlineto{\pgfqpoint{2.875049in}{2.251768in}}%
\pgfpathlineto{\pgfqpoint{2.886021in}{2.252860in}}%
\pgfpathlineto{\pgfqpoint{2.886472in}{2.253028in}}%
\pgfpathlineto{\pgfqpoint{2.894118in}{2.254120in}}%
\pgfpathlineto{\pgfqpoint{2.894118in}{2.254204in}}%
\pgfpathlineto{\pgfqpoint{2.901126in}{2.255296in}}%
\pgfpathlineto{\pgfqpoint{2.902178in}{2.255800in}}%
\pgfpathlineto{\pgfqpoint{2.908547in}{2.256892in}}%
\pgfpathlineto{\pgfqpoint{2.909449in}{2.257396in}}%
\pgfpathlineto{\pgfqpoint{2.914992in}{2.258488in}}%
\pgfpathlineto{\pgfqpoint{2.916100in}{2.258824in}}%
\pgfpathlineto{\pgfqpoint{2.918806in}{2.259916in}}%
\pgfpathlineto{\pgfqpoint{2.919876in}{2.260840in}}%
\pgfpathlineto{\pgfqpoint{2.920628in}{2.261848in}}%
\pgfpathlineto{\pgfqpoint{2.920853in}{2.263444in}}%
\pgfpathlineto{\pgfqpoint{2.920853in}{2.263444in}}%
\pgfusepath{stroke}%
\end{pgfscope}%
\begin{pgfscope}%
\pgfsetrectcap%
\pgfsetmiterjoin%
\pgfsetlinewidth{0.803000pt}%
\definecolor{currentstroke}{rgb}{0.000000,0.000000,0.000000}%
\pgfsetstrokecolor{currentstroke}%
\pgfsetdash{}{0pt}%
\pgfpathmoveto{\pgfqpoint{0.553581in}{0.499444in}}%
\pgfpathlineto{\pgfqpoint{0.553581in}{2.347444in}}%
\pgfusepath{stroke}%
\end{pgfscope}%
\begin{pgfscope}%
\pgfsetrectcap%
\pgfsetmiterjoin%
\pgfsetlinewidth{0.803000pt}%
\definecolor{currentstroke}{rgb}{0.000000,0.000000,0.000000}%
\pgfsetstrokecolor{currentstroke}%
\pgfsetdash{}{0pt}%
\pgfpathmoveto{\pgfqpoint{3.033581in}{0.499444in}}%
\pgfpathlineto{\pgfqpoint{3.033581in}{2.347444in}}%
\pgfusepath{stroke}%
\end{pgfscope}%
\begin{pgfscope}%
\pgfsetrectcap%
\pgfsetmiterjoin%
\pgfsetlinewidth{0.803000pt}%
\definecolor{currentstroke}{rgb}{0.000000,0.000000,0.000000}%
\pgfsetstrokecolor{currentstroke}%
\pgfsetdash{}{0pt}%
\pgfpathmoveto{\pgfqpoint{0.553581in}{0.499444in}}%
\pgfpathlineto{\pgfqpoint{3.033581in}{0.499444in}}%
\pgfusepath{stroke}%
\end{pgfscope}%
\begin{pgfscope}%
\pgfsetrectcap%
\pgfsetmiterjoin%
\pgfsetlinewidth{0.803000pt}%
\definecolor{currentstroke}{rgb}{0.000000,0.000000,0.000000}%
\pgfsetstrokecolor{currentstroke}%
\pgfsetdash{}{0pt}%
\pgfpathmoveto{\pgfqpoint{0.553581in}{2.347444in}}%
\pgfpathlineto{\pgfqpoint{3.033581in}{2.347444in}}%
\pgfusepath{stroke}%
\end{pgfscope}%
\begin{pgfscope}%
\pgfsetbuttcap%
\pgfsetmiterjoin%
\definecolor{currentfill}{rgb}{1.000000,1.000000,1.000000}%
\pgfsetfillcolor{currentfill}%
\pgfsetlinewidth{1.003750pt}%
\definecolor{currentstroke}{rgb}{1.000000,1.000000,1.000000}%
\pgfsetstrokecolor{currentstroke}%
\pgfsetdash{}{0pt}%
\pgfpathmoveto{\pgfqpoint{1.738420in}{2.054860in}}%
\pgfpathlineto{\pgfqpoint{2.165920in}{2.054860in}}%
\pgfpathlineto{\pgfqpoint{2.165920in}{2.289305in}}%
\pgfpathlineto{\pgfqpoint{1.738420in}{2.289305in}}%
\pgfpathlineto{\pgfqpoint{1.738420in}{2.054860in}}%
\pgfpathclose%
\pgfusepath{stroke,fill}%
\end{pgfscope}%
\begin{pgfscope}%
\definecolor{textcolor}{rgb}{0.000000,0.000000,0.000000}%
\pgfsetstrokecolor{textcolor}%
\pgfsetfillcolor{textcolor}%
\pgftext[x=1.793975in,y=2.137360in,left,base]{\color{textcolor}\rmfamily\fontsize{10.000000}{12.000000}\selectfont 0.418}%
\end{pgfscope}%
\begin{pgfscope}%
\pgfsetbuttcap%
\pgfsetmiterjoin%
\definecolor{currentfill}{rgb}{1.000000,1.000000,1.000000}%
\pgfsetfillcolor{currentfill}%
\pgfsetlinewidth{1.003750pt}%
\definecolor{currentstroke}{rgb}{1.000000,1.000000,1.000000}%
\pgfsetstrokecolor{currentstroke}%
\pgfsetdash{}{0pt}%
\pgfpathmoveto{\pgfqpoint{0.771915in}{1.340944in}}%
\pgfpathlineto{\pgfqpoint{1.199415in}{1.340944in}}%
\pgfpathlineto{\pgfqpoint{1.199415in}{1.575389in}}%
\pgfpathlineto{\pgfqpoint{0.771915in}{1.575389in}}%
\pgfpathlineto{\pgfqpoint{0.771915in}{1.340944in}}%
\pgfpathclose%
\pgfusepath{stroke,fill}%
\end{pgfscope}%
\begin{pgfscope}%
\definecolor{textcolor}{rgb}{0.000000,0.000000,0.000000}%
\pgfsetstrokecolor{textcolor}%
\pgfsetfillcolor{textcolor}%
\pgftext[x=0.827470in,y=1.423444in,left,base]{\color{textcolor}\rmfamily\fontsize{10.000000}{12.000000}\selectfont 0.582}%
\end{pgfscope}%
\begin{pgfscope}%
\definecolor{textcolor}{rgb}{0.000000,0.000000,0.000000}%
\pgfsetstrokecolor{textcolor}%
\pgfsetfillcolor{textcolor}%
\pgftext[x=1.793581in,y=2.430778in,,base]{\color{textcolor}\rmfamily\fontsize{12.000000}{14.400000}\selectfont ROC Curve}%
\end{pgfscope}%
\begin{pgfscope}%
\pgfsetbuttcap%
\pgfsetmiterjoin%
\definecolor{currentfill}{rgb}{1.000000,1.000000,1.000000}%
\pgfsetfillcolor{currentfill}%
\pgfsetfillopacity{0.800000}%
\pgfsetlinewidth{1.003750pt}%
\definecolor{currentstroke}{rgb}{0.800000,0.800000,0.800000}%
\pgfsetstrokecolor{currentstroke}%
\pgfsetstrokeopacity{0.800000}%
\pgfsetdash{}{0pt}%
\pgfpathmoveto{\pgfqpoint{0.800942in}{0.568889in}}%
\pgfpathlineto{\pgfqpoint{2.936358in}{0.568889in}}%
\pgfpathquadraticcurveto{\pgfqpoint{2.964136in}{0.568889in}}{\pgfqpoint{2.964136in}{0.596666in}}%
\pgfpathlineto{\pgfqpoint{2.964136in}{0.791111in}}%
\pgfpathquadraticcurveto{\pgfqpoint{2.964136in}{0.818888in}}{\pgfqpoint{2.936358in}{0.818888in}}%
\pgfpathlineto{\pgfqpoint{0.800942in}{0.818888in}}%
\pgfpathquadraticcurveto{\pgfqpoint{0.773164in}{0.818888in}}{\pgfqpoint{0.773164in}{0.791111in}}%
\pgfpathlineto{\pgfqpoint{0.773164in}{0.596666in}}%
\pgfpathquadraticcurveto{\pgfqpoint{0.773164in}{0.568889in}}{\pgfqpoint{0.800942in}{0.568889in}}%
\pgfpathlineto{\pgfqpoint{0.800942in}{0.568889in}}%
\pgfpathclose%
\pgfusepath{stroke,fill}%
\end{pgfscope}%
\begin{pgfscope}%
\pgfsetrectcap%
\pgfsetroundjoin%
\pgfsetlinewidth{1.505625pt}%
\definecolor{currentstroke}{rgb}{0.121569,0.466667,0.705882}%
\pgfsetstrokecolor{currentstroke}%
\pgfsetdash{}{0pt}%
\pgfpathmoveto{\pgfqpoint{0.828720in}{0.707777in}}%
\pgfpathlineto{\pgfqpoint{0.967608in}{0.707777in}}%
\pgfpathlineto{\pgfqpoint{1.106497in}{0.707777in}}%
\pgfusepath{stroke}%
\end{pgfscope}%
\begin{pgfscope}%
\definecolor{textcolor}{rgb}{0.000000,0.000000,0.000000}%
\pgfsetstrokecolor{textcolor}%
\pgfsetfillcolor{textcolor}%
\pgftext[x=1.217608in,y=0.659166in,left,base]{\color{textcolor}\rmfamily\fontsize{10.000000}{12.000000}\selectfont Area Under Curve = 0.847)}%
\end{pgfscope}%
\end{pgfpicture}%
\makeatother%
\endgroup%

\end{tabular}
\end{center}

\begin{center}
\begin{tabular}{cc}
\begin{tabular}{cc|c|c|}
	&\multicolumn{1}{c}{}& \multicolumn{2}{c}{Prediction} \cr
	&\multicolumn{1}{c}{} & \multicolumn{1}{c}{N} & \multicolumn{1}{c}{P} \cr\cline{3-4}
	\multirow{2}{*}{Actual}&N & 67.2\% & 18.5\% \vrule width 0pt height 10pt depth 2pt \cr\cline{3-4}
	&P & 3.06\% & 11.22\% \vrule width 0pt height 10pt depth 2pt \cr\cline{3-4}
\end{tabular}
&
\begin{tabular}{ll}
0.784 & Accuracy \cr 
0.785 & Balanced Accuracy \cr 
0.377 & Precision \cr 
0.784 & Balanced Precision \cr 
0.786 & Recall \cr 
0.510 & F1 \cr 
0.785 & Balanced F1 \cr 
0.543 & Gmean \cr 	\end{tabular}
\end{tabular}
\end{center}

Another way is to linearly transform the probabilities.   Whether the distribution was clustered to the left or right, or clustered at the center, is not necessarily relevant, so we want to see it spread out.  We have arbitrarily chosen a transformation to put next the original models in our results to see if it will make a better model; tuning the transformation is an avenue for future work.  We have chosen to take the 0.05 quantile of the negative class and map it to $p=0.05$, and the 0.95 quantile of the positive class and map it to $p=0.95$.  This linear transformation gives the same metrics as the shift, and the ROC curve is the same except for the two labeled medians, now at 0.305 and 0.688.

\begin{center}
\begin{tabular}{p{0.5\textwidth} p{0.5\textwidth}}
  \vspace{0pt} %% Creator: Matplotlib, PGF backend
%%
%% To include the figure in your LaTeX document, write
%%   \input{<filename>.pgf}
%%
%% Make sure the required packages are loaded in your preamble
%%   \usepackage{pgf}
%%
%% Also ensure that all the required font packages are loaded; for instance,
%% the lmodern package is sometimes necessary when using math font.
%%   \usepackage{lmodern}
%%
%% Figures using additional raster images can only be included by \input if
%% they are in the same directory as the main LaTeX file. For loading figures
%% from other directories you can use the `import` package
%%   \usepackage{import}
%%
%% and then include the figures with
%%   \import{<path to file>}{<filename>.pgf}
%%
%% Matplotlib used the following preamble
%%   
%%   \usepackage{fontspec}
%%   \makeatletter\@ifpackageloaded{underscore}{}{\usepackage[strings]{underscore}}\makeatother
%%
\begingroup%
\makeatletter%
\begin{pgfpicture}%
\pgfpathrectangle{\pgfpointorigin}{\pgfqpoint{3.095000in}{3.243944in}}%
\pgfusepath{use as bounding box, clip}%
\begin{pgfscope}%
\pgfsetbuttcap%
\pgfsetmiterjoin%
\definecolor{currentfill}{rgb}{1.000000,1.000000,1.000000}%
\pgfsetfillcolor{currentfill}%
\pgfsetlinewidth{0.000000pt}%
\definecolor{currentstroke}{rgb}{1.000000,1.000000,1.000000}%
\pgfsetstrokecolor{currentstroke}%
\pgfsetdash{}{0pt}%
\pgfpathmoveto{\pgfqpoint{0.000000in}{0.000000in}}%
\pgfpathlineto{\pgfqpoint{3.095000in}{0.000000in}}%
\pgfpathlineto{\pgfqpoint{3.095000in}{3.243944in}}%
\pgfpathlineto{\pgfqpoint{0.000000in}{3.243944in}}%
\pgfpathlineto{\pgfqpoint{0.000000in}{0.000000in}}%
\pgfpathclose%
\pgfusepath{fill}%
\end{pgfscope}%
\begin{pgfscope}%
\pgfsetbuttcap%
\pgfsetmiterjoin%
\definecolor{currentfill}{rgb}{1.000000,1.000000,1.000000}%
\pgfsetfillcolor{currentfill}%
\pgfsetlinewidth{0.000000pt}%
\definecolor{currentstroke}{rgb}{0.000000,0.000000,0.000000}%
\pgfsetstrokecolor{currentstroke}%
\pgfsetstrokeopacity{0.000000}%
\pgfsetdash{}{0pt}%
\pgfpathmoveto{\pgfqpoint{0.515000in}{1.096944in}}%
\pgfpathlineto{\pgfqpoint{2.995000in}{1.096944in}}%
\pgfpathlineto{\pgfqpoint{2.995000in}{2.944944in}}%
\pgfpathlineto{\pgfqpoint{0.515000in}{2.944944in}}%
\pgfpathlineto{\pgfqpoint{0.515000in}{1.096944in}}%
\pgfpathclose%
\pgfusepath{fill}%
\end{pgfscope}%
\begin{pgfscope}%
\pgfpathrectangle{\pgfqpoint{0.515000in}{1.096944in}}{\pgfqpoint{2.480000in}{1.848000in}}%
\pgfusepath{clip}%
\pgfsetbuttcap%
\pgfsetmiterjoin%
\pgfsetlinewidth{1.003750pt}%
\definecolor{currentstroke}{rgb}{0.000000,0.000000,0.000000}%
\pgfsetstrokecolor{currentstroke}%
\pgfsetdash{}{0pt}%
\pgfpathmoveto{\pgfqpoint{0.505000in}{1.096944in}}%
\pgfpathlineto{\pgfqpoint{0.577627in}{1.096944in}}%
\pgfpathlineto{\pgfqpoint{0.577627in}{2.115939in}}%
\pgfpathlineto{\pgfqpoint{0.505000in}{2.115939in}}%
\pgfusepath{stroke}%
\end{pgfscope}%
\begin{pgfscope}%
\pgfpathrectangle{\pgfqpoint{0.515000in}{1.096944in}}{\pgfqpoint{2.480000in}{1.848000in}}%
\pgfusepath{clip}%
\pgfsetbuttcap%
\pgfsetmiterjoin%
\pgfsetlinewidth{1.003750pt}%
\definecolor{currentstroke}{rgb}{0.000000,0.000000,0.000000}%
\pgfsetstrokecolor{currentstroke}%
\pgfsetdash{}{0pt}%
\pgfpathmoveto{\pgfqpoint{0.727930in}{1.096944in}}%
\pgfpathlineto{\pgfqpoint{0.828132in}{1.096944in}}%
\pgfpathlineto{\pgfqpoint{0.828132in}{2.697147in}}%
\pgfpathlineto{\pgfqpoint{0.727930in}{2.697147in}}%
\pgfpathlineto{\pgfqpoint{0.727930in}{1.096944in}}%
\pgfpathclose%
\pgfusepath{stroke}%
\end{pgfscope}%
\begin{pgfscope}%
\pgfpathrectangle{\pgfqpoint{0.515000in}{1.096944in}}{\pgfqpoint{2.480000in}{1.848000in}}%
\pgfusepath{clip}%
\pgfsetbuttcap%
\pgfsetmiterjoin%
\pgfsetlinewidth{1.003750pt}%
\definecolor{currentstroke}{rgb}{0.000000,0.000000,0.000000}%
\pgfsetstrokecolor{currentstroke}%
\pgfsetdash{}{0pt}%
\pgfpathmoveto{\pgfqpoint{0.978435in}{1.096944in}}%
\pgfpathlineto{\pgfqpoint{1.078637in}{1.096944in}}%
\pgfpathlineto{\pgfqpoint{1.078637in}{2.856944in}}%
\pgfpathlineto{\pgfqpoint{0.978435in}{2.856944in}}%
\pgfpathlineto{\pgfqpoint{0.978435in}{1.096944in}}%
\pgfpathclose%
\pgfusepath{stroke}%
\end{pgfscope}%
\begin{pgfscope}%
\pgfpathrectangle{\pgfqpoint{0.515000in}{1.096944in}}{\pgfqpoint{2.480000in}{1.848000in}}%
\pgfusepath{clip}%
\pgfsetbuttcap%
\pgfsetmiterjoin%
\pgfsetlinewidth{1.003750pt}%
\definecolor{currentstroke}{rgb}{0.000000,0.000000,0.000000}%
\pgfsetstrokecolor{currentstroke}%
\pgfsetdash{}{0pt}%
\pgfpathmoveto{\pgfqpoint{1.228940in}{1.096944in}}%
\pgfpathlineto{\pgfqpoint{1.329142in}{1.096944in}}%
\pgfpathlineto{\pgfqpoint{1.329142in}{2.617137in}}%
\pgfpathlineto{\pgfqpoint{1.228940in}{2.617137in}}%
\pgfpathlineto{\pgfqpoint{1.228940in}{1.096944in}}%
\pgfpathclose%
\pgfusepath{stroke}%
\end{pgfscope}%
\begin{pgfscope}%
\pgfpathrectangle{\pgfqpoint{0.515000in}{1.096944in}}{\pgfqpoint{2.480000in}{1.848000in}}%
\pgfusepath{clip}%
\pgfsetbuttcap%
\pgfsetmiterjoin%
\pgfsetlinewidth{1.003750pt}%
\definecolor{currentstroke}{rgb}{0.000000,0.000000,0.000000}%
\pgfsetstrokecolor{currentstroke}%
\pgfsetdash{}{0pt}%
\pgfpathmoveto{\pgfqpoint{1.479445in}{1.096944in}}%
\pgfpathlineto{\pgfqpoint{1.579647in}{1.096944in}}%
\pgfpathlineto{\pgfqpoint{1.579647in}{2.230781in}}%
\pgfpathlineto{\pgfqpoint{1.479445in}{2.230781in}}%
\pgfpathlineto{\pgfqpoint{1.479445in}{1.096944in}}%
\pgfpathclose%
\pgfusepath{stroke}%
\end{pgfscope}%
\begin{pgfscope}%
\pgfpathrectangle{\pgfqpoint{0.515000in}{1.096944in}}{\pgfqpoint{2.480000in}{1.848000in}}%
\pgfusepath{clip}%
\pgfsetbuttcap%
\pgfsetmiterjoin%
\pgfsetlinewidth{1.003750pt}%
\definecolor{currentstroke}{rgb}{0.000000,0.000000,0.000000}%
\pgfsetstrokecolor{currentstroke}%
\pgfsetdash{}{0pt}%
\pgfpathmoveto{\pgfqpoint{1.729950in}{1.096944in}}%
\pgfpathlineto{\pgfqpoint{1.830152in}{1.096944in}}%
\pgfpathlineto{\pgfqpoint{1.830152in}{1.847477in}}%
\pgfpathlineto{\pgfqpoint{1.729950in}{1.847477in}}%
\pgfpathlineto{\pgfqpoint{1.729950in}{1.096944in}}%
\pgfpathclose%
\pgfusepath{stroke}%
\end{pgfscope}%
\begin{pgfscope}%
\pgfpathrectangle{\pgfqpoint{0.515000in}{1.096944in}}{\pgfqpoint{2.480000in}{1.848000in}}%
\pgfusepath{clip}%
\pgfsetbuttcap%
\pgfsetmiterjoin%
\pgfsetlinewidth{1.003750pt}%
\definecolor{currentstroke}{rgb}{0.000000,0.000000,0.000000}%
\pgfsetstrokecolor{currentstroke}%
\pgfsetdash{}{0pt}%
\pgfpathmoveto{\pgfqpoint{1.980455in}{1.096944in}}%
\pgfpathlineto{\pgfqpoint{2.080657in}{1.096944in}}%
\pgfpathlineto{\pgfqpoint{2.080657in}{1.652029in}}%
\pgfpathlineto{\pgfqpoint{1.980455in}{1.652029in}}%
\pgfpathlineto{\pgfqpoint{1.980455in}{1.096944in}}%
\pgfpathclose%
\pgfusepath{stroke}%
\end{pgfscope}%
\begin{pgfscope}%
\pgfpathrectangle{\pgfqpoint{0.515000in}{1.096944in}}{\pgfqpoint{2.480000in}{1.848000in}}%
\pgfusepath{clip}%
\pgfsetbuttcap%
\pgfsetmiterjoin%
\pgfsetlinewidth{1.003750pt}%
\definecolor{currentstroke}{rgb}{0.000000,0.000000,0.000000}%
\pgfsetstrokecolor{currentstroke}%
\pgfsetdash{}{0pt}%
\pgfpathmoveto{\pgfqpoint{2.230960in}{1.096944in}}%
\pgfpathlineto{\pgfqpoint{2.331162in}{1.096944in}}%
\pgfpathlineto{\pgfqpoint{2.331162in}{1.382897in}}%
\pgfpathlineto{\pgfqpoint{2.230960in}{1.382897in}}%
\pgfpathlineto{\pgfqpoint{2.230960in}{1.096944in}}%
\pgfpathclose%
\pgfusepath{stroke}%
\end{pgfscope}%
\begin{pgfscope}%
\pgfpathrectangle{\pgfqpoint{0.515000in}{1.096944in}}{\pgfqpoint{2.480000in}{1.848000in}}%
\pgfusepath{clip}%
\pgfsetbuttcap%
\pgfsetmiterjoin%
\pgfsetlinewidth{1.003750pt}%
\definecolor{currentstroke}{rgb}{0.000000,0.000000,0.000000}%
\pgfsetstrokecolor{currentstroke}%
\pgfsetdash{}{0pt}%
\pgfpathmoveto{\pgfqpoint{2.481465in}{1.096944in}}%
\pgfpathlineto{\pgfqpoint{2.581667in}{1.096944in}}%
\pgfpathlineto{\pgfqpoint{2.581667in}{1.262919in}}%
\pgfpathlineto{\pgfqpoint{2.481465in}{1.262919in}}%
\pgfpathlineto{\pgfqpoint{2.481465in}{1.096944in}}%
\pgfpathclose%
\pgfusepath{stroke}%
\end{pgfscope}%
\begin{pgfscope}%
\pgfpathrectangle{\pgfqpoint{0.515000in}{1.096944in}}{\pgfqpoint{2.480000in}{1.848000in}}%
\pgfusepath{clip}%
\pgfsetbuttcap%
\pgfsetmiterjoin%
\pgfsetlinewidth{1.003750pt}%
\definecolor{currentstroke}{rgb}{0.000000,0.000000,0.000000}%
\pgfsetstrokecolor{currentstroke}%
\pgfsetdash{}{0pt}%
\pgfpathmoveto{\pgfqpoint{2.731970in}{1.096944in}}%
\pgfpathlineto{\pgfqpoint{2.832172in}{1.096944in}}%
\pgfpathlineto{\pgfqpoint{2.832172in}{1.237539in}}%
\pgfpathlineto{\pgfqpoint{2.731970in}{1.237539in}}%
\pgfpathlineto{\pgfqpoint{2.731970in}{1.096944in}}%
\pgfpathclose%
\pgfusepath{stroke}%
\end{pgfscope}%
\begin{pgfscope}%
\pgfpathrectangle{\pgfqpoint{0.515000in}{1.096944in}}{\pgfqpoint{2.480000in}{1.848000in}}%
\pgfusepath{clip}%
\pgfsetbuttcap%
\pgfsetmiterjoin%
\definecolor{currentfill}{rgb}{0.000000,0.000000,0.000000}%
\pgfsetfillcolor{currentfill}%
\pgfsetlinewidth{0.000000pt}%
\definecolor{currentstroke}{rgb}{0.000000,0.000000,0.000000}%
\pgfsetstrokecolor{currentstroke}%
\pgfsetstrokeopacity{0.000000}%
\pgfsetdash{}{0pt}%
\pgfpathmoveto{\pgfqpoint{0.577627in}{1.096944in}}%
\pgfpathlineto{\pgfqpoint{0.677829in}{1.096944in}}%
\pgfpathlineto{\pgfqpoint{0.677829in}{1.122920in}}%
\pgfpathlineto{\pgfqpoint{0.577627in}{1.122920in}}%
\pgfpathlineto{\pgfqpoint{0.577627in}{1.096944in}}%
\pgfpathclose%
\pgfusepath{fill}%
\end{pgfscope}%
\begin{pgfscope}%
\pgfpathrectangle{\pgfqpoint{0.515000in}{1.096944in}}{\pgfqpoint{2.480000in}{1.848000in}}%
\pgfusepath{clip}%
\pgfsetbuttcap%
\pgfsetmiterjoin%
\definecolor{currentfill}{rgb}{0.000000,0.000000,0.000000}%
\pgfsetfillcolor{currentfill}%
\pgfsetlinewidth{0.000000pt}%
\definecolor{currentstroke}{rgb}{0.000000,0.000000,0.000000}%
\pgfsetstrokecolor{currentstroke}%
\pgfsetstrokeopacity{0.000000}%
\pgfsetdash{}{0pt}%
\pgfpathmoveto{\pgfqpoint{0.828132in}{1.096944in}}%
\pgfpathlineto{\pgfqpoint{0.928334in}{1.096944in}}%
\pgfpathlineto{\pgfqpoint{0.928334in}{1.126864in}}%
\pgfpathlineto{\pgfqpoint{0.828132in}{1.126864in}}%
\pgfpathlineto{\pgfqpoint{0.828132in}{1.096944in}}%
\pgfpathclose%
\pgfusepath{fill}%
\end{pgfscope}%
\begin{pgfscope}%
\pgfpathrectangle{\pgfqpoint{0.515000in}{1.096944in}}{\pgfqpoint{2.480000in}{1.848000in}}%
\pgfusepath{clip}%
\pgfsetbuttcap%
\pgfsetmiterjoin%
\definecolor{currentfill}{rgb}{0.000000,0.000000,0.000000}%
\pgfsetfillcolor{currentfill}%
\pgfsetlinewidth{0.000000pt}%
\definecolor{currentstroke}{rgb}{0.000000,0.000000,0.000000}%
\pgfsetstrokecolor{currentstroke}%
\pgfsetstrokeopacity{0.000000}%
\pgfsetdash{}{0pt}%
\pgfpathmoveto{\pgfqpoint{1.078637in}{1.096944in}}%
\pgfpathlineto{\pgfqpoint{1.178839in}{1.096944in}}%
\pgfpathlineto{\pgfqpoint{1.178839in}{1.148821in}}%
\pgfpathlineto{\pgfqpoint{1.078637in}{1.148821in}}%
\pgfpathlineto{\pgfqpoint{1.078637in}{1.096944in}}%
\pgfpathclose%
\pgfusepath{fill}%
\end{pgfscope}%
\begin{pgfscope}%
\pgfpathrectangle{\pgfqpoint{0.515000in}{1.096944in}}{\pgfqpoint{2.480000in}{1.848000in}}%
\pgfusepath{clip}%
\pgfsetbuttcap%
\pgfsetmiterjoin%
\definecolor{currentfill}{rgb}{0.000000,0.000000,0.000000}%
\pgfsetfillcolor{currentfill}%
\pgfsetlinewidth{0.000000pt}%
\definecolor{currentstroke}{rgb}{0.000000,0.000000,0.000000}%
\pgfsetstrokecolor{currentstroke}%
\pgfsetstrokeopacity{0.000000}%
\pgfsetdash{}{0pt}%
\pgfpathmoveto{\pgfqpoint{1.329142in}{1.096944in}}%
\pgfpathlineto{\pgfqpoint{1.429344in}{1.096944in}}%
\pgfpathlineto{\pgfqpoint{1.429344in}{1.179783in}}%
\pgfpathlineto{\pgfqpoint{1.329142in}{1.179783in}}%
\pgfpathlineto{\pgfqpoint{1.329142in}{1.096944in}}%
\pgfpathclose%
\pgfusepath{fill}%
\end{pgfscope}%
\begin{pgfscope}%
\pgfpathrectangle{\pgfqpoint{0.515000in}{1.096944in}}{\pgfqpoint{2.480000in}{1.848000in}}%
\pgfusepath{clip}%
\pgfsetbuttcap%
\pgfsetmiterjoin%
\definecolor{currentfill}{rgb}{0.000000,0.000000,0.000000}%
\pgfsetfillcolor{currentfill}%
\pgfsetlinewidth{0.000000pt}%
\definecolor{currentstroke}{rgb}{0.000000,0.000000,0.000000}%
\pgfsetstrokecolor{currentstroke}%
\pgfsetstrokeopacity{0.000000}%
\pgfsetdash{}{0pt}%
\pgfpathmoveto{\pgfqpoint{1.579647in}{1.096944in}}%
\pgfpathlineto{\pgfqpoint{1.679849in}{1.096944in}}%
\pgfpathlineto{\pgfqpoint{1.679849in}{1.230543in}}%
\pgfpathlineto{\pgfqpoint{1.579647in}{1.230543in}}%
\pgfpathlineto{\pgfqpoint{1.579647in}{1.096944in}}%
\pgfpathclose%
\pgfusepath{fill}%
\end{pgfscope}%
\begin{pgfscope}%
\pgfpathrectangle{\pgfqpoint{0.515000in}{1.096944in}}{\pgfqpoint{2.480000in}{1.848000in}}%
\pgfusepath{clip}%
\pgfsetbuttcap%
\pgfsetmiterjoin%
\definecolor{currentfill}{rgb}{0.000000,0.000000,0.000000}%
\pgfsetfillcolor{currentfill}%
\pgfsetlinewidth{0.000000pt}%
\definecolor{currentstroke}{rgb}{0.000000,0.000000,0.000000}%
\pgfsetstrokecolor{currentstroke}%
\pgfsetstrokeopacity{0.000000}%
\pgfsetdash{}{0pt}%
\pgfpathmoveto{\pgfqpoint{1.830152in}{1.096944in}}%
\pgfpathlineto{\pgfqpoint{1.930354in}{1.096944in}}%
\pgfpathlineto{\pgfqpoint{1.930354in}{1.292839in}}%
\pgfpathlineto{\pgfqpoint{1.830152in}{1.292839in}}%
\pgfpathlineto{\pgfqpoint{1.830152in}{1.096944in}}%
\pgfpathclose%
\pgfusepath{fill}%
\end{pgfscope}%
\begin{pgfscope}%
\pgfpathrectangle{\pgfqpoint{0.515000in}{1.096944in}}{\pgfqpoint{2.480000in}{1.848000in}}%
\pgfusepath{clip}%
\pgfsetbuttcap%
\pgfsetmiterjoin%
\definecolor{currentfill}{rgb}{0.000000,0.000000,0.000000}%
\pgfsetfillcolor{currentfill}%
\pgfsetlinewidth{0.000000pt}%
\definecolor{currentstroke}{rgb}{0.000000,0.000000,0.000000}%
\pgfsetstrokecolor{currentstroke}%
\pgfsetstrokeopacity{0.000000}%
\pgfsetdash{}{0pt}%
\pgfpathmoveto{\pgfqpoint{2.080657in}{1.096944in}}%
\pgfpathlineto{\pgfqpoint{2.180859in}{1.096944in}}%
\pgfpathlineto{\pgfqpoint{2.180859in}{1.353646in}}%
\pgfpathlineto{\pgfqpoint{2.080657in}{1.353646in}}%
\pgfpathlineto{\pgfqpoint{2.080657in}{1.096944in}}%
\pgfpathclose%
\pgfusepath{fill}%
\end{pgfscope}%
\begin{pgfscope}%
\pgfpathrectangle{\pgfqpoint{0.515000in}{1.096944in}}{\pgfqpoint{2.480000in}{1.848000in}}%
\pgfusepath{clip}%
\pgfsetbuttcap%
\pgfsetmiterjoin%
\definecolor{currentfill}{rgb}{0.000000,0.000000,0.000000}%
\pgfsetfillcolor{currentfill}%
\pgfsetlinewidth{0.000000pt}%
\definecolor{currentstroke}{rgb}{0.000000,0.000000,0.000000}%
\pgfsetstrokecolor{currentstroke}%
\pgfsetstrokeopacity{0.000000}%
\pgfsetdash{}{0pt}%
\pgfpathmoveto{\pgfqpoint{2.331162in}{1.096944in}}%
\pgfpathlineto{\pgfqpoint{2.431364in}{1.096944in}}%
\pgfpathlineto{\pgfqpoint{2.431364in}{1.385948in}}%
\pgfpathlineto{\pgfqpoint{2.331162in}{1.385948in}}%
\pgfpathlineto{\pgfqpoint{2.331162in}{1.096944in}}%
\pgfpathclose%
\pgfusepath{fill}%
\end{pgfscope}%
\begin{pgfscope}%
\pgfpathrectangle{\pgfqpoint{0.515000in}{1.096944in}}{\pgfqpoint{2.480000in}{1.848000in}}%
\pgfusepath{clip}%
\pgfsetbuttcap%
\pgfsetmiterjoin%
\definecolor{currentfill}{rgb}{0.000000,0.000000,0.000000}%
\pgfsetfillcolor{currentfill}%
\pgfsetlinewidth{0.000000pt}%
\definecolor{currentstroke}{rgb}{0.000000,0.000000,0.000000}%
\pgfsetstrokecolor{currentstroke}%
\pgfsetstrokeopacity{0.000000}%
\pgfsetdash{}{0pt}%
\pgfpathmoveto{\pgfqpoint{2.581667in}{1.096944in}}%
\pgfpathlineto{\pgfqpoint{2.681869in}{1.096944in}}%
\pgfpathlineto{\pgfqpoint{2.681869in}{1.360494in}}%
\pgfpathlineto{\pgfqpoint{2.581667in}{1.360494in}}%
\pgfpathlineto{\pgfqpoint{2.581667in}{1.096944in}}%
\pgfpathclose%
\pgfusepath{fill}%
\end{pgfscope}%
\begin{pgfscope}%
\pgfpathrectangle{\pgfqpoint{0.515000in}{1.096944in}}{\pgfqpoint{2.480000in}{1.848000in}}%
\pgfusepath{clip}%
\pgfsetbuttcap%
\pgfsetmiterjoin%
\definecolor{currentfill}{rgb}{0.000000,0.000000,0.000000}%
\pgfsetfillcolor{currentfill}%
\pgfsetlinewidth{0.000000pt}%
\definecolor{currentstroke}{rgb}{0.000000,0.000000,0.000000}%
\pgfsetstrokecolor{currentstroke}%
\pgfsetstrokeopacity{0.000000}%
\pgfsetdash{}{0pt}%
\pgfpathmoveto{\pgfqpoint{2.832172in}{1.096944in}}%
\pgfpathlineto{\pgfqpoint{2.932374in}{1.096944in}}%
\pgfpathlineto{\pgfqpoint{2.932374in}{1.256146in}}%
\pgfpathlineto{\pgfqpoint{2.832172in}{1.256146in}}%
\pgfpathlineto{\pgfqpoint{2.832172in}{1.096944in}}%
\pgfpathclose%
\pgfusepath{fill}%
\end{pgfscope}%
\begin{pgfscope}%
\pgfsetbuttcap%
\pgfsetroundjoin%
\definecolor{currentfill}{rgb}{0.000000,0.000000,0.000000}%
\pgfsetfillcolor{currentfill}%
\pgfsetlinewidth{0.803000pt}%
\definecolor{currentstroke}{rgb}{0.000000,0.000000,0.000000}%
\pgfsetstrokecolor{currentstroke}%
\pgfsetdash{}{0pt}%
\pgfsys@defobject{currentmarker}{\pgfqpoint{0.000000in}{-0.048611in}}{\pgfqpoint{0.000000in}{0.000000in}}{%
\pgfpathmoveto{\pgfqpoint{0.000000in}{0.000000in}}%
\pgfpathlineto{\pgfqpoint{0.000000in}{-0.048611in}}%
\pgfusepath{stroke,fill}%
}%
\begin{pgfscope}%
\pgfsys@transformshift{0.577627in}{1.096944in}%
\pgfsys@useobject{currentmarker}{}%
\end{pgfscope}%
\end{pgfscope}%
\begin{pgfscope}%
\definecolor{textcolor}{rgb}{0.000000,0.000000,0.000000}%
\pgfsetstrokecolor{textcolor}%
\pgfsetfillcolor{textcolor}%
\pgftext[x=0.612349in, y=0.282083in, left, base,rotate=90.000000]{\color{textcolor}\rmfamily\fontsize{10.000000}{12.000000}\selectfont (-0.001, 0.1]}%
\end{pgfscope}%
\begin{pgfscope}%
\pgfsetbuttcap%
\pgfsetroundjoin%
\definecolor{currentfill}{rgb}{0.000000,0.000000,0.000000}%
\pgfsetfillcolor{currentfill}%
\pgfsetlinewidth{0.803000pt}%
\definecolor{currentstroke}{rgb}{0.000000,0.000000,0.000000}%
\pgfsetstrokecolor{currentstroke}%
\pgfsetdash{}{0pt}%
\pgfsys@defobject{currentmarker}{\pgfqpoint{0.000000in}{-0.048611in}}{\pgfqpoint{0.000000in}{0.000000in}}{%
\pgfpathmoveto{\pgfqpoint{0.000000in}{0.000000in}}%
\pgfpathlineto{\pgfqpoint{0.000000in}{-0.048611in}}%
\pgfusepath{stroke,fill}%
}%
\begin{pgfscope}%
\pgfsys@transformshift{0.828132in}{1.096944in}%
\pgfsys@useobject{currentmarker}{}%
\end{pgfscope}%
\end{pgfscope}%
\begin{pgfscope}%
\definecolor{textcolor}{rgb}{0.000000,0.000000,0.000000}%
\pgfsetstrokecolor{textcolor}%
\pgfsetfillcolor{textcolor}%
\pgftext[x=0.862854in, y=0.467222in, left, base,rotate=90.000000]{\color{textcolor}\rmfamily\fontsize{10.000000}{12.000000}\selectfont (0.1, 0.2]}%
\end{pgfscope}%
\begin{pgfscope}%
\pgfsetbuttcap%
\pgfsetroundjoin%
\definecolor{currentfill}{rgb}{0.000000,0.000000,0.000000}%
\pgfsetfillcolor{currentfill}%
\pgfsetlinewidth{0.803000pt}%
\definecolor{currentstroke}{rgb}{0.000000,0.000000,0.000000}%
\pgfsetstrokecolor{currentstroke}%
\pgfsetdash{}{0pt}%
\pgfsys@defobject{currentmarker}{\pgfqpoint{0.000000in}{-0.048611in}}{\pgfqpoint{0.000000in}{0.000000in}}{%
\pgfpathmoveto{\pgfqpoint{0.000000in}{0.000000in}}%
\pgfpathlineto{\pgfqpoint{0.000000in}{-0.048611in}}%
\pgfusepath{stroke,fill}%
}%
\begin{pgfscope}%
\pgfsys@transformshift{1.078637in}{1.096944in}%
\pgfsys@useobject{currentmarker}{}%
\end{pgfscope}%
\end{pgfscope}%
\begin{pgfscope}%
\definecolor{textcolor}{rgb}{0.000000,0.000000,0.000000}%
\pgfsetstrokecolor{textcolor}%
\pgfsetfillcolor{textcolor}%
\pgftext[x=1.113359in, y=0.467222in, left, base,rotate=90.000000]{\color{textcolor}\rmfamily\fontsize{10.000000}{12.000000}\selectfont (0.2, 0.3]}%
\end{pgfscope}%
\begin{pgfscope}%
\pgfsetbuttcap%
\pgfsetroundjoin%
\definecolor{currentfill}{rgb}{0.000000,0.000000,0.000000}%
\pgfsetfillcolor{currentfill}%
\pgfsetlinewidth{0.803000pt}%
\definecolor{currentstroke}{rgb}{0.000000,0.000000,0.000000}%
\pgfsetstrokecolor{currentstroke}%
\pgfsetdash{}{0pt}%
\pgfsys@defobject{currentmarker}{\pgfqpoint{0.000000in}{-0.048611in}}{\pgfqpoint{0.000000in}{0.000000in}}{%
\pgfpathmoveto{\pgfqpoint{0.000000in}{0.000000in}}%
\pgfpathlineto{\pgfqpoint{0.000000in}{-0.048611in}}%
\pgfusepath{stroke,fill}%
}%
\begin{pgfscope}%
\pgfsys@transformshift{1.329142in}{1.096944in}%
\pgfsys@useobject{currentmarker}{}%
\end{pgfscope}%
\end{pgfscope}%
\begin{pgfscope}%
\definecolor{textcolor}{rgb}{0.000000,0.000000,0.000000}%
\pgfsetstrokecolor{textcolor}%
\pgfsetfillcolor{textcolor}%
\pgftext[x=1.363864in, y=0.467222in, left, base,rotate=90.000000]{\color{textcolor}\rmfamily\fontsize{10.000000}{12.000000}\selectfont (0.3, 0.4]}%
\end{pgfscope}%
\begin{pgfscope}%
\pgfsetbuttcap%
\pgfsetroundjoin%
\definecolor{currentfill}{rgb}{0.000000,0.000000,0.000000}%
\pgfsetfillcolor{currentfill}%
\pgfsetlinewidth{0.803000pt}%
\definecolor{currentstroke}{rgb}{0.000000,0.000000,0.000000}%
\pgfsetstrokecolor{currentstroke}%
\pgfsetdash{}{0pt}%
\pgfsys@defobject{currentmarker}{\pgfqpoint{0.000000in}{-0.048611in}}{\pgfqpoint{0.000000in}{0.000000in}}{%
\pgfpathmoveto{\pgfqpoint{0.000000in}{0.000000in}}%
\pgfpathlineto{\pgfqpoint{0.000000in}{-0.048611in}}%
\pgfusepath{stroke,fill}%
}%
\begin{pgfscope}%
\pgfsys@transformshift{1.579647in}{1.096944in}%
\pgfsys@useobject{currentmarker}{}%
\end{pgfscope}%
\end{pgfscope}%
\begin{pgfscope}%
\definecolor{textcolor}{rgb}{0.000000,0.000000,0.000000}%
\pgfsetstrokecolor{textcolor}%
\pgfsetfillcolor{textcolor}%
\pgftext[x=1.614369in, y=0.467222in, left, base,rotate=90.000000]{\color{textcolor}\rmfamily\fontsize{10.000000}{12.000000}\selectfont (0.4, 0.5]}%
\end{pgfscope}%
\begin{pgfscope}%
\pgfsetbuttcap%
\pgfsetroundjoin%
\definecolor{currentfill}{rgb}{0.000000,0.000000,0.000000}%
\pgfsetfillcolor{currentfill}%
\pgfsetlinewidth{0.803000pt}%
\definecolor{currentstroke}{rgb}{0.000000,0.000000,0.000000}%
\pgfsetstrokecolor{currentstroke}%
\pgfsetdash{}{0pt}%
\pgfsys@defobject{currentmarker}{\pgfqpoint{0.000000in}{-0.048611in}}{\pgfqpoint{0.000000in}{0.000000in}}{%
\pgfpathmoveto{\pgfqpoint{0.000000in}{0.000000in}}%
\pgfpathlineto{\pgfqpoint{0.000000in}{-0.048611in}}%
\pgfusepath{stroke,fill}%
}%
\begin{pgfscope}%
\pgfsys@transformshift{1.830152in}{1.096944in}%
\pgfsys@useobject{currentmarker}{}%
\end{pgfscope}%
\end{pgfscope}%
\begin{pgfscope}%
\definecolor{textcolor}{rgb}{0.000000,0.000000,0.000000}%
\pgfsetstrokecolor{textcolor}%
\pgfsetfillcolor{textcolor}%
\pgftext[x=1.864874in, y=0.467222in, left, base,rotate=90.000000]{\color{textcolor}\rmfamily\fontsize{10.000000}{12.000000}\selectfont (0.5, 0.6]}%
\end{pgfscope}%
\begin{pgfscope}%
\pgfsetbuttcap%
\pgfsetroundjoin%
\definecolor{currentfill}{rgb}{0.000000,0.000000,0.000000}%
\pgfsetfillcolor{currentfill}%
\pgfsetlinewidth{0.803000pt}%
\definecolor{currentstroke}{rgb}{0.000000,0.000000,0.000000}%
\pgfsetstrokecolor{currentstroke}%
\pgfsetdash{}{0pt}%
\pgfsys@defobject{currentmarker}{\pgfqpoint{0.000000in}{-0.048611in}}{\pgfqpoint{0.000000in}{0.000000in}}{%
\pgfpathmoveto{\pgfqpoint{0.000000in}{0.000000in}}%
\pgfpathlineto{\pgfqpoint{0.000000in}{-0.048611in}}%
\pgfusepath{stroke,fill}%
}%
\begin{pgfscope}%
\pgfsys@transformshift{2.080657in}{1.096944in}%
\pgfsys@useobject{currentmarker}{}%
\end{pgfscope}%
\end{pgfscope}%
\begin{pgfscope}%
\definecolor{textcolor}{rgb}{0.000000,0.000000,0.000000}%
\pgfsetstrokecolor{textcolor}%
\pgfsetfillcolor{textcolor}%
\pgftext[x=2.115379in, y=0.467222in, left, base,rotate=90.000000]{\color{textcolor}\rmfamily\fontsize{10.000000}{12.000000}\selectfont (0.6, 0.7]}%
\end{pgfscope}%
\begin{pgfscope}%
\pgfsetbuttcap%
\pgfsetroundjoin%
\definecolor{currentfill}{rgb}{0.000000,0.000000,0.000000}%
\pgfsetfillcolor{currentfill}%
\pgfsetlinewidth{0.803000pt}%
\definecolor{currentstroke}{rgb}{0.000000,0.000000,0.000000}%
\pgfsetstrokecolor{currentstroke}%
\pgfsetdash{}{0pt}%
\pgfsys@defobject{currentmarker}{\pgfqpoint{0.000000in}{-0.048611in}}{\pgfqpoint{0.000000in}{0.000000in}}{%
\pgfpathmoveto{\pgfqpoint{0.000000in}{0.000000in}}%
\pgfpathlineto{\pgfqpoint{0.000000in}{-0.048611in}}%
\pgfusepath{stroke,fill}%
}%
\begin{pgfscope}%
\pgfsys@transformshift{2.331162in}{1.096944in}%
\pgfsys@useobject{currentmarker}{}%
\end{pgfscope}%
\end{pgfscope}%
\begin{pgfscope}%
\definecolor{textcolor}{rgb}{0.000000,0.000000,0.000000}%
\pgfsetstrokecolor{textcolor}%
\pgfsetfillcolor{textcolor}%
\pgftext[x=2.365884in, y=0.467222in, left, base,rotate=90.000000]{\color{textcolor}\rmfamily\fontsize{10.000000}{12.000000}\selectfont (0.7, 0.8]}%
\end{pgfscope}%
\begin{pgfscope}%
\pgfsetbuttcap%
\pgfsetroundjoin%
\definecolor{currentfill}{rgb}{0.000000,0.000000,0.000000}%
\pgfsetfillcolor{currentfill}%
\pgfsetlinewidth{0.803000pt}%
\definecolor{currentstroke}{rgb}{0.000000,0.000000,0.000000}%
\pgfsetstrokecolor{currentstroke}%
\pgfsetdash{}{0pt}%
\pgfsys@defobject{currentmarker}{\pgfqpoint{0.000000in}{-0.048611in}}{\pgfqpoint{0.000000in}{0.000000in}}{%
\pgfpathmoveto{\pgfqpoint{0.000000in}{0.000000in}}%
\pgfpathlineto{\pgfqpoint{0.000000in}{-0.048611in}}%
\pgfusepath{stroke,fill}%
}%
\begin{pgfscope}%
\pgfsys@transformshift{2.581667in}{1.096944in}%
\pgfsys@useobject{currentmarker}{}%
\end{pgfscope}%
\end{pgfscope}%
\begin{pgfscope}%
\definecolor{textcolor}{rgb}{0.000000,0.000000,0.000000}%
\pgfsetstrokecolor{textcolor}%
\pgfsetfillcolor{textcolor}%
\pgftext[x=2.616389in, y=0.467222in, left, base,rotate=90.000000]{\color{textcolor}\rmfamily\fontsize{10.000000}{12.000000}\selectfont (0.8, 0.9]}%
\end{pgfscope}%
\begin{pgfscope}%
\pgfsetbuttcap%
\pgfsetroundjoin%
\definecolor{currentfill}{rgb}{0.000000,0.000000,0.000000}%
\pgfsetfillcolor{currentfill}%
\pgfsetlinewidth{0.803000pt}%
\definecolor{currentstroke}{rgb}{0.000000,0.000000,0.000000}%
\pgfsetstrokecolor{currentstroke}%
\pgfsetdash{}{0pt}%
\pgfsys@defobject{currentmarker}{\pgfqpoint{0.000000in}{-0.048611in}}{\pgfqpoint{0.000000in}{0.000000in}}{%
\pgfpathmoveto{\pgfqpoint{0.000000in}{0.000000in}}%
\pgfpathlineto{\pgfqpoint{0.000000in}{-0.048611in}}%
\pgfusepath{stroke,fill}%
}%
\begin{pgfscope}%
\pgfsys@transformshift{2.832172in}{1.096944in}%
\pgfsys@useobject{currentmarker}{}%
\end{pgfscope}%
\end{pgfscope}%
\begin{pgfscope}%
\definecolor{textcolor}{rgb}{0.000000,0.000000,0.000000}%
\pgfsetstrokecolor{textcolor}%
\pgfsetfillcolor{textcolor}%
\pgftext[x=2.866894in, y=0.467222in, left, base,rotate=90.000000]{\color{textcolor}\rmfamily\fontsize{10.000000}{12.000000}\selectfont (0.9, 1.0]}%
\end{pgfscope}%
\begin{pgfscope}%
\definecolor{textcolor}{rgb}{0.000000,0.000000,0.000000}%
\pgfsetstrokecolor{textcolor}%
\pgfsetfillcolor{textcolor}%
\pgftext[x=1.755000in,y=0.226527in,,top]{\color{textcolor}\rmfamily\fontsize{10.000000}{12.000000}\selectfont Range of Prediction}%
\end{pgfscope}%
\begin{pgfscope}%
\pgfsetbuttcap%
\pgfsetroundjoin%
\definecolor{currentfill}{rgb}{0.000000,0.000000,0.000000}%
\pgfsetfillcolor{currentfill}%
\pgfsetlinewidth{0.803000pt}%
\definecolor{currentstroke}{rgb}{0.000000,0.000000,0.000000}%
\pgfsetstrokecolor{currentstroke}%
\pgfsetdash{}{0pt}%
\pgfsys@defobject{currentmarker}{\pgfqpoint{-0.048611in}{0.000000in}}{\pgfqpoint{-0.000000in}{0.000000in}}{%
\pgfpathmoveto{\pgfqpoint{-0.000000in}{0.000000in}}%
\pgfpathlineto{\pgfqpoint{-0.048611in}{0.000000in}}%
\pgfusepath{stroke,fill}%
}%
\begin{pgfscope}%
\pgfsys@transformshift{0.515000in}{1.096944in}%
\pgfsys@useobject{currentmarker}{}%
\end{pgfscope}%
\end{pgfscope}%
\begin{pgfscope}%
\definecolor{textcolor}{rgb}{0.000000,0.000000,0.000000}%
\pgfsetstrokecolor{textcolor}%
\pgfsetfillcolor{textcolor}%
\pgftext[x=0.348333in, y=1.048750in, left, base]{\color{textcolor}\rmfamily\fontsize{10.000000}{12.000000}\selectfont \(\displaystyle {0}\)}%
\end{pgfscope}%
\begin{pgfscope}%
\pgfsetbuttcap%
\pgfsetroundjoin%
\definecolor{currentfill}{rgb}{0.000000,0.000000,0.000000}%
\pgfsetfillcolor{currentfill}%
\pgfsetlinewidth{0.803000pt}%
\definecolor{currentstroke}{rgb}{0.000000,0.000000,0.000000}%
\pgfsetstrokecolor{currentstroke}%
\pgfsetdash{}{0pt}%
\pgfsys@defobject{currentmarker}{\pgfqpoint{-0.048611in}{0.000000in}}{\pgfqpoint{-0.000000in}{0.000000in}}{%
\pgfpathmoveto{\pgfqpoint{-0.000000in}{0.000000in}}%
\pgfpathlineto{\pgfqpoint{-0.048611in}{0.000000in}}%
\pgfusepath{stroke,fill}%
}%
\begin{pgfscope}%
\pgfsys@transformshift{0.515000in}{1.617940in}%
\pgfsys@useobject{currentmarker}{}%
\end{pgfscope}%
\end{pgfscope}%
\begin{pgfscope}%
\definecolor{textcolor}{rgb}{0.000000,0.000000,0.000000}%
\pgfsetstrokecolor{textcolor}%
\pgfsetfillcolor{textcolor}%
\pgftext[x=0.348333in, y=1.569746in, left, base]{\color{textcolor}\rmfamily\fontsize{10.000000}{12.000000}\selectfont \(\displaystyle {5}\)}%
\end{pgfscope}%
\begin{pgfscope}%
\pgfsetbuttcap%
\pgfsetroundjoin%
\definecolor{currentfill}{rgb}{0.000000,0.000000,0.000000}%
\pgfsetfillcolor{currentfill}%
\pgfsetlinewidth{0.803000pt}%
\definecolor{currentstroke}{rgb}{0.000000,0.000000,0.000000}%
\pgfsetstrokecolor{currentstroke}%
\pgfsetdash{}{0pt}%
\pgfsys@defobject{currentmarker}{\pgfqpoint{-0.048611in}{0.000000in}}{\pgfqpoint{-0.000000in}{0.000000in}}{%
\pgfpathmoveto{\pgfqpoint{-0.000000in}{0.000000in}}%
\pgfpathlineto{\pgfqpoint{-0.048611in}{0.000000in}}%
\pgfusepath{stroke,fill}%
}%
\begin{pgfscope}%
\pgfsys@transformshift{0.515000in}{2.138937in}%
\pgfsys@useobject{currentmarker}{}%
\end{pgfscope}%
\end{pgfscope}%
\begin{pgfscope}%
\definecolor{textcolor}{rgb}{0.000000,0.000000,0.000000}%
\pgfsetstrokecolor{textcolor}%
\pgfsetfillcolor{textcolor}%
\pgftext[x=0.278889in, y=2.090742in, left, base]{\color{textcolor}\rmfamily\fontsize{10.000000}{12.000000}\selectfont \(\displaystyle {10}\)}%
\end{pgfscope}%
\begin{pgfscope}%
\pgfsetbuttcap%
\pgfsetroundjoin%
\definecolor{currentfill}{rgb}{0.000000,0.000000,0.000000}%
\pgfsetfillcolor{currentfill}%
\pgfsetlinewidth{0.803000pt}%
\definecolor{currentstroke}{rgb}{0.000000,0.000000,0.000000}%
\pgfsetstrokecolor{currentstroke}%
\pgfsetdash{}{0pt}%
\pgfsys@defobject{currentmarker}{\pgfqpoint{-0.048611in}{0.000000in}}{\pgfqpoint{-0.000000in}{0.000000in}}{%
\pgfpathmoveto{\pgfqpoint{-0.000000in}{0.000000in}}%
\pgfpathlineto{\pgfqpoint{-0.048611in}{0.000000in}}%
\pgfusepath{stroke,fill}%
}%
\begin{pgfscope}%
\pgfsys@transformshift{0.515000in}{2.659933in}%
\pgfsys@useobject{currentmarker}{}%
\end{pgfscope}%
\end{pgfscope}%
\begin{pgfscope}%
\definecolor{textcolor}{rgb}{0.000000,0.000000,0.000000}%
\pgfsetstrokecolor{textcolor}%
\pgfsetfillcolor{textcolor}%
\pgftext[x=0.278889in, y=2.611739in, left, base]{\color{textcolor}\rmfamily\fontsize{10.000000}{12.000000}\selectfont \(\displaystyle {15}\)}%
\end{pgfscope}%
\begin{pgfscope}%
\definecolor{textcolor}{rgb}{0.000000,0.000000,0.000000}%
\pgfsetstrokecolor{textcolor}%
\pgfsetfillcolor{textcolor}%
\pgftext[x=0.223333in,y=2.020944in,,bottom,rotate=90.000000]{\color{textcolor}\rmfamily\fontsize{10.000000}{12.000000}\selectfont Percent of Data Set}%
\end{pgfscope}%
\begin{pgfscope}%
\pgfsetrectcap%
\pgfsetmiterjoin%
\pgfsetlinewidth{0.803000pt}%
\definecolor{currentstroke}{rgb}{0.000000,0.000000,0.000000}%
\pgfsetstrokecolor{currentstroke}%
\pgfsetdash{}{0pt}%
\pgfpathmoveto{\pgfqpoint{0.515000in}{1.096944in}}%
\pgfpathlineto{\pgfqpoint{0.515000in}{2.944944in}}%
\pgfusepath{stroke}%
\end{pgfscope}%
\begin{pgfscope}%
\pgfsetrectcap%
\pgfsetmiterjoin%
\pgfsetlinewidth{0.803000pt}%
\definecolor{currentstroke}{rgb}{0.000000,0.000000,0.000000}%
\pgfsetstrokecolor{currentstroke}%
\pgfsetdash{}{0pt}%
\pgfpathmoveto{\pgfqpoint{2.995000in}{1.096944in}}%
\pgfpathlineto{\pgfqpoint{2.995000in}{2.944944in}}%
\pgfusepath{stroke}%
\end{pgfscope}%
\begin{pgfscope}%
\pgfsetrectcap%
\pgfsetmiterjoin%
\pgfsetlinewidth{0.803000pt}%
\definecolor{currentstroke}{rgb}{0.000000,0.000000,0.000000}%
\pgfsetstrokecolor{currentstroke}%
\pgfsetdash{}{0pt}%
\pgfpathmoveto{\pgfqpoint{0.515000in}{1.096944in}}%
\pgfpathlineto{\pgfqpoint{2.995000in}{1.096944in}}%
\pgfusepath{stroke}%
\end{pgfscope}%
\begin{pgfscope}%
\pgfsetrectcap%
\pgfsetmiterjoin%
\pgfsetlinewidth{0.803000pt}%
\definecolor{currentstroke}{rgb}{0.000000,0.000000,0.000000}%
\pgfsetstrokecolor{currentstroke}%
\pgfsetdash{}{0pt}%
\pgfpathmoveto{\pgfqpoint{0.515000in}{2.944944in}}%
\pgfpathlineto{\pgfqpoint{2.995000in}{2.944944in}}%
\pgfusepath{stroke}%
\end{pgfscope}%
\begin{pgfscope}%
\definecolor{textcolor}{rgb}{0.000000,0.000000,0.000000}%
\pgfsetstrokecolor{textcolor}%
\pgfsetfillcolor{textcolor}%
\pgftext[x=1.755000in,y=3.028277in,,base]{\color{textcolor}\rmfamily\fontsize{12.000000}{14.400000}\selectfont Probability Distribution}%
\end{pgfscope}%
\begin{pgfscope}%
\pgfsetbuttcap%
\pgfsetmiterjoin%
\definecolor{currentfill}{rgb}{1.000000,1.000000,1.000000}%
\pgfsetfillcolor{currentfill}%
\pgfsetfillopacity{0.800000}%
\pgfsetlinewidth{1.003750pt}%
\definecolor{currentstroke}{rgb}{0.800000,0.800000,0.800000}%
\pgfsetstrokecolor{currentstroke}%
\pgfsetstrokeopacity{0.800000}%
\pgfsetdash{}{0pt}%
\pgfpathmoveto{\pgfqpoint{1.560833in}{2.444250in}}%
\pgfpathlineto{\pgfqpoint{2.897778in}{2.444250in}}%
\pgfpathquadraticcurveto{\pgfqpoint{2.925556in}{2.444250in}}{\pgfqpoint{2.925556in}{2.472028in}}%
\pgfpathlineto{\pgfqpoint{2.925556in}{2.847722in}}%
\pgfpathquadraticcurveto{\pgfqpoint{2.925556in}{2.875500in}}{\pgfqpoint{2.897778in}{2.875500in}}%
\pgfpathlineto{\pgfqpoint{1.560833in}{2.875500in}}%
\pgfpathquadraticcurveto{\pgfqpoint{1.533056in}{2.875500in}}{\pgfqpoint{1.533056in}{2.847722in}}%
\pgfpathlineto{\pgfqpoint{1.533056in}{2.472028in}}%
\pgfpathquadraticcurveto{\pgfqpoint{1.533056in}{2.444250in}}{\pgfqpoint{1.560833in}{2.444250in}}%
\pgfpathlineto{\pgfqpoint{1.560833in}{2.444250in}}%
\pgfpathclose%
\pgfusepath{stroke,fill}%
\end{pgfscope}%
\begin{pgfscope}%
\pgfsetbuttcap%
\pgfsetmiterjoin%
\pgfsetlinewidth{1.003750pt}%
\definecolor{currentstroke}{rgb}{0.000000,0.000000,0.000000}%
\pgfsetstrokecolor{currentstroke}%
\pgfsetdash{}{0pt}%
\pgfpathmoveto{\pgfqpoint{1.588611in}{2.722027in}}%
\pgfpathlineto{\pgfqpoint{1.866389in}{2.722027in}}%
\pgfpathlineto{\pgfqpoint{1.866389in}{2.819250in}}%
\pgfpathlineto{\pgfqpoint{1.588611in}{2.819250in}}%
\pgfpathlineto{\pgfqpoint{1.588611in}{2.722027in}}%
\pgfpathclose%
\pgfusepath{stroke}%
\end{pgfscope}%
\begin{pgfscope}%
\definecolor{textcolor}{rgb}{0.000000,0.000000,0.000000}%
\pgfsetstrokecolor{textcolor}%
\pgfsetfillcolor{textcolor}%
\pgftext[x=1.977500in,y=2.722027in,left,base]{\color{textcolor}\rmfamily\fontsize{10.000000}{12.000000}\selectfont Negative Class}%
\end{pgfscope}%
\begin{pgfscope}%
\pgfsetbuttcap%
\pgfsetmiterjoin%
\definecolor{currentfill}{rgb}{0.000000,0.000000,0.000000}%
\pgfsetfillcolor{currentfill}%
\pgfsetlinewidth{0.000000pt}%
\definecolor{currentstroke}{rgb}{0.000000,0.000000,0.000000}%
\pgfsetstrokecolor{currentstroke}%
\pgfsetstrokeopacity{0.000000}%
\pgfsetdash{}{0pt}%
\pgfpathmoveto{\pgfqpoint{1.588611in}{2.526750in}}%
\pgfpathlineto{\pgfqpoint{1.866389in}{2.526750in}}%
\pgfpathlineto{\pgfqpoint{1.866389in}{2.623972in}}%
\pgfpathlineto{\pgfqpoint{1.588611in}{2.623972in}}%
\pgfpathlineto{\pgfqpoint{1.588611in}{2.526750in}}%
\pgfpathclose%
\pgfusepath{fill}%
\end{pgfscope}%
\begin{pgfscope}%
\definecolor{textcolor}{rgb}{0.000000,0.000000,0.000000}%
\pgfsetstrokecolor{textcolor}%
\pgfsetfillcolor{textcolor}%
\pgftext[x=1.977500in,y=2.526750in,left,base]{\color{textcolor}\rmfamily\fontsize{10.000000}{12.000000}\selectfont Positive Class}%
\end{pgfscope}%
\end{pgfpicture}%
\makeatother%
\endgroup%

  &
  \vspace{0pt} %% Creator: Matplotlib, PGF backend
%%
%% To include the figure in your LaTeX document, write
%%   \input{<filename>.pgf}
%%
%% Make sure the required packages are loaded in your preamble
%%   \usepackage{pgf}
%%
%% Also ensure that all the required font packages are loaded; for instance,
%% the lmodern package is sometimes necessary when using math font.
%%   \usepackage{lmodern}
%%
%% Figures using additional raster images can only be included by \input if
%% they are in the same directory as the main LaTeX file. For loading figures
%% from other directories you can use the `import` package
%%   \usepackage{import}
%%
%% and then include the figures with
%%   \import{<path to file>}{<filename>.pgf}
%%
%% Matplotlib used the following preamble
%%   
%%   \usepackage{fontspec}
%%   \makeatletter\@ifpackageloaded{underscore}{}{\usepackage[strings]{underscore}}\makeatother
%%
\begingroup%
\makeatletter%
\begin{pgfpicture}%
\pgfpathrectangle{\pgfpointorigin}{\pgfqpoint{3.144311in}{2.646444in}}%
\pgfusepath{use as bounding box, clip}%
\begin{pgfscope}%
\pgfsetbuttcap%
\pgfsetmiterjoin%
\definecolor{currentfill}{rgb}{1.000000,1.000000,1.000000}%
\pgfsetfillcolor{currentfill}%
\pgfsetlinewidth{0.000000pt}%
\definecolor{currentstroke}{rgb}{1.000000,1.000000,1.000000}%
\pgfsetstrokecolor{currentstroke}%
\pgfsetdash{}{0pt}%
\pgfpathmoveto{\pgfqpoint{0.000000in}{0.000000in}}%
\pgfpathlineto{\pgfqpoint{3.144311in}{0.000000in}}%
\pgfpathlineto{\pgfqpoint{3.144311in}{2.646444in}}%
\pgfpathlineto{\pgfqpoint{0.000000in}{2.646444in}}%
\pgfpathlineto{\pgfqpoint{0.000000in}{0.000000in}}%
\pgfpathclose%
\pgfusepath{fill}%
\end{pgfscope}%
\begin{pgfscope}%
\pgfsetbuttcap%
\pgfsetmiterjoin%
\definecolor{currentfill}{rgb}{1.000000,1.000000,1.000000}%
\pgfsetfillcolor{currentfill}%
\pgfsetlinewidth{0.000000pt}%
\definecolor{currentstroke}{rgb}{0.000000,0.000000,0.000000}%
\pgfsetstrokecolor{currentstroke}%
\pgfsetstrokeopacity{0.000000}%
\pgfsetdash{}{0pt}%
\pgfpathmoveto{\pgfqpoint{0.553581in}{0.499444in}}%
\pgfpathlineto{\pgfqpoint{3.033581in}{0.499444in}}%
\pgfpathlineto{\pgfqpoint{3.033581in}{2.347444in}}%
\pgfpathlineto{\pgfqpoint{0.553581in}{2.347444in}}%
\pgfpathlineto{\pgfqpoint{0.553581in}{0.499444in}}%
\pgfpathclose%
\pgfusepath{fill}%
\end{pgfscope}%
\begin{pgfscope}%
\pgfsetbuttcap%
\pgfsetroundjoin%
\definecolor{currentfill}{rgb}{0.000000,0.000000,0.000000}%
\pgfsetfillcolor{currentfill}%
\pgfsetlinewidth{0.803000pt}%
\definecolor{currentstroke}{rgb}{0.000000,0.000000,0.000000}%
\pgfsetstrokecolor{currentstroke}%
\pgfsetdash{}{0pt}%
\pgfsys@defobject{currentmarker}{\pgfqpoint{0.000000in}{-0.048611in}}{\pgfqpoint{0.000000in}{0.000000in}}{%
\pgfpathmoveto{\pgfqpoint{0.000000in}{0.000000in}}%
\pgfpathlineto{\pgfqpoint{0.000000in}{-0.048611in}}%
\pgfusepath{stroke,fill}%
}%
\begin{pgfscope}%
\pgfsys@transformshift{0.666308in}{0.499444in}%
\pgfsys@useobject{currentmarker}{}%
\end{pgfscope}%
\end{pgfscope}%
\begin{pgfscope}%
\definecolor{textcolor}{rgb}{0.000000,0.000000,0.000000}%
\pgfsetstrokecolor{textcolor}%
\pgfsetfillcolor{textcolor}%
\pgftext[x=0.666308in,y=0.402222in,,top]{\color{textcolor}\rmfamily\fontsize{10.000000}{12.000000}\selectfont \(\displaystyle {0.00}\)}%
\end{pgfscope}%
\begin{pgfscope}%
\pgfsetbuttcap%
\pgfsetroundjoin%
\definecolor{currentfill}{rgb}{0.000000,0.000000,0.000000}%
\pgfsetfillcolor{currentfill}%
\pgfsetlinewidth{0.803000pt}%
\definecolor{currentstroke}{rgb}{0.000000,0.000000,0.000000}%
\pgfsetstrokecolor{currentstroke}%
\pgfsetdash{}{0pt}%
\pgfsys@defobject{currentmarker}{\pgfqpoint{0.000000in}{-0.048611in}}{\pgfqpoint{0.000000in}{0.000000in}}{%
\pgfpathmoveto{\pgfqpoint{0.000000in}{0.000000in}}%
\pgfpathlineto{\pgfqpoint{0.000000in}{-0.048611in}}%
\pgfusepath{stroke,fill}%
}%
\begin{pgfscope}%
\pgfsys@transformshift{1.229944in}{0.499444in}%
\pgfsys@useobject{currentmarker}{}%
\end{pgfscope}%
\end{pgfscope}%
\begin{pgfscope}%
\definecolor{textcolor}{rgb}{0.000000,0.000000,0.000000}%
\pgfsetstrokecolor{textcolor}%
\pgfsetfillcolor{textcolor}%
\pgftext[x=1.229944in,y=0.402222in,,top]{\color{textcolor}\rmfamily\fontsize{10.000000}{12.000000}\selectfont \(\displaystyle {0.25}\)}%
\end{pgfscope}%
\begin{pgfscope}%
\pgfsetbuttcap%
\pgfsetroundjoin%
\definecolor{currentfill}{rgb}{0.000000,0.000000,0.000000}%
\pgfsetfillcolor{currentfill}%
\pgfsetlinewidth{0.803000pt}%
\definecolor{currentstroke}{rgb}{0.000000,0.000000,0.000000}%
\pgfsetstrokecolor{currentstroke}%
\pgfsetdash{}{0pt}%
\pgfsys@defobject{currentmarker}{\pgfqpoint{0.000000in}{-0.048611in}}{\pgfqpoint{0.000000in}{0.000000in}}{%
\pgfpathmoveto{\pgfqpoint{0.000000in}{0.000000in}}%
\pgfpathlineto{\pgfqpoint{0.000000in}{-0.048611in}}%
\pgfusepath{stroke,fill}%
}%
\begin{pgfscope}%
\pgfsys@transformshift{1.793581in}{0.499444in}%
\pgfsys@useobject{currentmarker}{}%
\end{pgfscope}%
\end{pgfscope}%
\begin{pgfscope}%
\definecolor{textcolor}{rgb}{0.000000,0.000000,0.000000}%
\pgfsetstrokecolor{textcolor}%
\pgfsetfillcolor{textcolor}%
\pgftext[x=1.793581in,y=0.402222in,,top]{\color{textcolor}\rmfamily\fontsize{10.000000}{12.000000}\selectfont \(\displaystyle {0.50}\)}%
\end{pgfscope}%
\begin{pgfscope}%
\pgfsetbuttcap%
\pgfsetroundjoin%
\definecolor{currentfill}{rgb}{0.000000,0.000000,0.000000}%
\pgfsetfillcolor{currentfill}%
\pgfsetlinewidth{0.803000pt}%
\definecolor{currentstroke}{rgb}{0.000000,0.000000,0.000000}%
\pgfsetstrokecolor{currentstroke}%
\pgfsetdash{}{0pt}%
\pgfsys@defobject{currentmarker}{\pgfqpoint{0.000000in}{-0.048611in}}{\pgfqpoint{0.000000in}{0.000000in}}{%
\pgfpathmoveto{\pgfqpoint{0.000000in}{0.000000in}}%
\pgfpathlineto{\pgfqpoint{0.000000in}{-0.048611in}}%
\pgfusepath{stroke,fill}%
}%
\begin{pgfscope}%
\pgfsys@transformshift{2.357217in}{0.499444in}%
\pgfsys@useobject{currentmarker}{}%
\end{pgfscope}%
\end{pgfscope}%
\begin{pgfscope}%
\definecolor{textcolor}{rgb}{0.000000,0.000000,0.000000}%
\pgfsetstrokecolor{textcolor}%
\pgfsetfillcolor{textcolor}%
\pgftext[x=2.357217in,y=0.402222in,,top]{\color{textcolor}\rmfamily\fontsize{10.000000}{12.000000}\selectfont \(\displaystyle {0.75}\)}%
\end{pgfscope}%
\begin{pgfscope}%
\pgfsetbuttcap%
\pgfsetroundjoin%
\definecolor{currentfill}{rgb}{0.000000,0.000000,0.000000}%
\pgfsetfillcolor{currentfill}%
\pgfsetlinewidth{0.803000pt}%
\definecolor{currentstroke}{rgb}{0.000000,0.000000,0.000000}%
\pgfsetstrokecolor{currentstroke}%
\pgfsetdash{}{0pt}%
\pgfsys@defobject{currentmarker}{\pgfqpoint{0.000000in}{-0.048611in}}{\pgfqpoint{0.000000in}{0.000000in}}{%
\pgfpathmoveto{\pgfqpoint{0.000000in}{0.000000in}}%
\pgfpathlineto{\pgfqpoint{0.000000in}{-0.048611in}}%
\pgfusepath{stroke,fill}%
}%
\begin{pgfscope}%
\pgfsys@transformshift{2.920853in}{0.499444in}%
\pgfsys@useobject{currentmarker}{}%
\end{pgfscope}%
\end{pgfscope}%
\begin{pgfscope}%
\definecolor{textcolor}{rgb}{0.000000,0.000000,0.000000}%
\pgfsetstrokecolor{textcolor}%
\pgfsetfillcolor{textcolor}%
\pgftext[x=2.920853in,y=0.402222in,,top]{\color{textcolor}\rmfamily\fontsize{10.000000}{12.000000}\selectfont \(\displaystyle {1.00}\)}%
\end{pgfscope}%
\begin{pgfscope}%
\definecolor{textcolor}{rgb}{0.000000,0.000000,0.000000}%
\pgfsetstrokecolor{textcolor}%
\pgfsetfillcolor{textcolor}%
\pgftext[x=1.793581in,y=0.223333in,,top]{\color{textcolor}\rmfamily\fontsize{10.000000}{12.000000}\selectfont False positive rate}%
\end{pgfscope}%
\begin{pgfscope}%
\pgfsetbuttcap%
\pgfsetroundjoin%
\definecolor{currentfill}{rgb}{0.000000,0.000000,0.000000}%
\pgfsetfillcolor{currentfill}%
\pgfsetlinewidth{0.803000pt}%
\definecolor{currentstroke}{rgb}{0.000000,0.000000,0.000000}%
\pgfsetstrokecolor{currentstroke}%
\pgfsetdash{}{0pt}%
\pgfsys@defobject{currentmarker}{\pgfqpoint{-0.048611in}{0.000000in}}{\pgfqpoint{-0.000000in}{0.000000in}}{%
\pgfpathmoveto{\pgfqpoint{-0.000000in}{0.000000in}}%
\pgfpathlineto{\pgfqpoint{-0.048611in}{0.000000in}}%
\pgfusepath{stroke,fill}%
}%
\begin{pgfscope}%
\pgfsys@transformshift{0.553581in}{0.583444in}%
\pgfsys@useobject{currentmarker}{}%
\end{pgfscope}%
\end{pgfscope}%
\begin{pgfscope}%
\definecolor{textcolor}{rgb}{0.000000,0.000000,0.000000}%
\pgfsetstrokecolor{textcolor}%
\pgfsetfillcolor{textcolor}%
\pgftext[x=0.278889in, y=0.535250in, left, base]{\color{textcolor}\rmfamily\fontsize{10.000000}{12.000000}\selectfont \(\displaystyle {0.0}\)}%
\end{pgfscope}%
\begin{pgfscope}%
\pgfsetbuttcap%
\pgfsetroundjoin%
\definecolor{currentfill}{rgb}{0.000000,0.000000,0.000000}%
\pgfsetfillcolor{currentfill}%
\pgfsetlinewidth{0.803000pt}%
\definecolor{currentstroke}{rgb}{0.000000,0.000000,0.000000}%
\pgfsetstrokecolor{currentstroke}%
\pgfsetdash{}{0pt}%
\pgfsys@defobject{currentmarker}{\pgfqpoint{-0.048611in}{0.000000in}}{\pgfqpoint{-0.000000in}{0.000000in}}{%
\pgfpathmoveto{\pgfqpoint{-0.000000in}{0.000000in}}%
\pgfpathlineto{\pgfqpoint{-0.048611in}{0.000000in}}%
\pgfusepath{stroke,fill}%
}%
\begin{pgfscope}%
\pgfsys@transformshift{0.553581in}{0.919444in}%
\pgfsys@useobject{currentmarker}{}%
\end{pgfscope}%
\end{pgfscope}%
\begin{pgfscope}%
\definecolor{textcolor}{rgb}{0.000000,0.000000,0.000000}%
\pgfsetstrokecolor{textcolor}%
\pgfsetfillcolor{textcolor}%
\pgftext[x=0.278889in, y=0.871250in, left, base]{\color{textcolor}\rmfamily\fontsize{10.000000}{12.000000}\selectfont \(\displaystyle {0.2}\)}%
\end{pgfscope}%
\begin{pgfscope}%
\pgfsetbuttcap%
\pgfsetroundjoin%
\definecolor{currentfill}{rgb}{0.000000,0.000000,0.000000}%
\pgfsetfillcolor{currentfill}%
\pgfsetlinewidth{0.803000pt}%
\definecolor{currentstroke}{rgb}{0.000000,0.000000,0.000000}%
\pgfsetstrokecolor{currentstroke}%
\pgfsetdash{}{0pt}%
\pgfsys@defobject{currentmarker}{\pgfqpoint{-0.048611in}{0.000000in}}{\pgfqpoint{-0.000000in}{0.000000in}}{%
\pgfpathmoveto{\pgfqpoint{-0.000000in}{0.000000in}}%
\pgfpathlineto{\pgfqpoint{-0.048611in}{0.000000in}}%
\pgfusepath{stroke,fill}%
}%
\begin{pgfscope}%
\pgfsys@transformshift{0.553581in}{1.255444in}%
\pgfsys@useobject{currentmarker}{}%
\end{pgfscope}%
\end{pgfscope}%
\begin{pgfscope}%
\definecolor{textcolor}{rgb}{0.000000,0.000000,0.000000}%
\pgfsetstrokecolor{textcolor}%
\pgfsetfillcolor{textcolor}%
\pgftext[x=0.278889in, y=1.207250in, left, base]{\color{textcolor}\rmfamily\fontsize{10.000000}{12.000000}\selectfont \(\displaystyle {0.4}\)}%
\end{pgfscope}%
\begin{pgfscope}%
\pgfsetbuttcap%
\pgfsetroundjoin%
\definecolor{currentfill}{rgb}{0.000000,0.000000,0.000000}%
\pgfsetfillcolor{currentfill}%
\pgfsetlinewidth{0.803000pt}%
\definecolor{currentstroke}{rgb}{0.000000,0.000000,0.000000}%
\pgfsetstrokecolor{currentstroke}%
\pgfsetdash{}{0pt}%
\pgfsys@defobject{currentmarker}{\pgfqpoint{-0.048611in}{0.000000in}}{\pgfqpoint{-0.000000in}{0.000000in}}{%
\pgfpathmoveto{\pgfqpoint{-0.000000in}{0.000000in}}%
\pgfpathlineto{\pgfqpoint{-0.048611in}{0.000000in}}%
\pgfusepath{stroke,fill}%
}%
\begin{pgfscope}%
\pgfsys@transformshift{0.553581in}{1.591444in}%
\pgfsys@useobject{currentmarker}{}%
\end{pgfscope}%
\end{pgfscope}%
\begin{pgfscope}%
\definecolor{textcolor}{rgb}{0.000000,0.000000,0.000000}%
\pgfsetstrokecolor{textcolor}%
\pgfsetfillcolor{textcolor}%
\pgftext[x=0.278889in, y=1.543250in, left, base]{\color{textcolor}\rmfamily\fontsize{10.000000}{12.000000}\selectfont \(\displaystyle {0.6}\)}%
\end{pgfscope}%
\begin{pgfscope}%
\pgfsetbuttcap%
\pgfsetroundjoin%
\definecolor{currentfill}{rgb}{0.000000,0.000000,0.000000}%
\pgfsetfillcolor{currentfill}%
\pgfsetlinewidth{0.803000pt}%
\definecolor{currentstroke}{rgb}{0.000000,0.000000,0.000000}%
\pgfsetstrokecolor{currentstroke}%
\pgfsetdash{}{0pt}%
\pgfsys@defobject{currentmarker}{\pgfqpoint{-0.048611in}{0.000000in}}{\pgfqpoint{-0.000000in}{0.000000in}}{%
\pgfpathmoveto{\pgfqpoint{-0.000000in}{0.000000in}}%
\pgfpathlineto{\pgfqpoint{-0.048611in}{0.000000in}}%
\pgfusepath{stroke,fill}%
}%
\begin{pgfscope}%
\pgfsys@transformshift{0.553581in}{1.927444in}%
\pgfsys@useobject{currentmarker}{}%
\end{pgfscope}%
\end{pgfscope}%
\begin{pgfscope}%
\definecolor{textcolor}{rgb}{0.000000,0.000000,0.000000}%
\pgfsetstrokecolor{textcolor}%
\pgfsetfillcolor{textcolor}%
\pgftext[x=0.278889in, y=1.879250in, left, base]{\color{textcolor}\rmfamily\fontsize{10.000000}{12.000000}\selectfont \(\displaystyle {0.8}\)}%
\end{pgfscope}%
\begin{pgfscope}%
\pgfsetbuttcap%
\pgfsetroundjoin%
\definecolor{currentfill}{rgb}{0.000000,0.000000,0.000000}%
\pgfsetfillcolor{currentfill}%
\pgfsetlinewidth{0.803000pt}%
\definecolor{currentstroke}{rgb}{0.000000,0.000000,0.000000}%
\pgfsetstrokecolor{currentstroke}%
\pgfsetdash{}{0pt}%
\pgfsys@defobject{currentmarker}{\pgfqpoint{-0.048611in}{0.000000in}}{\pgfqpoint{-0.000000in}{0.000000in}}{%
\pgfpathmoveto{\pgfqpoint{-0.000000in}{0.000000in}}%
\pgfpathlineto{\pgfqpoint{-0.048611in}{0.000000in}}%
\pgfusepath{stroke,fill}%
}%
\begin{pgfscope}%
\pgfsys@transformshift{0.553581in}{2.263444in}%
\pgfsys@useobject{currentmarker}{}%
\end{pgfscope}%
\end{pgfscope}%
\begin{pgfscope}%
\definecolor{textcolor}{rgb}{0.000000,0.000000,0.000000}%
\pgfsetstrokecolor{textcolor}%
\pgfsetfillcolor{textcolor}%
\pgftext[x=0.278889in, y=2.215250in, left, base]{\color{textcolor}\rmfamily\fontsize{10.000000}{12.000000}\selectfont \(\displaystyle {1.0}\)}%
\end{pgfscope}%
\begin{pgfscope}%
\definecolor{textcolor}{rgb}{0.000000,0.000000,0.000000}%
\pgfsetstrokecolor{textcolor}%
\pgfsetfillcolor{textcolor}%
\pgftext[x=0.223333in,y=1.423444in,,bottom,rotate=90.000000]{\color{textcolor}\rmfamily\fontsize{10.000000}{12.000000}\selectfont True positive rate}%
\end{pgfscope}%
\begin{pgfscope}%
\pgfpathrectangle{\pgfqpoint{0.553581in}{0.499444in}}{\pgfqpoint{2.480000in}{1.848000in}}%
\pgfusepath{clip}%
\pgfsetbuttcap%
\pgfsetroundjoin%
\pgfsetlinewidth{1.505625pt}%
\definecolor{currentstroke}{rgb}{0.000000,0.000000,0.000000}%
\pgfsetstrokecolor{currentstroke}%
\pgfsetdash{{5.550000pt}{2.400000pt}}{0.000000pt}%
\pgfpathmoveto{\pgfqpoint{0.666308in}{0.583444in}}%
\pgfpathlineto{\pgfqpoint{2.920853in}{2.263444in}}%
\pgfusepath{stroke}%
\end{pgfscope}%
\begin{pgfscope}%
\pgfpathrectangle{\pgfqpoint{0.553581in}{0.499444in}}{\pgfqpoint{2.480000in}{1.848000in}}%
\pgfusepath{clip}%
\pgfsetrectcap%
\pgfsetroundjoin%
\pgfsetlinewidth{1.505625pt}%
\definecolor{currentstroke}{rgb}{0.121569,0.466667,0.705882}%
\pgfsetstrokecolor{currentstroke}%
\pgfsetdash{}{0pt}%
\pgfpathmoveto{\pgfqpoint{0.666308in}{0.583444in}}%
\pgfpathlineto{\pgfqpoint{0.680737in}{0.626452in}}%
\pgfpathlineto{\pgfqpoint{0.681846in}{0.632500in}}%
\pgfpathlineto{\pgfqpoint{0.682109in}{0.633508in}}%
\pgfpathlineto{\pgfqpoint{0.683198in}{0.639472in}}%
\pgfpathlineto{\pgfqpoint{0.683424in}{0.640564in}}%
\pgfpathlineto{\pgfqpoint{0.684513in}{0.645772in}}%
\pgfpathlineto{\pgfqpoint{0.684701in}{0.646696in}}%
\pgfpathlineto{\pgfqpoint{0.685791in}{0.654340in}}%
\pgfpathlineto{\pgfqpoint{0.686016in}{0.655432in}}%
\pgfpathlineto{\pgfqpoint{0.687106in}{0.662152in}}%
\pgfpathlineto{\pgfqpoint{0.687388in}{0.663244in}}%
\pgfpathlineto{\pgfqpoint{0.688496in}{0.669796in}}%
\pgfpathlineto{\pgfqpoint{0.688703in}{0.670720in}}%
\pgfpathlineto{\pgfqpoint{0.689774in}{0.678196in}}%
\pgfpathlineto{\pgfqpoint{0.690075in}{0.679204in}}%
\pgfpathlineto{\pgfqpoint{0.691183in}{0.686512in}}%
\pgfpathlineto{\pgfqpoint{0.691352in}{0.687520in}}%
\pgfpathlineto{\pgfqpoint{0.692442in}{0.694912in}}%
\pgfpathlineto{\pgfqpoint{0.692611in}{0.696004in}}%
\pgfpathlineto{\pgfqpoint{0.693719in}{0.703900in}}%
\pgfpathlineto{\pgfqpoint{0.693907in}{0.704908in}}%
\pgfpathlineto{\pgfqpoint{0.695016in}{0.712972in}}%
\pgfpathlineto{\pgfqpoint{0.695260in}{0.714064in}}%
\pgfpathlineto{\pgfqpoint{0.696369in}{0.723136in}}%
\pgfpathlineto{\pgfqpoint{0.696538in}{0.724144in}}%
\pgfpathlineto{\pgfqpoint{0.697646in}{0.734056in}}%
\pgfpathlineto{\pgfqpoint{0.697890in}{0.735148in}}%
\pgfpathlineto{\pgfqpoint{0.698999in}{0.744976in}}%
\pgfpathlineto{\pgfqpoint{0.699112in}{0.746068in}}%
\pgfpathlineto{\pgfqpoint{0.700201in}{0.753208in}}%
\pgfpathlineto{\pgfqpoint{0.700389in}{0.754300in}}%
\pgfpathlineto{\pgfqpoint{0.701460in}{0.762448in}}%
\pgfpathlineto{\pgfqpoint{0.701855in}{0.763204in}}%
\pgfpathlineto{\pgfqpoint{0.702963in}{0.772864in}}%
\pgfpathlineto{\pgfqpoint{0.703189in}{0.773788in}}%
\pgfpathlineto{\pgfqpoint{0.704241in}{0.780844in}}%
\pgfpathlineto{\pgfqpoint{0.704504in}{0.781936in}}%
\pgfpathlineto{\pgfqpoint{0.705612in}{0.790672in}}%
\pgfpathlineto{\pgfqpoint{0.705781in}{0.791680in}}%
\pgfpathlineto{\pgfqpoint{0.706890in}{0.800836in}}%
\pgfpathlineto{\pgfqpoint{0.707040in}{0.801760in}}%
\pgfpathlineto{\pgfqpoint{0.708149in}{0.810496in}}%
\pgfpathlineto{\pgfqpoint{0.708243in}{0.811588in}}%
\pgfpathlineto{\pgfqpoint{0.709351in}{0.820156in}}%
\pgfpathlineto{\pgfqpoint{0.709520in}{0.821248in}}%
\pgfpathlineto{\pgfqpoint{0.710629in}{0.829564in}}%
\pgfpathlineto{\pgfqpoint{0.710892in}{0.830656in}}%
\pgfpathlineto{\pgfqpoint{0.712000in}{0.840316in}}%
\pgfpathlineto{\pgfqpoint{0.712113in}{0.841408in}}%
\pgfpathlineto{\pgfqpoint{0.713203in}{0.847540in}}%
\pgfpathlineto{\pgfqpoint{0.713296in}{0.848632in}}%
\pgfpathlineto{\pgfqpoint{0.714292in}{0.856444in}}%
\pgfpathlineto{\pgfqpoint{0.714593in}{0.857200in}}%
\pgfpathlineto{\pgfqpoint{0.715701in}{0.866944in}}%
\pgfpathlineto{\pgfqpoint{0.715852in}{0.867868in}}%
\pgfpathlineto{\pgfqpoint{0.716960in}{0.876520in}}%
\pgfpathlineto{\pgfqpoint{0.717092in}{0.877528in}}%
\pgfpathlineto{\pgfqpoint{0.718200in}{0.886936in}}%
\pgfpathlineto{\pgfqpoint{0.718407in}{0.887860in}}%
\pgfpathlineto{\pgfqpoint{0.719515in}{0.898108in}}%
\pgfpathlineto{\pgfqpoint{0.719609in}{0.899032in}}%
\pgfpathlineto{\pgfqpoint{0.720718in}{0.908440in}}%
\pgfpathlineto{\pgfqpoint{0.720812in}{0.909448in}}%
\pgfpathlineto{\pgfqpoint{0.721901in}{0.918772in}}%
\pgfpathlineto{\pgfqpoint{0.722108in}{0.919864in}}%
\pgfpathlineto{\pgfqpoint{0.723198in}{0.927676in}}%
\pgfpathlineto{\pgfqpoint{0.723273in}{0.928600in}}%
\pgfpathlineto{\pgfqpoint{0.724381in}{0.935320in}}%
\pgfpathlineto{\pgfqpoint{0.724588in}{0.936412in}}%
\pgfpathlineto{\pgfqpoint{0.725696in}{0.944560in}}%
\pgfpathlineto{\pgfqpoint{0.725903in}{0.945652in}}%
\pgfpathlineto{\pgfqpoint{0.726974in}{0.952624in}}%
\pgfpathlineto{\pgfqpoint{0.727218in}{0.953296in}}%
\pgfpathlineto{\pgfqpoint{0.728308in}{0.963964in}}%
\pgfpathlineto{\pgfqpoint{0.728515in}{0.964972in}}%
\pgfpathlineto{\pgfqpoint{0.729623in}{0.972616in}}%
\pgfpathlineto{\pgfqpoint{0.729811in}{0.973624in}}%
\pgfpathlineto{\pgfqpoint{0.730919in}{0.980008in}}%
\pgfpathlineto{\pgfqpoint{0.731220in}{0.980932in}}%
\pgfpathlineto{\pgfqpoint{0.732291in}{0.986980in}}%
\pgfpathlineto{\pgfqpoint{0.732498in}{0.987736in}}%
\pgfpathlineto{\pgfqpoint{0.733606in}{0.995044in}}%
\pgfpathlineto{\pgfqpoint{0.733775in}{0.995968in}}%
\pgfpathlineto{\pgfqpoint{0.734865in}{1.005040in}}%
\pgfpathlineto{\pgfqpoint{0.735053in}{1.005964in}}%
\pgfpathlineto{\pgfqpoint{0.736124in}{1.012516in}}%
\pgfpathlineto{\pgfqpoint{0.736424in}{1.013608in}}%
\pgfpathlineto{\pgfqpoint{0.737514in}{1.022176in}}%
\pgfpathlineto{\pgfqpoint{0.737664in}{1.023268in}}%
\pgfpathlineto{\pgfqpoint{0.738754in}{1.029736in}}%
\pgfpathlineto{\pgfqpoint{0.738998in}{1.030660in}}%
\pgfpathlineto{\pgfqpoint{0.740107in}{1.039480in}}%
\pgfpathlineto{\pgfqpoint{0.740389in}{1.040572in}}%
\pgfpathlineto{\pgfqpoint{0.741441in}{1.046704in}}%
\pgfpathlineto{\pgfqpoint{0.741704in}{1.047796in}}%
\pgfpathlineto{\pgfqpoint{0.742793in}{1.055440in}}%
\pgfpathlineto{\pgfqpoint{0.743188in}{1.056532in}}%
\pgfpathlineto{\pgfqpoint{0.744296in}{1.064680in}}%
\pgfpathlineto{\pgfqpoint{0.744409in}{1.065772in}}%
\pgfpathlineto{\pgfqpoint{0.745518in}{1.073416in}}%
\pgfpathlineto{\pgfqpoint{0.745893in}{1.074340in}}%
\pgfpathlineto{\pgfqpoint{0.747002in}{1.081732in}}%
\pgfpathlineto{\pgfqpoint{0.747115in}{1.082572in}}%
\pgfpathlineto{\pgfqpoint{0.748129in}{1.089292in}}%
\pgfpathlineto{\pgfqpoint{0.748768in}{1.090384in}}%
\pgfpathlineto{\pgfqpoint{0.749876in}{1.098112in}}%
\pgfpathlineto{\pgfqpoint{0.750008in}{1.099120in}}%
\pgfpathlineto{\pgfqpoint{0.751116in}{1.104328in}}%
\pgfpathlineto{\pgfqpoint{0.751267in}{1.105084in}}%
\pgfpathlineto{\pgfqpoint{0.752375in}{1.110460in}}%
\pgfpathlineto{\pgfqpoint{0.752676in}{1.111468in}}%
\pgfpathlineto{\pgfqpoint{0.753766in}{1.116844in}}%
\pgfpathlineto{\pgfqpoint{0.753935in}{1.117852in}}%
\pgfpathlineto{\pgfqpoint{0.755043in}{1.123984in}}%
\pgfpathlineto{\pgfqpoint{0.755456in}{1.125076in}}%
\pgfpathlineto{\pgfqpoint{0.756565in}{1.132132in}}%
\pgfpathlineto{\pgfqpoint{0.756922in}{1.133224in}}%
\pgfpathlineto{\pgfqpoint{0.758012in}{1.139440in}}%
\pgfpathlineto{\pgfqpoint{0.758406in}{1.140532in}}%
\pgfpathlineto{\pgfqpoint{0.759515in}{1.147924in}}%
\pgfpathlineto{\pgfqpoint{0.759609in}{1.148512in}}%
\pgfpathlineto{\pgfqpoint{0.760717in}{1.155652in}}%
\pgfpathlineto{\pgfqpoint{0.760867in}{1.156744in}}%
\pgfpathlineto{\pgfqpoint{0.761976in}{1.162456in}}%
\pgfpathlineto{\pgfqpoint{0.762201in}{1.163380in}}%
\pgfpathlineto{\pgfqpoint{0.763310in}{1.168504in}}%
\pgfpathlineto{\pgfqpoint{0.763498in}{1.169512in}}%
\pgfpathlineto{\pgfqpoint{0.764606in}{1.177660in}}%
\pgfpathlineto{\pgfqpoint{0.764926in}{1.178752in}}%
\pgfpathlineto{\pgfqpoint{0.766015in}{1.184548in}}%
\pgfpathlineto{\pgfqpoint{0.766241in}{1.185556in}}%
\pgfpathlineto{\pgfqpoint{0.767349in}{1.190596in}}%
\pgfpathlineto{\pgfqpoint{0.767763in}{1.191688in}}%
\pgfpathlineto{\pgfqpoint{0.768833in}{1.197148in}}%
\pgfpathlineto{\pgfqpoint{0.769115in}{1.198156in}}%
\pgfpathlineto{\pgfqpoint{0.770186in}{1.202692in}}%
\pgfpathlineto{\pgfqpoint{0.770562in}{1.203784in}}%
\pgfpathlineto{\pgfqpoint{0.771670in}{1.208656in}}%
\pgfpathlineto{\pgfqpoint{0.771802in}{1.209748in}}%
\pgfpathlineto{\pgfqpoint{0.772892in}{1.216216in}}%
\pgfpathlineto{\pgfqpoint{0.773136in}{1.217308in}}%
\pgfpathlineto{\pgfqpoint{0.774244in}{1.222684in}}%
\pgfpathlineto{\pgfqpoint{0.774376in}{1.223692in}}%
\pgfpathlineto{\pgfqpoint{0.775484in}{1.228732in}}%
\pgfpathlineto{\pgfqpoint{0.775541in}{1.229236in}}%
\pgfpathlineto{\pgfqpoint{0.776612in}{1.233856in}}%
\pgfpathlineto{\pgfqpoint{0.776950in}{1.234864in}}%
\pgfpathlineto{\pgfqpoint{0.778039in}{1.240492in}}%
\pgfpathlineto{\pgfqpoint{0.778227in}{1.241416in}}%
\pgfpathlineto{\pgfqpoint{0.779336in}{1.246120in}}%
\pgfpathlineto{\pgfqpoint{0.779580in}{1.247212in}}%
\pgfpathlineto{\pgfqpoint{0.780670in}{1.252504in}}%
\pgfpathlineto{\pgfqpoint{0.781008in}{1.253428in}}%
\pgfpathlineto{\pgfqpoint{0.782098in}{1.259392in}}%
\pgfpathlineto{\pgfqpoint{0.782417in}{1.260400in}}%
\pgfpathlineto{\pgfqpoint{0.783526in}{1.263424in}}%
\pgfpathlineto{\pgfqpoint{0.783676in}{1.264180in}}%
\pgfpathlineto{\pgfqpoint{0.784709in}{1.269808in}}%
\pgfpathlineto{\pgfqpoint{0.784991in}{1.270564in}}%
\pgfpathlineto{\pgfqpoint{0.786081in}{1.276276in}}%
\pgfpathlineto{\pgfqpoint{0.786475in}{1.277368in}}%
\pgfpathlineto{\pgfqpoint{0.787565in}{1.280560in}}%
\pgfpathlineto{\pgfqpoint{0.787828in}{1.281484in}}%
\pgfpathlineto{\pgfqpoint{0.788936in}{1.286524in}}%
\pgfpathlineto{\pgfqpoint{0.789312in}{1.287448in}}%
\pgfpathlineto{\pgfqpoint{0.790421in}{1.294168in}}%
\pgfpathlineto{\pgfqpoint{0.790740in}{1.295176in}}%
\pgfpathlineto{\pgfqpoint{0.791849in}{1.300972in}}%
\pgfpathlineto{\pgfqpoint{0.792093in}{1.301812in}}%
\pgfpathlineto{\pgfqpoint{0.793089in}{1.305592in}}%
\pgfpathlineto{\pgfqpoint{0.793784in}{1.306600in}}%
\pgfpathlineto{\pgfqpoint{0.794892in}{1.310548in}}%
\pgfpathlineto{\pgfqpoint{0.795230in}{1.311640in}}%
\pgfpathlineto{\pgfqpoint{0.796339in}{1.315924in}}%
\pgfpathlineto{\pgfqpoint{0.796639in}{1.317016in}}%
\pgfpathlineto{\pgfqpoint{0.797748in}{1.322392in}}%
\pgfpathlineto{\pgfqpoint{0.798049in}{1.323400in}}%
\pgfpathlineto{\pgfqpoint{0.799138in}{1.327432in}}%
\pgfpathlineto{\pgfqpoint{0.799458in}{1.328524in}}%
\pgfpathlineto{\pgfqpoint{0.800529in}{1.333060in}}%
\pgfpathlineto{\pgfqpoint{0.800848in}{1.333900in}}%
\pgfpathlineto{\pgfqpoint{0.801938in}{1.338604in}}%
\pgfpathlineto{\pgfqpoint{0.802313in}{1.339696in}}%
\pgfpathlineto{\pgfqpoint{0.803384in}{1.343728in}}%
\pgfpathlineto{\pgfqpoint{0.803723in}{1.344820in}}%
\pgfpathlineto{\pgfqpoint{0.804812in}{1.347676in}}%
\pgfpathlineto{\pgfqpoint{0.805038in}{1.348600in}}%
\pgfpathlineto{\pgfqpoint{0.806127in}{1.355152in}}%
\pgfpathlineto{\pgfqpoint{0.806447in}{1.356160in}}%
\pgfpathlineto{\pgfqpoint{0.807536in}{1.361368in}}%
\pgfpathlineto{\pgfqpoint{0.807987in}{1.362376in}}%
\pgfpathlineto{\pgfqpoint{0.809077in}{1.366408in}}%
\pgfpathlineto{\pgfqpoint{0.809490in}{1.367500in}}%
\pgfpathlineto{\pgfqpoint{0.810561in}{1.370272in}}%
\pgfpathlineto{\pgfqpoint{0.810881in}{1.371280in}}%
\pgfpathlineto{\pgfqpoint{0.811952in}{1.374724in}}%
\pgfpathlineto{\pgfqpoint{0.812346in}{1.375732in}}%
\pgfpathlineto{\pgfqpoint{0.813455in}{1.378756in}}%
\pgfpathlineto{\pgfqpoint{0.813774in}{1.379764in}}%
\pgfpathlineto{\pgfqpoint{0.814807in}{1.382872in}}%
\pgfpathlineto{\pgfqpoint{0.815183in}{1.383964in}}%
\pgfpathlineto{\pgfqpoint{0.816273in}{1.388248in}}%
\pgfpathlineto{\pgfqpoint{0.816705in}{1.389340in}}%
\pgfpathlineto{\pgfqpoint{0.817795in}{1.393624in}}%
\pgfpathlineto{\pgfqpoint{0.818133in}{1.394632in}}%
\pgfpathlineto{\pgfqpoint{0.819241in}{1.398580in}}%
\pgfpathlineto{\pgfqpoint{0.819692in}{1.399672in}}%
\pgfpathlineto{\pgfqpoint{0.820763in}{1.402192in}}%
\pgfpathlineto{\pgfqpoint{0.821176in}{1.403284in}}%
\pgfpathlineto{\pgfqpoint{0.822210in}{1.407148in}}%
\pgfpathlineto{\pgfqpoint{0.822792in}{1.408240in}}%
\pgfpathlineto{\pgfqpoint{0.823901in}{1.411768in}}%
\pgfpathlineto{\pgfqpoint{0.824276in}{1.412860in}}%
\pgfpathlineto{\pgfqpoint{0.825329in}{1.417900in}}%
\pgfpathlineto{\pgfqpoint{0.825648in}{1.418908in}}%
\pgfpathlineto{\pgfqpoint{0.826663in}{1.421428in}}%
\pgfpathlineto{\pgfqpoint{0.827038in}{1.422352in}}%
\pgfpathlineto{\pgfqpoint{0.828147in}{1.424620in}}%
\pgfpathlineto{\pgfqpoint{0.828391in}{1.425712in}}%
\pgfpathlineto{\pgfqpoint{0.829499in}{1.429744in}}%
\pgfpathlineto{\pgfqpoint{0.830026in}{1.430752in}}%
\pgfpathlineto{\pgfqpoint{0.831021in}{1.434196in}}%
\pgfpathlineto{\pgfqpoint{0.831453in}{1.435288in}}%
\pgfpathlineto{\pgfqpoint{0.832562in}{1.439152in}}%
\pgfpathlineto{\pgfqpoint{0.833032in}{1.440076in}}%
\pgfpathlineto{\pgfqpoint{0.834121in}{1.443520in}}%
\pgfpathlineto{\pgfqpoint{0.834553in}{1.444612in}}%
\pgfpathlineto{\pgfqpoint{0.835662in}{1.448056in}}%
\pgfpathlineto{\pgfqpoint{0.836113in}{1.448980in}}%
\pgfpathlineto{\pgfqpoint{0.837203in}{1.452256in}}%
\pgfpathlineto{\pgfqpoint{0.837484in}{1.453264in}}%
\pgfpathlineto{\pgfqpoint{0.838593in}{1.456036in}}%
\pgfpathlineto{\pgfqpoint{0.838969in}{1.457044in}}%
\pgfpathlineto{\pgfqpoint{0.840077in}{1.461076in}}%
\pgfpathlineto{\pgfqpoint{0.840453in}{1.462168in}}%
\pgfpathlineto{\pgfqpoint{0.841505in}{1.465108in}}%
\pgfpathlineto{\pgfqpoint{0.842163in}{1.466116in}}%
\pgfpathlineto{\pgfqpoint{0.843233in}{1.469560in}}%
\pgfpathlineto{\pgfqpoint{0.843741in}{1.470484in}}%
\pgfpathlineto{\pgfqpoint{0.844849in}{1.474852in}}%
\pgfpathlineto{\pgfqpoint{0.845244in}{1.475692in}}%
\pgfpathlineto{\pgfqpoint{0.846352in}{1.479388in}}%
\pgfpathlineto{\pgfqpoint{0.846916in}{1.480396in}}%
\pgfpathlineto{\pgfqpoint{0.848024in}{1.482580in}}%
\pgfpathlineto{\pgfqpoint{0.848325in}{1.483672in}}%
\pgfpathlineto{\pgfqpoint{0.849433in}{1.488208in}}%
\pgfpathlineto{\pgfqpoint{0.849772in}{1.489300in}}%
\pgfpathlineto{\pgfqpoint{0.850843in}{1.492324in}}%
\pgfpathlineto{\pgfqpoint{0.851387in}{1.493332in}}%
\pgfpathlineto{\pgfqpoint{0.852364in}{1.496020in}}%
\pgfpathlineto{\pgfqpoint{0.853041in}{1.497112in}}%
\pgfpathlineto{\pgfqpoint{0.854055in}{1.499464in}}%
\pgfpathlineto{\pgfqpoint{0.854563in}{1.500556in}}%
\pgfpathlineto{\pgfqpoint{0.855671in}{1.504252in}}%
\pgfpathlineto{\pgfqpoint{0.856122in}{1.505344in}}%
\pgfpathlineto{\pgfqpoint{0.857230in}{1.509124in}}%
\pgfpathlineto{\pgfqpoint{0.857700in}{1.510216in}}%
\pgfpathlineto{\pgfqpoint{0.858771in}{1.511812in}}%
\pgfpathlineto{\pgfqpoint{0.859372in}{1.512904in}}%
\pgfpathlineto{\pgfqpoint{0.859466in}{1.513576in}}%
\pgfpathlineto{\pgfqpoint{0.876093in}{1.514668in}}%
\pgfpathlineto{\pgfqpoint{0.877164in}{1.517692in}}%
\pgfpathlineto{\pgfqpoint{0.877615in}{1.518784in}}%
\pgfpathlineto{\pgfqpoint{0.878724in}{1.522060in}}%
\pgfpathlineto{\pgfqpoint{0.879137in}{1.523068in}}%
\pgfpathlineto{\pgfqpoint{0.880227in}{1.527268in}}%
\pgfpathlineto{\pgfqpoint{0.880565in}{1.528276in}}%
\pgfpathlineto{\pgfqpoint{0.881579in}{1.529872in}}%
\pgfpathlineto{\pgfqpoint{0.882181in}{1.530964in}}%
\pgfpathlineto{\pgfqpoint{0.883289in}{1.533400in}}%
\pgfpathlineto{\pgfqpoint{0.883740in}{1.534408in}}%
\pgfpathlineto{\pgfqpoint{0.884849in}{1.538020in}}%
\pgfpathlineto{\pgfqpoint{0.885318in}{1.539112in}}%
\pgfpathlineto{\pgfqpoint{0.886352in}{1.541884in}}%
\pgfpathlineto{\pgfqpoint{0.886972in}{1.542976in}}%
\pgfpathlineto{\pgfqpoint{0.888024in}{1.545496in}}%
\pgfpathlineto{\pgfqpoint{0.888587in}{1.546588in}}%
\pgfpathlineto{\pgfqpoint{0.889583in}{1.549444in}}%
\pgfpathlineto{\pgfqpoint{0.890072in}{1.550536in}}%
\pgfpathlineto{\pgfqpoint{0.891180in}{1.552552in}}%
\pgfpathlineto{\pgfqpoint{0.891725in}{1.553644in}}%
\pgfpathlineto{\pgfqpoint{0.892815in}{1.555828in}}%
\pgfpathlineto{\pgfqpoint{0.893190in}{1.556920in}}%
\pgfpathlineto{\pgfqpoint{0.894299in}{1.560028in}}%
\pgfpathlineto{\pgfqpoint{0.894524in}{1.561120in}}%
\pgfpathlineto{\pgfqpoint{0.895614in}{1.563640in}}%
\pgfpathlineto{\pgfqpoint{0.896196in}{1.564732in}}%
\pgfpathlineto{\pgfqpoint{0.897230in}{1.567420in}}%
\pgfpathlineto{\pgfqpoint{0.897906in}{1.568512in}}%
\pgfpathlineto{\pgfqpoint{0.899015in}{1.570864in}}%
\pgfpathlineto{\pgfqpoint{0.900029in}{1.571872in}}%
\pgfpathlineto{\pgfqpoint{0.901138in}{1.574308in}}%
\pgfpathlineto{\pgfqpoint{0.901495in}{1.575316in}}%
\pgfpathlineto{\pgfqpoint{0.902584in}{1.578256in}}%
\pgfpathlineto{\pgfqpoint{0.903148in}{1.579264in}}%
\pgfpathlineto{\pgfqpoint{0.904256in}{1.581952in}}%
\pgfpathlineto{\pgfqpoint{0.905064in}{1.583044in}}%
\pgfpathlineto{\pgfqpoint{0.906135in}{1.585144in}}%
\pgfpathlineto{\pgfqpoint{0.906849in}{1.586152in}}%
\pgfpathlineto{\pgfqpoint{0.907883in}{1.589344in}}%
\pgfpathlineto{\pgfqpoint{0.908596in}{1.590436in}}%
\pgfpathlineto{\pgfqpoint{0.909667in}{1.593208in}}%
\pgfpathlineto{\pgfqpoint{0.910456in}{1.594216in}}%
\pgfpathlineto{\pgfqpoint{0.911565in}{1.597324in}}%
\pgfpathlineto{\pgfqpoint{0.912298in}{1.598332in}}%
\pgfpathlineto{\pgfqpoint{0.913312in}{1.600768in}}%
\pgfpathlineto{\pgfqpoint{0.913575in}{1.601776in}}%
\pgfpathlineto{\pgfqpoint{0.914684in}{1.604296in}}%
\pgfpathlineto{\pgfqpoint{0.915153in}{1.605304in}}%
\pgfpathlineto{\pgfqpoint{0.916262in}{1.607068in}}%
\pgfpathlineto{\pgfqpoint{0.916919in}{1.608160in}}%
\pgfpathlineto{\pgfqpoint{0.917972in}{1.610932in}}%
\pgfpathlineto{\pgfqpoint{0.919024in}{1.612024in}}%
\pgfpathlineto{\pgfqpoint{0.920001in}{1.614376in}}%
\pgfpathlineto{\pgfqpoint{0.921353in}{1.615468in}}%
\pgfpathlineto{\pgfqpoint{0.922274in}{1.617820in}}%
\pgfpathlineto{\pgfqpoint{0.923589in}{1.618912in}}%
\pgfpathlineto{\pgfqpoint{0.924585in}{1.620592in}}%
\pgfpathlineto{\pgfqpoint{0.925524in}{1.621684in}}%
\pgfpathlineto{\pgfqpoint{0.926576in}{1.623616in}}%
\pgfpathlineto{\pgfqpoint{0.927478in}{1.624624in}}%
\pgfpathlineto{\pgfqpoint{0.928587in}{1.627312in}}%
\pgfpathlineto{\pgfqpoint{0.929150in}{1.628320in}}%
\pgfpathlineto{\pgfqpoint{0.930240in}{1.630084in}}%
\pgfpathlineto{\pgfqpoint{0.930879in}{1.631176in}}%
\pgfpathlineto{\pgfqpoint{0.931987in}{1.633528in}}%
\pgfpathlineto{\pgfqpoint{0.932476in}{1.634368in}}%
\pgfpathlineto{\pgfqpoint{0.933547in}{1.636804in}}%
\pgfpathlineto{\pgfqpoint{0.934186in}{1.637812in}}%
\pgfpathlineto{\pgfqpoint{0.935256in}{1.639660in}}%
\pgfpathlineto{\pgfqpoint{0.935764in}{1.640668in}}%
\pgfpathlineto{\pgfqpoint{0.936722in}{1.643020in}}%
\pgfpathlineto{\pgfqpoint{0.937699in}{1.644112in}}%
\pgfpathlineto{\pgfqpoint{0.938807in}{1.645876in}}%
\pgfpathlineto{\pgfqpoint{0.939784in}{1.646968in}}%
\pgfpathlineto{\pgfqpoint{0.940818in}{1.649824in}}%
\pgfpathlineto{\pgfqpoint{0.941269in}{1.650916in}}%
\pgfpathlineto{\pgfqpoint{0.942358in}{1.653100in}}%
\pgfpathlineto{\pgfqpoint{0.943410in}{1.654192in}}%
\pgfpathlineto{\pgfqpoint{0.944463in}{1.656292in}}%
\pgfpathlineto{\pgfqpoint{0.945083in}{1.657384in}}%
\pgfpathlineto{\pgfqpoint{0.946135in}{1.659568in}}%
\pgfpathlineto{\pgfqpoint{0.946849in}{1.660660in}}%
\pgfpathlineto{\pgfqpoint{0.947919in}{1.662592in}}%
\pgfpathlineto{\pgfqpoint{0.948539in}{1.663600in}}%
\pgfpathlineto{\pgfqpoint{0.949648in}{1.665700in}}%
\pgfpathlineto{\pgfqpoint{0.950381in}{1.666792in}}%
\pgfpathlineto{\pgfqpoint{0.951470in}{1.669564in}}%
\pgfpathlineto{\pgfqpoint{0.952034in}{1.670656in}}%
\pgfpathlineto{\pgfqpoint{0.953067in}{1.672420in}}%
\pgfpathlineto{\pgfqpoint{0.953950in}{1.673428in}}%
\pgfpathlineto{\pgfqpoint{0.954890in}{1.675108in}}%
\pgfpathlineto{\pgfqpoint{0.955810in}{1.676200in}}%
\pgfpathlineto{\pgfqpoint{0.956750in}{1.677964in}}%
\pgfpathlineto{\pgfqpoint{0.957483in}{1.678888in}}%
\pgfpathlineto{\pgfqpoint{0.958591in}{1.680820in}}%
\pgfpathlineto{\pgfqpoint{0.959079in}{1.681828in}}%
\pgfpathlineto{\pgfqpoint{0.960132in}{1.684432in}}%
\pgfpathlineto{\pgfqpoint{0.961090in}{1.685524in}}%
\pgfpathlineto{\pgfqpoint{0.962086in}{1.687960in}}%
\pgfpathlineto{\pgfqpoint{0.962856in}{1.688800in}}%
\pgfpathlineto{\pgfqpoint{0.963964in}{1.690900in}}%
\pgfpathlineto{\pgfqpoint{0.964716in}{1.691908in}}%
\pgfpathlineto{\pgfqpoint{0.965618in}{1.693672in}}%
\pgfpathlineto{\pgfqpoint{0.966426in}{1.694764in}}%
\pgfpathlineto{\pgfqpoint{0.967515in}{1.696696in}}%
\pgfpathlineto{\pgfqpoint{0.968417in}{1.697788in}}%
\pgfpathlineto{\pgfqpoint{0.969507in}{1.699552in}}%
\pgfpathlineto{\pgfqpoint{0.970164in}{1.700560in}}%
\pgfpathlineto{\pgfqpoint{0.971273in}{1.703248in}}%
\pgfpathlineto{\pgfqpoint{0.971855in}{1.704340in}}%
\pgfpathlineto{\pgfqpoint{0.972964in}{1.707112in}}%
\pgfpathlineto{\pgfqpoint{0.973922in}{1.708120in}}%
\pgfpathlineto{\pgfqpoint{0.974993in}{1.710304in}}%
\pgfpathlineto{\pgfqpoint{0.975538in}{1.711312in}}%
\pgfpathlineto{\pgfqpoint{0.976609in}{1.713076in}}%
\pgfpathlineto{\pgfqpoint{0.977435in}{1.714168in}}%
\pgfpathlineto{\pgfqpoint{0.978506in}{1.716352in}}%
\pgfpathlineto{\pgfqpoint{0.979164in}{1.717444in}}%
\pgfpathlineto{\pgfqpoint{0.980272in}{1.719544in}}%
\pgfpathlineto{\pgfqpoint{0.980686in}{1.720636in}}%
\pgfpathlineto{\pgfqpoint{0.981700in}{1.721980in}}%
\pgfpathlineto{\pgfqpoint{0.982433in}{1.723072in}}%
\pgfpathlineto{\pgfqpoint{0.983541in}{1.724836in}}%
\pgfpathlineto{\pgfqpoint{0.984293in}{1.725928in}}%
\pgfpathlineto{\pgfqpoint{0.985401in}{1.728364in}}%
\pgfpathlineto{\pgfqpoint{0.986228in}{1.729372in}}%
\pgfpathlineto{\pgfqpoint{0.987336in}{1.731304in}}%
\pgfpathlineto{\pgfqpoint{0.988858in}{1.732228in}}%
\pgfpathlineto{\pgfqpoint{0.989854in}{1.734580in}}%
\pgfpathlineto{\pgfqpoint{0.990549in}{1.735672in}}%
\pgfpathlineto{\pgfqpoint{0.991451in}{1.736932in}}%
\pgfpathlineto{\pgfqpoint{0.992616in}{1.738024in}}%
\pgfpathlineto{\pgfqpoint{0.993630in}{1.739368in}}%
\pgfpathlineto{\pgfqpoint{0.994795in}{1.740460in}}%
\pgfpathlineto{\pgfqpoint{0.995885in}{1.742476in}}%
\pgfpathlineto{\pgfqpoint{0.996824in}{1.743568in}}%
\pgfpathlineto{\pgfqpoint{0.997895in}{1.746340in}}%
\pgfpathlineto{\pgfqpoint{0.998628in}{1.747264in}}%
\pgfpathlineto{\pgfqpoint{0.999473in}{1.749112in}}%
\pgfpathlineto{\pgfqpoint{1.000563in}{1.750204in}}%
\pgfpathlineto{\pgfqpoint{1.001484in}{1.751968in}}%
\pgfpathlineto{\pgfqpoint{1.002479in}{1.753060in}}%
\pgfpathlineto{\pgfqpoint{1.003569in}{1.754824in}}%
\pgfpathlineto{\pgfqpoint{1.004621in}{1.755916in}}%
\pgfpathlineto{\pgfqpoint{1.005692in}{1.757008in}}%
\pgfpathlineto{\pgfqpoint{1.006575in}{1.758100in}}%
\pgfpathlineto{\pgfqpoint{1.007477in}{1.759444in}}%
\pgfpathlineto{\pgfqpoint{1.008341in}{1.760452in}}%
\pgfpathlineto{\pgfqpoint{1.009450in}{1.761796in}}%
\pgfpathlineto{\pgfqpoint{1.010220in}{1.762888in}}%
\pgfpathlineto{\pgfqpoint{1.011272in}{1.764652in}}%
\pgfpathlineto{\pgfqpoint{1.013001in}{1.765660in}}%
\pgfpathlineto{\pgfqpoint{1.014109in}{1.767172in}}%
\pgfpathlineto{\pgfqpoint{1.015312in}{1.768264in}}%
\pgfpathlineto{\pgfqpoint{1.016420in}{1.769776in}}%
\pgfpathlineto{\pgfqpoint{1.016965in}{1.770784in}}%
\pgfpathlineto{\pgfqpoint{1.017923in}{1.772212in}}%
\pgfpathlineto{\pgfqpoint{1.018919in}{1.773304in}}%
\pgfpathlineto{\pgfqpoint{1.019746in}{1.774480in}}%
\pgfpathlineto{\pgfqpoint{1.020967in}{1.775572in}}%
\pgfpathlineto{\pgfqpoint{1.022056in}{1.776832in}}%
\pgfpathlineto{\pgfqpoint{1.022827in}{1.777756in}}%
\pgfpathlineto{\pgfqpoint{1.023916in}{1.779604in}}%
\pgfpathlineto{\pgfqpoint{1.025232in}{1.780696in}}%
\pgfpathlineto{\pgfqpoint{1.026321in}{1.782964in}}%
\pgfpathlineto{\pgfqpoint{1.027693in}{1.784056in}}%
\pgfpathlineto{\pgfqpoint{1.028783in}{1.785568in}}%
\pgfpathlineto{\pgfqpoint{1.030192in}{1.786660in}}%
\pgfpathlineto{\pgfqpoint{1.031187in}{1.787836in}}%
\pgfpathlineto{\pgfqpoint{1.032503in}{1.788928in}}%
\pgfpathlineto{\pgfqpoint{1.033536in}{1.790524in}}%
\pgfpathlineto{\pgfqpoint{1.034550in}{1.791532in}}%
\pgfpathlineto{\pgfqpoint{1.035640in}{1.792708in}}%
\pgfpathlineto{\pgfqpoint{1.036392in}{1.793800in}}%
\pgfpathlineto{\pgfqpoint{1.037181in}{1.795648in}}%
\pgfpathlineto{\pgfqpoint{1.038364in}{1.796740in}}%
\pgfpathlineto{\pgfqpoint{1.039473in}{1.798672in}}%
\pgfpathlineto{\pgfqpoint{1.040938in}{1.799680in}}%
\pgfpathlineto{\pgfqpoint{1.042028in}{1.800856in}}%
\pgfpathlineto{\pgfqpoint{1.043099in}{1.801948in}}%
\pgfpathlineto{\pgfqpoint{1.044207in}{1.803208in}}%
\pgfpathlineto{\pgfqpoint{1.045466in}{1.804300in}}%
\pgfpathlineto{\pgfqpoint{1.046575in}{1.804720in}}%
\pgfpathlineto{\pgfqpoint{1.047289in}{1.805812in}}%
\pgfpathlineto{\pgfqpoint{1.048228in}{1.806904in}}%
\pgfpathlineto{\pgfqpoint{1.049299in}{1.807996in}}%
\pgfpathlineto{\pgfqpoint{1.050389in}{1.809088in}}%
\pgfpathlineto{\pgfqpoint{1.051347in}{1.810180in}}%
\pgfpathlineto{\pgfqpoint{1.052436in}{1.811608in}}%
\pgfpathlineto{\pgfqpoint{1.052944in}{1.812700in}}%
\pgfpathlineto{\pgfqpoint{1.053770in}{1.813540in}}%
\pgfpathlineto{\pgfqpoint{1.054729in}{1.814632in}}%
\pgfpathlineto{\pgfqpoint{1.055724in}{1.815976in}}%
\pgfpathlineto{\pgfqpoint{1.056833in}{1.817068in}}%
\pgfpathlineto{\pgfqpoint{1.057923in}{1.818916in}}%
\pgfpathlineto{\pgfqpoint{1.058806in}{1.820008in}}%
\pgfpathlineto{\pgfqpoint{1.059914in}{1.821520in}}%
\pgfpathlineto{\pgfqpoint{1.060872in}{1.822612in}}%
\pgfpathlineto{\pgfqpoint{1.061943in}{1.823788in}}%
\pgfpathlineto{\pgfqpoint{1.063258in}{1.824880in}}%
\pgfpathlineto{\pgfqpoint{1.064198in}{1.825972in}}%
\pgfpathlineto{\pgfqpoint{1.065024in}{1.827064in}}%
\pgfpathlineto{\pgfqpoint{1.066001in}{1.828408in}}%
\pgfpathlineto{\pgfqpoint{1.067335in}{1.829500in}}%
\pgfpathlineto{\pgfqpoint{1.068406in}{1.830760in}}%
\pgfpathlineto{\pgfqpoint{1.069533in}{1.831852in}}%
\pgfpathlineto{\pgfqpoint{1.070360in}{1.833028in}}%
\pgfpathlineto{\pgfqpoint{1.071262in}{1.834120in}}%
\pgfpathlineto{\pgfqpoint{1.072183in}{1.834960in}}%
\pgfpathlineto{\pgfqpoint{1.073892in}{1.836052in}}%
\pgfpathlineto{\pgfqpoint{1.075001in}{1.837144in}}%
\pgfpathlineto{\pgfqpoint{1.075978in}{1.838152in}}%
\pgfpathlineto{\pgfqpoint{1.077030in}{1.839160in}}%
\pgfpathlineto{\pgfqpoint{1.078383in}{1.840168in}}%
\pgfpathlineto{\pgfqpoint{1.079040in}{1.840924in}}%
\pgfpathlineto{\pgfqpoint{1.080844in}{1.841932in}}%
\pgfpathlineto{\pgfqpoint{1.081877in}{1.842688in}}%
\pgfpathlineto{\pgfqpoint{1.082741in}{1.843696in}}%
\pgfpathlineto{\pgfqpoint{1.083850in}{1.845292in}}%
\pgfpathlineto{\pgfqpoint{1.084921in}{1.846300in}}%
\pgfpathlineto{\pgfqpoint{1.086029in}{1.847812in}}%
\pgfpathlineto{\pgfqpoint{1.087382in}{1.848904in}}%
\pgfpathlineto{\pgfqpoint{1.088265in}{1.849912in}}%
\pgfpathlineto{\pgfqpoint{1.090238in}{1.851004in}}%
\pgfpathlineto{\pgfqpoint{1.091233in}{1.851844in}}%
\pgfpathlineto{\pgfqpoint{1.092680in}{1.852768in}}%
\pgfpathlineto{\pgfqpoint{1.093582in}{1.853440in}}%
\pgfpathlineto{\pgfqpoint{1.094747in}{1.854532in}}%
\pgfpathlineto{\pgfqpoint{1.095836in}{1.855372in}}%
\pgfpathlineto{\pgfqpoint{1.097133in}{1.856464in}}%
\pgfpathlineto{\pgfqpoint{1.098241in}{1.858312in}}%
\pgfpathlineto{\pgfqpoint{1.100045in}{1.859404in}}%
\pgfpathlineto{\pgfqpoint{1.100853in}{1.860580in}}%
\pgfpathlineto{\pgfqpoint{1.102281in}{1.861588in}}%
\pgfpathlineto{\pgfqpoint{1.103295in}{1.863268in}}%
\pgfpathlineto{\pgfqpoint{1.105155in}{1.864360in}}%
\pgfpathlineto{\pgfqpoint{1.106113in}{1.865284in}}%
\pgfpathlineto{\pgfqpoint{1.107654in}{1.866376in}}%
\pgfpathlineto{\pgfqpoint{1.108744in}{1.867552in}}%
\pgfpathlineto{\pgfqpoint{1.110284in}{1.868644in}}%
\pgfpathlineto{\pgfqpoint{1.111393in}{1.870072in}}%
\pgfpathlineto{\pgfqpoint{1.112746in}{1.871164in}}%
\pgfpathlineto{\pgfqpoint{1.113835in}{1.872340in}}%
\pgfpathlineto{\pgfqpoint{1.115226in}{1.873348in}}%
\pgfpathlineto{\pgfqpoint{1.116296in}{1.874608in}}%
\pgfpathlineto{\pgfqpoint{1.118513in}{1.875700in}}%
\pgfpathlineto{\pgfqpoint{1.119622in}{1.876960in}}%
\pgfpathlineto{\pgfqpoint{1.120937in}{1.877968in}}%
\pgfpathlineto{\pgfqpoint{1.121952in}{1.878556in}}%
\pgfpathlineto{\pgfqpoint{1.123643in}{1.879648in}}%
\pgfpathlineto{\pgfqpoint{1.124695in}{1.880740in}}%
\pgfpathlineto{\pgfqpoint{1.126066in}{1.881832in}}%
\pgfpathlineto{\pgfqpoint{1.127137in}{1.883260in}}%
\pgfpathlineto{\pgfqpoint{1.128678in}{1.884268in}}%
\pgfpathlineto{\pgfqpoint{1.129786in}{1.885612in}}%
\pgfpathlineto{\pgfqpoint{1.130876in}{1.886704in}}%
\pgfpathlineto{\pgfqpoint{1.131909in}{1.887628in}}%
\pgfpathlineto{\pgfqpoint{1.133844in}{1.888720in}}%
\pgfpathlineto{\pgfqpoint{1.134878in}{1.889644in}}%
\pgfpathlineto{\pgfqpoint{1.136850in}{1.890736in}}%
\pgfpathlineto{\pgfqpoint{1.137903in}{1.891576in}}%
\pgfpathlineto{\pgfqpoint{1.139349in}{1.892668in}}%
\pgfpathlineto{\pgfqpoint{1.140458in}{1.893676in}}%
\pgfpathlineto{\pgfqpoint{1.142449in}{1.894768in}}%
\pgfpathlineto{\pgfqpoint{1.143520in}{1.896448in}}%
\pgfpathlineto{\pgfqpoint{1.145549in}{1.897540in}}%
\pgfpathlineto{\pgfqpoint{1.146601in}{1.898380in}}%
\pgfpathlineto{\pgfqpoint{1.148273in}{1.899472in}}%
\pgfpathlineto{\pgfqpoint{1.149363in}{1.900144in}}%
\pgfpathlineto{\pgfqpoint{1.150472in}{1.901236in}}%
\pgfpathlineto{\pgfqpoint{1.151505in}{1.901824in}}%
\pgfpathlineto{\pgfqpoint{1.153064in}{1.902916in}}%
\pgfpathlineto{\pgfqpoint{1.154060in}{1.903924in}}%
\pgfpathlineto{\pgfqpoint{1.155826in}{1.904932in}}%
\pgfpathlineto{\pgfqpoint{1.156916in}{1.905940in}}%
\pgfpathlineto{\pgfqpoint{1.158043in}{1.907032in}}%
\pgfpathlineto{\pgfqpoint{1.159133in}{1.908124in}}%
\pgfpathlineto{\pgfqpoint{1.161030in}{1.909216in}}%
\pgfpathlineto{\pgfqpoint{1.162007in}{1.910056in}}%
\pgfpathlineto{\pgfqpoint{1.163642in}{1.911148in}}%
\pgfpathlineto{\pgfqpoint{1.164713in}{1.911988in}}%
\pgfpathlineto{\pgfqpoint{1.166873in}{1.913080in}}%
\pgfpathlineto{\pgfqpoint{1.167926in}{1.914004in}}%
\pgfpathlineto{\pgfqpoint{1.170086in}{1.915096in}}%
\pgfpathlineto{\pgfqpoint{1.171138in}{1.916020in}}%
\pgfpathlineto{\pgfqpoint{1.173017in}{1.917112in}}%
\pgfpathlineto{\pgfqpoint{1.173881in}{1.917868in}}%
\pgfpathlineto{\pgfqpoint{1.175816in}{1.918960in}}%
\pgfpathlineto{\pgfqpoint{1.176831in}{1.919716in}}%
\pgfpathlineto{\pgfqpoint{1.178935in}{1.920808in}}%
\pgfpathlineto{\pgfqpoint{1.179912in}{1.922236in}}%
\pgfpathlineto{\pgfqpoint{1.180889in}{1.923328in}}%
\pgfpathlineto{\pgfqpoint{1.181735in}{1.924252in}}%
\pgfpathlineto{\pgfqpoint{1.184759in}{1.925344in}}%
\pgfpathlineto{\pgfqpoint{1.185718in}{1.926604in}}%
\pgfpathlineto{\pgfqpoint{1.187634in}{1.927696in}}%
\pgfpathlineto{\pgfqpoint{1.188649in}{1.928536in}}%
\pgfpathlineto{\pgfqpoint{1.190509in}{1.929628in}}%
\pgfpathlineto{\pgfqpoint{1.191429in}{1.930300in}}%
\pgfpathlineto{\pgfqpoint{1.193759in}{1.931392in}}%
\pgfpathlineto{\pgfqpoint{1.194867in}{1.932064in}}%
\pgfpathlineto{\pgfqpoint{1.196295in}{1.933156in}}%
\pgfpathlineto{\pgfqpoint{1.197404in}{1.934164in}}%
\pgfpathlineto{\pgfqpoint{1.199226in}{1.935256in}}%
\pgfpathlineto{\pgfqpoint{1.200241in}{1.935676in}}%
\pgfpathlineto{\pgfqpoint{1.202608in}{1.936768in}}%
\pgfpathlineto{\pgfqpoint{1.203435in}{1.937944in}}%
\pgfpathlineto{\pgfqpoint{1.205915in}{1.939036in}}%
\pgfpathlineto{\pgfqpoint{1.206910in}{1.939456in}}%
\pgfpathlineto{\pgfqpoint{1.209015in}{1.940548in}}%
\pgfpathlineto{\pgfqpoint{1.209484in}{1.940968in}}%
\pgfpathlineto{\pgfqpoint{1.212096in}{1.942060in}}%
\pgfpathlineto{\pgfqpoint{1.212960in}{1.942396in}}%
\pgfpathlineto{\pgfqpoint{1.215985in}{1.943488in}}%
\pgfpathlineto{\pgfqpoint{1.217075in}{1.944076in}}%
\pgfpathlineto{\pgfqpoint{1.218822in}{1.945084in}}%
\pgfpathlineto{\pgfqpoint{1.219912in}{1.946512in}}%
\pgfpathlineto{\pgfqpoint{1.222035in}{1.947604in}}%
\pgfpathlineto{\pgfqpoint{1.222655in}{1.948276in}}%
\pgfpathlineto{\pgfqpoint{1.224721in}{1.949368in}}%
\pgfpathlineto{\pgfqpoint{1.225473in}{1.949788in}}%
\pgfpathlineto{\pgfqpoint{1.227596in}{1.950880in}}%
\pgfpathlineto{\pgfqpoint{1.228704in}{1.951384in}}%
\pgfpathlineto{\pgfqpoint{1.230903in}{1.952476in}}%
\pgfpathlineto{\pgfqpoint{1.231955in}{1.953316in}}%
\pgfpathlineto{\pgfqpoint{1.233589in}{1.954240in}}%
\pgfpathlineto{\pgfqpoint{1.234585in}{1.955164in}}%
\pgfpathlineto{\pgfqpoint{1.235844in}{1.956256in}}%
\pgfpathlineto{\pgfqpoint{1.236595in}{1.956844in}}%
\pgfpathlineto{\pgfqpoint{1.237854in}{1.957852in}}%
\pgfpathlineto{\pgfqpoint{1.238963in}{1.958944in}}%
\pgfpathlineto{\pgfqpoint{1.241461in}{1.960036in}}%
\pgfpathlineto{\pgfqpoint{1.242513in}{1.960960in}}%
\pgfpathlineto{\pgfqpoint{1.244937in}{1.962052in}}%
\pgfpathlineto{\pgfqpoint{1.245482in}{1.962472in}}%
\pgfpathlineto{\pgfqpoint{1.249108in}{1.963564in}}%
\pgfpathlineto{\pgfqpoint{1.250123in}{1.964656in}}%
\pgfpathlineto{\pgfqpoint{1.251607in}{1.965748in}}%
\pgfpathlineto{\pgfqpoint{1.252415in}{1.966252in}}%
\pgfpathlineto{\pgfqpoint{1.254275in}{1.967344in}}%
\pgfpathlineto{\pgfqpoint{1.255346in}{1.968016in}}%
\pgfpathlineto{\pgfqpoint{1.256830in}{1.969024in}}%
\pgfpathlineto{\pgfqpoint{1.257938in}{1.969948in}}%
\pgfpathlineto{\pgfqpoint{1.259798in}{1.971040in}}%
\pgfpathlineto{\pgfqpoint{1.260775in}{1.971628in}}%
\pgfpathlineto{\pgfqpoint{1.263086in}{1.972720in}}%
\pgfpathlineto{\pgfqpoint{1.264063in}{1.973392in}}%
\pgfpathlineto{\pgfqpoint{1.266318in}{1.974484in}}%
\pgfpathlineto{\pgfqpoint{1.267351in}{1.975072in}}%
\pgfpathlineto{\pgfqpoint{1.270132in}{1.976164in}}%
\pgfpathlineto{\pgfqpoint{1.271127in}{1.976668in}}%
\pgfpathlineto{\pgfqpoint{1.272649in}{1.977760in}}%
\pgfpathlineto{\pgfqpoint{1.273457in}{1.978348in}}%
\pgfpathlineto{\pgfqpoint{1.276087in}{1.979440in}}%
\pgfpathlineto{\pgfqpoint{1.276970in}{1.980112in}}%
\pgfpathlineto{\pgfqpoint{1.279300in}{1.981204in}}%
\pgfpathlineto{\pgfqpoint{1.280239in}{1.982044in}}%
\pgfpathlineto{\pgfqpoint{1.282513in}{1.983136in}}%
\pgfpathlineto{\pgfqpoint{1.283302in}{1.983304in}}%
\pgfpathlineto{\pgfqpoint{1.285575in}{1.984396in}}%
\pgfpathlineto{\pgfqpoint{1.286515in}{1.985068in}}%
\pgfpathlineto{\pgfqpoint{1.288149in}{1.986160in}}%
\pgfpathlineto{\pgfqpoint{1.289220in}{1.986916in}}%
\pgfpathlineto{\pgfqpoint{1.291982in}{1.987924in}}%
\pgfpathlineto{\pgfqpoint{1.292564in}{1.988596in}}%
\pgfpathlineto{\pgfqpoint{1.294950in}{1.989604in}}%
\pgfpathlineto{\pgfqpoint{1.295646in}{1.989856in}}%
\pgfpathlineto{\pgfqpoint{1.298069in}{1.990948in}}%
\pgfpathlineto{\pgfqpoint{1.298952in}{1.991620in}}%
\pgfpathlineto{\pgfqpoint{1.301075in}{1.992628in}}%
\pgfpathlineto{\pgfqpoint{1.302033in}{1.993468in}}%
\pgfpathlineto{\pgfqpoint{1.303706in}{1.994560in}}%
\pgfpathlineto{\pgfqpoint{1.304401in}{1.994980in}}%
\pgfpathlineto{\pgfqpoint{1.306129in}{1.996072in}}%
\pgfpathlineto{\pgfqpoint{1.307238in}{1.997080in}}%
\pgfpathlineto{\pgfqpoint{1.308778in}{1.998172in}}%
\pgfpathlineto{\pgfqpoint{1.309699in}{1.998592in}}%
\pgfpathlineto{\pgfqpoint{1.311747in}{1.999600in}}%
\pgfpathlineto{\pgfqpoint{1.312592in}{2.000272in}}%
\pgfpathlineto{\pgfqpoint{1.315072in}{2.001364in}}%
\pgfpathlineto{\pgfqpoint{1.315955in}{2.001784in}}%
\pgfpathlineto{\pgfqpoint{1.318003in}{2.002876in}}%
\pgfpathlineto{\pgfqpoint{1.318623in}{2.003296in}}%
\pgfpathlineto{\pgfqpoint{1.320990in}{2.004388in}}%
\pgfpathlineto{\pgfqpoint{1.322024in}{2.004808in}}%
\pgfpathlineto{\pgfqpoint{1.323433in}{2.005900in}}%
\pgfpathlineto{\pgfqpoint{1.324391in}{2.006488in}}%
\pgfpathlineto{\pgfqpoint{1.328299in}{2.007580in}}%
\pgfpathlineto{\pgfqpoint{1.329201in}{2.007916in}}%
\pgfpathlineto{\pgfqpoint{1.331286in}{2.009008in}}%
\pgfpathlineto{\pgfqpoint{1.332338in}{2.009428in}}%
\pgfpathlineto{\pgfqpoint{1.334330in}{2.010520in}}%
\pgfpathlineto{\pgfqpoint{1.335382in}{2.011192in}}%
\pgfpathlineto{\pgfqpoint{1.336735in}{2.012284in}}%
\pgfpathlineto{\pgfqpoint{1.337599in}{2.012956in}}%
\pgfpathlineto{\pgfqpoint{1.339590in}{2.014048in}}%
\pgfpathlineto{\pgfqpoint{1.340549in}{2.014720in}}%
\pgfpathlineto{\pgfqpoint{1.343160in}{2.015728in}}%
\pgfpathlineto{\pgfqpoint{1.344062in}{2.016148in}}%
\pgfpathlineto{\pgfqpoint{1.346035in}{2.017240in}}%
\pgfpathlineto{\pgfqpoint{1.346974in}{2.017660in}}%
\pgfpathlineto{\pgfqpoint{1.350337in}{2.018752in}}%
\pgfpathlineto{\pgfqpoint{1.350901in}{2.019172in}}%
\pgfpathlineto{\pgfqpoint{1.355598in}{2.020264in}}%
\pgfpathlineto{\pgfqpoint{1.356499in}{2.020768in}}%
\pgfpathlineto{\pgfqpoint{1.358716in}{2.021860in}}%
\pgfpathlineto{\pgfqpoint{1.359618in}{2.022448in}}%
\pgfpathlineto{\pgfqpoint{1.363583in}{2.023540in}}%
\pgfpathlineto{\pgfqpoint{1.364635in}{2.024044in}}%
\pgfpathlineto{\pgfqpoint{1.366720in}{2.025136in}}%
\pgfpathlineto{\pgfqpoint{1.367810in}{2.025892in}}%
\pgfpathlineto{\pgfqpoint{1.372619in}{2.026984in}}%
\pgfpathlineto{\pgfqpoint{1.373672in}{2.027404in}}%
\pgfpathlineto{\pgfqpoint{1.376471in}{2.028412in}}%
\pgfpathlineto{\pgfqpoint{1.377579in}{2.029168in}}%
\pgfpathlineto{\pgfqpoint{1.380473in}{2.030260in}}%
\pgfpathlineto{\pgfqpoint{1.381243in}{2.030596in}}%
\pgfpathlineto{\pgfqpoint{1.383892in}{2.031688in}}%
\pgfpathlineto{\pgfqpoint{1.384080in}{2.031856in}}%
\pgfpathlineto{\pgfqpoint{1.389942in}{2.032948in}}%
\pgfpathlineto{\pgfqpoint{1.390637in}{2.033536in}}%
\pgfpathlineto{\pgfqpoint{1.395334in}{2.034628in}}%
\pgfpathlineto{\pgfqpoint{1.396349in}{2.034796in}}%
\pgfpathlineto{\pgfqpoint{1.398603in}{2.035804in}}%
\pgfpathlineto{\pgfqpoint{1.399561in}{2.036140in}}%
\pgfpathlineto{\pgfqpoint{1.402793in}{2.037232in}}%
\pgfpathlineto{\pgfqpoint{1.403901in}{2.037652in}}%
\pgfpathlineto{\pgfqpoint{1.406137in}{2.038744in}}%
\pgfpathlineto{\pgfqpoint{1.406983in}{2.039164in}}%
\pgfpathlineto{\pgfqpoint{1.409218in}{2.040256in}}%
\pgfpathlineto{\pgfqpoint{1.410233in}{2.040844in}}%
\pgfpathlineto{\pgfqpoint{1.412450in}{2.041852in}}%
\pgfpathlineto{\pgfqpoint{1.413464in}{2.042692in}}%
\pgfpathlineto{\pgfqpoint{1.418086in}{2.043784in}}%
\pgfpathlineto{\pgfqpoint{1.419119in}{2.044120in}}%
\pgfpathlineto{\pgfqpoint{1.424023in}{2.045212in}}%
\pgfpathlineto{\pgfqpoint{1.424887in}{2.045716in}}%
\pgfpathlineto{\pgfqpoint{1.428701in}{2.046808in}}%
\pgfpathlineto{\pgfqpoint{1.429810in}{2.047564in}}%
\pgfpathlineto{\pgfqpoint{1.432797in}{2.048656in}}%
\pgfpathlineto{\pgfqpoint{1.433586in}{2.049160in}}%
\pgfpathlineto{\pgfqpoint{1.435390in}{2.050252in}}%
\pgfpathlineto{\pgfqpoint{1.435859in}{2.050588in}}%
\pgfpathlineto{\pgfqpoint{1.438321in}{2.051680in}}%
\pgfpathlineto{\pgfqpoint{1.439335in}{2.052268in}}%
\pgfpathlineto{\pgfqpoint{1.443487in}{2.053360in}}%
\pgfpathlineto{\pgfqpoint{1.444558in}{2.054116in}}%
\pgfpathlineto{\pgfqpoint{1.449988in}{2.055208in}}%
\pgfpathlineto{\pgfqpoint{1.450383in}{2.055628in}}%
\pgfpathlineto{\pgfqpoint{1.454309in}{2.056720in}}%
\pgfpathlineto{\pgfqpoint{1.455418in}{2.057308in}}%
\pgfpathlineto{\pgfqpoint{1.458236in}{2.058400in}}%
\pgfpathlineto{\pgfqpoint{1.459307in}{2.058904in}}%
\pgfpathlineto{\pgfqpoint{1.462764in}{2.059996in}}%
\pgfpathlineto{\pgfqpoint{1.463872in}{2.060920in}}%
\pgfpathlineto{\pgfqpoint{1.467630in}{2.062012in}}%
\pgfpathlineto{\pgfqpoint{1.468569in}{2.062432in}}%
\pgfpathlineto{\pgfqpoint{1.471406in}{2.063524in}}%
\pgfpathlineto{\pgfqpoint{1.472458in}{2.064028in}}%
\pgfpathlineto{\pgfqpoint{1.476460in}{2.065120in}}%
\pgfpathlineto{\pgfqpoint{1.477362in}{2.065456in}}%
\pgfpathlineto{\pgfqpoint{1.479729in}{2.066548in}}%
\pgfpathlineto{\pgfqpoint{1.480800in}{2.067388in}}%
\pgfpathlineto{\pgfqpoint{1.484144in}{2.068480in}}%
\pgfpathlineto{\pgfqpoint{1.485009in}{2.068816in}}%
\pgfpathlineto{\pgfqpoint{1.489649in}{2.069908in}}%
\pgfpathlineto{\pgfqpoint{1.490701in}{2.070328in}}%
\pgfpathlineto{\pgfqpoint{1.494666in}{2.071420in}}%
\pgfpathlineto{\pgfqpoint{1.495455in}{2.071588in}}%
\pgfpathlineto{\pgfqpoint{1.501467in}{2.072680in}}%
\pgfpathlineto{\pgfqpoint{1.502387in}{2.073268in}}%
\pgfpathlineto{\pgfqpoint{1.505168in}{2.074360in}}%
\pgfpathlineto{\pgfqpoint{1.505844in}{2.074780in}}%
\pgfpathlineto{\pgfqpoint{1.511330in}{2.075872in}}%
\pgfpathlineto{\pgfqpoint{1.512383in}{2.076292in}}%
\pgfpathlineto{\pgfqpoint{1.515821in}{2.077384in}}%
\pgfpathlineto{\pgfqpoint{1.516723in}{2.077636in}}%
\pgfpathlineto{\pgfqpoint{1.521250in}{2.078728in}}%
\pgfpathlineto{\pgfqpoint{1.521250in}{2.078812in}}%
\pgfpathlineto{\pgfqpoint{1.526210in}{2.079904in}}%
\pgfpathlineto{\pgfqpoint{1.527281in}{2.080408in}}%
\pgfpathlineto{\pgfqpoint{1.530626in}{2.081500in}}%
\pgfpathlineto{\pgfqpoint{1.531640in}{2.081920in}}%
\pgfpathlineto{\pgfqpoint{1.534984in}{2.083012in}}%
\pgfpathlineto{\pgfqpoint{1.535999in}{2.083432in}}%
\pgfpathlineto{\pgfqpoint{1.538967in}{2.084524in}}%
\pgfpathlineto{\pgfqpoint{1.539888in}{2.084944in}}%
\pgfpathlineto{\pgfqpoint{1.544435in}{2.086036in}}%
\pgfpathlineto{\pgfqpoint{1.545130in}{2.086456in}}%
\pgfpathlineto{\pgfqpoint{1.548399in}{2.087548in}}%
\pgfpathlineto{\pgfqpoint{1.548399in}{2.087632in}}%
\pgfpathlineto{\pgfqpoint{1.553378in}{2.088724in}}%
\pgfpathlineto{\pgfqpoint{1.554467in}{2.089228in}}%
\pgfpathlineto{\pgfqpoint{1.559503in}{2.090320in}}%
\pgfpathlineto{\pgfqpoint{1.560461in}{2.090488in}}%
\pgfpathlineto{\pgfqpoint{1.563598in}{2.091580in}}%
\pgfpathlineto{\pgfqpoint{1.564275in}{2.091916in}}%
\pgfpathlineto{\pgfqpoint{1.569817in}{2.093008in}}%
\pgfpathlineto{\pgfqpoint{1.570756in}{2.093764in}}%
\pgfpathlineto{\pgfqpoint{1.574552in}{2.094856in}}%
\pgfpathlineto{\pgfqpoint{1.575040in}{2.095192in}}%
\pgfpathlineto{\pgfqpoint{1.580958in}{2.096284in}}%
\pgfpathlineto{\pgfqpoint{1.580958in}{2.096368in}}%
\pgfpathlineto{\pgfqpoint{1.588924in}{2.097460in}}%
\pgfpathlineto{\pgfqpoint{1.589394in}{2.097796in}}%
\pgfpathlineto{\pgfqpoint{1.594354in}{2.098888in}}%
\pgfpathlineto{\pgfqpoint{1.594673in}{2.099056in}}%
\pgfpathlineto{\pgfqpoint{1.601212in}{2.100148in}}%
\pgfpathlineto{\pgfqpoint{1.602019in}{2.100484in}}%
\pgfpathlineto{\pgfqpoint{1.606735in}{2.101576in}}%
\pgfpathlineto{\pgfqpoint{1.607581in}{2.102080in}}%
\pgfpathlineto{\pgfqpoint{1.613405in}{2.103172in}}%
\pgfpathlineto{\pgfqpoint{1.613856in}{2.103508in}}%
\pgfpathlineto{\pgfqpoint{1.618797in}{2.104600in}}%
\pgfpathlineto{\pgfqpoint{1.619473in}{2.105020in}}%
\pgfpathlineto{\pgfqpoint{1.629299in}{2.106112in}}%
\pgfpathlineto{\pgfqpoint{1.630164in}{2.106364in}}%
\pgfpathlineto{\pgfqpoint{1.633376in}{2.107456in}}%
\pgfpathlineto{\pgfqpoint{1.633921in}{2.107708in}}%
\pgfpathlineto{\pgfqpoint{1.639032in}{2.108800in}}%
\pgfpathlineto{\pgfqpoint{1.639520in}{2.108968in}}%
\pgfpathlineto{\pgfqpoint{1.648444in}{2.109976in}}%
\pgfpathlineto{\pgfqpoint{1.649403in}{2.110228in}}%
\pgfpathlineto{\pgfqpoint{1.655959in}{2.111320in}}%
\pgfpathlineto{\pgfqpoint{1.657049in}{2.111824in}}%
\pgfpathlineto{\pgfqpoint{1.662159in}{2.112916in}}%
\pgfpathlineto{\pgfqpoint{1.663155in}{2.113504in}}%
\pgfpathlineto{\pgfqpoint{1.669750in}{2.114596in}}%
\pgfpathlineto{\pgfqpoint{1.670802in}{2.114932in}}%
\pgfpathlineto{\pgfqpoint{1.678167in}{2.116024in}}%
\pgfpathlineto{\pgfqpoint{1.679106in}{2.116360in}}%
\pgfpathlineto{\pgfqpoint{1.686114in}{2.117452in}}%
\pgfpathlineto{\pgfqpoint{1.686959in}{2.117620in}}%
\pgfpathlineto{\pgfqpoint{1.690041in}{2.118712in}}%
\pgfpathlineto{\pgfqpoint{1.690999in}{2.118964in}}%
\pgfpathlineto{\pgfqpoint{1.697462in}{2.120056in}}%
\pgfpathlineto{\pgfqpoint{1.698007in}{2.120392in}}%
\pgfpathlineto{\pgfqpoint{1.706987in}{2.121484in}}%
\pgfpathlineto{\pgfqpoint{1.707927in}{2.121736in}}%
\pgfpathlineto{\pgfqpoint{1.712830in}{2.122828in}}%
\pgfpathlineto{\pgfqpoint{1.713338in}{2.123164in}}%
\pgfpathlineto{\pgfqpoint{1.718129in}{2.124256in}}%
\pgfpathlineto{\pgfqpoint{1.718129in}{2.124340in}}%
\pgfpathlineto{\pgfqpoint{1.724084in}{2.125432in}}%
\pgfpathlineto{\pgfqpoint{1.725024in}{2.125600in}}%
\pgfpathlineto{\pgfqpoint{1.731675in}{2.126692in}}%
\pgfpathlineto{\pgfqpoint{1.732614in}{2.126944in}}%
\pgfpathlineto{\pgfqpoint{1.738363in}{2.128036in}}%
\pgfpathlineto{\pgfqpoint{1.739453in}{2.128372in}}%
\pgfpathlineto{\pgfqpoint{1.744544in}{2.129464in}}%
\pgfpathlineto{\pgfqpoint{1.744544in}{2.129548in}}%
\pgfpathlineto{\pgfqpoint{1.754840in}{2.130640in}}%
\pgfpathlineto{\pgfqpoint{1.755159in}{2.130808in}}%
\pgfpathlineto{\pgfqpoint{1.761979in}{2.131900in}}%
\pgfpathlineto{\pgfqpoint{1.762938in}{2.132152in}}%
\pgfpathlineto{\pgfqpoint{1.769006in}{2.133244in}}%
\pgfpathlineto{\pgfqpoint{1.769006in}{2.133328in}}%
\pgfpathlineto{\pgfqpoint{1.777310in}{2.134420in}}%
\pgfpathlineto{\pgfqpoint{1.778306in}{2.134588in}}%
\pgfpathlineto{\pgfqpoint{1.783679in}{2.135680in}}%
\pgfpathlineto{\pgfqpoint{1.784356in}{2.135848in}}%
\pgfpathlineto{\pgfqpoint{1.791646in}{2.136940in}}%
\pgfpathlineto{\pgfqpoint{1.792604in}{2.137276in}}%
\pgfpathlineto{\pgfqpoint{1.800025in}{2.138368in}}%
\pgfpathlineto{\pgfqpoint{1.800983in}{2.138788in}}%
\pgfpathlineto{\pgfqpoint{1.807484in}{2.139880in}}%
\pgfpathlineto{\pgfqpoint{1.808066in}{2.140300in}}%
\pgfpathlineto{\pgfqpoint{1.813515in}{2.141392in}}%
\pgfpathlineto{\pgfqpoint{1.814097in}{2.141644in}}%
\pgfpathlineto{\pgfqpoint{1.821011in}{2.142736in}}%
\pgfpathlineto{\pgfqpoint{1.822101in}{2.143072in}}%
\pgfpathlineto{\pgfqpoint{1.828338in}{2.144164in}}%
\pgfpathlineto{\pgfqpoint{1.829296in}{2.144836in}}%
\pgfpathlineto{\pgfqpoint{1.832321in}{2.145928in}}%
\pgfpathlineto{\pgfqpoint{1.832979in}{2.146264in}}%
\pgfpathlineto{\pgfqpoint{1.842730in}{2.147356in}}%
\pgfpathlineto{\pgfqpoint{1.842730in}{2.147440in}}%
\pgfpathlineto{\pgfqpoint{1.850639in}{2.148532in}}%
\pgfpathlineto{\pgfqpoint{1.851184in}{2.148952in}}%
\pgfpathlineto{\pgfqpoint{1.858718in}{2.150044in}}%
\pgfpathlineto{\pgfqpoint{1.858718in}{2.150128in}}%
\pgfpathlineto{\pgfqpoint{1.866572in}{2.151220in}}%
\pgfpathlineto{\pgfqpoint{1.867304in}{2.151472in}}%
\pgfpathlineto{\pgfqpoint{1.874068in}{2.152564in}}%
\pgfpathlineto{\pgfqpoint{1.874632in}{2.152900in}}%
\pgfpathlineto{\pgfqpoint{1.881301in}{2.153992in}}%
\pgfpathlineto{\pgfqpoint{1.882203in}{2.154412in}}%
\pgfpathlineto{\pgfqpoint{1.889887in}{2.155504in}}%
\pgfpathlineto{\pgfqpoint{1.890432in}{2.155924in}}%
\pgfpathlineto{\pgfqpoint{1.902419in}{2.157016in}}%
\pgfpathlineto{\pgfqpoint{1.903358in}{2.157268in}}%
\pgfpathlineto{\pgfqpoint{1.911268in}{2.158360in}}%
\pgfpathlineto{\pgfqpoint{1.912038in}{2.158696in}}%
\pgfpathlineto{\pgfqpoint{1.916641in}{2.159788in}}%
\pgfpathlineto{\pgfqpoint{1.917618in}{2.160124in}}%
\pgfpathlineto{\pgfqpoint{1.927970in}{2.161216in}}%
\pgfpathlineto{\pgfqpoint{1.929079in}{2.161468in}}%
\pgfpathlineto{\pgfqpoint{1.939206in}{2.162560in}}%
\pgfpathlineto{\pgfqpoint{1.940239in}{2.162980in}}%
\pgfpathlineto{\pgfqpoint{1.957843in}{2.164072in}}%
\pgfpathlineto{\pgfqpoint{1.958519in}{2.164324in}}%
\pgfpathlineto{\pgfqpoint{1.967575in}{2.165416in}}%
\pgfpathlineto{\pgfqpoint{1.967895in}{2.165584in}}%
\pgfpathlineto{\pgfqpoint{1.981103in}{2.166676in}}%
\pgfpathlineto{\pgfqpoint{1.981422in}{2.166844in}}%
\pgfpathlineto{\pgfqpoint{1.989069in}{2.167936in}}%
\pgfpathlineto{\pgfqpoint{1.990027in}{2.168272in}}%
\pgfpathlineto{\pgfqpoint{1.999045in}{2.169364in}}%
\pgfpathlineto{\pgfqpoint{2.000135in}{2.169784in}}%
\pgfpathlineto{\pgfqpoint{2.009886in}{2.170876in}}%
\pgfpathlineto{\pgfqpoint{2.010581in}{2.171212in}}%
\pgfpathlineto{\pgfqpoint{2.020332in}{2.172304in}}%
\pgfpathlineto{\pgfqpoint{2.020332in}{2.172388in}}%
\pgfpathlineto{\pgfqpoint{2.027959in}{2.173480in}}%
\pgfpathlineto{\pgfqpoint{2.028561in}{2.173648in}}%
\pgfpathlineto{\pgfqpoint{2.037466in}{2.174740in}}%
\pgfpathlineto{\pgfqpoint{2.037466in}{2.174824in}}%
\pgfpathlineto{\pgfqpoint{2.044887in}{2.175916in}}%
\pgfpathlineto{\pgfqpoint{2.045226in}{2.176168in}}%
\pgfpathlineto{\pgfqpoint{2.053962in}{2.177260in}}%
\pgfpathlineto{\pgfqpoint{2.054281in}{2.177428in}}%
\pgfpathlineto{\pgfqpoint{2.064972in}{2.178520in}}%
\pgfpathlineto{\pgfqpoint{2.065592in}{2.178688in}}%
\pgfpathlineto{\pgfqpoint{2.071040in}{2.179780in}}%
\pgfpathlineto{\pgfqpoint{2.072130in}{2.180200in}}%
\pgfpathlineto{\pgfqpoint{2.082294in}{2.181292in}}%
\pgfpathlineto{\pgfqpoint{2.082613in}{2.181544in}}%
\pgfpathlineto{\pgfqpoint{2.093247in}{2.182636in}}%
\pgfpathlineto{\pgfqpoint{2.093247in}{2.182720in}}%
\pgfpathlineto{\pgfqpoint{2.103618in}{2.183812in}}%
\pgfpathlineto{\pgfqpoint{2.104483in}{2.184064in}}%
\pgfpathlineto{\pgfqpoint{2.118536in}{2.185156in}}%
\pgfpathlineto{\pgfqpoint{2.119269in}{2.185324in}}%
\pgfpathlineto{\pgfqpoint{2.127253in}{2.186416in}}%
\pgfpathlineto{\pgfqpoint{2.128212in}{2.186752in}}%
\pgfpathlineto{\pgfqpoint{2.140536in}{2.187844in}}%
\pgfpathlineto{\pgfqpoint{2.141419in}{2.188012in}}%
\pgfpathlineto{\pgfqpoint{2.153012in}{2.189104in}}%
\pgfpathlineto{\pgfqpoint{2.153632in}{2.189440in}}%
\pgfpathlineto{\pgfqpoint{2.163307in}{2.190532in}}%
\pgfpathlineto{\pgfqpoint{2.164266in}{2.190700in}}%
\pgfpathlineto{\pgfqpoint{2.176496in}{2.191792in}}%
\pgfpathlineto{\pgfqpoint{2.176496in}{2.191876in}}%
\pgfpathlineto{\pgfqpoint{2.186980in}{2.192968in}}%
\pgfpathlineto{\pgfqpoint{2.187619in}{2.193136in}}%
\pgfpathlineto{\pgfqpoint{2.198441in}{2.194228in}}%
\pgfpathlineto{\pgfqpoint{2.199249in}{2.194396in}}%
\pgfpathlineto{\pgfqpoint{2.210540in}{2.195488in}}%
\pgfpathlineto{\pgfqpoint{2.211573in}{2.195740in}}%
\pgfpathlineto{\pgfqpoint{2.222715in}{2.196832in}}%
\pgfpathlineto{\pgfqpoint{2.222715in}{2.196916in}}%
\pgfpathlineto{\pgfqpoint{2.237275in}{2.198008in}}%
\pgfpathlineto{\pgfqpoint{2.238365in}{2.198176in}}%
\pgfpathlineto{\pgfqpoint{2.249055in}{2.199268in}}%
\pgfpathlineto{\pgfqpoint{2.249675in}{2.199520in}}%
\pgfpathlineto{\pgfqpoint{2.261944in}{2.200612in}}%
\pgfpathlineto{\pgfqpoint{2.262564in}{2.200864in}}%
\pgfpathlineto{\pgfqpoint{2.274475in}{2.201956in}}%
\pgfpathlineto{\pgfqpoint{2.274475in}{2.202040in}}%
\pgfpathlineto{\pgfqpoint{2.293752in}{2.203132in}}%
\pgfpathlineto{\pgfqpoint{2.293752in}{2.203216in}}%
\pgfpathlineto{\pgfqpoint{2.311337in}{2.204308in}}%
\pgfpathlineto{\pgfqpoint{2.311337in}{2.204392in}}%
\pgfpathlineto{\pgfqpoint{2.328979in}{2.205484in}}%
\pgfpathlineto{\pgfqpoint{2.329881in}{2.205652in}}%
\pgfpathlineto{\pgfqpoint{2.345212in}{2.206744in}}%
\pgfpathlineto{\pgfqpoint{2.345212in}{2.206828in}}%
\pgfpathlineto{\pgfqpoint{2.355939in}{2.207920in}}%
\pgfpathlineto{\pgfqpoint{2.355939in}{2.208004in}}%
\pgfpathlineto{\pgfqpoint{2.366198in}{2.209096in}}%
\pgfpathlineto{\pgfqpoint{2.367193in}{2.209432in}}%
\pgfpathlineto{\pgfqpoint{2.378259in}{2.210524in}}%
\pgfpathlineto{\pgfqpoint{2.378259in}{2.210608in}}%
\pgfpathlineto{\pgfqpoint{2.397367in}{2.211700in}}%
\pgfpathlineto{\pgfqpoint{2.397761in}{2.211868in}}%
\pgfpathlineto{\pgfqpoint{2.412529in}{2.212960in}}%
\pgfpathlineto{\pgfqpoint{2.413224in}{2.213380in}}%
\pgfpathlineto{\pgfqpoint{2.428216in}{2.214472in}}%
\pgfpathlineto{\pgfqpoint{2.428874in}{2.214640in}}%
\pgfpathlineto{\pgfqpoint{2.437779in}{2.215732in}}%
\pgfpathlineto{\pgfqpoint{2.437779in}{2.215816in}}%
\pgfpathlineto{\pgfqpoint{2.453505in}{2.216908in}}%
\pgfpathlineto{\pgfqpoint{2.453505in}{2.216992in}}%
\pgfpathlineto{\pgfqpoint{2.470771in}{2.218084in}}%
\pgfpathlineto{\pgfqpoint{2.471429in}{2.218336in}}%
\pgfpathlineto{\pgfqpoint{2.485557in}{2.219428in}}%
\pgfpathlineto{\pgfqpoint{2.485970in}{2.219596in}}%
\pgfpathlineto{\pgfqpoint{2.506055in}{2.220688in}}%
\pgfpathlineto{\pgfqpoint{2.506055in}{2.220772in}}%
\pgfpathlineto{\pgfqpoint{2.519413in}{2.221864in}}%
\pgfpathlineto{\pgfqpoint{2.519789in}{2.222032in}}%
\pgfpathlineto{\pgfqpoint{2.529464in}{2.223124in}}%
\pgfpathlineto{\pgfqpoint{2.529558in}{2.223292in}}%
\pgfpathlineto{\pgfqpoint{2.547557in}{2.224384in}}%
\pgfpathlineto{\pgfqpoint{2.548177in}{2.224552in}}%
\pgfpathlineto{\pgfqpoint{2.567472in}{2.225644in}}%
\pgfpathlineto{\pgfqpoint{2.567886in}{2.225812in}}%
\pgfpathlineto{\pgfqpoint{2.584532in}{2.226904in}}%
\pgfpathlineto{\pgfqpoint{2.585452in}{2.227072in}}%
\pgfpathlineto{\pgfqpoint{2.595616in}{2.228164in}}%
\pgfpathlineto{\pgfqpoint{2.596105in}{2.228332in}}%
\pgfpathlineto{\pgfqpoint{2.612018in}{2.229424in}}%
\pgfpathlineto{\pgfqpoint{2.612300in}{2.229592in}}%
\pgfpathlineto{\pgfqpoint{2.626767in}{2.230684in}}%
\pgfpathlineto{\pgfqpoint{2.626767in}{2.230768in}}%
\pgfpathlineto{\pgfqpoint{2.639881in}{2.231860in}}%
\pgfpathlineto{\pgfqpoint{2.640933in}{2.232028in}}%
\pgfpathlineto{\pgfqpoint{2.653821in}{2.233120in}}%
\pgfpathlineto{\pgfqpoint{2.654836in}{2.233288in}}%
\pgfpathlineto{\pgfqpoint{2.667292in}{2.234380in}}%
\pgfpathlineto{\pgfqpoint{2.668269in}{2.234716in}}%
\pgfpathlineto{\pgfqpoint{2.684934in}{2.235808in}}%
\pgfpathlineto{\pgfqpoint{2.685404in}{2.235976in}}%
\pgfpathlineto{\pgfqpoint{2.702163in}{2.237068in}}%
\pgfpathlineto{\pgfqpoint{2.702670in}{2.237320in}}%
\pgfpathlineto{\pgfqpoint{2.719278in}{2.238412in}}%
\pgfpathlineto{\pgfqpoint{2.720067in}{2.238580in}}%
\pgfpathlineto{\pgfqpoint{2.736169in}{2.239672in}}%
\pgfpathlineto{\pgfqpoint{2.737221in}{2.240092in}}%
\pgfpathlineto{\pgfqpoint{2.756685in}{2.241184in}}%
\pgfpathlineto{\pgfqpoint{2.756685in}{2.241352in}}%
\pgfpathlineto{\pgfqpoint{2.772166in}{2.242444in}}%
\pgfpathlineto{\pgfqpoint{2.773275in}{2.242780in}}%
\pgfpathlineto{\pgfqpoint{2.792551in}{2.243872in}}%
\pgfpathlineto{\pgfqpoint{2.792683in}{2.244040in}}%
\pgfpathlineto{\pgfqpoint{2.803655in}{2.245048in}}%
\pgfpathlineto{\pgfqpoint{2.803655in}{2.245216in}}%
\pgfpathlineto{\pgfqpoint{2.818741in}{2.246308in}}%
\pgfpathlineto{\pgfqpoint{2.818741in}{2.246392in}}%
\pgfpathlineto{\pgfqpoint{2.834730in}{2.247484in}}%
\pgfpathlineto{\pgfqpoint{2.835331in}{2.247820in}}%
\pgfpathlineto{\pgfqpoint{2.844875in}{2.248912in}}%
\pgfpathlineto{\pgfqpoint{2.844875in}{2.248996in}}%
\pgfpathlineto{\pgfqpoint{2.859868in}{2.250088in}}%
\pgfpathlineto{\pgfqpoint{2.860789in}{2.250508in}}%
\pgfpathlineto{\pgfqpoint{2.873996in}{2.251600in}}%
\pgfpathlineto{\pgfqpoint{2.875049in}{2.251768in}}%
\pgfpathlineto{\pgfqpoint{2.886021in}{2.252860in}}%
\pgfpathlineto{\pgfqpoint{2.886472in}{2.253028in}}%
\pgfpathlineto{\pgfqpoint{2.890192in}{2.253196in}}%
\pgfpathlineto{\pgfqpoint{2.920853in}{2.263444in}}%
\pgfpathlineto{\pgfqpoint{2.920853in}{2.263444in}}%
\pgfusepath{stroke}%
\end{pgfscope}%
\begin{pgfscope}%
\pgfsetrectcap%
\pgfsetmiterjoin%
\pgfsetlinewidth{0.803000pt}%
\definecolor{currentstroke}{rgb}{0.000000,0.000000,0.000000}%
\pgfsetstrokecolor{currentstroke}%
\pgfsetdash{}{0pt}%
\pgfpathmoveto{\pgfqpoint{0.553581in}{0.499444in}}%
\pgfpathlineto{\pgfqpoint{0.553581in}{2.347444in}}%
\pgfusepath{stroke}%
\end{pgfscope}%
\begin{pgfscope}%
\pgfsetrectcap%
\pgfsetmiterjoin%
\pgfsetlinewidth{0.803000pt}%
\definecolor{currentstroke}{rgb}{0.000000,0.000000,0.000000}%
\pgfsetstrokecolor{currentstroke}%
\pgfsetdash{}{0pt}%
\pgfpathmoveto{\pgfqpoint{3.033581in}{0.499444in}}%
\pgfpathlineto{\pgfqpoint{3.033581in}{2.347444in}}%
\pgfusepath{stroke}%
\end{pgfscope}%
\begin{pgfscope}%
\pgfsetrectcap%
\pgfsetmiterjoin%
\pgfsetlinewidth{0.803000pt}%
\definecolor{currentstroke}{rgb}{0.000000,0.000000,0.000000}%
\pgfsetstrokecolor{currentstroke}%
\pgfsetdash{}{0pt}%
\pgfpathmoveto{\pgfqpoint{0.553581in}{0.499444in}}%
\pgfpathlineto{\pgfqpoint{3.033581in}{0.499444in}}%
\pgfusepath{stroke}%
\end{pgfscope}%
\begin{pgfscope}%
\pgfsetrectcap%
\pgfsetmiterjoin%
\pgfsetlinewidth{0.803000pt}%
\definecolor{currentstroke}{rgb}{0.000000,0.000000,0.000000}%
\pgfsetstrokecolor{currentstroke}%
\pgfsetdash{}{0pt}%
\pgfpathmoveto{\pgfqpoint{0.553581in}{2.347444in}}%
\pgfpathlineto{\pgfqpoint{3.033581in}{2.347444in}}%
\pgfusepath{stroke}%
\end{pgfscope}%
\begin{pgfscope}%
\pgfsetbuttcap%
\pgfsetmiterjoin%
\definecolor{currentfill}{rgb}{1.000000,1.000000,1.000000}%
\pgfsetfillcolor{currentfill}%
\pgfsetlinewidth{1.003750pt}%
\definecolor{currentstroke}{rgb}{1.000000,1.000000,1.000000}%
\pgfsetstrokecolor{currentstroke}%
\pgfsetdash{}{0pt}%
\pgfpathmoveto{\pgfqpoint{1.738420in}{2.054860in}}%
\pgfpathlineto{\pgfqpoint{2.165920in}{2.054860in}}%
\pgfpathlineto{\pgfqpoint{2.165920in}{2.289305in}}%
\pgfpathlineto{\pgfqpoint{1.738420in}{2.289305in}}%
\pgfpathlineto{\pgfqpoint{1.738420in}{2.054860in}}%
\pgfpathclose%
\pgfusepath{stroke,fill}%
\end{pgfscope}%
\begin{pgfscope}%
\definecolor{textcolor}{rgb}{0.000000,0.000000,0.000000}%
\pgfsetstrokecolor{textcolor}%
\pgfsetfillcolor{textcolor}%
\pgftext[x=1.793975in,y=2.137360in,left,base]{\color{textcolor}\rmfamily\fontsize{10.000000}{12.000000}\selectfont 0.305}%
\end{pgfscope}%
\begin{pgfscope}%
\pgfsetbuttcap%
\pgfsetmiterjoin%
\definecolor{currentfill}{rgb}{1.000000,1.000000,1.000000}%
\pgfsetfillcolor{currentfill}%
\pgfsetlinewidth{1.003750pt}%
\definecolor{currentstroke}{rgb}{1.000000,1.000000,1.000000}%
\pgfsetstrokecolor{currentstroke}%
\pgfsetdash{}{0pt}%
\pgfpathmoveto{\pgfqpoint{0.771915in}{1.340944in}}%
\pgfpathlineto{\pgfqpoint{1.199415in}{1.340944in}}%
\pgfpathlineto{\pgfqpoint{1.199415in}{1.575389in}}%
\pgfpathlineto{\pgfqpoint{0.771915in}{1.575389in}}%
\pgfpathlineto{\pgfqpoint{0.771915in}{1.340944in}}%
\pgfpathclose%
\pgfusepath{stroke,fill}%
\end{pgfscope}%
\begin{pgfscope}%
\definecolor{textcolor}{rgb}{0.000000,0.000000,0.000000}%
\pgfsetstrokecolor{textcolor}%
\pgfsetfillcolor{textcolor}%
\pgftext[x=0.827470in,y=1.423444in,left,base]{\color{textcolor}\rmfamily\fontsize{10.000000}{12.000000}\selectfont 0.688}%
\end{pgfscope}%
\begin{pgfscope}%
\definecolor{textcolor}{rgb}{0.000000,0.000000,0.000000}%
\pgfsetstrokecolor{textcolor}%
\pgfsetfillcolor{textcolor}%
\pgftext[x=1.793581in,y=2.430778in,,base]{\color{textcolor}\rmfamily\fontsize{12.000000}{14.400000}\selectfont ROC Curve}%
\end{pgfscope}%
\begin{pgfscope}%
\pgfsetbuttcap%
\pgfsetmiterjoin%
\definecolor{currentfill}{rgb}{1.000000,1.000000,1.000000}%
\pgfsetfillcolor{currentfill}%
\pgfsetfillopacity{0.800000}%
\pgfsetlinewidth{1.003750pt}%
\definecolor{currentstroke}{rgb}{0.800000,0.800000,0.800000}%
\pgfsetstrokecolor{currentstroke}%
\pgfsetstrokeopacity{0.800000}%
\pgfsetdash{}{0pt}%
\pgfpathmoveto{\pgfqpoint{0.800942in}{0.568889in}}%
\pgfpathlineto{\pgfqpoint{2.936358in}{0.568889in}}%
\pgfpathquadraticcurveto{\pgfqpoint{2.964136in}{0.568889in}}{\pgfqpoint{2.964136in}{0.596666in}}%
\pgfpathlineto{\pgfqpoint{2.964136in}{0.791111in}}%
\pgfpathquadraticcurveto{\pgfqpoint{2.964136in}{0.818888in}}{\pgfqpoint{2.936358in}{0.818888in}}%
\pgfpathlineto{\pgfqpoint{0.800942in}{0.818888in}}%
\pgfpathquadraticcurveto{\pgfqpoint{0.773164in}{0.818888in}}{\pgfqpoint{0.773164in}{0.791111in}}%
\pgfpathlineto{\pgfqpoint{0.773164in}{0.596666in}}%
\pgfpathquadraticcurveto{\pgfqpoint{0.773164in}{0.568889in}}{\pgfqpoint{0.800942in}{0.568889in}}%
\pgfpathlineto{\pgfqpoint{0.800942in}{0.568889in}}%
\pgfpathclose%
\pgfusepath{stroke,fill}%
\end{pgfscope}%
\begin{pgfscope}%
\pgfsetrectcap%
\pgfsetroundjoin%
\pgfsetlinewidth{1.505625pt}%
\definecolor{currentstroke}{rgb}{0.121569,0.466667,0.705882}%
\pgfsetstrokecolor{currentstroke}%
\pgfsetdash{}{0pt}%
\pgfpathmoveto{\pgfqpoint{0.828720in}{0.707777in}}%
\pgfpathlineto{\pgfqpoint{0.967608in}{0.707777in}}%
\pgfpathlineto{\pgfqpoint{1.106497in}{0.707777in}}%
\pgfusepath{stroke}%
\end{pgfscope}%
\begin{pgfscope}%
\definecolor{textcolor}{rgb}{0.000000,0.000000,0.000000}%
\pgfsetstrokecolor{textcolor}%
\pgfsetfillcolor{textcolor}%
\pgftext[x=1.217608in,y=0.659166in,left,base]{\color{textcolor}\rmfamily\fontsize{10.000000}{12.000000}\selectfont Area Under Curve = 0.847)}%
\end{pgfscope}%
\end{pgfpicture}%
\makeatother%
\endgroup%

\end{tabular}
\end{center}

\begin{center}
\begin{tabular}{cc}
\begin{tabular}{cc|c|c|}
	&\multicolumn{1}{c}{}& \multicolumn{2}{c}{Prediction} \cr
	&\multicolumn{1}{c}{} & \multicolumn{1}{c}{N} & \multicolumn{1}{c}{P} \cr\cline{3-4}
	\multirow{2}{*}{Actual}&N & 67.5\% & 18.2\% \vrule width 0pt height 10pt depth 2pt \cr\cline{3-4}
	&P & 3.11\% & 11.2\% \vrule width 0pt height 10pt depth 2pt \cr\cline{3-4}
\end{tabular}
&
\begin{tabular}{ll}
0.787 & Accuracy \cr 
0.785 & Balanced Accuracy \cr 
0.380 & Precision \cr 
0.786 & Balanced Precision \cr 
0.782 & Recall \cr 
0.512 & F1 \cr 
0.784 & Balanced F1 \cr 
0.547 & Gmean \cr 
	\end{tabular}
\end{tabular}
\end{center}




%%%%%
In the ideal results above, the algorithm learned a useful model from the patterns in the data.  The results below illustrate the worst case scenario, where the algorithm does not learn a good model, usually because the data does not have a pattern that predicts the target variable.  In the ROC curve, the median values of the probabilities for the two classes are so close that the labels are on top of each other.  

\begin{center}
\begin{tabular}{p{0.5\textwidth} p{0.5\textwidth}}
  \vspace{0pt} %% Creator: Matplotlib, PGF backend
%%
%% To include the figure in your LaTeX document, write
%%   \input{<filename>.pgf}
%%
%% Make sure the required packages are loaded in your preamble
%%   \usepackage{pgf}
%%
%% Also ensure that all the required font packages are loaded; for instance,
%% the lmodern package is sometimes necessary when using math font.
%%   \usepackage{lmodern}
%%
%% Figures using additional raster images can only be included by \input if
%% they are in the same directory as the main LaTeX file. For loading figures
%% from other directories you can use the `import` package
%%   \usepackage{import}
%%
%% and then include the figures with
%%   \import{<path to file>}{<filename>.pgf}
%%
%% Matplotlib used the following preamble
%%   
%%   \usepackage{fontspec}
%%   \makeatletter\@ifpackageloaded{underscore}{}{\usepackage[strings]{underscore}}\makeatother
%%
\begingroup%
\makeatletter%
\begin{pgfpicture}%
\pgfpathrectangle{\pgfpointorigin}{\pgfqpoint{2.153750in}{1.654444in}}%
\pgfusepath{use as bounding box, clip}%
\begin{pgfscope}%
\pgfsetbuttcap%
\pgfsetmiterjoin%
\definecolor{currentfill}{rgb}{1.000000,1.000000,1.000000}%
\pgfsetfillcolor{currentfill}%
\pgfsetlinewidth{0.000000pt}%
\definecolor{currentstroke}{rgb}{1.000000,1.000000,1.000000}%
\pgfsetstrokecolor{currentstroke}%
\pgfsetdash{}{0pt}%
\pgfpathmoveto{\pgfqpoint{0.000000in}{0.000000in}}%
\pgfpathlineto{\pgfqpoint{2.153750in}{0.000000in}}%
\pgfpathlineto{\pgfqpoint{2.153750in}{1.654444in}}%
\pgfpathlineto{\pgfqpoint{0.000000in}{1.654444in}}%
\pgfpathlineto{\pgfqpoint{0.000000in}{0.000000in}}%
\pgfpathclose%
\pgfusepath{fill}%
\end{pgfscope}%
\begin{pgfscope}%
\pgfsetbuttcap%
\pgfsetmiterjoin%
\definecolor{currentfill}{rgb}{1.000000,1.000000,1.000000}%
\pgfsetfillcolor{currentfill}%
\pgfsetlinewidth{0.000000pt}%
\definecolor{currentstroke}{rgb}{0.000000,0.000000,0.000000}%
\pgfsetstrokecolor{currentstroke}%
\pgfsetstrokeopacity{0.000000}%
\pgfsetdash{}{0pt}%
\pgfpathmoveto{\pgfqpoint{0.465000in}{0.449444in}}%
\pgfpathlineto{\pgfqpoint{2.015000in}{0.449444in}}%
\pgfpathlineto{\pgfqpoint{2.015000in}{1.604444in}}%
\pgfpathlineto{\pgfqpoint{0.465000in}{1.604444in}}%
\pgfpathlineto{\pgfqpoint{0.465000in}{0.449444in}}%
\pgfpathclose%
\pgfusepath{fill}%
\end{pgfscope}%
\begin{pgfscope}%
\pgfpathrectangle{\pgfqpoint{0.465000in}{0.449444in}}{\pgfqpoint{1.550000in}{1.155000in}}%
\pgfusepath{clip}%
\pgfsetbuttcap%
\pgfsetmiterjoin%
\pgfsetlinewidth{1.003750pt}%
\definecolor{currentstroke}{rgb}{0.000000,0.000000,0.000000}%
\pgfsetstrokecolor{currentstroke}%
\pgfsetdash{}{0pt}%
\pgfpathmoveto{\pgfqpoint{0.455000in}{0.449444in}}%
\pgfpathlineto{\pgfqpoint{0.502805in}{0.449444in}}%
\pgfpathlineto{\pgfqpoint{0.502805in}{1.418751in}}%
\pgfpathlineto{\pgfqpoint{0.455000in}{1.418751in}}%
\pgfusepath{stroke}%
\end{pgfscope}%
\begin{pgfscope}%
\pgfpathrectangle{\pgfqpoint{0.465000in}{0.449444in}}{\pgfqpoint{1.550000in}{1.155000in}}%
\pgfusepath{clip}%
\pgfsetbuttcap%
\pgfsetmiterjoin%
\pgfsetlinewidth{1.003750pt}%
\definecolor{currentstroke}{rgb}{0.000000,0.000000,0.000000}%
\pgfsetstrokecolor{currentstroke}%
\pgfsetdash{}{0pt}%
\pgfpathmoveto{\pgfqpoint{0.593537in}{0.449444in}}%
\pgfpathlineto{\pgfqpoint{0.654025in}{0.449444in}}%
\pgfpathlineto{\pgfqpoint{0.654025in}{1.549444in}}%
\pgfpathlineto{\pgfqpoint{0.593537in}{1.549444in}}%
\pgfpathlineto{\pgfqpoint{0.593537in}{0.449444in}}%
\pgfpathclose%
\pgfusepath{stroke}%
\end{pgfscope}%
\begin{pgfscope}%
\pgfpathrectangle{\pgfqpoint{0.465000in}{0.449444in}}{\pgfqpoint{1.550000in}{1.155000in}}%
\pgfusepath{clip}%
\pgfsetbuttcap%
\pgfsetmiterjoin%
\pgfsetlinewidth{1.003750pt}%
\definecolor{currentstroke}{rgb}{0.000000,0.000000,0.000000}%
\pgfsetstrokecolor{currentstroke}%
\pgfsetdash{}{0pt}%
\pgfpathmoveto{\pgfqpoint{0.744756in}{0.449444in}}%
\pgfpathlineto{\pgfqpoint{0.805244in}{0.449444in}}%
\pgfpathlineto{\pgfqpoint{0.805244in}{1.200929in}}%
\pgfpathlineto{\pgfqpoint{0.744756in}{1.200929in}}%
\pgfpathlineto{\pgfqpoint{0.744756in}{0.449444in}}%
\pgfpathclose%
\pgfusepath{stroke}%
\end{pgfscope}%
\begin{pgfscope}%
\pgfpathrectangle{\pgfqpoint{0.465000in}{0.449444in}}{\pgfqpoint{1.550000in}{1.155000in}}%
\pgfusepath{clip}%
\pgfsetbuttcap%
\pgfsetmiterjoin%
\pgfsetlinewidth{1.003750pt}%
\definecolor{currentstroke}{rgb}{0.000000,0.000000,0.000000}%
\pgfsetstrokecolor{currentstroke}%
\pgfsetdash{}{0pt}%
\pgfpathmoveto{\pgfqpoint{0.895976in}{0.449444in}}%
\pgfpathlineto{\pgfqpoint{0.956464in}{0.449444in}}%
\pgfpathlineto{\pgfqpoint{0.956464in}{1.462315in}}%
\pgfpathlineto{\pgfqpoint{0.895976in}{1.462315in}}%
\pgfpathlineto{\pgfqpoint{0.895976in}{0.449444in}}%
\pgfpathclose%
\pgfusepath{stroke}%
\end{pgfscope}%
\begin{pgfscope}%
\pgfpathrectangle{\pgfqpoint{0.465000in}{0.449444in}}{\pgfqpoint{1.550000in}{1.155000in}}%
\pgfusepath{clip}%
\pgfsetbuttcap%
\pgfsetmiterjoin%
\pgfsetlinewidth{1.003750pt}%
\definecolor{currentstroke}{rgb}{0.000000,0.000000,0.000000}%
\pgfsetstrokecolor{currentstroke}%
\pgfsetdash{}{0pt}%
\pgfpathmoveto{\pgfqpoint{1.047195in}{0.449444in}}%
\pgfpathlineto{\pgfqpoint{1.107683in}{0.449444in}}%
\pgfpathlineto{\pgfqpoint{1.107683in}{1.168256in}}%
\pgfpathlineto{\pgfqpoint{1.047195in}{1.168256in}}%
\pgfpathlineto{\pgfqpoint{1.047195in}{0.449444in}}%
\pgfpathclose%
\pgfusepath{stroke}%
\end{pgfscope}%
\begin{pgfscope}%
\pgfpathrectangle{\pgfqpoint{0.465000in}{0.449444in}}{\pgfqpoint{1.550000in}{1.155000in}}%
\pgfusepath{clip}%
\pgfsetbuttcap%
\pgfsetmiterjoin%
\pgfsetlinewidth{1.003750pt}%
\definecolor{currentstroke}{rgb}{0.000000,0.000000,0.000000}%
\pgfsetstrokecolor{currentstroke}%
\pgfsetdash{}{0pt}%
\pgfpathmoveto{\pgfqpoint{1.198415in}{0.449444in}}%
\pgfpathlineto{\pgfqpoint{1.258903in}{0.449444in}}%
\pgfpathlineto{\pgfqpoint{1.258903in}{1.473207in}}%
\pgfpathlineto{\pgfqpoint{1.198415in}{1.473207in}}%
\pgfpathlineto{\pgfqpoint{1.198415in}{0.449444in}}%
\pgfpathclose%
\pgfusepath{stroke}%
\end{pgfscope}%
\begin{pgfscope}%
\pgfpathrectangle{\pgfqpoint{0.465000in}{0.449444in}}{\pgfqpoint{1.550000in}{1.155000in}}%
\pgfusepath{clip}%
\pgfsetbuttcap%
\pgfsetmiterjoin%
\pgfsetlinewidth{1.003750pt}%
\definecolor{currentstroke}{rgb}{0.000000,0.000000,0.000000}%
\pgfsetstrokecolor{currentstroke}%
\pgfsetdash{}{0pt}%
\pgfpathmoveto{\pgfqpoint{1.349634in}{0.449444in}}%
\pgfpathlineto{\pgfqpoint{1.410122in}{0.449444in}}%
\pgfpathlineto{\pgfqpoint{1.410122in}{1.375187in}}%
\pgfpathlineto{\pgfqpoint{1.349634in}{1.375187in}}%
\pgfpathlineto{\pgfqpoint{1.349634in}{0.449444in}}%
\pgfpathclose%
\pgfusepath{stroke}%
\end{pgfscope}%
\begin{pgfscope}%
\pgfpathrectangle{\pgfqpoint{0.465000in}{0.449444in}}{\pgfqpoint{1.550000in}{1.155000in}}%
\pgfusepath{clip}%
\pgfsetbuttcap%
\pgfsetmiterjoin%
\pgfsetlinewidth{1.003750pt}%
\definecolor{currentstroke}{rgb}{0.000000,0.000000,0.000000}%
\pgfsetstrokecolor{currentstroke}%
\pgfsetdash{}{0pt}%
\pgfpathmoveto{\pgfqpoint{1.500854in}{0.449444in}}%
\pgfpathlineto{\pgfqpoint{1.561342in}{0.449444in}}%
\pgfpathlineto{\pgfqpoint{1.561342in}{1.277167in}}%
\pgfpathlineto{\pgfqpoint{1.500854in}{1.277167in}}%
\pgfpathlineto{\pgfqpoint{1.500854in}{0.449444in}}%
\pgfpathclose%
\pgfusepath{stroke}%
\end{pgfscope}%
\begin{pgfscope}%
\pgfpathrectangle{\pgfqpoint{0.465000in}{0.449444in}}{\pgfqpoint{1.550000in}{1.155000in}}%
\pgfusepath{clip}%
\pgfsetbuttcap%
\pgfsetmiterjoin%
\pgfsetlinewidth{1.003750pt}%
\definecolor{currentstroke}{rgb}{0.000000,0.000000,0.000000}%
\pgfsetstrokecolor{currentstroke}%
\pgfsetdash{}{0pt}%
\pgfpathmoveto{\pgfqpoint{1.652073in}{0.449444in}}%
\pgfpathlineto{\pgfqpoint{1.712561in}{0.449444in}}%
\pgfpathlineto{\pgfqpoint{1.712561in}{1.386078in}}%
\pgfpathlineto{\pgfqpoint{1.652073in}{1.386078in}}%
\pgfpathlineto{\pgfqpoint{1.652073in}{0.449444in}}%
\pgfpathclose%
\pgfusepath{stroke}%
\end{pgfscope}%
\begin{pgfscope}%
\pgfpathrectangle{\pgfqpoint{0.465000in}{0.449444in}}{\pgfqpoint{1.550000in}{1.155000in}}%
\pgfusepath{clip}%
\pgfsetbuttcap%
\pgfsetmiterjoin%
\pgfsetlinewidth{1.003750pt}%
\definecolor{currentstroke}{rgb}{0.000000,0.000000,0.000000}%
\pgfsetstrokecolor{currentstroke}%
\pgfsetdash{}{0pt}%
\pgfpathmoveto{\pgfqpoint{1.803293in}{0.449444in}}%
\pgfpathlineto{\pgfqpoint{1.863781in}{0.449444in}}%
\pgfpathlineto{\pgfqpoint{1.863781in}{1.440533in}}%
\pgfpathlineto{\pgfqpoint{1.803293in}{1.440533in}}%
\pgfpathlineto{\pgfqpoint{1.803293in}{0.449444in}}%
\pgfpathclose%
\pgfusepath{stroke}%
\end{pgfscope}%
\begin{pgfscope}%
\pgfpathrectangle{\pgfqpoint{0.465000in}{0.449444in}}{\pgfqpoint{1.550000in}{1.155000in}}%
\pgfusepath{clip}%
\pgfsetbuttcap%
\pgfsetmiterjoin%
\definecolor{currentfill}{rgb}{0.000000,0.000000,0.000000}%
\pgfsetfillcolor{currentfill}%
\pgfsetlinewidth{0.000000pt}%
\definecolor{currentstroke}{rgb}{0.000000,0.000000,0.000000}%
\pgfsetstrokecolor{currentstroke}%
\pgfsetstrokeopacity{0.000000}%
\pgfsetdash{}{0pt}%
\pgfpathmoveto{\pgfqpoint{0.502805in}{0.449444in}}%
\pgfpathlineto{\pgfqpoint{0.563293in}{0.449444in}}%
\pgfpathlineto{\pgfqpoint{0.563293in}{0.547464in}}%
\pgfpathlineto{\pgfqpoint{0.502805in}{0.547464in}}%
\pgfpathlineto{\pgfqpoint{0.502805in}{0.449444in}}%
\pgfpathclose%
\pgfusepath{fill}%
\end{pgfscope}%
\begin{pgfscope}%
\pgfpathrectangle{\pgfqpoint{0.465000in}{0.449444in}}{\pgfqpoint{1.550000in}{1.155000in}}%
\pgfusepath{clip}%
\pgfsetbuttcap%
\pgfsetmiterjoin%
\definecolor{currentfill}{rgb}{0.000000,0.000000,0.000000}%
\pgfsetfillcolor{currentfill}%
\pgfsetlinewidth{0.000000pt}%
\definecolor{currentstroke}{rgb}{0.000000,0.000000,0.000000}%
\pgfsetstrokecolor{currentstroke}%
\pgfsetstrokeopacity{0.000000}%
\pgfsetdash{}{0pt}%
\pgfpathmoveto{\pgfqpoint{0.654025in}{0.449444in}}%
\pgfpathlineto{\pgfqpoint{0.714512in}{0.449444in}}%
\pgfpathlineto{\pgfqpoint{0.714512in}{0.591028in}}%
\pgfpathlineto{\pgfqpoint{0.654025in}{0.591028in}}%
\pgfpathlineto{\pgfqpoint{0.654025in}{0.449444in}}%
\pgfpathclose%
\pgfusepath{fill}%
\end{pgfscope}%
\begin{pgfscope}%
\pgfpathrectangle{\pgfqpoint{0.465000in}{0.449444in}}{\pgfqpoint{1.550000in}{1.155000in}}%
\pgfusepath{clip}%
\pgfsetbuttcap%
\pgfsetmiterjoin%
\definecolor{currentfill}{rgb}{0.000000,0.000000,0.000000}%
\pgfsetfillcolor{currentfill}%
\pgfsetlinewidth{0.000000pt}%
\definecolor{currentstroke}{rgb}{0.000000,0.000000,0.000000}%
\pgfsetstrokecolor{currentstroke}%
\pgfsetstrokeopacity{0.000000}%
\pgfsetdash{}{0pt}%
\pgfpathmoveto{\pgfqpoint{0.805244in}{0.449444in}}%
\pgfpathlineto{\pgfqpoint{0.865732in}{0.449444in}}%
\pgfpathlineto{\pgfqpoint{0.865732in}{0.591028in}}%
\pgfpathlineto{\pgfqpoint{0.805244in}{0.591028in}}%
\pgfpathlineto{\pgfqpoint{0.805244in}{0.449444in}}%
\pgfpathclose%
\pgfusepath{fill}%
\end{pgfscope}%
\begin{pgfscope}%
\pgfpathrectangle{\pgfqpoint{0.465000in}{0.449444in}}{\pgfqpoint{1.550000in}{1.155000in}}%
\pgfusepath{clip}%
\pgfsetbuttcap%
\pgfsetmiterjoin%
\definecolor{currentfill}{rgb}{0.000000,0.000000,0.000000}%
\pgfsetfillcolor{currentfill}%
\pgfsetlinewidth{0.000000pt}%
\definecolor{currentstroke}{rgb}{0.000000,0.000000,0.000000}%
\pgfsetstrokecolor{currentstroke}%
\pgfsetstrokeopacity{0.000000}%
\pgfsetdash{}{0pt}%
\pgfpathmoveto{\pgfqpoint{0.956464in}{0.449444in}}%
\pgfpathlineto{\pgfqpoint{1.016951in}{0.449444in}}%
\pgfpathlineto{\pgfqpoint{1.016951in}{0.601919in}}%
\pgfpathlineto{\pgfqpoint{0.956464in}{0.601919in}}%
\pgfpathlineto{\pgfqpoint{0.956464in}{0.449444in}}%
\pgfpathclose%
\pgfusepath{fill}%
\end{pgfscope}%
\begin{pgfscope}%
\pgfpathrectangle{\pgfqpoint{0.465000in}{0.449444in}}{\pgfqpoint{1.550000in}{1.155000in}}%
\pgfusepath{clip}%
\pgfsetbuttcap%
\pgfsetmiterjoin%
\definecolor{currentfill}{rgb}{0.000000,0.000000,0.000000}%
\pgfsetfillcolor{currentfill}%
\pgfsetlinewidth{0.000000pt}%
\definecolor{currentstroke}{rgb}{0.000000,0.000000,0.000000}%
\pgfsetstrokecolor{currentstroke}%
\pgfsetstrokeopacity{0.000000}%
\pgfsetdash{}{0pt}%
\pgfpathmoveto{\pgfqpoint{1.107683in}{0.449444in}}%
\pgfpathlineto{\pgfqpoint{1.168171in}{0.449444in}}%
\pgfpathlineto{\pgfqpoint{1.168171in}{0.623702in}}%
\pgfpathlineto{\pgfqpoint{1.107683in}{0.623702in}}%
\pgfpathlineto{\pgfqpoint{1.107683in}{0.449444in}}%
\pgfpathclose%
\pgfusepath{fill}%
\end{pgfscope}%
\begin{pgfscope}%
\pgfpathrectangle{\pgfqpoint{0.465000in}{0.449444in}}{\pgfqpoint{1.550000in}{1.155000in}}%
\pgfusepath{clip}%
\pgfsetbuttcap%
\pgfsetmiterjoin%
\definecolor{currentfill}{rgb}{0.000000,0.000000,0.000000}%
\pgfsetfillcolor{currentfill}%
\pgfsetlinewidth{0.000000pt}%
\definecolor{currentstroke}{rgb}{0.000000,0.000000,0.000000}%
\pgfsetstrokecolor{currentstroke}%
\pgfsetstrokeopacity{0.000000}%
\pgfsetdash{}{0pt}%
\pgfpathmoveto{\pgfqpoint{1.258903in}{0.449444in}}%
\pgfpathlineto{\pgfqpoint{1.319391in}{0.449444in}}%
\pgfpathlineto{\pgfqpoint{1.319391in}{0.623702in}}%
\pgfpathlineto{\pgfqpoint{1.258903in}{0.623702in}}%
\pgfpathlineto{\pgfqpoint{1.258903in}{0.449444in}}%
\pgfpathclose%
\pgfusepath{fill}%
\end{pgfscope}%
\begin{pgfscope}%
\pgfpathrectangle{\pgfqpoint{0.465000in}{0.449444in}}{\pgfqpoint{1.550000in}{1.155000in}}%
\pgfusepath{clip}%
\pgfsetbuttcap%
\pgfsetmiterjoin%
\definecolor{currentfill}{rgb}{0.000000,0.000000,0.000000}%
\pgfsetfillcolor{currentfill}%
\pgfsetlinewidth{0.000000pt}%
\definecolor{currentstroke}{rgb}{0.000000,0.000000,0.000000}%
\pgfsetstrokecolor{currentstroke}%
\pgfsetstrokeopacity{0.000000}%
\pgfsetdash{}{0pt}%
\pgfpathmoveto{\pgfqpoint{1.410122in}{0.449444in}}%
\pgfpathlineto{\pgfqpoint{1.470610in}{0.449444in}}%
\pgfpathlineto{\pgfqpoint{1.470610in}{0.547464in}}%
\pgfpathlineto{\pgfqpoint{1.410122in}{0.547464in}}%
\pgfpathlineto{\pgfqpoint{1.410122in}{0.449444in}}%
\pgfpathclose%
\pgfusepath{fill}%
\end{pgfscope}%
\begin{pgfscope}%
\pgfpathrectangle{\pgfqpoint{0.465000in}{0.449444in}}{\pgfqpoint{1.550000in}{1.155000in}}%
\pgfusepath{clip}%
\pgfsetbuttcap%
\pgfsetmiterjoin%
\definecolor{currentfill}{rgb}{0.000000,0.000000,0.000000}%
\pgfsetfillcolor{currentfill}%
\pgfsetlinewidth{0.000000pt}%
\definecolor{currentstroke}{rgb}{0.000000,0.000000,0.000000}%
\pgfsetstrokecolor{currentstroke}%
\pgfsetstrokeopacity{0.000000}%
\pgfsetdash{}{0pt}%
\pgfpathmoveto{\pgfqpoint{1.561342in}{0.449444in}}%
\pgfpathlineto{\pgfqpoint{1.621830in}{0.449444in}}%
\pgfpathlineto{\pgfqpoint{1.621830in}{0.721721in}}%
\pgfpathlineto{\pgfqpoint{1.561342in}{0.721721in}}%
\pgfpathlineto{\pgfqpoint{1.561342in}{0.449444in}}%
\pgfpathclose%
\pgfusepath{fill}%
\end{pgfscope}%
\begin{pgfscope}%
\pgfpathrectangle{\pgfqpoint{0.465000in}{0.449444in}}{\pgfqpoint{1.550000in}{1.155000in}}%
\pgfusepath{clip}%
\pgfsetbuttcap%
\pgfsetmiterjoin%
\definecolor{currentfill}{rgb}{0.000000,0.000000,0.000000}%
\pgfsetfillcolor{currentfill}%
\pgfsetlinewidth{0.000000pt}%
\definecolor{currentstroke}{rgb}{0.000000,0.000000,0.000000}%
\pgfsetstrokecolor{currentstroke}%
\pgfsetstrokeopacity{0.000000}%
\pgfsetdash{}{0pt}%
\pgfpathmoveto{\pgfqpoint{1.712561in}{0.449444in}}%
\pgfpathlineto{\pgfqpoint{1.773049in}{0.449444in}}%
\pgfpathlineto{\pgfqpoint{1.773049in}{0.623702in}}%
\pgfpathlineto{\pgfqpoint{1.712561in}{0.623702in}}%
\pgfpathlineto{\pgfqpoint{1.712561in}{0.449444in}}%
\pgfpathclose%
\pgfusepath{fill}%
\end{pgfscope}%
\begin{pgfscope}%
\pgfpathrectangle{\pgfqpoint{0.465000in}{0.449444in}}{\pgfqpoint{1.550000in}{1.155000in}}%
\pgfusepath{clip}%
\pgfsetbuttcap%
\pgfsetmiterjoin%
\definecolor{currentfill}{rgb}{0.000000,0.000000,0.000000}%
\pgfsetfillcolor{currentfill}%
\pgfsetlinewidth{0.000000pt}%
\definecolor{currentstroke}{rgb}{0.000000,0.000000,0.000000}%
\pgfsetstrokecolor{currentstroke}%
\pgfsetstrokeopacity{0.000000}%
\pgfsetdash{}{0pt}%
\pgfpathmoveto{\pgfqpoint{1.863781in}{0.449444in}}%
\pgfpathlineto{\pgfqpoint{1.924269in}{0.449444in}}%
\pgfpathlineto{\pgfqpoint{1.924269in}{0.656375in}}%
\pgfpathlineto{\pgfqpoint{1.863781in}{0.656375in}}%
\pgfpathlineto{\pgfqpoint{1.863781in}{0.449444in}}%
\pgfpathclose%
\pgfusepath{fill}%
\end{pgfscope}%
\begin{pgfscope}%
\pgfsetbuttcap%
\pgfsetroundjoin%
\definecolor{currentfill}{rgb}{0.000000,0.000000,0.000000}%
\pgfsetfillcolor{currentfill}%
\pgfsetlinewidth{0.803000pt}%
\definecolor{currentstroke}{rgb}{0.000000,0.000000,0.000000}%
\pgfsetstrokecolor{currentstroke}%
\pgfsetdash{}{0pt}%
\pgfsys@defobject{currentmarker}{\pgfqpoint{0.000000in}{-0.048611in}}{\pgfqpoint{0.000000in}{0.000000in}}{%
\pgfpathmoveto{\pgfqpoint{0.000000in}{0.000000in}}%
\pgfpathlineto{\pgfqpoint{0.000000in}{-0.048611in}}%
\pgfusepath{stroke,fill}%
}%
\begin{pgfscope}%
\pgfsys@transformshift{0.502805in}{0.449444in}%
\pgfsys@useobject{currentmarker}{}%
\end{pgfscope}%
\end{pgfscope}%
\begin{pgfscope}%
\definecolor{textcolor}{rgb}{0.000000,0.000000,0.000000}%
\pgfsetstrokecolor{textcolor}%
\pgfsetfillcolor{textcolor}%
\pgftext[x=0.502805in,y=0.352222in,,top]{\color{textcolor}\rmfamily\fontsize{10.000000}{12.000000}\selectfont 0.0}%
\end{pgfscope}%
\begin{pgfscope}%
\pgfsetbuttcap%
\pgfsetroundjoin%
\definecolor{currentfill}{rgb}{0.000000,0.000000,0.000000}%
\pgfsetfillcolor{currentfill}%
\pgfsetlinewidth{0.803000pt}%
\definecolor{currentstroke}{rgb}{0.000000,0.000000,0.000000}%
\pgfsetstrokecolor{currentstroke}%
\pgfsetdash{}{0pt}%
\pgfsys@defobject{currentmarker}{\pgfqpoint{0.000000in}{-0.048611in}}{\pgfqpoint{0.000000in}{0.000000in}}{%
\pgfpathmoveto{\pgfqpoint{0.000000in}{0.000000in}}%
\pgfpathlineto{\pgfqpoint{0.000000in}{-0.048611in}}%
\pgfusepath{stroke,fill}%
}%
\begin{pgfscope}%
\pgfsys@transformshift{0.880854in}{0.449444in}%
\pgfsys@useobject{currentmarker}{}%
\end{pgfscope}%
\end{pgfscope}%
\begin{pgfscope}%
\definecolor{textcolor}{rgb}{0.000000,0.000000,0.000000}%
\pgfsetstrokecolor{textcolor}%
\pgfsetfillcolor{textcolor}%
\pgftext[x=0.880854in,y=0.352222in,,top]{\color{textcolor}\rmfamily\fontsize{10.000000}{12.000000}\selectfont 0.25}%
\end{pgfscope}%
\begin{pgfscope}%
\pgfsetbuttcap%
\pgfsetroundjoin%
\definecolor{currentfill}{rgb}{0.000000,0.000000,0.000000}%
\pgfsetfillcolor{currentfill}%
\pgfsetlinewidth{0.803000pt}%
\definecolor{currentstroke}{rgb}{0.000000,0.000000,0.000000}%
\pgfsetstrokecolor{currentstroke}%
\pgfsetdash{}{0pt}%
\pgfsys@defobject{currentmarker}{\pgfqpoint{0.000000in}{-0.048611in}}{\pgfqpoint{0.000000in}{0.000000in}}{%
\pgfpathmoveto{\pgfqpoint{0.000000in}{0.000000in}}%
\pgfpathlineto{\pgfqpoint{0.000000in}{-0.048611in}}%
\pgfusepath{stroke,fill}%
}%
\begin{pgfscope}%
\pgfsys@transformshift{1.258903in}{0.449444in}%
\pgfsys@useobject{currentmarker}{}%
\end{pgfscope}%
\end{pgfscope}%
\begin{pgfscope}%
\definecolor{textcolor}{rgb}{0.000000,0.000000,0.000000}%
\pgfsetstrokecolor{textcolor}%
\pgfsetfillcolor{textcolor}%
\pgftext[x=1.258903in,y=0.352222in,,top]{\color{textcolor}\rmfamily\fontsize{10.000000}{12.000000}\selectfont 0.5}%
\end{pgfscope}%
\begin{pgfscope}%
\pgfsetbuttcap%
\pgfsetroundjoin%
\definecolor{currentfill}{rgb}{0.000000,0.000000,0.000000}%
\pgfsetfillcolor{currentfill}%
\pgfsetlinewidth{0.803000pt}%
\definecolor{currentstroke}{rgb}{0.000000,0.000000,0.000000}%
\pgfsetstrokecolor{currentstroke}%
\pgfsetdash{}{0pt}%
\pgfsys@defobject{currentmarker}{\pgfqpoint{0.000000in}{-0.048611in}}{\pgfqpoint{0.000000in}{0.000000in}}{%
\pgfpathmoveto{\pgfqpoint{0.000000in}{0.000000in}}%
\pgfpathlineto{\pgfqpoint{0.000000in}{-0.048611in}}%
\pgfusepath{stroke,fill}%
}%
\begin{pgfscope}%
\pgfsys@transformshift{1.636951in}{0.449444in}%
\pgfsys@useobject{currentmarker}{}%
\end{pgfscope}%
\end{pgfscope}%
\begin{pgfscope}%
\definecolor{textcolor}{rgb}{0.000000,0.000000,0.000000}%
\pgfsetstrokecolor{textcolor}%
\pgfsetfillcolor{textcolor}%
\pgftext[x=1.636951in,y=0.352222in,,top]{\color{textcolor}\rmfamily\fontsize{10.000000}{12.000000}\selectfont 0.75}%
\end{pgfscope}%
\begin{pgfscope}%
\pgfsetbuttcap%
\pgfsetroundjoin%
\definecolor{currentfill}{rgb}{0.000000,0.000000,0.000000}%
\pgfsetfillcolor{currentfill}%
\pgfsetlinewidth{0.803000pt}%
\definecolor{currentstroke}{rgb}{0.000000,0.000000,0.000000}%
\pgfsetstrokecolor{currentstroke}%
\pgfsetdash{}{0pt}%
\pgfsys@defobject{currentmarker}{\pgfqpoint{0.000000in}{-0.048611in}}{\pgfqpoint{0.000000in}{0.000000in}}{%
\pgfpathmoveto{\pgfqpoint{0.000000in}{0.000000in}}%
\pgfpathlineto{\pgfqpoint{0.000000in}{-0.048611in}}%
\pgfusepath{stroke,fill}%
}%
\begin{pgfscope}%
\pgfsys@transformshift{2.015000in}{0.449444in}%
\pgfsys@useobject{currentmarker}{}%
\end{pgfscope}%
\end{pgfscope}%
\begin{pgfscope}%
\definecolor{textcolor}{rgb}{0.000000,0.000000,0.000000}%
\pgfsetstrokecolor{textcolor}%
\pgfsetfillcolor{textcolor}%
\pgftext[x=2.015000in,y=0.352222in,,top]{\color{textcolor}\rmfamily\fontsize{10.000000}{12.000000}\selectfont 1.0}%
\end{pgfscope}%
\begin{pgfscope}%
\definecolor{textcolor}{rgb}{0.000000,0.000000,0.000000}%
\pgfsetstrokecolor{textcolor}%
\pgfsetfillcolor{textcolor}%
\pgftext[x=1.240000in,y=0.173333in,,top]{\color{textcolor}\rmfamily\fontsize{10.000000}{12.000000}\selectfont \(\displaystyle p\)}%
\end{pgfscope}%
\begin{pgfscope}%
\pgfsetbuttcap%
\pgfsetroundjoin%
\definecolor{currentfill}{rgb}{0.000000,0.000000,0.000000}%
\pgfsetfillcolor{currentfill}%
\pgfsetlinewidth{0.803000pt}%
\definecolor{currentstroke}{rgb}{0.000000,0.000000,0.000000}%
\pgfsetstrokecolor{currentstroke}%
\pgfsetdash{}{0pt}%
\pgfsys@defobject{currentmarker}{\pgfqpoint{-0.048611in}{0.000000in}}{\pgfqpoint{-0.000000in}{0.000000in}}{%
\pgfpathmoveto{\pgfqpoint{-0.000000in}{0.000000in}}%
\pgfpathlineto{\pgfqpoint{-0.048611in}{0.000000in}}%
\pgfusepath{stroke,fill}%
}%
\begin{pgfscope}%
\pgfsys@transformshift{0.465000in}{0.449444in}%
\pgfsys@useobject{currentmarker}{}%
\end{pgfscope}%
\end{pgfscope}%
\begin{pgfscope}%
\definecolor{textcolor}{rgb}{0.000000,0.000000,0.000000}%
\pgfsetstrokecolor{textcolor}%
\pgfsetfillcolor{textcolor}%
\pgftext[x=0.298333in, y=0.401250in, left, base]{\color{textcolor}\rmfamily\fontsize{10.000000}{12.000000}\selectfont \(\displaystyle {0}\)}%
\end{pgfscope}%
\begin{pgfscope}%
\pgfsetbuttcap%
\pgfsetroundjoin%
\definecolor{currentfill}{rgb}{0.000000,0.000000,0.000000}%
\pgfsetfillcolor{currentfill}%
\pgfsetlinewidth{0.803000pt}%
\definecolor{currentstroke}{rgb}{0.000000,0.000000,0.000000}%
\pgfsetstrokecolor{currentstroke}%
\pgfsetdash{}{0pt}%
\pgfsys@defobject{currentmarker}{\pgfqpoint{-0.048611in}{0.000000in}}{\pgfqpoint{-0.000000in}{0.000000in}}{%
\pgfpathmoveto{\pgfqpoint{-0.000000in}{0.000000in}}%
\pgfpathlineto{\pgfqpoint{-0.048611in}{0.000000in}}%
\pgfusepath{stroke,fill}%
}%
\begin{pgfscope}%
\pgfsys@transformshift{0.465000in}{0.993999in}%
\pgfsys@useobject{currentmarker}{}%
\end{pgfscope}%
\end{pgfscope}%
\begin{pgfscope}%
\definecolor{textcolor}{rgb}{0.000000,0.000000,0.000000}%
\pgfsetstrokecolor{textcolor}%
\pgfsetfillcolor{textcolor}%
\pgftext[x=0.298333in, y=0.945804in, left, base]{\color{textcolor}\rmfamily\fontsize{10.000000}{12.000000}\selectfont \(\displaystyle {5}\)}%
\end{pgfscope}%
\begin{pgfscope}%
\pgfsetbuttcap%
\pgfsetroundjoin%
\definecolor{currentfill}{rgb}{0.000000,0.000000,0.000000}%
\pgfsetfillcolor{currentfill}%
\pgfsetlinewidth{0.803000pt}%
\definecolor{currentstroke}{rgb}{0.000000,0.000000,0.000000}%
\pgfsetstrokecolor{currentstroke}%
\pgfsetdash{}{0pt}%
\pgfsys@defobject{currentmarker}{\pgfqpoint{-0.048611in}{0.000000in}}{\pgfqpoint{-0.000000in}{0.000000in}}{%
\pgfpathmoveto{\pgfqpoint{-0.000000in}{0.000000in}}%
\pgfpathlineto{\pgfqpoint{-0.048611in}{0.000000in}}%
\pgfusepath{stroke,fill}%
}%
\begin{pgfscope}%
\pgfsys@transformshift{0.465000in}{1.538553in}%
\pgfsys@useobject{currentmarker}{}%
\end{pgfscope}%
\end{pgfscope}%
\begin{pgfscope}%
\definecolor{textcolor}{rgb}{0.000000,0.000000,0.000000}%
\pgfsetstrokecolor{textcolor}%
\pgfsetfillcolor{textcolor}%
\pgftext[x=0.228889in, y=1.490359in, left, base]{\color{textcolor}\rmfamily\fontsize{10.000000}{12.000000}\selectfont \(\displaystyle {10}\)}%
\end{pgfscope}%
\begin{pgfscope}%
\definecolor{textcolor}{rgb}{0.000000,0.000000,0.000000}%
\pgfsetstrokecolor{textcolor}%
\pgfsetfillcolor{textcolor}%
\pgftext[x=0.173333in,y=1.026944in,,bottom,rotate=90.000000]{\color{textcolor}\rmfamily\fontsize{10.000000}{12.000000}\selectfont Percent of Data Set}%
\end{pgfscope}%
\begin{pgfscope}%
\pgfsetrectcap%
\pgfsetmiterjoin%
\pgfsetlinewidth{0.803000pt}%
\definecolor{currentstroke}{rgb}{0.000000,0.000000,0.000000}%
\pgfsetstrokecolor{currentstroke}%
\pgfsetdash{}{0pt}%
\pgfpathmoveto{\pgfqpoint{0.465000in}{0.449444in}}%
\pgfpathlineto{\pgfqpoint{0.465000in}{1.604444in}}%
\pgfusepath{stroke}%
\end{pgfscope}%
\begin{pgfscope}%
\pgfsetrectcap%
\pgfsetmiterjoin%
\pgfsetlinewidth{0.803000pt}%
\definecolor{currentstroke}{rgb}{0.000000,0.000000,0.000000}%
\pgfsetstrokecolor{currentstroke}%
\pgfsetdash{}{0pt}%
\pgfpathmoveto{\pgfqpoint{2.015000in}{0.449444in}}%
\pgfpathlineto{\pgfqpoint{2.015000in}{1.604444in}}%
\pgfusepath{stroke}%
\end{pgfscope}%
\begin{pgfscope}%
\pgfsetrectcap%
\pgfsetmiterjoin%
\pgfsetlinewidth{0.803000pt}%
\definecolor{currentstroke}{rgb}{0.000000,0.000000,0.000000}%
\pgfsetstrokecolor{currentstroke}%
\pgfsetdash{}{0pt}%
\pgfpathmoveto{\pgfqpoint{0.465000in}{0.449444in}}%
\pgfpathlineto{\pgfqpoint{2.015000in}{0.449444in}}%
\pgfusepath{stroke}%
\end{pgfscope}%
\begin{pgfscope}%
\pgfsetrectcap%
\pgfsetmiterjoin%
\pgfsetlinewidth{0.803000pt}%
\definecolor{currentstroke}{rgb}{0.000000,0.000000,0.000000}%
\pgfsetstrokecolor{currentstroke}%
\pgfsetdash{}{0pt}%
\pgfpathmoveto{\pgfqpoint{0.465000in}{1.604444in}}%
\pgfpathlineto{\pgfqpoint{2.015000in}{1.604444in}}%
\pgfusepath{stroke}%
\end{pgfscope}%
\begin{pgfscope}%
\pgfsetbuttcap%
\pgfsetmiterjoin%
\definecolor{currentfill}{rgb}{1.000000,1.000000,1.000000}%
\pgfsetfillcolor{currentfill}%
\pgfsetfillopacity{0.800000}%
\pgfsetlinewidth{1.003750pt}%
\definecolor{currentstroke}{rgb}{0.800000,0.800000,0.800000}%
\pgfsetstrokecolor{currentstroke}%
\pgfsetstrokeopacity{0.800000}%
\pgfsetdash{}{0pt}%
\pgfpathmoveto{\pgfqpoint{1.238056in}{1.104445in}}%
\pgfpathlineto{\pgfqpoint{1.917778in}{1.104445in}}%
\pgfpathquadraticcurveto{\pgfqpoint{1.945556in}{1.104445in}}{\pgfqpoint{1.945556in}{1.132222in}}%
\pgfpathlineto{\pgfqpoint{1.945556in}{1.507222in}}%
\pgfpathquadraticcurveto{\pgfqpoint{1.945556in}{1.535000in}}{\pgfqpoint{1.917778in}{1.535000in}}%
\pgfpathlineto{\pgfqpoint{1.238056in}{1.535000in}}%
\pgfpathquadraticcurveto{\pgfqpoint{1.210278in}{1.535000in}}{\pgfqpoint{1.210278in}{1.507222in}}%
\pgfpathlineto{\pgfqpoint{1.210278in}{1.132222in}}%
\pgfpathquadraticcurveto{\pgfqpoint{1.210278in}{1.104445in}}{\pgfqpoint{1.238056in}{1.104445in}}%
\pgfpathlineto{\pgfqpoint{1.238056in}{1.104445in}}%
\pgfpathclose%
\pgfusepath{stroke,fill}%
\end{pgfscope}%
\begin{pgfscope}%
\pgfsetbuttcap%
\pgfsetmiterjoin%
\pgfsetlinewidth{1.003750pt}%
\definecolor{currentstroke}{rgb}{0.000000,0.000000,0.000000}%
\pgfsetstrokecolor{currentstroke}%
\pgfsetdash{}{0pt}%
\pgfpathmoveto{\pgfqpoint{1.265834in}{1.382222in}}%
\pgfpathlineto{\pgfqpoint{1.543611in}{1.382222in}}%
\pgfpathlineto{\pgfqpoint{1.543611in}{1.479444in}}%
\pgfpathlineto{\pgfqpoint{1.265834in}{1.479444in}}%
\pgfpathlineto{\pgfqpoint{1.265834in}{1.382222in}}%
\pgfpathclose%
\pgfusepath{stroke}%
\end{pgfscope}%
\begin{pgfscope}%
\definecolor{textcolor}{rgb}{0.000000,0.000000,0.000000}%
\pgfsetstrokecolor{textcolor}%
\pgfsetfillcolor{textcolor}%
\pgftext[x=1.654722in,y=1.382222in,left,base]{\color{textcolor}\rmfamily\fontsize{10.000000}{12.000000}\selectfont Neg}%
\end{pgfscope}%
\begin{pgfscope}%
\pgfsetbuttcap%
\pgfsetmiterjoin%
\definecolor{currentfill}{rgb}{0.000000,0.000000,0.000000}%
\pgfsetfillcolor{currentfill}%
\pgfsetlinewidth{0.000000pt}%
\definecolor{currentstroke}{rgb}{0.000000,0.000000,0.000000}%
\pgfsetstrokecolor{currentstroke}%
\pgfsetstrokeopacity{0.000000}%
\pgfsetdash{}{0pt}%
\pgfpathmoveto{\pgfqpoint{1.265834in}{1.186944in}}%
\pgfpathlineto{\pgfqpoint{1.543611in}{1.186944in}}%
\pgfpathlineto{\pgfqpoint{1.543611in}{1.284167in}}%
\pgfpathlineto{\pgfqpoint{1.265834in}{1.284167in}}%
\pgfpathlineto{\pgfqpoint{1.265834in}{1.186944in}}%
\pgfpathclose%
\pgfusepath{fill}%
\end{pgfscope}%
\begin{pgfscope}%
\definecolor{textcolor}{rgb}{0.000000,0.000000,0.000000}%
\pgfsetstrokecolor{textcolor}%
\pgfsetfillcolor{textcolor}%
\pgftext[x=1.654722in,y=1.186944in,left,base]{\color{textcolor}\rmfamily\fontsize{10.000000}{12.000000}\selectfont Pos}%
\end{pgfscope}%
\end{pgfpicture}%
\makeatother%
\endgroup%

  &
  \vspace{0pt} %% Creator: Matplotlib, PGF backend
%%
%% To include the figure in your LaTeX document, write
%%   \input{<filename>.pgf}
%%
%% Make sure the required packages are loaded in your preamble
%%   \usepackage{pgf}
%%
%% Also ensure that all the required font packages are loaded; for instance,
%% the lmodern package is sometimes necessary when using math font.
%%   \usepackage{lmodern}
%%
%% Figures using additional raster images can only be included by \input if
%% they are in the same directory as the main LaTeX file. For loading figures
%% from other directories you can use the `import` package
%%   \usepackage{import}
%%
%% and then include the figures with
%%   \import{<path to file>}{<filename>.pgf}
%%
%% Matplotlib used the following preamble
%%   
%%   \usepackage{fontspec}
%%   \makeatletter\@ifpackageloaded{underscore}{}{\usepackage[strings]{underscore}}\makeatother
%%
\begingroup%
\makeatletter%
\begin{pgfpicture}%
\pgfpathrectangle{\pgfpointorigin}{\pgfqpoint{2.121861in}{1.654444in}}%
\pgfusepath{use as bounding box, clip}%
\begin{pgfscope}%
\pgfsetbuttcap%
\pgfsetmiterjoin%
\definecolor{currentfill}{rgb}{1.000000,1.000000,1.000000}%
\pgfsetfillcolor{currentfill}%
\pgfsetlinewidth{0.000000pt}%
\definecolor{currentstroke}{rgb}{1.000000,1.000000,1.000000}%
\pgfsetstrokecolor{currentstroke}%
\pgfsetdash{}{0pt}%
\pgfpathmoveto{\pgfqpoint{0.000000in}{0.000000in}}%
\pgfpathlineto{\pgfqpoint{2.121861in}{0.000000in}}%
\pgfpathlineto{\pgfqpoint{2.121861in}{1.654444in}}%
\pgfpathlineto{\pgfqpoint{0.000000in}{1.654444in}}%
\pgfpathlineto{\pgfqpoint{0.000000in}{0.000000in}}%
\pgfpathclose%
\pgfusepath{fill}%
\end{pgfscope}%
\begin{pgfscope}%
\pgfsetbuttcap%
\pgfsetmiterjoin%
\definecolor{currentfill}{rgb}{1.000000,1.000000,1.000000}%
\pgfsetfillcolor{currentfill}%
\pgfsetlinewidth{0.000000pt}%
\definecolor{currentstroke}{rgb}{0.000000,0.000000,0.000000}%
\pgfsetstrokecolor{currentstroke}%
\pgfsetstrokeopacity{0.000000}%
\pgfsetdash{}{0pt}%
\pgfpathmoveto{\pgfqpoint{0.503581in}{0.449444in}}%
\pgfpathlineto{\pgfqpoint{2.053581in}{0.449444in}}%
\pgfpathlineto{\pgfqpoint{2.053581in}{1.604444in}}%
\pgfpathlineto{\pgfqpoint{0.503581in}{1.604444in}}%
\pgfpathlineto{\pgfqpoint{0.503581in}{0.449444in}}%
\pgfpathclose%
\pgfusepath{fill}%
\end{pgfscope}%
\begin{pgfscope}%
\pgfsetbuttcap%
\pgfsetroundjoin%
\definecolor{currentfill}{rgb}{0.000000,0.000000,0.000000}%
\pgfsetfillcolor{currentfill}%
\pgfsetlinewidth{0.803000pt}%
\definecolor{currentstroke}{rgb}{0.000000,0.000000,0.000000}%
\pgfsetstrokecolor{currentstroke}%
\pgfsetdash{}{0pt}%
\pgfsys@defobject{currentmarker}{\pgfqpoint{0.000000in}{-0.048611in}}{\pgfqpoint{0.000000in}{0.000000in}}{%
\pgfpathmoveto{\pgfqpoint{0.000000in}{0.000000in}}%
\pgfpathlineto{\pgfqpoint{0.000000in}{-0.048611in}}%
\pgfusepath{stroke,fill}%
}%
\begin{pgfscope}%
\pgfsys@transformshift{0.574035in}{0.449444in}%
\pgfsys@useobject{currentmarker}{}%
\end{pgfscope}%
\end{pgfscope}%
\begin{pgfscope}%
\definecolor{textcolor}{rgb}{0.000000,0.000000,0.000000}%
\pgfsetstrokecolor{textcolor}%
\pgfsetfillcolor{textcolor}%
\pgftext[x=0.574035in,y=0.352222in,,top]{\color{textcolor}\rmfamily\fontsize{10.000000}{12.000000}\selectfont \(\displaystyle {0.0}\)}%
\end{pgfscope}%
\begin{pgfscope}%
\pgfsetbuttcap%
\pgfsetroundjoin%
\definecolor{currentfill}{rgb}{0.000000,0.000000,0.000000}%
\pgfsetfillcolor{currentfill}%
\pgfsetlinewidth{0.803000pt}%
\definecolor{currentstroke}{rgb}{0.000000,0.000000,0.000000}%
\pgfsetstrokecolor{currentstroke}%
\pgfsetdash{}{0pt}%
\pgfsys@defobject{currentmarker}{\pgfqpoint{0.000000in}{-0.048611in}}{\pgfqpoint{0.000000in}{0.000000in}}{%
\pgfpathmoveto{\pgfqpoint{0.000000in}{0.000000in}}%
\pgfpathlineto{\pgfqpoint{0.000000in}{-0.048611in}}%
\pgfusepath{stroke,fill}%
}%
\begin{pgfscope}%
\pgfsys@transformshift{1.278581in}{0.449444in}%
\pgfsys@useobject{currentmarker}{}%
\end{pgfscope}%
\end{pgfscope}%
\begin{pgfscope}%
\definecolor{textcolor}{rgb}{0.000000,0.000000,0.000000}%
\pgfsetstrokecolor{textcolor}%
\pgfsetfillcolor{textcolor}%
\pgftext[x=1.278581in,y=0.352222in,,top]{\color{textcolor}\rmfamily\fontsize{10.000000}{12.000000}\selectfont \(\displaystyle {0.5}\)}%
\end{pgfscope}%
\begin{pgfscope}%
\pgfsetbuttcap%
\pgfsetroundjoin%
\definecolor{currentfill}{rgb}{0.000000,0.000000,0.000000}%
\pgfsetfillcolor{currentfill}%
\pgfsetlinewidth{0.803000pt}%
\definecolor{currentstroke}{rgb}{0.000000,0.000000,0.000000}%
\pgfsetstrokecolor{currentstroke}%
\pgfsetdash{}{0pt}%
\pgfsys@defobject{currentmarker}{\pgfqpoint{0.000000in}{-0.048611in}}{\pgfqpoint{0.000000in}{0.000000in}}{%
\pgfpathmoveto{\pgfqpoint{0.000000in}{0.000000in}}%
\pgfpathlineto{\pgfqpoint{0.000000in}{-0.048611in}}%
\pgfusepath{stroke,fill}%
}%
\begin{pgfscope}%
\pgfsys@transformshift{1.983126in}{0.449444in}%
\pgfsys@useobject{currentmarker}{}%
\end{pgfscope}%
\end{pgfscope}%
\begin{pgfscope}%
\definecolor{textcolor}{rgb}{0.000000,0.000000,0.000000}%
\pgfsetstrokecolor{textcolor}%
\pgfsetfillcolor{textcolor}%
\pgftext[x=1.983126in,y=0.352222in,,top]{\color{textcolor}\rmfamily\fontsize{10.000000}{12.000000}\selectfont \(\displaystyle {1.0}\)}%
\end{pgfscope}%
\begin{pgfscope}%
\definecolor{textcolor}{rgb}{0.000000,0.000000,0.000000}%
\pgfsetstrokecolor{textcolor}%
\pgfsetfillcolor{textcolor}%
\pgftext[x=1.278581in,y=0.173333in,,top]{\color{textcolor}\rmfamily\fontsize{10.000000}{12.000000}\selectfont False positive rate}%
\end{pgfscope}%
\begin{pgfscope}%
\pgfsetbuttcap%
\pgfsetroundjoin%
\definecolor{currentfill}{rgb}{0.000000,0.000000,0.000000}%
\pgfsetfillcolor{currentfill}%
\pgfsetlinewidth{0.803000pt}%
\definecolor{currentstroke}{rgb}{0.000000,0.000000,0.000000}%
\pgfsetstrokecolor{currentstroke}%
\pgfsetdash{}{0pt}%
\pgfsys@defobject{currentmarker}{\pgfqpoint{-0.048611in}{0.000000in}}{\pgfqpoint{-0.000000in}{0.000000in}}{%
\pgfpathmoveto{\pgfqpoint{-0.000000in}{0.000000in}}%
\pgfpathlineto{\pgfqpoint{-0.048611in}{0.000000in}}%
\pgfusepath{stroke,fill}%
}%
\begin{pgfscope}%
\pgfsys@transformshift{0.503581in}{0.501944in}%
\pgfsys@useobject{currentmarker}{}%
\end{pgfscope}%
\end{pgfscope}%
\begin{pgfscope}%
\definecolor{textcolor}{rgb}{0.000000,0.000000,0.000000}%
\pgfsetstrokecolor{textcolor}%
\pgfsetfillcolor{textcolor}%
\pgftext[x=0.228889in, y=0.453750in, left, base]{\color{textcolor}\rmfamily\fontsize{10.000000}{12.000000}\selectfont \(\displaystyle {0.0}\)}%
\end{pgfscope}%
\begin{pgfscope}%
\pgfsetbuttcap%
\pgfsetroundjoin%
\definecolor{currentfill}{rgb}{0.000000,0.000000,0.000000}%
\pgfsetfillcolor{currentfill}%
\pgfsetlinewidth{0.803000pt}%
\definecolor{currentstroke}{rgb}{0.000000,0.000000,0.000000}%
\pgfsetstrokecolor{currentstroke}%
\pgfsetdash{}{0pt}%
\pgfsys@defobject{currentmarker}{\pgfqpoint{-0.048611in}{0.000000in}}{\pgfqpoint{-0.000000in}{0.000000in}}{%
\pgfpathmoveto{\pgfqpoint{-0.000000in}{0.000000in}}%
\pgfpathlineto{\pgfqpoint{-0.048611in}{0.000000in}}%
\pgfusepath{stroke,fill}%
}%
\begin{pgfscope}%
\pgfsys@transformshift{0.503581in}{1.026944in}%
\pgfsys@useobject{currentmarker}{}%
\end{pgfscope}%
\end{pgfscope}%
\begin{pgfscope}%
\definecolor{textcolor}{rgb}{0.000000,0.000000,0.000000}%
\pgfsetstrokecolor{textcolor}%
\pgfsetfillcolor{textcolor}%
\pgftext[x=0.228889in, y=0.978750in, left, base]{\color{textcolor}\rmfamily\fontsize{10.000000}{12.000000}\selectfont \(\displaystyle {0.5}\)}%
\end{pgfscope}%
\begin{pgfscope}%
\pgfsetbuttcap%
\pgfsetroundjoin%
\definecolor{currentfill}{rgb}{0.000000,0.000000,0.000000}%
\pgfsetfillcolor{currentfill}%
\pgfsetlinewidth{0.803000pt}%
\definecolor{currentstroke}{rgb}{0.000000,0.000000,0.000000}%
\pgfsetstrokecolor{currentstroke}%
\pgfsetdash{}{0pt}%
\pgfsys@defobject{currentmarker}{\pgfqpoint{-0.048611in}{0.000000in}}{\pgfqpoint{-0.000000in}{0.000000in}}{%
\pgfpathmoveto{\pgfqpoint{-0.000000in}{0.000000in}}%
\pgfpathlineto{\pgfqpoint{-0.048611in}{0.000000in}}%
\pgfusepath{stroke,fill}%
}%
\begin{pgfscope}%
\pgfsys@transformshift{0.503581in}{1.551944in}%
\pgfsys@useobject{currentmarker}{}%
\end{pgfscope}%
\end{pgfscope}%
\begin{pgfscope}%
\definecolor{textcolor}{rgb}{0.000000,0.000000,0.000000}%
\pgfsetstrokecolor{textcolor}%
\pgfsetfillcolor{textcolor}%
\pgftext[x=0.228889in, y=1.503750in, left, base]{\color{textcolor}\rmfamily\fontsize{10.000000}{12.000000}\selectfont \(\displaystyle {1.0}\)}%
\end{pgfscope}%
\begin{pgfscope}%
\definecolor{textcolor}{rgb}{0.000000,0.000000,0.000000}%
\pgfsetstrokecolor{textcolor}%
\pgfsetfillcolor{textcolor}%
\pgftext[x=0.173333in,y=1.026944in,,bottom,rotate=90.000000]{\color{textcolor}\rmfamily\fontsize{10.000000}{12.000000}\selectfont True positive rate}%
\end{pgfscope}%
\begin{pgfscope}%
\pgfpathrectangle{\pgfqpoint{0.503581in}{0.449444in}}{\pgfqpoint{1.550000in}{1.155000in}}%
\pgfusepath{clip}%
\pgfsetbuttcap%
\pgfsetroundjoin%
\pgfsetlinewidth{1.505625pt}%
\definecolor{currentstroke}{rgb}{0.000000,0.000000,0.000000}%
\pgfsetstrokecolor{currentstroke}%
\pgfsetdash{{5.550000pt}{2.400000pt}}{0.000000pt}%
\pgfpathmoveto{\pgfqpoint{0.574035in}{0.501944in}}%
\pgfpathlineto{\pgfqpoint{1.983126in}{1.551944in}}%
\pgfusepath{stroke}%
\end{pgfscope}%
\begin{pgfscope}%
\pgfpathrectangle{\pgfqpoint{0.503581in}{0.449444in}}{\pgfqpoint{1.550000in}{1.155000in}}%
\pgfusepath{clip}%
\pgfsetrectcap%
\pgfsetroundjoin%
\pgfsetlinewidth{1.505625pt}%
\definecolor{currentstroke}{rgb}{0.000000,0.000000,0.000000}%
\pgfsetstrokecolor{currentstroke}%
\pgfsetdash{}{0pt}%
\pgfpathmoveto{\pgfqpoint{0.574035in}{0.501944in}}%
\pgfpathlineto{\pgfqpoint{0.585640in}{0.501944in}}%
\pgfpathlineto{\pgfqpoint{0.585640in}{0.508944in}}%
\pgfpathlineto{\pgfqpoint{0.587297in}{0.508944in}}%
\pgfpathlineto{\pgfqpoint{0.587297in}{0.515944in}}%
\pgfpathlineto{\pgfqpoint{0.600559in}{0.515944in}}%
\pgfpathlineto{\pgfqpoint{0.600559in}{0.529944in}}%
\pgfpathlineto{\pgfqpoint{0.602217in}{0.529944in}}%
\pgfpathlineto{\pgfqpoint{0.602217in}{0.536944in}}%
\pgfpathlineto{\pgfqpoint{0.603875in}{0.536944in}}%
\pgfpathlineto{\pgfqpoint{0.603875in}{0.543944in}}%
\pgfpathlineto{\pgfqpoint{0.605533in}{0.543944in}}%
\pgfpathlineto{\pgfqpoint{0.605533in}{0.550944in}}%
\pgfpathlineto{\pgfqpoint{0.607190in}{0.550944in}}%
\pgfpathlineto{\pgfqpoint{0.607190in}{0.557944in}}%
\pgfpathlineto{\pgfqpoint{0.618795in}{0.557944in}}%
\pgfpathlineto{\pgfqpoint{0.618795in}{0.564944in}}%
\pgfpathlineto{\pgfqpoint{0.625426in}{0.564944in}}%
\pgfpathlineto{\pgfqpoint{0.625426in}{0.571944in}}%
\pgfpathlineto{\pgfqpoint{0.640345in}{0.571944in}}%
\pgfpathlineto{\pgfqpoint{0.640345in}{0.578944in}}%
\pgfpathlineto{\pgfqpoint{0.651950in}{0.578944in}}%
\pgfpathlineto{\pgfqpoint{0.651950in}{0.585944in}}%
\pgfpathlineto{\pgfqpoint{0.673500in}{0.585944in}}%
\pgfpathlineto{\pgfqpoint{0.673500in}{0.599944in}}%
\pgfpathlineto{\pgfqpoint{0.688420in}{0.599944in}}%
\pgfpathlineto{\pgfqpoint{0.688420in}{0.606944in}}%
\pgfpathlineto{\pgfqpoint{0.698367in}{0.606944in}}%
\pgfpathlineto{\pgfqpoint{0.698367in}{0.613944in}}%
\pgfpathlineto{\pgfqpoint{0.701682in}{0.613944in}}%
\pgfpathlineto{\pgfqpoint{0.701682in}{0.620944in}}%
\pgfpathlineto{\pgfqpoint{0.708313in}{0.620944in}}%
\pgfpathlineto{\pgfqpoint{0.708313in}{0.627944in}}%
\pgfpathlineto{\pgfqpoint{0.723233in}{0.627944in}}%
\pgfpathlineto{\pgfqpoint{0.723233in}{0.634944in}}%
\pgfpathlineto{\pgfqpoint{0.736495in}{0.634944in}}%
\pgfpathlineto{\pgfqpoint{0.736495in}{0.641944in}}%
\pgfpathlineto{\pgfqpoint{0.743126in}{0.641944in}}%
\pgfpathlineto{\pgfqpoint{0.743126in}{0.648944in}}%
\pgfpathlineto{\pgfqpoint{0.758046in}{0.648944in}}%
\pgfpathlineto{\pgfqpoint{0.758046in}{0.655944in}}%
\pgfpathlineto{\pgfqpoint{0.784570in}{0.655944in}}%
\pgfpathlineto{\pgfqpoint{0.784570in}{0.669944in}}%
\pgfpathlineto{\pgfqpoint{0.789543in}{0.669944in}}%
\pgfpathlineto{\pgfqpoint{0.789543in}{0.683944in}}%
\pgfpathlineto{\pgfqpoint{0.806121in}{0.683944in}}%
\pgfpathlineto{\pgfqpoint{0.806121in}{0.697944in}}%
\pgfpathlineto{\pgfqpoint{0.816067in}{0.697944in}}%
\pgfpathlineto{\pgfqpoint{0.816067in}{0.704944in}}%
\pgfpathlineto{\pgfqpoint{0.821041in}{0.704944in}}%
\pgfpathlineto{\pgfqpoint{0.821041in}{0.711944in}}%
\pgfpathlineto{\pgfqpoint{0.822698in}{0.711944in}}%
\pgfpathlineto{\pgfqpoint{0.822698in}{0.718944in}}%
\pgfpathlineto{\pgfqpoint{0.827672in}{0.718944in}}%
\pgfpathlineto{\pgfqpoint{0.827672in}{0.725944in}}%
\pgfpathlineto{\pgfqpoint{0.829329in}{0.725944in}}%
\pgfpathlineto{\pgfqpoint{0.829329in}{0.739944in}}%
\pgfpathlineto{\pgfqpoint{0.852538in}{0.739944in}}%
\pgfpathlineto{\pgfqpoint{0.852538in}{0.746944in}}%
\pgfpathlineto{\pgfqpoint{0.877404in}{0.746944in}}%
\pgfpathlineto{\pgfqpoint{0.877404in}{0.760944in}}%
\pgfpathlineto{\pgfqpoint{0.880720in}{0.760944in}}%
\pgfpathlineto{\pgfqpoint{0.880720in}{0.774944in}}%
\pgfpathlineto{\pgfqpoint{0.882377in}{0.774944in}}%
\pgfpathlineto{\pgfqpoint{0.882377in}{0.781944in}}%
\pgfpathlineto{\pgfqpoint{0.889009in}{0.781944in}}%
\pgfpathlineto{\pgfqpoint{0.889009in}{0.788944in}}%
\pgfpathlineto{\pgfqpoint{0.892324in}{0.788944in}}%
\pgfpathlineto{\pgfqpoint{0.892324in}{0.795944in}}%
\pgfpathlineto{\pgfqpoint{0.908902in}{0.795944in}}%
\pgfpathlineto{\pgfqpoint{0.908902in}{0.802944in}}%
\pgfpathlineto{\pgfqpoint{0.910559in}{0.802944in}}%
\pgfpathlineto{\pgfqpoint{0.910559in}{0.809944in}}%
\pgfpathlineto{\pgfqpoint{0.918848in}{0.809944in}}%
\pgfpathlineto{\pgfqpoint{0.918848in}{0.816944in}}%
\pgfpathlineto{\pgfqpoint{0.922164in}{0.816944in}}%
\pgfpathlineto{\pgfqpoint{0.922164in}{0.823944in}}%
\pgfpathlineto{\pgfqpoint{0.923821in}{0.823944in}}%
\pgfpathlineto{\pgfqpoint{0.923821in}{0.830944in}}%
\pgfpathlineto{\pgfqpoint{0.942057in}{0.830944in}}%
\pgfpathlineto{\pgfqpoint{0.942057in}{0.837944in}}%
\pgfpathlineto{\pgfqpoint{0.945372in}{0.837944in}}%
\pgfpathlineto{\pgfqpoint{0.945372in}{0.844944in}}%
\pgfpathlineto{\pgfqpoint{0.947030in}{0.844944in}}%
\pgfpathlineto{\pgfqpoint{0.947030in}{0.851944in}}%
\pgfpathlineto{\pgfqpoint{0.953661in}{0.851944in}}%
\pgfpathlineto{\pgfqpoint{0.953661in}{0.865944in}}%
\pgfpathlineto{\pgfqpoint{0.956976in}{0.865944in}}%
\pgfpathlineto{\pgfqpoint{0.956976in}{0.872944in}}%
\pgfpathlineto{\pgfqpoint{0.960292in}{0.872944in}}%
\pgfpathlineto{\pgfqpoint{0.960292in}{0.886944in}}%
\pgfpathlineto{\pgfqpoint{0.963607in}{0.886944in}}%
\pgfpathlineto{\pgfqpoint{0.963607in}{0.893944in}}%
\pgfpathlineto{\pgfqpoint{0.976869in}{0.893944in}}%
\pgfpathlineto{\pgfqpoint{0.976869in}{0.900944in}}%
\pgfpathlineto{\pgfqpoint{0.981843in}{0.900944in}}%
\pgfpathlineto{\pgfqpoint{0.981843in}{0.907944in}}%
\pgfpathlineto{\pgfqpoint{0.983500in}{0.907944in}}%
\pgfpathlineto{\pgfqpoint{0.983500in}{0.914944in}}%
\pgfpathlineto{\pgfqpoint{0.988474in}{0.914944in}}%
\pgfpathlineto{\pgfqpoint{0.988474in}{0.921944in}}%
\pgfpathlineto{\pgfqpoint{1.031575in}{0.921944in}}%
\pgfpathlineto{\pgfqpoint{1.031575in}{0.928944in}}%
\pgfpathlineto{\pgfqpoint{1.043180in}{0.928944in}}%
\pgfpathlineto{\pgfqpoint{1.043180in}{0.935944in}}%
\pgfpathlineto{\pgfqpoint{1.056442in}{0.935944in}}%
\pgfpathlineto{\pgfqpoint{1.056442in}{0.942944in}}%
\pgfpathlineto{\pgfqpoint{1.064730in}{0.942944in}}%
\pgfpathlineto{\pgfqpoint{1.064730in}{0.949944in}}%
\pgfpathlineto{\pgfqpoint{1.092912in}{0.949944in}}%
\pgfpathlineto{\pgfqpoint{1.092912in}{0.956944in}}%
\pgfpathlineto{\pgfqpoint{1.102859in}{0.956944in}}%
\pgfpathlineto{\pgfqpoint{1.102859in}{0.963944in}}%
\pgfpathlineto{\pgfqpoint{1.111148in}{0.963944in}}%
\pgfpathlineto{\pgfqpoint{1.111148in}{0.970944in}}%
\pgfpathlineto{\pgfqpoint{1.121094in}{0.970944in}}%
\pgfpathlineto{\pgfqpoint{1.121094in}{0.977944in}}%
\pgfpathlineto{\pgfqpoint{1.132698in}{0.977944in}}%
\pgfpathlineto{\pgfqpoint{1.132698in}{0.984944in}}%
\pgfpathlineto{\pgfqpoint{1.134356in}{0.984944in}}%
\pgfpathlineto{\pgfqpoint{1.134356in}{0.991944in}}%
\pgfpathlineto{\pgfqpoint{1.147618in}{0.991944in}}%
\pgfpathlineto{\pgfqpoint{1.147618in}{0.998944in}}%
\pgfpathlineto{\pgfqpoint{1.150934in}{0.998944in}}%
\pgfpathlineto{\pgfqpoint{1.150934in}{1.005944in}}%
\pgfpathlineto{\pgfqpoint{1.165853in}{1.005944in}}%
\pgfpathlineto{\pgfqpoint{1.165853in}{1.012944in}}%
\pgfpathlineto{\pgfqpoint{1.167511in}{1.012944in}}%
\pgfpathlineto{\pgfqpoint{1.167511in}{1.019944in}}%
\pgfpathlineto{\pgfqpoint{1.174142in}{1.019944in}}%
\pgfpathlineto{\pgfqpoint{1.174142in}{1.026944in}}%
\pgfpathlineto{\pgfqpoint{1.185746in}{1.026944in}}%
\pgfpathlineto{\pgfqpoint{1.185746in}{1.033944in}}%
\pgfpathlineto{\pgfqpoint{1.187404in}{1.033944in}}%
\pgfpathlineto{\pgfqpoint{1.187404in}{1.040944in}}%
\pgfpathlineto{\pgfqpoint{1.190720in}{1.040944in}}%
\pgfpathlineto{\pgfqpoint{1.190720in}{1.047944in}}%
\pgfpathlineto{\pgfqpoint{1.200666in}{1.047944in}}%
\pgfpathlineto{\pgfqpoint{1.200666in}{1.054944in}}%
\pgfpathlineto{\pgfqpoint{1.208955in}{1.054944in}}%
\pgfpathlineto{\pgfqpoint{1.208955in}{1.061944in}}%
\pgfpathlineto{\pgfqpoint{1.238795in}{1.061944in}}%
\pgfpathlineto{\pgfqpoint{1.238795in}{1.068944in}}%
\pgfpathlineto{\pgfqpoint{1.243768in}{1.068944in}}%
\pgfpathlineto{\pgfqpoint{1.243768in}{1.075944in}}%
\pgfpathlineto{\pgfqpoint{1.265319in}{1.075944in}}%
\pgfpathlineto{\pgfqpoint{1.265319in}{1.082944in}}%
\pgfpathlineto{\pgfqpoint{1.278581in}{1.082944in}}%
\pgfpathlineto{\pgfqpoint{1.278581in}{1.096944in}}%
\pgfpathlineto{\pgfqpoint{1.313394in}{1.096944in}}%
\pgfpathlineto{\pgfqpoint{1.313394in}{1.117944in}}%
\pgfpathlineto{\pgfqpoint{1.336602in}{1.117944in}}%
\pgfpathlineto{\pgfqpoint{1.336602in}{1.124944in}}%
\pgfpathlineto{\pgfqpoint{1.351522in}{1.124944in}}%
\pgfpathlineto{\pgfqpoint{1.351522in}{1.131944in}}%
\pgfpathlineto{\pgfqpoint{1.353180in}{1.131944in}}%
\pgfpathlineto{\pgfqpoint{1.353180in}{1.138944in}}%
\pgfpathlineto{\pgfqpoint{1.354837in}{1.138944in}}%
\pgfpathlineto{\pgfqpoint{1.354837in}{1.145944in}}%
\pgfpathlineto{\pgfqpoint{1.361468in}{1.145944in}}%
\pgfpathlineto{\pgfqpoint{1.361468in}{1.152944in}}%
\pgfpathlineto{\pgfqpoint{1.363126in}{1.152944in}}%
\pgfpathlineto{\pgfqpoint{1.363126in}{1.159944in}}%
\pgfpathlineto{\pgfqpoint{1.378046in}{1.159944in}}%
\pgfpathlineto{\pgfqpoint{1.378046in}{1.166944in}}%
\pgfpathlineto{\pgfqpoint{1.384677in}{1.166944in}}%
\pgfpathlineto{\pgfqpoint{1.384677in}{1.173944in}}%
\pgfpathlineto{\pgfqpoint{1.394623in}{1.173944in}}%
\pgfpathlineto{\pgfqpoint{1.394623in}{1.187944in}}%
\pgfpathlineto{\pgfqpoint{1.396281in}{1.187944in}}%
\pgfpathlineto{\pgfqpoint{1.396281in}{1.194944in}}%
\pgfpathlineto{\pgfqpoint{1.397939in}{1.194944in}}%
\pgfpathlineto{\pgfqpoint{1.397939in}{1.208944in}}%
\pgfpathlineto{\pgfqpoint{1.399597in}{1.208944in}}%
\pgfpathlineto{\pgfqpoint{1.399597in}{1.215944in}}%
\pgfpathlineto{\pgfqpoint{1.407886in}{1.215944in}}%
\pgfpathlineto{\pgfqpoint{1.407886in}{1.222944in}}%
\pgfpathlineto{\pgfqpoint{1.434410in}{1.222944in}}%
\pgfpathlineto{\pgfqpoint{1.434410in}{1.229944in}}%
\pgfpathlineto{\pgfqpoint{1.454303in}{1.229944in}}%
\pgfpathlineto{\pgfqpoint{1.454303in}{1.236944in}}%
\pgfpathlineto{\pgfqpoint{1.474196in}{1.236944in}}%
\pgfpathlineto{\pgfqpoint{1.474196in}{1.243944in}}%
\pgfpathlineto{\pgfqpoint{1.484142in}{1.243944in}}%
\pgfpathlineto{\pgfqpoint{1.484142in}{1.250944in}}%
\pgfpathlineto{\pgfqpoint{1.485800in}{1.250944in}}%
\pgfpathlineto{\pgfqpoint{1.485800in}{1.257944in}}%
\pgfpathlineto{\pgfqpoint{1.494089in}{1.257944in}}%
\pgfpathlineto{\pgfqpoint{1.494089in}{1.264944in}}%
\pgfpathlineto{\pgfqpoint{1.504035in}{1.264944in}}%
\pgfpathlineto{\pgfqpoint{1.504035in}{1.271944in}}%
\pgfpathlineto{\pgfqpoint{1.505693in}{1.271944in}}%
\pgfpathlineto{\pgfqpoint{1.505693in}{1.278944in}}%
\pgfpathlineto{\pgfqpoint{1.509009in}{1.278944in}}%
\pgfpathlineto{\pgfqpoint{1.509009in}{1.285944in}}%
\pgfpathlineto{\pgfqpoint{1.513982in}{1.285944in}}%
\pgfpathlineto{\pgfqpoint{1.513982in}{1.292944in}}%
\pgfpathlineto{\pgfqpoint{1.543821in}{1.292944in}}%
\pgfpathlineto{\pgfqpoint{1.543821in}{1.299944in}}%
\pgfpathlineto{\pgfqpoint{1.547137in}{1.299944in}}%
\pgfpathlineto{\pgfqpoint{1.547137in}{1.306944in}}%
\pgfpathlineto{\pgfqpoint{1.572003in}{1.306944in}}%
\pgfpathlineto{\pgfqpoint{1.572003in}{1.313944in}}%
\pgfpathlineto{\pgfqpoint{1.578634in}{1.313944in}}%
\pgfpathlineto{\pgfqpoint{1.578634in}{1.320944in}}%
\pgfpathlineto{\pgfqpoint{1.588581in}{1.320944in}}%
\pgfpathlineto{\pgfqpoint{1.588581in}{1.327944in}}%
\pgfpathlineto{\pgfqpoint{1.593554in}{1.327944in}}%
\pgfpathlineto{\pgfqpoint{1.593554in}{1.334944in}}%
\pgfpathlineto{\pgfqpoint{1.598527in}{1.334944in}}%
\pgfpathlineto{\pgfqpoint{1.598527in}{1.341944in}}%
\pgfpathlineto{\pgfqpoint{1.600185in}{1.341944in}}%
\pgfpathlineto{\pgfqpoint{1.600185in}{1.348944in}}%
\pgfpathlineto{\pgfqpoint{1.615105in}{1.348944in}}%
\pgfpathlineto{\pgfqpoint{1.615105in}{1.355944in}}%
\pgfpathlineto{\pgfqpoint{1.639971in}{1.355944in}}%
\pgfpathlineto{\pgfqpoint{1.639971in}{1.362944in}}%
\pgfpathlineto{\pgfqpoint{1.643287in}{1.362944in}}%
\pgfpathlineto{\pgfqpoint{1.643287in}{1.369944in}}%
\pgfpathlineto{\pgfqpoint{1.651575in}{1.369944in}}%
\pgfpathlineto{\pgfqpoint{1.651575in}{1.376944in}}%
\pgfpathlineto{\pgfqpoint{1.668153in}{1.376944in}}%
\pgfpathlineto{\pgfqpoint{1.668153in}{1.397944in}}%
\pgfpathlineto{\pgfqpoint{1.691361in}{1.397944in}}%
\pgfpathlineto{\pgfqpoint{1.691361in}{1.404944in}}%
\pgfpathlineto{\pgfqpoint{1.697992in}{1.404944in}}%
\pgfpathlineto{\pgfqpoint{1.697992in}{1.411944in}}%
\pgfpathlineto{\pgfqpoint{1.717886in}{1.411944in}}%
\pgfpathlineto{\pgfqpoint{1.717886in}{1.425944in}}%
\pgfpathlineto{\pgfqpoint{1.721201in}{1.425944in}}%
\pgfpathlineto{\pgfqpoint{1.721201in}{1.432944in}}%
\pgfpathlineto{\pgfqpoint{1.757672in}{1.432944in}}%
\pgfpathlineto{\pgfqpoint{1.757672in}{1.439944in}}%
\pgfpathlineto{\pgfqpoint{1.770934in}{1.439944in}}%
\pgfpathlineto{\pgfqpoint{1.770934in}{1.446944in}}%
\pgfpathlineto{\pgfqpoint{1.777565in}{1.446944in}}%
\pgfpathlineto{\pgfqpoint{1.777565in}{1.453944in}}%
\pgfpathlineto{\pgfqpoint{1.787511in}{1.453944in}}%
\pgfpathlineto{\pgfqpoint{1.787511in}{1.460944in}}%
\pgfpathlineto{\pgfqpoint{1.825640in}{1.460944in}}%
\pgfpathlineto{\pgfqpoint{1.825640in}{1.467944in}}%
\pgfpathlineto{\pgfqpoint{1.827297in}{1.467944in}}%
\pgfpathlineto{\pgfqpoint{1.827297in}{1.481944in}}%
\pgfpathlineto{\pgfqpoint{1.828955in}{1.481944in}}%
\pgfpathlineto{\pgfqpoint{1.828955in}{1.488944in}}%
\pgfpathlineto{\pgfqpoint{1.835586in}{1.488944in}}%
\pgfpathlineto{\pgfqpoint{1.835586in}{1.495944in}}%
\pgfpathlineto{\pgfqpoint{1.857137in}{1.495944in}}%
\pgfpathlineto{\pgfqpoint{1.857137in}{1.509944in}}%
\pgfpathlineto{\pgfqpoint{1.867083in}{1.509944in}}%
\pgfpathlineto{\pgfqpoint{1.867083in}{1.516944in}}%
\pgfpathlineto{\pgfqpoint{1.896923in}{1.516944in}}%
\pgfpathlineto{\pgfqpoint{1.896923in}{1.523944in}}%
\pgfpathlineto{\pgfqpoint{1.906869in}{1.523944in}}%
\pgfpathlineto{\pgfqpoint{1.906869in}{1.530944in}}%
\pgfpathlineto{\pgfqpoint{1.910185in}{1.530944in}}%
\pgfpathlineto{\pgfqpoint{1.910185in}{1.537944in}}%
\pgfpathlineto{\pgfqpoint{1.911843in}{1.537944in}}%
\pgfpathlineto{\pgfqpoint{1.911843in}{1.544944in}}%
\pgfpathlineto{\pgfqpoint{1.963233in}{1.544944in}}%
\pgfpathlineto{\pgfqpoint{1.963233in}{1.551944in}}%
\pgfpathlineto{\pgfqpoint{1.983126in}{1.551944in}}%
\pgfpathlineto{\pgfqpoint{1.983126in}{1.551944in}}%
\pgfusepath{stroke}%
\end{pgfscope}%
\begin{pgfscope}%
\pgfsetrectcap%
\pgfsetmiterjoin%
\pgfsetlinewidth{0.803000pt}%
\definecolor{currentstroke}{rgb}{0.000000,0.000000,0.000000}%
\pgfsetstrokecolor{currentstroke}%
\pgfsetdash{}{0pt}%
\pgfpathmoveto{\pgfqpoint{0.503581in}{0.449444in}}%
\pgfpathlineto{\pgfqpoint{0.503581in}{1.604444in}}%
\pgfusepath{stroke}%
\end{pgfscope}%
\begin{pgfscope}%
\pgfsetrectcap%
\pgfsetmiterjoin%
\pgfsetlinewidth{0.803000pt}%
\definecolor{currentstroke}{rgb}{0.000000,0.000000,0.000000}%
\pgfsetstrokecolor{currentstroke}%
\pgfsetdash{}{0pt}%
\pgfpathmoveto{\pgfqpoint{2.053581in}{0.449444in}}%
\pgfpathlineto{\pgfqpoint{2.053581in}{1.604444in}}%
\pgfusepath{stroke}%
\end{pgfscope}%
\begin{pgfscope}%
\pgfsetrectcap%
\pgfsetmiterjoin%
\pgfsetlinewidth{0.803000pt}%
\definecolor{currentstroke}{rgb}{0.000000,0.000000,0.000000}%
\pgfsetstrokecolor{currentstroke}%
\pgfsetdash{}{0pt}%
\pgfpathmoveto{\pgfqpoint{0.503581in}{0.449444in}}%
\pgfpathlineto{\pgfqpoint{2.053581in}{0.449444in}}%
\pgfusepath{stroke}%
\end{pgfscope}%
\begin{pgfscope}%
\pgfsetrectcap%
\pgfsetmiterjoin%
\pgfsetlinewidth{0.803000pt}%
\definecolor{currentstroke}{rgb}{0.000000,0.000000,0.000000}%
\pgfsetstrokecolor{currentstroke}%
\pgfsetdash{}{0pt}%
\pgfpathmoveto{\pgfqpoint{0.503581in}{1.604444in}}%
\pgfpathlineto{\pgfqpoint{2.053581in}{1.604444in}}%
\pgfusepath{stroke}%
\end{pgfscope}%
\begin{pgfscope}%
\pgfsetbuttcap%
\pgfsetmiterjoin%
\definecolor{currentfill}{rgb}{1.000000,1.000000,1.000000}%
\pgfsetfillcolor{currentfill}%
\pgfsetfillopacity{0.800000}%
\pgfsetlinewidth{1.003750pt}%
\definecolor{currentstroke}{rgb}{0.800000,0.800000,0.800000}%
\pgfsetstrokecolor{currentstroke}%
\pgfsetstrokeopacity{0.800000}%
\pgfsetdash{}{0pt}%
\pgfpathmoveto{\pgfqpoint{0.782747in}{0.518889in}}%
\pgfpathlineto{\pgfqpoint{1.956358in}{0.518889in}}%
\pgfpathquadraticcurveto{\pgfqpoint{1.984136in}{0.518889in}}{\pgfqpoint{1.984136in}{0.546666in}}%
\pgfpathlineto{\pgfqpoint{1.984136in}{0.726388in}}%
\pgfpathquadraticcurveto{\pgfqpoint{1.984136in}{0.754166in}}{\pgfqpoint{1.956358in}{0.754166in}}%
\pgfpathlineto{\pgfqpoint{0.782747in}{0.754166in}}%
\pgfpathquadraticcurveto{\pgfqpoint{0.754970in}{0.754166in}}{\pgfqpoint{0.754970in}{0.726388in}}%
\pgfpathlineto{\pgfqpoint{0.754970in}{0.546666in}}%
\pgfpathquadraticcurveto{\pgfqpoint{0.754970in}{0.518889in}}{\pgfqpoint{0.782747in}{0.518889in}}%
\pgfpathlineto{\pgfqpoint{0.782747in}{0.518889in}}%
\pgfpathclose%
\pgfusepath{stroke,fill}%
\end{pgfscope}%
\begin{pgfscope}%
\pgfsetrectcap%
\pgfsetroundjoin%
\pgfsetlinewidth{1.505625pt}%
\definecolor{currentstroke}{rgb}{0.000000,0.000000,0.000000}%
\pgfsetstrokecolor{currentstroke}%
\pgfsetdash{}{0pt}%
\pgfpathmoveto{\pgfqpoint{0.810525in}{0.650000in}}%
\pgfpathlineto{\pgfqpoint{0.949414in}{0.650000in}}%
\pgfpathlineto{\pgfqpoint{1.088303in}{0.650000in}}%
\pgfusepath{stroke}%
\end{pgfscope}%
\begin{pgfscope}%
\definecolor{textcolor}{rgb}{0.000000,0.000000,0.000000}%
\pgfsetstrokecolor{textcolor}%
\pgfsetfillcolor{textcolor}%
\pgftext[x=1.199414in,y=0.601388in,left,base]{\color{textcolor}\rmfamily\fontsize{10.000000}{12.000000}\selectfont AUC=0.552}%
\end{pgfscope}%
\end{pgfpicture}%
\makeatother%
\endgroup%

\end{tabular}
\end{center}

\begin{center}
\begin{tabular}{cc}
\begin{tabular}{cc|c|c|}
	&\multicolumn{1}{c}{}& \multicolumn{2}{c}{Prediction} \cr
	&\multicolumn{1}{c}{} & \multicolumn{1}{c}{N} & \multicolumn{1}{c}{P} \cr\cline{3-4}
	\multirow{2}{*}{Actual}&N & 41.1\% & 44.6\% \vrule width 0pt height 10pt depth 2pt \cr\cline{3-4}
	&P & 5.71\% & 8.57\% \vrule width 0pt height 10pt depth 2pt \cr\cline{3-4}
\end{tabular}
&
\begin{tabular}{ll}
0.497 & Accuracy \cr 
0.540 & Balanced Accuracy \cr 
0.161 & Precision \cr 
0.536 & Balanced Precision \cr 
0.600 & Recall \cr 
0.254 & F1 \cr 
0.566 & Balanced F1 \cr 
0.278 & Gmean \cr 	\end{tabular}
\end{tabular}
\end{center}






