% Analysis of Results

Our ML algorithms assign to each sample (feature vector, crash person) a probability $p \in [0,1]$ that the person needs an ambulance.  The histogram below left shows the percentage of the dataset in each range of $p$, showing the percentages for the negative class (``Does not need an ambulance'') and the positive class (``Needs an ambulance'').  On the right, the Receiver Operating Characteristic (ROC) curve, and particularly the area under the curve (AUC), is a metric for how well the model separates the two classes, with $AUC=1.0$ being perfect and $AUC=0.5$ (the dashed line) being just random assignment with no insight.  

We would love to have results like in the graphs below, where the machine learning (ML) algorithm nearly perfectly separates the two classes.  There is some overlap between $p=0.6$ and $p=0.8$ with some samples the algorithm misclassifies, but the model clearly separates most samples.  Having an AUC of 0.996 would be amazing.  

[Put in \verb|BRFC_Hard_alpha_0_5_Train_Pred_Wide.pgf| once we have it.)

\begin{comment}

\noindent\begin{tabular}{@{\hspace{-6pt}}p{4.3in} @{\hspace{-6pt}}p{2.0in}}
	\vskip 0pt
	\hfil Raw Model Output
	
	%% Creator: Matplotlib, PGF backend
%%
%% To include the figure in your LaTeX document, write
%%   \input{<filename>.pgf}
%%
%% Make sure the required packages are loaded in your preamble
%%   \usepackage{pgf}
%%
%% Also ensure that all the required font packages are loaded; for instance,
%% the lmodern package is sometimes necessary when using math font.
%%   \usepackage{lmodern}
%%
%% Figures using additional raster images can only be included by \input if
%% they are in the same directory as the main LaTeX file. For loading figures
%% from other directories you can use the `import` package
%%   \usepackage{import}
%%
%% and then include the figures with
%%   \import{<path to file>}{<filename>.pgf}
%%
%% Matplotlib used the following preamble
%%   
%%   \usepackage{fontspec}
%%   \makeatletter\@ifpackageloaded{underscore}{}{\usepackage[strings]{underscore}}\makeatother
%%
\begingroup%
\makeatletter%
\begin{pgfpicture}%
\pgfpathrectangle{\pgfpointorigin}{\pgfqpoint{4.509306in}{1.754444in}}%
\pgfusepath{use as bounding box, clip}%
\begin{pgfscope}%
\pgfsetbuttcap%
\pgfsetmiterjoin%
\definecolor{currentfill}{rgb}{1.000000,1.000000,1.000000}%
\pgfsetfillcolor{currentfill}%
\pgfsetlinewidth{0.000000pt}%
\definecolor{currentstroke}{rgb}{1.000000,1.000000,1.000000}%
\pgfsetstrokecolor{currentstroke}%
\pgfsetdash{}{0pt}%
\pgfpathmoveto{\pgfqpoint{0.000000in}{0.000000in}}%
\pgfpathlineto{\pgfqpoint{4.509306in}{0.000000in}}%
\pgfpathlineto{\pgfqpoint{4.509306in}{1.754444in}}%
\pgfpathlineto{\pgfqpoint{0.000000in}{1.754444in}}%
\pgfpathlineto{\pgfqpoint{0.000000in}{0.000000in}}%
\pgfpathclose%
\pgfusepath{fill}%
\end{pgfscope}%
\begin{pgfscope}%
\pgfsetbuttcap%
\pgfsetmiterjoin%
\definecolor{currentfill}{rgb}{1.000000,1.000000,1.000000}%
\pgfsetfillcolor{currentfill}%
\pgfsetlinewidth{0.000000pt}%
\definecolor{currentstroke}{rgb}{0.000000,0.000000,0.000000}%
\pgfsetstrokecolor{currentstroke}%
\pgfsetstrokeopacity{0.000000}%
\pgfsetdash{}{0pt}%
\pgfpathmoveto{\pgfqpoint{0.445556in}{0.499444in}}%
\pgfpathlineto{\pgfqpoint{4.320556in}{0.499444in}}%
\pgfpathlineto{\pgfqpoint{4.320556in}{1.654444in}}%
\pgfpathlineto{\pgfqpoint{0.445556in}{1.654444in}}%
\pgfpathlineto{\pgfqpoint{0.445556in}{0.499444in}}%
\pgfpathclose%
\pgfusepath{fill}%
\end{pgfscope}%
\begin{pgfscope}%
\pgfpathrectangle{\pgfqpoint{0.445556in}{0.499444in}}{\pgfqpoint{3.875000in}{1.155000in}}%
\pgfusepath{clip}%
\pgfsetbuttcap%
\pgfsetmiterjoin%
\pgfsetlinewidth{1.003750pt}%
\definecolor{currentstroke}{rgb}{0.000000,0.000000,0.000000}%
\pgfsetstrokecolor{currentstroke}%
\pgfsetdash{}{0pt}%
\pgfpathmoveto{\pgfqpoint{0.435556in}{0.499444in}}%
\pgfpathlineto{\pgfqpoint{0.483922in}{0.499444in}}%
\pgfpathlineto{\pgfqpoint{0.483922in}{0.632510in}}%
\pgfpathlineto{\pgfqpoint{0.435556in}{0.632510in}}%
\pgfusepath{stroke}%
\end{pgfscope}%
\begin{pgfscope}%
\pgfpathrectangle{\pgfqpoint{0.445556in}{0.499444in}}{\pgfqpoint{3.875000in}{1.155000in}}%
\pgfusepath{clip}%
\pgfsetbuttcap%
\pgfsetmiterjoin%
\pgfsetlinewidth{1.003750pt}%
\definecolor{currentstroke}{rgb}{0.000000,0.000000,0.000000}%
\pgfsetstrokecolor{currentstroke}%
\pgfsetdash{}{0pt}%
\pgfpathmoveto{\pgfqpoint{0.576001in}{0.499444in}}%
\pgfpathlineto{\pgfqpoint{0.637387in}{0.499444in}}%
\pgfpathlineto{\pgfqpoint{0.637387in}{0.855505in}}%
\pgfpathlineto{\pgfqpoint{0.576001in}{0.855505in}}%
\pgfpathlineto{\pgfqpoint{0.576001in}{0.499444in}}%
\pgfpathclose%
\pgfusepath{stroke}%
\end{pgfscope}%
\begin{pgfscope}%
\pgfpathrectangle{\pgfqpoint{0.445556in}{0.499444in}}{\pgfqpoint{3.875000in}{1.155000in}}%
\pgfusepath{clip}%
\pgfsetbuttcap%
\pgfsetmiterjoin%
\pgfsetlinewidth{1.003750pt}%
\definecolor{currentstroke}{rgb}{0.000000,0.000000,0.000000}%
\pgfsetstrokecolor{currentstroke}%
\pgfsetdash{}{0pt}%
\pgfpathmoveto{\pgfqpoint{0.729467in}{0.499444in}}%
\pgfpathlineto{\pgfqpoint{0.790853in}{0.499444in}}%
\pgfpathlineto{\pgfqpoint{0.790853in}{1.087595in}}%
\pgfpathlineto{\pgfqpoint{0.729467in}{1.087595in}}%
\pgfpathlineto{\pgfqpoint{0.729467in}{0.499444in}}%
\pgfpathclose%
\pgfusepath{stroke}%
\end{pgfscope}%
\begin{pgfscope}%
\pgfpathrectangle{\pgfqpoint{0.445556in}{0.499444in}}{\pgfqpoint{3.875000in}{1.155000in}}%
\pgfusepath{clip}%
\pgfsetbuttcap%
\pgfsetmiterjoin%
\pgfsetlinewidth{1.003750pt}%
\definecolor{currentstroke}{rgb}{0.000000,0.000000,0.000000}%
\pgfsetstrokecolor{currentstroke}%
\pgfsetdash{}{0pt}%
\pgfpathmoveto{\pgfqpoint{0.882932in}{0.499444in}}%
\pgfpathlineto{\pgfqpoint{0.944318in}{0.499444in}}%
\pgfpathlineto{\pgfqpoint{0.944318in}{1.269324in}}%
\pgfpathlineto{\pgfqpoint{0.882932in}{1.269324in}}%
\pgfpathlineto{\pgfqpoint{0.882932in}{0.499444in}}%
\pgfpathclose%
\pgfusepath{stroke}%
\end{pgfscope}%
\begin{pgfscope}%
\pgfpathrectangle{\pgfqpoint{0.445556in}{0.499444in}}{\pgfqpoint{3.875000in}{1.155000in}}%
\pgfusepath{clip}%
\pgfsetbuttcap%
\pgfsetmiterjoin%
\pgfsetlinewidth{1.003750pt}%
\definecolor{currentstroke}{rgb}{0.000000,0.000000,0.000000}%
\pgfsetstrokecolor{currentstroke}%
\pgfsetdash{}{0pt}%
\pgfpathmoveto{\pgfqpoint{1.036397in}{0.499444in}}%
\pgfpathlineto{\pgfqpoint{1.097783in}{0.499444in}}%
\pgfpathlineto{\pgfqpoint{1.097783in}{1.428301in}}%
\pgfpathlineto{\pgfqpoint{1.036397in}{1.428301in}}%
\pgfpathlineto{\pgfqpoint{1.036397in}{0.499444in}}%
\pgfpathclose%
\pgfusepath{stroke}%
\end{pgfscope}%
\begin{pgfscope}%
\pgfpathrectangle{\pgfqpoint{0.445556in}{0.499444in}}{\pgfqpoint{3.875000in}{1.155000in}}%
\pgfusepath{clip}%
\pgfsetbuttcap%
\pgfsetmiterjoin%
\pgfsetlinewidth{1.003750pt}%
\definecolor{currentstroke}{rgb}{0.000000,0.000000,0.000000}%
\pgfsetstrokecolor{currentstroke}%
\pgfsetdash{}{0pt}%
\pgfpathmoveto{\pgfqpoint{1.189863in}{0.499444in}}%
\pgfpathlineto{\pgfqpoint{1.251249in}{0.499444in}}%
\pgfpathlineto{\pgfqpoint{1.251249in}{1.521593in}}%
\pgfpathlineto{\pgfqpoint{1.189863in}{1.521593in}}%
\pgfpathlineto{\pgfqpoint{1.189863in}{0.499444in}}%
\pgfpathclose%
\pgfusepath{stroke}%
\end{pgfscope}%
\begin{pgfscope}%
\pgfpathrectangle{\pgfqpoint{0.445556in}{0.499444in}}{\pgfqpoint{3.875000in}{1.155000in}}%
\pgfusepath{clip}%
\pgfsetbuttcap%
\pgfsetmiterjoin%
\pgfsetlinewidth{1.003750pt}%
\definecolor{currentstroke}{rgb}{0.000000,0.000000,0.000000}%
\pgfsetstrokecolor{currentstroke}%
\pgfsetdash{}{0pt}%
\pgfpathmoveto{\pgfqpoint{1.343328in}{0.499444in}}%
\pgfpathlineto{\pgfqpoint{1.404714in}{0.499444in}}%
\pgfpathlineto{\pgfqpoint{1.404714in}{1.590875in}}%
\pgfpathlineto{\pgfqpoint{1.343328in}{1.590875in}}%
\pgfpathlineto{\pgfqpoint{1.343328in}{0.499444in}}%
\pgfpathclose%
\pgfusepath{stroke}%
\end{pgfscope}%
\begin{pgfscope}%
\pgfpathrectangle{\pgfqpoint{0.445556in}{0.499444in}}{\pgfqpoint{3.875000in}{1.155000in}}%
\pgfusepath{clip}%
\pgfsetbuttcap%
\pgfsetmiterjoin%
\pgfsetlinewidth{1.003750pt}%
\definecolor{currentstroke}{rgb}{0.000000,0.000000,0.000000}%
\pgfsetstrokecolor{currentstroke}%
\pgfsetdash{}{0pt}%
\pgfpathmoveto{\pgfqpoint{1.496793in}{0.499444in}}%
\pgfpathlineto{\pgfqpoint{1.558179in}{0.499444in}}%
\pgfpathlineto{\pgfqpoint{1.558179in}{1.599444in}}%
\pgfpathlineto{\pgfqpoint{1.496793in}{1.599444in}}%
\pgfpathlineto{\pgfqpoint{1.496793in}{0.499444in}}%
\pgfpathclose%
\pgfusepath{stroke}%
\end{pgfscope}%
\begin{pgfscope}%
\pgfpathrectangle{\pgfqpoint{0.445556in}{0.499444in}}{\pgfqpoint{3.875000in}{1.155000in}}%
\pgfusepath{clip}%
\pgfsetbuttcap%
\pgfsetmiterjoin%
\pgfsetlinewidth{1.003750pt}%
\definecolor{currentstroke}{rgb}{0.000000,0.000000,0.000000}%
\pgfsetstrokecolor{currentstroke}%
\pgfsetdash{}{0pt}%
\pgfpathmoveto{\pgfqpoint{1.650259in}{0.499444in}}%
\pgfpathlineto{\pgfqpoint{1.711645in}{0.499444in}}%
\pgfpathlineto{\pgfqpoint{1.711645in}{1.566953in}}%
\pgfpathlineto{\pgfqpoint{1.650259in}{1.566953in}}%
\pgfpathlineto{\pgfqpoint{1.650259in}{0.499444in}}%
\pgfpathclose%
\pgfusepath{stroke}%
\end{pgfscope}%
\begin{pgfscope}%
\pgfpathrectangle{\pgfqpoint{0.445556in}{0.499444in}}{\pgfqpoint{3.875000in}{1.155000in}}%
\pgfusepath{clip}%
\pgfsetbuttcap%
\pgfsetmiterjoin%
\pgfsetlinewidth{1.003750pt}%
\definecolor{currentstroke}{rgb}{0.000000,0.000000,0.000000}%
\pgfsetstrokecolor{currentstroke}%
\pgfsetdash{}{0pt}%
\pgfpathmoveto{\pgfqpoint{1.803724in}{0.499444in}}%
\pgfpathlineto{\pgfqpoint{1.865110in}{0.499444in}}%
\pgfpathlineto{\pgfqpoint{1.865110in}{1.506942in}}%
\pgfpathlineto{\pgfqpoint{1.803724in}{1.506942in}}%
\pgfpathlineto{\pgfqpoint{1.803724in}{0.499444in}}%
\pgfpathclose%
\pgfusepath{stroke}%
\end{pgfscope}%
\begin{pgfscope}%
\pgfpathrectangle{\pgfqpoint{0.445556in}{0.499444in}}{\pgfqpoint{3.875000in}{1.155000in}}%
\pgfusepath{clip}%
\pgfsetbuttcap%
\pgfsetmiterjoin%
\pgfsetlinewidth{1.003750pt}%
\definecolor{currentstroke}{rgb}{0.000000,0.000000,0.000000}%
\pgfsetstrokecolor{currentstroke}%
\pgfsetdash{}{0pt}%
\pgfpathmoveto{\pgfqpoint{1.957189in}{0.499444in}}%
\pgfpathlineto{\pgfqpoint{2.018575in}{0.499444in}}%
\pgfpathlineto{\pgfqpoint{2.018575in}{1.416545in}}%
\pgfpathlineto{\pgfqpoint{1.957189in}{1.416545in}}%
\pgfpathlineto{\pgfqpoint{1.957189in}{0.499444in}}%
\pgfpathclose%
\pgfusepath{stroke}%
\end{pgfscope}%
\begin{pgfscope}%
\pgfpathrectangle{\pgfqpoint{0.445556in}{0.499444in}}{\pgfqpoint{3.875000in}{1.155000in}}%
\pgfusepath{clip}%
\pgfsetbuttcap%
\pgfsetmiterjoin%
\pgfsetlinewidth{1.003750pt}%
\definecolor{currentstroke}{rgb}{0.000000,0.000000,0.000000}%
\pgfsetstrokecolor{currentstroke}%
\pgfsetdash{}{0pt}%
\pgfpathmoveto{\pgfqpoint{2.110655in}{0.499444in}}%
\pgfpathlineto{\pgfqpoint{2.172041in}{0.499444in}}%
\pgfpathlineto{\pgfqpoint{2.172041in}{1.298482in}}%
\pgfpathlineto{\pgfqpoint{2.110655in}{1.298482in}}%
\pgfpathlineto{\pgfqpoint{2.110655in}{0.499444in}}%
\pgfpathclose%
\pgfusepath{stroke}%
\end{pgfscope}%
\begin{pgfscope}%
\pgfpathrectangle{\pgfqpoint{0.445556in}{0.499444in}}{\pgfqpoint{3.875000in}{1.155000in}}%
\pgfusepath{clip}%
\pgfsetbuttcap%
\pgfsetmiterjoin%
\pgfsetlinewidth{1.003750pt}%
\definecolor{currentstroke}{rgb}{0.000000,0.000000,0.000000}%
\pgfsetstrokecolor{currentstroke}%
\pgfsetdash{}{0pt}%
\pgfpathmoveto{\pgfqpoint{2.264120in}{0.499444in}}%
\pgfpathlineto{\pgfqpoint{2.325506in}{0.499444in}}%
\pgfpathlineto{\pgfqpoint{2.325506in}{1.165884in}}%
\pgfpathlineto{\pgfqpoint{2.264120in}{1.165884in}}%
\pgfpathlineto{\pgfqpoint{2.264120in}{0.499444in}}%
\pgfpathclose%
\pgfusepath{stroke}%
\end{pgfscope}%
\begin{pgfscope}%
\pgfpathrectangle{\pgfqpoint{0.445556in}{0.499444in}}{\pgfqpoint{3.875000in}{1.155000in}}%
\pgfusepath{clip}%
\pgfsetbuttcap%
\pgfsetmiterjoin%
\pgfsetlinewidth{1.003750pt}%
\definecolor{currentstroke}{rgb}{0.000000,0.000000,0.000000}%
\pgfsetstrokecolor{currentstroke}%
\pgfsetdash{}{0pt}%
\pgfpathmoveto{\pgfqpoint{2.417585in}{0.499444in}}%
\pgfpathlineto{\pgfqpoint{2.478972in}{0.499444in}}%
\pgfpathlineto{\pgfqpoint{2.478972in}{1.041650in}}%
\pgfpathlineto{\pgfqpoint{2.417585in}{1.041650in}}%
\pgfpathlineto{\pgfqpoint{2.417585in}{0.499444in}}%
\pgfpathclose%
\pgfusepath{stroke}%
\end{pgfscope}%
\begin{pgfscope}%
\pgfpathrectangle{\pgfqpoint{0.445556in}{0.499444in}}{\pgfqpoint{3.875000in}{1.155000in}}%
\pgfusepath{clip}%
\pgfsetbuttcap%
\pgfsetmiterjoin%
\pgfsetlinewidth{1.003750pt}%
\definecolor{currentstroke}{rgb}{0.000000,0.000000,0.000000}%
\pgfsetstrokecolor{currentstroke}%
\pgfsetdash{}{0pt}%
\pgfpathmoveto{\pgfqpoint{2.571051in}{0.499444in}}%
\pgfpathlineto{\pgfqpoint{2.632437in}{0.499444in}}%
\pgfpathlineto{\pgfqpoint{2.632437in}{0.916218in}}%
\pgfpathlineto{\pgfqpoint{2.571051in}{0.916218in}}%
\pgfpathlineto{\pgfqpoint{2.571051in}{0.499444in}}%
\pgfpathclose%
\pgfusepath{stroke}%
\end{pgfscope}%
\begin{pgfscope}%
\pgfpathrectangle{\pgfqpoint{0.445556in}{0.499444in}}{\pgfqpoint{3.875000in}{1.155000in}}%
\pgfusepath{clip}%
\pgfsetbuttcap%
\pgfsetmiterjoin%
\pgfsetlinewidth{1.003750pt}%
\definecolor{currentstroke}{rgb}{0.000000,0.000000,0.000000}%
\pgfsetstrokecolor{currentstroke}%
\pgfsetdash{}{0pt}%
\pgfpathmoveto{\pgfqpoint{2.724516in}{0.499444in}}%
\pgfpathlineto{\pgfqpoint{2.785902in}{0.499444in}}%
\pgfpathlineto{\pgfqpoint{2.785902in}{0.806958in}}%
\pgfpathlineto{\pgfqpoint{2.724516in}{0.806958in}}%
\pgfpathlineto{\pgfqpoint{2.724516in}{0.499444in}}%
\pgfpathclose%
\pgfusepath{stroke}%
\end{pgfscope}%
\begin{pgfscope}%
\pgfpathrectangle{\pgfqpoint{0.445556in}{0.499444in}}{\pgfqpoint{3.875000in}{1.155000in}}%
\pgfusepath{clip}%
\pgfsetbuttcap%
\pgfsetmiterjoin%
\pgfsetlinewidth{1.003750pt}%
\definecolor{currentstroke}{rgb}{0.000000,0.000000,0.000000}%
\pgfsetstrokecolor{currentstroke}%
\pgfsetdash{}{0pt}%
\pgfpathmoveto{\pgfqpoint{2.877981in}{0.499444in}}%
\pgfpathlineto{\pgfqpoint{2.939368in}{0.499444in}}%
\pgfpathlineto{\pgfqpoint{2.939368in}{0.716063in}}%
\pgfpathlineto{\pgfqpoint{2.877981in}{0.716063in}}%
\pgfpathlineto{\pgfqpoint{2.877981in}{0.499444in}}%
\pgfpathclose%
\pgfusepath{stroke}%
\end{pgfscope}%
\begin{pgfscope}%
\pgfpathrectangle{\pgfqpoint{0.445556in}{0.499444in}}{\pgfqpoint{3.875000in}{1.155000in}}%
\pgfusepath{clip}%
\pgfsetbuttcap%
\pgfsetmiterjoin%
\pgfsetlinewidth{1.003750pt}%
\definecolor{currentstroke}{rgb}{0.000000,0.000000,0.000000}%
\pgfsetstrokecolor{currentstroke}%
\pgfsetdash{}{0pt}%
\pgfpathmoveto{\pgfqpoint{3.031447in}{0.499444in}}%
\pgfpathlineto{\pgfqpoint{3.092833in}{0.499444in}}%
\pgfpathlineto{\pgfqpoint{3.092833in}{0.645904in}}%
\pgfpathlineto{\pgfqpoint{3.031447in}{0.645904in}}%
\pgfpathlineto{\pgfqpoint{3.031447in}{0.499444in}}%
\pgfpathclose%
\pgfusepath{stroke}%
\end{pgfscope}%
\begin{pgfscope}%
\pgfpathrectangle{\pgfqpoint{0.445556in}{0.499444in}}{\pgfqpoint{3.875000in}{1.155000in}}%
\pgfusepath{clip}%
\pgfsetbuttcap%
\pgfsetmiterjoin%
\pgfsetlinewidth{1.003750pt}%
\definecolor{currentstroke}{rgb}{0.000000,0.000000,0.000000}%
\pgfsetstrokecolor{currentstroke}%
\pgfsetdash{}{0pt}%
\pgfpathmoveto{\pgfqpoint{3.184912in}{0.499444in}}%
\pgfpathlineto{\pgfqpoint{3.246298in}{0.499444in}}%
\pgfpathlineto{\pgfqpoint{3.246298in}{0.598936in}}%
\pgfpathlineto{\pgfqpoint{3.184912in}{0.598936in}}%
\pgfpathlineto{\pgfqpoint{3.184912in}{0.499444in}}%
\pgfpathclose%
\pgfusepath{stroke}%
\end{pgfscope}%
\begin{pgfscope}%
\pgfpathrectangle{\pgfqpoint{0.445556in}{0.499444in}}{\pgfqpoint{3.875000in}{1.155000in}}%
\pgfusepath{clip}%
\pgfsetbuttcap%
\pgfsetmiterjoin%
\pgfsetlinewidth{1.003750pt}%
\definecolor{currentstroke}{rgb}{0.000000,0.000000,0.000000}%
\pgfsetstrokecolor{currentstroke}%
\pgfsetdash{}{0pt}%
\pgfpathmoveto{\pgfqpoint{3.338377in}{0.499444in}}%
\pgfpathlineto{\pgfqpoint{3.399764in}{0.499444in}}%
\pgfpathlineto{\pgfqpoint{3.399764in}{0.560040in}}%
\pgfpathlineto{\pgfqpoint{3.338377in}{0.560040in}}%
\pgfpathlineto{\pgfqpoint{3.338377in}{0.499444in}}%
\pgfpathclose%
\pgfusepath{stroke}%
\end{pgfscope}%
\begin{pgfscope}%
\pgfpathrectangle{\pgfqpoint{0.445556in}{0.499444in}}{\pgfqpoint{3.875000in}{1.155000in}}%
\pgfusepath{clip}%
\pgfsetbuttcap%
\pgfsetmiterjoin%
\pgfsetlinewidth{1.003750pt}%
\definecolor{currentstroke}{rgb}{0.000000,0.000000,0.000000}%
\pgfsetstrokecolor{currentstroke}%
\pgfsetdash{}{0pt}%
\pgfpathmoveto{\pgfqpoint{3.491843in}{0.499444in}}%
\pgfpathlineto{\pgfqpoint{3.553229in}{0.499444in}}%
\pgfpathlineto{\pgfqpoint{3.553229in}{0.535591in}}%
\pgfpathlineto{\pgfqpoint{3.491843in}{0.535591in}}%
\pgfpathlineto{\pgfqpoint{3.491843in}{0.499444in}}%
\pgfpathclose%
\pgfusepath{stroke}%
\end{pgfscope}%
\begin{pgfscope}%
\pgfpathrectangle{\pgfqpoint{0.445556in}{0.499444in}}{\pgfqpoint{3.875000in}{1.155000in}}%
\pgfusepath{clip}%
\pgfsetbuttcap%
\pgfsetmiterjoin%
\pgfsetlinewidth{1.003750pt}%
\definecolor{currentstroke}{rgb}{0.000000,0.000000,0.000000}%
\pgfsetstrokecolor{currentstroke}%
\pgfsetdash{}{0pt}%
\pgfpathmoveto{\pgfqpoint{3.645308in}{0.499444in}}%
\pgfpathlineto{\pgfqpoint{3.706694in}{0.499444in}}%
\pgfpathlineto{\pgfqpoint{3.706694in}{0.512137in}}%
\pgfpathlineto{\pgfqpoint{3.645308in}{0.512137in}}%
\pgfpathlineto{\pgfqpoint{3.645308in}{0.499444in}}%
\pgfpathclose%
\pgfusepath{stroke}%
\end{pgfscope}%
\begin{pgfscope}%
\pgfpathrectangle{\pgfqpoint{0.445556in}{0.499444in}}{\pgfqpoint{3.875000in}{1.155000in}}%
\pgfusepath{clip}%
\pgfsetbuttcap%
\pgfsetmiterjoin%
\pgfsetlinewidth{1.003750pt}%
\definecolor{currentstroke}{rgb}{0.000000,0.000000,0.000000}%
\pgfsetstrokecolor{currentstroke}%
\pgfsetdash{}{0pt}%
\pgfpathmoveto{\pgfqpoint{3.798774in}{0.499444in}}%
\pgfpathlineto{\pgfqpoint{3.860160in}{0.499444in}}%
\pgfpathlineto{\pgfqpoint{3.860160in}{0.503305in}}%
\pgfpathlineto{\pgfqpoint{3.798774in}{0.503305in}}%
\pgfpathlineto{\pgfqpoint{3.798774in}{0.499444in}}%
\pgfpathclose%
\pgfusepath{stroke}%
\end{pgfscope}%
\begin{pgfscope}%
\pgfpathrectangle{\pgfqpoint{0.445556in}{0.499444in}}{\pgfqpoint{3.875000in}{1.155000in}}%
\pgfusepath{clip}%
\pgfsetbuttcap%
\pgfsetmiterjoin%
\pgfsetlinewidth{1.003750pt}%
\definecolor{currentstroke}{rgb}{0.000000,0.000000,0.000000}%
\pgfsetstrokecolor{currentstroke}%
\pgfsetdash{}{0pt}%
\pgfpathmoveto{\pgfqpoint{3.952239in}{0.499444in}}%
\pgfpathlineto{\pgfqpoint{4.013625in}{0.499444in}}%
\pgfpathlineto{\pgfqpoint{4.013625in}{0.500234in}}%
\pgfpathlineto{\pgfqpoint{3.952239in}{0.500234in}}%
\pgfpathlineto{\pgfqpoint{3.952239in}{0.499444in}}%
\pgfpathclose%
\pgfusepath{stroke}%
\end{pgfscope}%
\begin{pgfscope}%
\pgfpathrectangle{\pgfqpoint{0.445556in}{0.499444in}}{\pgfqpoint{3.875000in}{1.155000in}}%
\pgfusepath{clip}%
\pgfsetbuttcap%
\pgfsetmiterjoin%
\pgfsetlinewidth{1.003750pt}%
\definecolor{currentstroke}{rgb}{0.000000,0.000000,0.000000}%
\pgfsetstrokecolor{currentstroke}%
\pgfsetdash{}{0pt}%
\pgfpathmoveto{\pgfqpoint{4.105704in}{0.499444in}}%
\pgfpathlineto{\pgfqpoint{4.167090in}{0.499444in}}%
\pgfpathlineto{\pgfqpoint{4.167090in}{0.499503in}}%
\pgfpathlineto{\pgfqpoint{4.105704in}{0.499503in}}%
\pgfpathlineto{\pgfqpoint{4.105704in}{0.499444in}}%
\pgfpathclose%
\pgfusepath{stroke}%
\end{pgfscope}%
\begin{pgfscope}%
\pgfpathrectangle{\pgfqpoint{0.445556in}{0.499444in}}{\pgfqpoint{3.875000in}{1.155000in}}%
\pgfusepath{clip}%
\pgfsetbuttcap%
\pgfsetmiterjoin%
\definecolor{currentfill}{rgb}{0.000000,0.000000,0.000000}%
\pgfsetfillcolor{currentfill}%
\pgfsetlinewidth{0.000000pt}%
\definecolor{currentstroke}{rgb}{0.000000,0.000000,0.000000}%
\pgfsetstrokecolor{currentstroke}%
\pgfsetstrokeopacity{0.000000}%
\pgfsetdash{}{0pt}%
\pgfpathmoveto{\pgfqpoint{0.483922in}{0.499444in}}%
\pgfpathlineto{\pgfqpoint{0.545308in}{0.499444in}}%
\pgfpathlineto{\pgfqpoint{0.545308in}{0.499444in}}%
\pgfpathlineto{\pgfqpoint{0.483922in}{0.499444in}}%
\pgfpathlineto{\pgfqpoint{0.483922in}{0.499444in}}%
\pgfpathclose%
\pgfusepath{fill}%
\end{pgfscope}%
\begin{pgfscope}%
\pgfpathrectangle{\pgfqpoint{0.445556in}{0.499444in}}{\pgfqpoint{3.875000in}{1.155000in}}%
\pgfusepath{clip}%
\pgfsetbuttcap%
\pgfsetmiterjoin%
\definecolor{currentfill}{rgb}{0.000000,0.000000,0.000000}%
\pgfsetfillcolor{currentfill}%
\pgfsetlinewidth{0.000000pt}%
\definecolor{currentstroke}{rgb}{0.000000,0.000000,0.000000}%
\pgfsetstrokecolor{currentstroke}%
\pgfsetstrokeopacity{0.000000}%
\pgfsetdash{}{0pt}%
\pgfpathmoveto{\pgfqpoint{0.637387in}{0.499444in}}%
\pgfpathlineto{\pgfqpoint{0.698774in}{0.499444in}}%
\pgfpathlineto{\pgfqpoint{0.698774in}{0.499444in}}%
\pgfpathlineto{\pgfqpoint{0.637387in}{0.499444in}}%
\pgfpathlineto{\pgfqpoint{0.637387in}{0.499444in}}%
\pgfpathclose%
\pgfusepath{fill}%
\end{pgfscope}%
\begin{pgfscope}%
\pgfpathrectangle{\pgfqpoint{0.445556in}{0.499444in}}{\pgfqpoint{3.875000in}{1.155000in}}%
\pgfusepath{clip}%
\pgfsetbuttcap%
\pgfsetmiterjoin%
\definecolor{currentfill}{rgb}{0.000000,0.000000,0.000000}%
\pgfsetfillcolor{currentfill}%
\pgfsetlinewidth{0.000000pt}%
\definecolor{currentstroke}{rgb}{0.000000,0.000000,0.000000}%
\pgfsetstrokecolor{currentstroke}%
\pgfsetstrokeopacity{0.000000}%
\pgfsetdash{}{0pt}%
\pgfpathmoveto{\pgfqpoint{0.790853in}{0.499444in}}%
\pgfpathlineto{\pgfqpoint{0.852239in}{0.499444in}}%
\pgfpathlineto{\pgfqpoint{0.852239in}{0.499444in}}%
\pgfpathlineto{\pgfqpoint{0.790853in}{0.499444in}}%
\pgfpathlineto{\pgfqpoint{0.790853in}{0.499444in}}%
\pgfpathclose%
\pgfusepath{fill}%
\end{pgfscope}%
\begin{pgfscope}%
\pgfpathrectangle{\pgfqpoint{0.445556in}{0.499444in}}{\pgfqpoint{3.875000in}{1.155000in}}%
\pgfusepath{clip}%
\pgfsetbuttcap%
\pgfsetmiterjoin%
\definecolor{currentfill}{rgb}{0.000000,0.000000,0.000000}%
\pgfsetfillcolor{currentfill}%
\pgfsetlinewidth{0.000000pt}%
\definecolor{currentstroke}{rgb}{0.000000,0.000000,0.000000}%
\pgfsetstrokecolor{currentstroke}%
\pgfsetstrokeopacity{0.000000}%
\pgfsetdash{}{0pt}%
\pgfpathmoveto{\pgfqpoint{0.944318in}{0.499444in}}%
\pgfpathlineto{\pgfqpoint{1.005704in}{0.499444in}}%
\pgfpathlineto{\pgfqpoint{1.005704in}{0.499444in}}%
\pgfpathlineto{\pgfqpoint{0.944318in}{0.499444in}}%
\pgfpathlineto{\pgfqpoint{0.944318in}{0.499444in}}%
\pgfpathclose%
\pgfusepath{fill}%
\end{pgfscope}%
\begin{pgfscope}%
\pgfpathrectangle{\pgfqpoint{0.445556in}{0.499444in}}{\pgfqpoint{3.875000in}{1.155000in}}%
\pgfusepath{clip}%
\pgfsetbuttcap%
\pgfsetmiterjoin%
\definecolor{currentfill}{rgb}{0.000000,0.000000,0.000000}%
\pgfsetfillcolor{currentfill}%
\pgfsetlinewidth{0.000000pt}%
\definecolor{currentstroke}{rgb}{0.000000,0.000000,0.000000}%
\pgfsetstrokecolor{currentstroke}%
\pgfsetstrokeopacity{0.000000}%
\pgfsetdash{}{0pt}%
\pgfpathmoveto{\pgfqpoint{1.097783in}{0.499444in}}%
\pgfpathlineto{\pgfqpoint{1.159170in}{0.499444in}}%
\pgfpathlineto{\pgfqpoint{1.159170in}{0.499444in}}%
\pgfpathlineto{\pgfqpoint{1.097783in}{0.499444in}}%
\pgfpathlineto{\pgfqpoint{1.097783in}{0.499444in}}%
\pgfpathclose%
\pgfusepath{fill}%
\end{pgfscope}%
\begin{pgfscope}%
\pgfpathrectangle{\pgfqpoint{0.445556in}{0.499444in}}{\pgfqpoint{3.875000in}{1.155000in}}%
\pgfusepath{clip}%
\pgfsetbuttcap%
\pgfsetmiterjoin%
\definecolor{currentfill}{rgb}{0.000000,0.000000,0.000000}%
\pgfsetfillcolor{currentfill}%
\pgfsetlinewidth{0.000000pt}%
\definecolor{currentstroke}{rgb}{0.000000,0.000000,0.000000}%
\pgfsetstrokecolor{currentstroke}%
\pgfsetstrokeopacity{0.000000}%
\pgfsetdash{}{0pt}%
\pgfpathmoveto{\pgfqpoint{1.251249in}{0.499444in}}%
\pgfpathlineto{\pgfqpoint{1.312635in}{0.499444in}}%
\pgfpathlineto{\pgfqpoint{1.312635in}{0.499444in}}%
\pgfpathlineto{\pgfqpoint{1.251249in}{0.499444in}}%
\pgfpathlineto{\pgfqpoint{1.251249in}{0.499444in}}%
\pgfpathclose%
\pgfusepath{fill}%
\end{pgfscope}%
\begin{pgfscope}%
\pgfpathrectangle{\pgfqpoint{0.445556in}{0.499444in}}{\pgfqpoint{3.875000in}{1.155000in}}%
\pgfusepath{clip}%
\pgfsetbuttcap%
\pgfsetmiterjoin%
\definecolor{currentfill}{rgb}{0.000000,0.000000,0.000000}%
\pgfsetfillcolor{currentfill}%
\pgfsetlinewidth{0.000000pt}%
\definecolor{currentstroke}{rgb}{0.000000,0.000000,0.000000}%
\pgfsetstrokecolor{currentstroke}%
\pgfsetstrokeopacity{0.000000}%
\pgfsetdash{}{0pt}%
\pgfpathmoveto{\pgfqpoint{1.404714in}{0.499444in}}%
\pgfpathlineto{\pgfqpoint{1.466100in}{0.499444in}}%
\pgfpathlineto{\pgfqpoint{1.466100in}{0.499444in}}%
\pgfpathlineto{\pgfqpoint{1.404714in}{0.499444in}}%
\pgfpathlineto{\pgfqpoint{1.404714in}{0.499444in}}%
\pgfpathclose%
\pgfusepath{fill}%
\end{pgfscope}%
\begin{pgfscope}%
\pgfpathrectangle{\pgfqpoint{0.445556in}{0.499444in}}{\pgfqpoint{3.875000in}{1.155000in}}%
\pgfusepath{clip}%
\pgfsetbuttcap%
\pgfsetmiterjoin%
\definecolor{currentfill}{rgb}{0.000000,0.000000,0.000000}%
\pgfsetfillcolor{currentfill}%
\pgfsetlinewidth{0.000000pt}%
\definecolor{currentstroke}{rgb}{0.000000,0.000000,0.000000}%
\pgfsetstrokecolor{currentstroke}%
\pgfsetstrokeopacity{0.000000}%
\pgfsetdash{}{0pt}%
\pgfpathmoveto{\pgfqpoint{1.558179in}{0.499444in}}%
\pgfpathlineto{\pgfqpoint{1.619566in}{0.499444in}}%
\pgfpathlineto{\pgfqpoint{1.619566in}{0.499444in}}%
\pgfpathlineto{\pgfqpoint{1.558179in}{0.499444in}}%
\pgfpathlineto{\pgfqpoint{1.558179in}{0.499444in}}%
\pgfpathclose%
\pgfusepath{fill}%
\end{pgfscope}%
\begin{pgfscope}%
\pgfpathrectangle{\pgfqpoint{0.445556in}{0.499444in}}{\pgfqpoint{3.875000in}{1.155000in}}%
\pgfusepath{clip}%
\pgfsetbuttcap%
\pgfsetmiterjoin%
\definecolor{currentfill}{rgb}{0.000000,0.000000,0.000000}%
\pgfsetfillcolor{currentfill}%
\pgfsetlinewidth{0.000000pt}%
\definecolor{currentstroke}{rgb}{0.000000,0.000000,0.000000}%
\pgfsetstrokecolor{currentstroke}%
\pgfsetstrokeopacity{0.000000}%
\pgfsetdash{}{0pt}%
\pgfpathmoveto{\pgfqpoint{1.711645in}{0.499444in}}%
\pgfpathlineto{\pgfqpoint{1.773031in}{0.499444in}}%
\pgfpathlineto{\pgfqpoint{1.773031in}{0.499444in}}%
\pgfpathlineto{\pgfqpoint{1.711645in}{0.499444in}}%
\pgfpathlineto{\pgfqpoint{1.711645in}{0.499444in}}%
\pgfpathclose%
\pgfusepath{fill}%
\end{pgfscope}%
\begin{pgfscope}%
\pgfpathrectangle{\pgfqpoint{0.445556in}{0.499444in}}{\pgfqpoint{3.875000in}{1.155000in}}%
\pgfusepath{clip}%
\pgfsetbuttcap%
\pgfsetmiterjoin%
\definecolor{currentfill}{rgb}{0.000000,0.000000,0.000000}%
\pgfsetfillcolor{currentfill}%
\pgfsetlinewidth{0.000000pt}%
\definecolor{currentstroke}{rgb}{0.000000,0.000000,0.000000}%
\pgfsetstrokecolor{currentstroke}%
\pgfsetstrokeopacity{0.000000}%
\pgfsetdash{}{0pt}%
\pgfpathmoveto{\pgfqpoint{1.865110in}{0.499444in}}%
\pgfpathlineto{\pgfqpoint{1.926496in}{0.499444in}}%
\pgfpathlineto{\pgfqpoint{1.926496in}{0.499444in}}%
\pgfpathlineto{\pgfqpoint{1.865110in}{0.499444in}}%
\pgfpathlineto{\pgfqpoint{1.865110in}{0.499444in}}%
\pgfpathclose%
\pgfusepath{fill}%
\end{pgfscope}%
\begin{pgfscope}%
\pgfpathrectangle{\pgfqpoint{0.445556in}{0.499444in}}{\pgfqpoint{3.875000in}{1.155000in}}%
\pgfusepath{clip}%
\pgfsetbuttcap%
\pgfsetmiterjoin%
\definecolor{currentfill}{rgb}{0.000000,0.000000,0.000000}%
\pgfsetfillcolor{currentfill}%
\pgfsetlinewidth{0.000000pt}%
\definecolor{currentstroke}{rgb}{0.000000,0.000000,0.000000}%
\pgfsetstrokecolor{currentstroke}%
\pgfsetstrokeopacity{0.000000}%
\pgfsetdash{}{0pt}%
\pgfpathmoveto{\pgfqpoint{2.018575in}{0.499444in}}%
\pgfpathlineto{\pgfqpoint{2.079962in}{0.499444in}}%
\pgfpathlineto{\pgfqpoint{2.079962in}{0.499444in}}%
\pgfpathlineto{\pgfqpoint{2.018575in}{0.499444in}}%
\pgfpathlineto{\pgfqpoint{2.018575in}{0.499444in}}%
\pgfpathclose%
\pgfusepath{fill}%
\end{pgfscope}%
\begin{pgfscope}%
\pgfpathrectangle{\pgfqpoint{0.445556in}{0.499444in}}{\pgfqpoint{3.875000in}{1.155000in}}%
\pgfusepath{clip}%
\pgfsetbuttcap%
\pgfsetmiterjoin%
\definecolor{currentfill}{rgb}{0.000000,0.000000,0.000000}%
\pgfsetfillcolor{currentfill}%
\pgfsetlinewidth{0.000000pt}%
\definecolor{currentstroke}{rgb}{0.000000,0.000000,0.000000}%
\pgfsetstrokecolor{currentstroke}%
\pgfsetstrokeopacity{0.000000}%
\pgfsetdash{}{0pt}%
\pgfpathmoveto{\pgfqpoint{2.172041in}{0.499444in}}%
\pgfpathlineto{\pgfqpoint{2.233427in}{0.499444in}}%
\pgfpathlineto{\pgfqpoint{2.233427in}{0.499444in}}%
\pgfpathlineto{\pgfqpoint{2.172041in}{0.499444in}}%
\pgfpathlineto{\pgfqpoint{2.172041in}{0.499444in}}%
\pgfpathclose%
\pgfusepath{fill}%
\end{pgfscope}%
\begin{pgfscope}%
\pgfpathrectangle{\pgfqpoint{0.445556in}{0.499444in}}{\pgfqpoint{3.875000in}{1.155000in}}%
\pgfusepath{clip}%
\pgfsetbuttcap%
\pgfsetmiterjoin%
\definecolor{currentfill}{rgb}{0.000000,0.000000,0.000000}%
\pgfsetfillcolor{currentfill}%
\pgfsetlinewidth{0.000000pt}%
\definecolor{currentstroke}{rgb}{0.000000,0.000000,0.000000}%
\pgfsetstrokecolor{currentstroke}%
\pgfsetstrokeopacity{0.000000}%
\pgfsetdash{}{0pt}%
\pgfpathmoveto{\pgfqpoint{2.325506in}{0.499444in}}%
\pgfpathlineto{\pgfqpoint{2.386892in}{0.499444in}}%
\pgfpathlineto{\pgfqpoint{2.386892in}{0.499561in}}%
\pgfpathlineto{\pgfqpoint{2.325506in}{0.499561in}}%
\pgfpathlineto{\pgfqpoint{2.325506in}{0.499444in}}%
\pgfpathclose%
\pgfusepath{fill}%
\end{pgfscope}%
\begin{pgfscope}%
\pgfpathrectangle{\pgfqpoint{0.445556in}{0.499444in}}{\pgfqpoint{3.875000in}{1.155000in}}%
\pgfusepath{clip}%
\pgfsetbuttcap%
\pgfsetmiterjoin%
\definecolor{currentfill}{rgb}{0.000000,0.000000,0.000000}%
\pgfsetfillcolor{currentfill}%
\pgfsetlinewidth{0.000000pt}%
\definecolor{currentstroke}{rgb}{0.000000,0.000000,0.000000}%
\pgfsetstrokecolor{currentstroke}%
\pgfsetstrokeopacity{0.000000}%
\pgfsetdash{}{0pt}%
\pgfpathmoveto{\pgfqpoint{2.478972in}{0.499444in}}%
\pgfpathlineto{\pgfqpoint{2.540358in}{0.499444in}}%
\pgfpathlineto{\pgfqpoint{2.540358in}{0.499854in}}%
\pgfpathlineto{\pgfqpoint{2.478972in}{0.499854in}}%
\pgfpathlineto{\pgfqpoint{2.478972in}{0.499444in}}%
\pgfpathclose%
\pgfusepath{fill}%
\end{pgfscope}%
\begin{pgfscope}%
\pgfpathrectangle{\pgfqpoint{0.445556in}{0.499444in}}{\pgfqpoint{3.875000in}{1.155000in}}%
\pgfusepath{clip}%
\pgfsetbuttcap%
\pgfsetmiterjoin%
\definecolor{currentfill}{rgb}{0.000000,0.000000,0.000000}%
\pgfsetfillcolor{currentfill}%
\pgfsetlinewidth{0.000000pt}%
\definecolor{currentstroke}{rgb}{0.000000,0.000000,0.000000}%
\pgfsetstrokecolor{currentstroke}%
\pgfsetstrokeopacity{0.000000}%
\pgfsetdash{}{0pt}%
\pgfpathmoveto{\pgfqpoint{2.632437in}{0.499444in}}%
\pgfpathlineto{\pgfqpoint{2.693823in}{0.499444in}}%
\pgfpathlineto{\pgfqpoint{2.693823in}{0.501521in}}%
\pgfpathlineto{\pgfqpoint{2.632437in}{0.501521in}}%
\pgfpathlineto{\pgfqpoint{2.632437in}{0.499444in}}%
\pgfpathclose%
\pgfusepath{fill}%
\end{pgfscope}%
\begin{pgfscope}%
\pgfpathrectangle{\pgfqpoint{0.445556in}{0.499444in}}{\pgfqpoint{3.875000in}{1.155000in}}%
\pgfusepath{clip}%
\pgfsetbuttcap%
\pgfsetmiterjoin%
\definecolor{currentfill}{rgb}{0.000000,0.000000,0.000000}%
\pgfsetfillcolor{currentfill}%
\pgfsetlinewidth{0.000000pt}%
\definecolor{currentstroke}{rgb}{0.000000,0.000000,0.000000}%
\pgfsetstrokecolor{currentstroke}%
\pgfsetstrokeopacity{0.000000}%
\pgfsetdash{}{0pt}%
\pgfpathmoveto{\pgfqpoint{2.785902in}{0.499444in}}%
\pgfpathlineto{\pgfqpoint{2.847288in}{0.499444in}}%
\pgfpathlineto{\pgfqpoint{2.847288in}{0.511171in}}%
\pgfpathlineto{\pgfqpoint{2.785902in}{0.511171in}}%
\pgfpathlineto{\pgfqpoint{2.785902in}{0.499444in}}%
\pgfpathclose%
\pgfusepath{fill}%
\end{pgfscope}%
\begin{pgfscope}%
\pgfpathrectangle{\pgfqpoint{0.445556in}{0.499444in}}{\pgfqpoint{3.875000in}{1.155000in}}%
\pgfusepath{clip}%
\pgfsetbuttcap%
\pgfsetmiterjoin%
\definecolor{currentfill}{rgb}{0.000000,0.000000,0.000000}%
\pgfsetfillcolor{currentfill}%
\pgfsetlinewidth{0.000000pt}%
\definecolor{currentstroke}{rgb}{0.000000,0.000000,0.000000}%
\pgfsetstrokecolor{currentstroke}%
\pgfsetstrokeopacity{0.000000}%
\pgfsetdash{}{0pt}%
\pgfpathmoveto{\pgfqpoint{2.939368in}{0.499444in}}%
\pgfpathlineto{\pgfqpoint{3.000754in}{0.499444in}}%
\pgfpathlineto{\pgfqpoint{3.000754in}{0.534889in}}%
\pgfpathlineto{\pgfqpoint{2.939368in}{0.534889in}}%
\pgfpathlineto{\pgfqpoint{2.939368in}{0.499444in}}%
\pgfpathclose%
\pgfusepath{fill}%
\end{pgfscope}%
\begin{pgfscope}%
\pgfpathrectangle{\pgfqpoint{0.445556in}{0.499444in}}{\pgfqpoint{3.875000in}{1.155000in}}%
\pgfusepath{clip}%
\pgfsetbuttcap%
\pgfsetmiterjoin%
\definecolor{currentfill}{rgb}{0.000000,0.000000,0.000000}%
\pgfsetfillcolor{currentfill}%
\pgfsetlinewidth{0.000000pt}%
\definecolor{currentstroke}{rgb}{0.000000,0.000000,0.000000}%
\pgfsetstrokecolor{currentstroke}%
\pgfsetstrokeopacity{0.000000}%
\pgfsetdash{}{0pt}%
\pgfpathmoveto{\pgfqpoint{3.092833in}{0.499444in}}%
\pgfpathlineto{\pgfqpoint{3.154219in}{0.499444in}}%
\pgfpathlineto{\pgfqpoint{3.154219in}{0.582998in}}%
\pgfpathlineto{\pgfqpoint{3.092833in}{0.582998in}}%
\pgfpathlineto{\pgfqpoint{3.092833in}{0.499444in}}%
\pgfpathclose%
\pgfusepath{fill}%
\end{pgfscope}%
\begin{pgfscope}%
\pgfpathrectangle{\pgfqpoint{0.445556in}{0.499444in}}{\pgfqpoint{3.875000in}{1.155000in}}%
\pgfusepath{clip}%
\pgfsetbuttcap%
\pgfsetmiterjoin%
\definecolor{currentfill}{rgb}{0.000000,0.000000,0.000000}%
\pgfsetfillcolor{currentfill}%
\pgfsetlinewidth{0.000000pt}%
\definecolor{currentstroke}{rgb}{0.000000,0.000000,0.000000}%
\pgfsetstrokecolor{currentstroke}%
\pgfsetstrokeopacity{0.000000}%
\pgfsetdash{}{0pt}%
\pgfpathmoveto{\pgfqpoint{3.246298in}{0.499444in}}%
\pgfpathlineto{\pgfqpoint{3.307684in}{0.499444in}}%
\pgfpathlineto{\pgfqpoint{3.307684in}{0.662340in}}%
\pgfpathlineto{\pgfqpoint{3.246298in}{0.662340in}}%
\pgfpathlineto{\pgfqpoint{3.246298in}{0.499444in}}%
\pgfpathclose%
\pgfusepath{fill}%
\end{pgfscope}%
\begin{pgfscope}%
\pgfpathrectangle{\pgfqpoint{0.445556in}{0.499444in}}{\pgfqpoint{3.875000in}{1.155000in}}%
\pgfusepath{clip}%
\pgfsetbuttcap%
\pgfsetmiterjoin%
\definecolor{currentfill}{rgb}{0.000000,0.000000,0.000000}%
\pgfsetfillcolor{currentfill}%
\pgfsetlinewidth{0.000000pt}%
\definecolor{currentstroke}{rgb}{0.000000,0.000000,0.000000}%
\pgfsetstrokecolor{currentstroke}%
\pgfsetstrokeopacity{0.000000}%
\pgfsetdash{}{0pt}%
\pgfpathmoveto{\pgfqpoint{3.399764in}{0.499444in}}%
\pgfpathlineto{\pgfqpoint{3.461150in}{0.499444in}}%
\pgfpathlineto{\pgfqpoint{3.461150in}{0.763967in}}%
\pgfpathlineto{\pgfqpoint{3.399764in}{0.763967in}}%
\pgfpathlineto{\pgfqpoint{3.399764in}{0.499444in}}%
\pgfpathclose%
\pgfusepath{fill}%
\end{pgfscope}%
\begin{pgfscope}%
\pgfpathrectangle{\pgfqpoint{0.445556in}{0.499444in}}{\pgfqpoint{3.875000in}{1.155000in}}%
\pgfusepath{clip}%
\pgfsetbuttcap%
\pgfsetmiterjoin%
\definecolor{currentfill}{rgb}{0.000000,0.000000,0.000000}%
\pgfsetfillcolor{currentfill}%
\pgfsetlinewidth{0.000000pt}%
\definecolor{currentstroke}{rgb}{0.000000,0.000000,0.000000}%
\pgfsetstrokecolor{currentstroke}%
\pgfsetstrokeopacity{0.000000}%
\pgfsetdash{}{0pt}%
\pgfpathmoveto{\pgfqpoint{3.553229in}{0.499444in}}%
\pgfpathlineto{\pgfqpoint{3.614615in}{0.499444in}}%
\pgfpathlineto{\pgfqpoint{3.614615in}{0.871794in}}%
\pgfpathlineto{\pgfqpoint{3.553229in}{0.871794in}}%
\pgfpathlineto{\pgfqpoint{3.553229in}{0.499444in}}%
\pgfpathclose%
\pgfusepath{fill}%
\end{pgfscope}%
\begin{pgfscope}%
\pgfpathrectangle{\pgfqpoint{0.445556in}{0.499444in}}{\pgfqpoint{3.875000in}{1.155000in}}%
\pgfusepath{clip}%
\pgfsetbuttcap%
\pgfsetmiterjoin%
\definecolor{currentfill}{rgb}{0.000000,0.000000,0.000000}%
\pgfsetfillcolor{currentfill}%
\pgfsetlinewidth{0.000000pt}%
\definecolor{currentstroke}{rgb}{0.000000,0.000000,0.000000}%
\pgfsetstrokecolor{currentstroke}%
\pgfsetstrokeopacity{0.000000}%
\pgfsetdash{}{0pt}%
\pgfpathmoveto{\pgfqpoint{3.706694in}{0.499444in}}%
\pgfpathlineto{\pgfqpoint{3.768080in}{0.499444in}}%
\pgfpathlineto{\pgfqpoint{3.768080in}{0.925810in}}%
\pgfpathlineto{\pgfqpoint{3.706694in}{0.925810in}}%
\pgfpathlineto{\pgfqpoint{3.706694in}{0.499444in}}%
\pgfpathclose%
\pgfusepath{fill}%
\end{pgfscope}%
\begin{pgfscope}%
\pgfpathrectangle{\pgfqpoint{0.445556in}{0.499444in}}{\pgfqpoint{3.875000in}{1.155000in}}%
\pgfusepath{clip}%
\pgfsetbuttcap%
\pgfsetmiterjoin%
\definecolor{currentfill}{rgb}{0.000000,0.000000,0.000000}%
\pgfsetfillcolor{currentfill}%
\pgfsetlinewidth{0.000000pt}%
\definecolor{currentstroke}{rgb}{0.000000,0.000000,0.000000}%
\pgfsetstrokecolor{currentstroke}%
\pgfsetstrokeopacity{0.000000}%
\pgfsetdash{}{0pt}%
\pgfpathmoveto{\pgfqpoint{3.860160in}{0.499444in}}%
\pgfpathlineto{\pgfqpoint{3.921546in}{0.499444in}}%
\pgfpathlineto{\pgfqpoint{3.921546in}{0.913586in}}%
\pgfpathlineto{\pgfqpoint{3.860160in}{0.913586in}}%
\pgfpathlineto{\pgfqpoint{3.860160in}{0.499444in}}%
\pgfpathclose%
\pgfusepath{fill}%
\end{pgfscope}%
\begin{pgfscope}%
\pgfpathrectangle{\pgfqpoint{0.445556in}{0.499444in}}{\pgfqpoint{3.875000in}{1.155000in}}%
\pgfusepath{clip}%
\pgfsetbuttcap%
\pgfsetmiterjoin%
\definecolor{currentfill}{rgb}{0.000000,0.000000,0.000000}%
\pgfsetfillcolor{currentfill}%
\pgfsetlinewidth{0.000000pt}%
\definecolor{currentstroke}{rgb}{0.000000,0.000000,0.000000}%
\pgfsetstrokecolor{currentstroke}%
\pgfsetstrokeopacity{0.000000}%
\pgfsetdash{}{0pt}%
\pgfpathmoveto{\pgfqpoint{4.013625in}{0.499444in}}%
\pgfpathlineto{\pgfqpoint{4.075011in}{0.499444in}}%
\pgfpathlineto{\pgfqpoint{4.075011in}{0.851907in}}%
\pgfpathlineto{\pgfqpoint{4.013625in}{0.851907in}}%
\pgfpathlineto{\pgfqpoint{4.013625in}{0.499444in}}%
\pgfpathclose%
\pgfusepath{fill}%
\end{pgfscope}%
\begin{pgfscope}%
\pgfpathrectangle{\pgfqpoint{0.445556in}{0.499444in}}{\pgfqpoint{3.875000in}{1.155000in}}%
\pgfusepath{clip}%
\pgfsetbuttcap%
\pgfsetmiterjoin%
\definecolor{currentfill}{rgb}{0.000000,0.000000,0.000000}%
\pgfsetfillcolor{currentfill}%
\pgfsetlinewidth{0.000000pt}%
\definecolor{currentstroke}{rgb}{0.000000,0.000000,0.000000}%
\pgfsetstrokecolor{currentstroke}%
\pgfsetstrokeopacity{0.000000}%
\pgfsetdash{}{0pt}%
\pgfpathmoveto{\pgfqpoint{4.167090in}{0.499444in}}%
\pgfpathlineto{\pgfqpoint{4.228476in}{0.499444in}}%
\pgfpathlineto{\pgfqpoint{4.228476in}{0.681583in}}%
\pgfpathlineto{\pgfqpoint{4.167090in}{0.681583in}}%
\pgfpathlineto{\pgfqpoint{4.167090in}{0.499444in}}%
\pgfpathclose%
\pgfusepath{fill}%
\end{pgfscope}%
\begin{pgfscope}%
\pgfsetbuttcap%
\pgfsetroundjoin%
\definecolor{currentfill}{rgb}{0.000000,0.000000,0.000000}%
\pgfsetfillcolor{currentfill}%
\pgfsetlinewidth{0.803000pt}%
\definecolor{currentstroke}{rgb}{0.000000,0.000000,0.000000}%
\pgfsetstrokecolor{currentstroke}%
\pgfsetdash{}{0pt}%
\pgfsys@defobject{currentmarker}{\pgfqpoint{0.000000in}{-0.048611in}}{\pgfqpoint{0.000000in}{0.000000in}}{%
\pgfpathmoveto{\pgfqpoint{0.000000in}{0.000000in}}%
\pgfpathlineto{\pgfqpoint{0.000000in}{-0.048611in}}%
\pgfusepath{stroke,fill}%
}%
\begin{pgfscope}%
\pgfsys@transformshift{0.483922in}{0.499444in}%
\pgfsys@useobject{currentmarker}{}%
\end{pgfscope}%
\end{pgfscope}%
\begin{pgfscope}%
\definecolor{textcolor}{rgb}{0.000000,0.000000,0.000000}%
\pgfsetstrokecolor{textcolor}%
\pgfsetfillcolor{textcolor}%
\pgftext[x=0.483922in,y=0.402222in,,top]{\color{textcolor}\rmfamily\fontsize{10.000000}{12.000000}\selectfont 0.0}%
\end{pgfscope}%
\begin{pgfscope}%
\pgfsetbuttcap%
\pgfsetroundjoin%
\definecolor{currentfill}{rgb}{0.000000,0.000000,0.000000}%
\pgfsetfillcolor{currentfill}%
\pgfsetlinewidth{0.803000pt}%
\definecolor{currentstroke}{rgb}{0.000000,0.000000,0.000000}%
\pgfsetstrokecolor{currentstroke}%
\pgfsetdash{}{0pt}%
\pgfsys@defobject{currentmarker}{\pgfqpoint{0.000000in}{-0.048611in}}{\pgfqpoint{0.000000in}{0.000000in}}{%
\pgfpathmoveto{\pgfqpoint{0.000000in}{0.000000in}}%
\pgfpathlineto{\pgfqpoint{0.000000in}{-0.048611in}}%
\pgfusepath{stroke,fill}%
}%
\begin{pgfscope}%
\pgfsys@transformshift{0.867585in}{0.499444in}%
\pgfsys@useobject{currentmarker}{}%
\end{pgfscope}%
\end{pgfscope}%
\begin{pgfscope}%
\definecolor{textcolor}{rgb}{0.000000,0.000000,0.000000}%
\pgfsetstrokecolor{textcolor}%
\pgfsetfillcolor{textcolor}%
\pgftext[x=0.867585in,y=0.402222in,,top]{\color{textcolor}\rmfamily\fontsize{10.000000}{12.000000}\selectfont 0.1}%
\end{pgfscope}%
\begin{pgfscope}%
\pgfsetbuttcap%
\pgfsetroundjoin%
\definecolor{currentfill}{rgb}{0.000000,0.000000,0.000000}%
\pgfsetfillcolor{currentfill}%
\pgfsetlinewidth{0.803000pt}%
\definecolor{currentstroke}{rgb}{0.000000,0.000000,0.000000}%
\pgfsetstrokecolor{currentstroke}%
\pgfsetdash{}{0pt}%
\pgfsys@defobject{currentmarker}{\pgfqpoint{0.000000in}{-0.048611in}}{\pgfqpoint{0.000000in}{0.000000in}}{%
\pgfpathmoveto{\pgfqpoint{0.000000in}{0.000000in}}%
\pgfpathlineto{\pgfqpoint{0.000000in}{-0.048611in}}%
\pgfusepath{stroke,fill}%
}%
\begin{pgfscope}%
\pgfsys@transformshift{1.251249in}{0.499444in}%
\pgfsys@useobject{currentmarker}{}%
\end{pgfscope}%
\end{pgfscope}%
\begin{pgfscope}%
\definecolor{textcolor}{rgb}{0.000000,0.000000,0.000000}%
\pgfsetstrokecolor{textcolor}%
\pgfsetfillcolor{textcolor}%
\pgftext[x=1.251249in,y=0.402222in,,top]{\color{textcolor}\rmfamily\fontsize{10.000000}{12.000000}\selectfont 0.2}%
\end{pgfscope}%
\begin{pgfscope}%
\pgfsetbuttcap%
\pgfsetroundjoin%
\definecolor{currentfill}{rgb}{0.000000,0.000000,0.000000}%
\pgfsetfillcolor{currentfill}%
\pgfsetlinewidth{0.803000pt}%
\definecolor{currentstroke}{rgb}{0.000000,0.000000,0.000000}%
\pgfsetstrokecolor{currentstroke}%
\pgfsetdash{}{0pt}%
\pgfsys@defobject{currentmarker}{\pgfqpoint{0.000000in}{-0.048611in}}{\pgfqpoint{0.000000in}{0.000000in}}{%
\pgfpathmoveto{\pgfqpoint{0.000000in}{0.000000in}}%
\pgfpathlineto{\pgfqpoint{0.000000in}{-0.048611in}}%
\pgfusepath{stroke,fill}%
}%
\begin{pgfscope}%
\pgfsys@transformshift{1.634912in}{0.499444in}%
\pgfsys@useobject{currentmarker}{}%
\end{pgfscope}%
\end{pgfscope}%
\begin{pgfscope}%
\definecolor{textcolor}{rgb}{0.000000,0.000000,0.000000}%
\pgfsetstrokecolor{textcolor}%
\pgfsetfillcolor{textcolor}%
\pgftext[x=1.634912in,y=0.402222in,,top]{\color{textcolor}\rmfamily\fontsize{10.000000}{12.000000}\selectfont 0.3}%
\end{pgfscope}%
\begin{pgfscope}%
\pgfsetbuttcap%
\pgfsetroundjoin%
\definecolor{currentfill}{rgb}{0.000000,0.000000,0.000000}%
\pgfsetfillcolor{currentfill}%
\pgfsetlinewidth{0.803000pt}%
\definecolor{currentstroke}{rgb}{0.000000,0.000000,0.000000}%
\pgfsetstrokecolor{currentstroke}%
\pgfsetdash{}{0pt}%
\pgfsys@defobject{currentmarker}{\pgfqpoint{0.000000in}{-0.048611in}}{\pgfqpoint{0.000000in}{0.000000in}}{%
\pgfpathmoveto{\pgfqpoint{0.000000in}{0.000000in}}%
\pgfpathlineto{\pgfqpoint{0.000000in}{-0.048611in}}%
\pgfusepath{stroke,fill}%
}%
\begin{pgfscope}%
\pgfsys@transformshift{2.018575in}{0.499444in}%
\pgfsys@useobject{currentmarker}{}%
\end{pgfscope}%
\end{pgfscope}%
\begin{pgfscope}%
\definecolor{textcolor}{rgb}{0.000000,0.000000,0.000000}%
\pgfsetstrokecolor{textcolor}%
\pgfsetfillcolor{textcolor}%
\pgftext[x=2.018575in,y=0.402222in,,top]{\color{textcolor}\rmfamily\fontsize{10.000000}{12.000000}\selectfont 0.4}%
\end{pgfscope}%
\begin{pgfscope}%
\pgfsetbuttcap%
\pgfsetroundjoin%
\definecolor{currentfill}{rgb}{0.000000,0.000000,0.000000}%
\pgfsetfillcolor{currentfill}%
\pgfsetlinewidth{0.803000pt}%
\definecolor{currentstroke}{rgb}{0.000000,0.000000,0.000000}%
\pgfsetstrokecolor{currentstroke}%
\pgfsetdash{}{0pt}%
\pgfsys@defobject{currentmarker}{\pgfqpoint{0.000000in}{-0.048611in}}{\pgfqpoint{0.000000in}{0.000000in}}{%
\pgfpathmoveto{\pgfqpoint{0.000000in}{0.000000in}}%
\pgfpathlineto{\pgfqpoint{0.000000in}{-0.048611in}}%
\pgfusepath{stroke,fill}%
}%
\begin{pgfscope}%
\pgfsys@transformshift{2.402239in}{0.499444in}%
\pgfsys@useobject{currentmarker}{}%
\end{pgfscope}%
\end{pgfscope}%
\begin{pgfscope}%
\definecolor{textcolor}{rgb}{0.000000,0.000000,0.000000}%
\pgfsetstrokecolor{textcolor}%
\pgfsetfillcolor{textcolor}%
\pgftext[x=2.402239in,y=0.402222in,,top]{\color{textcolor}\rmfamily\fontsize{10.000000}{12.000000}\selectfont 0.5}%
\end{pgfscope}%
\begin{pgfscope}%
\pgfsetbuttcap%
\pgfsetroundjoin%
\definecolor{currentfill}{rgb}{0.000000,0.000000,0.000000}%
\pgfsetfillcolor{currentfill}%
\pgfsetlinewidth{0.803000pt}%
\definecolor{currentstroke}{rgb}{0.000000,0.000000,0.000000}%
\pgfsetstrokecolor{currentstroke}%
\pgfsetdash{}{0pt}%
\pgfsys@defobject{currentmarker}{\pgfqpoint{0.000000in}{-0.048611in}}{\pgfqpoint{0.000000in}{0.000000in}}{%
\pgfpathmoveto{\pgfqpoint{0.000000in}{0.000000in}}%
\pgfpathlineto{\pgfqpoint{0.000000in}{-0.048611in}}%
\pgfusepath{stroke,fill}%
}%
\begin{pgfscope}%
\pgfsys@transformshift{2.785902in}{0.499444in}%
\pgfsys@useobject{currentmarker}{}%
\end{pgfscope}%
\end{pgfscope}%
\begin{pgfscope}%
\definecolor{textcolor}{rgb}{0.000000,0.000000,0.000000}%
\pgfsetstrokecolor{textcolor}%
\pgfsetfillcolor{textcolor}%
\pgftext[x=2.785902in,y=0.402222in,,top]{\color{textcolor}\rmfamily\fontsize{10.000000}{12.000000}\selectfont 0.6}%
\end{pgfscope}%
\begin{pgfscope}%
\pgfsetbuttcap%
\pgfsetroundjoin%
\definecolor{currentfill}{rgb}{0.000000,0.000000,0.000000}%
\pgfsetfillcolor{currentfill}%
\pgfsetlinewidth{0.803000pt}%
\definecolor{currentstroke}{rgb}{0.000000,0.000000,0.000000}%
\pgfsetstrokecolor{currentstroke}%
\pgfsetdash{}{0pt}%
\pgfsys@defobject{currentmarker}{\pgfqpoint{0.000000in}{-0.048611in}}{\pgfqpoint{0.000000in}{0.000000in}}{%
\pgfpathmoveto{\pgfqpoint{0.000000in}{0.000000in}}%
\pgfpathlineto{\pgfqpoint{0.000000in}{-0.048611in}}%
\pgfusepath{stroke,fill}%
}%
\begin{pgfscope}%
\pgfsys@transformshift{3.169566in}{0.499444in}%
\pgfsys@useobject{currentmarker}{}%
\end{pgfscope}%
\end{pgfscope}%
\begin{pgfscope}%
\definecolor{textcolor}{rgb}{0.000000,0.000000,0.000000}%
\pgfsetstrokecolor{textcolor}%
\pgfsetfillcolor{textcolor}%
\pgftext[x=3.169566in,y=0.402222in,,top]{\color{textcolor}\rmfamily\fontsize{10.000000}{12.000000}\selectfont 0.7}%
\end{pgfscope}%
\begin{pgfscope}%
\pgfsetbuttcap%
\pgfsetroundjoin%
\definecolor{currentfill}{rgb}{0.000000,0.000000,0.000000}%
\pgfsetfillcolor{currentfill}%
\pgfsetlinewidth{0.803000pt}%
\definecolor{currentstroke}{rgb}{0.000000,0.000000,0.000000}%
\pgfsetstrokecolor{currentstroke}%
\pgfsetdash{}{0pt}%
\pgfsys@defobject{currentmarker}{\pgfqpoint{0.000000in}{-0.048611in}}{\pgfqpoint{0.000000in}{0.000000in}}{%
\pgfpathmoveto{\pgfqpoint{0.000000in}{0.000000in}}%
\pgfpathlineto{\pgfqpoint{0.000000in}{-0.048611in}}%
\pgfusepath{stroke,fill}%
}%
\begin{pgfscope}%
\pgfsys@transformshift{3.553229in}{0.499444in}%
\pgfsys@useobject{currentmarker}{}%
\end{pgfscope}%
\end{pgfscope}%
\begin{pgfscope}%
\definecolor{textcolor}{rgb}{0.000000,0.000000,0.000000}%
\pgfsetstrokecolor{textcolor}%
\pgfsetfillcolor{textcolor}%
\pgftext[x=3.553229in,y=0.402222in,,top]{\color{textcolor}\rmfamily\fontsize{10.000000}{12.000000}\selectfont 0.8}%
\end{pgfscope}%
\begin{pgfscope}%
\pgfsetbuttcap%
\pgfsetroundjoin%
\definecolor{currentfill}{rgb}{0.000000,0.000000,0.000000}%
\pgfsetfillcolor{currentfill}%
\pgfsetlinewidth{0.803000pt}%
\definecolor{currentstroke}{rgb}{0.000000,0.000000,0.000000}%
\pgfsetstrokecolor{currentstroke}%
\pgfsetdash{}{0pt}%
\pgfsys@defobject{currentmarker}{\pgfqpoint{0.000000in}{-0.048611in}}{\pgfqpoint{0.000000in}{0.000000in}}{%
\pgfpathmoveto{\pgfqpoint{0.000000in}{0.000000in}}%
\pgfpathlineto{\pgfqpoint{0.000000in}{-0.048611in}}%
\pgfusepath{stroke,fill}%
}%
\begin{pgfscope}%
\pgfsys@transformshift{3.936892in}{0.499444in}%
\pgfsys@useobject{currentmarker}{}%
\end{pgfscope}%
\end{pgfscope}%
\begin{pgfscope}%
\definecolor{textcolor}{rgb}{0.000000,0.000000,0.000000}%
\pgfsetstrokecolor{textcolor}%
\pgfsetfillcolor{textcolor}%
\pgftext[x=3.936892in,y=0.402222in,,top]{\color{textcolor}\rmfamily\fontsize{10.000000}{12.000000}\selectfont 0.9}%
\end{pgfscope}%
\begin{pgfscope}%
\pgfsetbuttcap%
\pgfsetroundjoin%
\definecolor{currentfill}{rgb}{0.000000,0.000000,0.000000}%
\pgfsetfillcolor{currentfill}%
\pgfsetlinewidth{0.803000pt}%
\definecolor{currentstroke}{rgb}{0.000000,0.000000,0.000000}%
\pgfsetstrokecolor{currentstroke}%
\pgfsetdash{}{0pt}%
\pgfsys@defobject{currentmarker}{\pgfqpoint{0.000000in}{-0.048611in}}{\pgfqpoint{0.000000in}{0.000000in}}{%
\pgfpathmoveto{\pgfqpoint{0.000000in}{0.000000in}}%
\pgfpathlineto{\pgfqpoint{0.000000in}{-0.048611in}}%
\pgfusepath{stroke,fill}%
}%
\begin{pgfscope}%
\pgfsys@transformshift{4.320556in}{0.499444in}%
\pgfsys@useobject{currentmarker}{}%
\end{pgfscope}%
\end{pgfscope}%
\begin{pgfscope}%
\definecolor{textcolor}{rgb}{0.000000,0.000000,0.000000}%
\pgfsetstrokecolor{textcolor}%
\pgfsetfillcolor{textcolor}%
\pgftext[x=4.320556in,y=0.402222in,,top]{\color{textcolor}\rmfamily\fontsize{10.000000}{12.000000}\selectfont 1.0}%
\end{pgfscope}%
\begin{pgfscope}%
\definecolor{textcolor}{rgb}{0.000000,0.000000,0.000000}%
\pgfsetstrokecolor{textcolor}%
\pgfsetfillcolor{textcolor}%
\pgftext[x=2.383056in,y=0.223333in,,top]{\color{textcolor}\rmfamily\fontsize{10.000000}{12.000000}\selectfont \(\displaystyle p\)}%
\end{pgfscope}%
\begin{pgfscope}%
\pgfsetbuttcap%
\pgfsetroundjoin%
\definecolor{currentfill}{rgb}{0.000000,0.000000,0.000000}%
\pgfsetfillcolor{currentfill}%
\pgfsetlinewidth{0.803000pt}%
\definecolor{currentstroke}{rgb}{0.000000,0.000000,0.000000}%
\pgfsetstrokecolor{currentstroke}%
\pgfsetdash{}{0pt}%
\pgfsys@defobject{currentmarker}{\pgfqpoint{-0.048611in}{0.000000in}}{\pgfqpoint{-0.000000in}{0.000000in}}{%
\pgfpathmoveto{\pgfqpoint{-0.000000in}{0.000000in}}%
\pgfpathlineto{\pgfqpoint{-0.048611in}{0.000000in}}%
\pgfusepath{stroke,fill}%
}%
\begin{pgfscope}%
\pgfsys@transformshift{0.445556in}{0.499444in}%
\pgfsys@useobject{currentmarker}{}%
\end{pgfscope}%
\end{pgfscope}%
\begin{pgfscope}%
\definecolor{textcolor}{rgb}{0.000000,0.000000,0.000000}%
\pgfsetstrokecolor{textcolor}%
\pgfsetfillcolor{textcolor}%
\pgftext[x=0.278889in, y=0.451250in, left, base]{\color{textcolor}\rmfamily\fontsize{10.000000}{12.000000}\selectfont \(\displaystyle {0}\)}%
\end{pgfscope}%
\begin{pgfscope}%
\pgfsetbuttcap%
\pgfsetroundjoin%
\definecolor{currentfill}{rgb}{0.000000,0.000000,0.000000}%
\pgfsetfillcolor{currentfill}%
\pgfsetlinewidth{0.803000pt}%
\definecolor{currentstroke}{rgb}{0.000000,0.000000,0.000000}%
\pgfsetstrokecolor{currentstroke}%
\pgfsetdash{}{0pt}%
\pgfsys@defobject{currentmarker}{\pgfqpoint{-0.048611in}{0.000000in}}{\pgfqpoint{-0.000000in}{0.000000in}}{%
\pgfpathmoveto{\pgfqpoint{-0.000000in}{0.000000in}}%
\pgfpathlineto{\pgfqpoint{-0.048611in}{0.000000in}}%
\pgfusepath{stroke,fill}%
}%
\begin{pgfscope}%
\pgfsys@transformshift{0.445556in}{0.791601in}%
\pgfsys@useobject{currentmarker}{}%
\end{pgfscope}%
\end{pgfscope}%
\begin{pgfscope}%
\definecolor{textcolor}{rgb}{0.000000,0.000000,0.000000}%
\pgfsetstrokecolor{textcolor}%
\pgfsetfillcolor{textcolor}%
\pgftext[x=0.278889in, y=0.743407in, left, base]{\color{textcolor}\rmfamily\fontsize{10.000000}{12.000000}\selectfont \(\displaystyle {2}\)}%
\end{pgfscope}%
\begin{pgfscope}%
\pgfsetbuttcap%
\pgfsetroundjoin%
\definecolor{currentfill}{rgb}{0.000000,0.000000,0.000000}%
\pgfsetfillcolor{currentfill}%
\pgfsetlinewidth{0.803000pt}%
\definecolor{currentstroke}{rgb}{0.000000,0.000000,0.000000}%
\pgfsetstrokecolor{currentstroke}%
\pgfsetdash{}{0pt}%
\pgfsys@defobject{currentmarker}{\pgfqpoint{-0.048611in}{0.000000in}}{\pgfqpoint{-0.000000in}{0.000000in}}{%
\pgfpathmoveto{\pgfqpoint{-0.000000in}{0.000000in}}%
\pgfpathlineto{\pgfqpoint{-0.048611in}{0.000000in}}%
\pgfusepath{stroke,fill}%
}%
\begin{pgfscope}%
\pgfsys@transformshift{0.445556in}{1.083759in}%
\pgfsys@useobject{currentmarker}{}%
\end{pgfscope}%
\end{pgfscope}%
\begin{pgfscope}%
\definecolor{textcolor}{rgb}{0.000000,0.000000,0.000000}%
\pgfsetstrokecolor{textcolor}%
\pgfsetfillcolor{textcolor}%
\pgftext[x=0.278889in, y=1.035564in, left, base]{\color{textcolor}\rmfamily\fontsize{10.000000}{12.000000}\selectfont \(\displaystyle {4}\)}%
\end{pgfscope}%
\begin{pgfscope}%
\pgfsetbuttcap%
\pgfsetroundjoin%
\definecolor{currentfill}{rgb}{0.000000,0.000000,0.000000}%
\pgfsetfillcolor{currentfill}%
\pgfsetlinewidth{0.803000pt}%
\definecolor{currentstroke}{rgb}{0.000000,0.000000,0.000000}%
\pgfsetstrokecolor{currentstroke}%
\pgfsetdash{}{0pt}%
\pgfsys@defobject{currentmarker}{\pgfqpoint{-0.048611in}{0.000000in}}{\pgfqpoint{-0.000000in}{0.000000in}}{%
\pgfpathmoveto{\pgfqpoint{-0.000000in}{0.000000in}}%
\pgfpathlineto{\pgfqpoint{-0.048611in}{0.000000in}}%
\pgfusepath{stroke,fill}%
}%
\begin{pgfscope}%
\pgfsys@transformshift{0.445556in}{1.375916in}%
\pgfsys@useobject{currentmarker}{}%
\end{pgfscope}%
\end{pgfscope}%
\begin{pgfscope}%
\definecolor{textcolor}{rgb}{0.000000,0.000000,0.000000}%
\pgfsetstrokecolor{textcolor}%
\pgfsetfillcolor{textcolor}%
\pgftext[x=0.278889in, y=1.327722in, left, base]{\color{textcolor}\rmfamily\fontsize{10.000000}{12.000000}\selectfont \(\displaystyle {6}\)}%
\end{pgfscope}%
\begin{pgfscope}%
\definecolor{textcolor}{rgb}{0.000000,0.000000,0.000000}%
\pgfsetstrokecolor{textcolor}%
\pgfsetfillcolor{textcolor}%
\pgftext[x=0.223333in,y=1.076944in,,bottom,rotate=90.000000]{\color{textcolor}\rmfamily\fontsize{10.000000}{12.000000}\selectfont Percent of Data Set}%
\end{pgfscope}%
\begin{pgfscope}%
\pgfsetrectcap%
\pgfsetmiterjoin%
\pgfsetlinewidth{0.803000pt}%
\definecolor{currentstroke}{rgb}{0.000000,0.000000,0.000000}%
\pgfsetstrokecolor{currentstroke}%
\pgfsetdash{}{0pt}%
\pgfpathmoveto{\pgfqpoint{0.445556in}{0.499444in}}%
\pgfpathlineto{\pgfqpoint{0.445556in}{1.654444in}}%
\pgfusepath{stroke}%
\end{pgfscope}%
\begin{pgfscope}%
\pgfsetrectcap%
\pgfsetmiterjoin%
\pgfsetlinewidth{0.803000pt}%
\definecolor{currentstroke}{rgb}{0.000000,0.000000,0.000000}%
\pgfsetstrokecolor{currentstroke}%
\pgfsetdash{}{0pt}%
\pgfpathmoveto{\pgfqpoint{4.320556in}{0.499444in}}%
\pgfpathlineto{\pgfqpoint{4.320556in}{1.654444in}}%
\pgfusepath{stroke}%
\end{pgfscope}%
\begin{pgfscope}%
\pgfsetrectcap%
\pgfsetmiterjoin%
\pgfsetlinewidth{0.803000pt}%
\definecolor{currentstroke}{rgb}{0.000000,0.000000,0.000000}%
\pgfsetstrokecolor{currentstroke}%
\pgfsetdash{}{0pt}%
\pgfpathmoveto{\pgfqpoint{0.445556in}{0.499444in}}%
\pgfpathlineto{\pgfqpoint{4.320556in}{0.499444in}}%
\pgfusepath{stroke}%
\end{pgfscope}%
\begin{pgfscope}%
\pgfsetrectcap%
\pgfsetmiterjoin%
\pgfsetlinewidth{0.803000pt}%
\definecolor{currentstroke}{rgb}{0.000000,0.000000,0.000000}%
\pgfsetstrokecolor{currentstroke}%
\pgfsetdash{}{0pt}%
\pgfpathmoveto{\pgfqpoint{0.445556in}{1.654444in}}%
\pgfpathlineto{\pgfqpoint{4.320556in}{1.654444in}}%
\pgfusepath{stroke}%
\end{pgfscope}%
\begin{pgfscope}%
\pgfsetbuttcap%
\pgfsetmiterjoin%
\definecolor{currentfill}{rgb}{1.000000,1.000000,1.000000}%
\pgfsetfillcolor{currentfill}%
\pgfsetfillopacity{0.800000}%
\pgfsetlinewidth{1.003750pt}%
\definecolor{currentstroke}{rgb}{0.800000,0.800000,0.800000}%
\pgfsetstrokecolor{currentstroke}%
\pgfsetstrokeopacity{0.800000}%
\pgfsetdash{}{0pt}%
\pgfpathmoveto{\pgfqpoint{3.543611in}{1.154445in}}%
\pgfpathlineto{\pgfqpoint{4.223333in}{1.154445in}}%
\pgfpathquadraticcurveto{\pgfqpoint{4.251111in}{1.154445in}}{\pgfqpoint{4.251111in}{1.182222in}}%
\pgfpathlineto{\pgfqpoint{4.251111in}{1.557222in}}%
\pgfpathquadraticcurveto{\pgfqpoint{4.251111in}{1.585000in}}{\pgfqpoint{4.223333in}{1.585000in}}%
\pgfpathlineto{\pgfqpoint{3.543611in}{1.585000in}}%
\pgfpathquadraticcurveto{\pgfqpoint{3.515833in}{1.585000in}}{\pgfqpoint{3.515833in}{1.557222in}}%
\pgfpathlineto{\pgfqpoint{3.515833in}{1.182222in}}%
\pgfpathquadraticcurveto{\pgfqpoint{3.515833in}{1.154445in}}{\pgfqpoint{3.543611in}{1.154445in}}%
\pgfpathlineto{\pgfqpoint{3.543611in}{1.154445in}}%
\pgfpathclose%
\pgfusepath{stroke,fill}%
\end{pgfscope}%
\begin{pgfscope}%
\pgfsetbuttcap%
\pgfsetmiterjoin%
\pgfsetlinewidth{1.003750pt}%
\definecolor{currentstroke}{rgb}{0.000000,0.000000,0.000000}%
\pgfsetstrokecolor{currentstroke}%
\pgfsetdash{}{0pt}%
\pgfpathmoveto{\pgfqpoint{3.571389in}{1.432222in}}%
\pgfpathlineto{\pgfqpoint{3.849167in}{1.432222in}}%
\pgfpathlineto{\pgfqpoint{3.849167in}{1.529444in}}%
\pgfpathlineto{\pgfqpoint{3.571389in}{1.529444in}}%
\pgfpathlineto{\pgfqpoint{3.571389in}{1.432222in}}%
\pgfpathclose%
\pgfusepath{stroke}%
\end{pgfscope}%
\begin{pgfscope}%
\definecolor{textcolor}{rgb}{0.000000,0.000000,0.000000}%
\pgfsetstrokecolor{textcolor}%
\pgfsetfillcolor{textcolor}%
\pgftext[x=3.960278in,y=1.432222in,left,base]{\color{textcolor}\rmfamily\fontsize{10.000000}{12.000000}\selectfont Neg}%
\end{pgfscope}%
\begin{pgfscope}%
\pgfsetbuttcap%
\pgfsetmiterjoin%
\definecolor{currentfill}{rgb}{0.000000,0.000000,0.000000}%
\pgfsetfillcolor{currentfill}%
\pgfsetlinewidth{0.000000pt}%
\definecolor{currentstroke}{rgb}{0.000000,0.000000,0.000000}%
\pgfsetstrokecolor{currentstroke}%
\pgfsetstrokeopacity{0.000000}%
\pgfsetdash{}{0pt}%
\pgfpathmoveto{\pgfqpoint{3.571389in}{1.236944in}}%
\pgfpathlineto{\pgfqpoint{3.849167in}{1.236944in}}%
\pgfpathlineto{\pgfqpoint{3.849167in}{1.334167in}}%
\pgfpathlineto{\pgfqpoint{3.571389in}{1.334167in}}%
\pgfpathlineto{\pgfqpoint{3.571389in}{1.236944in}}%
\pgfpathclose%
\pgfusepath{fill}%
\end{pgfscope}%
\begin{pgfscope}%
\definecolor{textcolor}{rgb}{0.000000,0.000000,0.000000}%
\pgfsetstrokecolor{textcolor}%
\pgfsetfillcolor{textcolor}%
\pgftext[x=3.960278in,y=1.236944in,left,base]{\color{textcolor}\rmfamily\fontsize{10.000000}{12.000000}\selectfont Pos}%
\end{pgfscope}%
\end{pgfpicture}%
\makeatother%
\endgroup%
	
&
	\vskip 0pt
	\hfil ROC Curve
	
	%% Creator: Matplotlib, PGF backend
%%
%% To include the figure in your LaTeX document, write
%%   \input{<filename>.pgf}
%%
%% Make sure the required packages are loaded in your preamble
%%   \usepackage{pgf}
%%
%% Also ensure that all the required font packages are loaded; for instance,
%% the lmodern package is sometimes necessary when using math font.
%%   \usepackage{lmodern}
%%
%% Figures using additional raster images can only be included by \input if
%% they are in the same directory as the main LaTeX file. For loading figures
%% from other directories you can use the `import` package
%%   \usepackage{import}
%%
%% and then include the figures with
%%   \import{<path to file>}{<filename>.pgf}
%%
%% Matplotlib used the following preamble
%%   
%%   \usepackage{fontspec}
%%   \makeatletter\@ifpackageloaded{underscore}{}{\usepackage[strings]{underscore}}\makeatother
%%
\begingroup%
\makeatletter%
\begin{pgfpicture}%
\pgfpathrectangle{\pgfpointorigin}{\pgfqpoint{2.221861in}{1.754444in}}%
\pgfusepath{use as bounding box, clip}%
\begin{pgfscope}%
\pgfsetbuttcap%
\pgfsetmiterjoin%
\definecolor{currentfill}{rgb}{1.000000,1.000000,1.000000}%
\pgfsetfillcolor{currentfill}%
\pgfsetlinewidth{0.000000pt}%
\definecolor{currentstroke}{rgb}{1.000000,1.000000,1.000000}%
\pgfsetstrokecolor{currentstroke}%
\pgfsetdash{}{0pt}%
\pgfpathmoveto{\pgfqpoint{0.000000in}{0.000000in}}%
\pgfpathlineto{\pgfqpoint{2.221861in}{0.000000in}}%
\pgfpathlineto{\pgfqpoint{2.221861in}{1.754444in}}%
\pgfpathlineto{\pgfqpoint{0.000000in}{1.754444in}}%
\pgfpathlineto{\pgfqpoint{0.000000in}{0.000000in}}%
\pgfpathclose%
\pgfusepath{fill}%
\end{pgfscope}%
\begin{pgfscope}%
\pgfsetbuttcap%
\pgfsetmiterjoin%
\definecolor{currentfill}{rgb}{1.000000,1.000000,1.000000}%
\pgfsetfillcolor{currentfill}%
\pgfsetlinewidth{0.000000pt}%
\definecolor{currentstroke}{rgb}{0.000000,0.000000,0.000000}%
\pgfsetstrokecolor{currentstroke}%
\pgfsetstrokeopacity{0.000000}%
\pgfsetdash{}{0pt}%
\pgfpathmoveto{\pgfqpoint{0.553581in}{0.499444in}}%
\pgfpathlineto{\pgfqpoint{2.103581in}{0.499444in}}%
\pgfpathlineto{\pgfqpoint{2.103581in}{1.654444in}}%
\pgfpathlineto{\pgfqpoint{0.553581in}{1.654444in}}%
\pgfpathlineto{\pgfqpoint{0.553581in}{0.499444in}}%
\pgfpathclose%
\pgfusepath{fill}%
\end{pgfscope}%
\begin{pgfscope}%
\pgfsetbuttcap%
\pgfsetroundjoin%
\definecolor{currentfill}{rgb}{0.000000,0.000000,0.000000}%
\pgfsetfillcolor{currentfill}%
\pgfsetlinewidth{0.803000pt}%
\definecolor{currentstroke}{rgb}{0.000000,0.000000,0.000000}%
\pgfsetstrokecolor{currentstroke}%
\pgfsetdash{}{0pt}%
\pgfsys@defobject{currentmarker}{\pgfqpoint{0.000000in}{-0.048611in}}{\pgfqpoint{0.000000in}{0.000000in}}{%
\pgfpathmoveto{\pgfqpoint{0.000000in}{0.000000in}}%
\pgfpathlineto{\pgfqpoint{0.000000in}{-0.048611in}}%
\pgfusepath{stroke,fill}%
}%
\begin{pgfscope}%
\pgfsys@transformshift{0.624035in}{0.499444in}%
\pgfsys@useobject{currentmarker}{}%
\end{pgfscope}%
\end{pgfscope}%
\begin{pgfscope}%
\definecolor{textcolor}{rgb}{0.000000,0.000000,0.000000}%
\pgfsetstrokecolor{textcolor}%
\pgfsetfillcolor{textcolor}%
\pgftext[x=0.624035in,y=0.402222in,,top]{\color{textcolor}\rmfamily\fontsize{10.000000}{12.000000}\selectfont \(\displaystyle {0.0}\)}%
\end{pgfscope}%
\begin{pgfscope}%
\pgfsetbuttcap%
\pgfsetroundjoin%
\definecolor{currentfill}{rgb}{0.000000,0.000000,0.000000}%
\pgfsetfillcolor{currentfill}%
\pgfsetlinewidth{0.803000pt}%
\definecolor{currentstroke}{rgb}{0.000000,0.000000,0.000000}%
\pgfsetstrokecolor{currentstroke}%
\pgfsetdash{}{0pt}%
\pgfsys@defobject{currentmarker}{\pgfqpoint{0.000000in}{-0.048611in}}{\pgfqpoint{0.000000in}{0.000000in}}{%
\pgfpathmoveto{\pgfqpoint{0.000000in}{0.000000in}}%
\pgfpathlineto{\pgfqpoint{0.000000in}{-0.048611in}}%
\pgfusepath{stroke,fill}%
}%
\begin{pgfscope}%
\pgfsys@transformshift{1.328581in}{0.499444in}%
\pgfsys@useobject{currentmarker}{}%
\end{pgfscope}%
\end{pgfscope}%
\begin{pgfscope}%
\definecolor{textcolor}{rgb}{0.000000,0.000000,0.000000}%
\pgfsetstrokecolor{textcolor}%
\pgfsetfillcolor{textcolor}%
\pgftext[x=1.328581in,y=0.402222in,,top]{\color{textcolor}\rmfamily\fontsize{10.000000}{12.000000}\selectfont \(\displaystyle {0.5}\)}%
\end{pgfscope}%
\begin{pgfscope}%
\pgfsetbuttcap%
\pgfsetroundjoin%
\definecolor{currentfill}{rgb}{0.000000,0.000000,0.000000}%
\pgfsetfillcolor{currentfill}%
\pgfsetlinewidth{0.803000pt}%
\definecolor{currentstroke}{rgb}{0.000000,0.000000,0.000000}%
\pgfsetstrokecolor{currentstroke}%
\pgfsetdash{}{0pt}%
\pgfsys@defobject{currentmarker}{\pgfqpoint{0.000000in}{-0.048611in}}{\pgfqpoint{0.000000in}{0.000000in}}{%
\pgfpathmoveto{\pgfqpoint{0.000000in}{0.000000in}}%
\pgfpathlineto{\pgfqpoint{0.000000in}{-0.048611in}}%
\pgfusepath{stroke,fill}%
}%
\begin{pgfscope}%
\pgfsys@transformshift{2.033126in}{0.499444in}%
\pgfsys@useobject{currentmarker}{}%
\end{pgfscope}%
\end{pgfscope}%
\begin{pgfscope}%
\definecolor{textcolor}{rgb}{0.000000,0.000000,0.000000}%
\pgfsetstrokecolor{textcolor}%
\pgfsetfillcolor{textcolor}%
\pgftext[x=2.033126in,y=0.402222in,,top]{\color{textcolor}\rmfamily\fontsize{10.000000}{12.000000}\selectfont \(\displaystyle {1.0}\)}%
\end{pgfscope}%
\begin{pgfscope}%
\definecolor{textcolor}{rgb}{0.000000,0.000000,0.000000}%
\pgfsetstrokecolor{textcolor}%
\pgfsetfillcolor{textcolor}%
\pgftext[x=1.328581in,y=0.223333in,,top]{\color{textcolor}\rmfamily\fontsize{10.000000}{12.000000}\selectfont False positive rate}%
\end{pgfscope}%
\begin{pgfscope}%
\pgfsetbuttcap%
\pgfsetroundjoin%
\definecolor{currentfill}{rgb}{0.000000,0.000000,0.000000}%
\pgfsetfillcolor{currentfill}%
\pgfsetlinewidth{0.803000pt}%
\definecolor{currentstroke}{rgb}{0.000000,0.000000,0.000000}%
\pgfsetstrokecolor{currentstroke}%
\pgfsetdash{}{0pt}%
\pgfsys@defobject{currentmarker}{\pgfqpoint{-0.048611in}{0.000000in}}{\pgfqpoint{-0.000000in}{0.000000in}}{%
\pgfpathmoveto{\pgfqpoint{-0.000000in}{0.000000in}}%
\pgfpathlineto{\pgfqpoint{-0.048611in}{0.000000in}}%
\pgfusepath{stroke,fill}%
}%
\begin{pgfscope}%
\pgfsys@transformshift{0.553581in}{0.551944in}%
\pgfsys@useobject{currentmarker}{}%
\end{pgfscope}%
\end{pgfscope}%
\begin{pgfscope}%
\definecolor{textcolor}{rgb}{0.000000,0.000000,0.000000}%
\pgfsetstrokecolor{textcolor}%
\pgfsetfillcolor{textcolor}%
\pgftext[x=0.278889in, y=0.503750in, left, base]{\color{textcolor}\rmfamily\fontsize{10.000000}{12.000000}\selectfont \(\displaystyle {0.0}\)}%
\end{pgfscope}%
\begin{pgfscope}%
\pgfsetbuttcap%
\pgfsetroundjoin%
\definecolor{currentfill}{rgb}{0.000000,0.000000,0.000000}%
\pgfsetfillcolor{currentfill}%
\pgfsetlinewidth{0.803000pt}%
\definecolor{currentstroke}{rgb}{0.000000,0.000000,0.000000}%
\pgfsetstrokecolor{currentstroke}%
\pgfsetdash{}{0pt}%
\pgfsys@defobject{currentmarker}{\pgfqpoint{-0.048611in}{0.000000in}}{\pgfqpoint{-0.000000in}{0.000000in}}{%
\pgfpathmoveto{\pgfqpoint{-0.000000in}{0.000000in}}%
\pgfpathlineto{\pgfqpoint{-0.048611in}{0.000000in}}%
\pgfusepath{stroke,fill}%
}%
\begin{pgfscope}%
\pgfsys@transformshift{0.553581in}{1.076944in}%
\pgfsys@useobject{currentmarker}{}%
\end{pgfscope}%
\end{pgfscope}%
\begin{pgfscope}%
\definecolor{textcolor}{rgb}{0.000000,0.000000,0.000000}%
\pgfsetstrokecolor{textcolor}%
\pgfsetfillcolor{textcolor}%
\pgftext[x=0.278889in, y=1.028750in, left, base]{\color{textcolor}\rmfamily\fontsize{10.000000}{12.000000}\selectfont \(\displaystyle {0.5}\)}%
\end{pgfscope}%
\begin{pgfscope}%
\pgfsetbuttcap%
\pgfsetroundjoin%
\definecolor{currentfill}{rgb}{0.000000,0.000000,0.000000}%
\pgfsetfillcolor{currentfill}%
\pgfsetlinewidth{0.803000pt}%
\definecolor{currentstroke}{rgb}{0.000000,0.000000,0.000000}%
\pgfsetstrokecolor{currentstroke}%
\pgfsetdash{}{0pt}%
\pgfsys@defobject{currentmarker}{\pgfqpoint{-0.048611in}{0.000000in}}{\pgfqpoint{-0.000000in}{0.000000in}}{%
\pgfpathmoveto{\pgfqpoint{-0.000000in}{0.000000in}}%
\pgfpathlineto{\pgfqpoint{-0.048611in}{0.000000in}}%
\pgfusepath{stroke,fill}%
}%
\begin{pgfscope}%
\pgfsys@transformshift{0.553581in}{1.601944in}%
\pgfsys@useobject{currentmarker}{}%
\end{pgfscope}%
\end{pgfscope}%
\begin{pgfscope}%
\definecolor{textcolor}{rgb}{0.000000,0.000000,0.000000}%
\pgfsetstrokecolor{textcolor}%
\pgfsetfillcolor{textcolor}%
\pgftext[x=0.278889in, y=1.553750in, left, base]{\color{textcolor}\rmfamily\fontsize{10.000000}{12.000000}\selectfont \(\displaystyle {1.0}\)}%
\end{pgfscope}%
\begin{pgfscope}%
\definecolor{textcolor}{rgb}{0.000000,0.000000,0.000000}%
\pgfsetstrokecolor{textcolor}%
\pgfsetfillcolor{textcolor}%
\pgftext[x=0.223333in,y=1.076944in,,bottom,rotate=90.000000]{\color{textcolor}\rmfamily\fontsize{10.000000}{12.000000}\selectfont True positive rate}%
\end{pgfscope}%
\begin{pgfscope}%
\pgfpathrectangle{\pgfqpoint{0.553581in}{0.499444in}}{\pgfqpoint{1.550000in}{1.155000in}}%
\pgfusepath{clip}%
\pgfsetbuttcap%
\pgfsetroundjoin%
\pgfsetlinewidth{1.505625pt}%
\definecolor{currentstroke}{rgb}{0.000000,0.000000,0.000000}%
\pgfsetstrokecolor{currentstroke}%
\pgfsetdash{{5.550000pt}{2.400000pt}}{0.000000pt}%
\pgfpathmoveto{\pgfqpoint{0.624035in}{0.551944in}}%
\pgfpathlineto{\pgfqpoint{2.033126in}{1.601944in}}%
\pgfusepath{stroke}%
\end{pgfscope}%
\begin{pgfscope}%
\pgfpathrectangle{\pgfqpoint{0.553581in}{0.499444in}}{\pgfqpoint{1.550000in}{1.155000in}}%
\pgfusepath{clip}%
\pgfsetrectcap%
\pgfsetroundjoin%
\pgfsetlinewidth{1.505625pt}%
\definecolor{currentstroke}{rgb}{0.000000,0.000000,0.000000}%
\pgfsetstrokecolor{currentstroke}%
\pgfsetdash{}{0pt}%
\pgfpathmoveto{\pgfqpoint{0.624035in}{0.551944in}}%
\pgfpathlineto{\pgfqpoint{0.625087in}{1.081068in}}%
\pgfpathlineto{\pgfqpoint{0.626766in}{1.222951in}}%
\pgfpathlineto{\pgfqpoint{0.630063in}{1.346143in}}%
\pgfpathlineto{\pgfqpoint{0.630170in}{1.346861in}}%
\pgfpathlineto{\pgfqpoint{0.633316in}{1.414483in}}%
\pgfpathlineto{\pgfqpoint{0.637112in}{1.467192in}}%
\pgfpathlineto{\pgfqpoint{0.642134in}{1.509564in}}%
\pgfpathlineto{\pgfqpoint{0.648493in}{1.541240in}}%
\pgfpathlineto{\pgfqpoint{0.648510in}{1.541293in}}%
\pgfpathlineto{\pgfqpoint{0.655596in}{1.563550in}}%
\pgfpathlineto{\pgfqpoint{0.660310in}{1.572583in}}%
\pgfpathlineto{\pgfqpoint{0.670656in}{1.585341in}}%
\pgfpathlineto{\pgfqpoint{0.682708in}{1.593071in}}%
\pgfpathlineto{\pgfqpoint{0.683120in}{1.593230in}}%
\pgfpathlineto{\pgfqpoint{0.697859in}{1.597421in}}%
\pgfpathlineto{\pgfqpoint{0.715458in}{1.600108in}}%
\pgfpathlineto{\pgfqpoint{0.747388in}{1.601439in}}%
\pgfpathlineto{\pgfqpoint{0.871277in}{1.601931in}}%
\pgfpathlineto{\pgfqpoint{2.033126in}{1.601944in}}%
\pgfpathlineto{\pgfqpoint{2.033126in}{1.601944in}}%
\pgfusepath{stroke}%
\end{pgfscope}%
\begin{pgfscope}%
\pgfsetrectcap%
\pgfsetmiterjoin%
\pgfsetlinewidth{0.803000pt}%
\definecolor{currentstroke}{rgb}{0.000000,0.000000,0.000000}%
\pgfsetstrokecolor{currentstroke}%
\pgfsetdash{}{0pt}%
\pgfpathmoveto{\pgfqpoint{0.553581in}{0.499444in}}%
\pgfpathlineto{\pgfqpoint{0.553581in}{1.654444in}}%
\pgfusepath{stroke}%
\end{pgfscope}%
\begin{pgfscope}%
\pgfsetrectcap%
\pgfsetmiterjoin%
\pgfsetlinewidth{0.803000pt}%
\definecolor{currentstroke}{rgb}{0.000000,0.000000,0.000000}%
\pgfsetstrokecolor{currentstroke}%
\pgfsetdash{}{0pt}%
\pgfpathmoveto{\pgfqpoint{2.103581in}{0.499444in}}%
\pgfpathlineto{\pgfqpoint{2.103581in}{1.654444in}}%
\pgfusepath{stroke}%
\end{pgfscope}%
\begin{pgfscope}%
\pgfsetrectcap%
\pgfsetmiterjoin%
\pgfsetlinewidth{0.803000pt}%
\definecolor{currentstroke}{rgb}{0.000000,0.000000,0.000000}%
\pgfsetstrokecolor{currentstroke}%
\pgfsetdash{}{0pt}%
\pgfpathmoveto{\pgfqpoint{0.553581in}{0.499444in}}%
\pgfpathlineto{\pgfqpoint{2.103581in}{0.499444in}}%
\pgfusepath{stroke}%
\end{pgfscope}%
\begin{pgfscope}%
\pgfsetrectcap%
\pgfsetmiterjoin%
\pgfsetlinewidth{0.803000pt}%
\definecolor{currentstroke}{rgb}{0.000000,0.000000,0.000000}%
\pgfsetstrokecolor{currentstroke}%
\pgfsetdash{}{0pt}%
\pgfpathmoveto{\pgfqpoint{0.553581in}{1.654444in}}%
\pgfpathlineto{\pgfqpoint{2.103581in}{1.654444in}}%
\pgfusepath{stroke}%
\end{pgfscope}%
\begin{pgfscope}%
\pgfsetbuttcap%
\pgfsetmiterjoin%
\definecolor{currentfill}{rgb}{1.000000,1.000000,1.000000}%
\pgfsetfillcolor{currentfill}%
\pgfsetfillopacity{0.800000}%
\pgfsetlinewidth{1.003750pt}%
\definecolor{currentstroke}{rgb}{0.800000,0.800000,0.800000}%
\pgfsetstrokecolor{currentstroke}%
\pgfsetstrokeopacity{0.800000}%
\pgfsetdash{}{0pt}%
\pgfpathmoveto{\pgfqpoint{0.832747in}{1.349722in}}%
\pgfpathlineto{\pgfqpoint{2.006358in}{1.349722in}}%
\pgfpathquadraticcurveto{\pgfqpoint{2.034136in}{1.349722in}}{\pgfqpoint{2.034136in}{1.377500in}}%
\pgfpathlineto{\pgfqpoint{2.034136in}{1.557222in}}%
\pgfpathquadraticcurveto{\pgfqpoint{2.034136in}{1.585000in}}{\pgfqpoint{2.006358in}{1.585000in}}%
\pgfpathlineto{\pgfqpoint{0.832747in}{1.585000in}}%
\pgfpathquadraticcurveto{\pgfqpoint{0.804970in}{1.585000in}}{\pgfqpoint{0.804970in}{1.557222in}}%
\pgfpathlineto{\pgfqpoint{0.804970in}{1.377500in}}%
\pgfpathquadraticcurveto{\pgfqpoint{0.804970in}{1.349722in}}{\pgfqpoint{0.832747in}{1.349722in}}%
\pgfpathlineto{\pgfqpoint{0.832747in}{1.349722in}}%
\pgfpathclose%
\pgfusepath{stroke,fill}%
\end{pgfscope}%
\begin{pgfscope}%
\pgfsetrectcap%
\pgfsetroundjoin%
\pgfsetlinewidth{1.505625pt}%
\definecolor{currentstroke}{rgb}{0.000000,0.000000,0.000000}%
\pgfsetstrokecolor{currentstroke}%
\pgfsetdash{}{0pt}%
\pgfpathmoveto{\pgfqpoint{0.860525in}{1.480833in}}%
\pgfpathlineto{\pgfqpoint{0.999414in}{1.480833in}}%
\pgfpathlineto{\pgfqpoint{1.138303in}{1.480833in}}%
\pgfusepath{stroke}%
\end{pgfscope}%
\begin{pgfscope}%
\definecolor{textcolor}{rgb}{0.000000,0.000000,0.000000}%
\pgfsetstrokecolor{textcolor}%
\pgfsetfillcolor{textcolor}%
\pgftext[x=1.249414in,y=1.432222in,left,base]{\color{textcolor}\rmfamily\fontsize{10.000000}{12.000000}\selectfont AUC=0.996}%
\end{pgfscope}%
\end{pgfpicture}%
\makeatother%
\endgroup%

	
\end{tabular}
\end{comment}

Unfortunately, our test results do not look quite that nice.  They do not separate the two classes as well.  Some distributions are clustered to one side or in the middle.  Some models give the results in $p \in [0,1]$ rounded to two decimal places so that we cannot hope for a level of detail beyond that, and one algorithm, Bagging, gives $p$ rounded to only one decimal place.  

Let us look at some examples.  In all of them, AUC is in the range $[0.7,0.8]$, so the various models separate the positive and negative classes about equally well overall, with none being dramatically better or worse.  We will later show how we investigated which models do a better job in the ranges of interest.  

\

%
\verb|BRFC_5_Fold_alpha_0_5_Hard_Test|

\

This model does not separate the negative and positive classes as well as the ideal, giving a much lower AUC (area under the ROC curve).  These results are actually from the same model as the ideal above, but the ideal are the results on the training set and below on the test set, showing overfitting.  

In these results, the 100 most frequent values comprised 93\% of the results, meaning that, while there is some noise making the distribution look continuous, it is mostly discrete to two decimal places, so we cannot hope for fine detail in tuning the decision threshold.  

\noindent\begin{tabular}{@{\hspace{-6pt}}p{4.3in} @{\hspace{-6pt}}p{2.0in}}
	\vskip 0pt
	\hfil Raw Model Output
	
	%% Creator: Matplotlib, PGF backend
%%
%% To include the figure in your LaTeX document, write
%%   \input{<filename>.pgf}
%%
%% Make sure the required packages are loaded in your preamble
%%   \usepackage{pgf}
%%
%% Also ensure that all the required font packages are loaded; for instance,
%% the lmodern package is sometimes necessary when using math font.
%%   \usepackage{lmodern}
%%
%% Figures using additional raster images can only be included by \input if
%% they are in the same directory as the main LaTeX file. For loading figures
%% from other directories you can use the `import` package
%%   \usepackage{import}
%%
%% and then include the figures with
%%   \import{<path to file>}{<filename>.pgf}
%%
%% Matplotlib used the following preamble
%%   
%%   \usepackage{fontspec}
%%   \makeatletter\@ifpackageloaded{underscore}{}{\usepackage[strings]{underscore}}\makeatother
%%
\begingroup%
\makeatletter%
\begin{pgfpicture}%
\pgfpathrectangle{\pgfpointorigin}{\pgfqpoint{4.041081in}{1.654444in}}%
\pgfusepath{use as bounding box, clip}%
\begin{pgfscope}%
\pgfsetbuttcap%
\pgfsetmiterjoin%
\definecolor{currentfill}{rgb}{1.000000,1.000000,1.000000}%
\pgfsetfillcolor{currentfill}%
\pgfsetlinewidth{0.000000pt}%
\definecolor{currentstroke}{rgb}{1.000000,1.000000,1.000000}%
\pgfsetstrokecolor{currentstroke}%
\pgfsetdash{}{0pt}%
\pgfpathmoveto{\pgfqpoint{0.000000in}{0.000000in}}%
\pgfpathlineto{\pgfqpoint{4.041081in}{0.000000in}}%
\pgfpathlineto{\pgfqpoint{4.041081in}{1.654444in}}%
\pgfpathlineto{\pgfqpoint{0.000000in}{1.654444in}}%
\pgfpathlineto{\pgfqpoint{0.000000in}{0.000000in}}%
\pgfpathclose%
\pgfusepath{fill}%
\end{pgfscope}%
\begin{pgfscope}%
\pgfsetbuttcap%
\pgfsetmiterjoin%
\definecolor{currentfill}{rgb}{1.000000,1.000000,1.000000}%
\pgfsetfillcolor{currentfill}%
\pgfsetlinewidth{0.000000pt}%
\definecolor{currentstroke}{rgb}{0.000000,0.000000,0.000000}%
\pgfsetstrokecolor{currentstroke}%
\pgfsetstrokeopacity{0.000000}%
\pgfsetdash{}{0pt}%
\pgfpathmoveto{\pgfqpoint{0.503581in}{0.449444in}}%
\pgfpathlineto{\pgfqpoint{3.991081in}{0.449444in}}%
\pgfpathlineto{\pgfqpoint{3.991081in}{1.604444in}}%
\pgfpathlineto{\pgfqpoint{0.503581in}{1.604444in}}%
\pgfpathlineto{\pgfqpoint{0.503581in}{0.449444in}}%
\pgfpathclose%
\pgfusepath{fill}%
\end{pgfscope}%
\begin{pgfscope}%
\pgfpathrectangle{\pgfqpoint{0.503581in}{0.449444in}}{\pgfqpoint{3.487500in}{1.155000in}}%
\pgfusepath{clip}%
\pgfsetbuttcap%
\pgfsetmiterjoin%
\pgfsetlinewidth{1.003750pt}%
\definecolor{currentstroke}{rgb}{0.000000,0.000000,0.000000}%
\pgfsetstrokecolor{currentstroke}%
\pgfsetdash{}{0pt}%
\pgfpathmoveto{\pgfqpoint{0.598694in}{0.449444in}}%
\pgfpathlineto{\pgfqpoint{0.662103in}{0.449444in}}%
\pgfpathlineto{\pgfqpoint{0.662103in}{0.459762in}}%
\pgfpathlineto{\pgfqpoint{0.598694in}{0.459762in}}%
\pgfpathlineto{\pgfqpoint{0.598694in}{0.449444in}}%
\pgfpathclose%
\pgfusepath{stroke}%
\end{pgfscope}%
\begin{pgfscope}%
\pgfpathrectangle{\pgfqpoint{0.503581in}{0.449444in}}{\pgfqpoint{3.487500in}{1.155000in}}%
\pgfusepath{clip}%
\pgfsetbuttcap%
\pgfsetmiterjoin%
\pgfsetlinewidth{1.003750pt}%
\definecolor{currentstroke}{rgb}{0.000000,0.000000,0.000000}%
\pgfsetstrokecolor{currentstroke}%
\pgfsetdash{}{0pt}%
\pgfpathmoveto{\pgfqpoint{0.757217in}{0.449444in}}%
\pgfpathlineto{\pgfqpoint{0.820626in}{0.449444in}}%
\pgfpathlineto{\pgfqpoint{0.820626in}{0.605286in}}%
\pgfpathlineto{\pgfqpoint{0.757217in}{0.605286in}}%
\pgfpathlineto{\pgfqpoint{0.757217in}{0.449444in}}%
\pgfpathclose%
\pgfusepath{stroke}%
\end{pgfscope}%
\begin{pgfscope}%
\pgfpathrectangle{\pgfqpoint{0.503581in}{0.449444in}}{\pgfqpoint{3.487500in}{1.155000in}}%
\pgfusepath{clip}%
\pgfsetbuttcap%
\pgfsetmiterjoin%
\pgfsetlinewidth{1.003750pt}%
\definecolor{currentstroke}{rgb}{0.000000,0.000000,0.000000}%
\pgfsetstrokecolor{currentstroke}%
\pgfsetdash{}{0pt}%
\pgfpathmoveto{\pgfqpoint{0.915740in}{0.449444in}}%
\pgfpathlineto{\pgfqpoint{0.979149in}{0.449444in}}%
\pgfpathlineto{\pgfqpoint{0.979149in}{0.870445in}}%
\pgfpathlineto{\pgfqpoint{0.915740in}{0.870445in}}%
\pgfpathlineto{\pgfqpoint{0.915740in}{0.449444in}}%
\pgfpathclose%
\pgfusepath{stroke}%
\end{pgfscope}%
\begin{pgfscope}%
\pgfpathrectangle{\pgfqpoint{0.503581in}{0.449444in}}{\pgfqpoint{3.487500in}{1.155000in}}%
\pgfusepath{clip}%
\pgfsetbuttcap%
\pgfsetmiterjoin%
\pgfsetlinewidth{1.003750pt}%
\definecolor{currentstroke}{rgb}{0.000000,0.000000,0.000000}%
\pgfsetstrokecolor{currentstroke}%
\pgfsetdash{}{0pt}%
\pgfpathmoveto{\pgfqpoint{1.074263in}{0.449444in}}%
\pgfpathlineto{\pgfqpoint{1.137672in}{0.449444in}}%
\pgfpathlineto{\pgfqpoint{1.137672in}{1.129048in}}%
\pgfpathlineto{\pgfqpoint{1.074263in}{1.129048in}}%
\pgfpathlineto{\pgfqpoint{1.074263in}{0.449444in}}%
\pgfpathclose%
\pgfusepath{stroke}%
\end{pgfscope}%
\begin{pgfscope}%
\pgfpathrectangle{\pgfqpoint{0.503581in}{0.449444in}}{\pgfqpoint{3.487500in}{1.155000in}}%
\pgfusepath{clip}%
\pgfsetbuttcap%
\pgfsetmiterjoin%
\pgfsetlinewidth{1.003750pt}%
\definecolor{currentstroke}{rgb}{0.000000,0.000000,0.000000}%
\pgfsetstrokecolor{currentstroke}%
\pgfsetdash{}{0pt}%
\pgfpathmoveto{\pgfqpoint{1.232785in}{0.449444in}}%
\pgfpathlineto{\pgfqpoint{1.296194in}{0.449444in}}%
\pgfpathlineto{\pgfqpoint{1.296194in}{1.310646in}}%
\pgfpathlineto{\pgfqpoint{1.232785in}{1.310646in}}%
\pgfpathlineto{\pgfqpoint{1.232785in}{0.449444in}}%
\pgfpathclose%
\pgfusepath{stroke}%
\end{pgfscope}%
\begin{pgfscope}%
\pgfpathrectangle{\pgfqpoint{0.503581in}{0.449444in}}{\pgfqpoint{3.487500in}{1.155000in}}%
\pgfusepath{clip}%
\pgfsetbuttcap%
\pgfsetmiterjoin%
\pgfsetlinewidth{1.003750pt}%
\definecolor{currentstroke}{rgb}{0.000000,0.000000,0.000000}%
\pgfsetstrokecolor{currentstroke}%
\pgfsetdash{}{0pt}%
\pgfpathmoveto{\pgfqpoint{1.391308in}{0.449444in}}%
\pgfpathlineto{\pgfqpoint{1.454717in}{0.449444in}}%
\pgfpathlineto{\pgfqpoint{1.454717in}{1.447080in}}%
\pgfpathlineto{\pgfqpoint{1.391308in}{1.447080in}}%
\pgfpathlineto{\pgfqpoint{1.391308in}{0.449444in}}%
\pgfpathclose%
\pgfusepath{stroke}%
\end{pgfscope}%
\begin{pgfscope}%
\pgfpathrectangle{\pgfqpoint{0.503581in}{0.449444in}}{\pgfqpoint{3.487500in}{1.155000in}}%
\pgfusepath{clip}%
\pgfsetbuttcap%
\pgfsetmiterjoin%
\pgfsetlinewidth{1.003750pt}%
\definecolor{currentstroke}{rgb}{0.000000,0.000000,0.000000}%
\pgfsetstrokecolor{currentstroke}%
\pgfsetdash{}{0pt}%
\pgfpathmoveto{\pgfqpoint{1.549831in}{0.449444in}}%
\pgfpathlineto{\pgfqpoint{1.613240in}{0.449444in}}%
\pgfpathlineto{\pgfqpoint{1.613240in}{1.518699in}}%
\pgfpathlineto{\pgfqpoint{1.549831in}{1.518699in}}%
\pgfpathlineto{\pgfqpoint{1.549831in}{0.449444in}}%
\pgfpathclose%
\pgfusepath{stroke}%
\end{pgfscope}%
\begin{pgfscope}%
\pgfpathrectangle{\pgfqpoint{0.503581in}{0.449444in}}{\pgfqpoint{3.487500in}{1.155000in}}%
\pgfusepath{clip}%
\pgfsetbuttcap%
\pgfsetmiterjoin%
\pgfsetlinewidth{1.003750pt}%
\definecolor{currentstroke}{rgb}{0.000000,0.000000,0.000000}%
\pgfsetstrokecolor{currentstroke}%
\pgfsetdash{}{0pt}%
\pgfpathmoveto{\pgfqpoint{1.708353in}{0.449444in}}%
\pgfpathlineto{\pgfqpoint{1.771763in}{0.449444in}}%
\pgfpathlineto{\pgfqpoint{1.771763in}{1.549444in}}%
\pgfpathlineto{\pgfqpoint{1.708353in}{1.549444in}}%
\pgfpathlineto{\pgfqpoint{1.708353in}{0.449444in}}%
\pgfpathclose%
\pgfusepath{stroke}%
\end{pgfscope}%
\begin{pgfscope}%
\pgfpathrectangle{\pgfqpoint{0.503581in}{0.449444in}}{\pgfqpoint{3.487500in}{1.155000in}}%
\pgfusepath{clip}%
\pgfsetbuttcap%
\pgfsetmiterjoin%
\pgfsetlinewidth{1.003750pt}%
\definecolor{currentstroke}{rgb}{0.000000,0.000000,0.000000}%
\pgfsetstrokecolor{currentstroke}%
\pgfsetdash{}{0pt}%
\pgfpathmoveto{\pgfqpoint{1.866876in}{0.449444in}}%
\pgfpathlineto{\pgfqpoint{1.930285in}{0.449444in}}%
\pgfpathlineto{\pgfqpoint{1.930285in}{1.543284in}}%
\pgfpathlineto{\pgfqpoint{1.866876in}{1.543284in}}%
\pgfpathlineto{\pgfqpoint{1.866876in}{0.449444in}}%
\pgfpathclose%
\pgfusepath{stroke}%
\end{pgfscope}%
\begin{pgfscope}%
\pgfpathrectangle{\pgfqpoint{0.503581in}{0.449444in}}{\pgfqpoint{3.487500in}{1.155000in}}%
\pgfusepath{clip}%
\pgfsetbuttcap%
\pgfsetmiterjoin%
\pgfsetlinewidth{1.003750pt}%
\definecolor{currentstroke}{rgb}{0.000000,0.000000,0.000000}%
\pgfsetstrokecolor{currentstroke}%
\pgfsetdash{}{0pt}%
\pgfpathmoveto{\pgfqpoint{2.025399in}{0.449444in}}%
\pgfpathlineto{\pgfqpoint{2.088808in}{0.449444in}}%
\pgfpathlineto{\pgfqpoint{2.088808in}{1.481510in}}%
\pgfpathlineto{\pgfqpoint{2.025399in}{1.481510in}}%
\pgfpathlineto{\pgfqpoint{2.025399in}{0.449444in}}%
\pgfpathclose%
\pgfusepath{stroke}%
\end{pgfscope}%
\begin{pgfscope}%
\pgfpathrectangle{\pgfqpoint{0.503581in}{0.449444in}}{\pgfqpoint{3.487500in}{1.155000in}}%
\pgfusepath{clip}%
\pgfsetbuttcap%
\pgfsetmiterjoin%
\pgfsetlinewidth{1.003750pt}%
\definecolor{currentstroke}{rgb}{0.000000,0.000000,0.000000}%
\pgfsetstrokecolor{currentstroke}%
\pgfsetdash{}{0pt}%
\pgfpathmoveto{\pgfqpoint{2.183922in}{0.449444in}}%
\pgfpathlineto{\pgfqpoint{2.247331in}{0.449444in}}%
\pgfpathlineto{\pgfqpoint{2.247331in}{1.379524in}}%
\pgfpathlineto{\pgfqpoint{2.183922in}{1.379524in}}%
\pgfpathlineto{\pgfqpoint{2.183922in}{0.449444in}}%
\pgfpathclose%
\pgfusepath{stroke}%
\end{pgfscope}%
\begin{pgfscope}%
\pgfpathrectangle{\pgfqpoint{0.503581in}{0.449444in}}{\pgfqpoint{3.487500in}{1.155000in}}%
\pgfusepath{clip}%
\pgfsetbuttcap%
\pgfsetmiterjoin%
\pgfsetlinewidth{1.003750pt}%
\definecolor{currentstroke}{rgb}{0.000000,0.000000,0.000000}%
\pgfsetstrokecolor{currentstroke}%
\pgfsetdash{}{0pt}%
\pgfpathmoveto{\pgfqpoint{2.342444in}{0.449444in}}%
\pgfpathlineto{\pgfqpoint{2.405853in}{0.449444in}}%
\pgfpathlineto{\pgfqpoint{2.405853in}{1.263857in}}%
\pgfpathlineto{\pgfqpoint{2.342444in}{1.263857in}}%
\pgfpathlineto{\pgfqpoint{2.342444in}{0.449444in}}%
\pgfpathclose%
\pgfusepath{stroke}%
\end{pgfscope}%
\begin{pgfscope}%
\pgfpathrectangle{\pgfqpoint{0.503581in}{0.449444in}}{\pgfqpoint{3.487500in}{1.155000in}}%
\pgfusepath{clip}%
\pgfsetbuttcap%
\pgfsetmiterjoin%
\pgfsetlinewidth{1.003750pt}%
\definecolor{currentstroke}{rgb}{0.000000,0.000000,0.000000}%
\pgfsetstrokecolor{currentstroke}%
\pgfsetdash{}{0pt}%
\pgfpathmoveto{\pgfqpoint{2.500967in}{0.449444in}}%
\pgfpathlineto{\pgfqpoint{2.564376in}{0.449444in}}%
\pgfpathlineto{\pgfqpoint{2.564376in}{1.126912in}}%
\pgfpathlineto{\pgfqpoint{2.500967in}{1.126912in}}%
\pgfpathlineto{\pgfqpoint{2.500967in}{0.449444in}}%
\pgfpathclose%
\pgfusepath{stroke}%
\end{pgfscope}%
\begin{pgfscope}%
\pgfpathrectangle{\pgfqpoint{0.503581in}{0.449444in}}{\pgfqpoint{3.487500in}{1.155000in}}%
\pgfusepath{clip}%
\pgfsetbuttcap%
\pgfsetmiterjoin%
\pgfsetlinewidth{1.003750pt}%
\definecolor{currentstroke}{rgb}{0.000000,0.000000,0.000000}%
\pgfsetstrokecolor{currentstroke}%
\pgfsetdash{}{0pt}%
\pgfpathmoveto{\pgfqpoint{2.659490in}{0.449444in}}%
\pgfpathlineto{\pgfqpoint{2.722899in}{0.449444in}}%
\pgfpathlineto{\pgfqpoint{2.722899in}{0.976211in}}%
\pgfpathlineto{\pgfqpoint{2.659490in}{0.976211in}}%
\pgfpathlineto{\pgfqpoint{2.659490in}{0.449444in}}%
\pgfpathclose%
\pgfusepath{stroke}%
\end{pgfscope}%
\begin{pgfscope}%
\pgfpathrectangle{\pgfqpoint{0.503581in}{0.449444in}}{\pgfqpoint{3.487500in}{1.155000in}}%
\pgfusepath{clip}%
\pgfsetbuttcap%
\pgfsetmiterjoin%
\pgfsetlinewidth{1.003750pt}%
\definecolor{currentstroke}{rgb}{0.000000,0.000000,0.000000}%
\pgfsetstrokecolor{currentstroke}%
\pgfsetdash{}{0pt}%
\pgfpathmoveto{\pgfqpoint{2.818013in}{0.449444in}}%
\pgfpathlineto{\pgfqpoint{2.881422in}{0.449444in}}%
\pgfpathlineto{\pgfqpoint{2.881422in}{0.839738in}}%
\pgfpathlineto{\pgfqpoint{2.818013in}{0.839738in}}%
\pgfpathlineto{\pgfqpoint{2.818013in}{0.449444in}}%
\pgfpathclose%
\pgfusepath{stroke}%
\end{pgfscope}%
\begin{pgfscope}%
\pgfpathrectangle{\pgfqpoint{0.503581in}{0.449444in}}{\pgfqpoint{3.487500in}{1.155000in}}%
\pgfusepath{clip}%
\pgfsetbuttcap%
\pgfsetmiterjoin%
\pgfsetlinewidth{1.003750pt}%
\definecolor{currentstroke}{rgb}{0.000000,0.000000,0.000000}%
\pgfsetstrokecolor{currentstroke}%
\pgfsetdash{}{0pt}%
\pgfpathmoveto{\pgfqpoint{2.976535in}{0.449444in}}%
\pgfpathlineto{\pgfqpoint{3.039944in}{0.449444in}}%
\pgfpathlineto{\pgfqpoint{3.039944in}{0.723523in}}%
\pgfpathlineto{\pgfqpoint{2.976535in}{0.723523in}}%
\pgfpathlineto{\pgfqpoint{2.976535in}{0.449444in}}%
\pgfpathclose%
\pgfusepath{stroke}%
\end{pgfscope}%
\begin{pgfscope}%
\pgfpathrectangle{\pgfqpoint{0.503581in}{0.449444in}}{\pgfqpoint{3.487500in}{1.155000in}}%
\pgfusepath{clip}%
\pgfsetbuttcap%
\pgfsetmiterjoin%
\pgfsetlinewidth{1.003750pt}%
\definecolor{currentstroke}{rgb}{0.000000,0.000000,0.000000}%
\pgfsetstrokecolor{currentstroke}%
\pgfsetdash{}{0pt}%
\pgfpathmoveto{\pgfqpoint{3.135058in}{0.449444in}}%
\pgfpathlineto{\pgfqpoint{3.198467in}{0.449444in}}%
\pgfpathlineto{\pgfqpoint{3.198467in}{0.630683in}}%
\pgfpathlineto{\pgfqpoint{3.135058in}{0.630683in}}%
\pgfpathlineto{\pgfqpoint{3.135058in}{0.449444in}}%
\pgfpathclose%
\pgfusepath{stroke}%
\end{pgfscope}%
\begin{pgfscope}%
\pgfpathrectangle{\pgfqpoint{0.503581in}{0.449444in}}{\pgfqpoint{3.487500in}{1.155000in}}%
\pgfusepath{clip}%
\pgfsetbuttcap%
\pgfsetmiterjoin%
\pgfsetlinewidth{1.003750pt}%
\definecolor{currentstroke}{rgb}{0.000000,0.000000,0.000000}%
\pgfsetstrokecolor{currentstroke}%
\pgfsetdash{}{0pt}%
\pgfpathmoveto{\pgfqpoint{3.293581in}{0.449444in}}%
\pgfpathlineto{\pgfqpoint{3.356990in}{0.449444in}}%
\pgfpathlineto{\pgfqpoint{3.356990in}{0.564034in}}%
\pgfpathlineto{\pgfqpoint{3.293581in}{0.564034in}}%
\pgfpathlineto{\pgfqpoint{3.293581in}{0.449444in}}%
\pgfpathclose%
\pgfusepath{stroke}%
\end{pgfscope}%
\begin{pgfscope}%
\pgfpathrectangle{\pgfqpoint{0.503581in}{0.449444in}}{\pgfqpoint{3.487500in}{1.155000in}}%
\pgfusepath{clip}%
\pgfsetbuttcap%
\pgfsetmiterjoin%
\pgfsetlinewidth{1.003750pt}%
\definecolor{currentstroke}{rgb}{0.000000,0.000000,0.000000}%
\pgfsetstrokecolor{currentstroke}%
\pgfsetdash{}{0pt}%
\pgfpathmoveto{\pgfqpoint{3.452103in}{0.449444in}}%
\pgfpathlineto{\pgfqpoint{3.515513in}{0.449444in}}%
\pgfpathlineto{\pgfqpoint{3.515513in}{0.520118in}}%
\pgfpathlineto{\pgfqpoint{3.452103in}{0.520118in}}%
\pgfpathlineto{\pgfqpoint{3.452103in}{0.449444in}}%
\pgfpathclose%
\pgfusepath{stroke}%
\end{pgfscope}%
\begin{pgfscope}%
\pgfpathrectangle{\pgfqpoint{0.503581in}{0.449444in}}{\pgfqpoint{3.487500in}{1.155000in}}%
\pgfusepath{clip}%
\pgfsetbuttcap%
\pgfsetmiterjoin%
\pgfsetlinewidth{1.003750pt}%
\definecolor{currentstroke}{rgb}{0.000000,0.000000,0.000000}%
\pgfsetstrokecolor{currentstroke}%
\pgfsetdash{}{0pt}%
\pgfpathmoveto{\pgfqpoint{3.610626in}{0.449444in}}%
\pgfpathlineto{\pgfqpoint{3.674035in}{0.449444in}}%
\pgfpathlineto{\pgfqpoint{3.674035in}{0.485499in}}%
\pgfpathlineto{\pgfqpoint{3.610626in}{0.485499in}}%
\pgfpathlineto{\pgfqpoint{3.610626in}{0.449444in}}%
\pgfpathclose%
\pgfusepath{stroke}%
\end{pgfscope}%
\begin{pgfscope}%
\pgfpathrectangle{\pgfqpoint{0.503581in}{0.449444in}}{\pgfqpoint{3.487500in}{1.155000in}}%
\pgfusepath{clip}%
\pgfsetbuttcap%
\pgfsetmiterjoin%
\pgfsetlinewidth{1.003750pt}%
\definecolor{currentstroke}{rgb}{0.000000,0.000000,0.000000}%
\pgfsetstrokecolor{currentstroke}%
\pgfsetdash{}{0pt}%
\pgfpathmoveto{\pgfqpoint{3.769149in}{0.449444in}}%
\pgfpathlineto{\pgfqpoint{3.832558in}{0.449444in}}%
\pgfpathlineto{\pgfqpoint{3.832558in}{0.457097in}}%
\pgfpathlineto{\pgfqpoint{3.769149in}{0.457097in}}%
\pgfpathlineto{\pgfqpoint{3.769149in}{0.449444in}}%
\pgfpathclose%
\pgfusepath{stroke}%
\end{pgfscope}%
\begin{pgfscope}%
\pgfpathrectangle{\pgfqpoint{0.503581in}{0.449444in}}{\pgfqpoint{3.487500in}{1.155000in}}%
\pgfusepath{clip}%
\pgfsetbuttcap%
\pgfsetmiterjoin%
\definecolor{currentfill}{rgb}{0.000000,0.000000,0.000000}%
\pgfsetfillcolor{currentfill}%
\pgfsetlinewidth{0.000000pt}%
\definecolor{currentstroke}{rgb}{0.000000,0.000000,0.000000}%
\pgfsetstrokecolor{currentstroke}%
\pgfsetstrokeopacity{0.000000}%
\pgfsetdash{}{0pt}%
\pgfpathmoveto{\pgfqpoint{0.662103in}{0.449444in}}%
\pgfpathlineto{\pgfqpoint{0.725513in}{0.449444in}}%
\pgfpathlineto{\pgfqpoint{0.725513in}{0.449444in}}%
\pgfpathlineto{\pgfqpoint{0.662103in}{0.449444in}}%
\pgfpathlineto{\pgfqpoint{0.662103in}{0.449444in}}%
\pgfpathclose%
\pgfusepath{fill}%
\end{pgfscope}%
\begin{pgfscope}%
\pgfpathrectangle{\pgfqpoint{0.503581in}{0.449444in}}{\pgfqpoint{3.487500in}{1.155000in}}%
\pgfusepath{clip}%
\pgfsetbuttcap%
\pgfsetmiterjoin%
\definecolor{currentfill}{rgb}{0.000000,0.000000,0.000000}%
\pgfsetfillcolor{currentfill}%
\pgfsetlinewidth{0.000000pt}%
\definecolor{currentstroke}{rgb}{0.000000,0.000000,0.000000}%
\pgfsetstrokecolor{currentstroke}%
\pgfsetstrokeopacity{0.000000}%
\pgfsetdash{}{0pt}%
\pgfpathmoveto{\pgfqpoint{0.820626in}{0.449444in}}%
\pgfpathlineto{\pgfqpoint{0.884035in}{0.449444in}}%
\pgfpathlineto{\pgfqpoint{0.884035in}{0.450880in}}%
\pgfpathlineto{\pgfqpoint{0.820626in}{0.450880in}}%
\pgfpathlineto{\pgfqpoint{0.820626in}{0.449444in}}%
\pgfpathclose%
\pgfusepath{fill}%
\end{pgfscope}%
\begin{pgfscope}%
\pgfpathrectangle{\pgfqpoint{0.503581in}{0.449444in}}{\pgfqpoint{3.487500in}{1.155000in}}%
\pgfusepath{clip}%
\pgfsetbuttcap%
\pgfsetmiterjoin%
\definecolor{currentfill}{rgb}{0.000000,0.000000,0.000000}%
\pgfsetfillcolor{currentfill}%
\pgfsetlinewidth{0.000000pt}%
\definecolor{currentstroke}{rgb}{0.000000,0.000000,0.000000}%
\pgfsetstrokecolor{currentstroke}%
\pgfsetstrokeopacity{0.000000}%
\pgfsetdash{}{0pt}%
\pgfpathmoveto{\pgfqpoint{0.979149in}{0.449444in}}%
\pgfpathlineto{\pgfqpoint{1.042558in}{0.449444in}}%
\pgfpathlineto{\pgfqpoint{1.042558in}{0.454546in}}%
\pgfpathlineto{\pgfqpoint{0.979149in}{0.454546in}}%
\pgfpathlineto{\pgfqpoint{0.979149in}{0.449444in}}%
\pgfpathclose%
\pgfusepath{fill}%
\end{pgfscope}%
\begin{pgfscope}%
\pgfpathrectangle{\pgfqpoint{0.503581in}{0.449444in}}{\pgfqpoint{3.487500in}{1.155000in}}%
\pgfusepath{clip}%
\pgfsetbuttcap%
\pgfsetmiterjoin%
\definecolor{currentfill}{rgb}{0.000000,0.000000,0.000000}%
\pgfsetfillcolor{currentfill}%
\pgfsetlinewidth{0.000000pt}%
\definecolor{currentstroke}{rgb}{0.000000,0.000000,0.000000}%
\pgfsetstrokecolor{currentstroke}%
\pgfsetstrokeopacity{0.000000}%
\pgfsetdash{}{0pt}%
\pgfpathmoveto{\pgfqpoint{1.137672in}{0.449444in}}%
\pgfpathlineto{\pgfqpoint{1.201081in}{0.449444in}}%
\pgfpathlineto{\pgfqpoint{1.201081in}{0.461236in}}%
\pgfpathlineto{\pgfqpoint{1.137672in}{0.461236in}}%
\pgfpathlineto{\pgfqpoint{1.137672in}{0.449444in}}%
\pgfpathclose%
\pgfusepath{fill}%
\end{pgfscope}%
\begin{pgfscope}%
\pgfpathrectangle{\pgfqpoint{0.503581in}{0.449444in}}{\pgfqpoint{3.487500in}{1.155000in}}%
\pgfusepath{clip}%
\pgfsetbuttcap%
\pgfsetmiterjoin%
\definecolor{currentfill}{rgb}{0.000000,0.000000,0.000000}%
\pgfsetfillcolor{currentfill}%
\pgfsetlinewidth{0.000000pt}%
\definecolor{currentstroke}{rgb}{0.000000,0.000000,0.000000}%
\pgfsetstrokecolor{currentstroke}%
\pgfsetstrokeopacity{0.000000}%
\pgfsetdash{}{0pt}%
\pgfpathmoveto{\pgfqpoint{1.296194in}{0.449444in}}%
\pgfpathlineto{\pgfqpoint{1.359603in}{0.449444in}}%
\pgfpathlineto{\pgfqpoint{1.359603in}{0.472555in}}%
\pgfpathlineto{\pgfqpoint{1.296194in}{0.472555in}}%
\pgfpathlineto{\pgfqpoint{1.296194in}{0.449444in}}%
\pgfpathclose%
\pgfusepath{fill}%
\end{pgfscope}%
\begin{pgfscope}%
\pgfpathrectangle{\pgfqpoint{0.503581in}{0.449444in}}{\pgfqpoint{3.487500in}{1.155000in}}%
\pgfusepath{clip}%
\pgfsetbuttcap%
\pgfsetmiterjoin%
\definecolor{currentfill}{rgb}{0.000000,0.000000,0.000000}%
\pgfsetfillcolor{currentfill}%
\pgfsetlinewidth{0.000000pt}%
\definecolor{currentstroke}{rgb}{0.000000,0.000000,0.000000}%
\pgfsetstrokecolor{currentstroke}%
\pgfsetstrokeopacity{0.000000}%
\pgfsetdash{}{0pt}%
\pgfpathmoveto{\pgfqpoint{1.454717in}{0.449444in}}%
\pgfpathlineto{\pgfqpoint{1.518126in}{0.449444in}}%
\pgfpathlineto{\pgfqpoint{1.518126in}{0.487049in}}%
\pgfpathlineto{\pgfqpoint{1.454717in}{0.487049in}}%
\pgfpathlineto{\pgfqpoint{1.454717in}{0.449444in}}%
\pgfpathclose%
\pgfusepath{fill}%
\end{pgfscope}%
\begin{pgfscope}%
\pgfpathrectangle{\pgfqpoint{0.503581in}{0.449444in}}{\pgfqpoint{3.487500in}{1.155000in}}%
\pgfusepath{clip}%
\pgfsetbuttcap%
\pgfsetmiterjoin%
\definecolor{currentfill}{rgb}{0.000000,0.000000,0.000000}%
\pgfsetfillcolor{currentfill}%
\pgfsetlinewidth{0.000000pt}%
\definecolor{currentstroke}{rgb}{0.000000,0.000000,0.000000}%
\pgfsetstrokecolor{currentstroke}%
\pgfsetstrokeopacity{0.000000}%
\pgfsetdash{}{0pt}%
\pgfpathmoveto{\pgfqpoint{1.613240in}{0.449444in}}%
\pgfpathlineto{\pgfqpoint{1.676649in}{0.449444in}}%
\pgfpathlineto{\pgfqpoint{1.676649in}{0.501713in}}%
\pgfpathlineto{\pgfqpoint{1.613240in}{0.501713in}}%
\pgfpathlineto{\pgfqpoint{1.613240in}{0.449444in}}%
\pgfpathclose%
\pgfusepath{fill}%
\end{pgfscope}%
\begin{pgfscope}%
\pgfpathrectangle{\pgfqpoint{0.503581in}{0.449444in}}{\pgfqpoint{3.487500in}{1.155000in}}%
\pgfusepath{clip}%
\pgfsetbuttcap%
\pgfsetmiterjoin%
\definecolor{currentfill}{rgb}{0.000000,0.000000,0.000000}%
\pgfsetfillcolor{currentfill}%
\pgfsetlinewidth{0.000000pt}%
\definecolor{currentstroke}{rgb}{0.000000,0.000000,0.000000}%
\pgfsetstrokecolor{currentstroke}%
\pgfsetstrokeopacity{0.000000}%
\pgfsetdash{}{0pt}%
\pgfpathmoveto{\pgfqpoint{1.771763in}{0.449444in}}%
\pgfpathlineto{\pgfqpoint{1.835172in}{0.449444in}}%
\pgfpathlineto{\pgfqpoint{1.835172in}{0.523992in}}%
\pgfpathlineto{\pgfqpoint{1.771763in}{0.523992in}}%
\pgfpathlineto{\pgfqpoint{1.771763in}{0.449444in}}%
\pgfpathclose%
\pgfusepath{fill}%
\end{pgfscope}%
\begin{pgfscope}%
\pgfpathrectangle{\pgfqpoint{0.503581in}{0.449444in}}{\pgfqpoint{3.487500in}{1.155000in}}%
\pgfusepath{clip}%
\pgfsetbuttcap%
\pgfsetmiterjoin%
\definecolor{currentfill}{rgb}{0.000000,0.000000,0.000000}%
\pgfsetfillcolor{currentfill}%
\pgfsetlinewidth{0.000000pt}%
\definecolor{currentstroke}{rgb}{0.000000,0.000000,0.000000}%
\pgfsetstrokecolor{currentstroke}%
\pgfsetstrokeopacity{0.000000}%
\pgfsetdash{}{0pt}%
\pgfpathmoveto{\pgfqpoint{1.930285in}{0.449444in}}%
\pgfpathlineto{\pgfqpoint{1.993694in}{0.449444in}}%
\pgfpathlineto{\pgfqpoint{1.993694in}{0.547859in}}%
\pgfpathlineto{\pgfqpoint{1.930285in}{0.547859in}}%
\pgfpathlineto{\pgfqpoint{1.930285in}{0.449444in}}%
\pgfpathclose%
\pgfusepath{fill}%
\end{pgfscope}%
\begin{pgfscope}%
\pgfpathrectangle{\pgfqpoint{0.503581in}{0.449444in}}{\pgfqpoint{3.487500in}{1.155000in}}%
\pgfusepath{clip}%
\pgfsetbuttcap%
\pgfsetmiterjoin%
\definecolor{currentfill}{rgb}{0.000000,0.000000,0.000000}%
\pgfsetfillcolor{currentfill}%
\pgfsetlinewidth{0.000000pt}%
\definecolor{currentstroke}{rgb}{0.000000,0.000000,0.000000}%
\pgfsetstrokecolor{currentstroke}%
\pgfsetstrokeopacity{0.000000}%
\pgfsetdash{}{0pt}%
\pgfpathmoveto{\pgfqpoint{2.088808in}{0.449444in}}%
\pgfpathlineto{\pgfqpoint{2.152217in}{0.449444in}}%
\pgfpathlineto{\pgfqpoint{2.152217in}{0.571725in}}%
\pgfpathlineto{\pgfqpoint{2.088808in}{0.571725in}}%
\pgfpathlineto{\pgfqpoint{2.088808in}{0.449444in}}%
\pgfpathclose%
\pgfusepath{fill}%
\end{pgfscope}%
\begin{pgfscope}%
\pgfpathrectangle{\pgfqpoint{0.503581in}{0.449444in}}{\pgfqpoint{3.487500in}{1.155000in}}%
\pgfusepath{clip}%
\pgfsetbuttcap%
\pgfsetmiterjoin%
\definecolor{currentfill}{rgb}{0.000000,0.000000,0.000000}%
\pgfsetfillcolor{currentfill}%
\pgfsetlinewidth{0.000000pt}%
\definecolor{currentstroke}{rgb}{0.000000,0.000000,0.000000}%
\pgfsetstrokecolor{currentstroke}%
\pgfsetstrokeopacity{0.000000}%
\pgfsetdash{}{0pt}%
\pgfpathmoveto{\pgfqpoint{2.247331in}{0.449444in}}%
\pgfpathlineto{\pgfqpoint{2.310740in}{0.449444in}}%
\pgfpathlineto{\pgfqpoint{2.310740in}{0.593438in}}%
\pgfpathlineto{\pgfqpoint{2.247331in}{0.593438in}}%
\pgfpathlineto{\pgfqpoint{2.247331in}{0.449444in}}%
\pgfpathclose%
\pgfusepath{fill}%
\end{pgfscope}%
\begin{pgfscope}%
\pgfpathrectangle{\pgfqpoint{0.503581in}{0.449444in}}{\pgfqpoint{3.487500in}{1.155000in}}%
\pgfusepath{clip}%
\pgfsetbuttcap%
\pgfsetmiterjoin%
\definecolor{currentfill}{rgb}{0.000000,0.000000,0.000000}%
\pgfsetfillcolor{currentfill}%
\pgfsetlinewidth{0.000000pt}%
\definecolor{currentstroke}{rgb}{0.000000,0.000000,0.000000}%
\pgfsetstrokecolor{currentstroke}%
\pgfsetstrokeopacity{0.000000}%
\pgfsetdash{}{0pt}%
\pgfpathmoveto{\pgfqpoint{2.405853in}{0.449444in}}%
\pgfpathlineto{\pgfqpoint{2.469263in}{0.449444in}}%
\pgfpathlineto{\pgfqpoint{2.469263in}{0.612939in}}%
\pgfpathlineto{\pgfqpoint{2.405853in}{0.612939in}}%
\pgfpathlineto{\pgfqpoint{2.405853in}{0.449444in}}%
\pgfpathclose%
\pgfusepath{fill}%
\end{pgfscope}%
\begin{pgfscope}%
\pgfpathrectangle{\pgfqpoint{0.503581in}{0.449444in}}{\pgfqpoint{3.487500in}{1.155000in}}%
\pgfusepath{clip}%
\pgfsetbuttcap%
\pgfsetmiterjoin%
\definecolor{currentfill}{rgb}{0.000000,0.000000,0.000000}%
\pgfsetfillcolor{currentfill}%
\pgfsetlinewidth{0.000000pt}%
\definecolor{currentstroke}{rgb}{0.000000,0.000000,0.000000}%
\pgfsetstrokecolor{currentstroke}%
\pgfsetstrokeopacity{0.000000}%
\pgfsetdash{}{0pt}%
\pgfpathmoveto{\pgfqpoint{2.564376in}{0.449444in}}%
\pgfpathlineto{\pgfqpoint{2.627785in}{0.449444in}}%
\pgfpathlineto{\pgfqpoint{2.627785in}{0.629209in}}%
\pgfpathlineto{\pgfqpoint{2.564376in}{0.629209in}}%
\pgfpathlineto{\pgfqpoint{2.564376in}{0.449444in}}%
\pgfpathclose%
\pgfusepath{fill}%
\end{pgfscope}%
\begin{pgfscope}%
\pgfpathrectangle{\pgfqpoint{0.503581in}{0.449444in}}{\pgfqpoint{3.487500in}{1.155000in}}%
\pgfusepath{clip}%
\pgfsetbuttcap%
\pgfsetmiterjoin%
\definecolor{currentfill}{rgb}{0.000000,0.000000,0.000000}%
\pgfsetfillcolor{currentfill}%
\pgfsetlinewidth{0.000000pt}%
\definecolor{currentstroke}{rgb}{0.000000,0.000000,0.000000}%
\pgfsetstrokecolor{currentstroke}%
\pgfsetstrokeopacity{0.000000}%
\pgfsetdash{}{0pt}%
\pgfpathmoveto{\pgfqpoint{2.722899in}{0.449444in}}%
\pgfpathlineto{\pgfqpoint{2.786308in}{0.449444in}}%
\pgfpathlineto{\pgfqpoint{2.786308in}{0.636484in}}%
\pgfpathlineto{\pgfqpoint{2.722899in}{0.636484in}}%
\pgfpathlineto{\pgfqpoint{2.722899in}{0.449444in}}%
\pgfpathclose%
\pgfusepath{fill}%
\end{pgfscope}%
\begin{pgfscope}%
\pgfpathrectangle{\pgfqpoint{0.503581in}{0.449444in}}{\pgfqpoint{3.487500in}{1.155000in}}%
\pgfusepath{clip}%
\pgfsetbuttcap%
\pgfsetmiterjoin%
\definecolor{currentfill}{rgb}{0.000000,0.000000,0.000000}%
\pgfsetfillcolor{currentfill}%
\pgfsetlinewidth{0.000000pt}%
\definecolor{currentstroke}{rgb}{0.000000,0.000000,0.000000}%
\pgfsetstrokecolor{currentstroke}%
\pgfsetstrokeopacity{0.000000}%
\pgfsetdash{}{0pt}%
\pgfpathmoveto{\pgfqpoint{2.881422in}{0.449444in}}%
\pgfpathlineto{\pgfqpoint{2.944831in}{0.449444in}}%
\pgfpathlineto{\pgfqpoint{2.944831in}{0.634273in}}%
\pgfpathlineto{\pgfqpoint{2.881422in}{0.634273in}}%
\pgfpathlineto{\pgfqpoint{2.881422in}{0.449444in}}%
\pgfpathclose%
\pgfusepath{fill}%
\end{pgfscope}%
\begin{pgfscope}%
\pgfpathrectangle{\pgfqpoint{0.503581in}{0.449444in}}{\pgfqpoint{3.487500in}{1.155000in}}%
\pgfusepath{clip}%
\pgfsetbuttcap%
\pgfsetmiterjoin%
\definecolor{currentfill}{rgb}{0.000000,0.000000,0.000000}%
\pgfsetfillcolor{currentfill}%
\pgfsetlinewidth{0.000000pt}%
\definecolor{currentstroke}{rgb}{0.000000,0.000000,0.000000}%
\pgfsetstrokecolor{currentstroke}%
\pgfsetstrokeopacity{0.000000}%
\pgfsetdash{}{0pt}%
\pgfpathmoveto{\pgfqpoint{3.039944in}{0.449444in}}%
\pgfpathlineto{\pgfqpoint{3.103353in}{0.449444in}}%
\pgfpathlineto{\pgfqpoint{3.103353in}{0.623880in}}%
\pgfpathlineto{\pgfqpoint{3.039944in}{0.623880in}}%
\pgfpathlineto{\pgfqpoint{3.039944in}{0.449444in}}%
\pgfpathclose%
\pgfusepath{fill}%
\end{pgfscope}%
\begin{pgfscope}%
\pgfpathrectangle{\pgfqpoint{0.503581in}{0.449444in}}{\pgfqpoint{3.487500in}{1.155000in}}%
\pgfusepath{clip}%
\pgfsetbuttcap%
\pgfsetmiterjoin%
\definecolor{currentfill}{rgb}{0.000000,0.000000,0.000000}%
\pgfsetfillcolor{currentfill}%
\pgfsetlinewidth{0.000000pt}%
\definecolor{currentstroke}{rgb}{0.000000,0.000000,0.000000}%
\pgfsetstrokecolor{currentstroke}%
\pgfsetstrokeopacity{0.000000}%
\pgfsetdash{}{0pt}%
\pgfpathmoveto{\pgfqpoint{3.198467in}{0.449444in}}%
\pgfpathlineto{\pgfqpoint{3.261876in}{0.449444in}}%
\pgfpathlineto{\pgfqpoint{3.261876in}{0.613808in}}%
\pgfpathlineto{\pgfqpoint{3.198467in}{0.613808in}}%
\pgfpathlineto{\pgfqpoint{3.198467in}{0.449444in}}%
\pgfpathclose%
\pgfusepath{fill}%
\end{pgfscope}%
\begin{pgfscope}%
\pgfpathrectangle{\pgfqpoint{0.503581in}{0.449444in}}{\pgfqpoint{3.487500in}{1.155000in}}%
\pgfusepath{clip}%
\pgfsetbuttcap%
\pgfsetmiterjoin%
\definecolor{currentfill}{rgb}{0.000000,0.000000,0.000000}%
\pgfsetfillcolor{currentfill}%
\pgfsetlinewidth{0.000000pt}%
\definecolor{currentstroke}{rgb}{0.000000,0.000000,0.000000}%
\pgfsetstrokecolor{currentstroke}%
\pgfsetstrokeopacity{0.000000}%
\pgfsetdash{}{0pt}%
\pgfpathmoveto{\pgfqpoint{3.356990in}{0.449444in}}%
\pgfpathlineto{\pgfqpoint{3.420399in}{0.449444in}}%
\pgfpathlineto{\pgfqpoint{3.420399in}{0.601280in}}%
\pgfpathlineto{\pgfqpoint{3.356990in}{0.601280in}}%
\pgfpathlineto{\pgfqpoint{3.356990in}{0.449444in}}%
\pgfpathclose%
\pgfusepath{fill}%
\end{pgfscope}%
\begin{pgfscope}%
\pgfpathrectangle{\pgfqpoint{0.503581in}{0.449444in}}{\pgfqpoint{3.487500in}{1.155000in}}%
\pgfusepath{clip}%
\pgfsetbuttcap%
\pgfsetmiterjoin%
\definecolor{currentfill}{rgb}{0.000000,0.000000,0.000000}%
\pgfsetfillcolor{currentfill}%
\pgfsetlinewidth{0.000000pt}%
\definecolor{currentstroke}{rgb}{0.000000,0.000000,0.000000}%
\pgfsetstrokecolor{currentstroke}%
\pgfsetstrokeopacity{0.000000}%
\pgfsetdash{}{0pt}%
\pgfpathmoveto{\pgfqpoint{3.515513in}{0.449444in}}%
\pgfpathlineto{\pgfqpoint{3.578922in}{0.449444in}}%
\pgfpathlineto{\pgfqpoint{3.578922in}{0.582685in}}%
\pgfpathlineto{\pgfqpoint{3.515513in}{0.582685in}}%
\pgfpathlineto{\pgfqpoint{3.515513in}{0.449444in}}%
\pgfpathclose%
\pgfusepath{fill}%
\end{pgfscope}%
\begin{pgfscope}%
\pgfpathrectangle{\pgfqpoint{0.503581in}{0.449444in}}{\pgfqpoint{3.487500in}{1.155000in}}%
\pgfusepath{clip}%
\pgfsetbuttcap%
\pgfsetmiterjoin%
\definecolor{currentfill}{rgb}{0.000000,0.000000,0.000000}%
\pgfsetfillcolor{currentfill}%
\pgfsetlinewidth{0.000000pt}%
\definecolor{currentstroke}{rgb}{0.000000,0.000000,0.000000}%
\pgfsetstrokecolor{currentstroke}%
\pgfsetstrokeopacity{0.000000}%
\pgfsetdash{}{0pt}%
\pgfpathmoveto{\pgfqpoint{3.674035in}{0.449444in}}%
\pgfpathlineto{\pgfqpoint{3.737444in}{0.449444in}}%
\pgfpathlineto{\pgfqpoint{3.737444in}{0.547651in}}%
\pgfpathlineto{\pgfqpoint{3.674035in}{0.547651in}}%
\pgfpathlineto{\pgfqpoint{3.674035in}{0.449444in}}%
\pgfpathclose%
\pgfusepath{fill}%
\end{pgfscope}%
\begin{pgfscope}%
\pgfpathrectangle{\pgfqpoint{0.503581in}{0.449444in}}{\pgfqpoint{3.487500in}{1.155000in}}%
\pgfusepath{clip}%
\pgfsetbuttcap%
\pgfsetmiterjoin%
\definecolor{currentfill}{rgb}{0.000000,0.000000,0.000000}%
\pgfsetfillcolor{currentfill}%
\pgfsetlinewidth{0.000000pt}%
\definecolor{currentstroke}{rgb}{0.000000,0.000000,0.000000}%
\pgfsetstrokecolor{currentstroke}%
\pgfsetstrokeopacity{0.000000}%
\pgfsetdash{}{0pt}%
\pgfpathmoveto{\pgfqpoint{3.832558in}{0.449444in}}%
\pgfpathlineto{\pgfqpoint{3.895967in}{0.449444in}}%
\pgfpathlineto{\pgfqpoint{3.895967in}{0.481701in}}%
\pgfpathlineto{\pgfqpoint{3.832558in}{0.481701in}}%
\pgfpathlineto{\pgfqpoint{3.832558in}{0.449444in}}%
\pgfpathclose%
\pgfusepath{fill}%
\end{pgfscope}%
\begin{pgfscope}%
\pgfsetbuttcap%
\pgfsetroundjoin%
\definecolor{currentfill}{rgb}{0.000000,0.000000,0.000000}%
\pgfsetfillcolor{currentfill}%
\pgfsetlinewidth{0.803000pt}%
\definecolor{currentstroke}{rgb}{0.000000,0.000000,0.000000}%
\pgfsetstrokecolor{currentstroke}%
\pgfsetdash{}{0pt}%
\pgfsys@defobject{currentmarker}{\pgfqpoint{0.000000in}{-0.048611in}}{\pgfqpoint{0.000000in}{0.000000in}}{%
\pgfpathmoveto{\pgfqpoint{0.000000in}{0.000000in}}%
\pgfpathlineto{\pgfqpoint{0.000000in}{-0.048611in}}%
\pgfusepath{stroke,fill}%
}%
\begin{pgfscope}%
\pgfsys@transformshift{0.503581in}{0.449444in}%
\pgfsys@useobject{currentmarker}{}%
\end{pgfscope}%
\end{pgfscope}%
\begin{pgfscope}%
\pgfsetbuttcap%
\pgfsetroundjoin%
\definecolor{currentfill}{rgb}{0.000000,0.000000,0.000000}%
\pgfsetfillcolor{currentfill}%
\pgfsetlinewidth{0.803000pt}%
\definecolor{currentstroke}{rgb}{0.000000,0.000000,0.000000}%
\pgfsetstrokecolor{currentstroke}%
\pgfsetdash{}{0pt}%
\pgfsys@defobject{currentmarker}{\pgfqpoint{0.000000in}{-0.048611in}}{\pgfqpoint{0.000000in}{0.000000in}}{%
\pgfpathmoveto{\pgfqpoint{0.000000in}{0.000000in}}%
\pgfpathlineto{\pgfqpoint{0.000000in}{-0.048611in}}%
\pgfusepath{stroke,fill}%
}%
\begin{pgfscope}%
\pgfsys@transformshift{0.662103in}{0.449444in}%
\pgfsys@useobject{currentmarker}{}%
\end{pgfscope}%
\end{pgfscope}%
\begin{pgfscope}%
\definecolor{textcolor}{rgb}{0.000000,0.000000,0.000000}%
\pgfsetstrokecolor{textcolor}%
\pgfsetfillcolor{textcolor}%
\pgftext[x=0.662103in,y=0.352222in,,top]{\color{textcolor}\rmfamily\fontsize{10.000000}{12.000000}\selectfont 0.0}%
\end{pgfscope}%
\begin{pgfscope}%
\pgfsetbuttcap%
\pgfsetroundjoin%
\definecolor{currentfill}{rgb}{0.000000,0.000000,0.000000}%
\pgfsetfillcolor{currentfill}%
\pgfsetlinewidth{0.803000pt}%
\definecolor{currentstroke}{rgb}{0.000000,0.000000,0.000000}%
\pgfsetstrokecolor{currentstroke}%
\pgfsetdash{}{0pt}%
\pgfsys@defobject{currentmarker}{\pgfqpoint{0.000000in}{-0.048611in}}{\pgfqpoint{0.000000in}{0.000000in}}{%
\pgfpathmoveto{\pgfqpoint{0.000000in}{0.000000in}}%
\pgfpathlineto{\pgfqpoint{0.000000in}{-0.048611in}}%
\pgfusepath{stroke,fill}%
}%
\begin{pgfscope}%
\pgfsys@transformshift{0.820626in}{0.449444in}%
\pgfsys@useobject{currentmarker}{}%
\end{pgfscope}%
\end{pgfscope}%
\begin{pgfscope}%
\pgfsetbuttcap%
\pgfsetroundjoin%
\definecolor{currentfill}{rgb}{0.000000,0.000000,0.000000}%
\pgfsetfillcolor{currentfill}%
\pgfsetlinewidth{0.803000pt}%
\definecolor{currentstroke}{rgb}{0.000000,0.000000,0.000000}%
\pgfsetstrokecolor{currentstroke}%
\pgfsetdash{}{0pt}%
\pgfsys@defobject{currentmarker}{\pgfqpoint{0.000000in}{-0.048611in}}{\pgfqpoint{0.000000in}{0.000000in}}{%
\pgfpathmoveto{\pgfqpoint{0.000000in}{0.000000in}}%
\pgfpathlineto{\pgfqpoint{0.000000in}{-0.048611in}}%
\pgfusepath{stroke,fill}%
}%
\begin{pgfscope}%
\pgfsys@transformshift{0.979149in}{0.449444in}%
\pgfsys@useobject{currentmarker}{}%
\end{pgfscope}%
\end{pgfscope}%
\begin{pgfscope}%
\definecolor{textcolor}{rgb}{0.000000,0.000000,0.000000}%
\pgfsetstrokecolor{textcolor}%
\pgfsetfillcolor{textcolor}%
\pgftext[x=0.979149in,y=0.352222in,,top]{\color{textcolor}\rmfamily\fontsize{10.000000}{12.000000}\selectfont 0.1}%
\end{pgfscope}%
\begin{pgfscope}%
\pgfsetbuttcap%
\pgfsetroundjoin%
\definecolor{currentfill}{rgb}{0.000000,0.000000,0.000000}%
\pgfsetfillcolor{currentfill}%
\pgfsetlinewidth{0.803000pt}%
\definecolor{currentstroke}{rgb}{0.000000,0.000000,0.000000}%
\pgfsetstrokecolor{currentstroke}%
\pgfsetdash{}{0pt}%
\pgfsys@defobject{currentmarker}{\pgfqpoint{0.000000in}{-0.048611in}}{\pgfqpoint{0.000000in}{0.000000in}}{%
\pgfpathmoveto{\pgfqpoint{0.000000in}{0.000000in}}%
\pgfpathlineto{\pgfqpoint{0.000000in}{-0.048611in}}%
\pgfusepath{stroke,fill}%
}%
\begin{pgfscope}%
\pgfsys@transformshift{1.137672in}{0.449444in}%
\pgfsys@useobject{currentmarker}{}%
\end{pgfscope}%
\end{pgfscope}%
\begin{pgfscope}%
\pgfsetbuttcap%
\pgfsetroundjoin%
\definecolor{currentfill}{rgb}{0.000000,0.000000,0.000000}%
\pgfsetfillcolor{currentfill}%
\pgfsetlinewidth{0.803000pt}%
\definecolor{currentstroke}{rgb}{0.000000,0.000000,0.000000}%
\pgfsetstrokecolor{currentstroke}%
\pgfsetdash{}{0pt}%
\pgfsys@defobject{currentmarker}{\pgfqpoint{0.000000in}{-0.048611in}}{\pgfqpoint{0.000000in}{0.000000in}}{%
\pgfpathmoveto{\pgfqpoint{0.000000in}{0.000000in}}%
\pgfpathlineto{\pgfqpoint{0.000000in}{-0.048611in}}%
\pgfusepath{stroke,fill}%
}%
\begin{pgfscope}%
\pgfsys@transformshift{1.296194in}{0.449444in}%
\pgfsys@useobject{currentmarker}{}%
\end{pgfscope}%
\end{pgfscope}%
\begin{pgfscope}%
\definecolor{textcolor}{rgb}{0.000000,0.000000,0.000000}%
\pgfsetstrokecolor{textcolor}%
\pgfsetfillcolor{textcolor}%
\pgftext[x=1.296194in,y=0.352222in,,top]{\color{textcolor}\rmfamily\fontsize{10.000000}{12.000000}\selectfont 0.2}%
\end{pgfscope}%
\begin{pgfscope}%
\pgfsetbuttcap%
\pgfsetroundjoin%
\definecolor{currentfill}{rgb}{0.000000,0.000000,0.000000}%
\pgfsetfillcolor{currentfill}%
\pgfsetlinewidth{0.803000pt}%
\definecolor{currentstroke}{rgb}{0.000000,0.000000,0.000000}%
\pgfsetstrokecolor{currentstroke}%
\pgfsetdash{}{0pt}%
\pgfsys@defobject{currentmarker}{\pgfqpoint{0.000000in}{-0.048611in}}{\pgfqpoint{0.000000in}{0.000000in}}{%
\pgfpathmoveto{\pgfqpoint{0.000000in}{0.000000in}}%
\pgfpathlineto{\pgfqpoint{0.000000in}{-0.048611in}}%
\pgfusepath{stroke,fill}%
}%
\begin{pgfscope}%
\pgfsys@transformshift{1.454717in}{0.449444in}%
\pgfsys@useobject{currentmarker}{}%
\end{pgfscope}%
\end{pgfscope}%
\begin{pgfscope}%
\pgfsetbuttcap%
\pgfsetroundjoin%
\definecolor{currentfill}{rgb}{0.000000,0.000000,0.000000}%
\pgfsetfillcolor{currentfill}%
\pgfsetlinewidth{0.803000pt}%
\definecolor{currentstroke}{rgb}{0.000000,0.000000,0.000000}%
\pgfsetstrokecolor{currentstroke}%
\pgfsetdash{}{0pt}%
\pgfsys@defobject{currentmarker}{\pgfqpoint{0.000000in}{-0.048611in}}{\pgfqpoint{0.000000in}{0.000000in}}{%
\pgfpathmoveto{\pgfqpoint{0.000000in}{0.000000in}}%
\pgfpathlineto{\pgfqpoint{0.000000in}{-0.048611in}}%
\pgfusepath{stroke,fill}%
}%
\begin{pgfscope}%
\pgfsys@transformshift{1.613240in}{0.449444in}%
\pgfsys@useobject{currentmarker}{}%
\end{pgfscope}%
\end{pgfscope}%
\begin{pgfscope}%
\definecolor{textcolor}{rgb}{0.000000,0.000000,0.000000}%
\pgfsetstrokecolor{textcolor}%
\pgfsetfillcolor{textcolor}%
\pgftext[x=1.613240in,y=0.352222in,,top]{\color{textcolor}\rmfamily\fontsize{10.000000}{12.000000}\selectfont 0.3}%
\end{pgfscope}%
\begin{pgfscope}%
\pgfsetbuttcap%
\pgfsetroundjoin%
\definecolor{currentfill}{rgb}{0.000000,0.000000,0.000000}%
\pgfsetfillcolor{currentfill}%
\pgfsetlinewidth{0.803000pt}%
\definecolor{currentstroke}{rgb}{0.000000,0.000000,0.000000}%
\pgfsetstrokecolor{currentstroke}%
\pgfsetdash{}{0pt}%
\pgfsys@defobject{currentmarker}{\pgfqpoint{0.000000in}{-0.048611in}}{\pgfqpoint{0.000000in}{0.000000in}}{%
\pgfpathmoveto{\pgfqpoint{0.000000in}{0.000000in}}%
\pgfpathlineto{\pgfqpoint{0.000000in}{-0.048611in}}%
\pgfusepath{stroke,fill}%
}%
\begin{pgfscope}%
\pgfsys@transformshift{1.771763in}{0.449444in}%
\pgfsys@useobject{currentmarker}{}%
\end{pgfscope}%
\end{pgfscope}%
\begin{pgfscope}%
\pgfsetbuttcap%
\pgfsetroundjoin%
\definecolor{currentfill}{rgb}{0.000000,0.000000,0.000000}%
\pgfsetfillcolor{currentfill}%
\pgfsetlinewidth{0.803000pt}%
\definecolor{currentstroke}{rgb}{0.000000,0.000000,0.000000}%
\pgfsetstrokecolor{currentstroke}%
\pgfsetdash{}{0pt}%
\pgfsys@defobject{currentmarker}{\pgfqpoint{0.000000in}{-0.048611in}}{\pgfqpoint{0.000000in}{0.000000in}}{%
\pgfpathmoveto{\pgfqpoint{0.000000in}{0.000000in}}%
\pgfpathlineto{\pgfqpoint{0.000000in}{-0.048611in}}%
\pgfusepath{stroke,fill}%
}%
\begin{pgfscope}%
\pgfsys@transformshift{1.930285in}{0.449444in}%
\pgfsys@useobject{currentmarker}{}%
\end{pgfscope}%
\end{pgfscope}%
\begin{pgfscope}%
\definecolor{textcolor}{rgb}{0.000000,0.000000,0.000000}%
\pgfsetstrokecolor{textcolor}%
\pgfsetfillcolor{textcolor}%
\pgftext[x=1.930285in,y=0.352222in,,top]{\color{textcolor}\rmfamily\fontsize{10.000000}{12.000000}\selectfont 0.4}%
\end{pgfscope}%
\begin{pgfscope}%
\pgfsetbuttcap%
\pgfsetroundjoin%
\definecolor{currentfill}{rgb}{0.000000,0.000000,0.000000}%
\pgfsetfillcolor{currentfill}%
\pgfsetlinewidth{0.803000pt}%
\definecolor{currentstroke}{rgb}{0.000000,0.000000,0.000000}%
\pgfsetstrokecolor{currentstroke}%
\pgfsetdash{}{0pt}%
\pgfsys@defobject{currentmarker}{\pgfqpoint{0.000000in}{-0.048611in}}{\pgfqpoint{0.000000in}{0.000000in}}{%
\pgfpathmoveto{\pgfqpoint{0.000000in}{0.000000in}}%
\pgfpathlineto{\pgfqpoint{0.000000in}{-0.048611in}}%
\pgfusepath{stroke,fill}%
}%
\begin{pgfscope}%
\pgfsys@transformshift{2.088808in}{0.449444in}%
\pgfsys@useobject{currentmarker}{}%
\end{pgfscope}%
\end{pgfscope}%
\begin{pgfscope}%
\pgfsetbuttcap%
\pgfsetroundjoin%
\definecolor{currentfill}{rgb}{0.000000,0.000000,0.000000}%
\pgfsetfillcolor{currentfill}%
\pgfsetlinewidth{0.803000pt}%
\definecolor{currentstroke}{rgb}{0.000000,0.000000,0.000000}%
\pgfsetstrokecolor{currentstroke}%
\pgfsetdash{}{0pt}%
\pgfsys@defobject{currentmarker}{\pgfqpoint{0.000000in}{-0.048611in}}{\pgfqpoint{0.000000in}{0.000000in}}{%
\pgfpathmoveto{\pgfqpoint{0.000000in}{0.000000in}}%
\pgfpathlineto{\pgfqpoint{0.000000in}{-0.048611in}}%
\pgfusepath{stroke,fill}%
}%
\begin{pgfscope}%
\pgfsys@transformshift{2.247331in}{0.449444in}%
\pgfsys@useobject{currentmarker}{}%
\end{pgfscope}%
\end{pgfscope}%
\begin{pgfscope}%
\definecolor{textcolor}{rgb}{0.000000,0.000000,0.000000}%
\pgfsetstrokecolor{textcolor}%
\pgfsetfillcolor{textcolor}%
\pgftext[x=2.247331in,y=0.352222in,,top]{\color{textcolor}\rmfamily\fontsize{10.000000}{12.000000}\selectfont 0.5}%
\end{pgfscope}%
\begin{pgfscope}%
\pgfsetbuttcap%
\pgfsetroundjoin%
\definecolor{currentfill}{rgb}{0.000000,0.000000,0.000000}%
\pgfsetfillcolor{currentfill}%
\pgfsetlinewidth{0.803000pt}%
\definecolor{currentstroke}{rgb}{0.000000,0.000000,0.000000}%
\pgfsetstrokecolor{currentstroke}%
\pgfsetdash{}{0pt}%
\pgfsys@defobject{currentmarker}{\pgfqpoint{0.000000in}{-0.048611in}}{\pgfqpoint{0.000000in}{0.000000in}}{%
\pgfpathmoveto{\pgfqpoint{0.000000in}{0.000000in}}%
\pgfpathlineto{\pgfqpoint{0.000000in}{-0.048611in}}%
\pgfusepath{stroke,fill}%
}%
\begin{pgfscope}%
\pgfsys@transformshift{2.405853in}{0.449444in}%
\pgfsys@useobject{currentmarker}{}%
\end{pgfscope}%
\end{pgfscope}%
\begin{pgfscope}%
\pgfsetbuttcap%
\pgfsetroundjoin%
\definecolor{currentfill}{rgb}{0.000000,0.000000,0.000000}%
\pgfsetfillcolor{currentfill}%
\pgfsetlinewidth{0.803000pt}%
\definecolor{currentstroke}{rgb}{0.000000,0.000000,0.000000}%
\pgfsetstrokecolor{currentstroke}%
\pgfsetdash{}{0pt}%
\pgfsys@defobject{currentmarker}{\pgfqpoint{0.000000in}{-0.048611in}}{\pgfqpoint{0.000000in}{0.000000in}}{%
\pgfpathmoveto{\pgfqpoint{0.000000in}{0.000000in}}%
\pgfpathlineto{\pgfqpoint{0.000000in}{-0.048611in}}%
\pgfusepath{stroke,fill}%
}%
\begin{pgfscope}%
\pgfsys@transformshift{2.564376in}{0.449444in}%
\pgfsys@useobject{currentmarker}{}%
\end{pgfscope}%
\end{pgfscope}%
\begin{pgfscope}%
\definecolor{textcolor}{rgb}{0.000000,0.000000,0.000000}%
\pgfsetstrokecolor{textcolor}%
\pgfsetfillcolor{textcolor}%
\pgftext[x=2.564376in,y=0.352222in,,top]{\color{textcolor}\rmfamily\fontsize{10.000000}{12.000000}\selectfont 0.6}%
\end{pgfscope}%
\begin{pgfscope}%
\pgfsetbuttcap%
\pgfsetroundjoin%
\definecolor{currentfill}{rgb}{0.000000,0.000000,0.000000}%
\pgfsetfillcolor{currentfill}%
\pgfsetlinewidth{0.803000pt}%
\definecolor{currentstroke}{rgb}{0.000000,0.000000,0.000000}%
\pgfsetstrokecolor{currentstroke}%
\pgfsetdash{}{0pt}%
\pgfsys@defobject{currentmarker}{\pgfqpoint{0.000000in}{-0.048611in}}{\pgfqpoint{0.000000in}{0.000000in}}{%
\pgfpathmoveto{\pgfqpoint{0.000000in}{0.000000in}}%
\pgfpathlineto{\pgfqpoint{0.000000in}{-0.048611in}}%
\pgfusepath{stroke,fill}%
}%
\begin{pgfscope}%
\pgfsys@transformshift{2.722899in}{0.449444in}%
\pgfsys@useobject{currentmarker}{}%
\end{pgfscope}%
\end{pgfscope}%
\begin{pgfscope}%
\pgfsetbuttcap%
\pgfsetroundjoin%
\definecolor{currentfill}{rgb}{0.000000,0.000000,0.000000}%
\pgfsetfillcolor{currentfill}%
\pgfsetlinewidth{0.803000pt}%
\definecolor{currentstroke}{rgb}{0.000000,0.000000,0.000000}%
\pgfsetstrokecolor{currentstroke}%
\pgfsetdash{}{0pt}%
\pgfsys@defobject{currentmarker}{\pgfqpoint{0.000000in}{-0.048611in}}{\pgfqpoint{0.000000in}{0.000000in}}{%
\pgfpathmoveto{\pgfqpoint{0.000000in}{0.000000in}}%
\pgfpathlineto{\pgfqpoint{0.000000in}{-0.048611in}}%
\pgfusepath{stroke,fill}%
}%
\begin{pgfscope}%
\pgfsys@transformshift{2.881422in}{0.449444in}%
\pgfsys@useobject{currentmarker}{}%
\end{pgfscope}%
\end{pgfscope}%
\begin{pgfscope}%
\definecolor{textcolor}{rgb}{0.000000,0.000000,0.000000}%
\pgfsetstrokecolor{textcolor}%
\pgfsetfillcolor{textcolor}%
\pgftext[x=2.881422in,y=0.352222in,,top]{\color{textcolor}\rmfamily\fontsize{10.000000}{12.000000}\selectfont 0.7}%
\end{pgfscope}%
\begin{pgfscope}%
\pgfsetbuttcap%
\pgfsetroundjoin%
\definecolor{currentfill}{rgb}{0.000000,0.000000,0.000000}%
\pgfsetfillcolor{currentfill}%
\pgfsetlinewidth{0.803000pt}%
\definecolor{currentstroke}{rgb}{0.000000,0.000000,0.000000}%
\pgfsetstrokecolor{currentstroke}%
\pgfsetdash{}{0pt}%
\pgfsys@defobject{currentmarker}{\pgfqpoint{0.000000in}{-0.048611in}}{\pgfqpoint{0.000000in}{0.000000in}}{%
\pgfpathmoveto{\pgfqpoint{0.000000in}{0.000000in}}%
\pgfpathlineto{\pgfqpoint{0.000000in}{-0.048611in}}%
\pgfusepath{stroke,fill}%
}%
\begin{pgfscope}%
\pgfsys@transformshift{3.039944in}{0.449444in}%
\pgfsys@useobject{currentmarker}{}%
\end{pgfscope}%
\end{pgfscope}%
\begin{pgfscope}%
\pgfsetbuttcap%
\pgfsetroundjoin%
\definecolor{currentfill}{rgb}{0.000000,0.000000,0.000000}%
\pgfsetfillcolor{currentfill}%
\pgfsetlinewidth{0.803000pt}%
\definecolor{currentstroke}{rgb}{0.000000,0.000000,0.000000}%
\pgfsetstrokecolor{currentstroke}%
\pgfsetdash{}{0pt}%
\pgfsys@defobject{currentmarker}{\pgfqpoint{0.000000in}{-0.048611in}}{\pgfqpoint{0.000000in}{0.000000in}}{%
\pgfpathmoveto{\pgfqpoint{0.000000in}{0.000000in}}%
\pgfpathlineto{\pgfqpoint{0.000000in}{-0.048611in}}%
\pgfusepath{stroke,fill}%
}%
\begin{pgfscope}%
\pgfsys@transformshift{3.198467in}{0.449444in}%
\pgfsys@useobject{currentmarker}{}%
\end{pgfscope}%
\end{pgfscope}%
\begin{pgfscope}%
\definecolor{textcolor}{rgb}{0.000000,0.000000,0.000000}%
\pgfsetstrokecolor{textcolor}%
\pgfsetfillcolor{textcolor}%
\pgftext[x=3.198467in,y=0.352222in,,top]{\color{textcolor}\rmfamily\fontsize{10.000000}{12.000000}\selectfont 0.8}%
\end{pgfscope}%
\begin{pgfscope}%
\pgfsetbuttcap%
\pgfsetroundjoin%
\definecolor{currentfill}{rgb}{0.000000,0.000000,0.000000}%
\pgfsetfillcolor{currentfill}%
\pgfsetlinewidth{0.803000pt}%
\definecolor{currentstroke}{rgb}{0.000000,0.000000,0.000000}%
\pgfsetstrokecolor{currentstroke}%
\pgfsetdash{}{0pt}%
\pgfsys@defobject{currentmarker}{\pgfqpoint{0.000000in}{-0.048611in}}{\pgfqpoint{0.000000in}{0.000000in}}{%
\pgfpathmoveto{\pgfqpoint{0.000000in}{0.000000in}}%
\pgfpathlineto{\pgfqpoint{0.000000in}{-0.048611in}}%
\pgfusepath{stroke,fill}%
}%
\begin{pgfscope}%
\pgfsys@transformshift{3.356990in}{0.449444in}%
\pgfsys@useobject{currentmarker}{}%
\end{pgfscope}%
\end{pgfscope}%
\begin{pgfscope}%
\pgfsetbuttcap%
\pgfsetroundjoin%
\definecolor{currentfill}{rgb}{0.000000,0.000000,0.000000}%
\pgfsetfillcolor{currentfill}%
\pgfsetlinewidth{0.803000pt}%
\definecolor{currentstroke}{rgb}{0.000000,0.000000,0.000000}%
\pgfsetstrokecolor{currentstroke}%
\pgfsetdash{}{0pt}%
\pgfsys@defobject{currentmarker}{\pgfqpoint{0.000000in}{-0.048611in}}{\pgfqpoint{0.000000in}{0.000000in}}{%
\pgfpathmoveto{\pgfqpoint{0.000000in}{0.000000in}}%
\pgfpathlineto{\pgfqpoint{0.000000in}{-0.048611in}}%
\pgfusepath{stroke,fill}%
}%
\begin{pgfscope}%
\pgfsys@transformshift{3.515513in}{0.449444in}%
\pgfsys@useobject{currentmarker}{}%
\end{pgfscope}%
\end{pgfscope}%
\begin{pgfscope}%
\definecolor{textcolor}{rgb}{0.000000,0.000000,0.000000}%
\pgfsetstrokecolor{textcolor}%
\pgfsetfillcolor{textcolor}%
\pgftext[x=3.515513in,y=0.352222in,,top]{\color{textcolor}\rmfamily\fontsize{10.000000}{12.000000}\selectfont 0.9}%
\end{pgfscope}%
\begin{pgfscope}%
\pgfsetbuttcap%
\pgfsetroundjoin%
\definecolor{currentfill}{rgb}{0.000000,0.000000,0.000000}%
\pgfsetfillcolor{currentfill}%
\pgfsetlinewidth{0.803000pt}%
\definecolor{currentstroke}{rgb}{0.000000,0.000000,0.000000}%
\pgfsetstrokecolor{currentstroke}%
\pgfsetdash{}{0pt}%
\pgfsys@defobject{currentmarker}{\pgfqpoint{0.000000in}{-0.048611in}}{\pgfqpoint{0.000000in}{0.000000in}}{%
\pgfpathmoveto{\pgfqpoint{0.000000in}{0.000000in}}%
\pgfpathlineto{\pgfqpoint{0.000000in}{-0.048611in}}%
\pgfusepath{stroke,fill}%
}%
\begin{pgfscope}%
\pgfsys@transformshift{3.674035in}{0.449444in}%
\pgfsys@useobject{currentmarker}{}%
\end{pgfscope}%
\end{pgfscope}%
\begin{pgfscope}%
\pgfsetbuttcap%
\pgfsetroundjoin%
\definecolor{currentfill}{rgb}{0.000000,0.000000,0.000000}%
\pgfsetfillcolor{currentfill}%
\pgfsetlinewidth{0.803000pt}%
\definecolor{currentstroke}{rgb}{0.000000,0.000000,0.000000}%
\pgfsetstrokecolor{currentstroke}%
\pgfsetdash{}{0pt}%
\pgfsys@defobject{currentmarker}{\pgfqpoint{0.000000in}{-0.048611in}}{\pgfqpoint{0.000000in}{0.000000in}}{%
\pgfpathmoveto{\pgfqpoint{0.000000in}{0.000000in}}%
\pgfpathlineto{\pgfqpoint{0.000000in}{-0.048611in}}%
\pgfusepath{stroke,fill}%
}%
\begin{pgfscope}%
\pgfsys@transformshift{3.832558in}{0.449444in}%
\pgfsys@useobject{currentmarker}{}%
\end{pgfscope}%
\end{pgfscope}%
\begin{pgfscope}%
\definecolor{textcolor}{rgb}{0.000000,0.000000,0.000000}%
\pgfsetstrokecolor{textcolor}%
\pgfsetfillcolor{textcolor}%
\pgftext[x=3.832558in,y=0.352222in,,top]{\color{textcolor}\rmfamily\fontsize{10.000000}{12.000000}\selectfont 1.0}%
\end{pgfscope}%
\begin{pgfscope}%
\pgfsetbuttcap%
\pgfsetroundjoin%
\definecolor{currentfill}{rgb}{0.000000,0.000000,0.000000}%
\pgfsetfillcolor{currentfill}%
\pgfsetlinewidth{0.803000pt}%
\definecolor{currentstroke}{rgb}{0.000000,0.000000,0.000000}%
\pgfsetstrokecolor{currentstroke}%
\pgfsetdash{}{0pt}%
\pgfsys@defobject{currentmarker}{\pgfqpoint{0.000000in}{-0.048611in}}{\pgfqpoint{0.000000in}{0.000000in}}{%
\pgfpathmoveto{\pgfqpoint{0.000000in}{0.000000in}}%
\pgfpathlineto{\pgfqpoint{0.000000in}{-0.048611in}}%
\pgfusepath{stroke,fill}%
}%
\begin{pgfscope}%
\pgfsys@transformshift{3.991081in}{0.449444in}%
\pgfsys@useobject{currentmarker}{}%
\end{pgfscope}%
\end{pgfscope}%
\begin{pgfscope}%
\definecolor{textcolor}{rgb}{0.000000,0.000000,0.000000}%
\pgfsetstrokecolor{textcolor}%
\pgfsetfillcolor{textcolor}%
\pgftext[x=2.247331in,y=0.173333in,,top]{\color{textcolor}\rmfamily\fontsize{10.000000}{12.000000}\selectfont \(\displaystyle p\)}%
\end{pgfscope}%
\begin{pgfscope}%
\pgfsetbuttcap%
\pgfsetroundjoin%
\definecolor{currentfill}{rgb}{0.000000,0.000000,0.000000}%
\pgfsetfillcolor{currentfill}%
\pgfsetlinewidth{0.803000pt}%
\definecolor{currentstroke}{rgb}{0.000000,0.000000,0.000000}%
\pgfsetstrokecolor{currentstroke}%
\pgfsetdash{}{0pt}%
\pgfsys@defobject{currentmarker}{\pgfqpoint{-0.048611in}{0.000000in}}{\pgfqpoint{-0.000000in}{0.000000in}}{%
\pgfpathmoveto{\pgfqpoint{-0.000000in}{0.000000in}}%
\pgfpathlineto{\pgfqpoint{-0.048611in}{0.000000in}}%
\pgfusepath{stroke,fill}%
}%
\begin{pgfscope}%
\pgfsys@transformshift{0.503581in}{0.449444in}%
\pgfsys@useobject{currentmarker}{}%
\end{pgfscope}%
\end{pgfscope}%
\begin{pgfscope}%
\definecolor{textcolor}{rgb}{0.000000,0.000000,0.000000}%
\pgfsetstrokecolor{textcolor}%
\pgfsetfillcolor{textcolor}%
\pgftext[x=0.228889in, y=0.401250in, left, base]{\color{textcolor}\rmfamily\fontsize{10.000000}{12.000000}\selectfont \(\displaystyle {0.0}\)}%
\end{pgfscope}%
\begin{pgfscope}%
\pgfsetbuttcap%
\pgfsetroundjoin%
\definecolor{currentfill}{rgb}{0.000000,0.000000,0.000000}%
\pgfsetfillcolor{currentfill}%
\pgfsetlinewidth{0.803000pt}%
\definecolor{currentstroke}{rgb}{0.000000,0.000000,0.000000}%
\pgfsetstrokecolor{currentstroke}%
\pgfsetdash{}{0pt}%
\pgfsys@defobject{currentmarker}{\pgfqpoint{-0.048611in}{0.000000in}}{\pgfqpoint{-0.000000in}{0.000000in}}{%
\pgfpathmoveto{\pgfqpoint{-0.000000in}{0.000000in}}%
\pgfpathlineto{\pgfqpoint{-0.048611in}{0.000000in}}%
\pgfusepath{stroke,fill}%
}%
\begin{pgfscope}%
\pgfsys@transformshift{0.503581in}{0.786547in}%
\pgfsys@useobject{currentmarker}{}%
\end{pgfscope}%
\end{pgfscope}%
\begin{pgfscope}%
\definecolor{textcolor}{rgb}{0.000000,0.000000,0.000000}%
\pgfsetstrokecolor{textcolor}%
\pgfsetfillcolor{textcolor}%
\pgftext[x=0.228889in, y=0.738352in, left, base]{\color{textcolor}\rmfamily\fontsize{10.000000}{12.000000}\selectfont \(\displaystyle {2.5}\)}%
\end{pgfscope}%
\begin{pgfscope}%
\pgfsetbuttcap%
\pgfsetroundjoin%
\definecolor{currentfill}{rgb}{0.000000,0.000000,0.000000}%
\pgfsetfillcolor{currentfill}%
\pgfsetlinewidth{0.803000pt}%
\definecolor{currentstroke}{rgb}{0.000000,0.000000,0.000000}%
\pgfsetstrokecolor{currentstroke}%
\pgfsetdash{}{0pt}%
\pgfsys@defobject{currentmarker}{\pgfqpoint{-0.048611in}{0.000000in}}{\pgfqpoint{-0.000000in}{0.000000in}}{%
\pgfpathmoveto{\pgfqpoint{-0.000000in}{0.000000in}}%
\pgfpathlineto{\pgfqpoint{-0.048611in}{0.000000in}}%
\pgfusepath{stroke,fill}%
}%
\begin{pgfscope}%
\pgfsys@transformshift{0.503581in}{1.123649in}%
\pgfsys@useobject{currentmarker}{}%
\end{pgfscope}%
\end{pgfscope}%
\begin{pgfscope}%
\definecolor{textcolor}{rgb}{0.000000,0.000000,0.000000}%
\pgfsetstrokecolor{textcolor}%
\pgfsetfillcolor{textcolor}%
\pgftext[x=0.228889in, y=1.075454in, left, base]{\color{textcolor}\rmfamily\fontsize{10.000000}{12.000000}\selectfont \(\displaystyle {5.0}\)}%
\end{pgfscope}%
\begin{pgfscope}%
\pgfsetbuttcap%
\pgfsetroundjoin%
\definecolor{currentfill}{rgb}{0.000000,0.000000,0.000000}%
\pgfsetfillcolor{currentfill}%
\pgfsetlinewidth{0.803000pt}%
\definecolor{currentstroke}{rgb}{0.000000,0.000000,0.000000}%
\pgfsetstrokecolor{currentstroke}%
\pgfsetdash{}{0pt}%
\pgfsys@defobject{currentmarker}{\pgfqpoint{-0.048611in}{0.000000in}}{\pgfqpoint{-0.000000in}{0.000000in}}{%
\pgfpathmoveto{\pgfqpoint{-0.000000in}{0.000000in}}%
\pgfpathlineto{\pgfqpoint{-0.048611in}{0.000000in}}%
\pgfusepath{stroke,fill}%
}%
\begin{pgfscope}%
\pgfsys@transformshift{0.503581in}{1.460751in}%
\pgfsys@useobject{currentmarker}{}%
\end{pgfscope}%
\end{pgfscope}%
\begin{pgfscope}%
\definecolor{textcolor}{rgb}{0.000000,0.000000,0.000000}%
\pgfsetstrokecolor{textcolor}%
\pgfsetfillcolor{textcolor}%
\pgftext[x=0.228889in, y=1.412557in, left, base]{\color{textcolor}\rmfamily\fontsize{10.000000}{12.000000}\selectfont \(\displaystyle {7.5}\)}%
\end{pgfscope}%
\begin{pgfscope}%
\definecolor{textcolor}{rgb}{0.000000,0.000000,0.000000}%
\pgfsetstrokecolor{textcolor}%
\pgfsetfillcolor{textcolor}%
\pgftext[x=0.173333in,y=1.026944in,,bottom,rotate=90.000000]{\color{textcolor}\rmfamily\fontsize{10.000000}{12.000000}\selectfont Percent of Data Set}%
\end{pgfscope}%
\begin{pgfscope}%
\pgfsetrectcap%
\pgfsetmiterjoin%
\pgfsetlinewidth{0.803000pt}%
\definecolor{currentstroke}{rgb}{0.000000,0.000000,0.000000}%
\pgfsetstrokecolor{currentstroke}%
\pgfsetdash{}{0pt}%
\pgfpathmoveto{\pgfqpoint{0.503581in}{0.449444in}}%
\pgfpathlineto{\pgfqpoint{0.503581in}{1.604444in}}%
\pgfusepath{stroke}%
\end{pgfscope}%
\begin{pgfscope}%
\pgfsetrectcap%
\pgfsetmiterjoin%
\pgfsetlinewidth{0.803000pt}%
\definecolor{currentstroke}{rgb}{0.000000,0.000000,0.000000}%
\pgfsetstrokecolor{currentstroke}%
\pgfsetdash{}{0pt}%
\pgfpathmoveto{\pgfqpoint{3.991081in}{0.449444in}}%
\pgfpathlineto{\pgfqpoint{3.991081in}{1.604444in}}%
\pgfusepath{stroke}%
\end{pgfscope}%
\begin{pgfscope}%
\pgfsetrectcap%
\pgfsetmiterjoin%
\pgfsetlinewidth{0.803000pt}%
\definecolor{currentstroke}{rgb}{0.000000,0.000000,0.000000}%
\pgfsetstrokecolor{currentstroke}%
\pgfsetdash{}{0pt}%
\pgfpathmoveto{\pgfqpoint{0.503581in}{0.449444in}}%
\pgfpathlineto{\pgfqpoint{3.991081in}{0.449444in}}%
\pgfusepath{stroke}%
\end{pgfscope}%
\begin{pgfscope}%
\pgfsetrectcap%
\pgfsetmiterjoin%
\pgfsetlinewidth{0.803000pt}%
\definecolor{currentstroke}{rgb}{0.000000,0.000000,0.000000}%
\pgfsetstrokecolor{currentstroke}%
\pgfsetdash{}{0pt}%
\pgfpathmoveto{\pgfqpoint{0.503581in}{1.604444in}}%
\pgfpathlineto{\pgfqpoint{3.991081in}{1.604444in}}%
\pgfusepath{stroke}%
\end{pgfscope}%
\begin{pgfscope}%
\pgfsetbuttcap%
\pgfsetmiterjoin%
\definecolor{currentfill}{rgb}{1.000000,1.000000,1.000000}%
\pgfsetfillcolor{currentfill}%
\pgfsetfillopacity{0.800000}%
\pgfsetlinewidth{1.003750pt}%
\definecolor{currentstroke}{rgb}{0.800000,0.800000,0.800000}%
\pgfsetstrokecolor{currentstroke}%
\pgfsetstrokeopacity{0.800000}%
\pgfsetdash{}{0pt}%
\pgfpathmoveto{\pgfqpoint{3.214136in}{1.104445in}}%
\pgfpathlineto{\pgfqpoint{3.893858in}{1.104445in}}%
\pgfpathquadraticcurveto{\pgfqpoint{3.921636in}{1.104445in}}{\pgfqpoint{3.921636in}{1.132222in}}%
\pgfpathlineto{\pgfqpoint{3.921636in}{1.507222in}}%
\pgfpathquadraticcurveto{\pgfqpoint{3.921636in}{1.535000in}}{\pgfqpoint{3.893858in}{1.535000in}}%
\pgfpathlineto{\pgfqpoint{3.214136in}{1.535000in}}%
\pgfpathquadraticcurveto{\pgfqpoint{3.186358in}{1.535000in}}{\pgfqpoint{3.186358in}{1.507222in}}%
\pgfpathlineto{\pgfqpoint{3.186358in}{1.132222in}}%
\pgfpathquadraticcurveto{\pgfqpoint{3.186358in}{1.104445in}}{\pgfqpoint{3.214136in}{1.104445in}}%
\pgfpathlineto{\pgfqpoint{3.214136in}{1.104445in}}%
\pgfpathclose%
\pgfusepath{stroke,fill}%
\end{pgfscope}%
\begin{pgfscope}%
\pgfsetbuttcap%
\pgfsetmiterjoin%
\pgfsetlinewidth{1.003750pt}%
\definecolor{currentstroke}{rgb}{0.000000,0.000000,0.000000}%
\pgfsetstrokecolor{currentstroke}%
\pgfsetdash{}{0pt}%
\pgfpathmoveto{\pgfqpoint{3.241914in}{1.382222in}}%
\pgfpathlineto{\pgfqpoint{3.519692in}{1.382222in}}%
\pgfpathlineto{\pgfqpoint{3.519692in}{1.479444in}}%
\pgfpathlineto{\pgfqpoint{3.241914in}{1.479444in}}%
\pgfpathlineto{\pgfqpoint{3.241914in}{1.382222in}}%
\pgfpathclose%
\pgfusepath{stroke}%
\end{pgfscope}%
\begin{pgfscope}%
\definecolor{textcolor}{rgb}{0.000000,0.000000,0.000000}%
\pgfsetstrokecolor{textcolor}%
\pgfsetfillcolor{textcolor}%
\pgftext[x=3.630803in,y=1.382222in,left,base]{\color{textcolor}\rmfamily\fontsize{10.000000}{12.000000}\selectfont Neg}%
\end{pgfscope}%
\begin{pgfscope}%
\pgfsetbuttcap%
\pgfsetmiterjoin%
\definecolor{currentfill}{rgb}{0.000000,0.000000,0.000000}%
\pgfsetfillcolor{currentfill}%
\pgfsetlinewidth{0.000000pt}%
\definecolor{currentstroke}{rgb}{0.000000,0.000000,0.000000}%
\pgfsetstrokecolor{currentstroke}%
\pgfsetstrokeopacity{0.000000}%
\pgfsetdash{}{0pt}%
\pgfpathmoveto{\pgfqpoint{3.241914in}{1.186944in}}%
\pgfpathlineto{\pgfqpoint{3.519692in}{1.186944in}}%
\pgfpathlineto{\pgfqpoint{3.519692in}{1.284167in}}%
\pgfpathlineto{\pgfqpoint{3.241914in}{1.284167in}}%
\pgfpathlineto{\pgfqpoint{3.241914in}{1.186944in}}%
\pgfpathclose%
\pgfusepath{fill}%
\end{pgfscope}%
\begin{pgfscope}%
\definecolor{textcolor}{rgb}{0.000000,0.000000,0.000000}%
\pgfsetstrokecolor{textcolor}%
\pgfsetfillcolor{textcolor}%
\pgftext[x=3.630803in,y=1.186944in,left,base]{\color{textcolor}\rmfamily\fontsize{10.000000}{12.000000}\selectfont Pos}%
\end{pgfscope}%
\end{pgfpicture}%
\makeatother%
\endgroup%
	
&
	\vskip 0pt
	\hfil ROC Curve
	
	%% Creator: Matplotlib, PGF backend
%%
%% To include the figure in your LaTeX document, write
%%   \input{<filename>.pgf}
%%
%% Make sure the required packages are loaded in your preamble
%%   \usepackage{pgf}
%%
%% Also ensure that all the required font packages are loaded; for instance,
%% the lmodern package is sometimes necessary when using math font.
%%   \usepackage{lmodern}
%%
%% Figures using additional raster images can only be included by \input if
%% they are in the same directory as the main LaTeX file. For loading figures
%% from other directories you can use the `import` package
%%   \usepackage{import}
%%
%% and then include the figures with
%%   \import{<path to file>}{<filename>.pgf}
%%
%% Matplotlib used the following preamble
%%   
%%   \usepackage{fontspec}
%%   \makeatletter\@ifpackageloaded{underscore}{}{\usepackage[strings]{underscore}}\makeatother
%%
\begingroup%
\makeatletter%
\begin{pgfpicture}%
\pgfpathrectangle{\pgfpointorigin}{\pgfqpoint{2.121861in}{1.654444in}}%
\pgfusepath{use as bounding box, clip}%
\begin{pgfscope}%
\pgfsetbuttcap%
\pgfsetmiterjoin%
\definecolor{currentfill}{rgb}{1.000000,1.000000,1.000000}%
\pgfsetfillcolor{currentfill}%
\pgfsetlinewidth{0.000000pt}%
\definecolor{currentstroke}{rgb}{1.000000,1.000000,1.000000}%
\pgfsetstrokecolor{currentstroke}%
\pgfsetdash{}{0pt}%
\pgfpathmoveto{\pgfqpoint{0.000000in}{0.000000in}}%
\pgfpathlineto{\pgfqpoint{2.121861in}{0.000000in}}%
\pgfpathlineto{\pgfqpoint{2.121861in}{1.654444in}}%
\pgfpathlineto{\pgfqpoint{0.000000in}{1.654444in}}%
\pgfpathlineto{\pgfqpoint{0.000000in}{0.000000in}}%
\pgfpathclose%
\pgfusepath{fill}%
\end{pgfscope}%
\begin{pgfscope}%
\pgfsetbuttcap%
\pgfsetmiterjoin%
\definecolor{currentfill}{rgb}{1.000000,1.000000,1.000000}%
\pgfsetfillcolor{currentfill}%
\pgfsetlinewidth{0.000000pt}%
\definecolor{currentstroke}{rgb}{0.000000,0.000000,0.000000}%
\pgfsetstrokecolor{currentstroke}%
\pgfsetstrokeopacity{0.000000}%
\pgfsetdash{}{0pt}%
\pgfpathmoveto{\pgfqpoint{0.503581in}{0.449444in}}%
\pgfpathlineto{\pgfqpoint{2.053581in}{0.449444in}}%
\pgfpathlineto{\pgfqpoint{2.053581in}{1.604444in}}%
\pgfpathlineto{\pgfqpoint{0.503581in}{1.604444in}}%
\pgfpathlineto{\pgfqpoint{0.503581in}{0.449444in}}%
\pgfpathclose%
\pgfusepath{fill}%
\end{pgfscope}%
\begin{pgfscope}%
\pgfsetbuttcap%
\pgfsetroundjoin%
\definecolor{currentfill}{rgb}{0.000000,0.000000,0.000000}%
\pgfsetfillcolor{currentfill}%
\pgfsetlinewidth{0.803000pt}%
\definecolor{currentstroke}{rgb}{0.000000,0.000000,0.000000}%
\pgfsetstrokecolor{currentstroke}%
\pgfsetdash{}{0pt}%
\pgfsys@defobject{currentmarker}{\pgfqpoint{0.000000in}{-0.048611in}}{\pgfqpoint{0.000000in}{0.000000in}}{%
\pgfpathmoveto{\pgfqpoint{0.000000in}{0.000000in}}%
\pgfpathlineto{\pgfqpoint{0.000000in}{-0.048611in}}%
\pgfusepath{stroke,fill}%
}%
\begin{pgfscope}%
\pgfsys@transformshift{0.574035in}{0.449444in}%
\pgfsys@useobject{currentmarker}{}%
\end{pgfscope}%
\end{pgfscope}%
\begin{pgfscope}%
\definecolor{textcolor}{rgb}{0.000000,0.000000,0.000000}%
\pgfsetstrokecolor{textcolor}%
\pgfsetfillcolor{textcolor}%
\pgftext[x=0.574035in,y=0.352222in,,top]{\color{textcolor}\rmfamily\fontsize{10.000000}{12.000000}\selectfont \(\displaystyle {0.0}\)}%
\end{pgfscope}%
\begin{pgfscope}%
\pgfsetbuttcap%
\pgfsetroundjoin%
\definecolor{currentfill}{rgb}{0.000000,0.000000,0.000000}%
\pgfsetfillcolor{currentfill}%
\pgfsetlinewidth{0.803000pt}%
\definecolor{currentstroke}{rgb}{0.000000,0.000000,0.000000}%
\pgfsetstrokecolor{currentstroke}%
\pgfsetdash{}{0pt}%
\pgfsys@defobject{currentmarker}{\pgfqpoint{0.000000in}{-0.048611in}}{\pgfqpoint{0.000000in}{0.000000in}}{%
\pgfpathmoveto{\pgfqpoint{0.000000in}{0.000000in}}%
\pgfpathlineto{\pgfqpoint{0.000000in}{-0.048611in}}%
\pgfusepath{stroke,fill}%
}%
\begin{pgfscope}%
\pgfsys@transformshift{1.278581in}{0.449444in}%
\pgfsys@useobject{currentmarker}{}%
\end{pgfscope}%
\end{pgfscope}%
\begin{pgfscope}%
\definecolor{textcolor}{rgb}{0.000000,0.000000,0.000000}%
\pgfsetstrokecolor{textcolor}%
\pgfsetfillcolor{textcolor}%
\pgftext[x=1.278581in,y=0.352222in,,top]{\color{textcolor}\rmfamily\fontsize{10.000000}{12.000000}\selectfont \(\displaystyle {0.5}\)}%
\end{pgfscope}%
\begin{pgfscope}%
\pgfsetbuttcap%
\pgfsetroundjoin%
\definecolor{currentfill}{rgb}{0.000000,0.000000,0.000000}%
\pgfsetfillcolor{currentfill}%
\pgfsetlinewidth{0.803000pt}%
\definecolor{currentstroke}{rgb}{0.000000,0.000000,0.000000}%
\pgfsetstrokecolor{currentstroke}%
\pgfsetdash{}{0pt}%
\pgfsys@defobject{currentmarker}{\pgfqpoint{0.000000in}{-0.048611in}}{\pgfqpoint{0.000000in}{0.000000in}}{%
\pgfpathmoveto{\pgfqpoint{0.000000in}{0.000000in}}%
\pgfpathlineto{\pgfqpoint{0.000000in}{-0.048611in}}%
\pgfusepath{stroke,fill}%
}%
\begin{pgfscope}%
\pgfsys@transformshift{1.983126in}{0.449444in}%
\pgfsys@useobject{currentmarker}{}%
\end{pgfscope}%
\end{pgfscope}%
\begin{pgfscope}%
\definecolor{textcolor}{rgb}{0.000000,0.000000,0.000000}%
\pgfsetstrokecolor{textcolor}%
\pgfsetfillcolor{textcolor}%
\pgftext[x=1.983126in,y=0.352222in,,top]{\color{textcolor}\rmfamily\fontsize{10.000000}{12.000000}\selectfont \(\displaystyle {1.0}\)}%
\end{pgfscope}%
\begin{pgfscope}%
\definecolor{textcolor}{rgb}{0.000000,0.000000,0.000000}%
\pgfsetstrokecolor{textcolor}%
\pgfsetfillcolor{textcolor}%
\pgftext[x=1.278581in,y=0.173333in,,top]{\color{textcolor}\rmfamily\fontsize{10.000000}{12.000000}\selectfont False positive rate}%
\end{pgfscope}%
\begin{pgfscope}%
\pgfsetbuttcap%
\pgfsetroundjoin%
\definecolor{currentfill}{rgb}{0.000000,0.000000,0.000000}%
\pgfsetfillcolor{currentfill}%
\pgfsetlinewidth{0.803000pt}%
\definecolor{currentstroke}{rgb}{0.000000,0.000000,0.000000}%
\pgfsetstrokecolor{currentstroke}%
\pgfsetdash{}{0pt}%
\pgfsys@defobject{currentmarker}{\pgfqpoint{-0.048611in}{0.000000in}}{\pgfqpoint{-0.000000in}{0.000000in}}{%
\pgfpathmoveto{\pgfqpoint{-0.000000in}{0.000000in}}%
\pgfpathlineto{\pgfqpoint{-0.048611in}{0.000000in}}%
\pgfusepath{stroke,fill}%
}%
\begin{pgfscope}%
\pgfsys@transformshift{0.503581in}{0.501944in}%
\pgfsys@useobject{currentmarker}{}%
\end{pgfscope}%
\end{pgfscope}%
\begin{pgfscope}%
\definecolor{textcolor}{rgb}{0.000000,0.000000,0.000000}%
\pgfsetstrokecolor{textcolor}%
\pgfsetfillcolor{textcolor}%
\pgftext[x=0.228889in, y=0.453750in, left, base]{\color{textcolor}\rmfamily\fontsize{10.000000}{12.000000}\selectfont \(\displaystyle {0.0}\)}%
\end{pgfscope}%
\begin{pgfscope}%
\pgfsetbuttcap%
\pgfsetroundjoin%
\definecolor{currentfill}{rgb}{0.000000,0.000000,0.000000}%
\pgfsetfillcolor{currentfill}%
\pgfsetlinewidth{0.803000pt}%
\definecolor{currentstroke}{rgb}{0.000000,0.000000,0.000000}%
\pgfsetstrokecolor{currentstroke}%
\pgfsetdash{}{0pt}%
\pgfsys@defobject{currentmarker}{\pgfqpoint{-0.048611in}{0.000000in}}{\pgfqpoint{-0.000000in}{0.000000in}}{%
\pgfpathmoveto{\pgfqpoint{-0.000000in}{0.000000in}}%
\pgfpathlineto{\pgfqpoint{-0.048611in}{0.000000in}}%
\pgfusepath{stroke,fill}%
}%
\begin{pgfscope}%
\pgfsys@transformshift{0.503581in}{1.026944in}%
\pgfsys@useobject{currentmarker}{}%
\end{pgfscope}%
\end{pgfscope}%
\begin{pgfscope}%
\definecolor{textcolor}{rgb}{0.000000,0.000000,0.000000}%
\pgfsetstrokecolor{textcolor}%
\pgfsetfillcolor{textcolor}%
\pgftext[x=0.228889in, y=0.978750in, left, base]{\color{textcolor}\rmfamily\fontsize{10.000000}{12.000000}\selectfont \(\displaystyle {0.5}\)}%
\end{pgfscope}%
\begin{pgfscope}%
\pgfsetbuttcap%
\pgfsetroundjoin%
\definecolor{currentfill}{rgb}{0.000000,0.000000,0.000000}%
\pgfsetfillcolor{currentfill}%
\pgfsetlinewidth{0.803000pt}%
\definecolor{currentstroke}{rgb}{0.000000,0.000000,0.000000}%
\pgfsetstrokecolor{currentstroke}%
\pgfsetdash{}{0pt}%
\pgfsys@defobject{currentmarker}{\pgfqpoint{-0.048611in}{0.000000in}}{\pgfqpoint{-0.000000in}{0.000000in}}{%
\pgfpathmoveto{\pgfqpoint{-0.000000in}{0.000000in}}%
\pgfpathlineto{\pgfqpoint{-0.048611in}{0.000000in}}%
\pgfusepath{stroke,fill}%
}%
\begin{pgfscope}%
\pgfsys@transformshift{0.503581in}{1.551944in}%
\pgfsys@useobject{currentmarker}{}%
\end{pgfscope}%
\end{pgfscope}%
\begin{pgfscope}%
\definecolor{textcolor}{rgb}{0.000000,0.000000,0.000000}%
\pgfsetstrokecolor{textcolor}%
\pgfsetfillcolor{textcolor}%
\pgftext[x=0.228889in, y=1.503750in, left, base]{\color{textcolor}\rmfamily\fontsize{10.000000}{12.000000}\selectfont \(\displaystyle {1.0}\)}%
\end{pgfscope}%
\begin{pgfscope}%
\definecolor{textcolor}{rgb}{0.000000,0.000000,0.000000}%
\pgfsetstrokecolor{textcolor}%
\pgfsetfillcolor{textcolor}%
\pgftext[x=0.173333in,y=1.026944in,,bottom,rotate=90.000000]{\color{textcolor}\rmfamily\fontsize{10.000000}{12.000000}\selectfont True positive rate}%
\end{pgfscope}%
\begin{pgfscope}%
\pgfpathrectangle{\pgfqpoint{0.503581in}{0.449444in}}{\pgfqpoint{1.550000in}{1.155000in}}%
\pgfusepath{clip}%
\pgfsetbuttcap%
\pgfsetroundjoin%
\pgfsetlinewidth{1.505625pt}%
\definecolor{currentstroke}{rgb}{0.000000,0.000000,0.000000}%
\pgfsetstrokecolor{currentstroke}%
\pgfsetdash{{5.550000pt}{2.400000pt}}{0.000000pt}%
\pgfpathmoveto{\pgfqpoint{0.574035in}{0.501944in}}%
\pgfpathlineto{\pgfqpoint{1.983126in}{1.551944in}}%
\pgfusepath{stroke}%
\end{pgfscope}%
\begin{pgfscope}%
\pgfpathrectangle{\pgfqpoint{0.503581in}{0.449444in}}{\pgfqpoint{1.550000in}{1.155000in}}%
\pgfusepath{clip}%
\pgfsetrectcap%
\pgfsetroundjoin%
\pgfsetlinewidth{1.505625pt}%
\definecolor{currentstroke}{rgb}{0.000000,0.000000,0.000000}%
\pgfsetstrokecolor{currentstroke}%
\pgfsetdash{}{0pt}%
\pgfpathmoveto{\pgfqpoint{0.574035in}{0.501944in}}%
\pgfpathlineto{\pgfqpoint{0.575529in}{0.525520in}}%
\pgfpathlineto{\pgfqpoint{0.584003in}{0.607736in}}%
\pgfpathlineto{\pgfqpoint{0.584112in}{0.608660in}}%
\pgfpathlineto{\pgfqpoint{0.590376in}{0.652729in}}%
\pgfpathlineto{\pgfqpoint{0.602228in}{0.715823in}}%
\pgfpathlineto{\pgfqpoint{0.614539in}{0.766380in}}%
\pgfpathlineto{\pgfqpoint{0.624544in}{0.800421in}}%
\pgfpathlineto{\pgfqpoint{0.643095in}{0.853594in}}%
\pgfpathlineto{\pgfqpoint{0.666765in}{0.909189in}}%
\pgfpathlineto{\pgfqpoint{0.695595in}{0.965912in}}%
\pgfpathlineto{\pgfqpoint{0.717587in}{1.004506in}}%
\pgfpathlineto{\pgfqpoint{0.729912in}{1.024269in}}%
\pgfpathlineto{\pgfqpoint{0.771207in}{1.081605in}}%
\pgfpathlineto{\pgfqpoint{0.802878in}{1.119878in}}%
\pgfpathlineto{\pgfqpoint{0.836798in}{1.155894in}}%
\pgfpathlineto{\pgfqpoint{0.854623in}{1.174131in}}%
\pgfpathlineto{\pgfqpoint{0.891904in}{1.208046in}}%
\pgfpathlineto{\pgfqpoint{0.893144in}{1.209125in}}%
\pgfpathlineto{\pgfqpoint{0.893211in}{1.209174in}}%
\pgfpathlineto{\pgfqpoint{0.933666in}{1.242350in}}%
\pgfpathlineto{\pgfqpoint{0.976452in}{1.273775in}}%
\pgfpathlineto{\pgfqpoint{1.021947in}{1.303683in}}%
\pgfpathlineto{\pgfqpoint{1.069419in}{1.332395in}}%
\pgfpathlineto{\pgfqpoint{1.093629in}{1.346809in}}%
\pgfpathlineto{\pgfqpoint{1.144426in}{1.371874in}}%
\pgfpathlineto{\pgfqpoint{1.196496in}{1.395333in}}%
\pgfpathlineto{\pgfqpoint{1.249983in}{1.417567in}}%
\pgfpathlineto{\pgfqpoint{1.329327in}{1.445374in}}%
\pgfpathlineto{\pgfqpoint{1.331179in}{1.445987in}}%
\pgfpathlineto{\pgfqpoint{1.385540in}{1.462804in}}%
\pgfpathlineto{\pgfqpoint{1.439360in}{1.477899in}}%
\pgfpathlineto{\pgfqpoint{1.493496in}{1.490523in}}%
\pgfpathlineto{\pgfqpoint{1.570799in}{1.506494in}}%
\pgfpathlineto{\pgfqpoint{1.572516in}{1.506649in}}%
\pgfpathlineto{\pgfqpoint{1.648759in}{1.519906in}}%
\pgfpathlineto{\pgfqpoint{1.721113in}{1.530615in}}%
\pgfpathlineto{\pgfqpoint{1.807356in}{1.540613in}}%
\pgfpathlineto{\pgfqpoint{1.880338in}{1.546663in}}%
\pgfpathlineto{\pgfqpoint{1.968738in}{1.551536in}}%
\pgfpathlineto{\pgfqpoint{1.983126in}{1.551944in}}%
\pgfpathlineto{\pgfqpoint{1.983126in}{1.551944in}}%
\pgfusepath{stroke}%
\end{pgfscope}%
\begin{pgfscope}%
\pgfsetrectcap%
\pgfsetmiterjoin%
\pgfsetlinewidth{0.803000pt}%
\definecolor{currentstroke}{rgb}{0.000000,0.000000,0.000000}%
\pgfsetstrokecolor{currentstroke}%
\pgfsetdash{}{0pt}%
\pgfpathmoveto{\pgfqpoint{0.503581in}{0.449444in}}%
\pgfpathlineto{\pgfqpoint{0.503581in}{1.604444in}}%
\pgfusepath{stroke}%
\end{pgfscope}%
\begin{pgfscope}%
\pgfsetrectcap%
\pgfsetmiterjoin%
\pgfsetlinewidth{0.803000pt}%
\definecolor{currentstroke}{rgb}{0.000000,0.000000,0.000000}%
\pgfsetstrokecolor{currentstroke}%
\pgfsetdash{}{0pt}%
\pgfpathmoveto{\pgfqpoint{2.053581in}{0.449444in}}%
\pgfpathlineto{\pgfqpoint{2.053581in}{1.604444in}}%
\pgfusepath{stroke}%
\end{pgfscope}%
\begin{pgfscope}%
\pgfsetrectcap%
\pgfsetmiterjoin%
\pgfsetlinewidth{0.803000pt}%
\definecolor{currentstroke}{rgb}{0.000000,0.000000,0.000000}%
\pgfsetstrokecolor{currentstroke}%
\pgfsetdash{}{0pt}%
\pgfpathmoveto{\pgfqpoint{0.503581in}{0.449444in}}%
\pgfpathlineto{\pgfqpoint{2.053581in}{0.449444in}}%
\pgfusepath{stroke}%
\end{pgfscope}%
\begin{pgfscope}%
\pgfsetrectcap%
\pgfsetmiterjoin%
\pgfsetlinewidth{0.803000pt}%
\definecolor{currentstroke}{rgb}{0.000000,0.000000,0.000000}%
\pgfsetstrokecolor{currentstroke}%
\pgfsetdash{}{0pt}%
\pgfpathmoveto{\pgfqpoint{0.503581in}{1.604444in}}%
\pgfpathlineto{\pgfqpoint{2.053581in}{1.604444in}}%
\pgfusepath{stroke}%
\end{pgfscope}%
\begin{pgfscope}%
\pgfsetbuttcap%
\pgfsetmiterjoin%
\definecolor{currentfill}{rgb}{1.000000,1.000000,1.000000}%
\pgfsetfillcolor{currentfill}%
\pgfsetfillopacity{0.800000}%
\pgfsetlinewidth{1.003750pt}%
\definecolor{currentstroke}{rgb}{0.800000,0.800000,0.800000}%
\pgfsetstrokecolor{currentstroke}%
\pgfsetstrokeopacity{0.800000}%
\pgfsetdash{}{0pt}%
\pgfpathmoveto{\pgfqpoint{0.782747in}{0.518889in}}%
\pgfpathlineto{\pgfqpoint{1.956358in}{0.518889in}}%
\pgfpathquadraticcurveto{\pgfqpoint{1.984136in}{0.518889in}}{\pgfqpoint{1.984136in}{0.546666in}}%
\pgfpathlineto{\pgfqpoint{1.984136in}{0.726388in}}%
\pgfpathquadraticcurveto{\pgfqpoint{1.984136in}{0.754166in}}{\pgfqpoint{1.956358in}{0.754166in}}%
\pgfpathlineto{\pgfqpoint{0.782747in}{0.754166in}}%
\pgfpathquadraticcurveto{\pgfqpoint{0.754970in}{0.754166in}}{\pgfqpoint{0.754970in}{0.726388in}}%
\pgfpathlineto{\pgfqpoint{0.754970in}{0.546666in}}%
\pgfpathquadraticcurveto{\pgfqpoint{0.754970in}{0.518889in}}{\pgfqpoint{0.782747in}{0.518889in}}%
\pgfpathlineto{\pgfqpoint{0.782747in}{0.518889in}}%
\pgfpathclose%
\pgfusepath{stroke,fill}%
\end{pgfscope}%
\begin{pgfscope}%
\pgfsetrectcap%
\pgfsetroundjoin%
\pgfsetlinewidth{1.505625pt}%
\definecolor{currentstroke}{rgb}{0.000000,0.000000,0.000000}%
\pgfsetstrokecolor{currentstroke}%
\pgfsetdash{}{0pt}%
\pgfpathmoveto{\pgfqpoint{0.810525in}{0.650000in}}%
\pgfpathlineto{\pgfqpoint{0.949414in}{0.650000in}}%
\pgfpathlineto{\pgfqpoint{1.088303in}{0.650000in}}%
\pgfusepath{stroke}%
\end{pgfscope}%
\begin{pgfscope}%
\definecolor{textcolor}{rgb}{0.000000,0.000000,0.000000}%
\pgfsetstrokecolor{textcolor}%
\pgfsetfillcolor{textcolor}%
\pgftext[x=1.199414in,y=0.601388in,left,base]{\color{textcolor}\rmfamily\fontsize{10.000000}{12.000000}\selectfont AUC=0.801}%
\end{pgfscope}%
\end{pgfpicture}%
\makeatother%
\endgroup%

\end{tabular}

\

\

%
\verb|AdaBoost_5_Fold_Hard_Test|

\

In this model the values are clustered very tightly, but in that small range the 214,070 samples return 210,442 different values of $p$, so there is much diversity that we can't see in this representation.  


\

\verb|AdaBoost_5_Fold_Hard_Test|



\noindent\begin{tabular}{@{\hspace{-6pt}}p{4.3in} @{\hspace{-6pt}}p{2.0in}}
	\vskip 0pt
	\hfil Raw Model Output
	
	%% Creator: Matplotlib, PGF backend
%%
%% To include the figure in your LaTeX document, write
%%   \input{<filename>.pgf}
%%
%% Make sure the required packages are loaded in your preamble
%%   \usepackage{pgf}
%%
%% Also ensure that all the required font packages are loaded; for instance,
%% the lmodern package is sometimes necessary when using math font.
%%   \usepackage{lmodern}
%%
%% Figures using additional raster images can only be included by \input if
%% they are in the same directory as the main LaTeX file. For loading figures
%% from other directories you can use the `import` package
%%   \usepackage{import}
%%
%% and then include the figures with
%%   \import{<path to file>}{<filename>.pgf}
%%
%% Matplotlib used the following preamble
%%   
%%   \usepackage{fontspec}
%%   \makeatletter\@ifpackageloaded{underscore}{}{\usepackage[strings]{underscore}}\makeatother
%%
\begingroup%
\makeatletter%
\begin{pgfpicture}%
\pgfpathrectangle{\pgfpointorigin}{\pgfqpoint{4.002500in}{1.654444in}}%
\pgfusepath{use as bounding box, clip}%
\begin{pgfscope}%
\pgfsetbuttcap%
\pgfsetmiterjoin%
\definecolor{currentfill}{rgb}{1.000000,1.000000,1.000000}%
\pgfsetfillcolor{currentfill}%
\pgfsetlinewidth{0.000000pt}%
\definecolor{currentstroke}{rgb}{1.000000,1.000000,1.000000}%
\pgfsetstrokecolor{currentstroke}%
\pgfsetdash{}{0pt}%
\pgfpathmoveto{\pgfqpoint{0.000000in}{0.000000in}}%
\pgfpathlineto{\pgfqpoint{4.002500in}{0.000000in}}%
\pgfpathlineto{\pgfqpoint{4.002500in}{1.654444in}}%
\pgfpathlineto{\pgfqpoint{0.000000in}{1.654444in}}%
\pgfpathlineto{\pgfqpoint{0.000000in}{0.000000in}}%
\pgfpathclose%
\pgfusepath{fill}%
\end{pgfscope}%
\begin{pgfscope}%
\pgfsetbuttcap%
\pgfsetmiterjoin%
\definecolor{currentfill}{rgb}{1.000000,1.000000,1.000000}%
\pgfsetfillcolor{currentfill}%
\pgfsetlinewidth{0.000000pt}%
\definecolor{currentstroke}{rgb}{0.000000,0.000000,0.000000}%
\pgfsetstrokecolor{currentstroke}%
\pgfsetstrokeopacity{0.000000}%
\pgfsetdash{}{0pt}%
\pgfpathmoveto{\pgfqpoint{0.465000in}{0.449444in}}%
\pgfpathlineto{\pgfqpoint{3.952500in}{0.449444in}}%
\pgfpathlineto{\pgfqpoint{3.952500in}{1.604444in}}%
\pgfpathlineto{\pgfqpoint{0.465000in}{1.604444in}}%
\pgfpathlineto{\pgfqpoint{0.465000in}{0.449444in}}%
\pgfpathclose%
\pgfusepath{fill}%
\end{pgfscope}%
\begin{pgfscope}%
\pgfpathrectangle{\pgfqpoint{0.465000in}{0.449444in}}{\pgfqpoint{3.487500in}{1.155000in}}%
\pgfusepath{clip}%
\pgfsetbuttcap%
\pgfsetmiterjoin%
\pgfsetlinewidth{1.003750pt}%
\definecolor{currentstroke}{rgb}{0.000000,0.000000,0.000000}%
\pgfsetstrokecolor{currentstroke}%
\pgfsetdash{}{0pt}%
\pgfpathmoveto{\pgfqpoint{0.560114in}{0.449444in}}%
\pgfpathlineto{\pgfqpoint{0.623523in}{0.449444in}}%
\pgfpathlineto{\pgfqpoint{0.623523in}{0.449444in}}%
\pgfpathlineto{\pgfqpoint{0.560114in}{0.449444in}}%
\pgfpathlineto{\pgfqpoint{0.560114in}{0.449444in}}%
\pgfpathclose%
\pgfusepath{stroke}%
\end{pgfscope}%
\begin{pgfscope}%
\pgfpathrectangle{\pgfqpoint{0.465000in}{0.449444in}}{\pgfqpoint{3.487500in}{1.155000in}}%
\pgfusepath{clip}%
\pgfsetbuttcap%
\pgfsetmiterjoin%
\pgfsetlinewidth{1.003750pt}%
\definecolor{currentstroke}{rgb}{0.000000,0.000000,0.000000}%
\pgfsetstrokecolor{currentstroke}%
\pgfsetdash{}{0pt}%
\pgfpathmoveto{\pgfqpoint{0.718637in}{0.449444in}}%
\pgfpathlineto{\pgfqpoint{0.782046in}{0.449444in}}%
\pgfpathlineto{\pgfqpoint{0.782046in}{0.449444in}}%
\pgfpathlineto{\pgfqpoint{0.718637in}{0.449444in}}%
\pgfpathlineto{\pgfqpoint{0.718637in}{0.449444in}}%
\pgfpathclose%
\pgfusepath{stroke}%
\end{pgfscope}%
\begin{pgfscope}%
\pgfpathrectangle{\pgfqpoint{0.465000in}{0.449444in}}{\pgfqpoint{3.487500in}{1.155000in}}%
\pgfusepath{clip}%
\pgfsetbuttcap%
\pgfsetmiterjoin%
\pgfsetlinewidth{1.003750pt}%
\definecolor{currentstroke}{rgb}{0.000000,0.000000,0.000000}%
\pgfsetstrokecolor{currentstroke}%
\pgfsetdash{}{0pt}%
\pgfpathmoveto{\pgfqpoint{0.877159in}{0.449444in}}%
\pgfpathlineto{\pgfqpoint{0.940568in}{0.449444in}}%
\pgfpathlineto{\pgfqpoint{0.940568in}{0.449444in}}%
\pgfpathlineto{\pgfqpoint{0.877159in}{0.449444in}}%
\pgfpathlineto{\pgfqpoint{0.877159in}{0.449444in}}%
\pgfpathclose%
\pgfusepath{stroke}%
\end{pgfscope}%
\begin{pgfscope}%
\pgfpathrectangle{\pgfqpoint{0.465000in}{0.449444in}}{\pgfqpoint{3.487500in}{1.155000in}}%
\pgfusepath{clip}%
\pgfsetbuttcap%
\pgfsetmiterjoin%
\pgfsetlinewidth{1.003750pt}%
\definecolor{currentstroke}{rgb}{0.000000,0.000000,0.000000}%
\pgfsetstrokecolor{currentstroke}%
\pgfsetdash{}{0pt}%
\pgfpathmoveto{\pgfqpoint{1.035682in}{0.449444in}}%
\pgfpathlineto{\pgfqpoint{1.099091in}{0.449444in}}%
\pgfpathlineto{\pgfqpoint{1.099091in}{0.449444in}}%
\pgfpathlineto{\pgfqpoint{1.035682in}{0.449444in}}%
\pgfpathlineto{\pgfqpoint{1.035682in}{0.449444in}}%
\pgfpathclose%
\pgfusepath{stroke}%
\end{pgfscope}%
\begin{pgfscope}%
\pgfpathrectangle{\pgfqpoint{0.465000in}{0.449444in}}{\pgfqpoint{3.487500in}{1.155000in}}%
\pgfusepath{clip}%
\pgfsetbuttcap%
\pgfsetmiterjoin%
\pgfsetlinewidth{1.003750pt}%
\definecolor{currentstroke}{rgb}{0.000000,0.000000,0.000000}%
\pgfsetstrokecolor{currentstroke}%
\pgfsetdash{}{0pt}%
\pgfpathmoveto{\pgfqpoint{1.194205in}{0.449444in}}%
\pgfpathlineto{\pgfqpoint{1.257614in}{0.449444in}}%
\pgfpathlineto{\pgfqpoint{1.257614in}{0.449444in}}%
\pgfpathlineto{\pgfqpoint{1.194205in}{0.449444in}}%
\pgfpathlineto{\pgfqpoint{1.194205in}{0.449444in}}%
\pgfpathclose%
\pgfusepath{stroke}%
\end{pgfscope}%
\begin{pgfscope}%
\pgfpathrectangle{\pgfqpoint{0.465000in}{0.449444in}}{\pgfqpoint{3.487500in}{1.155000in}}%
\pgfusepath{clip}%
\pgfsetbuttcap%
\pgfsetmiterjoin%
\pgfsetlinewidth{1.003750pt}%
\definecolor{currentstroke}{rgb}{0.000000,0.000000,0.000000}%
\pgfsetstrokecolor{currentstroke}%
\pgfsetdash{}{0pt}%
\pgfpathmoveto{\pgfqpoint{1.352728in}{0.449444in}}%
\pgfpathlineto{\pgfqpoint{1.416137in}{0.449444in}}%
\pgfpathlineto{\pgfqpoint{1.416137in}{0.449444in}}%
\pgfpathlineto{\pgfqpoint{1.352728in}{0.449444in}}%
\pgfpathlineto{\pgfqpoint{1.352728in}{0.449444in}}%
\pgfpathclose%
\pgfusepath{stroke}%
\end{pgfscope}%
\begin{pgfscope}%
\pgfpathrectangle{\pgfqpoint{0.465000in}{0.449444in}}{\pgfqpoint{3.487500in}{1.155000in}}%
\pgfusepath{clip}%
\pgfsetbuttcap%
\pgfsetmiterjoin%
\pgfsetlinewidth{1.003750pt}%
\definecolor{currentstroke}{rgb}{0.000000,0.000000,0.000000}%
\pgfsetstrokecolor{currentstroke}%
\pgfsetdash{}{0pt}%
\pgfpathmoveto{\pgfqpoint{1.511250in}{0.449444in}}%
\pgfpathlineto{\pgfqpoint{1.574659in}{0.449444in}}%
\pgfpathlineto{\pgfqpoint{1.574659in}{0.449444in}}%
\pgfpathlineto{\pgfqpoint{1.511250in}{0.449444in}}%
\pgfpathlineto{\pgfqpoint{1.511250in}{0.449444in}}%
\pgfpathclose%
\pgfusepath{stroke}%
\end{pgfscope}%
\begin{pgfscope}%
\pgfpathrectangle{\pgfqpoint{0.465000in}{0.449444in}}{\pgfqpoint{3.487500in}{1.155000in}}%
\pgfusepath{clip}%
\pgfsetbuttcap%
\pgfsetmiterjoin%
\pgfsetlinewidth{1.003750pt}%
\definecolor{currentstroke}{rgb}{0.000000,0.000000,0.000000}%
\pgfsetstrokecolor{currentstroke}%
\pgfsetdash{}{0pt}%
\pgfpathmoveto{\pgfqpoint{1.669773in}{0.449444in}}%
\pgfpathlineto{\pgfqpoint{1.733182in}{0.449444in}}%
\pgfpathlineto{\pgfqpoint{1.733182in}{0.449444in}}%
\pgfpathlineto{\pgfqpoint{1.669773in}{0.449444in}}%
\pgfpathlineto{\pgfqpoint{1.669773in}{0.449444in}}%
\pgfpathclose%
\pgfusepath{stroke}%
\end{pgfscope}%
\begin{pgfscope}%
\pgfpathrectangle{\pgfqpoint{0.465000in}{0.449444in}}{\pgfqpoint{3.487500in}{1.155000in}}%
\pgfusepath{clip}%
\pgfsetbuttcap%
\pgfsetmiterjoin%
\pgfsetlinewidth{1.003750pt}%
\definecolor{currentstroke}{rgb}{0.000000,0.000000,0.000000}%
\pgfsetstrokecolor{currentstroke}%
\pgfsetdash{}{0pt}%
\pgfpathmoveto{\pgfqpoint{1.828296in}{0.449444in}}%
\pgfpathlineto{\pgfqpoint{1.891705in}{0.449444in}}%
\pgfpathlineto{\pgfqpoint{1.891705in}{0.449444in}}%
\pgfpathlineto{\pgfqpoint{1.828296in}{0.449444in}}%
\pgfpathlineto{\pgfqpoint{1.828296in}{0.449444in}}%
\pgfpathclose%
\pgfusepath{stroke}%
\end{pgfscope}%
\begin{pgfscope}%
\pgfpathrectangle{\pgfqpoint{0.465000in}{0.449444in}}{\pgfqpoint{3.487500in}{1.155000in}}%
\pgfusepath{clip}%
\pgfsetbuttcap%
\pgfsetmiterjoin%
\pgfsetlinewidth{1.003750pt}%
\definecolor{currentstroke}{rgb}{0.000000,0.000000,0.000000}%
\pgfsetstrokecolor{currentstroke}%
\pgfsetdash{}{0pt}%
\pgfpathmoveto{\pgfqpoint{1.986818in}{0.449444in}}%
\pgfpathlineto{\pgfqpoint{2.050228in}{0.449444in}}%
\pgfpathlineto{\pgfqpoint{2.050228in}{0.449444in}}%
\pgfpathlineto{\pgfqpoint{1.986818in}{0.449444in}}%
\pgfpathlineto{\pgfqpoint{1.986818in}{0.449444in}}%
\pgfpathclose%
\pgfusepath{stroke}%
\end{pgfscope}%
\begin{pgfscope}%
\pgfpathrectangle{\pgfqpoint{0.465000in}{0.449444in}}{\pgfqpoint{3.487500in}{1.155000in}}%
\pgfusepath{clip}%
\pgfsetbuttcap%
\pgfsetmiterjoin%
\pgfsetlinewidth{1.003750pt}%
\definecolor{currentstroke}{rgb}{0.000000,0.000000,0.000000}%
\pgfsetstrokecolor{currentstroke}%
\pgfsetdash{}{0pt}%
\pgfpathmoveto{\pgfqpoint{2.145341in}{0.449444in}}%
\pgfpathlineto{\pgfqpoint{2.208750in}{0.449444in}}%
\pgfpathlineto{\pgfqpoint{2.208750in}{1.549444in}}%
\pgfpathlineto{\pgfqpoint{2.145341in}{1.549444in}}%
\pgfpathlineto{\pgfqpoint{2.145341in}{0.449444in}}%
\pgfpathclose%
\pgfusepath{stroke}%
\end{pgfscope}%
\begin{pgfscope}%
\pgfpathrectangle{\pgfqpoint{0.465000in}{0.449444in}}{\pgfqpoint{3.487500in}{1.155000in}}%
\pgfusepath{clip}%
\pgfsetbuttcap%
\pgfsetmiterjoin%
\pgfsetlinewidth{1.003750pt}%
\definecolor{currentstroke}{rgb}{0.000000,0.000000,0.000000}%
\pgfsetstrokecolor{currentstroke}%
\pgfsetdash{}{0pt}%
\pgfpathmoveto{\pgfqpoint{2.303864in}{0.449444in}}%
\pgfpathlineto{\pgfqpoint{2.367273in}{0.449444in}}%
\pgfpathlineto{\pgfqpoint{2.367273in}{0.466924in}}%
\pgfpathlineto{\pgfqpoint{2.303864in}{0.466924in}}%
\pgfpathlineto{\pgfqpoint{2.303864in}{0.449444in}}%
\pgfpathclose%
\pgfusepath{stroke}%
\end{pgfscope}%
\begin{pgfscope}%
\pgfpathrectangle{\pgfqpoint{0.465000in}{0.449444in}}{\pgfqpoint{3.487500in}{1.155000in}}%
\pgfusepath{clip}%
\pgfsetbuttcap%
\pgfsetmiterjoin%
\pgfsetlinewidth{1.003750pt}%
\definecolor{currentstroke}{rgb}{0.000000,0.000000,0.000000}%
\pgfsetstrokecolor{currentstroke}%
\pgfsetdash{}{0pt}%
\pgfpathmoveto{\pgfqpoint{2.462387in}{0.449444in}}%
\pgfpathlineto{\pgfqpoint{2.525796in}{0.449444in}}%
\pgfpathlineto{\pgfqpoint{2.525796in}{0.449444in}}%
\pgfpathlineto{\pgfqpoint{2.462387in}{0.449444in}}%
\pgfpathlineto{\pgfqpoint{2.462387in}{0.449444in}}%
\pgfpathclose%
\pgfusepath{stroke}%
\end{pgfscope}%
\begin{pgfscope}%
\pgfpathrectangle{\pgfqpoint{0.465000in}{0.449444in}}{\pgfqpoint{3.487500in}{1.155000in}}%
\pgfusepath{clip}%
\pgfsetbuttcap%
\pgfsetmiterjoin%
\pgfsetlinewidth{1.003750pt}%
\definecolor{currentstroke}{rgb}{0.000000,0.000000,0.000000}%
\pgfsetstrokecolor{currentstroke}%
\pgfsetdash{}{0pt}%
\pgfpathmoveto{\pgfqpoint{2.620909in}{0.449444in}}%
\pgfpathlineto{\pgfqpoint{2.684318in}{0.449444in}}%
\pgfpathlineto{\pgfqpoint{2.684318in}{0.449444in}}%
\pgfpathlineto{\pgfqpoint{2.620909in}{0.449444in}}%
\pgfpathlineto{\pgfqpoint{2.620909in}{0.449444in}}%
\pgfpathclose%
\pgfusepath{stroke}%
\end{pgfscope}%
\begin{pgfscope}%
\pgfpathrectangle{\pgfqpoint{0.465000in}{0.449444in}}{\pgfqpoint{3.487500in}{1.155000in}}%
\pgfusepath{clip}%
\pgfsetbuttcap%
\pgfsetmiterjoin%
\pgfsetlinewidth{1.003750pt}%
\definecolor{currentstroke}{rgb}{0.000000,0.000000,0.000000}%
\pgfsetstrokecolor{currentstroke}%
\pgfsetdash{}{0pt}%
\pgfpathmoveto{\pgfqpoint{2.779432in}{0.449444in}}%
\pgfpathlineto{\pgfqpoint{2.842841in}{0.449444in}}%
\pgfpathlineto{\pgfqpoint{2.842841in}{0.449444in}}%
\pgfpathlineto{\pgfqpoint{2.779432in}{0.449444in}}%
\pgfpathlineto{\pgfqpoint{2.779432in}{0.449444in}}%
\pgfpathclose%
\pgfusepath{stroke}%
\end{pgfscope}%
\begin{pgfscope}%
\pgfpathrectangle{\pgfqpoint{0.465000in}{0.449444in}}{\pgfqpoint{3.487500in}{1.155000in}}%
\pgfusepath{clip}%
\pgfsetbuttcap%
\pgfsetmiterjoin%
\pgfsetlinewidth{1.003750pt}%
\definecolor{currentstroke}{rgb}{0.000000,0.000000,0.000000}%
\pgfsetstrokecolor{currentstroke}%
\pgfsetdash{}{0pt}%
\pgfpathmoveto{\pgfqpoint{2.937955in}{0.449444in}}%
\pgfpathlineto{\pgfqpoint{3.001364in}{0.449444in}}%
\pgfpathlineto{\pgfqpoint{3.001364in}{0.449444in}}%
\pgfpathlineto{\pgfqpoint{2.937955in}{0.449444in}}%
\pgfpathlineto{\pgfqpoint{2.937955in}{0.449444in}}%
\pgfpathclose%
\pgfusepath{stroke}%
\end{pgfscope}%
\begin{pgfscope}%
\pgfpathrectangle{\pgfqpoint{0.465000in}{0.449444in}}{\pgfqpoint{3.487500in}{1.155000in}}%
\pgfusepath{clip}%
\pgfsetbuttcap%
\pgfsetmiterjoin%
\pgfsetlinewidth{1.003750pt}%
\definecolor{currentstroke}{rgb}{0.000000,0.000000,0.000000}%
\pgfsetstrokecolor{currentstroke}%
\pgfsetdash{}{0pt}%
\pgfpathmoveto{\pgfqpoint{3.096478in}{0.449444in}}%
\pgfpathlineto{\pgfqpoint{3.159887in}{0.449444in}}%
\pgfpathlineto{\pgfqpoint{3.159887in}{0.449444in}}%
\pgfpathlineto{\pgfqpoint{3.096478in}{0.449444in}}%
\pgfpathlineto{\pgfqpoint{3.096478in}{0.449444in}}%
\pgfpathclose%
\pgfusepath{stroke}%
\end{pgfscope}%
\begin{pgfscope}%
\pgfpathrectangle{\pgfqpoint{0.465000in}{0.449444in}}{\pgfqpoint{3.487500in}{1.155000in}}%
\pgfusepath{clip}%
\pgfsetbuttcap%
\pgfsetmiterjoin%
\pgfsetlinewidth{1.003750pt}%
\definecolor{currentstroke}{rgb}{0.000000,0.000000,0.000000}%
\pgfsetstrokecolor{currentstroke}%
\pgfsetdash{}{0pt}%
\pgfpathmoveto{\pgfqpoint{3.255000in}{0.449444in}}%
\pgfpathlineto{\pgfqpoint{3.318409in}{0.449444in}}%
\pgfpathlineto{\pgfqpoint{3.318409in}{0.449444in}}%
\pgfpathlineto{\pgfqpoint{3.255000in}{0.449444in}}%
\pgfpathlineto{\pgfqpoint{3.255000in}{0.449444in}}%
\pgfpathclose%
\pgfusepath{stroke}%
\end{pgfscope}%
\begin{pgfscope}%
\pgfpathrectangle{\pgfqpoint{0.465000in}{0.449444in}}{\pgfqpoint{3.487500in}{1.155000in}}%
\pgfusepath{clip}%
\pgfsetbuttcap%
\pgfsetmiterjoin%
\pgfsetlinewidth{1.003750pt}%
\definecolor{currentstroke}{rgb}{0.000000,0.000000,0.000000}%
\pgfsetstrokecolor{currentstroke}%
\pgfsetdash{}{0pt}%
\pgfpathmoveto{\pgfqpoint{3.413523in}{0.449444in}}%
\pgfpathlineto{\pgfqpoint{3.476932in}{0.449444in}}%
\pgfpathlineto{\pgfqpoint{3.476932in}{0.449444in}}%
\pgfpathlineto{\pgfqpoint{3.413523in}{0.449444in}}%
\pgfpathlineto{\pgfqpoint{3.413523in}{0.449444in}}%
\pgfpathclose%
\pgfusepath{stroke}%
\end{pgfscope}%
\begin{pgfscope}%
\pgfpathrectangle{\pgfqpoint{0.465000in}{0.449444in}}{\pgfqpoint{3.487500in}{1.155000in}}%
\pgfusepath{clip}%
\pgfsetbuttcap%
\pgfsetmiterjoin%
\pgfsetlinewidth{1.003750pt}%
\definecolor{currentstroke}{rgb}{0.000000,0.000000,0.000000}%
\pgfsetstrokecolor{currentstroke}%
\pgfsetdash{}{0pt}%
\pgfpathmoveto{\pgfqpoint{3.572046in}{0.449444in}}%
\pgfpathlineto{\pgfqpoint{3.635455in}{0.449444in}}%
\pgfpathlineto{\pgfqpoint{3.635455in}{0.449444in}}%
\pgfpathlineto{\pgfqpoint{3.572046in}{0.449444in}}%
\pgfpathlineto{\pgfqpoint{3.572046in}{0.449444in}}%
\pgfpathclose%
\pgfusepath{stroke}%
\end{pgfscope}%
\begin{pgfscope}%
\pgfpathrectangle{\pgfqpoint{0.465000in}{0.449444in}}{\pgfqpoint{3.487500in}{1.155000in}}%
\pgfusepath{clip}%
\pgfsetbuttcap%
\pgfsetmiterjoin%
\pgfsetlinewidth{1.003750pt}%
\definecolor{currentstroke}{rgb}{0.000000,0.000000,0.000000}%
\pgfsetstrokecolor{currentstroke}%
\pgfsetdash{}{0pt}%
\pgfpathmoveto{\pgfqpoint{3.730568in}{0.449444in}}%
\pgfpathlineto{\pgfqpoint{3.793978in}{0.449444in}}%
\pgfpathlineto{\pgfqpoint{3.793978in}{0.449444in}}%
\pgfpathlineto{\pgfqpoint{3.730568in}{0.449444in}}%
\pgfpathlineto{\pgfqpoint{3.730568in}{0.449444in}}%
\pgfpathclose%
\pgfusepath{stroke}%
\end{pgfscope}%
\begin{pgfscope}%
\pgfpathrectangle{\pgfqpoint{0.465000in}{0.449444in}}{\pgfqpoint{3.487500in}{1.155000in}}%
\pgfusepath{clip}%
\pgfsetbuttcap%
\pgfsetmiterjoin%
\definecolor{currentfill}{rgb}{0.000000,0.000000,0.000000}%
\pgfsetfillcolor{currentfill}%
\pgfsetlinewidth{0.000000pt}%
\definecolor{currentstroke}{rgb}{0.000000,0.000000,0.000000}%
\pgfsetstrokecolor{currentstroke}%
\pgfsetstrokeopacity{0.000000}%
\pgfsetdash{}{0pt}%
\pgfpathmoveto{\pgfqpoint{0.623523in}{0.449444in}}%
\pgfpathlineto{\pgfqpoint{0.686932in}{0.449444in}}%
\pgfpathlineto{\pgfqpoint{0.686932in}{0.449444in}}%
\pgfpathlineto{\pgfqpoint{0.623523in}{0.449444in}}%
\pgfpathlineto{\pgfqpoint{0.623523in}{0.449444in}}%
\pgfpathclose%
\pgfusepath{fill}%
\end{pgfscope}%
\begin{pgfscope}%
\pgfpathrectangle{\pgfqpoint{0.465000in}{0.449444in}}{\pgfqpoint{3.487500in}{1.155000in}}%
\pgfusepath{clip}%
\pgfsetbuttcap%
\pgfsetmiterjoin%
\definecolor{currentfill}{rgb}{0.000000,0.000000,0.000000}%
\pgfsetfillcolor{currentfill}%
\pgfsetlinewidth{0.000000pt}%
\definecolor{currentstroke}{rgb}{0.000000,0.000000,0.000000}%
\pgfsetstrokecolor{currentstroke}%
\pgfsetstrokeopacity{0.000000}%
\pgfsetdash{}{0pt}%
\pgfpathmoveto{\pgfqpoint{0.782046in}{0.449444in}}%
\pgfpathlineto{\pgfqpoint{0.845455in}{0.449444in}}%
\pgfpathlineto{\pgfqpoint{0.845455in}{0.449444in}}%
\pgfpathlineto{\pgfqpoint{0.782046in}{0.449444in}}%
\pgfpathlineto{\pgfqpoint{0.782046in}{0.449444in}}%
\pgfpathclose%
\pgfusepath{fill}%
\end{pgfscope}%
\begin{pgfscope}%
\pgfpathrectangle{\pgfqpoint{0.465000in}{0.449444in}}{\pgfqpoint{3.487500in}{1.155000in}}%
\pgfusepath{clip}%
\pgfsetbuttcap%
\pgfsetmiterjoin%
\definecolor{currentfill}{rgb}{0.000000,0.000000,0.000000}%
\pgfsetfillcolor{currentfill}%
\pgfsetlinewidth{0.000000pt}%
\definecolor{currentstroke}{rgb}{0.000000,0.000000,0.000000}%
\pgfsetstrokecolor{currentstroke}%
\pgfsetstrokeopacity{0.000000}%
\pgfsetdash{}{0pt}%
\pgfpathmoveto{\pgfqpoint{0.940568in}{0.449444in}}%
\pgfpathlineto{\pgfqpoint{1.003978in}{0.449444in}}%
\pgfpathlineto{\pgfqpoint{1.003978in}{0.449444in}}%
\pgfpathlineto{\pgfqpoint{0.940568in}{0.449444in}}%
\pgfpathlineto{\pgfqpoint{0.940568in}{0.449444in}}%
\pgfpathclose%
\pgfusepath{fill}%
\end{pgfscope}%
\begin{pgfscope}%
\pgfpathrectangle{\pgfqpoint{0.465000in}{0.449444in}}{\pgfqpoint{3.487500in}{1.155000in}}%
\pgfusepath{clip}%
\pgfsetbuttcap%
\pgfsetmiterjoin%
\definecolor{currentfill}{rgb}{0.000000,0.000000,0.000000}%
\pgfsetfillcolor{currentfill}%
\pgfsetlinewidth{0.000000pt}%
\definecolor{currentstroke}{rgb}{0.000000,0.000000,0.000000}%
\pgfsetstrokecolor{currentstroke}%
\pgfsetstrokeopacity{0.000000}%
\pgfsetdash{}{0pt}%
\pgfpathmoveto{\pgfqpoint{1.099091in}{0.449444in}}%
\pgfpathlineto{\pgfqpoint{1.162500in}{0.449444in}}%
\pgfpathlineto{\pgfqpoint{1.162500in}{0.449444in}}%
\pgfpathlineto{\pgfqpoint{1.099091in}{0.449444in}}%
\pgfpathlineto{\pgfqpoint{1.099091in}{0.449444in}}%
\pgfpathclose%
\pgfusepath{fill}%
\end{pgfscope}%
\begin{pgfscope}%
\pgfpathrectangle{\pgfqpoint{0.465000in}{0.449444in}}{\pgfqpoint{3.487500in}{1.155000in}}%
\pgfusepath{clip}%
\pgfsetbuttcap%
\pgfsetmiterjoin%
\definecolor{currentfill}{rgb}{0.000000,0.000000,0.000000}%
\pgfsetfillcolor{currentfill}%
\pgfsetlinewidth{0.000000pt}%
\definecolor{currentstroke}{rgb}{0.000000,0.000000,0.000000}%
\pgfsetstrokecolor{currentstroke}%
\pgfsetstrokeopacity{0.000000}%
\pgfsetdash{}{0pt}%
\pgfpathmoveto{\pgfqpoint{1.257614in}{0.449444in}}%
\pgfpathlineto{\pgfqpoint{1.321023in}{0.449444in}}%
\pgfpathlineto{\pgfqpoint{1.321023in}{0.449444in}}%
\pgfpathlineto{\pgfqpoint{1.257614in}{0.449444in}}%
\pgfpathlineto{\pgfqpoint{1.257614in}{0.449444in}}%
\pgfpathclose%
\pgfusepath{fill}%
\end{pgfscope}%
\begin{pgfscope}%
\pgfpathrectangle{\pgfqpoint{0.465000in}{0.449444in}}{\pgfqpoint{3.487500in}{1.155000in}}%
\pgfusepath{clip}%
\pgfsetbuttcap%
\pgfsetmiterjoin%
\definecolor{currentfill}{rgb}{0.000000,0.000000,0.000000}%
\pgfsetfillcolor{currentfill}%
\pgfsetlinewidth{0.000000pt}%
\definecolor{currentstroke}{rgb}{0.000000,0.000000,0.000000}%
\pgfsetstrokecolor{currentstroke}%
\pgfsetstrokeopacity{0.000000}%
\pgfsetdash{}{0pt}%
\pgfpathmoveto{\pgfqpoint{1.416137in}{0.449444in}}%
\pgfpathlineto{\pgfqpoint{1.479546in}{0.449444in}}%
\pgfpathlineto{\pgfqpoint{1.479546in}{0.449444in}}%
\pgfpathlineto{\pgfqpoint{1.416137in}{0.449444in}}%
\pgfpathlineto{\pgfqpoint{1.416137in}{0.449444in}}%
\pgfpathclose%
\pgfusepath{fill}%
\end{pgfscope}%
\begin{pgfscope}%
\pgfpathrectangle{\pgfqpoint{0.465000in}{0.449444in}}{\pgfqpoint{3.487500in}{1.155000in}}%
\pgfusepath{clip}%
\pgfsetbuttcap%
\pgfsetmiterjoin%
\definecolor{currentfill}{rgb}{0.000000,0.000000,0.000000}%
\pgfsetfillcolor{currentfill}%
\pgfsetlinewidth{0.000000pt}%
\definecolor{currentstroke}{rgb}{0.000000,0.000000,0.000000}%
\pgfsetstrokecolor{currentstroke}%
\pgfsetstrokeopacity{0.000000}%
\pgfsetdash{}{0pt}%
\pgfpathmoveto{\pgfqpoint{1.574659in}{0.449444in}}%
\pgfpathlineto{\pgfqpoint{1.638068in}{0.449444in}}%
\pgfpathlineto{\pgfqpoint{1.638068in}{0.449444in}}%
\pgfpathlineto{\pgfqpoint{1.574659in}{0.449444in}}%
\pgfpathlineto{\pgfqpoint{1.574659in}{0.449444in}}%
\pgfpathclose%
\pgfusepath{fill}%
\end{pgfscope}%
\begin{pgfscope}%
\pgfpathrectangle{\pgfqpoint{0.465000in}{0.449444in}}{\pgfqpoint{3.487500in}{1.155000in}}%
\pgfusepath{clip}%
\pgfsetbuttcap%
\pgfsetmiterjoin%
\definecolor{currentfill}{rgb}{0.000000,0.000000,0.000000}%
\pgfsetfillcolor{currentfill}%
\pgfsetlinewidth{0.000000pt}%
\definecolor{currentstroke}{rgb}{0.000000,0.000000,0.000000}%
\pgfsetstrokecolor{currentstroke}%
\pgfsetstrokeopacity{0.000000}%
\pgfsetdash{}{0pt}%
\pgfpathmoveto{\pgfqpoint{1.733182in}{0.449444in}}%
\pgfpathlineto{\pgfqpoint{1.796591in}{0.449444in}}%
\pgfpathlineto{\pgfqpoint{1.796591in}{0.449444in}}%
\pgfpathlineto{\pgfqpoint{1.733182in}{0.449444in}}%
\pgfpathlineto{\pgfqpoint{1.733182in}{0.449444in}}%
\pgfpathclose%
\pgfusepath{fill}%
\end{pgfscope}%
\begin{pgfscope}%
\pgfpathrectangle{\pgfqpoint{0.465000in}{0.449444in}}{\pgfqpoint{3.487500in}{1.155000in}}%
\pgfusepath{clip}%
\pgfsetbuttcap%
\pgfsetmiterjoin%
\definecolor{currentfill}{rgb}{0.000000,0.000000,0.000000}%
\pgfsetfillcolor{currentfill}%
\pgfsetlinewidth{0.000000pt}%
\definecolor{currentstroke}{rgb}{0.000000,0.000000,0.000000}%
\pgfsetstrokecolor{currentstroke}%
\pgfsetstrokeopacity{0.000000}%
\pgfsetdash{}{0pt}%
\pgfpathmoveto{\pgfqpoint{1.891705in}{0.449444in}}%
\pgfpathlineto{\pgfqpoint{1.955114in}{0.449444in}}%
\pgfpathlineto{\pgfqpoint{1.955114in}{0.449444in}}%
\pgfpathlineto{\pgfqpoint{1.891705in}{0.449444in}}%
\pgfpathlineto{\pgfqpoint{1.891705in}{0.449444in}}%
\pgfpathclose%
\pgfusepath{fill}%
\end{pgfscope}%
\begin{pgfscope}%
\pgfpathrectangle{\pgfqpoint{0.465000in}{0.449444in}}{\pgfqpoint{3.487500in}{1.155000in}}%
\pgfusepath{clip}%
\pgfsetbuttcap%
\pgfsetmiterjoin%
\definecolor{currentfill}{rgb}{0.000000,0.000000,0.000000}%
\pgfsetfillcolor{currentfill}%
\pgfsetlinewidth{0.000000pt}%
\definecolor{currentstroke}{rgb}{0.000000,0.000000,0.000000}%
\pgfsetstrokecolor{currentstroke}%
\pgfsetstrokeopacity{0.000000}%
\pgfsetdash{}{0pt}%
\pgfpathmoveto{\pgfqpoint{2.050228in}{0.449444in}}%
\pgfpathlineto{\pgfqpoint{2.113637in}{0.449444in}}%
\pgfpathlineto{\pgfqpoint{2.113637in}{0.449444in}}%
\pgfpathlineto{\pgfqpoint{2.050228in}{0.449444in}}%
\pgfpathlineto{\pgfqpoint{2.050228in}{0.449444in}}%
\pgfpathclose%
\pgfusepath{fill}%
\end{pgfscope}%
\begin{pgfscope}%
\pgfpathrectangle{\pgfqpoint{0.465000in}{0.449444in}}{\pgfqpoint{3.487500in}{1.155000in}}%
\pgfusepath{clip}%
\pgfsetbuttcap%
\pgfsetmiterjoin%
\definecolor{currentfill}{rgb}{0.000000,0.000000,0.000000}%
\pgfsetfillcolor{currentfill}%
\pgfsetlinewidth{0.000000pt}%
\definecolor{currentstroke}{rgb}{0.000000,0.000000,0.000000}%
\pgfsetstrokecolor{currentstroke}%
\pgfsetstrokeopacity{0.000000}%
\pgfsetdash{}{0pt}%
\pgfpathmoveto{\pgfqpoint{2.208750in}{0.449444in}}%
\pgfpathlineto{\pgfqpoint{2.272159in}{0.449444in}}%
\pgfpathlineto{\pgfqpoint{2.272159in}{0.628961in}}%
\pgfpathlineto{\pgfqpoint{2.208750in}{0.628961in}}%
\pgfpathlineto{\pgfqpoint{2.208750in}{0.449444in}}%
\pgfpathclose%
\pgfusepath{fill}%
\end{pgfscope}%
\begin{pgfscope}%
\pgfpathrectangle{\pgfqpoint{0.465000in}{0.449444in}}{\pgfqpoint{3.487500in}{1.155000in}}%
\pgfusepath{clip}%
\pgfsetbuttcap%
\pgfsetmiterjoin%
\definecolor{currentfill}{rgb}{0.000000,0.000000,0.000000}%
\pgfsetfillcolor{currentfill}%
\pgfsetlinewidth{0.000000pt}%
\definecolor{currentstroke}{rgb}{0.000000,0.000000,0.000000}%
\pgfsetstrokecolor{currentstroke}%
\pgfsetstrokeopacity{0.000000}%
\pgfsetdash{}{0pt}%
\pgfpathmoveto{\pgfqpoint{2.367273in}{0.449444in}}%
\pgfpathlineto{\pgfqpoint{2.430682in}{0.449444in}}%
\pgfpathlineto{\pgfqpoint{2.430682in}{0.469129in}}%
\pgfpathlineto{\pgfqpoint{2.367273in}{0.469129in}}%
\pgfpathlineto{\pgfqpoint{2.367273in}{0.449444in}}%
\pgfpathclose%
\pgfusepath{fill}%
\end{pgfscope}%
\begin{pgfscope}%
\pgfpathrectangle{\pgfqpoint{0.465000in}{0.449444in}}{\pgfqpoint{3.487500in}{1.155000in}}%
\pgfusepath{clip}%
\pgfsetbuttcap%
\pgfsetmiterjoin%
\definecolor{currentfill}{rgb}{0.000000,0.000000,0.000000}%
\pgfsetfillcolor{currentfill}%
\pgfsetlinewidth{0.000000pt}%
\definecolor{currentstroke}{rgb}{0.000000,0.000000,0.000000}%
\pgfsetstrokecolor{currentstroke}%
\pgfsetstrokeopacity{0.000000}%
\pgfsetdash{}{0pt}%
\pgfpathmoveto{\pgfqpoint{2.525796in}{0.449444in}}%
\pgfpathlineto{\pgfqpoint{2.589205in}{0.449444in}}%
\pgfpathlineto{\pgfqpoint{2.589205in}{0.449444in}}%
\pgfpathlineto{\pgfqpoint{2.525796in}{0.449444in}}%
\pgfpathlineto{\pgfqpoint{2.525796in}{0.449444in}}%
\pgfpathclose%
\pgfusepath{fill}%
\end{pgfscope}%
\begin{pgfscope}%
\pgfpathrectangle{\pgfqpoint{0.465000in}{0.449444in}}{\pgfqpoint{3.487500in}{1.155000in}}%
\pgfusepath{clip}%
\pgfsetbuttcap%
\pgfsetmiterjoin%
\definecolor{currentfill}{rgb}{0.000000,0.000000,0.000000}%
\pgfsetfillcolor{currentfill}%
\pgfsetlinewidth{0.000000pt}%
\definecolor{currentstroke}{rgb}{0.000000,0.000000,0.000000}%
\pgfsetstrokecolor{currentstroke}%
\pgfsetstrokeopacity{0.000000}%
\pgfsetdash{}{0pt}%
\pgfpathmoveto{\pgfqpoint{2.684318in}{0.449444in}}%
\pgfpathlineto{\pgfqpoint{2.747728in}{0.449444in}}%
\pgfpathlineto{\pgfqpoint{2.747728in}{0.449444in}}%
\pgfpathlineto{\pgfqpoint{2.684318in}{0.449444in}}%
\pgfpathlineto{\pgfqpoint{2.684318in}{0.449444in}}%
\pgfpathclose%
\pgfusepath{fill}%
\end{pgfscope}%
\begin{pgfscope}%
\pgfpathrectangle{\pgfqpoint{0.465000in}{0.449444in}}{\pgfqpoint{3.487500in}{1.155000in}}%
\pgfusepath{clip}%
\pgfsetbuttcap%
\pgfsetmiterjoin%
\definecolor{currentfill}{rgb}{0.000000,0.000000,0.000000}%
\pgfsetfillcolor{currentfill}%
\pgfsetlinewidth{0.000000pt}%
\definecolor{currentstroke}{rgb}{0.000000,0.000000,0.000000}%
\pgfsetstrokecolor{currentstroke}%
\pgfsetstrokeopacity{0.000000}%
\pgfsetdash{}{0pt}%
\pgfpathmoveto{\pgfqpoint{2.842841in}{0.449444in}}%
\pgfpathlineto{\pgfqpoint{2.906250in}{0.449444in}}%
\pgfpathlineto{\pgfqpoint{2.906250in}{0.449444in}}%
\pgfpathlineto{\pgfqpoint{2.842841in}{0.449444in}}%
\pgfpathlineto{\pgfqpoint{2.842841in}{0.449444in}}%
\pgfpathclose%
\pgfusepath{fill}%
\end{pgfscope}%
\begin{pgfscope}%
\pgfpathrectangle{\pgfqpoint{0.465000in}{0.449444in}}{\pgfqpoint{3.487500in}{1.155000in}}%
\pgfusepath{clip}%
\pgfsetbuttcap%
\pgfsetmiterjoin%
\definecolor{currentfill}{rgb}{0.000000,0.000000,0.000000}%
\pgfsetfillcolor{currentfill}%
\pgfsetlinewidth{0.000000pt}%
\definecolor{currentstroke}{rgb}{0.000000,0.000000,0.000000}%
\pgfsetstrokecolor{currentstroke}%
\pgfsetstrokeopacity{0.000000}%
\pgfsetdash{}{0pt}%
\pgfpathmoveto{\pgfqpoint{3.001364in}{0.449444in}}%
\pgfpathlineto{\pgfqpoint{3.064773in}{0.449444in}}%
\pgfpathlineto{\pgfqpoint{3.064773in}{0.449444in}}%
\pgfpathlineto{\pgfqpoint{3.001364in}{0.449444in}}%
\pgfpathlineto{\pgfqpoint{3.001364in}{0.449444in}}%
\pgfpathclose%
\pgfusepath{fill}%
\end{pgfscope}%
\begin{pgfscope}%
\pgfpathrectangle{\pgfqpoint{0.465000in}{0.449444in}}{\pgfqpoint{3.487500in}{1.155000in}}%
\pgfusepath{clip}%
\pgfsetbuttcap%
\pgfsetmiterjoin%
\definecolor{currentfill}{rgb}{0.000000,0.000000,0.000000}%
\pgfsetfillcolor{currentfill}%
\pgfsetlinewidth{0.000000pt}%
\definecolor{currentstroke}{rgb}{0.000000,0.000000,0.000000}%
\pgfsetstrokecolor{currentstroke}%
\pgfsetstrokeopacity{0.000000}%
\pgfsetdash{}{0pt}%
\pgfpathmoveto{\pgfqpoint{3.159887in}{0.449444in}}%
\pgfpathlineto{\pgfqpoint{3.223296in}{0.449444in}}%
\pgfpathlineto{\pgfqpoint{3.223296in}{0.449444in}}%
\pgfpathlineto{\pgfqpoint{3.159887in}{0.449444in}}%
\pgfpathlineto{\pgfqpoint{3.159887in}{0.449444in}}%
\pgfpathclose%
\pgfusepath{fill}%
\end{pgfscope}%
\begin{pgfscope}%
\pgfpathrectangle{\pgfqpoint{0.465000in}{0.449444in}}{\pgfqpoint{3.487500in}{1.155000in}}%
\pgfusepath{clip}%
\pgfsetbuttcap%
\pgfsetmiterjoin%
\definecolor{currentfill}{rgb}{0.000000,0.000000,0.000000}%
\pgfsetfillcolor{currentfill}%
\pgfsetlinewidth{0.000000pt}%
\definecolor{currentstroke}{rgb}{0.000000,0.000000,0.000000}%
\pgfsetstrokecolor{currentstroke}%
\pgfsetstrokeopacity{0.000000}%
\pgfsetdash{}{0pt}%
\pgfpathmoveto{\pgfqpoint{3.318409in}{0.449444in}}%
\pgfpathlineto{\pgfqpoint{3.381818in}{0.449444in}}%
\pgfpathlineto{\pgfqpoint{3.381818in}{0.449444in}}%
\pgfpathlineto{\pgfqpoint{3.318409in}{0.449444in}}%
\pgfpathlineto{\pgfqpoint{3.318409in}{0.449444in}}%
\pgfpathclose%
\pgfusepath{fill}%
\end{pgfscope}%
\begin{pgfscope}%
\pgfpathrectangle{\pgfqpoint{0.465000in}{0.449444in}}{\pgfqpoint{3.487500in}{1.155000in}}%
\pgfusepath{clip}%
\pgfsetbuttcap%
\pgfsetmiterjoin%
\definecolor{currentfill}{rgb}{0.000000,0.000000,0.000000}%
\pgfsetfillcolor{currentfill}%
\pgfsetlinewidth{0.000000pt}%
\definecolor{currentstroke}{rgb}{0.000000,0.000000,0.000000}%
\pgfsetstrokecolor{currentstroke}%
\pgfsetstrokeopacity{0.000000}%
\pgfsetdash{}{0pt}%
\pgfpathmoveto{\pgfqpoint{3.476932in}{0.449444in}}%
\pgfpathlineto{\pgfqpoint{3.540341in}{0.449444in}}%
\pgfpathlineto{\pgfqpoint{3.540341in}{0.449444in}}%
\pgfpathlineto{\pgfqpoint{3.476932in}{0.449444in}}%
\pgfpathlineto{\pgfqpoint{3.476932in}{0.449444in}}%
\pgfpathclose%
\pgfusepath{fill}%
\end{pgfscope}%
\begin{pgfscope}%
\pgfpathrectangle{\pgfqpoint{0.465000in}{0.449444in}}{\pgfqpoint{3.487500in}{1.155000in}}%
\pgfusepath{clip}%
\pgfsetbuttcap%
\pgfsetmiterjoin%
\definecolor{currentfill}{rgb}{0.000000,0.000000,0.000000}%
\pgfsetfillcolor{currentfill}%
\pgfsetlinewidth{0.000000pt}%
\definecolor{currentstroke}{rgb}{0.000000,0.000000,0.000000}%
\pgfsetstrokecolor{currentstroke}%
\pgfsetstrokeopacity{0.000000}%
\pgfsetdash{}{0pt}%
\pgfpathmoveto{\pgfqpoint{3.635455in}{0.449444in}}%
\pgfpathlineto{\pgfqpoint{3.698864in}{0.449444in}}%
\pgfpathlineto{\pgfqpoint{3.698864in}{0.449444in}}%
\pgfpathlineto{\pgfqpoint{3.635455in}{0.449444in}}%
\pgfpathlineto{\pgfqpoint{3.635455in}{0.449444in}}%
\pgfpathclose%
\pgfusepath{fill}%
\end{pgfscope}%
\begin{pgfscope}%
\pgfpathrectangle{\pgfqpoint{0.465000in}{0.449444in}}{\pgfqpoint{3.487500in}{1.155000in}}%
\pgfusepath{clip}%
\pgfsetbuttcap%
\pgfsetmiterjoin%
\definecolor{currentfill}{rgb}{0.000000,0.000000,0.000000}%
\pgfsetfillcolor{currentfill}%
\pgfsetlinewidth{0.000000pt}%
\definecolor{currentstroke}{rgb}{0.000000,0.000000,0.000000}%
\pgfsetstrokecolor{currentstroke}%
\pgfsetstrokeopacity{0.000000}%
\pgfsetdash{}{0pt}%
\pgfpathmoveto{\pgfqpoint{3.793978in}{0.449444in}}%
\pgfpathlineto{\pgfqpoint{3.857387in}{0.449444in}}%
\pgfpathlineto{\pgfqpoint{3.857387in}{0.449444in}}%
\pgfpathlineto{\pgfqpoint{3.793978in}{0.449444in}}%
\pgfpathlineto{\pgfqpoint{3.793978in}{0.449444in}}%
\pgfpathclose%
\pgfusepath{fill}%
\end{pgfscope}%
\begin{pgfscope}%
\pgfsetbuttcap%
\pgfsetroundjoin%
\definecolor{currentfill}{rgb}{0.000000,0.000000,0.000000}%
\pgfsetfillcolor{currentfill}%
\pgfsetlinewidth{0.803000pt}%
\definecolor{currentstroke}{rgb}{0.000000,0.000000,0.000000}%
\pgfsetstrokecolor{currentstroke}%
\pgfsetdash{}{0pt}%
\pgfsys@defobject{currentmarker}{\pgfqpoint{0.000000in}{-0.048611in}}{\pgfqpoint{0.000000in}{0.000000in}}{%
\pgfpathmoveto{\pgfqpoint{0.000000in}{0.000000in}}%
\pgfpathlineto{\pgfqpoint{0.000000in}{-0.048611in}}%
\pgfusepath{stroke,fill}%
}%
\begin{pgfscope}%
\pgfsys@transformshift{0.465000in}{0.449444in}%
\pgfsys@useobject{currentmarker}{}%
\end{pgfscope}%
\end{pgfscope}%
\begin{pgfscope}%
\pgfsetbuttcap%
\pgfsetroundjoin%
\definecolor{currentfill}{rgb}{0.000000,0.000000,0.000000}%
\pgfsetfillcolor{currentfill}%
\pgfsetlinewidth{0.803000pt}%
\definecolor{currentstroke}{rgb}{0.000000,0.000000,0.000000}%
\pgfsetstrokecolor{currentstroke}%
\pgfsetdash{}{0pt}%
\pgfsys@defobject{currentmarker}{\pgfqpoint{0.000000in}{-0.048611in}}{\pgfqpoint{0.000000in}{0.000000in}}{%
\pgfpathmoveto{\pgfqpoint{0.000000in}{0.000000in}}%
\pgfpathlineto{\pgfqpoint{0.000000in}{-0.048611in}}%
\pgfusepath{stroke,fill}%
}%
\begin{pgfscope}%
\pgfsys@transformshift{0.623523in}{0.449444in}%
\pgfsys@useobject{currentmarker}{}%
\end{pgfscope}%
\end{pgfscope}%
\begin{pgfscope}%
\definecolor{textcolor}{rgb}{0.000000,0.000000,0.000000}%
\pgfsetstrokecolor{textcolor}%
\pgfsetfillcolor{textcolor}%
\pgftext[x=0.623523in,y=0.352222in,,top]{\color{textcolor}\rmfamily\fontsize{10.000000}{12.000000}\selectfont 0.0}%
\end{pgfscope}%
\begin{pgfscope}%
\pgfsetbuttcap%
\pgfsetroundjoin%
\definecolor{currentfill}{rgb}{0.000000,0.000000,0.000000}%
\pgfsetfillcolor{currentfill}%
\pgfsetlinewidth{0.803000pt}%
\definecolor{currentstroke}{rgb}{0.000000,0.000000,0.000000}%
\pgfsetstrokecolor{currentstroke}%
\pgfsetdash{}{0pt}%
\pgfsys@defobject{currentmarker}{\pgfqpoint{0.000000in}{-0.048611in}}{\pgfqpoint{0.000000in}{0.000000in}}{%
\pgfpathmoveto{\pgfqpoint{0.000000in}{0.000000in}}%
\pgfpathlineto{\pgfqpoint{0.000000in}{-0.048611in}}%
\pgfusepath{stroke,fill}%
}%
\begin{pgfscope}%
\pgfsys@transformshift{0.782046in}{0.449444in}%
\pgfsys@useobject{currentmarker}{}%
\end{pgfscope}%
\end{pgfscope}%
\begin{pgfscope}%
\pgfsetbuttcap%
\pgfsetroundjoin%
\definecolor{currentfill}{rgb}{0.000000,0.000000,0.000000}%
\pgfsetfillcolor{currentfill}%
\pgfsetlinewidth{0.803000pt}%
\definecolor{currentstroke}{rgb}{0.000000,0.000000,0.000000}%
\pgfsetstrokecolor{currentstroke}%
\pgfsetdash{}{0pt}%
\pgfsys@defobject{currentmarker}{\pgfqpoint{0.000000in}{-0.048611in}}{\pgfqpoint{0.000000in}{0.000000in}}{%
\pgfpathmoveto{\pgfqpoint{0.000000in}{0.000000in}}%
\pgfpathlineto{\pgfqpoint{0.000000in}{-0.048611in}}%
\pgfusepath{stroke,fill}%
}%
\begin{pgfscope}%
\pgfsys@transformshift{0.940568in}{0.449444in}%
\pgfsys@useobject{currentmarker}{}%
\end{pgfscope}%
\end{pgfscope}%
\begin{pgfscope}%
\definecolor{textcolor}{rgb}{0.000000,0.000000,0.000000}%
\pgfsetstrokecolor{textcolor}%
\pgfsetfillcolor{textcolor}%
\pgftext[x=0.940568in,y=0.352222in,,top]{\color{textcolor}\rmfamily\fontsize{10.000000}{12.000000}\selectfont 0.1}%
\end{pgfscope}%
\begin{pgfscope}%
\pgfsetbuttcap%
\pgfsetroundjoin%
\definecolor{currentfill}{rgb}{0.000000,0.000000,0.000000}%
\pgfsetfillcolor{currentfill}%
\pgfsetlinewidth{0.803000pt}%
\definecolor{currentstroke}{rgb}{0.000000,0.000000,0.000000}%
\pgfsetstrokecolor{currentstroke}%
\pgfsetdash{}{0pt}%
\pgfsys@defobject{currentmarker}{\pgfqpoint{0.000000in}{-0.048611in}}{\pgfqpoint{0.000000in}{0.000000in}}{%
\pgfpathmoveto{\pgfqpoint{0.000000in}{0.000000in}}%
\pgfpathlineto{\pgfqpoint{0.000000in}{-0.048611in}}%
\pgfusepath{stroke,fill}%
}%
\begin{pgfscope}%
\pgfsys@transformshift{1.099091in}{0.449444in}%
\pgfsys@useobject{currentmarker}{}%
\end{pgfscope}%
\end{pgfscope}%
\begin{pgfscope}%
\pgfsetbuttcap%
\pgfsetroundjoin%
\definecolor{currentfill}{rgb}{0.000000,0.000000,0.000000}%
\pgfsetfillcolor{currentfill}%
\pgfsetlinewidth{0.803000pt}%
\definecolor{currentstroke}{rgb}{0.000000,0.000000,0.000000}%
\pgfsetstrokecolor{currentstroke}%
\pgfsetdash{}{0pt}%
\pgfsys@defobject{currentmarker}{\pgfqpoint{0.000000in}{-0.048611in}}{\pgfqpoint{0.000000in}{0.000000in}}{%
\pgfpathmoveto{\pgfqpoint{0.000000in}{0.000000in}}%
\pgfpathlineto{\pgfqpoint{0.000000in}{-0.048611in}}%
\pgfusepath{stroke,fill}%
}%
\begin{pgfscope}%
\pgfsys@transformshift{1.257614in}{0.449444in}%
\pgfsys@useobject{currentmarker}{}%
\end{pgfscope}%
\end{pgfscope}%
\begin{pgfscope}%
\definecolor{textcolor}{rgb}{0.000000,0.000000,0.000000}%
\pgfsetstrokecolor{textcolor}%
\pgfsetfillcolor{textcolor}%
\pgftext[x=1.257614in,y=0.352222in,,top]{\color{textcolor}\rmfamily\fontsize{10.000000}{12.000000}\selectfont 0.2}%
\end{pgfscope}%
\begin{pgfscope}%
\pgfsetbuttcap%
\pgfsetroundjoin%
\definecolor{currentfill}{rgb}{0.000000,0.000000,0.000000}%
\pgfsetfillcolor{currentfill}%
\pgfsetlinewidth{0.803000pt}%
\definecolor{currentstroke}{rgb}{0.000000,0.000000,0.000000}%
\pgfsetstrokecolor{currentstroke}%
\pgfsetdash{}{0pt}%
\pgfsys@defobject{currentmarker}{\pgfqpoint{0.000000in}{-0.048611in}}{\pgfqpoint{0.000000in}{0.000000in}}{%
\pgfpathmoveto{\pgfqpoint{0.000000in}{0.000000in}}%
\pgfpathlineto{\pgfqpoint{0.000000in}{-0.048611in}}%
\pgfusepath{stroke,fill}%
}%
\begin{pgfscope}%
\pgfsys@transformshift{1.416137in}{0.449444in}%
\pgfsys@useobject{currentmarker}{}%
\end{pgfscope}%
\end{pgfscope}%
\begin{pgfscope}%
\pgfsetbuttcap%
\pgfsetroundjoin%
\definecolor{currentfill}{rgb}{0.000000,0.000000,0.000000}%
\pgfsetfillcolor{currentfill}%
\pgfsetlinewidth{0.803000pt}%
\definecolor{currentstroke}{rgb}{0.000000,0.000000,0.000000}%
\pgfsetstrokecolor{currentstroke}%
\pgfsetdash{}{0pt}%
\pgfsys@defobject{currentmarker}{\pgfqpoint{0.000000in}{-0.048611in}}{\pgfqpoint{0.000000in}{0.000000in}}{%
\pgfpathmoveto{\pgfqpoint{0.000000in}{0.000000in}}%
\pgfpathlineto{\pgfqpoint{0.000000in}{-0.048611in}}%
\pgfusepath{stroke,fill}%
}%
\begin{pgfscope}%
\pgfsys@transformshift{1.574659in}{0.449444in}%
\pgfsys@useobject{currentmarker}{}%
\end{pgfscope}%
\end{pgfscope}%
\begin{pgfscope}%
\definecolor{textcolor}{rgb}{0.000000,0.000000,0.000000}%
\pgfsetstrokecolor{textcolor}%
\pgfsetfillcolor{textcolor}%
\pgftext[x=1.574659in,y=0.352222in,,top]{\color{textcolor}\rmfamily\fontsize{10.000000}{12.000000}\selectfont 0.3}%
\end{pgfscope}%
\begin{pgfscope}%
\pgfsetbuttcap%
\pgfsetroundjoin%
\definecolor{currentfill}{rgb}{0.000000,0.000000,0.000000}%
\pgfsetfillcolor{currentfill}%
\pgfsetlinewidth{0.803000pt}%
\definecolor{currentstroke}{rgb}{0.000000,0.000000,0.000000}%
\pgfsetstrokecolor{currentstroke}%
\pgfsetdash{}{0pt}%
\pgfsys@defobject{currentmarker}{\pgfqpoint{0.000000in}{-0.048611in}}{\pgfqpoint{0.000000in}{0.000000in}}{%
\pgfpathmoveto{\pgfqpoint{0.000000in}{0.000000in}}%
\pgfpathlineto{\pgfqpoint{0.000000in}{-0.048611in}}%
\pgfusepath{stroke,fill}%
}%
\begin{pgfscope}%
\pgfsys@transformshift{1.733182in}{0.449444in}%
\pgfsys@useobject{currentmarker}{}%
\end{pgfscope}%
\end{pgfscope}%
\begin{pgfscope}%
\pgfsetbuttcap%
\pgfsetroundjoin%
\definecolor{currentfill}{rgb}{0.000000,0.000000,0.000000}%
\pgfsetfillcolor{currentfill}%
\pgfsetlinewidth{0.803000pt}%
\definecolor{currentstroke}{rgb}{0.000000,0.000000,0.000000}%
\pgfsetstrokecolor{currentstroke}%
\pgfsetdash{}{0pt}%
\pgfsys@defobject{currentmarker}{\pgfqpoint{0.000000in}{-0.048611in}}{\pgfqpoint{0.000000in}{0.000000in}}{%
\pgfpathmoveto{\pgfqpoint{0.000000in}{0.000000in}}%
\pgfpathlineto{\pgfqpoint{0.000000in}{-0.048611in}}%
\pgfusepath{stroke,fill}%
}%
\begin{pgfscope}%
\pgfsys@transformshift{1.891705in}{0.449444in}%
\pgfsys@useobject{currentmarker}{}%
\end{pgfscope}%
\end{pgfscope}%
\begin{pgfscope}%
\definecolor{textcolor}{rgb}{0.000000,0.000000,0.000000}%
\pgfsetstrokecolor{textcolor}%
\pgfsetfillcolor{textcolor}%
\pgftext[x=1.891705in,y=0.352222in,,top]{\color{textcolor}\rmfamily\fontsize{10.000000}{12.000000}\selectfont 0.4}%
\end{pgfscope}%
\begin{pgfscope}%
\pgfsetbuttcap%
\pgfsetroundjoin%
\definecolor{currentfill}{rgb}{0.000000,0.000000,0.000000}%
\pgfsetfillcolor{currentfill}%
\pgfsetlinewidth{0.803000pt}%
\definecolor{currentstroke}{rgb}{0.000000,0.000000,0.000000}%
\pgfsetstrokecolor{currentstroke}%
\pgfsetdash{}{0pt}%
\pgfsys@defobject{currentmarker}{\pgfqpoint{0.000000in}{-0.048611in}}{\pgfqpoint{0.000000in}{0.000000in}}{%
\pgfpathmoveto{\pgfqpoint{0.000000in}{0.000000in}}%
\pgfpathlineto{\pgfqpoint{0.000000in}{-0.048611in}}%
\pgfusepath{stroke,fill}%
}%
\begin{pgfscope}%
\pgfsys@transformshift{2.050228in}{0.449444in}%
\pgfsys@useobject{currentmarker}{}%
\end{pgfscope}%
\end{pgfscope}%
\begin{pgfscope}%
\pgfsetbuttcap%
\pgfsetroundjoin%
\definecolor{currentfill}{rgb}{0.000000,0.000000,0.000000}%
\pgfsetfillcolor{currentfill}%
\pgfsetlinewidth{0.803000pt}%
\definecolor{currentstroke}{rgb}{0.000000,0.000000,0.000000}%
\pgfsetstrokecolor{currentstroke}%
\pgfsetdash{}{0pt}%
\pgfsys@defobject{currentmarker}{\pgfqpoint{0.000000in}{-0.048611in}}{\pgfqpoint{0.000000in}{0.000000in}}{%
\pgfpathmoveto{\pgfqpoint{0.000000in}{0.000000in}}%
\pgfpathlineto{\pgfqpoint{0.000000in}{-0.048611in}}%
\pgfusepath{stroke,fill}%
}%
\begin{pgfscope}%
\pgfsys@transformshift{2.208750in}{0.449444in}%
\pgfsys@useobject{currentmarker}{}%
\end{pgfscope}%
\end{pgfscope}%
\begin{pgfscope}%
\definecolor{textcolor}{rgb}{0.000000,0.000000,0.000000}%
\pgfsetstrokecolor{textcolor}%
\pgfsetfillcolor{textcolor}%
\pgftext[x=2.208750in,y=0.352222in,,top]{\color{textcolor}\rmfamily\fontsize{10.000000}{12.000000}\selectfont 0.5}%
\end{pgfscope}%
\begin{pgfscope}%
\pgfsetbuttcap%
\pgfsetroundjoin%
\definecolor{currentfill}{rgb}{0.000000,0.000000,0.000000}%
\pgfsetfillcolor{currentfill}%
\pgfsetlinewidth{0.803000pt}%
\definecolor{currentstroke}{rgb}{0.000000,0.000000,0.000000}%
\pgfsetstrokecolor{currentstroke}%
\pgfsetdash{}{0pt}%
\pgfsys@defobject{currentmarker}{\pgfqpoint{0.000000in}{-0.048611in}}{\pgfqpoint{0.000000in}{0.000000in}}{%
\pgfpathmoveto{\pgfqpoint{0.000000in}{0.000000in}}%
\pgfpathlineto{\pgfqpoint{0.000000in}{-0.048611in}}%
\pgfusepath{stroke,fill}%
}%
\begin{pgfscope}%
\pgfsys@transformshift{2.367273in}{0.449444in}%
\pgfsys@useobject{currentmarker}{}%
\end{pgfscope}%
\end{pgfscope}%
\begin{pgfscope}%
\pgfsetbuttcap%
\pgfsetroundjoin%
\definecolor{currentfill}{rgb}{0.000000,0.000000,0.000000}%
\pgfsetfillcolor{currentfill}%
\pgfsetlinewidth{0.803000pt}%
\definecolor{currentstroke}{rgb}{0.000000,0.000000,0.000000}%
\pgfsetstrokecolor{currentstroke}%
\pgfsetdash{}{0pt}%
\pgfsys@defobject{currentmarker}{\pgfqpoint{0.000000in}{-0.048611in}}{\pgfqpoint{0.000000in}{0.000000in}}{%
\pgfpathmoveto{\pgfqpoint{0.000000in}{0.000000in}}%
\pgfpathlineto{\pgfqpoint{0.000000in}{-0.048611in}}%
\pgfusepath{stroke,fill}%
}%
\begin{pgfscope}%
\pgfsys@transformshift{2.525796in}{0.449444in}%
\pgfsys@useobject{currentmarker}{}%
\end{pgfscope}%
\end{pgfscope}%
\begin{pgfscope}%
\definecolor{textcolor}{rgb}{0.000000,0.000000,0.000000}%
\pgfsetstrokecolor{textcolor}%
\pgfsetfillcolor{textcolor}%
\pgftext[x=2.525796in,y=0.352222in,,top]{\color{textcolor}\rmfamily\fontsize{10.000000}{12.000000}\selectfont 0.6}%
\end{pgfscope}%
\begin{pgfscope}%
\pgfsetbuttcap%
\pgfsetroundjoin%
\definecolor{currentfill}{rgb}{0.000000,0.000000,0.000000}%
\pgfsetfillcolor{currentfill}%
\pgfsetlinewidth{0.803000pt}%
\definecolor{currentstroke}{rgb}{0.000000,0.000000,0.000000}%
\pgfsetstrokecolor{currentstroke}%
\pgfsetdash{}{0pt}%
\pgfsys@defobject{currentmarker}{\pgfqpoint{0.000000in}{-0.048611in}}{\pgfqpoint{0.000000in}{0.000000in}}{%
\pgfpathmoveto{\pgfqpoint{0.000000in}{0.000000in}}%
\pgfpathlineto{\pgfqpoint{0.000000in}{-0.048611in}}%
\pgfusepath{stroke,fill}%
}%
\begin{pgfscope}%
\pgfsys@transformshift{2.684318in}{0.449444in}%
\pgfsys@useobject{currentmarker}{}%
\end{pgfscope}%
\end{pgfscope}%
\begin{pgfscope}%
\pgfsetbuttcap%
\pgfsetroundjoin%
\definecolor{currentfill}{rgb}{0.000000,0.000000,0.000000}%
\pgfsetfillcolor{currentfill}%
\pgfsetlinewidth{0.803000pt}%
\definecolor{currentstroke}{rgb}{0.000000,0.000000,0.000000}%
\pgfsetstrokecolor{currentstroke}%
\pgfsetdash{}{0pt}%
\pgfsys@defobject{currentmarker}{\pgfqpoint{0.000000in}{-0.048611in}}{\pgfqpoint{0.000000in}{0.000000in}}{%
\pgfpathmoveto{\pgfqpoint{0.000000in}{0.000000in}}%
\pgfpathlineto{\pgfqpoint{0.000000in}{-0.048611in}}%
\pgfusepath{stroke,fill}%
}%
\begin{pgfscope}%
\pgfsys@transformshift{2.842841in}{0.449444in}%
\pgfsys@useobject{currentmarker}{}%
\end{pgfscope}%
\end{pgfscope}%
\begin{pgfscope}%
\definecolor{textcolor}{rgb}{0.000000,0.000000,0.000000}%
\pgfsetstrokecolor{textcolor}%
\pgfsetfillcolor{textcolor}%
\pgftext[x=2.842841in,y=0.352222in,,top]{\color{textcolor}\rmfamily\fontsize{10.000000}{12.000000}\selectfont 0.7}%
\end{pgfscope}%
\begin{pgfscope}%
\pgfsetbuttcap%
\pgfsetroundjoin%
\definecolor{currentfill}{rgb}{0.000000,0.000000,0.000000}%
\pgfsetfillcolor{currentfill}%
\pgfsetlinewidth{0.803000pt}%
\definecolor{currentstroke}{rgb}{0.000000,0.000000,0.000000}%
\pgfsetstrokecolor{currentstroke}%
\pgfsetdash{}{0pt}%
\pgfsys@defobject{currentmarker}{\pgfqpoint{0.000000in}{-0.048611in}}{\pgfqpoint{0.000000in}{0.000000in}}{%
\pgfpathmoveto{\pgfqpoint{0.000000in}{0.000000in}}%
\pgfpathlineto{\pgfqpoint{0.000000in}{-0.048611in}}%
\pgfusepath{stroke,fill}%
}%
\begin{pgfscope}%
\pgfsys@transformshift{3.001364in}{0.449444in}%
\pgfsys@useobject{currentmarker}{}%
\end{pgfscope}%
\end{pgfscope}%
\begin{pgfscope}%
\pgfsetbuttcap%
\pgfsetroundjoin%
\definecolor{currentfill}{rgb}{0.000000,0.000000,0.000000}%
\pgfsetfillcolor{currentfill}%
\pgfsetlinewidth{0.803000pt}%
\definecolor{currentstroke}{rgb}{0.000000,0.000000,0.000000}%
\pgfsetstrokecolor{currentstroke}%
\pgfsetdash{}{0pt}%
\pgfsys@defobject{currentmarker}{\pgfqpoint{0.000000in}{-0.048611in}}{\pgfqpoint{0.000000in}{0.000000in}}{%
\pgfpathmoveto{\pgfqpoint{0.000000in}{0.000000in}}%
\pgfpathlineto{\pgfqpoint{0.000000in}{-0.048611in}}%
\pgfusepath{stroke,fill}%
}%
\begin{pgfscope}%
\pgfsys@transformshift{3.159887in}{0.449444in}%
\pgfsys@useobject{currentmarker}{}%
\end{pgfscope}%
\end{pgfscope}%
\begin{pgfscope}%
\definecolor{textcolor}{rgb}{0.000000,0.000000,0.000000}%
\pgfsetstrokecolor{textcolor}%
\pgfsetfillcolor{textcolor}%
\pgftext[x=3.159887in,y=0.352222in,,top]{\color{textcolor}\rmfamily\fontsize{10.000000}{12.000000}\selectfont 0.8}%
\end{pgfscope}%
\begin{pgfscope}%
\pgfsetbuttcap%
\pgfsetroundjoin%
\definecolor{currentfill}{rgb}{0.000000,0.000000,0.000000}%
\pgfsetfillcolor{currentfill}%
\pgfsetlinewidth{0.803000pt}%
\definecolor{currentstroke}{rgb}{0.000000,0.000000,0.000000}%
\pgfsetstrokecolor{currentstroke}%
\pgfsetdash{}{0pt}%
\pgfsys@defobject{currentmarker}{\pgfqpoint{0.000000in}{-0.048611in}}{\pgfqpoint{0.000000in}{0.000000in}}{%
\pgfpathmoveto{\pgfqpoint{0.000000in}{0.000000in}}%
\pgfpathlineto{\pgfqpoint{0.000000in}{-0.048611in}}%
\pgfusepath{stroke,fill}%
}%
\begin{pgfscope}%
\pgfsys@transformshift{3.318409in}{0.449444in}%
\pgfsys@useobject{currentmarker}{}%
\end{pgfscope}%
\end{pgfscope}%
\begin{pgfscope}%
\pgfsetbuttcap%
\pgfsetroundjoin%
\definecolor{currentfill}{rgb}{0.000000,0.000000,0.000000}%
\pgfsetfillcolor{currentfill}%
\pgfsetlinewidth{0.803000pt}%
\definecolor{currentstroke}{rgb}{0.000000,0.000000,0.000000}%
\pgfsetstrokecolor{currentstroke}%
\pgfsetdash{}{0pt}%
\pgfsys@defobject{currentmarker}{\pgfqpoint{0.000000in}{-0.048611in}}{\pgfqpoint{0.000000in}{0.000000in}}{%
\pgfpathmoveto{\pgfqpoint{0.000000in}{0.000000in}}%
\pgfpathlineto{\pgfqpoint{0.000000in}{-0.048611in}}%
\pgfusepath{stroke,fill}%
}%
\begin{pgfscope}%
\pgfsys@transformshift{3.476932in}{0.449444in}%
\pgfsys@useobject{currentmarker}{}%
\end{pgfscope}%
\end{pgfscope}%
\begin{pgfscope}%
\definecolor{textcolor}{rgb}{0.000000,0.000000,0.000000}%
\pgfsetstrokecolor{textcolor}%
\pgfsetfillcolor{textcolor}%
\pgftext[x=3.476932in,y=0.352222in,,top]{\color{textcolor}\rmfamily\fontsize{10.000000}{12.000000}\selectfont 0.9}%
\end{pgfscope}%
\begin{pgfscope}%
\pgfsetbuttcap%
\pgfsetroundjoin%
\definecolor{currentfill}{rgb}{0.000000,0.000000,0.000000}%
\pgfsetfillcolor{currentfill}%
\pgfsetlinewidth{0.803000pt}%
\definecolor{currentstroke}{rgb}{0.000000,0.000000,0.000000}%
\pgfsetstrokecolor{currentstroke}%
\pgfsetdash{}{0pt}%
\pgfsys@defobject{currentmarker}{\pgfqpoint{0.000000in}{-0.048611in}}{\pgfqpoint{0.000000in}{0.000000in}}{%
\pgfpathmoveto{\pgfqpoint{0.000000in}{0.000000in}}%
\pgfpathlineto{\pgfqpoint{0.000000in}{-0.048611in}}%
\pgfusepath{stroke,fill}%
}%
\begin{pgfscope}%
\pgfsys@transformshift{3.635455in}{0.449444in}%
\pgfsys@useobject{currentmarker}{}%
\end{pgfscope}%
\end{pgfscope}%
\begin{pgfscope}%
\pgfsetbuttcap%
\pgfsetroundjoin%
\definecolor{currentfill}{rgb}{0.000000,0.000000,0.000000}%
\pgfsetfillcolor{currentfill}%
\pgfsetlinewidth{0.803000pt}%
\definecolor{currentstroke}{rgb}{0.000000,0.000000,0.000000}%
\pgfsetstrokecolor{currentstroke}%
\pgfsetdash{}{0pt}%
\pgfsys@defobject{currentmarker}{\pgfqpoint{0.000000in}{-0.048611in}}{\pgfqpoint{0.000000in}{0.000000in}}{%
\pgfpathmoveto{\pgfqpoint{0.000000in}{0.000000in}}%
\pgfpathlineto{\pgfqpoint{0.000000in}{-0.048611in}}%
\pgfusepath{stroke,fill}%
}%
\begin{pgfscope}%
\pgfsys@transformshift{3.793978in}{0.449444in}%
\pgfsys@useobject{currentmarker}{}%
\end{pgfscope}%
\end{pgfscope}%
\begin{pgfscope}%
\definecolor{textcolor}{rgb}{0.000000,0.000000,0.000000}%
\pgfsetstrokecolor{textcolor}%
\pgfsetfillcolor{textcolor}%
\pgftext[x=3.793978in,y=0.352222in,,top]{\color{textcolor}\rmfamily\fontsize{10.000000}{12.000000}\selectfont 1.0}%
\end{pgfscope}%
\begin{pgfscope}%
\pgfsetbuttcap%
\pgfsetroundjoin%
\definecolor{currentfill}{rgb}{0.000000,0.000000,0.000000}%
\pgfsetfillcolor{currentfill}%
\pgfsetlinewidth{0.803000pt}%
\definecolor{currentstroke}{rgb}{0.000000,0.000000,0.000000}%
\pgfsetstrokecolor{currentstroke}%
\pgfsetdash{}{0pt}%
\pgfsys@defobject{currentmarker}{\pgfqpoint{0.000000in}{-0.048611in}}{\pgfqpoint{0.000000in}{0.000000in}}{%
\pgfpathmoveto{\pgfqpoint{0.000000in}{0.000000in}}%
\pgfpathlineto{\pgfqpoint{0.000000in}{-0.048611in}}%
\pgfusepath{stroke,fill}%
}%
\begin{pgfscope}%
\pgfsys@transformshift{3.952500in}{0.449444in}%
\pgfsys@useobject{currentmarker}{}%
\end{pgfscope}%
\end{pgfscope}%
\begin{pgfscope}%
\definecolor{textcolor}{rgb}{0.000000,0.000000,0.000000}%
\pgfsetstrokecolor{textcolor}%
\pgfsetfillcolor{textcolor}%
\pgftext[x=2.208750in,y=0.173333in,,top]{\color{textcolor}\rmfamily\fontsize{10.000000}{12.000000}\selectfont \(\displaystyle p\)}%
\end{pgfscope}%
\begin{pgfscope}%
\pgfsetbuttcap%
\pgfsetroundjoin%
\definecolor{currentfill}{rgb}{0.000000,0.000000,0.000000}%
\pgfsetfillcolor{currentfill}%
\pgfsetlinewidth{0.803000pt}%
\definecolor{currentstroke}{rgb}{0.000000,0.000000,0.000000}%
\pgfsetstrokecolor{currentstroke}%
\pgfsetdash{}{0pt}%
\pgfsys@defobject{currentmarker}{\pgfqpoint{-0.048611in}{0.000000in}}{\pgfqpoint{-0.000000in}{0.000000in}}{%
\pgfpathmoveto{\pgfqpoint{-0.000000in}{0.000000in}}%
\pgfpathlineto{\pgfqpoint{-0.048611in}{0.000000in}}%
\pgfusepath{stroke,fill}%
}%
\begin{pgfscope}%
\pgfsys@transformshift{0.465000in}{0.449444in}%
\pgfsys@useobject{currentmarker}{}%
\end{pgfscope}%
\end{pgfscope}%
\begin{pgfscope}%
\definecolor{textcolor}{rgb}{0.000000,0.000000,0.000000}%
\pgfsetstrokecolor{textcolor}%
\pgfsetfillcolor{textcolor}%
\pgftext[x=0.298333in, y=0.401250in, left, base]{\color{textcolor}\rmfamily\fontsize{10.000000}{12.000000}\selectfont \(\displaystyle {0}\)}%
\end{pgfscope}%
\begin{pgfscope}%
\pgfsetbuttcap%
\pgfsetroundjoin%
\definecolor{currentfill}{rgb}{0.000000,0.000000,0.000000}%
\pgfsetfillcolor{currentfill}%
\pgfsetlinewidth{0.803000pt}%
\definecolor{currentstroke}{rgb}{0.000000,0.000000,0.000000}%
\pgfsetstrokecolor{currentstroke}%
\pgfsetdash{}{0pt}%
\pgfsys@defobject{currentmarker}{\pgfqpoint{-0.048611in}{0.000000in}}{\pgfqpoint{-0.000000in}{0.000000in}}{%
\pgfpathmoveto{\pgfqpoint{-0.000000in}{0.000000in}}%
\pgfpathlineto{\pgfqpoint{-0.048611in}{0.000000in}}%
\pgfusepath{stroke,fill}%
}%
\begin{pgfscope}%
\pgfsys@transformshift{0.465000in}{0.778615in}%
\pgfsys@useobject{currentmarker}{}%
\end{pgfscope}%
\end{pgfscope}%
\begin{pgfscope}%
\definecolor{textcolor}{rgb}{0.000000,0.000000,0.000000}%
\pgfsetstrokecolor{textcolor}%
\pgfsetfillcolor{textcolor}%
\pgftext[x=0.228889in, y=0.730420in, left, base]{\color{textcolor}\rmfamily\fontsize{10.000000}{12.000000}\selectfont \(\displaystyle {25}\)}%
\end{pgfscope}%
\begin{pgfscope}%
\pgfsetbuttcap%
\pgfsetroundjoin%
\definecolor{currentfill}{rgb}{0.000000,0.000000,0.000000}%
\pgfsetfillcolor{currentfill}%
\pgfsetlinewidth{0.803000pt}%
\definecolor{currentstroke}{rgb}{0.000000,0.000000,0.000000}%
\pgfsetstrokecolor{currentstroke}%
\pgfsetdash{}{0pt}%
\pgfsys@defobject{currentmarker}{\pgfqpoint{-0.048611in}{0.000000in}}{\pgfqpoint{-0.000000in}{0.000000in}}{%
\pgfpathmoveto{\pgfqpoint{-0.000000in}{0.000000in}}%
\pgfpathlineto{\pgfqpoint{-0.048611in}{0.000000in}}%
\pgfusepath{stroke,fill}%
}%
\begin{pgfscope}%
\pgfsys@transformshift{0.465000in}{1.107785in}%
\pgfsys@useobject{currentmarker}{}%
\end{pgfscope}%
\end{pgfscope}%
\begin{pgfscope}%
\definecolor{textcolor}{rgb}{0.000000,0.000000,0.000000}%
\pgfsetstrokecolor{textcolor}%
\pgfsetfillcolor{textcolor}%
\pgftext[x=0.228889in, y=1.059591in, left, base]{\color{textcolor}\rmfamily\fontsize{10.000000}{12.000000}\selectfont \(\displaystyle {50}\)}%
\end{pgfscope}%
\begin{pgfscope}%
\pgfsetbuttcap%
\pgfsetroundjoin%
\definecolor{currentfill}{rgb}{0.000000,0.000000,0.000000}%
\pgfsetfillcolor{currentfill}%
\pgfsetlinewidth{0.803000pt}%
\definecolor{currentstroke}{rgb}{0.000000,0.000000,0.000000}%
\pgfsetstrokecolor{currentstroke}%
\pgfsetdash{}{0pt}%
\pgfsys@defobject{currentmarker}{\pgfqpoint{-0.048611in}{0.000000in}}{\pgfqpoint{-0.000000in}{0.000000in}}{%
\pgfpathmoveto{\pgfqpoint{-0.000000in}{0.000000in}}%
\pgfpathlineto{\pgfqpoint{-0.048611in}{0.000000in}}%
\pgfusepath{stroke,fill}%
}%
\begin{pgfscope}%
\pgfsys@transformshift{0.465000in}{1.436955in}%
\pgfsys@useobject{currentmarker}{}%
\end{pgfscope}%
\end{pgfscope}%
\begin{pgfscope}%
\definecolor{textcolor}{rgb}{0.000000,0.000000,0.000000}%
\pgfsetstrokecolor{textcolor}%
\pgfsetfillcolor{textcolor}%
\pgftext[x=0.228889in, y=1.388761in, left, base]{\color{textcolor}\rmfamily\fontsize{10.000000}{12.000000}\selectfont \(\displaystyle {75}\)}%
\end{pgfscope}%
\begin{pgfscope}%
\definecolor{textcolor}{rgb}{0.000000,0.000000,0.000000}%
\pgfsetstrokecolor{textcolor}%
\pgfsetfillcolor{textcolor}%
\pgftext[x=0.173333in,y=1.026944in,,bottom,rotate=90.000000]{\color{textcolor}\rmfamily\fontsize{10.000000}{12.000000}\selectfont Percent of Data Set}%
\end{pgfscope}%
\begin{pgfscope}%
\pgfsetrectcap%
\pgfsetmiterjoin%
\pgfsetlinewidth{0.803000pt}%
\definecolor{currentstroke}{rgb}{0.000000,0.000000,0.000000}%
\pgfsetstrokecolor{currentstroke}%
\pgfsetdash{}{0pt}%
\pgfpathmoveto{\pgfqpoint{0.465000in}{0.449444in}}%
\pgfpathlineto{\pgfqpoint{0.465000in}{1.604444in}}%
\pgfusepath{stroke}%
\end{pgfscope}%
\begin{pgfscope}%
\pgfsetrectcap%
\pgfsetmiterjoin%
\pgfsetlinewidth{0.803000pt}%
\definecolor{currentstroke}{rgb}{0.000000,0.000000,0.000000}%
\pgfsetstrokecolor{currentstroke}%
\pgfsetdash{}{0pt}%
\pgfpathmoveto{\pgfqpoint{3.952500in}{0.449444in}}%
\pgfpathlineto{\pgfqpoint{3.952500in}{1.604444in}}%
\pgfusepath{stroke}%
\end{pgfscope}%
\begin{pgfscope}%
\pgfsetrectcap%
\pgfsetmiterjoin%
\pgfsetlinewidth{0.803000pt}%
\definecolor{currentstroke}{rgb}{0.000000,0.000000,0.000000}%
\pgfsetstrokecolor{currentstroke}%
\pgfsetdash{}{0pt}%
\pgfpathmoveto{\pgfqpoint{0.465000in}{0.449444in}}%
\pgfpathlineto{\pgfqpoint{3.952500in}{0.449444in}}%
\pgfusepath{stroke}%
\end{pgfscope}%
\begin{pgfscope}%
\pgfsetrectcap%
\pgfsetmiterjoin%
\pgfsetlinewidth{0.803000pt}%
\definecolor{currentstroke}{rgb}{0.000000,0.000000,0.000000}%
\pgfsetstrokecolor{currentstroke}%
\pgfsetdash{}{0pt}%
\pgfpathmoveto{\pgfqpoint{0.465000in}{1.604444in}}%
\pgfpathlineto{\pgfqpoint{3.952500in}{1.604444in}}%
\pgfusepath{stroke}%
\end{pgfscope}%
\begin{pgfscope}%
\pgfsetbuttcap%
\pgfsetmiterjoin%
\definecolor{currentfill}{rgb}{1.000000,1.000000,1.000000}%
\pgfsetfillcolor{currentfill}%
\pgfsetfillopacity{0.800000}%
\pgfsetlinewidth{1.003750pt}%
\definecolor{currentstroke}{rgb}{0.800000,0.800000,0.800000}%
\pgfsetstrokecolor{currentstroke}%
\pgfsetstrokeopacity{0.800000}%
\pgfsetdash{}{0pt}%
\pgfpathmoveto{\pgfqpoint{3.175556in}{1.104445in}}%
\pgfpathlineto{\pgfqpoint{3.855278in}{1.104445in}}%
\pgfpathquadraticcurveto{\pgfqpoint{3.883056in}{1.104445in}}{\pgfqpoint{3.883056in}{1.132222in}}%
\pgfpathlineto{\pgfqpoint{3.883056in}{1.507222in}}%
\pgfpathquadraticcurveto{\pgfqpoint{3.883056in}{1.535000in}}{\pgfqpoint{3.855278in}{1.535000in}}%
\pgfpathlineto{\pgfqpoint{3.175556in}{1.535000in}}%
\pgfpathquadraticcurveto{\pgfqpoint{3.147778in}{1.535000in}}{\pgfqpoint{3.147778in}{1.507222in}}%
\pgfpathlineto{\pgfqpoint{3.147778in}{1.132222in}}%
\pgfpathquadraticcurveto{\pgfqpoint{3.147778in}{1.104445in}}{\pgfqpoint{3.175556in}{1.104445in}}%
\pgfpathlineto{\pgfqpoint{3.175556in}{1.104445in}}%
\pgfpathclose%
\pgfusepath{stroke,fill}%
\end{pgfscope}%
\begin{pgfscope}%
\pgfsetbuttcap%
\pgfsetmiterjoin%
\pgfsetlinewidth{1.003750pt}%
\definecolor{currentstroke}{rgb}{0.000000,0.000000,0.000000}%
\pgfsetstrokecolor{currentstroke}%
\pgfsetdash{}{0pt}%
\pgfpathmoveto{\pgfqpoint{3.203334in}{1.382222in}}%
\pgfpathlineto{\pgfqpoint{3.481111in}{1.382222in}}%
\pgfpathlineto{\pgfqpoint{3.481111in}{1.479444in}}%
\pgfpathlineto{\pgfqpoint{3.203334in}{1.479444in}}%
\pgfpathlineto{\pgfqpoint{3.203334in}{1.382222in}}%
\pgfpathclose%
\pgfusepath{stroke}%
\end{pgfscope}%
\begin{pgfscope}%
\definecolor{textcolor}{rgb}{0.000000,0.000000,0.000000}%
\pgfsetstrokecolor{textcolor}%
\pgfsetfillcolor{textcolor}%
\pgftext[x=3.592223in,y=1.382222in,left,base]{\color{textcolor}\rmfamily\fontsize{10.000000}{12.000000}\selectfont Neg}%
\end{pgfscope}%
\begin{pgfscope}%
\pgfsetbuttcap%
\pgfsetmiterjoin%
\definecolor{currentfill}{rgb}{0.000000,0.000000,0.000000}%
\pgfsetfillcolor{currentfill}%
\pgfsetlinewidth{0.000000pt}%
\definecolor{currentstroke}{rgb}{0.000000,0.000000,0.000000}%
\pgfsetstrokecolor{currentstroke}%
\pgfsetstrokeopacity{0.000000}%
\pgfsetdash{}{0pt}%
\pgfpathmoveto{\pgfqpoint{3.203334in}{1.186944in}}%
\pgfpathlineto{\pgfqpoint{3.481111in}{1.186944in}}%
\pgfpathlineto{\pgfqpoint{3.481111in}{1.284167in}}%
\pgfpathlineto{\pgfqpoint{3.203334in}{1.284167in}}%
\pgfpathlineto{\pgfqpoint{3.203334in}{1.186944in}}%
\pgfpathclose%
\pgfusepath{fill}%
\end{pgfscope}%
\begin{pgfscope}%
\definecolor{textcolor}{rgb}{0.000000,0.000000,0.000000}%
\pgfsetstrokecolor{textcolor}%
\pgfsetfillcolor{textcolor}%
\pgftext[x=3.592223in,y=1.186944in,left,base]{\color{textcolor}\rmfamily\fontsize{10.000000}{12.000000}\selectfont Pos}%
\end{pgfscope}%
\end{pgfpicture}%
\makeatother%
\endgroup%
	
&
	\vskip 0pt
	\hfil ROC Curve
	
	%% Creator: Matplotlib, PGF backend
%%
%% To include the figure in your LaTeX document, write
%%   \input{<filename>.pgf}
%%
%% Make sure the required packages are loaded in your preamble
%%   \usepackage{pgf}
%%
%% Also ensure that all the required font packages are loaded; for instance,
%% the lmodern package is sometimes necessary when using math font.
%%   \usepackage{lmodern}
%%
%% Figures using additional raster images can only be included by \input if
%% they are in the same directory as the main LaTeX file. For loading figures
%% from other directories you can use the `import` package
%%   \usepackage{import}
%%
%% and then include the figures with
%%   \import{<path to file>}{<filename>.pgf}
%%
%% Matplotlib used the following preamble
%%   
%%   \usepackage{fontspec}
%%   \makeatletter\@ifpackageloaded{underscore}{}{\usepackage[strings]{underscore}}\makeatother
%%
\begingroup%
\makeatletter%
\begin{pgfpicture}%
\pgfpathrectangle{\pgfpointorigin}{\pgfqpoint{2.121861in}{1.654444in}}%
\pgfusepath{use as bounding box, clip}%
\begin{pgfscope}%
\pgfsetbuttcap%
\pgfsetmiterjoin%
\definecolor{currentfill}{rgb}{1.000000,1.000000,1.000000}%
\pgfsetfillcolor{currentfill}%
\pgfsetlinewidth{0.000000pt}%
\definecolor{currentstroke}{rgb}{1.000000,1.000000,1.000000}%
\pgfsetstrokecolor{currentstroke}%
\pgfsetdash{}{0pt}%
\pgfpathmoveto{\pgfqpoint{0.000000in}{0.000000in}}%
\pgfpathlineto{\pgfqpoint{2.121861in}{0.000000in}}%
\pgfpathlineto{\pgfqpoint{2.121861in}{1.654444in}}%
\pgfpathlineto{\pgfqpoint{0.000000in}{1.654444in}}%
\pgfpathlineto{\pgfqpoint{0.000000in}{0.000000in}}%
\pgfpathclose%
\pgfusepath{fill}%
\end{pgfscope}%
\begin{pgfscope}%
\pgfsetbuttcap%
\pgfsetmiterjoin%
\definecolor{currentfill}{rgb}{1.000000,1.000000,1.000000}%
\pgfsetfillcolor{currentfill}%
\pgfsetlinewidth{0.000000pt}%
\definecolor{currentstroke}{rgb}{0.000000,0.000000,0.000000}%
\pgfsetstrokecolor{currentstroke}%
\pgfsetstrokeopacity{0.000000}%
\pgfsetdash{}{0pt}%
\pgfpathmoveto{\pgfqpoint{0.503581in}{0.449444in}}%
\pgfpathlineto{\pgfqpoint{2.053581in}{0.449444in}}%
\pgfpathlineto{\pgfqpoint{2.053581in}{1.604444in}}%
\pgfpathlineto{\pgfqpoint{0.503581in}{1.604444in}}%
\pgfpathlineto{\pgfqpoint{0.503581in}{0.449444in}}%
\pgfpathclose%
\pgfusepath{fill}%
\end{pgfscope}%
\begin{pgfscope}%
\pgfsetbuttcap%
\pgfsetroundjoin%
\definecolor{currentfill}{rgb}{0.000000,0.000000,0.000000}%
\pgfsetfillcolor{currentfill}%
\pgfsetlinewidth{0.803000pt}%
\definecolor{currentstroke}{rgb}{0.000000,0.000000,0.000000}%
\pgfsetstrokecolor{currentstroke}%
\pgfsetdash{}{0pt}%
\pgfsys@defobject{currentmarker}{\pgfqpoint{0.000000in}{-0.048611in}}{\pgfqpoint{0.000000in}{0.000000in}}{%
\pgfpathmoveto{\pgfqpoint{0.000000in}{0.000000in}}%
\pgfpathlineto{\pgfqpoint{0.000000in}{-0.048611in}}%
\pgfusepath{stroke,fill}%
}%
\begin{pgfscope}%
\pgfsys@transformshift{0.574035in}{0.449444in}%
\pgfsys@useobject{currentmarker}{}%
\end{pgfscope}%
\end{pgfscope}%
\begin{pgfscope}%
\definecolor{textcolor}{rgb}{0.000000,0.000000,0.000000}%
\pgfsetstrokecolor{textcolor}%
\pgfsetfillcolor{textcolor}%
\pgftext[x=0.574035in,y=0.352222in,,top]{\color{textcolor}\rmfamily\fontsize{10.000000}{12.000000}\selectfont \(\displaystyle {0.0}\)}%
\end{pgfscope}%
\begin{pgfscope}%
\pgfsetbuttcap%
\pgfsetroundjoin%
\definecolor{currentfill}{rgb}{0.000000,0.000000,0.000000}%
\pgfsetfillcolor{currentfill}%
\pgfsetlinewidth{0.803000pt}%
\definecolor{currentstroke}{rgb}{0.000000,0.000000,0.000000}%
\pgfsetstrokecolor{currentstroke}%
\pgfsetdash{}{0pt}%
\pgfsys@defobject{currentmarker}{\pgfqpoint{0.000000in}{-0.048611in}}{\pgfqpoint{0.000000in}{0.000000in}}{%
\pgfpathmoveto{\pgfqpoint{0.000000in}{0.000000in}}%
\pgfpathlineto{\pgfqpoint{0.000000in}{-0.048611in}}%
\pgfusepath{stroke,fill}%
}%
\begin{pgfscope}%
\pgfsys@transformshift{1.278581in}{0.449444in}%
\pgfsys@useobject{currentmarker}{}%
\end{pgfscope}%
\end{pgfscope}%
\begin{pgfscope}%
\definecolor{textcolor}{rgb}{0.000000,0.000000,0.000000}%
\pgfsetstrokecolor{textcolor}%
\pgfsetfillcolor{textcolor}%
\pgftext[x=1.278581in,y=0.352222in,,top]{\color{textcolor}\rmfamily\fontsize{10.000000}{12.000000}\selectfont \(\displaystyle {0.5}\)}%
\end{pgfscope}%
\begin{pgfscope}%
\pgfsetbuttcap%
\pgfsetroundjoin%
\definecolor{currentfill}{rgb}{0.000000,0.000000,0.000000}%
\pgfsetfillcolor{currentfill}%
\pgfsetlinewidth{0.803000pt}%
\definecolor{currentstroke}{rgb}{0.000000,0.000000,0.000000}%
\pgfsetstrokecolor{currentstroke}%
\pgfsetdash{}{0pt}%
\pgfsys@defobject{currentmarker}{\pgfqpoint{0.000000in}{-0.048611in}}{\pgfqpoint{0.000000in}{0.000000in}}{%
\pgfpathmoveto{\pgfqpoint{0.000000in}{0.000000in}}%
\pgfpathlineto{\pgfqpoint{0.000000in}{-0.048611in}}%
\pgfusepath{stroke,fill}%
}%
\begin{pgfscope}%
\pgfsys@transformshift{1.983126in}{0.449444in}%
\pgfsys@useobject{currentmarker}{}%
\end{pgfscope}%
\end{pgfscope}%
\begin{pgfscope}%
\definecolor{textcolor}{rgb}{0.000000,0.000000,0.000000}%
\pgfsetstrokecolor{textcolor}%
\pgfsetfillcolor{textcolor}%
\pgftext[x=1.983126in,y=0.352222in,,top]{\color{textcolor}\rmfamily\fontsize{10.000000}{12.000000}\selectfont \(\displaystyle {1.0}\)}%
\end{pgfscope}%
\begin{pgfscope}%
\definecolor{textcolor}{rgb}{0.000000,0.000000,0.000000}%
\pgfsetstrokecolor{textcolor}%
\pgfsetfillcolor{textcolor}%
\pgftext[x=1.278581in,y=0.173333in,,top]{\color{textcolor}\rmfamily\fontsize{10.000000}{12.000000}\selectfont False positive rate}%
\end{pgfscope}%
\begin{pgfscope}%
\pgfsetbuttcap%
\pgfsetroundjoin%
\definecolor{currentfill}{rgb}{0.000000,0.000000,0.000000}%
\pgfsetfillcolor{currentfill}%
\pgfsetlinewidth{0.803000pt}%
\definecolor{currentstroke}{rgb}{0.000000,0.000000,0.000000}%
\pgfsetstrokecolor{currentstroke}%
\pgfsetdash{}{0pt}%
\pgfsys@defobject{currentmarker}{\pgfqpoint{-0.048611in}{0.000000in}}{\pgfqpoint{-0.000000in}{0.000000in}}{%
\pgfpathmoveto{\pgfqpoint{-0.000000in}{0.000000in}}%
\pgfpathlineto{\pgfqpoint{-0.048611in}{0.000000in}}%
\pgfusepath{stroke,fill}%
}%
\begin{pgfscope}%
\pgfsys@transformshift{0.503581in}{0.501944in}%
\pgfsys@useobject{currentmarker}{}%
\end{pgfscope}%
\end{pgfscope}%
\begin{pgfscope}%
\definecolor{textcolor}{rgb}{0.000000,0.000000,0.000000}%
\pgfsetstrokecolor{textcolor}%
\pgfsetfillcolor{textcolor}%
\pgftext[x=0.228889in, y=0.453750in, left, base]{\color{textcolor}\rmfamily\fontsize{10.000000}{12.000000}\selectfont \(\displaystyle {0.0}\)}%
\end{pgfscope}%
\begin{pgfscope}%
\pgfsetbuttcap%
\pgfsetroundjoin%
\definecolor{currentfill}{rgb}{0.000000,0.000000,0.000000}%
\pgfsetfillcolor{currentfill}%
\pgfsetlinewidth{0.803000pt}%
\definecolor{currentstroke}{rgb}{0.000000,0.000000,0.000000}%
\pgfsetstrokecolor{currentstroke}%
\pgfsetdash{}{0pt}%
\pgfsys@defobject{currentmarker}{\pgfqpoint{-0.048611in}{0.000000in}}{\pgfqpoint{-0.000000in}{0.000000in}}{%
\pgfpathmoveto{\pgfqpoint{-0.000000in}{0.000000in}}%
\pgfpathlineto{\pgfqpoint{-0.048611in}{0.000000in}}%
\pgfusepath{stroke,fill}%
}%
\begin{pgfscope}%
\pgfsys@transformshift{0.503581in}{1.026944in}%
\pgfsys@useobject{currentmarker}{}%
\end{pgfscope}%
\end{pgfscope}%
\begin{pgfscope}%
\definecolor{textcolor}{rgb}{0.000000,0.000000,0.000000}%
\pgfsetstrokecolor{textcolor}%
\pgfsetfillcolor{textcolor}%
\pgftext[x=0.228889in, y=0.978750in, left, base]{\color{textcolor}\rmfamily\fontsize{10.000000}{12.000000}\selectfont \(\displaystyle {0.5}\)}%
\end{pgfscope}%
\begin{pgfscope}%
\pgfsetbuttcap%
\pgfsetroundjoin%
\definecolor{currentfill}{rgb}{0.000000,0.000000,0.000000}%
\pgfsetfillcolor{currentfill}%
\pgfsetlinewidth{0.803000pt}%
\definecolor{currentstroke}{rgb}{0.000000,0.000000,0.000000}%
\pgfsetstrokecolor{currentstroke}%
\pgfsetdash{}{0pt}%
\pgfsys@defobject{currentmarker}{\pgfqpoint{-0.048611in}{0.000000in}}{\pgfqpoint{-0.000000in}{0.000000in}}{%
\pgfpathmoveto{\pgfqpoint{-0.000000in}{0.000000in}}%
\pgfpathlineto{\pgfqpoint{-0.048611in}{0.000000in}}%
\pgfusepath{stroke,fill}%
}%
\begin{pgfscope}%
\pgfsys@transformshift{0.503581in}{1.551944in}%
\pgfsys@useobject{currentmarker}{}%
\end{pgfscope}%
\end{pgfscope}%
\begin{pgfscope}%
\definecolor{textcolor}{rgb}{0.000000,0.000000,0.000000}%
\pgfsetstrokecolor{textcolor}%
\pgfsetfillcolor{textcolor}%
\pgftext[x=0.228889in, y=1.503750in, left, base]{\color{textcolor}\rmfamily\fontsize{10.000000}{12.000000}\selectfont \(\displaystyle {1.0}\)}%
\end{pgfscope}%
\begin{pgfscope}%
\definecolor{textcolor}{rgb}{0.000000,0.000000,0.000000}%
\pgfsetstrokecolor{textcolor}%
\pgfsetfillcolor{textcolor}%
\pgftext[x=0.173333in,y=1.026944in,,bottom,rotate=90.000000]{\color{textcolor}\rmfamily\fontsize{10.000000}{12.000000}\selectfont True positive rate}%
\end{pgfscope}%
\begin{pgfscope}%
\pgfpathrectangle{\pgfqpoint{0.503581in}{0.449444in}}{\pgfqpoint{1.550000in}{1.155000in}}%
\pgfusepath{clip}%
\pgfsetbuttcap%
\pgfsetroundjoin%
\pgfsetlinewidth{1.505625pt}%
\definecolor{currentstroke}{rgb}{0.000000,0.000000,0.000000}%
\pgfsetstrokecolor{currentstroke}%
\pgfsetdash{{5.550000pt}{2.400000pt}}{0.000000pt}%
\pgfpathmoveto{\pgfqpoint{0.574035in}{0.501944in}}%
\pgfpathlineto{\pgfqpoint{1.983126in}{1.551944in}}%
\pgfusepath{stroke}%
\end{pgfscope}%
\begin{pgfscope}%
\pgfpathrectangle{\pgfqpoint{0.503581in}{0.449444in}}{\pgfqpoint{1.550000in}{1.155000in}}%
\pgfusepath{clip}%
\pgfsetrectcap%
\pgfsetroundjoin%
\pgfsetlinewidth{1.505625pt}%
\definecolor{currentstroke}{rgb}{0.000000,0.000000,0.000000}%
\pgfsetstrokecolor{currentstroke}%
\pgfsetdash{}{0pt}%
\pgfpathmoveto{\pgfqpoint{0.574035in}{0.501944in}}%
\pgfpathlineto{\pgfqpoint{0.575145in}{0.511486in}}%
\pgfpathlineto{\pgfqpoint{0.575273in}{0.512565in}}%
\pgfpathlineto{\pgfqpoint{0.576383in}{0.519753in}}%
\pgfpathlineto{\pgfqpoint{0.576527in}{0.520803in}}%
\pgfpathlineto{\pgfqpoint{0.577637in}{0.527203in}}%
\pgfpathlineto{\pgfqpoint{0.577849in}{0.528312in}}%
\pgfpathlineto{\pgfqpoint{0.578959in}{0.534002in}}%
\pgfpathlineto{\pgfqpoint{0.579224in}{0.535110in}}%
\pgfpathlineto{\pgfqpoint{0.580322in}{0.540499in}}%
\pgfpathlineto{\pgfqpoint{0.580569in}{0.541608in}}%
\pgfpathlineto{\pgfqpoint{0.581676in}{0.547083in}}%
\pgfpathlineto{\pgfqpoint{0.581902in}{0.548173in}}%
\pgfpathlineto{\pgfqpoint{0.583012in}{0.553464in}}%
\pgfpathlineto{\pgfqpoint{0.583268in}{0.554573in}}%
\pgfpathlineto{\pgfqpoint{0.584373in}{0.559416in}}%
\pgfpathlineto{\pgfqpoint{0.584661in}{0.560506in}}%
\pgfpathlineto{\pgfqpoint{0.584661in}{0.560515in}}%
\pgfpathlineto{\pgfqpoint{0.585769in}{0.564980in}}%
\pgfpathlineto{\pgfqpoint{0.586002in}{0.566088in}}%
\pgfpathlineto{\pgfqpoint{0.587111in}{0.570300in}}%
\pgfpathlineto{\pgfqpoint{0.587360in}{0.571399in}}%
\pgfpathlineto{\pgfqpoint{0.588468in}{0.575892in}}%
\pgfpathlineto{\pgfqpoint{0.588717in}{0.577001in}}%
\pgfpathlineto{\pgfqpoint{0.589822in}{0.581602in}}%
\pgfpathlineto{\pgfqpoint{0.590166in}{0.582691in}}%
\pgfpathlineto{\pgfqpoint{0.591276in}{0.586951in}}%
\pgfpathlineto{\pgfqpoint{0.591546in}{0.588060in}}%
\pgfpathlineto{\pgfqpoint{0.592651in}{0.592767in}}%
\pgfpathlineto{\pgfqpoint{0.592998in}{0.593876in}}%
\pgfpathlineto{\pgfqpoint{0.594106in}{0.597708in}}%
\pgfpathlineto{\pgfqpoint{0.594341in}{0.598817in}}%
\pgfpathlineto{\pgfqpoint{0.595437in}{0.603291in}}%
\pgfpathlineto{\pgfqpoint{0.595753in}{0.604400in}}%
\pgfpathlineto{\pgfqpoint{0.596863in}{0.608300in}}%
\pgfpathlineto{\pgfqpoint{0.597179in}{0.609409in}}%
\pgfpathlineto{\pgfqpoint{0.598289in}{0.613562in}}%
\pgfpathlineto{\pgfqpoint{0.598578in}{0.614641in}}%
\pgfpathlineto{\pgfqpoint{0.599685in}{0.618269in}}%
\pgfpathlineto{\pgfqpoint{0.599985in}{0.619359in}}%
\pgfpathlineto{\pgfqpoint{0.601093in}{0.623084in}}%
\pgfpathlineto{\pgfqpoint{0.601451in}{0.624193in}}%
\pgfpathlineto{\pgfqpoint{0.602559in}{0.628248in}}%
\pgfpathlineto{\pgfqpoint{0.602959in}{0.629357in}}%
\pgfpathlineto{\pgfqpoint{0.604064in}{0.632654in}}%
\pgfpathlineto{\pgfqpoint{0.604460in}{0.633763in}}%
\pgfpathlineto{\pgfqpoint{0.605569in}{0.637595in}}%
\pgfpathlineto{\pgfqpoint{0.605907in}{0.638704in}}%
\pgfpathlineto{\pgfqpoint{0.607014in}{0.642089in}}%
\pgfpathlineto{\pgfqpoint{0.607359in}{0.643198in}}%
\pgfpathlineto{\pgfqpoint{0.608464in}{0.646660in}}%
\pgfpathlineto{\pgfqpoint{0.608822in}{0.647769in}}%
\pgfpathlineto{\pgfqpoint{0.609932in}{0.651484in}}%
\pgfpathlineto{\pgfqpoint{0.610362in}{0.652593in}}%
\pgfpathlineto{\pgfqpoint{0.611468in}{0.656581in}}%
\pgfpathlineto{\pgfqpoint{0.611840in}{0.657670in}}%
\pgfpathlineto{\pgfqpoint{0.612947in}{0.661687in}}%
\pgfpathlineto{\pgfqpoint{0.613415in}{0.662786in}}%
\pgfpathlineto{\pgfqpoint{0.614525in}{0.666453in}}%
\pgfpathlineto{\pgfqpoint{0.614886in}{0.667552in}}%
\pgfpathlineto{\pgfqpoint{0.615989in}{0.670762in}}%
\pgfpathlineto{\pgfqpoint{0.616375in}{0.671870in}}%
\pgfpathlineto{\pgfqpoint{0.617485in}{0.675090in}}%
\pgfpathlineto{\pgfqpoint{0.617866in}{0.676198in}}%
\pgfpathlineto{\pgfqpoint{0.618969in}{0.679428in}}%
\pgfpathlineto{\pgfqpoint{0.619355in}{0.680527in}}%
\pgfpathlineto{\pgfqpoint{0.620465in}{0.683532in}}%
\pgfpathlineto{\pgfqpoint{0.620844in}{0.684641in}}%
\pgfpathlineto{\pgfqpoint{0.621952in}{0.687948in}}%
\pgfpathlineto{\pgfqpoint{0.622327in}{0.689037in}}%
\pgfpathlineto{\pgfqpoint{0.623434in}{0.692169in}}%
\pgfpathlineto{\pgfqpoint{0.623774in}{0.693278in}}%
\pgfpathlineto{\pgfqpoint{0.624881in}{0.696662in}}%
\pgfpathlineto{\pgfqpoint{0.625363in}{0.697752in}}%
\pgfpathlineto{\pgfqpoint{0.626473in}{0.701234in}}%
\pgfpathlineto{\pgfqpoint{0.626833in}{0.702333in}}%
\pgfpathlineto{\pgfqpoint{0.627943in}{0.705163in}}%
\pgfpathlineto{\pgfqpoint{0.628318in}{0.706272in}}%
\pgfpathlineto{\pgfqpoint{0.629418in}{0.709238in}}%
\pgfpathlineto{\pgfqpoint{0.629872in}{0.710347in}}%
\pgfpathlineto{\pgfqpoint{0.630968in}{0.713265in}}%
\pgfpathlineto{\pgfqpoint{0.631459in}{0.714374in}}%
\pgfpathlineto{\pgfqpoint{0.632564in}{0.716834in}}%
\pgfpathlineto{\pgfqpoint{0.633139in}{0.717943in}}%
\pgfpathlineto{\pgfqpoint{0.634242in}{0.720735in}}%
\pgfpathlineto{\pgfqpoint{0.634667in}{0.721834in}}%
\pgfpathlineto{\pgfqpoint{0.635770in}{0.724878in}}%
\pgfpathlineto{\pgfqpoint{0.636268in}{0.725977in}}%
\pgfpathlineto{\pgfqpoint{0.637376in}{0.728681in}}%
\pgfpathlineto{\pgfqpoint{0.637857in}{0.729780in}}%
\pgfpathlineto{\pgfqpoint{0.638967in}{0.732620in}}%
\pgfpathlineto{\pgfqpoint{0.639374in}{0.733709in}}%
\pgfpathlineto{\pgfqpoint{0.640484in}{0.736404in}}%
\pgfpathlineto{\pgfqpoint{0.640931in}{0.737512in}}%
\pgfpathlineto{\pgfqpoint{0.642029in}{0.740382in}}%
\pgfpathlineto{\pgfqpoint{0.642476in}{0.741471in}}%
\pgfpathlineto{\pgfqpoint{0.643572in}{0.744039in}}%
\pgfpathlineto{\pgfqpoint{0.643584in}{0.744039in}}%
\pgfpathlineto{\pgfqpoint{0.644019in}{0.745138in}}%
\pgfpathlineto{\pgfqpoint{0.645128in}{0.747637in}}%
\pgfpathlineto{\pgfqpoint{0.645594in}{0.748736in}}%
\pgfpathlineto{\pgfqpoint{0.646701in}{0.751265in}}%
\pgfpathlineto{\pgfqpoint{0.647125in}{0.752374in}}%
\pgfpathlineto{\pgfqpoint{0.648235in}{0.755117in}}%
\pgfpathlineto{\pgfqpoint{0.648726in}{0.756226in}}%
\pgfpathlineto{\pgfqpoint{0.649833in}{0.758929in}}%
\pgfpathlineto{\pgfqpoint{0.650315in}{0.760009in}}%
\pgfpathlineto{\pgfqpoint{0.651425in}{0.762616in}}%
\pgfpathlineto{\pgfqpoint{0.651846in}{0.763705in}}%
\pgfpathlineto{\pgfqpoint{0.652956in}{0.766273in}}%
\pgfpathlineto{\pgfqpoint{0.653465in}{0.767381in}}%
\pgfpathlineto{\pgfqpoint{0.654573in}{0.769891in}}%
\pgfpathlineto{\pgfqpoint{0.655078in}{0.770990in}}%
\pgfpathlineto{\pgfqpoint{0.656173in}{0.773334in}}%
\pgfpathlineto{\pgfqpoint{0.656185in}{0.773334in}}%
\pgfpathlineto{\pgfqpoint{0.656655in}{0.774433in}}%
\pgfpathlineto{\pgfqpoint{0.657760in}{0.777283in}}%
\pgfpathlineto{\pgfqpoint{0.658309in}{0.778392in}}%
\pgfpathlineto{\pgfqpoint{0.659417in}{0.780424in}}%
\pgfpathlineto{\pgfqpoint{0.659894in}{0.781523in}}%
\pgfpathlineto{\pgfqpoint{0.661004in}{0.784169in}}%
\pgfpathlineto{\pgfqpoint{0.661502in}{0.785278in}}%
\pgfpathlineto{\pgfqpoint{0.662605in}{0.787379in}}%
\pgfpathlineto{\pgfqpoint{0.663114in}{0.788487in}}%
\pgfpathlineto{\pgfqpoint{0.664212in}{0.791074in}}%
\pgfpathlineto{\pgfqpoint{0.664850in}{0.792174in}}%
\pgfpathlineto{\pgfqpoint{0.665955in}{0.794333in}}%
\pgfpathlineto{\pgfqpoint{0.666434in}{0.795422in}}%
\pgfpathlineto{\pgfqpoint{0.667544in}{0.797815in}}%
\pgfpathlineto{\pgfqpoint{0.668268in}{0.798914in}}%
\pgfpathlineto{\pgfqpoint{0.669366in}{0.801248in}}%
\pgfpathlineto{\pgfqpoint{0.669964in}{0.802357in}}%
\pgfpathlineto{\pgfqpoint{0.671067in}{0.804925in}}%
\pgfpathlineto{\pgfqpoint{0.671653in}{0.806033in}}%
\pgfpathlineto{\pgfqpoint{0.672758in}{0.808426in}}%
\pgfpathlineto{\pgfqpoint{0.673310in}{0.809535in}}%
\pgfpathlineto{\pgfqpoint{0.674415in}{0.811927in}}%
\pgfpathlineto{\pgfqpoint{0.675032in}{0.813026in}}%
\pgfpathlineto{\pgfqpoint{0.676141in}{0.815195in}}%
\pgfpathlineto{\pgfqpoint{0.676635in}{0.816294in}}%
\pgfpathlineto{\pgfqpoint{0.677742in}{0.818590in}}%
\pgfpathlineto{\pgfqpoint{0.678291in}{0.819669in}}%
\pgfpathlineto{\pgfqpoint{0.679401in}{0.821984in}}%
\pgfpathlineto{\pgfqpoint{0.679997in}{0.823093in}}%
\pgfpathlineto{\pgfqpoint{0.681104in}{0.825427in}}%
\pgfpathlineto{\pgfqpoint{0.681621in}{0.826536in}}%
\pgfpathlineto{\pgfqpoint{0.682731in}{0.828598in}}%
\pgfpathlineto{\pgfqpoint{0.683266in}{0.829707in}}%
\pgfpathlineto{\pgfqpoint{0.684369in}{0.831730in}}%
\pgfpathlineto{\pgfqpoint{0.684983in}{0.832839in}}%
\pgfpathlineto{\pgfqpoint{0.686093in}{0.834920in}}%
\pgfpathlineto{\pgfqpoint{0.686607in}{0.836029in}}%
\pgfpathlineto{\pgfqpoint{0.687710in}{0.838188in}}%
\pgfpathlineto{\pgfqpoint{0.688368in}{0.839287in}}%
\pgfpathlineto{\pgfqpoint{0.689478in}{0.841252in}}%
\pgfpathlineto{\pgfqpoint{0.690102in}{0.842361in}}%
\pgfpathlineto{\pgfqpoint{0.691212in}{0.844423in}}%
\pgfpathlineto{\pgfqpoint{0.691996in}{0.845522in}}%
\pgfpathlineto{\pgfqpoint{0.693099in}{0.847681in}}%
\pgfpathlineto{\pgfqpoint{0.693657in}{0.848790in}}%
\pgfpathlineto{\pgfqpoint{0.694767in}{0.850560in}}%
\pgfpathlineto{\pgfqpoint{0.695363in}{0.851669in}}%
\pgfpathlineto{\pgfqpoint{0.696470in}{0.853507in}}%
\pgfpathlineto{\pgfqpoint{0.697108in}{0.854616in}}%
\pgfpathlineto{\pgfqpoint{0.698208in}{0.856464in}}%
\pgfpathlineto{\pgfqpoint{0.698936in}{0.857563in}}%
\pgfpathlineto{\pgfqpoint{0.700046in}{0.859644in}}%
\pgfpathlineto{\pgfqpoint{0.700840in}{0.860753in}}%
\pgfpathlineto{\pgfqpoint{0.701950in}{0.862620in}}%
\pgfpathlineto{\pgfqpoint{0.702657in}{0.863729in}}%
\pgfpathlineto{\pgfqpoint{0.703767in}{0.865538in}}%
\pgfpathlineto{\pgfqpoint{0.704353in}{0.866647in}}%
\pgfpathlineto{\pgfqpoint{0.705461in}{0.868816in}}%
\pgfpathlineto{\pgfqpoint{0.706138in}{0.869915in}}%
\pgfpathlineto{\pgfqpoint{0.707241in}{0.871792in}}%
\pgfpathlineto{\pgfqpoint{0.707724in}{0.872901in}}%
\pgfpathlineto{\pgfqpoint{0.708830in}{0.874729in}}%
\pgfpathlineto{\pgfqpoint{0.709397in}{0.875828in}}%
\pgfpathlineto{\pgfqpoint{0.710505in}{0.877540in}}%
\pgfpathlineto{\pgfqpoint{0.711070in}{0.878649in}}%
\pgfpathlineto{\pgfqpoint{0.712173in}{0.880662in}}%
\pgfpathlineto{\pgfqpoint{0.712899in}{0.881761in}}%
\pgfpathlineto{\pgfqpoint{0.714009in}{0.883775in}}%
\pgfpathlineto{\pgfqpoint{0.714735in}{0.884884in}}%
\pgfpathlineto{\pgfqpoint{0.715842in}{0.886712in}}%
\pgfpathlineto{\pgfqpoint{0.716603in}{0.887821in}}%
\pgfpathlineto{\pgfqpoint{0.717713in}{0.890058in}}%
\pgfpathlineto{\pgfqpoint{0.718369in}{0.891167in}}%
\pgfpathlineto{\pgfqpoint{0.719456in}{0.892820in}}%
\pgfpathlineto{\pgfqpoint{0.720207in}{0.893919in}}%
\pgfpathlineto{\pgfqpoint{0.721310in}{0.895602in}}%
\pgfpathlineto{\pgfqpoint{0.722087in}{0.896711in}}%
\pgfpathlineto{\pgfqpoint{0.723197in}{0.898617in}}%
\pgfpathlineto{\pgfqpoint{0.723777in}{0.899726in}}%
\pgfpathlineto{\pgfqpoint{0.724884in}{0.901564in}}%
\pgfpathlineto{\pgfqpoint{0.725664in}{0.902673in}}%
\pgfpathlineto{\pgfqpoint{0.726755in}{0.904414in}}%
\pgfpathlineto{\pgfqpoint{0.727539in}{0.905522in}}%
\pgfpathlineto{\pgfqpoint{0.728628in}{0.907419in}}%
\pgfpathlineto{\pgfqpoint{0.728632in}{0.907419in}}%
\pgfpathlineto{\pgfqpoint{0.729337in}{0.908528in}}%
\pgfpathlineto{\pgfqpoint{0.730445in}{0.910259in}}%
\pgfpathlineto{\pgfqpoint{0.731134in}{0.911368in}}%
\pgfpathlineto{\pgfqpoint{0.732244in}{0.912982in}}%
\pgfpathlineto{\pgfqpoint{0.732963in}{0.914091in}}%
\pgfpathlineto{\pgfqpoint{0.734072in}{0.915842in}}%
\pgfpathlineto{\pgfqpoint{0.734733in}{0.916951in}}%
\pgfpathlineto{\pgfqpoint{0.735841in}{0.918770in}}%
\pgfpathlineto{\pgfqpoint{0.736513in}{0.919849in}}%
\pgfpathlineto{\pgfqpoint{0.737623in}{0.921726in}}%
\pgfpathlineto{\pgfqpoint{0.738368in}{0.922835in}}%
\pgfpathlineto{\pgfqpoint{0.739473in}{0.924654in}}%
\pgfpathlineto{\pgfqpoint{0.740196in}{0.925763in}}%
\pgfpathlineto{\pgfqpoint{0.741304in}{0.927552in}}%
\pgfpathlineto{\pgfqpoint{0.742169in}{0.928661in}}%
\pgfpathlineto{\pgfqpoint{0.743277in}{0.930334in}}%
\pgfpathlineto{\pgfqpoint{0.743991in}{0.931433in}}%
\pgfpathlineto{\pgfqpoint{0.745099in}{0.932921in}}%
\pgfpathlineto{\pgfqpoint{0.745739in}{0.934020in}}%
\pgfpathlineto{\pgfqpoint{0.746839in}{0.935333in}}%
\pgfpathlineto{\pgfqpoint{0.747663in}{0.936432in}}%
\pgfpathlineto{\pgfqpoint{0.748756in}{0.937950in}}%
\pgfpathlineto{\pgfqpoint{0.749552in}{0.939058in}}%
\pgfpathlineto{\pgfqpoint{0.750657in}{0.940634in}}%
\pgfpathlineto{\pgfqpoint{0.751290in}{0.941743in}}%
\pgfpathlineto{\pgfqpoint{0.752379in}{0.943299in}}%
\pgfpathlineto{\pgfqpoint{0.753126in}{0.944408in}}%
\pgfpathlineto{\pgfqpoint{0.754236in}{0.946314in}}%
\pgfpathlineto{\pgfqpoint{0.754911in}{0.947423in}}%
\pgfpathlineto{\pgfqpoint{0.756018in}{0.949037in}}%
\pgfpathlineto{\pgfqpoint{0.756818in}{0.950146in}}%
\pgfpathlineto{\pgfqpoint{0.757921in}{0.951566in}}%
\pgfpathlineto{\pgfqpoint{0.758519in}{0.952675in}}%
\pgfpathlineto{\pgfqpoint{0.759622in}{0.954562in}}%
\pgfpathlineto{\pgfqpoint{0.760390in}{0.955671in}}%
\pgfpathlineto{\pgfqpoint{0.761432in}{0.956984in}}%
\pgfpathlineto{\pgfqpoint{0.762354in}{0.958093in}}%
\pgfpathlineto{\pgfqpoint{0.763461in}{0.959736in}}%
\pgfpathlineto{\pgfqpoint{0.764308in}{0.960826in}}%
\pgfpathlineto{\pgfqpoint{0.765418in}{0.962508in}}%
\pgfpathlineto{\pgfqpoint{0.766223in}{0.963617in}}%
\pgfpathlineto{\pgfqpoint{0.767333in}{0.964998in}}%
\pgfpathlineto{\pgfqpoint{0.768080in}{0.966107in}}%
\pgfpathlineto{\pgfqpoint{0.769190in}{0.967663in}}%
\pgfpathlineto{\pgfqpoint{0.770239in}{0.968772in}}%
\pgfpathlineto{\pgfqpoint{0.771349in}{0.970484in}}%
\pgfpathlineto{\pgfqpoint{0.772221in}{0.971573in}}%
\pgfpathlineto{\pgfqpoint{0.773310in}{0.972915in}}%
\pgfpathlineto{\pgfqpoint{0.774220in}{0.974024in}}%
\pgfpathlineto{\pgfqpoint{0.775328in}{0.975522in}}%
\pgfpathlineto{\pgfqpoint{0.776242in}{0.976631in}}%
\pgfpathlineto{\pgfqpoint{0.777345in}{0.978245in}}%
\pgfpathlineto{\pgfqpoint{0.778059in}{0.979354in}}%
\pgfpathlineto{\pgfqpoint{0.779162in}{0.980774in}}%
\pgfpathlineto{\pgfqpoint{0.780177in}{0.981873in}}%
\pgfpathlineto{\pgfqpoint{0.781286in}{0.983274in}}%
\pgfpathlineto{\pgfqpoint{0.782189in}{0.984382in}}%
\pgfpathlineto{\pgfqpoint{0.783294in}{0.985715in}}%
\pgfpathlineto{\pgfqpoint{0.784199in}{0.986824in}}%
\pgfpathlineto{\pgfqpoint{0.785300in}{0.988341in}}%
\pgfpathlineto{\pgfqpoint{0.786159in}{0.989450in}}%
\pgfpathlineto{\pgfqpoint{0.787266in}{0.991171in}}%
\pgfpathlineto{\pgfqpoint{0.788094in}{0.992280in}}%
\pgfpathlineto{\pgfqpoint{0.789204in}{0.993817in}}%
\pgfpathlineto{\pgfqpoint{0.790170in}{0.994926in}}%
\pgfpathlineto{\pgfqpoint{0.791266in}{0.996433in}}%
\pgfpathlineto{\pgfqpoint{0.792029in}{0.997542in}}%
\pgfpathlineto{\pgfqpoint{0.793139in}{0.998933in}}%
\pgfpathlineto{\pgfqpoint{0.794149in}{1.000042in}}%
\pgfpathlineto{\pgfqpoint{0.795242in}{1.001539in}}%
\pgfpathlineto{\pgfqpoint{0.796138in}{1.002648in}}%
\pgfpathlineto{\pgfqpoint{0.797231in}{1.004000in}}%
\pgfpathlineto{\pgfqpoint{0.797964in}{1.005070in}}%
\pgfpathlineto{\pgfqpoint{0.799051in}{1.006801in}}%
\pgfpathlineto{\pgfqpoint{0.799740in}{1.007910in}}%
\pgfpathlineto{\pgfqpoint{0.800850in}{1.009291in}}%
\pgfpathlineto{\pgfqpoint{0.801755in}{1.010400in}}%
\pgfpathlineto{\pgfqpoint{0.802853in}{1.011723in}}%
\pgfpathlineto{\pgfqpoint{0.803902in}{1.012831in}}%
\pgfpathlineto{\pgfqpoint{0.805010in}{1.014164in}}%
\pgfpathlineto{\pgfqpoint{0.805789in}{1.015273in}}%
\pgfpathlineto{\pgfqpoint{0.806894in}{1.016741in}}%
\pgfpathlineto{\pgfqpoint{0.807665in}{1.017850in}}%
\pgfpathlineto{\pgfqpoint{0.808772in}{1.019299in}}%
\pgfpathlineto{\pgfqpoint{0.809703in}{1.020408in}}%
\pgfpathlineto{\pgfqpoint{0.810813in}{1.021624in}}%
\pgfpathlineto{\pgfqpoint{0.811692in}{1.022733in}}%
\pgfpathlineto{\pgfqpoint{0.812781in}{1.024211in}}%
\pgfpathlineto{\pgfqpoint{0.813828in}{1.025320in}}%
\pgfpathlineto{\pgfqpoint{0.814936in}{1.026575in}}%
\pgfpathlineto{\pgfqpoint{0.815787in}{1.027683in}}%
\pgfpathlineto{\pgfqpoint{0.816871in}{1.028850in}}%
\pgfpathlineto{\pgfqpoint{0.817772in}{1.029959in}}%
\pgfpathlineto{\pgfqpoint{0.818882in}{1.031389in}}%
\pgfpathlineto{\pgfqpoint{0.819556in}{1.032498in}}%
\pgfpathlineto{\pgfqpoint{0.820659in}{1.033782in}}%
\pgfpathlineto{\pgfqpoint{0.821681in}{1.034890in}}%
\pgfpathlineto{\pgfqpoint{0.822791in}{1.036194in}}%
\pgfpathlineto{\pgfqpoint{0.823900in}{1.037293in}}%
\pgfpathlineto{\pgfqpoint{0.825010in}{1.038596in}}%
\pgfpathlineto{\pgfqpoint{0.825848in}{1.039705in}}%
\pgfpathlineto{\pgfqpoint{0.826958in}{1.041028in}}%
\pgfpathlineto{\pgfqpoint{0.827961in}{1.042127in}}%
\pgfpathlineto{\pgfqpoint{0.827961in}{1.042136in}}%
\pgfpathlineto{\pgfqpoint{0.829054in}{1.043323in}}%
\pgfpathlineto{\pgfqpoint{0.829941in}{1.044432in}}%
\pgfpathlineto{\pgfqpoint{0.831044in}{1.045706in}}%
\pgfpathlineto{\pgfqpoint{0.832188in}{1.046815in}}%
\pgfpathlineto{\pgfqpoint{0.833298in}{1.047953in}}%
\pgfpathlineto{\pgfqpoint{0.834333in}{1.049062in}}%
\pgfpathlineto{\pgfqpoint{0.835422in}{1.050511in}}%
\pgfpathlineto{\pgfqpoint{0.836444in}{1.051619in}}%
\pgfpathlineto{\pgfqpoint{0.837537in}{1.052942in}}%
\pgfpathlineto{\pgfqpoint{0.838373in}{1.054051in}}%
\pgfpathlineto{\pgfqpoint{0.839473in}{1.055296in}}%
\pgfpathlineto{\pgfqpoint{0.840332in}{1.056405in}}%
\pgfpathlineto{\pgfqpoint{0.841430in}{1.057494in}}%
\pgfpathlineto{\pgfqpoint{0.842323in}{1.058603in}}%
\pgfpathlineto{\pgfqpoint{0.843375in}{1.059712in}}%
\pgfpathlineto{\pgfqpoint{0.844331in}{1.060820in}}%
\pgfpathlineto{\pgfqpoint{0.845439in}{1.061978in}}%
\pgfpathlineto{\pgfqpoint{0.846449in}{1.063087in}}%
\pgfpathlineto{\pgfqpoint{0.847552in}{1.064176in}}%
\pgfpathlineto{\pgfqpoint{0.848678in}{1.065285in}}%
\pgfpathlineto{\pgfqpoint{0.849788in}{1.066510in}}%
\pgfpathlineto{\pgfqpoint{0.850800in}{1.067619in}}%
\pgfpathlineto{\pgfqpoint{0.851910in}{1.068883in}}%
\pgfpathlineto{\pgfqpoint{0.852901in}{1.069992in}}%
\pgfpathlineto{\pgfqpoint{0.854008in}{1.071120in}}%
\pgfpathlineto{\pgfqpoint{0.855048in}{1.072229in}}%
\pgfpathlineto{\pgfqpoint{0.856144in}{1.073396in}}%
\pgfpathlineto{\pgfqpoint{0.856154in}{1.073396in}}%
\pgfpathlineto{\pgfqpoint{0.857005in}{1.074505in}}%
\pgfpathlineto{\pgfqpoint{0.858094in}{1.075760in}}%
\pgfpathlineto{\pgfqpoint{0.859043in}{1.076869in}}%
\pgfpathlineto{\pgfqpoint{0.860146in}{1.078337in}}%
\pgfpathlineto{\pgfqpoint{0.861110in}{1.079446in}}%
\pgfpathlineto{\pgfqpoint{0.862215in}{1.080458in}}%
\pgfpathlineto{\pgfqpoint{0.863373in}{1.081537in}}%
\pgfpathlineto{\pgfqpoint{0.864476in}{1.082665in}}%
\pgfpathlineto{\pgfqpoint{0.865419in}{1.083755in}}%
\pgfpathlineto{\pgfqpoint{0.866521in}{1.085039in}}%
\pgfpathlineto{\pgfqpoint{0.867478in}{1.086147in}}%
\pgfpathlineto{\pgfqpoint{0.868588in}{1.087159in}}%
\pgfpathlineto{\pgfqpoint{0.869621in}{1.088268in}}%
\pgfpathlineto{\pgfqpoint{0.871059in}{1.089542in}}%
\pgfpathlineto{\pgfqpoint{0.872138in}{1.090651in}}%
\pgfpathlineto{\pgfqpoint{0.873232in}{1.091935in}}%
\pgfpathlineto{\pgfqpoint{0.874316in}{1.093034in}}%
\pgfpathlineto{\pgfqpoint{0.875421in}{1.094123in}}%
\pgfpathlineto{\pgfqpoint{0.876401in}{1.095232in}}%
\pgfpathlineto{\pgfqpoint{0.877511in}{1.096506in}}%
\pgfpathlineto{\pgfqpoint{0.878502in}{1.097615in}}%
\pgfpathlineto{\pgfqpoint{0.879609in}{1.098782in}}%
\pgfpathlineto{\pgfqpoint{0.880870in}{1.099891in}}%
\pgfpathlineto{\pgfqpoint{0.881976in}{1.101116in}}%
\pgfpathlineto{\pgfqpoint{0.882995in}{1.102225in}}%
\pgfpathlineto{\pgfqpoint{0.884105in}{1.103577in}}%
\pgfpathlineto{\pgfqpoint{0.885147in}{1.104686in}}%
\pgfpathlineto{\pgfqpoint{0.886252in}{1.105872in}}%
\pgfpathlineto{\pgfqpoint{0.887385in}{1.106981in}}%
\pgfpathlineto{\pgfqpoint{0.888488in}{1.107934in}}%
\pgfpathlineto{\pgfqpoint{0.889628in}{1.109043in}}%
\pgfpathlineto{\pgfqpoint{0.890733in}{1.110142in}}%
\pgfpathlineto{\pgfqpoint{0.891829in}{1.111251in}}%
\pgfpathlineto{\pgfqpoint{0.892925in}{1.112428in}}%
\pgfpathlineto{\pgfqpoint{0.894312in}{1.113536in}}%
\pgfpathlineto{\pgfqpoint{0.895415in}{1.114567in}}%
\pgfpathlineto{\pgfqpoint{0.896476in}{1.115676in}}%
\pgfpathlineto{\pgfqpoint{0.897581in}{1.116610in}}%
\pgfpathlineto{\pgfqpoint{0.898693in}{1.117719in}}%
\pgfpathlineto{\pgfqpoint{0.899801in}{1.118740in}}%
\pgfpathlineto{\pgfqpoint{0.901027in}{1.119839in}}%
\pgfpathlineto{\pgfqpoint{0.902137in}{1.120957in}}%
\pgfpathlineto{\pgfqpoint{0.903149in}{1.122057in}}%
\pgfpathlineto{\pgfqpoint{0.904249in}{1.123224in}}%
\pgfpathlineto{\pgfqpoint{0.905357in}{1.124332in}}%
\pgfpathlineto{\pgfqpoint{0.906462in}{1.125500in}}%
\pgfpathlineto{\pgfqpoint{0.907688in}{1.126608in}}%
\pgfpathlineto{\pgfqpoint{0.908777in}{1.127717in}}%
\pgfpathlineto{\pgfqpoint{0.908798in}{1.127717in}}%
\pgfpathlineto{\pgfqpoint{0.909952in}{1.128826in}}%
\pgfpathlineto{\pgfqpoint{0.911053in}{1.129828in}}%
\pgfpathlineto{\pgfqpoint{0.911946in}{1.130937in}}%
\pgfpathlineto{\pgfqpoint{0.913040in}{1.131890in}}%
\pgfpathlineto{\pgfqpoint{0.914271in}{1.132998in}}%
\pgfpathlineto{\pgfqpoint{0.915378in}{1.133981in}}%
\pgfpathlineto{\pgfqpoint{0.916595in}{1.135070in}}%
\pgfpathlineto{\pgfqpoint{0.916595in}{1.135080in}}%
\pgfpathlineto{\pgfqpoint{0.917703in}{1.136403in}}%
\pgfpathlineto{\pgfqpoint{0.918912in}{1.137502in}}%
\pgfpathlineto{\pgfqpoint{0.919987in}{1.138513in}}%
\pgfpathlineto{\pgfqpoint{0.921363in}{1.139622in}}%
\pgfpathlineto{\pgfqpoint{0.922456in}{1.140575in}}%
\pgfpathlineto{\pgfqpoint{0.922472in}{1.140575in}}%
\pgfpathlineto{\pgfqpoint{0.923445in}{1.141684in}}%
\pgfpathlineto{\pgfqpoint{0.924552in}{1.142832in}}%
\pgfpathlineto{\pgfqpoint{0.925904in}{1.143940in}}%
\pgfpathlineto{\pgfqpoint{0.927012in}{1.144913in}}%
\pgfpathlineto{\pgfqpoint{0.928173in}{1.146022in}}%
\pgfpathlineto{\pgfqpoint{0.929271in}{1.147024in}}%
\pgfpathlineto{\pgfqpoint{0.930379in}{1.148094in}}%
\pgfpathlineto{\pgfqpoint{0.931472in}{1.149202in}}%
\pgfpathlineto{\pgfqpoint{0.932710in}{1.150311in}}%
\pgfpathlineto{\pgfqpoint{0.933820in}{1.151371in}}%
\pgfpathlineto{\pgfqpoint{0.935041in}{1.152480in}}%
\pgfpathlineto{\pgfqpoint{0.936151in}{1.153540in}}%
\pgfpathlineto{\pgfqpoint{0.937361in}{1.154649in}}%
\pgfpathlineto{\pgfqpoint{0.938422in}{1.155447in}}%
\pgfpathlineto{\pgfqpoint{0.939762in}{1.156546in}}%
\pgfpathlineto{\pgfqpoint{0.940849in}{1.157518in}}%
\pgfpathlineto{\pgfqpoint{0.942080in}{1.158627in}}%
\pgfpathlineto{\pgfqpoint{0.943176in}{1.159512in}}%
\pgfpathlineto{\pgfqpoint{0.944283in}{1.160621in}}%
\pgfpathlineto{\pgfqpoint{0.945379in}{1.161778in}}%
\pgfpathlineto{\pgfqpoint{0.945391in}{1.161778in}}%
\pgfpathlineto{\pgfqpoint{0.946554in}{1.162887in}}%
\pgfpathlineto{\pgfqpoint{0.947631in}{1.163830in}}%
\pgfpathlineto{\pgfqpoint{0.948792in}{1.164939in}}%
\pgfpathlineto{\pgfqpoint{0.949898in}{1.166252in}}%
\pgfpathlineto{\pgfqpoint{0.949902in}{1.166252in}}%
\pgfpathlineto{\pgfqpoint{0.951321in}{1.167439in}}%
\pgfpathlineto{\pgfqpoint{0.952659in}{1.168528in}}%
\pgfpathlineto{\pgfqpoint{0.953767in}{1.169481in}}%
\pgfpathlineto{\pgfqpoint{0.954942in}{1.170590in}}%
\pgfpathlineto{\pgfqpoint{0.956052in}{1.171592in}}%
\pgfpathlineto{\pgfqpoint{0.957273in}{1.172701in}}%
\pgfpathlineto{\pgfqpoint{0.958383in}{1.173518in}}%
\pgfpathlineto{\pgfqpoint{0.959844in}{1.174627in}}%
\pgfpathlineto{\pgfqpoint{0.960945in}{1.175628in}}%
\pgfpathlineto{\pgfqpoint{0.962339in}{1.176727in}}%
\pgfpathlineto{\pgfqpoint{0.963441in}{1.177817in}}%
\pgfpathlineto{\pgfqpoint{0.964675in}{1.178896in}}%
\pgfpathlineto{\pgfqpoint{0.965754in}{1.180015in}}%
\pgfpathlineto{\pgfqpoint{0.967125in}{1.181124in}}%
\pgfpathlineto{\pgfqpoint{0.968225in}{1.182155in}}%
\pgfpathlineto{\pgfqpoint{0.969412in}{1.183244in}}%
\pgfpathlineto{\pgfqpoint{0.970517in}{1.184246in}}%
\pgfpathlineto{\pgfqpoint{0.972008in}{1.185355in}}%
\pgfpathlineto{\pgfqpoint{0.973051in}{1.186269in}}%
\pgfpathlineto{\pgfqpoint{0.973083in}{1.186269in}}%
\pgfpathlineto{\pgfqpoint{0.974561in}{1.187378in}}%
\pgfpathlineto{\pgfqpoint{0.975664in}{1.188263in}}%
\pgfpathlineto{\pgfqpoint{0.976925in}{1.189371in}}%
\pgfpathlineto{\pgfqpoint{0.978023in}{1.190432in}}%
\pgfpathlineto{\pgfqpoint{0.979403in}{1.191540in}}%
\pgfpathlineto{\pgfqpoint{0.980473in}{1.192474in}}%
\pgfpathlineto{\pgfqpoint{0.980499in}{1.192474in}}%
\pgfpathlineto{\pgfqpoint{0.981871in}{1.193583in}}%
\pgfpathlineto{\pgfqpoint{0.982970in}{1.194653in}}%
\pgfpathlineto{\pgfqpoint{0.984319in}{1.195762in}}%
\pgfpathlineto{\pgfqpoint{0.985408in}{1.196530in}}%
\pgfpathlineto{\pgfqpoint{0.986813in}{1.197639in}}%
\pgfpathlineto{\pgfqpoint{0.987914in}{1.198689in}}%
\pgfpathlineto{\pgfqpoint{0.989547in}{1.199788in}}%
\pgfpathlineto{\pgfqpoint{0.990657in}{1.200761in}}%
\pgfpathlineto{\pgfqpoint{0.992030in}{1.201870in}}%
\pgfpathlineto{\pgfqpoint{0.993121in}{1.202784in}}%
\pgfpathlineto{\pgfqpoint{0.994299in}{1.203893in}}%
\pgfpathlineto{\pgfqpoint{0.995373in}{1.204856in}}%
\pgfpathlineto{\pgfqpoint{0.996909in}{1.205964in}}%
\pgfpathlineto{\pgfqpoint{0.998010in}{1.206820in}}%
\pgfpathlineto{\pgfqpoint{0.999517in}{1.207929in}}%
\pgfpathlineto{\pgfqpoint{1.000625in}{1.208814in}}%
\pgfpathlineto{\pgfqpoint{1.002158in}{1.209923in}}%
\pgfpathlineto{\pgfqpoint{1.003231in}{1.210779in}}%
\pgfpathlineto{\pgfqpoint{1.004394in}{1.211888in}}%
\pgfpathlineto{\pgfqpoint{1.005502in}{1.212928in}}%
\pgfpathlineto{\pgfqpoint{1.006947in}{1.214037in}}%
\pgfpathlineto{\pgfqpoint{1.008056in}{1.215049in}}%
\pgfpathlineto{\pgfqpoint{1.009418in}{1.216157in}}%
\pgfpathlineto{\pgfqpoint{1.010518in}{1.216828in}}%
\pgfpathlineto{\pgfqpoint{1.012049in}{1.217937in}}%
\pgfpathlineto{\pgfqpoint{1.013152in}{1.218725in}}%
\pgfpathlineto{\pgfqpoint{1.014546in}{1.219834in}}%
\pgfpathlineto{\pgfqpoint{1.015628in}{1.220758in}}%
\pgfpathlineto{\pgfqpoint{1.017135in}{1.221867in}}%
\pgfpathlineto{\pgfqpoint{1.018243in}{1.222839in}}%
\pgfpathlineto{\pgfqpoint{1.019620in}{1.223948in}}%
\pgfpathlineto{\pgfqpoint{1.020714in}{1.224814in}}%
\pgfpathlineto{\pgfqpoint{1.022019in}{1.225922in}}%
\pgfpathlineto{\pgfqpoint{1.023120in}{1.226837in}}%
\pgfpathlineto{\pgfqpoint{1.024686in}{1.227936in}}%
\pgfpathlineto{\pgfqpoint{1.025788in}{1.228831in}}%
\pgfpathlineto{\pgfqpoint{1.027364in}{1.229939in}}%
\pgfpathlineto{\pgfqpoint{1.028474in}{1.230717in}}%
\pgfpathlineto{\pgfqpoint{1.029914in}{1.231826in}}%
\pgfpathlineto{\pgfqpoint{1.030991in}{1.232692in}}%
\pgfpathlineto{\pgfqpoint{1.032578in}{1.233801in}}%
\pgfpathlineto{\pgfqpoint{1.033625in}{1.234530in}}%
\pgfpathlineto{\pgfqpoint{1.034995in}{1.235639in}}%
\pgfpathlineto{\pgfqpoint{1.036103in}{1.236495in}}%
\pgfpathlineto{\pgfqpoint{1.037492in}{1.237604in}}%
\pgfpathlineto{\pgfqpoint{1.038592in}{1.238460in}}%
\pgfpathlineto{\pgfqpoint{1.040354in}{1.239568in}}%
\pgfpathlineto{\pgfqpoint{1.041457in}{1.240531in}}%
\pgfpathlineto{\pgfqpoint{1.042999in}{1.241630in}}%
\pgfpathlineto{\pgfqpoint{1.044104in}{1.242486in}}%
\pgfpathlineto{\pgfqpoint{1.045722in}{1.243595in}}%
\pgfpathlineto{\pgfqpoint{1.046822in}{1.244354in}}%
\pgfpathlineto{\pgfqpoint{1.048321in}{1.245462in}}%
\pgfpathlineto{\pgfqpoint{1.049356in}{1.246289in}}%
\pgfpathlineto{\pgfqpoint{1.050906in}{1.247398in}}%
\pgfpathlineto{\pgfqpoint{1.052006in}{1.248127in}}%
\pgfpathlineto{\pgfqpoint{1.053660in}{1.249236in}}%
\pgfpathlineto{\pgfqpoint{1.054768in}{1.249975in}}%
\pgfpathlineto{\pgfqpoint{1.056443in}{1.251084in}}%
\pgfpathlineto{\pgfqpoint{1.057527in}{1.251872in}}%
\pgfpathlineto{\pgfqpoint{1.057546in}{1.251872in}}%
\pgfpathlineto{\pgfqpoint{1.059186in}{1.252981in}}%
\pgfpathlineto{\pgfqpoint{1.060271in}{1.253817in}}%
\pgfpathlineto{\pgfqpoint{1.061850in}{1.254926in}}%
\pgfpathlineto{\pgfqpoint{1.062949in}{1.255626in}}%
\pgfpathlineto{\pgfqpoint{1.064615in}{1.256735in}}%
\pgfpathlineto{\pgfqpoint{1.065717in}{1.257474in}}%
\pgfpathlineto{\pgfqpoint{1.067246in}{1.258583in}}%
\pgfpathlineto{\pgfqpoint{1.068356in}{1.259205in}}%
\pgfpathlineto{\pgfqpoint{1.070180in}{1.260314in}}%
\pgfpathlineto{\pgfqpoint{1.071281in}{1.261005in}}%
\pgfpathlineto{\pgfqpoint{1.073263in}{1.262114in}}%
\pgfpathlineto{\pgfqpoint{1.074354in}{1.262804in}}%
\pgfpathlineto{\pgfqpoint{1.075611in}{1.263913in}}%
\pgfpathlineto{\pgfqpoint{1.076644in}{1.264672in}}%
\pgfpathlineto{\pgfqpoint{1.076693in}{1.264672in}}%
\pgfpathlineto{\pgfqpoint{1.078089in}{1.265780in}}%
\pgfpathlineto{\pgfqpoint{1.079199in}{1.266520in}}%
\pgfpathlineto{\pgfqpoint{1.080678in}{1.267628in}}%
\pgfpathlineto{\pgfqpoint{1.081765in}{1.268562in}}%
\pgfpathlineto{\pgfqpoint{1.083554in}{1.269661in}}%
\pgfpathlineto{\pgfqpoint{1.084664in}{1.270507in}}%
\pgfpathlineto{\pgfqpoint{1.086190in}{1.271616in}}%
\pgfpathlineto{\pgfqpoint{1.087277in}{1.272443in}}%
\pgfpathlineto{\pgfqpoint{1.088836in}{1.273552in}}%
\pgfpathlineto{\pgfqpoint{1.089943in}{1.274330in}}%
\pgfpathlineto{\pgfqpoint{1.091565in}{1.275429in}}%
\pgfpathlineto{\pgfqpoint{1.092656in}{1.276255in}}%
\pgfpathlineto{\pgfqpoint{1.094632in}{1.277364in}}%
\pgfpathlineto{\pgfqpoint{1.095665in}{1.278035in}}%
\pgfpathlineto{\pgfqpoint{1.097212in}{1.279144in}}%
\pgfpathlineto{\pgfqpoint{1.098310in}{1.279961in}}%
\pgfpathlineto{\pgfqpoint{1.100041in}{1.281060in}}%
\pgfpathlineto{\pgfqpoint{1.101151in}{1.281780in}}%
\pgfpathlineto{\pgfqpoint{1.102445in}{1.282889in}}%
\pgfpathlineto{\pgfqpoint{1.103534in}{1.283813in}}%
\pgfpathlineto{\pgfqpoint{1.105016in}{1.284921in}}%
\pgfpathlineto{\pgfqpoint{1.106105in}{1.285593in}}%
\pgfpathlineto{\pgfqpoint{1.107573in}{1.286701in}}%
\pgfpathlineto{\pgfqpoint{1.108576in}{1.287324in}}%
\pgfpathlineto{\pgfqpoint{1.110714in}{1.288433in}}%
\pgfpathlineto{\pgfqpoint{1.111819in}{1.289045in}}%
\pgfpathlineto{\pgfqpoint{1.113783in}{1.290154in}}%
\pgfpathlineto{\pgfqpoint{1.114888in}{1.290854in}}%
\pgfpathlineto{\pgfqpoint{1.116652in}{1.291963in}}%
\pgfpathlineto{\pgfqpoint{1.117759in}{1.292732in}}%
\pgfpathlineto{\pgfqpoint{1.119511in}{1.293840in}}%
\pgfpathlineto{\pgfqpoint{1.120593in}{1.294560in}}%
\pgfpathlineto{\pgfqpoint{1.122408in}{1.295669in}}%
\pgfpathlineto{\pgfqpoint{1.123502in}{1.296389in}}%
\pgfpathlineto{\pgfqpoint{1.125400in}{1.297497in}}%
\pgfpathlineto{\pgfqpoint{1.126503in}{1.298110in}}%
\pgfpathlineto{\pgfqpoint{1.128453in}{1.299219in}}%
\pgfpathlineto{\pgfqpoint{1.129535in}{1.299948in}}%
\pgfpathlineto{\pgfqpoint{1.129551in}{1.299948in}}%
\pgfpathlineto{\pgfqpoint{1.131227in}{1.301057in}}%
\pgfpathlineto{\pgfqpoint{1.132313in}{1.301621in}}%
\pgfpathlineto{\pgfqpoint{1.134198in}{1.302720in}}%
\pgfpathlineto{\pgfqpoint{1.135291in}{1.303411in}}%
\pgfpathlineto{\pgfqpoint{1.137241in}{1.304510in}}%
\pgfpathlineto{\pgfqpoint{1.138337in}{1.305123in}}%
\pgfpathlineto{\pgfqpoint{1.139796in}{1.306232in}}%
\pgfpathlineto{\pgfqpoint{1.140899in}{1.306883in}}%
\pgfpathlineto{\pgfqpoint{1.142809in}{1.307992in}}%
\pgfpathlineto{\pgfqpoint{1.143919in}{1.308653in}}%
\pgfpathlineto{\pgfqpoint{1.145738in}{1.309752in}}%
\pgfpathlineto{\pgfqpoint{1.146846in}{1.310433in}}%
\pgfpathlineto{\pgfqpoint{1.148637in}{1.311542in}}%
\pgfpathlineto{\pgfqpoint{1.149747in}{1.312252in}}%
\pgfpathlineto{\pgfqpoint{1.151753in}{1.313361in}}%
\pgfpathlineto{\pgfqpoint{1.152849in}{1.313886in}}%
\pgfpathlineto{\pgfqpoint{1.154745in}{1.314995in}}%
\pgfpathlineto{\pgfqpoint{1.155832in}{1.315540in}}%
\pgfpathlineto{\pgfqpoint{1.157756in}{1.316648in}}%
\pgfpathlineto{\pgfqpoint{1.158822in}{1.317271in}}%
\pgfpathlineto{\pgfqpoint{1.160511in}{1.318380in}}%
\pgfpathlineto{\pgfqpoint{1.161618in}{1.319090in}}%
\pgfpathlineto{\pgfqpoint{1.163531in}{1.320198in}}%
\pgfpathlineto{\pgfqpoint{1.164606in}{1.320928in}}%
\pgfpathlineto{\pgfqpoint{1.166181in}{1.322027in}}%
\pgfpathlineto{\pgfqpoint{1.167282in}{1.322562in}}%
\pgfpathlineto{\pgfqpoint{1.169231in}{1.323671in}}%
\pgfpathlineto{\pgfqpoint{1.170327in}{1.324283in}}%
\pgfpathlineto{\pgfqpoint{1.170341in}{1.324283in}}%
\pgfpathlineto{\pgfqpoint{1.172286in}{1.325392in}}%
\pgfpathlineto{\pgfqpoint{1.173396in}{1.325917in}}%
\pgfpathlineto{\pgfqpoint{1.175018in}{1.327026in}}%
\pgfpathlineto{\pgfqpoint{1.176118in}{1.327590in}}%
\pgfpathlineto{\pgfqpoint{1.178217in}{1.328699in}}%
\pgfpathlineto{\pgfqpoint{1.179327in}{1.329448in}}%
\pgfpathlineto{\pgfqpoint{1.181079in}{1.330557in}}%
\pgfpathlineto{\pgfqpoint{1.182170in}{1.331218in}}%
\pgfpathlineto{\pgfqpoint{1.183974in}{1.332327in}}%
\pgfpathlineto{\pgfqpoint{1.185083in}{1.333076in}}%
\pgfpathlineto{\pgfqpoint{1.187375in}{1.334185in}}%
\pgfpathlineto{\pgfqpoint{1.188485in}{1.334710in}}%
\pgfpathlineto{\pgfqpoint{1.190612in}{1.335819in}}%
\pgfpathlineto{\pgfqpoint{1.191719in}{1.336626in}}%
\pgfpathlineto{\pgfqpoint{1.193418in}{1.337735in}}%
\pgfpathlineto{\pgfqpoint{1.194523in}{1.338445in}}%
\pgfpathlineto{\pgfqpoint{1.196114in}{1.339553in}}%
\pgfpathlineto{\pgfqpoint{1.197213in}{1.340137in}}%
\pgfpathlineto{\pgfqpoint{1.199067in}{1.341246in}}%
\pgfpathlineto{\pgfqpoint{1.200151in}{1.341907in}}%
\pgfpathlineto{\pgfqpoint{1.202131in}{1.343016in}}%
\pgfpathlineto{\pgfqpoint{1.203237in}{1.343619in}}%
\pgfpathlineto{\pgfqpoint{1.205144in}{1.344728in}}%
\pgfpathlineto{\pgfqpoint{1.206238in}{1.345263in}}%
\pgfpathlineto{\pgfqpoint{1.208041in}{1.346372in}}%
\pgfpathlineto{\pgfqpoint{1.209149in}{1.347033in}}%
\pgfpathlineto{\pgfqpoint{1.211357in}{1.348142in}}%
\pgfpathlineto{\pgfqpoint{1.212697in}{1.349017in}}%
\pgfpathlineto{\pgfqpoint{1.214654in}{1.350126in}}%
\pgfpathlineto{\pgfqpoint{1.215757in}{1.350777in}}%
\pgfpathlineto{\pgfqpoint{1.217576in}{1.351886in}}%
\pgfpathlineto{\pgfqpoint{1.218679in}{1.352411in}}%
\pgfpathlineto{\pgfqpoint{1.221029in}{1.353520in}}%
\pgfpathlineto{\pgfqpoint{1.222123in}{1.354182in}}%
\pgfpathlineto{\pgfqpoint{1.224149in}{1.355290in}}%
\pgfpathlineto{\pgfqpoint{1.225254in}{1.355874in}}%
\pgfpathlineto{\pgfqpoint{1.227532in}{1.356983in}}%
\pgfpathlineto{\pgfqpoint{1.228628in}{1.357508in}}%
\pgfpathlineto{\pgfqpoint{1.230457in}{1.358617in}}%
\pgfpathlineto{\pgfqpoint{1.231546in}{1.359181in}}%
\pgfpathlineto{\pgfqpoint{1.233645in}{1.360290in}}%
\pgfpathlineto{\pgfqpoint{1.234754in}{1.360844in}}%
\pgfpathlineto{\pgfqpoint{1.237070in}{1.361953in}}%
\pgfpathlineto{\pgfqpoint{1.238165in}{1.362400in}}%
\pgfpathlineto{\pgfqpoint{1.240408in}{1.363509in}}%
\pgfpathlineto{\pgfqpoint{1.241509in}{1.364200in}}%
\pgfpathlineto{\pgfqpoint{1.241516in}{1.364200in}}%
\pgfpathlineto{\pgfqpoint{1.243468in}{1.365308in}}%
\pgfpathlineto{\pgfqpoint{1.244569in}{1.365863in}}%
\pgfpathlineto{\pgfqpoint{1.246867in}{1.366972in}}%
\pgfpathlineto{\pgfqpoint{1.247977in}{1.367623in}}%
\pgfpathlineto{\pgfqpoint{1.250232in}{1.368732in}}%
\pgfpathlineto{\pgfqpoint{1.251330in}{1.369335in}}%
\pgfpathlineto{\pgfqpoint{1.253291in}{1.370444in}}%
\pgfpathlineto{\pgfqpoint{1.254352in}{1.371115in}}%
\pgfpathlineto{\pgfqpoint{1.256854in}{1.372224in}}%
\pgfpathlineto{\pgfqpoint{1.257950in}{1.372778in}}%
\pgfpathlineto{\pgfqpoint{1.260241in}{1.373887in}}%
\pgfpathlineto{\pgfqpoint{1.261340in}{1.374509in}}%
\pgfpathlineto{\pgfqpoint{1.263629in}{1.375618in}}%
\pgfpathlineto{\pgfqpoint{1.264695in}{1.376250in}}%
\pgfpathlineto{\pgfqpoint{1.264702in}{1.376250in}}%
\pgfpathlineto{\pgfqpoint{1.267268in}{1.377359in}}%
\pgfpathlineto{\pgfqpoint{1.268378in}{1.377904in}}%
\pgfpathlineto{\pgfqpoint{1.270714in}{1.379013in}}%
\pgfpathlineto{\pgfqpoint{1.271817in}{1.379635in}}%
\pgfpathlineto{\pgfqpoint{1.273795in}{1.380744in}}%
\pgfpathlineto{\pgfqpoint{1.274902in}{1.381337in}}%
\pgfpathlineto{\pgfqpoint{1.277005in}{1.382446in}}%
\pgfpathlineto{\pgfqpoint{1.278108in}{1.383020in}}%
\pgfpathlineto{\pgfqpoint{1.280845in}{1.384129in}}%
\pgfpathlineto{\pgfqpoint{1.281920in}{1.384771in}}%
\pgfpathlineto{\pgfqpoint{1.284651in}{1.385879in}}%
\pgfpathlineto{\pgfqpoint{1.285717in}{1.386366in}}%
\pgfpathlineto{\pgfqpoint{1.287957in}{1.387474in}}%
\pgfpathlineto{\pgfqpoint{1.289000in}{1.387863in}}%
\pgfpathlineto{\pgfqpoint{1.291555in}{1.388972in}}%
\pgfpathlineto{\pgfqpoint{1.292583in}{1.389459in}}%
\pgfpathlineto{\pgfqpoint{1.292648in}{1.389459in}}%
\pgfpathlineto{\pgfqpoint{1.294726in}{1.390567in}}%
\pgfpathlineto{\pgfqpoint{1.295824in}{1.391093in}}%
\pgfpathlineto{\pgfqpoint{1.298339in}{1.392201in}}%
\pgfpathlineto{\pgfqpoint{1.299407in}{1.392678in}}%
\pgfpathlineto{\pgfqpoint{1.301857in}{1.393787in}}%
\pgfpathlineto{\pgfqpoint{1.302960in}{1.394244in}}%
\pgfpathlineto{\pgfqpoint{1.305543in}{1.395353in}}%
\pgfpathlineto{\pgfqpoint{1.306613in}{1.395751in}}%
\pgfpathlineto{\pgfqpoint{1.308993in}{1.396860in}}%
\pgfpathlineto{\pgfqpoint{1.310003in}{1.397288in}}%
\pgfpathlineto{\pgfqpoint{1.312681in}{1.398397in}}%
\pgfpathlineto{\pgfqpoint{1.313768in}{1.398942in}}%
\pgfpathlineto{\pgfqpoint{1.316076in}{1.400050in}}%
\pgfpathlineto{\pgfqpoint{1.317183in}{1.400614in}}%
\pgfpathlineto{\pgfqpoint{1.319869in}{1.401723in}}%
\pgfpathlineto{\pgfqpoint{1.320906in}{1.402093in}}%
\pgfpathlineto{\pgfqpoint{1.323919in}{1.403202in}}%
\pgfpathlineto{\pgfqpoint{1.325013in}{1.403824in}}%
\pgfpathlineto{\pgfqpoint{1.327928in}{1.404933in}}%
\pgfpathlineto{\pgfqpoint{1.328985in}{1.405390in}}%
\pgfpathlineto{\pgfqpoint{1.331991in}{1.406499in}}%
\pgfpathlineto{\pgfqpoint{1.333096in}{1.406898in}}%
\pgfpathlineto{\pgfqpoint{1.335446in}{1.407997in}}%
\pgfpathlineto{\pgfqpoint{1.336556in}{1.408493in}}%
\pgfpathlineto{\pgfqpoint{1.339369in}{1.409601in}}%
\pgfpathlineto{\pgfqpoint{1.340472in}{1.410097in}}%
\pgfpathlineto{\pgfqpoint{1.343241in}{1.411206in}}%
\pgfpathlineto{\pgfqpoint{1.344281in}{1.411508in}}%
\pgfpathlineto{\pgfqpoint{1.347117in}{1.412617in}}%
\pgfpathlineto{\pgfqpoint{1.348224in}{1.413045in}}%
\pgfpathlineto{\pgfqpoint{1.351282in}{1.414153in}}%
\pgfpathlineto{\pgfqpoint{1.352364in}{1.414640in}}%
\pgfpathlineto{\pgfqpoint{1.354423in}{1.415748in}}%
\pgfpathlineto{\pgfqpoint{1.355512in}{1.416108in}}%
\pgfpathlineto{\pgfqpoint{1.358315in}{1.417217in}}%
\pgfpathlineto{\pgfqpoint{1.359369in}{1.417664in}}%
\pgfpathlineto{\pgfqpoint{1.359414in}{1.417664in}}%
\pgfpathlineto{\pgfqpoint{1.362266in}{1.418764in}}%
\pgfpathlineto{\pgfqpoint{1.363364in}{1.419230in}}%
\pgfpathlineto{\pgfqpoint{1.365919in}{1.420339in}}%
\pgfpathlineto{\pgfqpoint{1.367020in}{1.420884in}}%
\pgfpathlineto{\pgfqpoint{1.369968in}{1.421993in}}%
\pgfpathlineto{\pgfqpoint{1.371073in}{1.422352in}}%
\pgfpathlineto{\pgfqpoint{1.373486in}{1.423461in}}%
\pgfpathlineto{\pgfqpoint{1.374556in}{1.423860in}}%
\pgfpathlineto{\pgfqpoint{1.377213in}{1.424969in}}%
\pgfpathlineto{\pgfqpoint{1.378309in}{1.425523in}}%
\pgfpathlineto{\pgfqpoint{1.381189in}{1.426632in}}%
\pgfpathlineto{\pgfqpoint{1.382288in}{1.427177in}}%
\pgfpathlineto{\pgfqpoint{1.384924in}{1.428285in}}%
\pgfpathlineto{\pgfqpoint{1.386034in}{1.428743in}}%
\pgfpathlineto{\pgfqpoint{1.388605in}{1.429851in}}%
\pgfpathlineto{\pgfqpoint{1.389708in}{1.430425in}}%
\pgfpathlineto{\pgfqpoint{1.392735in}{1.431534in}}%
\pgfpathlineto{\pgfqpoint{1.393819in}{1.432001in}}%
\pgfpathlineto{\pgfqpoint{1.396325in}{1.433110in}}%
\pgfpathlineto{\pgfqpoint{1.397281in}{1.433528in}}%
\pgfpathlineto{\pgfqpoint{1.400222in}{1.434637in}}%
\pgfpathlineto{\pgfqpoint{1.401274in}{1.435055in}}%
\pgfpathlineto{\pgfqpoint{1.404573in}{1.436164in}}%
\pgfpathlineto{\pgfqpoint{1.405676in}{1.436660in}}%
\pgfpathlineto{\pgfqpoint{1.408664in}{1.437759in}}%
\pgfpathlineto{\pgfqpoint{1.409743in}{1.438313in}}%
\pgfpathlineto{\pgfqpoint{1.412961in}{1.439422in}}%
\pgfpathlineto{\pgfqpoint{1.414015in}{1.439899in}}%
\pgfpathlineto{\pgfqpoint{1.414071in}{1.439899in}}%
\pgfpathlineto{\pgfqpoint{1.416609in}{1.441007in}}%
\pgfpathlineto{\pgfqpoint{1.417719in}{1.441484in}}%
\pgfpathlineto{\pgfqpoint{1.420981in}{1.442593in}}%
\pgfpathlineto{\pgfqpoint{1.422038in}{1.443001in}}%
\pgfpathlineto{\pgfqpoint{1.425493in}{1.444110in}}%
\pgfpathlineto{\pgfqpoint{1.426600in}{1.444596in}}%
\pgfpathlineto{\pgfqpoint{1.429916in}{1.445705in}}%
\pgfpathlineto{\pgfqpoint{1.430991in}{1.446123in}}%
\pgfpathlineto{\pgfqpoint{1.433839in}{1.447232in}}%
\pgfpathlineto{\pgfqpoint{1.434949in}{1.447582in}}%
\pgfpathlineto{\pgfqpoint{1.438101in}{1.448691in}}%
\pgfpathlineto{\pgfqpoint{1.439195in}{1.449041in}}%
\pgfpathlineto{\pgfqpoint{1.442248in}{1.450150in}}%
\pgfpathlineto{\pgfqpoint{1.443322in}{1.450471in}}%
\pgfpathlineto{\pgfqpoint{1.446671in}{1.451580in}}%
\pgfpathlineto{\pgfqpoint{1.447778in}{1.451920in}}%
\pgfpathlineto{\pgfqpoint{1.450884in}{1.453029in}}%
\pgfpathlineto{\pgfqpoint{1.451990in}{1.453457in}}%
\pgfpathlineto{\pgfqpoint{1.455559in}{1.454566in}}%
\pgfpathlineto{\pgfqpoint{1.456634in}{1.455081in}}%
\pgfpathlineto{\pgfqpoint{1.460070in}{1.456190in}}%
\pgfpathlineto{\pgfqpoint{1.461071in}{1.456530in}}%
\pgfpathlineto{\pgfqpoint{1.461155in}{1.456530in}}%
\pgfpathlineto{\pgfqpoint{1.464991in}{1.457629in}}%
\pgfpathlineto{\pgfqpoint{1.466071in}{1.458009in}}%
\pgfpathlineto{\pgfqpoint{1.469929in}{1.459117in}}%
\pgfpathlineto{\pgfqpoint{1.471022in}{1.459400in}}%
\pgfpathlineto{\pgfqpoint{1.475029in}{1.460508in}}%
\pgfpathlineto{\pgfqpoint{1.476118in}{1.460897in}}%
\pgfpathlineto{\pgfqpoint{1.479243in}{1.462006in}}%
\pgfpathlineto{\pgfqpoint{1.480336in}{1.462434in}}%
\pgfpathlineto{\pgfqpoint{1.483919in}{1.463543in}}%
\pgfpathlineto{\pgfqpoint{1.485024in}{1.463864in}}%
\pgfpathlineto{\pgfqpoint{1.488347in}{1.464973in}}%
\pgfpathlineto{\pgfqpoint{1.489578in}{1.465371in}}%
\pgfpathlineto{\pgfqpoint{1.492968in}{1.466480in}}%
\pgfpathlineto{\pgfqpoint{1.494066in}{1.466782in}}%
\pgfpathlineto{\pgfqpoint{1.497698in}{1.467890in}}%
\pgfpathlineto{\pgfqpoint{1.498801in}{1.468241in}}%
\pgfpathlineto{\pgfqpoint{1.502065in}{1.469349in}}%
\pgfpathlineto{\pgfqpoint{1.503068in}{1.469680in}}%
\pgfpathlineto{\pgfqpoint{1.507121in}{1.470789in}}%
\pgfpathlineto{\pgfqpoint{1.508227in}{1.471139in}}%
\pgfpathlineto{\pgfqpoint{1.512540in}{1.472248in}}%
\pgfpathlineto{\pgfqpoint{1.513590in}{1.472676in}}%
\pgfpathlineto{\pgfqpoint{1.517077in}{1.473785in}}%
\pgfpathlineto{\pgfqpoint{1.518078in}{1.474115in}}%
\pgfpathlineto{\pgfqpoint{1.521591in}{1.475214in}}%
\pgfpathlineto{\pgfqpoint{1.522687in}{1.475467in}}%
\pgfpathlineto{\pgfqpoint{1.527148in}{1.476576in}}%
\pgfpathlineto{\pgfqpoint{1.528241in}{1.476887in}}%
\pgfpathlineto{\pgfqpoint{1.528257in}{1.476887in}}%
\pgfpathlineto{\pgfqpoint{1.531678in}{1.477996in}}%
\pgfpathlineto{\pgfqpoint{1.532771in}{1.478259in}}%
\pgfpathlineto{\pgfqpoint{1.536599in}{1.479367in}}%
\pgfpathlineto{\pgfqpoint{1.537688in}{1.479698in}}%
\pgfpathlineto{\pgfqpoint{1.537704in}{1.479698in}}%
\pgfpathlineto{\pgfqpoint{1.542099in}{1.480807in}}%
\pgfpathlineto{\pgfqpoint{1.543186in}{1.481128in}}%
\pgfpathlineto{\pgfqpoint{1.546974in}{1.482237in}}%
\pgfpathlineto{\pgfqpoint{1.548058in}{1.482596in}}%
\pgfpathlineto{\pgfqpoint{1.552116in}{1.483705in}}%
\pgfpathlineto{\pgfqpoint{1.553184in}{1.484016in}}%
\pgfpathlineto{\pgfqpoint{1.557542in}{1.485125in}}%
\pgfpathlineto{\pgfqpoint{1.558617in}{1.485495in}}%
\pgfpathlineto{\pgfqpoint{1.562879in}{1.486604in}}%
\pgfpathlineto{\pgfqpoint{1.563954in}{1.486934in}}%
\pgfpathlineto{\pgfqpoint{1.563963in}{1.486934in}}%
\pgfpathlineto{\pgfqpoint{1.568191in}{1.488043in}}%
\pgfpathlineto{\pgfqpoint{1.569261in}{1.488364in}}%
\pgfpathlineto{\pgfqpoint{1.569299in}{1.488364in}}%
\pgfpathlineto{\pgfqpoint{1.573922in}{1.489473in}}%
\pgfpathlineto{\pgfqpoint{1.574901in}{1.489726in}}%
\pgfpathlineto{\pgfqpoint{1.575011in}{1.489726in}}%
\pgfpathlineto{\pgfqpoint{1.580083in}{1.490835in}}%
\pgfpathlineto{\pgfqpoint{1.581172in}{1.491117in}}%
\pgfpathlineto{\pgfqpoint{1.585595in}{1.492225in}}%
\pgfpathlineto{\pgfqpoint{1.586705in}{1.492517in}}%
\pgfpathlineto{\pgfqpoint{1.591982in}{1.493626in}}%
\pgfpathlineto{\pgfqpoint{1.593068in}{1.493889in}}%
\pgfpathlineto{\pgfqpoint{1.597343in}{1.494997in}}%
\pgfpathlineto{\pgfqpoint{1.598427in}{1.495221in}}%
\pgfpathlineto{\pgfqpoint{1.602475in}{1.496330in}}%
\pgfpathlineto{\pgfqpoint{1.603576in}{1.496651in}}%
\pgfpathlineto{\pgfqpoint{1.608346in}{1.497750in}}%
\pgfpathlineto{\pgfqpoint{1.609449in}{1.498051in}}%
\pgfpathlineto{\pgfqpoint{1.614081in}{1.499160in}}%
\pgfpathlineto{\pgfqpoint{1.615191in}{1.499374in}}%
\pgfpathlineto{\pgfqpoint{1.619589in}{1.500483in}}%
\pgfpathlineto{\pgfqpoint{1.620670in}{1.500755in}}%
\pgfpathlineto{\pgfqpoint{1.625103in}{1.501864in}}%
\pgfpathlineto{\pgfqpoint{1.626189in}{1.502117in}}%
\pgfpathlineto{\pgfqpoint{1.631010in}{1.503216in}}%
\pgfpathlineto{\pgfqpoint{1.632037in}{1.503459in}}%
\pgfpathlineto{\pgfqpoint{1.632106in}{1.503459in}}%
\pgfpathlineto{\pgfqpoint{1.636909in}{1.504568in}}%
\pgfpathlineto{\pgfqpoint{1.637993in}{1.504918in}}%
\pgfpathlineto{\pgfqpoint{1.642432in}{1.506027in}}%
\pgfpathlineto{\pgfqpoint{1.643535in}{1.506289in}}%
\pgfpathlineto{\pgfqpoint{1.648338in}{1.507398in}}%
\pgfpathlineto{\pgfqpoint{1.649424in}{1.507544in}}%
\pgfpathlineto{\pgfqpoint{1.654499in}{1.508653in}}%
\pgfpathlineto{\pgfqpoint{1.655590in}{1.508925in}}%
\pgfpathlineto{\pgfqpoint{1.660681in}{1.510034in}}%
\pgfpathlineto{\pgfqpoint{1.661716in}{1.510277in}}%
\pgfpathlineto{\pgfqpoint{1.661754in}{1.510277in}}%
\pgfpathlineto{\pgfqpoint{1.666374in}{1.511386in}}%
\pgfpathlineto{\pgfqpoint{1.667356in}{1.511590in}}%
\pgfpathlineto{\pgfqpoint{1.673052in}{1.512699in}}%
\pgfpathlineto{\pgfqpoint{1.673834in}{1.512874in}}%
\pgfpathlineto{\pgfqpoint{1.674122in}{1.512874in}}%
\pgfpathlineto{\pgfqpoint{1.679023in}{1.513983in}}%
\pgfpathlineto{\pgfqpoint{1.680121in}{1.514236in}}%
\pgfpathlineto{\pgfqpoint{1.685221in}{1.515345in}}%
\pgfpathlineto{\pgfqpoint{1.686280in}{1.515559in}}%
\pgfpathlineto{\pgfqpoint{1.691566in}{1.516667in}}%
\pgfpathlineto{\pgfqpoint{1.692566in}{1.516833in}}%
\pgfpathlineto{\pgfqpoint{1.698102in}{1.517941in}}%
\pgfpathlineto{\pgfqpoint{1.699193in}{1.518087in}}%
\pgfpathlineto{\pgfqpoint{1.704347in}{1.519186in}}%
\pgfpathlineto{\pgfqpoint{1.705373in}{1.519391in}}%
\pgfpathlineto{\pgfqpoint{1.712064in}{1.520499in}}%
\pgfpathlineto{\pgfqpoint{1.713093in}{1.520713in}}%
\pgfpathlineto{\pgfqpoint{1.719731in}{1.521822in}}%
\pgfpathlineto{\pgfqpoint{1.720804in}{1.522046in}}%
\pgfpathlineto{\pgfqpoint{1.727128in}{1.523155in}}%
\pgfpathlineto{\pgfqpoint{1.728079in}{1.523330in}}%
\pgfpathlineto{\pgfqpoint{1.728200in}{1.523330in}}%
\pgfpathlineto{\pgfqpoint{1.735860in}{1.524439in}}%
\pgfpathlineto{\pgfqpoint{1.736874in}{1.524614in}}%
\pgfpathlineto{\pgfqpoint{1.736919in}{1.524614in}}%
\pgfpathlineto{\pgfqpoint{1.743340in}{1.525722in}}%
\pgfpathlineto{\pgfqpoint{1.744415in}{1.525966in}}%
\pgfpathlineto{\pgfqpoint{1.750162in}{1.527074in}}%
\pgfpathlineto{\pgfqpoint{1.751212in}{1.527269in}}%
\pgfpathlineto{\pgfqpoint{1.759330in}{1.528378in}}%
\pgfpathlineto{\pgfqpoint{1.760300in}{1.528543in}}%
\pgfpathlineto{\pgfqpoint{1.760393in}{1.528543in}}%
\pgfpathlineto{\pgfqpoint{1.767715in}{1.529652in}}%
\pgfpathlineto{\pgfqpoint{1.768711in}{1.529905in}}%
\pgfpathlineto{\pgfqpoint{1.776091in}{1.531004in}}%
\pgfpathlineto{\pgfqpoint{1.777155in}{1.531188in}}%
\pgfpathlineto{\pgfqpoint{1.777192in}{1.531188in}}%
\pgfpathlineto{\pgfqpoint{1.785352in}{1.532297in}}%
\pgfpathlineto{\pgfqpoint{1.786457in}{1.532531in}}%
\pgfpathlineto{\pgfqpoint{1.792949in}{1.533639in}}%
\pgfpathlineto{\pgfqpoint{1.793998in}{1.533863in}}%
\pgfpathlineto{\pgfqpoint{1.802712in}{1.534972in}}%
\pgfpathlineto{\pgfqpoint{1.803663in}{1.535108in}}%
\pgfpathlineto{\pgfqpoint{1.803728in}{1.535108in}}%
\pgfpathlineto{\pgfqpoint{1.811614in}{1.536217in}}%
\pgfpathlineto{\pgfqpoint{1.812435in}{1.536334in}}%
\pgfpathlineto{\pgfqpoint{1.812668in}{1.536334in}}%
\pgfpathlineto{\pgfqpoint{1.820497in}{1.537442in}}%
\pgfpathlineto{\pgfqpoint{1.821600in}{1.537598in}}%
\pgfpathlineto{\pgfqpoint{1.830495in}{1.538707in}}%
\pgfpathlineto{\pgfqpoint{1.831563in}{1.538911in}}%
\pgfpathlineto{\pgfqpoint{1.840160in}{1.540020in}}%
\pgfpathlineto{\pgfqpoint{1.841231in}{1.540175in}}%
\pgfpathlineto{\pgfqpoint{1.841254in}{1.540175in}}%
\pgfpathlineto{\pgfqpoint{1.851040in}{1.541284in}}%
\pgfpathlineto{\pgfqpoint{1.852087in}{1.541372in}}%
\pgfpathlineto{\pgfqpoint{1.852101in}{1.541372in}}%
\pgfpathlineto{\pgfqpoint{1.861938in}{1.542481in}}%
\pgfpathlineto{\pgfqpoint{1.863013in}{1.542636in}}%
\pgfpathlineto{\pgfqpoint{1.873516in}{1.543745in}}%
\pgfpathlineto{\pgfqpoint{1.874312in}{1.543852in}}%
\pgfpathlineto{\pgfqpoint{1.874607in}{1.543852in}}%
\pgfpathlineto{\pgfqpoint{1.885224in}{1.544961in}}%
\pgfpathlineto{\pgfqpoint{1.886325in}{1.545116in}}%
\pgfpathlineto{\pgfqpoint{1.899150in}{1.546225in}}%
\pgfpathlineto{\pgfqpoint{1.900225in}{1.546361in}}%
\pgfpathlineto{\pgfqpoint{1.912191in}{1.547460in}}%
\pgfpathlineto{\pgfqpoint{1.913217in}{1.547616in}}%
\pgfpathlineto{\pgfqpoint{1.928104in}{1.548725in}}%
\pgfpathlineto{\pgfqpoint{1.929213in}{1.548842in}}%
\pgfpathlineto{\pgfqpoint{1.944886in}{1.549950in}}%
\pgfpathlineto{\pgfqpoint{1.945936in}{1.550057in}}%
\pgfpathlineto{\pgfqpoint{1.963612in}{1.551166in}}%
\pgfpathlineto{\pgfqpoint{1.964508in}{1.551234in}}%
\pgfpathlineto{\pgfqpoint{1.964703in}{1.551234in}}%
\pgfpathlineto{\pgfqpoint{1.983126in}{1.551944in}}%
\pgfpathlineto{\pgfqpoint{1.983126in}{1.551944in}}%
\pgfusepath{stroke}%
\end{pgfscope}%
\begin{pgfscope}%
\pgfsetrectcap%
\pgfsetmiterjoin%
\pgfsetlinewidth{0.803000pt}%
\definecolor{currentstroke}{rgb}{0.000000,0.000000,0.000000}%
\pgfsetstrokecolor{currentstroke}%
\pgfsetdash{}{0pt}%
\pgfpathmoveto{\pgfqpoint{0.503581in}{0.449444in}}%
\pgfpathlineto{\pgfqpoint{0.503581in}{1.604444in}}%
\pgfusepath{stroke}%
\end{pgfscope}%
\begin{pgfscope}%
\pgfsetrectcap%
\pgfsetmiterjoin%
\pgfsetlinewidth{0.803000pt}%
\definecolor{currentstroke}{rgb}{0.000000,0.000000,0.000000}%
\pgfsetstrokecolor{currentstroke}%
\pgfsetdash{}{0pt}%
\pgfpathmoveto{\pgfqpoint{2.053581in}{0.449444in}}%
\pgfpathlineto{\pgfqpoint{2.053581in}{1.604444in}}%
\pgfusepath{stroke}%
\end{pgfscope}%
\begin{pgfscope}%
\pgfsetrectcap%
\pgfsetmiterjoin%
\pgfsetlinewidth{0.803000pt}%
\definecolor{currentstroke}{rgb}{0.000000,0.000000,0.000000}%
\pgfsetstrokecolor{currentstroke}%
\pgfsetdash{}{0pt}%
\pgfpathmoveto{\pgfqpoint{0.503581in}{0.449444in}}%
\pgfpathlineto{\pgfqpoint{2.053581in}{0.449444in}}%
\pgfusepath{stroke}%
\end{pgfscope}%
\begin{pgfscope}%
\pgfsetrectcap%
\pgfsetmiterjoin%
\pgfsetlinewidth{0.803000pt}%
\definecolor{currentstroke}{rgb}{0.000000,0.000000,0.000000}%
\pgfsetstrokecolor{currentstroke}%
\pgfsetdash{}{0pt}%
\pgfpathmoveto{\pgfqpoint{0.503581in}{1.604444in}}%
\pgfpathlineto{\pgfqpoint{2.053581in}{1.604444in}}%
\pgfusepath{stroke}%
\end{pgfscope}%
\begin{pgfscope}%
\pgfsetbuttcap%
\pgfsetmiterjoin%
\definecolor{currentfill}{rgb}{1.000000,1.000000,1.000000}%
\pgfsetfillcolor{currentfill}%
\pgfsetfillopacity{0.800000}%
\pgfsetlinewidth{1.003750pt}%
\definecolor{currentstroke}{rgb}{0.800000,0.800000,0.800000}%
\pgfsetstrokecolor{currentstroke}%
\pgfsetstrokeopacity{0.800000}%
\pgfsetdash{}{0pt}%
\pgfpathmoveto{\pgfqpoint{0.782747in}{0.518889in}}%
\pgfpathlineto{\pgfqpoint{1.956358in}{0.518889in}}%
\pgfpathquadraticcurveto{\pgfqpoint{1.984136in}{0.518889in}}{\pgfqpoint{1.984136in}{0.546666in}}%
\pgfpathlineto{\pgfqpoint{1.984136in}{0.726388in}}%
\pgfpathquadraticcurveto{\pgfqpoint{1.984136in}{0.754166in}}{\pgfqpoint{1.956358in}{0.754166in}}%
\pgfpathlineto{\pgfqpoint{0.782747in}{0.754166in}}%
\pgfpathquadraticcurveto{\pgfqpoint{0.754970in}{0.754166in}}{\pgfqpoint{0.754970in}{0.726388in}}%
\pgfpathlineto{\pgfqpoint{0.754970in}{0.546666in}}%
\pgfpathquadraticcurveto{\pgfqpoint{0.754970in}{0.518889in}}{\pgfqpoint{0.782747in}{0.518889in}}%
\pgfpathlineto{\pgfqpoint{0.782747in}{0.518889in}}%
\pgfpathclose%
\pgfusepath{stroke,fill}%
\end{pgfscope}%
\begin{pgfscope}%
\pgfsetrectcap%
\pgfsetroundjoin%
\pgfsetlinewidth{1.505625pt}%
\definecolor{currentstroke}{rgb}{0.000000,0.000000,0.000000}%
\pgfsetstrokecolor{currentstroke}%
\pgfsetdash{}{0pt}%
\pgfpathmoveto{\pgfqpoint{0.810525in}{0.650000in}}%
\pgfpathlineto{\pgfqpoint{0.949414in}{0.650000in}}%
\pgfpathlineto{\pgfqpoint{1.088303in}{0.650000in}}%
\pgfusepath{stroke}%
\end{pgfscope}%
\begin{pgfscope}%
\definecolor{textcolor}{rgb}{0.000000,0.000000,0.000000}%
\pgfsetstrokecolor{textcolor}%
\pgfsetfillcolor{textcolor}%
\pgftext[x=1.199414in,y=0.601388in,left,base]{\color{textcolor}\rmfamily\fontsize{10.000000}{12.000000}\selectfont AUC=0.753}%
\end{pgfscope}%
\end{pgfpicture}%
\makeatother%
\endgroup%

\end{tabular}


\

In this work we used two methods to give the results of different models similar distributions.  This case illustrates directly transforming the \verb|y_proba| values.  

To make a useful visualization of the results where we can see the interplay between the negative and positive classes, we can transform the data.  A transformation that preserves rank will have no effect on the ROC curve.  [Cite]  For the graph below, we mapped the smallest value in the set to 0 and the largest to 1.  

\

\verb|AdaBoost_5_Fold_Hard_Test_Transformed_100|

%
\noindent\begin{tabular}{@{\hspace{-6pt}}p{4.3in} @{\hspace{-6pt}}p{2.0in}}
	\vskip 0pt
	\hfil Raw Model Output
	
	%% Creator: Matplotlib, PGF backend
%%
%% To include the figure in your LaTeX document, write
%%   \input{<filename>.pgf}
%%
%% Make sure the required packages are loaded in your preamble
%%   \usepackage{pgf}
%%
%% Also ensure that all the required font packages are loaded; for instance,
%% the lmodern package is sometimes necessary when using math font.
%%   \usepackage{lmodern}
%%
%% Figures using additional raster images can only be included by \input if
%% they are in the same directory as the main LaTeX file. For loading figures
%% from other directories you can use the `import` package
%%   \usepackage{import}
%%
%% and then include the figures with
%%   \import{<path to file>}{<filename>.pgf}
%%
%% Matplotlib used the following preamble
%%   
%%   \usepackage{fontspec}
%%   \makeatletter\@ifpackageloaded{underscore}{}{\usepackage[strings]{underscore}}\makeatother
%%
\begingroup%
\makeatletter%
\begin{pgfpicture}%
\pgfpathrectangle{\pgfpointorigin}{\pgfqpoint{4.102500in}{1.794370in}}%
\pgfusepath{use as bounding box, clip}%
\begin{pgfscope}%
\pgfsetbuttcap%
\pgfsetmiterjoin%
\definecolor{currentfill}{rgb}{1.000000,1.000000,1.000000}%
\pgfsetfillcolor{currentfill}%
\pgfsetlinewidth{0.000000pt}%
\definecolor{currentstroke}{rgb}{1.000000,1.000000,1.000000}%
\pgfsetstrokecolor{currentstroke}%
\pgfsetdash{}{0pt}%
\pgfpathmoveto{\pgfqpoint{0.000000in}{0.000000in}}%
\pgfpathlineto{\pgfqpoint{4.102500in}{0.000000in}}%
\pgfpathlineto{\pgfqpoint{4.102500in}{1.794370in}}%
\pgfpathlineto{\pgfqpoint{0.000000in}{1.794370in}}%
\pgfpathlineto{\pgfqpoint{0.000000in}{0.000000in}}%
\pgfpathclose%
\pgfusepath{fill}%
\end{pgfscope}%
\begin{pgfscope}%
\pgfsetbuttcap%
\pgfsetmiterjoin%
\definecolor{currentfill}{rgb}{1.000000,1.000000,1.000000}%
\pgfsetfillcolor{currentfill}%
\pgfsetlinewidth{0.000000pt}%
\definecolor{currentstroke}{rgb}{0.000000,0.000000,0.000000}%
\pgfsetstrokecolor{currentstroke}%
\pgfsetstrokeopacity{0.000000}%
\pgfsetdash{}{0pt}%
\pgfpathmoveto{\pgfqpoint{0.515000in}{0.499444in}}%
\pgfpathlineto{\pgfqpoint{4.002500in}{0.499444in}}%
\pgfpathlineto{\pgfqpoint{4.002500in}{1.654444in}}%
\pgfpathlineto{\pgfqpoint{0.515000in}{1.654444in}}%
\pgfpathlineto{\pgfqpoint{0.515000in}{0.499444in}}%
\pgfpathclose%
\pgfusepath{fill}%
\end{pgfscope}%
\begin{pgfscope}%
\pgfpathrectangle{\pgfqpoint{0.515000in}{0.499444in}}{\pgfqpoint{3.487500in}{1.155000in}}%
\pgfusepath{clip}%
\pgfsetbuttcap%
\pgfsetmiterjoin%
\pgfsetlinewidth{1.003750pt}%
\definecolor{currentstroke}{rgb}{0.000000,0.000000,0.000000}%
\pgfsetstrokecolor{currentstroke}%
\pgfsetdash{}{0pt}%
\pgfpathmoveto{\pgfqpoint{0.610114in}{0.499444in}}%
\pgfpathlineto{\pgfqpoint{0.673523in}{0.499444in}}%
\pgfpathlineto{\pgfqpoint{0.673523in}{0.499455in}}%
\pgfpathlineto{\pgfqpoint{0.610114in}{0.499455in}}%
\pgfpathlineto{\pgfqpoint{0.610114in}{0.499444in}}%
\pgfpathclose%
\pgfusepath{stroke}%
\end{pgfscope}%
\begin{pgfscope}%
\pgfpathrectangle{\pgfqpoint{0.515000in}{0.499444in}}{\pgfqpoint{3.487500in}{1.155000in}}%
\pgfusepath{clip}%
\pgfsetbuttcap%
\pgfsetmiterjoin%
\pgfsetlinewidth{1.003750pt}%
\definecolor{currentstroke}{rgb}{0.000000,0.000000,0.000000}%
\pgfsetstrokecolor{currentstroke}%
\pgfsetdash{}{0pt}%
\pgfpathmoveto{\pgfqpoint{0.768637in}{0.499444in}}%
\pgfpathlineto{\pgfqpoint{0.832046in}{0.499444in}}%
\pgfpathlineto{\pgfqpoint{0.832046in}{0.499926in}}%
\pgfpathlineto{\pgfqpoint{0.768637in}{0.499926in}}%
\pgfpathlineto{\pgfqpoint{0.768637in}{0.499444in}}%
\pgfpathclose%
\pgfusepath{stroke}%
\end{pgfscope}%
\begin{pgfscope}%
\pgfpathrectangle{\pgfqpoint{0.515000in}{0.499444in}}{\pgfqpoint{3.487500in}{1.155000in}}%
\pgfusepath{clip}%
\pgfsetbuttcap%
\pgfsetmiterjoin%
\pgfsetlinewidth{1.003750pt}%
\definecolor{currentstroke}{rgb}{0.000000,0.000000,0.000000}%
\pgfsetstrokecolor{currentstroke}%
\pgfsetdash{}{0pt}%
\pgfpathmoveto{\pgfqpoint{0.927159in}{0.499444in}}%
\pgfpathlineto{\pgfqpoint{0.990568in}{0.499444in}}%
\pgfpathlineto{\pgfqpoint{0.990568in}{0.505787in}}%
\pgfpathlineto{\pgfqpoint{0.927159in}{0.505787in}}%
\pgfpathlineto{\pgfqpoint{0.927159in}{0.499444in}}%
\pgfpathclose%
\pgfusepath{stroke}%
\end{pgfscope}%
\begin{pgfscope}%
\pgfpathrectangle{\pgfqpoint{0.515000in}{0.499444in}}{\pgfqpoint{3.487500in}{1.155000in}}%
\pgfusepath{clip}%
\pgfsetbuttcap%
\pgfsetmiterjoin%
\pgfsetlinewidth{1.003750pt}%
\definecolor{currentstroke}{rgb}{0.000000,0.000000,0.000000}%
\pgfsetstrokecolor{currentstroke}%
\pgfsetdash{}{0pt}%
\pgfpathmoveto{\pgfqpoint{1.085682in}{0.499444in}}%
\pgfpathlineto{\pgfqpoint{1.149091in}{0.499444in}}%
\pgfpathlineto{\pgfqpoint{1.149091in}{0.537917in}}%
\pgfpathlineto{\pgfqpoint{1.085682in}{0.537917in}}%
\pgfpathlineto{\pgfqpoint{1.085682in}{0.499444in}}%
\pgfpathclose%
\pgfusepath{stroke}%
\end{pgfscope}%
\begin{pgfscope}%
\pgfpathrectangle{\pgfqpoint{0.515000in}{0.499444in}}{\pgfqpoint{3.487500in}{1.155000in}}%
\pgfusepath{clip}%
\pgfsetbuttcap%
\pgfsetmiterjoin%
\pgfsetlinewidth{1.003750pt}%
\definecolor{currentstroke}{rgb}{0.000000,0.000000,0.000000}%
\pgfsetstrokecolor{currentstroke}%
\pgfsetdash{}{0pt}%
\pgfpathmoveto{\pgfqpoint{1.244205in}{0.499444in}}%
\pgfpathlineto{\pgfqpoint{1.307614in}{0.499444in}}%
\pgfpathlineto{\pgfqpoint{1.307614in}{0.630836in}}%
\pgfpathlineto{\pgfqpoint{1.244205in}{0.630836in}}%
\pgfpathlineto{\pgfqpoint{1.244205in}{0.499444in}}%
\pgfpathclose%
\pgfusepath{stroke}%
\end{pgfscope}%
\begin{pgfscope}%
\pgfpathrectangle{\pgfqpoint{0.515000in}{0.499444in}}{\pgfqpoint{3.487500in}{1.155000in}}%
\pgfusepath{clip}%
\pgfsetbuttcap%
\pgfsetmiterjoin%
\pgfsetlinewidth{1.003750pt}%
\definecolor{currentstroke}{rgb}{0.000000,0.000000,0.000000}%
\pgfsetstrokecolor{currentstroke}%
\pgfsetdash{}{0pt}%
\pgfpathmoveto{\pgfqpoint{1.402728in}{0.499444in}}%
\pgfpathlineto{\pgfqpoint{1.466137in}{0.499444in}}%
\pgfpathlineto{\pgfqpoint{1.466137in}{0.818357in}}%
\pgfpathlineto{\pgfqpoint{1.402728in}{0.818357in}}%
\pgfpathlineto{\pgfqpoint{1.402728in}{0.499444in}}%
\pgfpathclose%
\pgfusepath{stroke}%
\end{pgfscope}%
\begin{pgfscope}%
\pgfpathrectangle{\pgfqpoint{0.515000in}{0.499444in}}{\pgfqpoint{3.487500in}{1.155000in}}%
\pgfusepath{clip}%
\pgfsetbuttcap%
\pgfsetmiterjoin%
\pgfsetlinewidth{1.003750pt}%
\definecolor{currentstroke}{rgb}{0.000000,0.000000,0.000000}%
\pgfsetstrokecolor{currentstroke}%
\pgfsetdash{}{0pt}%
\pgfpathmoveto{\pgfqpoint{1.561250in}{0.499444in}}%
\pgfpathlineto{\pgfqpoint{1.624659in}{0.499444in}}%
\pgfpathlineto{\pgfqpoint{1.624659in}{1.083605in}}%
\pgfpathlineto{\pgfqpoint{1.561250in}{1.083605in}}%
\pgfpathlineto{\pgfqpoint{1.561250in}{0.499444in}}%
\pgfpathclose%
\pgfusepath{stroke}%
\end{pgfscope}%
\begin{pgfscope}%
\pgfpathrectangle{\pgfqpoint{0.515000in}{0.499444in}}{\pgfqpoint{3.487500in}{1.155000in}}%
\pgfusepath{clip}%
\pgfsetbuttcap%
\pgfsetmiterjoin%
\pgfsetlinewidth{1.003750pt}%
\definecolor{currentstroke}{rgb}{0.000000,0.000000,0.000000}%
\pgfsetstrokecolor{currentstroke}%
\pgfsetdash{}{0pt}%
\pgfpathmoveto{\pgfqpoint{1.719773in}{0.499444in}}%
\pgfpathlineto{\pgfqpoint{1.783182in}{0.499444in}}%
\pgfpathlineto{\pgfqpoint{1.783182in}{1.374747in}}%
\pgfpathlineto{\pgfqpoint{1.719773in}{1.374747in}}%
\pgfpathlineto{\pgfqpoint{1.719773in}{0.499444in}}%
\pgfpathclose%
\pgfusepath{stroke}%
\end{pgfscope}%
\begin{pgfscope}%
\pgfpathrectangle{\pgfqpoint{0.515000in}{0.499444in}}{\pgfqpoint{3.487500in}{1.155000in}}%
\pgfusepath{clip}%
\pgfsetbuttcap%
\pgfsetmiterjoin%
\pgfsetlinewidth{1.003750pt}%
\definecolor{currentstroke}{rgb}{0.000000,0.000000,0.000000}%
\pgfsetstrokecolor{currentstroke}%
\pgfsetdash{}{0pt}%
\pgfpathmoveto{\pgfqpoint{1.878296in}{0.499444in}}%
\pgfpathlineto{\pgfqpoint{1.941705in}{0.499444in}}%
\pgfpathlineto{\pgfqpoint{1.941705in}{1.583427in}}%
\pgfpathlineto{\pgfqpoint{1.878296in}{1.583427in}}%
\pgfpathlineto{\pgfqpoint{1.878296in}{0.499444in}}%
\pgfpathclose%
\pgfusepath{stroke}%
\end{pgfscope}%
\begin{pgfscope}%
\pgfpathrectangle{\pgfqpoint{0.515000in}{0.499444in}}{\pgfqpoint{3.487500in}{1.155000in}}%
\pgfusepath{clip}%
\pgfsetbuttcap%
\pgfsetmiterjoin%
\pgfsetlinewidth{1.003750pt}%
\definecolor{currentstroke}{rgb}{0.000000,0.000000,0.000000}%
\pgfsetstrokecolor{currentstroke}%
\pgfsetdash{}{0pt}%
\pgfpathmoveto{\pgfqpoint{2.036818in}{0.499444in}}%
\pgfpathlineto{\pgfqpoint{2.100228in}{0.499444in}}%
\pgfpathlineto{\pgfqpoint{2.100228in}{1.599444in}}%
\pgfpathlineto{\pgfqpoint{2.036818in}{1.599444in}}%
\pgfpathlineto{\pgfqpoint{2.036818in}{0.499444in}}%
\pgfpathclose%
\pgfusepath{stroke}%
\end{pgfscope}%
\begin{pgfscope}%
\pgfpathrectangle{\pgfqpoint{0.515000in}{0.499444in}}{\pgfqpoint{3.487500in}{1.155000in}}%
\pgfusepath{clip}%
\pgfsetbuttcap%
\pgfsetmiterjoin%
\pgfsetlinewidth{1.003750pt}%
\definecolor{currentstroke}{rgb}{0.000000,0.000000,0.000000}%
\pgfsetstrokecolor{currentstroke}%
\pgfsetdash{}{0pt}%
\pgfpathmoveto{\pgfqpoint{2.195341in}{0.499444in}}%
\pgfpathlineto{\pgfqpoint{2.258750in}{0.499444in}}%
\pgfpathlineto{\pgfqpoint{2.258750in}{1.433597in}}%
\pgfpathlineto{\pgfqpoint{2.195341in}{1.433597in}}%
\pgfpathlineto{\pgfqpoint{2.195341in}{0.499444in}}%
\pgfpathclose%
\pgfusepath{stroke}%
\end{pgfscope}%
\begin{pgfscope}%
\pgfpathrectangle{\pgfqpoint{0.515000in}{0.499444in}}{\pgfqpoint{3.487500in}{1.155000in}}%
\pgfusepath{clip}%
\pgfsetbuttcap%
\pgfsetmiterjoin%
\pgfsetlinewidth{1.003750pt}%
\definecolor{currentstroke}{rgb}{0.000000,0.000000,0.000000}%
\pgfsetstrokecolor{currentstroke}%
\pgfsetdash{}{0pt}%
\pgfpathmoveto{\pgfqpoint{2.353864in}{0.499444in}}%
\pgfpathlineto{\pgfqpoint{2.417273in}{0.499444in}}%
\pgfpathlineto{\pgfqpoint{2.417273in}{1.153779in}}%
\pgfpathlineto{\pgfqpoint{2.353864in}{1.153779in}}%
\pgfpathlineto{\pgfqpoint{2.353864in}{0.499444in}}%
\pgfpathclose%
\pgfusepath{stroke}%
\end{pgfscope}%
\begin{pgfscope}%
\pgfpathrectangle{\pgfqpoint{0.515000in}{0.499444in}}{\pgfqpoint{3.487500in}{1.155000in}}%
\pgfusepath{clip}%
\pgfsetbuttcap%
\pgfsetmiterjoin%
\pgfsetlinewidth{1.003750pt}%
\definecolor{currentstroke}{rgb}{0.000000,0.000000,0.000000}%
\pgfsetstrokecolor{currentstroke}%
\pgfsetdash{}{0pt}%
\pgfpathmoveto{\pgfqpoint{2.512387in}{0.499444in}}%
\pgfpathlineto{\pgfqpoint{2.575796in}{0.499444in}}%
\pgfpathlineto{\pgfqpoint{2.575796in}{0.889345in}}%
\pgfpathlineto{\pgfqpoint{2.512387in}{0.889345in}}%
\pgfpathlineto{\pgfqpoint{2.512387in}{0.499444in}}%
\pgfpathclose%
\pgfusepath{stroke}%
\end{pgfscope}%
\begin{pgfscope}%
\pgfpathrectangle{\pgfqpoint{0.515000in}{0.499444in}}{\pgfqpoint{3.487500in}{1.155000in}}%
\pgfusepath{clip}%
\pgfsetbuttcap%
\pgfsetmiterjoin%
\pgfsetlinewidth{1.003750pt}%
\definecolor{currentstroke}{rgb}{0.000000,0.000000,0.000000}%
\pgfsetstrokecolor{currentstroke}%
\pgfsetdash{}{0pt}%
\pgfpathmoveto{\pgfqpoint{2.670909in}{0.499444in}}%
\pgfpathlineto{\pgfqpoint{2.734318in}{0.499444in}}%
\pgfpathlineto{\pgfqpoint{2.734318in}{0.704042in}}%
\pgfpathlineto{\pgfqpoint{2.670909in}{0.704042in}}%
\pgfpathlineto{\pgfqpoint{2.670909in}{0.499444in}}%
\pgfpathclose%
\pgfusepath{stroke}%
\end{pgfscope}%
\begin{pgfscope}%
\pgfpathrectangle{\pgfqpoint{0.515000in}{0.499444in}}{\pgfqpoint{3.487500in}{1.155000in}}%
\pgfusepath{clip}%
\pgfsetbuttcap%
\pgfsetmiterjoin%
\pgfsetlinewidth{1.003750pt}%
\definecolor{currentstroke}{rgb}{0.000000,0.000000,0.000000}%
\pgfsetstrokecolor{currentstroke}%
\pgfsetdash{}{0pt}%
\pgfpathmoveto{\pgfqpoint{2.829432in}{0.499444in}}%
\pgfpathlineto{\pgfqpoint{2.892841in}{0.499444in}}%
\pgfpathlineto{\pgfqpoint{2.892841in}{0.597227in}}%
\pgfpathlineto{\pgfqpoint{2.829432in}{0.597227in}}%
\pgfpathlineto{\pgfqpoint{2.829432in}{0.499444in}}%
\pgfpathclose%
\pgfusepath{stroke}%
\end{pgfscope}%
\begin{pgfscope}%
\pgfpathrectangle{\pgfqpoint{0.515000in}{0.499444in}}{\pgfqpoint{3.487500in}{1.155000in}}%
\pgfusepath{clip}%
\pgfsetbuttcap%
\pgfsetmiterjoin%
\pgfsetlinewidth{1.003750pt}%
\definecolor{currentstroke}{rgb}{0.000000,0.000000,0.000000}%
\pgfsetstrokecolor{currentstroke}%
\pgfsetdash{}{0pt}%
\pgfpathmoveto{\pgfqpoint{2.987955in}{0.499444in}}%
\pgfpathlineto{\pgfqpoint{3.051364in}{0.499444in}}%
\pgfpathlineto{\pgfqpoint{3.051364in}{0.542192in}}%
\pgfpathlineto{\pgfqpoint{2.987955in}{0.542192in}}%
\pgfpathlineto{\pgfqpoint{2.987955in}{0.499444in}}%
\pgfpathclose%
\pgfusepath{stroke}%
\end{pgfscope}%
\begin{pgfscope}%
\pgfpathrectangle{\pgfqpoint{0.515000in}{0.499444in}}{\pgfqpoint{3.487500in}{1.155000in}}%
\pgfusepath{clip}%
\pgfsetbuttcap%
\pgfsetmiterjoin%
\pgfsetlinewidth{1.003750pt}%
\definecolor{currentstroke}{rgb}{0.000000,0.000000,0.000000}%
\pgfsetstrokecolor{currentstroke}%
\pgfsetdash{}{0pt}%
\pgfpathmoveto{\pgfqpoint{3.146478in}{0.499444in}}%
\pgfpathlineto{\pgfqpoint{3.209887in}{0.499444in}}%
\pgfpathlineto{\pgfqpoint{3.209887in}{0.516447in}}%
\pgfpathlineto{\pgfqpoint{3.146478in}{0.516447in}}%
\pgfpathlineto{\pgfqpoint{3.146478in}{0.499444in}}%
\pgfpathclose%
\pgfusepath{stroke}%
\end{pgfscope}%
\begin{pgfscope}%
\pgfpathrectangle{\pgfqpoint{0.515000in}{0.499444in}}{\pgfqpoint{3.487500in}{1.155000in}}%
\pgfusepath{clip}%
\pgfsetbuttcap%
\pgfsetmiterjoin%
\pgfsetlinewidth{1.003750pt}%
\definecolor{currentstroke}{rgb}{0.000000,0.000000,0.000000}%
\pgfsetstrokecolor{currentstroke}%
\pgfsetdash{}{0pt}%
\pgfpathmoveto{\pgfqpoint{3.305000in}{0.499444in}}%
\pgfpathlineto{\pgfqpoint{3.368409in}{0.499444in}}%
\pgfpathlineto{\pgfqpoint{3.368409in}{0.505894in}}%
\pgfpathlineto{\pgfqpoint{3.305000in}{0.505894in}}%
\pgfpathlineto{\pgfqpoint{3.305000in}{0.499444in}}%
\pgfpathclose%
\pgfusepath{stroke}%
\end{pgfscope}%
\begin{pgfscope}%
\pgfpathrectangle{\pgfqpoint{0.515000in}{0.499444in}}{\pgfqpoint{3.487500in}{1.155000in}}%
\pgfusepath{clip}%
\pgfsetbuttcap%
\pgfsetmiterjoin%
\pgfsetlinewidth{1.003750pt}%
\definecolor{currentstroke}{rgb}{0.000000,0.000000,0.000000}%
\pgfsetstrokecolor{currentstroke}%
\pgfsetdash{}{0pt}%
\pgfpathmoveto{\pgfqpoint{3.463523in}{0.499444in}}%
\pgfpathlineto{\pgfqpoint{3.526932in}{0.499444in}}%
\pgfpathlineto{\pgfqpoint{3.526932in}{0.501469in}}%
\pgfpathlineto{\pgfqpoint{3.463523in}{0.501469in}}%
\pgfpathlineto{\pgfqpoint{3.463523in}{0.499444in}}%
\pgfpathclose%
\pgfusepath{stroke}%
\end{pgfscope}%
\begin{pgfscope}%
\pgfpathrectangle{\pgfqpoint{0.515000in}{0.499444in}}{\pgfqpoint{3.487500in}{1.155000in}}%
\pgfusepath{clip}%
\pgfsetbuttcap%
\pgfsetmiterjoin%
\pgfsetlinewidth{1.003750pt}%
\definecolor{currentstroke}{rgb}{0.000000,0.000000,0.000000}%
\pgfsetstrokecolor{currentstroke}%
\pgfsetdash{}{0pt}%
\pgfpathmoveto{\pgfqpoint{3.622046in}{0.499444in}}%
\pgfpathlineto{\pgfqpoint{3.685455in}{0.499444in}}%
\pgfpathlineto{\pgfqpoint{3.685455in}{0.499637in}}%
\pgfpathlineto{\pgfqpoint{3.622046in}{0.499637in}}%
\pgfpathlineto{\pgfqpoint{3.622046in}{0.499444in}}%
\pgfpathclose%
\pgfusepath{stroke}%
\end{pgfscope}%
\begin{pgfscope}%
\pgfpathrectangle{\pgfqpoint{0.515000in}{0.499444in}}{\pgfqpoint{3.487500in}{1.155000in}}%
\pgfusepath{clip}%
\pgfsetbuttcap%
\pgfsetmiterjoin%
\pgfsetlinewidth{1.003750pt}%
\definecolor{currentstroke}{rgb}{0.000000,0.000000,0.000000}%
\pgfsetstrokecolor{currentstroke}%
\pgfsetdash{}{0pt}%
\pgfpathmoveto{\pgfqpoint{3.780568in}{0.499444in}}%
\pgfpathlineto{\pgfqpoint{3.843978in}{0.499444in}}%
\pgfpathlineto{\pgfqpoint{3.843978in}{0.499476in}}%
\pgfpathlineto{\pgfqpoint{3.780568in}{0.499476in}}%
\pgfpathlineto{\pgfqpoint{3.780568in}{0.499444in}}%
\pgfpathclose%
\pgfusepath{stroke}%
\end{pgfscope}%
\begin{pgfscope}%
\pgfpathrectangle{\pgfqpoint{0.515000in}{0.499444in}}{\pgfqpoint{3.487500in}{1.155000in}}%
\pgfusepath{clip}%
\pgfsetbuttcap%
\pgfsetmiterjoin%
\definecolor{currentfill}{rgb}{0.000000,0.000000,0.000000}%
\pgfsetfillcolor{currentfill}%
\pgfsetlinewidth{0.000000pt}%
\definecolor{currentstroke}{rgb}{0.000000,0.000000,0.000000}%
\pgfsetstrokecolor{currentstroke}%
\pgfsetstrokeopacity{0.000000}%
\pgfsetdash{}{0pt}%
\pgfpathmoveto{\pgfqpoint{0.673523in}{0.499444in}}%
\pgfpathlineto{\pgfqpoint{0.736932in}{0.499444in}}%
\pgfpathlineto{\pgfqpoint{0.736932in}{0.499444in}}%
\pgfpathlineto{\pgfqpoint{0.673523in}{0.499444in}}%
\pgfpathlineto{\pgfqpoint{0.673523in}{0.499444in}}%
\pgfpathclose%
\pgfusepath{fill}%
\end{pgfscope}%
\begin{pgfscope}%
\pgfpathrectangle{\pgfqpoint{0.515000in}{0.499444in}}{\pgfqpoint{3.487500in}{1.155000in}}%
\pgfusepath{clip}%
\pgfsetbuttcap%
\pgfsetmiterjoin%
\definecolor{currentfill}{rgb}{0.000000,0.000000,0.000000}%
\pgfsetfillcolor{currentfill}%
\pgfsetlinewidth{0.000000pt}%
\definecolor{currentstroke}{rgb}{0.000000,0.000000,0.000000}%
\pgfsetstrokecolor{currentstroke}%
\pgfsetstrokeopacity{0.000000}%
\pgfsetdash{}{0pt}%
\pgfpathmoveto{\pgfqpoint{0.832046in}{0.499444in}}%
\pgfpathlineto{\pgfqpoint{0.895455in}{0.499444in}}%
\pgfpathlineto{\pgfqpoint{0.895455in}{0.499455in}}%
\pgfpathlineto{\pgfqpoint{0.832046in}{0.499455in}}%
\pgfpathlineto{\pgfqpoint{0.832046in}{0.499444in}}%
\pgfpathclose%
\pgfusepath{fill}%
\end{pgfscope}%
\begin{pgfscope}%
\pgfpathrectangle{\pgfqpoint{0.515000in}{0.499444in}}{\pgfqpoint{3.487500in}{1.155000in}}%
\pgfusepath{clip}%
\pgfsetbuttcap%
\pgfsetmiterjoin%
\definecolor{currentfill}{rgb}{0.000000,0.000000,0.000000}%
\pgfsetfillcolor{currentfill}%
\pgfsetlinewidth{0.000000pt}%
\definecolor{currentstroke}{rgb}{0.000000,0.000000,0.000000}%
\pgfsetstrokecolor{currentstroke}%
\pgfsetstrokeopacity{0.000000}%
\pgfsetdash{}{0pt}%
\pgfpathmoveto{\pgfqpoint{0.990568in}{0.499444in}}%
\pgfpathlineto{\pgfqpoint{1.053978in}{0.499444in}}%
\pgfpathlineto{\pgfqpoint{1.053978in}{0.499487in}}%
\pgfpathlineto{\pgfqpoint{0.990568in}{0.499487in}}%
\pgfpathlineto{\pgfqpoint{0.990568in}{0.499444in}}%
\pgfpathclose%
\pgfusepath{fill}%
\end{pgfscope}%
\begin{pgfscope}%
\pgfpathrectangle{\pgfqpoint{0.515000in}{0.499444in}}{\pgfqpoint{3.487500in}{1.155000in}}%
\pgfusepath{clip}%
\pgfsetbuttcap%
\pgfsetmiterjoin%
\definecolor{currentfill}{rgb}{0.000000,0.000000,0.000000}%
\pgfsetfillcolor{currentfill}%
\pgfsetlinewidth{0.000000pt}%
\definecolor{currentstroke}{rgb}{0.000000,0.000000,0.000000}%
\pgfsetstrokecolor{currentstroke}%
\pgfsetstrokeopacity{0.000000}%
\pgfsetdash{}{0pt}%
\pgfpathmoveto{\pgfqpoint{1.149091in}{0.499444in}}%
\pgfpathlineto{\pgfqpoint{1.212500in}{0.499444in}}%
\pgfpathlineto{\pgfqpoint{1.212500in}{0.499766in}}%
\pgfpathlineto{\pgfqpoint{1.149091in}{0.499766in}}%
\pgfpathlineto{\pgfqpoint{1.149091in}{0.499444in}}%
\pgfpathclose%
\pgfusepath{fill}%
\end{pgfscope}%
\begin{pgfscope}%
\pgfpathrectangle{\pgfqpoint{0.515000in}{0.499444in}}{\pgfqpoint{3.487500in}{1.155000in}}%
\pgfusepath{clip}%
\pgfsetbuttcap%
\pgfsetmiterjoin%
\definecolor{currentfill}{rgb}{0.000000,0.000000,0.000000}%
\pgfsetfillcolor{currentfill}%
\pgfsetlinewidth{0.000000pt}%
\definecolor{currentstroke}{rgb}{0.000000,0.000000,0.000000}%
\pgfsetstrokecolor{currentstroke}%
\pgfsetstrokeopacity{0.000000}%
\pgfsetdash{}{0pt}%
\pgfpathmoveto{\pgfqpoint{1.307614in}{0.499444in}}%
\pgfpathlineto{\pgfqpoint{1.371023in}{0.499444in}}%
\pgfpathlineto{\pgfqpoint{1.371023in}{0.501276in}}%
\pgfpathlineto{\pgfqpoint{1.307614in}{0.501276in}}%
\pgfpathlineto{\pgfqpoint{1.307614in}{0.499444in}}%
\pgfpathclose%
\pgfusepath{fill}%
\end{pgfscope}%
\begin{pgfscope}%
\pgfpathrectangle{\pgfqpoint{0.515000in}{0.499444in}}{\pgfqpoint{3.487500in}{1.155000in}}%
\pgfusepath{clip}%
\pgfsetbuttcap%
\pgfsetmiterjoin%
\definecolor{currentfill}{rgb}{0.000000,0.000000,0.000000}%
\pgfsetfillcolor{currentfill}%
\pgfsetlinewidth{0.000000pt}%
\definecolor{currentstroke}{rgb}{0.000000,0.000000,0.000000}%
\pgfsetstrokecolor{currentstroke}%
\pgfsetstrokeopacity{0.000000}%
\pgfsetdash{}{0pt}%
\pgfpathmoveto{\pgfqpoint{1.466137in}{0.499444in}}%
\pgfpathlineto{\pgfqpoint{1.529546in}{0.499444in}}%
\pgfpathlineto{\pgfqpoint{1.529546in}{0.506054in}}%
\pgfpathlineto{\pgfqpoint{1.466137in}{0.506054in}}%
\pgfpathlineto{\pgfqpoint{1.466137in}{0.499444in}}%
\pgfpathclose%
\pgfusepath{fill}%
\end{pgfscope}%
\begin{pgfscope}%
\pgfpathrectangle{\pgfqpoint{0.515000in}{0.499444in}}{\pgfqpoint{3.487500in}{1.155000in}}%
\pgfusepath{clip}%
\pgfsetbuttcap%
\pgfsetmiterjoin%
\definecolor{currentfill}{rgb}{0.000000,0.000000,0.000000}%
\pgfsetfillcolor{currentfill}%
\pgfsetlinewidth{0.000000pt}%
\definecolor{currentstroke}{rgb}{0.000000,0.000000,0.000000}%
\pgfsetstrokecolor{currentstroke}%
\pgfsetstrokeopacity{0.000000}%
\pgfsetdash{}{0pt}%
\pgfpathmoveto{\pgfqpoint{1.624659in}{0.499444in}}%
\pgfpathlineto{\pgfqpoint{1.688068in}{0.499444in}}%
\pgfpathlineto{\pgfqpoint{1.688068in}{0.518332in}}%
\pgfpathlineto{\pgfqpoint{1.624659in}{0.518332in}}%
\pgfpathlineto{\pgfqpoint{1.624659in}{0.499444in}}%
\pgfpathclose%
\pgfusepath{fill}%
\end{pgfscope}%
\begin{pgfscope}%
\pgfpathrectangle{\pgfqpoint{0.515000in}{0.499444in}}{\pgfqpoint{3.487500in}{1.155000in}}%
\pgfusepath{clip}%
\pgfsetbuttcap%
\pgfsetmiterjoin%
\definecolor{currentfill}{rgb}{0.000000,0.000000,0.000000}%
\pgfsetfillcolor{currentfill}%
\pgfsetlinewidth{0.000000pt}%
\definecolor{currentstroke}{rgb}{0.000000,0.000000,0.000000}%
\pgfsetstrokecolor{currentstroke}%
\pgfsetstrokeopacity{0.000000}%
\pgfsetdash{}{0pt}%
\pgfpathmoveto{\pgfqpoint{1.783182in}{0.499444in}}%
\pgfpathlineto{\pgfqpoint{1.846591in}{0.499444in}}%
\pgfpathlineto{\pgfqpoint{1.846591in}{0.544956in}}%
\pgfpathlineto{\pgfqpoint{1.783182in}{0.544956in}}%
\pgfpathlineto{\pgfqpoint{1.783182in}{0.499444in}}%
\pgfpathclose%
\pgfusepath{fill}%
\end{pgfscope}%
\begin{pgfscope}%
\pgfpathrectangle{\pgfqpoint{0.515000in}{0.499444in}}{\pgfqpoint{3.487500in}{1.155000in}}%
\pgfusepath{clip}%
\pgfsetbuttcap%
\pgfsetmiterjoin%
\definecolor{currentfill}{rgb}{0.000000,0.000000,0.000000}%
\pgfsetfillcolor{currentfill}%
\pgfsetlinewidth{0.000000pt}%
\definecolor{currentstroke}{rgb}{0.000000,0.000000,0.000000}%
\pgfsetstrokecolor{currentstroke}%
\pgfsetstrokeopacity{0.000000}%
\pgfsetdash{}{0pt}%
\pgfpathmoveto{\pgfqpoint{1.941705in}{0.499444in}}%
\pgfpathlineto{\pgfqpoint{2.005114in}{0.499444in}}%
\pgfpathlineto{\pgfqpoint{2.005114in}{0.590414in}}%
\pgfpathlineto{\pgfqpoint{1.941705in}{0.590414in}}%
\pgfpathlineto{\pgfqpoint{1.941705in}{0.499444in}}%
\pgfpathclose%
\pgfusepath{fill}%
\end{pgfscope}%
\begin{pgfscope}%
\pgfpathrectangle{\pgfqpoint{0.515000in}{0.499444in}}{\pgfqpoint{3.487500in}{1.155000in}}%
\pgfusepath{clip}%
\pgfsetbuttcap%
\pgfsetmiterjoin%
\definecolor{currentfill}{rgb}{0.000000,0.000000,0.000000}%
\pgfsetfillcolor{currentfill}%
\pgfsetlinewidth{0.000000pt}%
\definecolor{currentstroke}{rgb}{0.000000,0.000000,0.000000}%
\pgfsetstrokecolor{currentstroke}%
\pgfsetstrokeopacity{0.000000}%
\pgfsetdash{}{0pt}%
\pgfpathmoveto{\pgfqpoint{2.100228in}{0.499444in}}%
\pgfpathlineto{\pgfqpoint{2.163637in}{0.499444in}}%
\pgfpathlineto{\pgfqpoint{2.163637in}{0.645589in}}%
\pgfpathlineto{\pgfqpoint{2.100228in}{0.645589in}}%
\pgfpathlineto{\pgfqpoint{2.100228in}{0.499444in}}%
\pgfpathclose%
\pgfusepath{fill}%
\end{pgfscope}%
\begin{pgfscope}%
\pgfpathrectangle{\pgfqpoint{0.515000in}{0.499444in}}{\pgfqpoint{3.487500in}{1.155000in}}%
\pgfusepath{clip}%
\pgfsetbuttcap%
\pgfsetmiterjoin%
\definecolor{currentfill}{rgb}{0.000000,0.000000,0.000000}%
\pgfsetfillcolor{currentfill}%
\pgfsetlinewidth{0.000000pt}%
\definecolor{currentstroke}{rgb}{0.000000,0.000000,0.000000}%
\pgfsetstrokecolor{currentstroke}%
\pgfsetstrokeopacity{0.000000}%
\pgfsetdash{}{0pt}%
\pgfpathmoveto{\pgfqpoint{2.258750in}{0.499444in}}%
\pgfpathlineto{\pgfqpoint{2.322159in}{0.499444in}}%
\pgfpathlineto{\pgfqpoint{2.322159in}{0.686525in}}%
\pgfpathlineto{\pgfqpoint{2.258750in}{0.686525in}}%
\pgfpathlineto{\pgfqpoint{2.258750in}{0.499444in}}%
\pgfpathclose%
\pgfusepath{fill}%
\end{pgfscope}%
\begin{pgfscope}%
\pgfpathrectangle{\pgfqpoint{0.515000in}{0.499444in}}{\pgfqpoint{3.487500in}{1.155000in}}%
\pgfusepath{clip}%
\pgfsetbuttcap%
\pgfsetmiterjoin%
\definecolor{currentfill}{rgb}{0.000000,0.000000,0.000000}%
\pgfsetfillcolor{currentfill}%
\pgfsetlinewidth{0.000000pt}%
\definecolor{currentstroke}{rgb}{0.000000,0.000000,0.000000}%
\pgfsetstrokecolor{currentstroke}%
\pgfsetstrokeopacity{0.000000}%
\pgfsetdash{}{0pt}%
\pgfpathmoveto{\pgfqpoint{2.417273in}{0.499444in}}%
\pgfpathlineto{\pgfqpoint{2.480682in}{0.499444in}}%
\pgfpathlineto{\pgfqpoint{2.480682in}{0.693414in}}%
\pgfpathlineto{\pgfqpoint{2.417273in}{0.693414in}}%
\pgfpathlineto{\pgfqpoint{2.417273in}{0.499444in}}%
\pgfpathclose%
\pgfusepath{fill}%
\end{pgfscope}%
\begin{pgfscope}%
\pgfpathrectangle{\pgfqpoint{0.515000in}{0.499444in}}{\pgfqpoint{3.487500in}{1.155000in}}%
\pgfusepath{clip}%
\pgfsetbuttcap%
\pgfsetmiterjoin%
\definecolor{currentfill}{rgb}{0.000000,0.000000,0.000000}%
\pgfsetfillcolor{currentfill}%
\pgfsetlinewidth{0.000000pt}%
\definecolor{currentstroke}{rgb}{0.000000,0.000000,0.000000}%
\pgfsetstrokecolor{currentstroke}%
\pgfsetstrokeopacity{0.000000}%
\pgfsetdash{}{0pt}%
\pgfpathmoveto{\pgfqpoint{2.575796in}{0.499444in}}%
\pgfpathlineto{\pgfqpoint{2.639205in}{0.499444in}}%
\pgfpathlineto{\pgfqpoint{2.639205in}{0.669534in}}%
\pgfpathlineto{\pgfqpoint{2.575796in}{0.669534in}}%
\pgfpathlineto{\pgfqpoint{2.575796in}{0.499444in}}%
\pgfpathclose%
\pgfusepath{fill}%
\end{pgfscope}%
\begin{pgfscope}%
\pgfpathrectangle{\pgfqpoint{0.515000in}{0.499444in}}{\pgfqpoint{3.487500in}{1.155000in}}%
\pgfusepath{clip}%
\pgfsetbuttcap%
\pgfsetmiterjoin%
\definecolor{currentfill}{rgb}{0.000000,0.000000,0.000000}%
\pgfsetfillcolor{currentfill}%
\pgfsetlinewidth{0.000000pt}%
\definecolor{currentstroke}{rgb}{0.000000,0.000000,0.000000}%
\pgfsetstrokecolor{currentstroke}%
\pgfsetstrokeopacity{0.000000}%
\pgfsetdash{}{0pt}%
\pgfpathmoveto{\pgfqpoint{2.734318in}{0.499444in}}%
\pgfpathlineto{\pgfqpoint{2.797728in}{0.499444in}}%
\pgfpathlineto{\pgfqpoint{2.797728in}{0.629347in}}%
\pgfpathlineto{\pgfqpoint{2.734318in}{0.629347in}}%
\pgfpathlineto{\pgfqpoint{2.734318in}{0.499444in}}%
\pgfpathclose%
\pgfusepath{fill}%
\end{pgfscope}%
\begin{pgfscope}%
\pgfpathrectangle{\pgfqpoint{0.515000in}{0.499444in}}{\pgfqpoint{3.487500in}{1.155000in}}%
\pgfusepath{clip}%
\pgfsetbuttcap%
\pgfsetmiterjoin%
\definecolor{currentfill}{rgb}{0.000000,0.000000,0.000000}%
\pgfsetfillcolor{currentfill}%
\pgfsetlinewidth{0.000000pt}%
\definecolor{currentstroke}{rgb}{0.000000,0.000000,0.000000}%
\pgfsetstrokecolor{currentstroke}%
\pgfsetstrokeopacity{0.000000}%
\pgfsetdash{}{0pt}%
\pgfpathmoveto{\pgfqpoint{2.892841in}{0.499444in}}%
\pgfpathlineto{\pgfqpoint{2.956250in}{0.499444in}}%
\pgfpathlineto{\pgfqpoint{2.956250in}{0.581275in}}%
\pgfpathlineto{\pgfqpoint{2.892841in}{0.581275in}}%
\pgfpathlineto{\pgfqpoint{2.892841in}{0.499444in}}%
\pgfpathclose%
\pgfusepath{fill}%
\end{pgfscope}%
\begin{pgfscope}%
\pgfpathrectangle{\pgfqpoint{0.515000in}{0.499444in}}{\pgfqpoint{3.487500in}{1.155000in}}%
\pgfusepath{clip}%
\pgfsetbuttcap%
\pgfsetmiterjoin%
\definecolor{currentfill}{rgb}{0.000000,0.000000,0.000000}%
\pgfsetfillcolor{currentfill}%
\pgfsetlinewidth{0.000000pt}%
\definecolor{currentstroke}{rgb}{0.000000,0.000000,0.000000}%
\pgfsetstrokecolor{currentstroke}%
\pgfsetstrokeopacity{0.000000}%
\pgfsetdash{}{0pt}%
\pgfpathmoveto{\pgfqpoint{3.051364in}{0.499444in}}%
\pgfpathlineto{\pgfqpoint{3.114773in}{0.499444in}}%
\pgfpathlineto{\pgfqpoint{3.114773in}{0.544291in}}%
\pgfpathlineto{\pgfqpoint{3.051364in}{0.544291in}}%
\pgfpathlineto{\pgfqpoint{3.051364in}{0.499444in}}%
\pgfpathclose%
\pgfusepath{fill}%
\end{pgfscope}%
\begin{pgfscope}%
\pgfpathrectangle{\pgfqpoint{0.515000in}{0.499444in}}{\pgfqpoint{3.487500in}{1.155000in}}%
\pgfusepath{clip}%
\pgfsetbuttcap%
\pgfsetmiterjoin%
\definecolor{currentfill}{rgb}{0.000000,0.000000,0.000000}%
\pgfsetfillcolor{currentfill}%
\pgfsetlinewidth{0.000000pt}%
\definecolor{currentstroke}{rgb}{0.000000,0.000000,0.000000}%
\pgfsetstrokecolor{currentstroke}%
\pgfsetstrokeopacity{0.000000}%
\pgfsetdash{}{0pt}%
\pgfpathmoveto{\pgfqpoint{3.209887in}{0.499444in}}%
\pgfpathlineto{\pgfqpoint{3.273296in}{0.499444in}}%
\pgfpathlineto{\pgfqpoint{3.273296in}{0.521364in}}%
\pgfpathlineto{\pgfqpoint{3.209887in}{0.521364in}}%
\pgfpathlineto{\pgfqpoint{3.209887in}{0.499444in}}%
\pgfpathclose%
\pgfusepath{fill}%
\end{pgfscope}%
\begin{pgfscope}%
\pgfpathrectangle{\pgfqpoint{0.515000in}{0.499444in}}{\pgfqpoint{3.487500in}{1.155000in}}%
\pgfusepath{clip}%
\pgfsetbuttcap%
\pgfsetmiterjoin%
\definecolor{currentfill}{rgb}{0.000000,0.000000,0.000000}%
\pgfsetfillcolor{currentfill}%
\pgfsetlinewidth{0.000000pt}%
\definecolor{currentstroke}{rgb}{0.000000,0.000000,0.000000}%
\pgfsetstrokecolor{currentstroke}%
\pgfsetstrokeopacity{0.000000}%
\pgfsetdash{}{0pt}%
\pgfpathmoveto{\pgfqpoint{3.368409in}{0.499444in}}%
\pgfpathlineto{\pgfqpoint{3.431818in}{0.499444in}}%
\pgfpathlineto{\pgfqpoint{3.431818in}{0.510918in}}%
\pgfpathlineto{\pgfqpoint{3.368409in}{0.510918in}}%
\pgfpathlineto{\pgfqpoint{3.368409in}{0.499444in}}%
\pgfpathclose%
\pgfusepath{fill}%
\end{pgfscope}%
\begin{pgfscope}%
\pgfpathrectangle{\pgfqpoint{0.515000in}{0.499444in}}{\pgfqpoint{3.487500in}{1.155000in}}%
\pgfusepath{clip}%
\pgfsetbuttcap%
\pgfsetmiterjoin%
\definecolor{currentfill}{rgb}{0.000000,0.000000,0.000000}%
\pgfsetfillcolor{currentfill}%
\pgfsetlinewidth{0.000000pt}%
\definecolor{currentstroke}{rgb}{0.000000,0.000000,0.000000}%
\pgfsetstrokecolor{currentstroke}%
\pgfsetstrokeopacity{0.000000}%
\pgfsetdash{}{0pt}%
\pgfpathmoveto{\pgfqpoint{3.526932in}{0.499444in}}%
\pgfpathlineto{\pgfqpoint{3.590341in}{0.499444in}}%
\pgfpathlineto{\pgfqpoint{3.590341in}{0.503783in}}%
\pgfpathlineto{\pgfqpoint{3.526932in}{0.503783in}}%
\pgfpathlineto{\pgfqpoint{3.526932in}{0.499444in}}%
\pgfpathclose%
\pgfusepath{fill}%
\end{pgfscope}%
\begin{pgfscope}%
\pgfpathrectangle{\pgfqpoint{0.515000in}{0.499444in}}{\pgfqpoint{3.487500in}{1.155000in}}%
\pgfusepath{clip}%
\pgfsetbuttcap%
\pgfsetmiterjoin%
\definecolor{currentfill}{rgb}{0.000000,0.000000,0.000000}%
\pgfsetfillcolor{currentfill}%
\pgfsetlinewidth{0.000000pt}%
\definecolor{currentstroke}{rgb}{0.000000,0.000000,0.000000}%
\pgfsetstrokecolor{currentstroke}%
\pgfsetstrokeopacity{0.000000}%
\pgfsetdash{}{0pt}%
\pgfpathmoveto{\pgfqpoint{3.685455in}{0.499444in}}%
\pgfpathlineto{\pgfqpoint{3.748864in}{0.499444in}}%
\pgfpathlineto{\pgfqpoint{3.748864in}{0.500162in}}%
\pgfpathlineto{\pgfqpoint{3.685455in}{0.500162in}}%
\pgfpathlineto{\pgfqpoint{3.685455in}{0.499444in}}%
\pgfpathclose%
\pgfusepath{fill}%
\end{pgfscope}%
\begin{pgfscope}%
\pgfpathrectangle{\pgfqpoint{0.515000in}{0.499444in}}{\pgfqpoint{3.487500in}{1.155000in}}%
\pgfusepath{clip}%
\pgfsetbuttcap%
\pgfsetmiterjoin%
\definecolor{currentfill}{rgb}{0.000000,0.000000,0.000000}%
\pgfsetfillcolor{currentfill}%
\pgfsetlinewidth{0.000000pt}%
\definecolor{currentstroke}{rgb}{0.000000,0.000000,0.000000}%
\pgfsetstrokecolor{currentstroke}%
\pgfsetstrokeopacity{0.000000}%
\pgfsetdash{}{0pt}%
\pgfpathmoveto{\pgfqpoint{3.843978in}{0.499444in}}%
\pgfpathlineto{\pgfqpoint{3.907387in}{0.499444in}}%
\pgfpathlineto{\pgfqpoint{3.907387in}{0.499541in}}%
\pgfpathlineto{\pgfqpoint{3.843978in}{0.499541in}}%
\pgfpathlineto{\pgfqpoint{3.843978in}{0.499444in}}%
\pgfpathclose%
\pgfusepath{fill}%
\end{pgfscope}%
\begin{pgfscope}%
\pgfsetbuttcap%
\pgfsetroundjoin%
\definecolor{currentfill}{rgb}{0.000000,0.000000,0.000000}%
\pgfsetfillcolor{currentfill}%
\pgfsetlinewidth{0.803000pt}%
\definecolor{currentstroke}{rgb}{0.000000,0.000000,0.000000}%
\pgfsetstrokecolor{currentstroke}%
\pgfsetdash{}{0pt}%
\pgfsys@defobject{currentmarker}{\pgfqpoint{0.000000in}{-0.048611in}}{\pgfqpoint{0.000000in}{0.000000in}}{%
\pgfpathmoveto{\pgfqpoint{0.000000in}{0.000000in}}%
\pgfpathlineto{\pgfqpoint{0.000000in}{-0.048611in}}%
\pgfusepath{stroke,fill}%
}%
\begin{pgfscope}%
\pgfsys@transformshift{0.515000in}{0.499444in}%
\pgfsys@useobject{currentmarker}{}%
\end{pgfscope}%
\end{pgfscope}%
\begin{pgfscope}%
\pgfsetbuttcap%
\pgfsetroundjoin%
\definecolor{currentfill}{rgb}{0.000000,0.000000,0.000000}%
\pgfsetfillcolor{currentfill}%
\pgfsetlinewidth{0.803000pt}%
\definecolor{currentstroke}{rgb}{0.000000,0.000000,0.000000}%
\pgfsetstrokecolor{currentstroke}%
\pgfsetdash{}{0pt}%
\pgfsys@defobject{currentmarker}{\pgfqpoint{0.000000in}{-0.048611in}}{\pgfqpoint{0.000000in}{0.000000in}}{%
\pgfpathmoveto{\pgfqpoint{0.000000in}{0.000000in}}%
\pgfpathlineto{\pgfqpoint{0.000000in}{-0.048611in}}%
\pgfusepath{stroke,fill}%
}%
\begin{pgfscope}%
\pgfsys@transformshift{0.673523in}{0.499444in}%
\pgfsys@useobject{currentmarker}{}%
\end{pgfscope}%
\end{pgfscope}%
\begin{pgfscope}%
\definecolor{textcolor}{rgb}{0.000000,0.000000,0.000000}%
\pgfsetstrokecolor{textcolor}%
\pgfsetfillcolor{textcolor}%
\pgftext[x=0.673523in,y=0.402222in,,top]{\color{textcolor}\rmfamily\fontsize{10.000000}{12.000000}\selectfont 0.0}%
\end{pgfscope}%
\begin{pgfscope}%
\pgfsetbuttcap%
\pgfsetroundjoin%
\definecolor{currentfill}{rgb}{0.000000,0.000000,0.000000}%
\pgfsetfillcolor{currentfill}%
\pgfsetlinewidth{0.803000pt}%
\definecolor{currentstroke}{rgb}{0.000000,0.000000,0.000000}%
\pgfsetstrokecolor{currentstroke}%
\pgfsetdash{}{0pt}%
\pgfsys@defobject{currentmarker}{\pgfqpoint{0.000000in}{-0.048611in}}{\pgfqpoint{0.000000in}{0.000000in}}{%
\pgfpathmoveto{\pgfqpoint{0.000000in}{0.000000in}}%
\pgfpathlineto{\pgfqpoint{0.000000in}{-0.048611in}}%
\pgfusepath{stroke,fill}%
}%
\begin{pgfscope}%
\pgfsys@transformshift{0.832046in}{0.499444in}%
\pgfsys@useobject{currentmarker}{}%
\end{pgfscope}%
\end{pgfscope}%
\begin{pgfscope}%
\pgfsetbuttcap%
\pgfsetroundjoin%
\definecolor{currentfill}{rgb}{0.000000,0.000000,0.000000}%
\pgfsetfillcolor{currentfill}%
\pgfsetlinewidth{0.803000pt}%
\definecolor{currentstroke}{rgb}{0.000000,0.000000,0.000000}%
\pgfsetstrokecolor{currentstroke}%
\pgfsetdash{}{0pt}%
\pgfsys@defobject{currentmarker}{\pgfqpoint{0.000000in}{-0.048611in}}{\pgfqpoint{0.000000in}{0.000000in}}{%
\pgfpathmoveto{\pgfqpoint{0.000000in}{0.000000in}}%
\pgfpathlineto{\pgfqpoint{0.000000in}{-0.048611in}}%
\pgfusepath{stroke,fill}%
}%
\begin{pgfscope}%
\pgfsys@transformshift{0.990568in}{0.499444in}%
\pgfsys@useobject{currentmarker}{}%
\end{pgfscope}%
\end{pgfscope}%
\begin{pgfscope}%
\definecolor{textcolor}{rgb}{0.000000,0.000000,0.000000}%
\pgfsetstrokecolor{textcolor}%
\pgfsetfillcolor{textcolor}%
\pgftext[x=0.990568in,y=0.402222in,,top]{\color{textcolor}\rmfamily\fontsize{10.000000}{12.000000}\selectfont 0.1}%
\end{pgfscope}%
\begin{pgfscope}%
\pgfsetbuttcap%
\pgfsetroundjoin%
\definecolor{currentfill}{rgb}{0.000000,0.000000,0.000000}%
\pgfsetfillcolor{currentfill}%
\pgfsetlinewidth{0.803000pt}%
\definecolor{currentstroke}{rgb}{0.000000,0.000000,0.000000}%
\pgfsetstrokecolor{currentstroke}%
\pgfsetdash{}{0pt}%
\pgfsys@defobject{currentmarker}{\pgfqpoint{0.000000in}{-0.048611in}}{\pgfqpoint{0.000000in}{0.000000in}}{%
\pgfpathmoveto{\pgfqpoint{0.000000in}{0.000000in}}%
\pgfpathlineto{\pgfqpoint{0.000000in}{-0.048611in}}%
\pgfusepath{stroke,fill}%
}%
\begin{pgfscope}%
\pgfsys@transformshift{1.149091in}{0.499444in}%
\pgfsys@useobject{currentmarker}{}%
\end{pgfscope}%
\end{pgfscope}%
\begin{pgfscope}%
\pgfsetbuttcap%
\pgfsetroundjoin%
\definecolor{currentfill}{rgb}{0.000000,0.000000,0.000000}%
\pgfsetfillcolor{currentfill}%
\pgfsetlinewidth{0.803000pt}%
\definecolor{currentstroke}{rgb}{0.000000,0.000000,0.000000}%
\pgfsetstrokecolor{currentstroke}%
\pgfsetdash{}{0pt}%
\pgfsys@defobject{currentmarker}{\pgfqpoint{0.000000in}{-0.048611in}}{\pgfqpoint{0.000000in}{0.000000in}}{%
\pgfpathmoveto{\pgfqpoint{0.000000in}{0.000000in}}%
\pgfpathlineto{\pgfqpoint{0.000000in}{-0.048611in}}%
\pgfusepath{stroke,fill}%
}%
\begin{pgfscope}%
\pgfsys@transformshift{1.307614in}{0.499444in}%
\pgfsys@useobject{currentmarker}{}%
\end{pgfscope}%
\end{pgfscope}%
\begin{pgfscope}%
\definecolor{textcolor}{rgb}{0.000000,0.000000,0.000000}%
\pgfsetstrokecolor{textcolor}%
\pgfsetfillcolor{textcolor}%
\pgftext[x=1.307614in,y=0.402222in,,top]{\color{textcolor}\rmfamily\fontsize{10.000000}{12.000000}\selectfont 0.2}%
\end{pgfscope}%
\begin{pgfscope}%
\pgfsetbuttcap%
\pgfsetroundjoin%
\definecolor{currentfill}{rgb}{0.000000,0.000000,0.000000}%
\pgfsetfillcolor{currentfill}%
\pgfsetlinewidth{0.803000pt}%
\definecolor{currentstroke}{rgb}{0.000000,0.000000,0.000000}%
\pgfsetstrokecolor{currentstroke}%
\pgfsetdash{}{0pt}%
\pgfsys@defobject{currentmarker}{\pgfqpoint{0.000000in}{-0.048611in}}{\pgfqpoint{0.000000in}{0.000000in}}{%
\pgfpathmoveto{\pgfqpoint{0.000000in}{0.000000in}}%
\pgfpathlineto{\pgfqpoint{0.000000in}{-0.048611in}}%
\pgfusepath{stroke,fill}%
}%
\begin{pgfscope}%
\pgfsys@transformshift{1.466137in}{0.499444in}%
\pgfsys@useobject{currentmarker}{}%
\end{pgfscope}%
\end{pgfscope}%
\begin{pgfscope}%
\pgfsetbuttcap%
\pgfsetroundjoin%
\definecolor{currentfill}{rgb}{0.000000,0.000000,0.000000}%
\pgfsetfillcolor{currentfill}%
\pgfsetlinewidth{0.803000pt}%
\definecolor{currentstroke}{rgb}{0.000000,0.000000,0.000000}%
\pgfsetstrokecolor{currentstroke}%
\pgfsetdash{}{0pt}%
\pgfsys@defobject{currentmarker}{\pgfqpoint{0.000000in}{-0.048611in}}{\pgfqpoint{0.000000in}{0.000000in}}{%
\pgfpathmoveto{\pgfqpoint{0.000000in}{0.000000in}}%
\pgfpathlineto{\pgfqpoint{0.000000in}{-0.048611in}}%
\pgfusepath{stroke,fill}%
}%
\begin{pgfscope}%
\pgfsys@transformshift{1.624659in}{0.499444in}%
\pgfsys@useobject{currentmarker}{}%
\end{pgfscope}%
\end{pgfscope}%
\begin{pgfscope}%
\definecolor{textcolor}{rgb}{0.000000,0.000000,0.000000}%
\pgfsetstrokecolor{textcolor}%
\pgfsetfillcolor{textcolor}%
\pgftext[x=1.624659in,y=0.402222in,,top]{\color{textcolor}\rmfamily\fontsize{10.000000}{12.000000}\selectfont 0.3}%
\end{pgfscope}%
\begin{pgfscope}%
\pgfsetbuttcap%
\pgfsetroundjoin%
\definecolor{currentfill}{rgb}{0.000000,0.000000,0.000000}%
\pgfsetfillcolor{currentfill}%
\pgfsetlinewidth{0.803000pt}%
\definecolor{currentstroke}{rgb}{0.000000,0.000000,0.000000}%
\pgfsetstrokecolor{currentstroke}%
\pgfsetdash{}{0pt}%
\pgfsys@defobject{currentmarker}{\pgfqpoint{0.000000in}{-0.048611in}}{\pgfqpoint{0.000000in}{0.000000in}}{%
\pgfpathmoveto{\pgfqpoint{0.000000in}{0.000000in}}%
\pgfpathlineto{\pgfqpoint{0.000000in}{-0.048611in}}%
\pgfusepath{stroke,fill}%
}%
\begin{pgfscope}%
\pgfsys@transformshift{1.783182in}{0.499444in}%
\pgfsys@useobject{currentmarker}{}%
\end{pgfscope}%
\end{pgfscope}%
\begin{pgfscope}%
\pgfsetbuttcap%
\pgfsetroundjoin%
\definecolor{currentfill}{rgb}{0.000000,0.000000,0.000000}%
\pgfsetfillcolor{currentfill}%
\pgfsetlinewidth{0.803000pt}%
\definecolor{currentstroke}{rgb}{0.000000,0.000000,0.000000}%
\pgfsetstrokecolor{currentstroke}%
\pgfsetdash{}{0pt}%
\pgfsys@defobject{currentmarker}{\pgfqpoint{0.000000in}{-0.048611in}}{\pgfqpoint{0.000000in}{0.000000in}}{%
\pgfpathmoveto{\pgfqpoint{0.000000in}{0.000000in}}%
\pgfpathlineto{\pgfqpoint{0.000000in}{-0.048611in}}%
\pgfusepath{stroke,fill}%
}%
\begin{pgfscope}%
\pgfsys@transformshift{1.941705in}{0.499444in}%
\pgfsys@useobject{currentmarker}{}%
\end{pgfscope}%
\end{pgfscope}%
\begin{pgfscope}%
\definecolor{textcolor}{rgb}{0.000000,0.000000,0.000000}%
\pgfsetstrokecolor{textcolor}%
\pgfsetfillcolor{textcolor}%
\pgftext[x=1.941705in,y=0.402222in,,top]{\color{textcolor}\rmfamily\fontsize{10.000000}{12.000000}\selectfont 0.4}%
\end{pgfscope}%
\begin{pgfscope}%
\pgfsetbuttcap%
\pgfsetroundjoin%
\definecolor{currentfill}{rgb}{0.000000,0.000000,0.000000}%
\pgfsetfillcolor{currentfill}%
\pgfsetlinewidth{0.803000pt}%
\definecolor{currentstroke}{rgb}{0.000000,0.000000,0.000000}%
\pgfsetstrokecolor{currentstroke}%
\pgfsetdash{}{0pt}%
\pgfsys@defobject{currentmarker}{\pgfqpoint{0.000000in}{-0.048611in}}{\pgfqpoint{0.000000in}{0.000000in}}{%
\pgfpathmoveto{\pgfqpoint{0.000000in}{0.000000in}}%
\pgfpathlineto{\pgfqpoint{0.000000in}{-0.048611in}}%
\pgfusepath{stroke,fill}%
}%
\begin{pgfscope}%
\pgfsys@transformshift{2.100228in}{0.499444in}%
\pgfsys@useobject{currentmarker}{}%
\end{pgfscope}%
\end{pgfscope}%
\begin{pgfscope}%
\pgfsetbuttcap%
\pgfsetroundjoin%
\definecolor{currentfill}{rgb}{0.000000,0.000000,0.000000}%
\pgfsetfillcolor{currentfill}%
\pgfsetlinewidth{0.803000pt}%
\definecolor{currentstroke}{rgb}{0.000000,0.000000,0.000000}%
\pgfsetstrokecolor{currentstroke}%
\pgfsetdash{}{0pt}%
\pgfsys@defobject{currentmarker}{\pgfqpoint{0.000000in}{-0.048611in}}{\pgfqpoint{0.000000in}{0.000000in}}{%
\pgfpathmoveto{\pgfqpoint{0.000000in}{0.000000in}}%
\pgfpathlineto{\pgfqpoint{0.000000in}{-0.048611in}}%
\pgfusepath{stroke,fill}%
}%
\begin{pgfscope}%
\pgfsys@transformshift{2.258750in}{0.499444in}%
\pgfsys@useobject{currentmarker}{}%
\end{pgfscope}%
\end{pgfscope}%
\begin{pgfscope}%
\definecolor{textcolor}{rgb}{0.000000,0.000000,0.000000}%
\pgfsetstrokecolor{textcolor}%
\pgfsetfillcolor{textcolor}%
\pgftext[x=2.258750in,y=0.402222in,,top]{\color{textcolor}\rmfamily\fontsize{10.000000}{12.000000}\selectfont 0.5}%
\end{pgfscope}%
\begin{pgfscope}%
\pgfsetbuttcap%
\pgfsetroundjoin%
\definecolor{currentfill}{rgb}{0.000000,0.000000,0.000000}%
\pgfsetfillcolor{currentfill}%
\pgfsetlinewidth{0.803000pt}%
\definecolor{currentstroke}{rgb}{0.000000,0.000000,0.000000}%
\pgfsetstrokecolor{currentstroke}%
\pgfsetdash{}{0pt}%
\pgfsys@defobject{currentmarker}{\pgfqpoint{0.000000in}{-0.048611in}}{\pgfqpoint{0.000000in}{0.000000in}}{%
\pgfpathmoveto{\pgfqpoint{0.000000in}{0.000000in}}%
\pgfpathlineto{\pgfqpoint{0.000000in}{-0.048611in}}%
\pgfusepath{stroke,fill}%
}%
\begin{pgfscope}%
\pgfsys@transformshift{2.417273in}{0.499444in}%
\pgfsys@useobject{currentmarker}{}%
\end{pgfscope}%
\end{pgfscope}%
\begin{pgfscope}%
\pgfsetbuttcap%
\pgfsetroundjoin%
\definecolor{currentfill}{rgb}{0.000000,0.000000,0.000000}%
\pgfsetfillcolor{currentfill}%
\pgfsetlinewidth{0.803000pt}%
\definecolor{currentstroke}{rgb}{0.000000,0.000000,0.000000}%
\pgfsetstrokecolor{currentstroke}%
\pgfsetdash{}{0pt}%
\pgfsys@defobject{currentmarker}{\pgfqpoint{0.000000in}{-0.048611in}}{\pgfqpoint{0.000000in}{0.000000in}}{%
\pgfpathmoveto{\pgfqpoint{0.000000in}{0.000000in}}%
\pgfpathlineto{\pgfqpoint{0.000000in}{-0.048611in}}%
\pgfusepath{stroke,fill}%
}%
\begin{pgfscope}%
\pgfsys@transformshift{2.575796in}{0.499444in}%
\pgfsys@useobject{currentmarker}{}%
\end{pgfscope}%
\end{pgfscope}%
\begin{pgfscope}%
\definecolor{textcolor}{rgb}{0.000000,0.000000,0.000000}%
\pgfsetstrokecolor{textcolor}%
\pgfsetfillcolor{textcolor}%
\pgftext[x=2.575796in,y=0.402222in,,top]{\color{textcolor}\rmfamily\fontsize{10.000000}{12.000000}\selectfont 0.6}%
\end{pgfscope}%
\begin{pgfscope}%
\pgfsetbuttcap%
\pgfsetroundjoin%
\definecolor{currentfill}{rgb}{0.000000,0.000000,0.000000}%
\pgfsetfillcolor{currentfill}%
\pgfsetlinewidth{0.803000pt}%
\definecolor{currentstroke}{rgb}{0.000000,0.000000,0.000000}%
\pgfsetstrokecolor{currentstroke}%
\pgfsetdash{}{0pt}%
\pgfsys@defobject{currentmarker}{\pgfqpoint{0.000000in}{-0.048611in}}{\pgfqpoint{0.000000in}{0.000000in}}{%
\pgfpathmoveto{\pgfqpoint{0.000000in}{0.000000in}}%
\pgfpathlineto{\pgfqpoint{0.000000in}{-0.048611in}}%
\pgfusepath{stroke,fill}%
}%
\begin{pgfscope}%
\pgfsys@transformshift{2.734318in}{0.499444in}%
\pgfsys@useobject{currentmarker}{}%
\end{pgfscope}%
\end{pgfscope}%
\begin{pgfscope}%
\pgfsetbuttcap%
\pgfsetroundjoin%
\definecolor{currentfill}{rgb}{0.000000,0.000000,0.000000}%
\pgfsetfillcolor{currentfill}%
\pgfsetlinewidth{0.803000pt}%
\definecolor{currentstroke}{rgb}{0.000000,0.000000,0.000000}%
\pgfsetstrokecolor{currentstroke}%
\pgfsetdash{}{0pt}%
\pgfsys@defobject{currentmarker}{\pgfqpoint{0.000000in}{-0.048611in}}{\pgfqpoint{0.000000in}{0.000000in}}{%
\pgfpathmoveto{\pgfqpoint{0.000000in}{0.000000in}}%
\pgfpathlineto{\pgfqpoint{0.000000in}{-0.048611in}}%
\pgfusepath{stroke,fill}%
}%
\begin{pgfscope}%
\pgfsys@transformshift{2.892841in}{0.499444in}%
\pgfsys@useobject{currentmarker}{}%
\end{pgfscope}%
\end{pgfscope}%
\begin{pgfscope}%
\definecolor{textcolor}{rgb}{0.000000,0.000000,0.000000}%
\pgfsetstrokecolor{textcolor}%
\pgfsetfillcolor{textcolor}%
\pgftext[x=2.892841in,y=0.402222in,,top]{\color{textcolor}\rmfamily\fontsize{10.000000}{12.000000}\selectfont 0.7}%
\end{pgfscope}%
\begin{pgfscope}%
\pgfsetbuttcap%
\pgfsetroundjoin%
\definecolor{currentfill}{rgb}{0.000000,0.000000,0.000000}%
\pgfsetfillcolor{currentfill}%
\pgfsetlinewidth{0.803000pt}%
\definecolor{currentstroke}{rgb}{0.000000,0.000000,0.000000}%
\pgfsetstrokecolor{currentstroke}%
\pgfsetdash{}{0pt}%
\pgfsys@defobject{currentmarker}{\pgfqpoint{0.000000in}{-0.048611in}}{\pgfqpoint{0.000000in}{0.000000in}}{%
\pgfpathmoveto{\pgfqpoint{0.000000in}{0.000000in}}%
\pgfpathlineto{\pgfqpoint{0.000000in}{-0.048611in}}%
\pgfusepath{stroke,fill}%
}%
\begin{pgfscope}%
\pgfsys@transformshift{3.051364in}{0.499444in}%
\pgfsys@useobject{currentmarker}{}%
\end{pgfscope}%
\end{pgfscope}%
\begin{pgfscope}%
\pgfsetbuttcap%
\pgfsetroundjoin%
\definecolor{currentfill}{rgb}{0.000000,0.000000,0.000000}%
\pgfsetfillcolor{currentfill}%
\pgfsetlinewidth{0.803000pt}%
\definecolor{currentstroke}{rgb}{0.000000,0.000000,0.000000}%
\pgfsetstrokecolor{currentstroke}%
\pgfsetdash{}{0pt}%
\pgfsys@defobject{currentmarker}{\pgfqpoint{0.000000in}{-0.048611in}}{\pgfqpoint{0.000000in}{0.000000in}}{%
\pgfpathmoveto{\pgfqpoint{0.000000in}{0.000000in}}%
\pgfpathlineto{\pgfqpoint{0.000000in}{-0.048611in}}%
\pgfusepath{stroke,fill}%
}%
\begin{pgfscope}%
\pgfsys@transformshift{3.209887in}{0.499444in}%
\pgfsys@useobject{currentmarker}{}%
\end{pgfscope}%
\end{pgfscope}%
\begin{pgfscope}%
\definecolor{textcolor}{rgb}{0.000000,0.000000,0.000000}%
\pgfsetstrokecolor{textcolor}%
\pgfsetfillcolor{textcolor}%
\pgftext[x=3.209887in,y=0.402222in,,top]{\color{textcolor}\rmfamily\fontsize{10.000000}{12.000000}\selectfont 0.8}%
\end{pgfscope}%
\begin{pgfscope}%
\pgfsetbuttcap%
\pgfsetroundjoin%
\definecolor{currentfill}{rgb}{0.000000,0.000000,0.000000}%
\pgfsetfillcolor{currentfill}%
\pgfsetlinewidth{0.803000pt}%
\definecolor{currentstroke}{rgb}{0.000000,0.000000,0.000000}%
\pgfsetstrokecolor{currentstroke}%
\pgfsetdash{}{0pt}%
\pgfsys@defobject{currentmarker}{\pgfqpoint{0.000000in}{-0.048611in}}{\pgfqpoint{0.000000in}{0.000000in}}{%
\pgfpathmoveto{\pgfqpoint{0.000000in}{0.000000in}}%
\pgfpathlineto{\pgfqpoint{0.000000in}{-0.048611in}}%
\pgfusepath{stroke,fill}%
}%
\begin{pgfscope}%
\pgfsys@transformshift{3.368409in}{0.499444in}%
\pgfsys@useobject{currentmarker}{}%
\end{pgfscope}%
\end{pgfscope}%
\begin{pgfscope}%
\pgfsetbuttcap%
\pgfsetroundjoin%
\definecolor{currentfill}{rgb}{0.000000,0.000000,0.000000}%
\pgfsetfillcolor{currentfill}%
\pgfsetlinewidth{0.803000pt}%
\definecolor{currentstroke}{rgb}{0.000000,0.000000,0.000000}%
\pgfsetstrokecolor{currentstroke}%
\pgfsetdash{}{0pt}%
\pgfsys@defobject{currentmarker}{\pgfqpoint{0.000000in}{-0.048611in}}{\pgfqpoint{0.000000in}{0.000000in}}{%
\pgfpathmoveto{\pgfqpoint{0.000000in}{0.000000in}}%
\pgfpathlineto{\pgfqpoint{0.000000in}{-0.048611in}}%
\pgfusepath{stroke,fill}%
}%
\begin{pgfscope}%
\pgfsys@transformshift{3.526932in}{0.499444in}%
\pgfsys@useobject{currentmarker}{}%
\end{pgfscope}%
\end{pgfscope}%
\begin{pgfscope}%
\definecolor{textcolor}{rgb}{0.000000,0.000000,0.000000}%
\pgfsetstrokecolor{textcolor}%
\pgfsetfillcolor{textcolor}%
\pgftext[x=3.526932in,y=0.402222in,,top]{\color{textcolor}\rmfamily\fontsize{10.000000}{12.000000}\selectfont 0.9}%
\end{pgfscope}%
\begin{pgfscope}%
\pgfsetbuttcap%
\pgfsetroundjoin%
\definecolor{currentfill}{rgb}{0.000000,0.000000,0.000000}%
\pgfsetfillcolor{currentfill}%
\pgfsetlinewidth{0.803000pt}%
\definecolor{currentstroke}{rgb}{0.000000,0.000000,0.000000}%
\pgfsetstrokecolor{currentstroke}%
\pgfsetdash{}{0pt}%
\pgfsys@defobject{currentmarker}{\pgfqpoint{0.000000in}{-0.048611in}}{\pgfqpoint{0.000000in}{0.000000in}}{%
\pgfpathmoveto{\pgfqpoint{0.000000in}{0.000000in}}%
\pgfpathlineto{\pgfqpoint{0.000000in}{-0.048611in}}%
\pgfusepath{stroke,fill}%
}%
\begin{pgfscope}%
\pgfsys@transformshift{3.685455in}{0.499444in}%
\pgfsys@useobject{currentmarker}{}%
\end{pgfscope}%
\end{pgfscope}%
\begin{pgfscope}%
\pgfsetbuttcap%
\pgfsetroundjoin%
\definecolor{currentfill}{rgb}{0.000000,0.000000,0.000000}%
\pgfsetfillcolor{currentfill}%
\pgfsetlinewidth{0.803000pt}%
\definecolor{currentstroke}{rgb}{0.000000,0.000000,0.000000}%
\pgfsetstrokecolor{currentstroke}%
\pgfsetdash{}{0pt}%
\pgfsys@defobject{currentmarker}{\pgfqpoint{0.000000in}{-0.048611in}}{\pgfqpoint{0.000000in}{0.000000in}}{%
\pgfpathmoveto{\pgfqpoint{0.000000in}{0.000000in}}%
\pgfpathlineto{\pgfqpoint{0.000000in}{-0.048611in}}%
\pgfusepath{stroke,fill}%
}%
\begin{pgfscope}%
\pgfsys@transformshift{3.843978in}{0.499444in}%
\pgfsys@useobject{currentmarker}{}%
\end{pgfscope}%
\end{pgfscope}%
\begin{pgfscope}%
\definecolor{textcolor}{rgb}{0.000000,0.000000,0.000000}%
\pgfsetstrokecolor{textcolor}%
\pgfsetfillcolor{textcolor}%
\pgftext[x=3.843978in,y=0.402222in,,top]{\color{textcolor}\rmfamily\fontsize{10.000000}{12.000000}\selectfont 1.0}%
\end{pgfscope}%
\begin{pgfscope}%
\pgfsetbuttcap%
\pgfsetroundjoin%
\definecolor{currentfill}{rgb}{0.000000,0.000000,0.000000}%
\pgfsetfillcolor{currentfill}%
\pgfsetlinewidth{0.803000pt}%
\definecolor{currentstroke}{rgb}{0.000000,0.000000,0.000000}%
\pgfsetstrokecolor{currentstroke}%
\pgfsetdash{}{0pt}%
\pgfsys@defobject{currentmarker}{\pgfqpoint{0.000000in}{-0.048611in}}{\pgfqpoint{0.000000in}{0.000000in}}{%
\pgfpathmoveto{\pgfqpoint{0.000000in}{0.000000in}}%
\pgfpathlineto{\pgfqpoint{0.000000in}{-0.048611in}}%
\pgfusepath{stroke,fill}%
}%
\begin{pgfscope}%
\pgfsys@transformshift{4.002500in}{0.499444in}%
\pgfsys@useobject{currentmarker}{}%
\end{pgfscope}%
\end{pgfscope}%
\begin{pgfscope}%
\definecolor{textcolor}{rgb}{0.000000,0.000000,0.000000}%
\pgfsetstrokecolor{textcolor}%
\pgfsetfillcolor{textcolor}%
\pgftext[x=2.258750in,y=0.223333in,,top]{\color{textcolor}\rmfamily\fontsize{10.000000}{12.000000}\selectfont \(\displaystyle p\)}%
\end{pgfscope}%
\begin{pgfscope}%
\pgfsetbuttcap%
\pgfsetroundjoin%
\definecolor{currentfill}{rgb}{0.000000,0.000000,0.000000}%
\pgfsetfillcolor{currentfill}%
\pgfsetlinewidth{0.803000pt}%
\definecolor{currentstroke}{rgb}{0.000000,0.000000,0.000000}%
\pgfsetstrokecolor{currentstroke}%
\pgfsetdash{}{0pt}%
\pgfsys@defobject{currentmarker}{\pgfqpoint{-0.048611in}{0.000000in}}{\pgfqpoint{-0.000000in}{0.000000in}}{%
\pgfpathmoveto{\pgfqpoint{-0.000000in}{0.000000in}}%
\pgfpathlineto{\pgfqpoint{-0.048611in}{0.000000in}}%
\pgfusepath{stroke,fill}%
}%
\begin{pgfscope}%
\pgfsys@transformshift{0.515000in}{0.499444in}%
\pgfsys@useobject{currentmarker}{}%
\end{pgfscope}%
\end{pgfscope}%
\begin{pgfscope}%
\definecolor{textcolor}{rgb}{0.000000,0.000000,0.000000}%
\pgfsetstrokecolor{textcolor}%
\pgfsetfillcolor{textcolor}%
\pgftext[x=0.348333in, y=0.451250in, left, base]{\color{textcolor}\rmfamily\fontsize{10.000000}{12.000000}\selectfont \(\displaystyle {0}\)}%
\end{pgfscope}%
\begin{pgfscope}%
\pgfsetbuttcap%
\pgfsetroundjoin%
\definecolor{currentfill}{rgb}{0.000000,0.000000,0.000000}%
\pgfsetfillcolor{currentfill}%
\pgfsetlinewidth{0.803000pt}%
\definecolor{currentstroke}{rgb}{0.000000,0.000000,0.000000}%
\pgfsetstrokecolor{currentstroke}%
\pgfsetdash{}{0pt}%
\pgfsys@defobject{currentmarker}{\pgfqpoint{-0.048611in}{0.000000in}}{\pgfqpoint{-0.000000in}{0.000000in}}{%
\pgfpathmoveto{\pgfqpoint{-0.000000in}{0.000000in}}%
\pgfpathlineto{\pgfqpoint{-0.048611in}{0.000000in}}%
\pgfusepath{stroke,fill}%
}%
\begin{pgfscope}%
\pgfsys@transformshift{0.515000in}{0.881688in}%
\pgfsys@useobject{currentmarker}{}%
\end{pgfscope}%
\end{pgfscope}%
\begin{pgfscope}%
\definecolor{textcolor}{rgb}{0.000000,0.000000,0.000000}%
\pgfsetstrokecolor{textcolor}%
\pgfsetfillcolor{textcolor}%
\pgftext[x=0.348333in, y=0.833494in, left, base]{\color{textcolor}\rmfamily\fontsize{10.000000}{12.000000}\selectfont \(\displaystyle {5}\)}%
\end{pgfscope}%
\begin{pgfscope}%
\pgfsetbuttcap%
\pgfsetroundjoin%
\definecolor{currentfill}{rgb}{0.000000,0.000000,0.000000}%
\pgfsetfillcolor{currentfill}%
\pgfsetlinewidth{0.803000pt}%
\definecolor{currentstroke}{rgb}{0.000000,0.000000,0.000000}%
\pgfsetstrokecolor{currentstroke}%
\pgfsetdash{}{0pt}%
\pgfsys@defobject{currentmarker}{\pgfqpoint{-0.048611in}{0.000000in}}{\pgfqpoint{-0.000000in}{0.000000in}}{%
\pgfpathmoveto{\pgfqpoint{-0.000000in}{0.000000in}}%
\pgfpathlineto{\pgfqpoint{-0.048611in}{0.000000in}}%
\pgfusepath{stroke,fill}%
}%
\begin{pgfscope}%
\pgfsys@transformshift{0.515000in}{1.263932in}%
\pgfsys@useobject{currentmarker}{}%
\end{pgfscope}%
\end{pgfscope}%
\begin{pgfscope}%
\definecolor{textcolor}{rgb}{0.000000,0.000000,0.000000}%
\pgfsetstrokecolor{textcolor}%
\pgfsetfillcolor{textcolor}%
\pgftext[x=0.278889in, y=1.215738in, left, base]{\color{textcolor}\rmfamily\fontsize{10.000000}{12.000000}\selectfont \(\displaystyle {10}\)}%
\end{pgfscope}%
\begin{pgfscope}%
\pgfsetbuttcap%
\pgfsetroundjoin%
\definecolor{currentfill}{rgb}{0.000000,0.000000,0.000000}%
\pgfsetfillcolor{currentfill}%
\pgfsetlinewidth{0.803000pt}%
\definecolor{currentstroke}{rgb}{0.000000,0.000000,0.000000}%
\pgfsetstrokecolor{currentstroke}%
\pgfsetdash{}{0pt}%
\pgfsys@defobject{currentmarker}{\pgfqpoint{-0.048611in}{0.000000in}}{\pgfqpoint{-0.000000in}{0.000000in}}{%
\pgfpathmoveto{\pgfqpoint{-0.000000in}{0.000000in}}%
\pgfpathlineto{\pgfqpoint{-0.048611in}{0.000000in}}%
\pgfusepath{stroke,fill}%
}%
\begin{pgfscope}%
\pgfsys@transformshift{0.515000in}{1.646176in}%
\pgfsys@useobject{currentmarker}{}%
\end{pgfscope}%
\end{pgfscope}%
\begin{pgfscope}%
\definecolor{textcolor}{rgb}{0.000000,0.000000,0.000000}%
\pgfsetstrokecolor{textcolor}%
\pgfsetfillcolor{textcolor}%
\pgftext[x=0.278889in, y=1.597981in, left, base]{\color{textcolor}\rmfamily\fontsize{10.000000}{12.000000}\selectfont \(\displaystyle {15}\)}%
\end{pgfscope}%
\begin{pgfscope}%
\definecolor{textcolor}{rgb}{0.000000,0.000000,0.000000}%
\pgfsetstrokecolor{textcolor}%
\pgfsetfillcolor{textcolor}%
\pgftext[x=0.223333in,y=1.076944in,,bottom,rotate=90.000000]{\color{textcolor}\rmfamily\fontsize{10.000000}{12.000000}\selectfont Percent of Data Set}%
\end{pgfscope}%
\begin{pgfscope}%
\pgfsetrectcap%
\pgfsetmiterjoin%
\pgfsetlinewidth{0.803000pt}%
\definecolor{currentstroke}{rgb}{0.000000,0.000000,0.000000}%
\pgfsetstrokecolor{currentstroke}%
\pgfsetdash{}{0pt}%
\pgfpathmoveto{\pgfqpoint{0.515000in}{0.499444in}}%
\pgfpathlineto{\pgfqpoint{0.515000in}{1.654444in}}%
\pgfusepath{stroke}%
\end{pgfscope}%
\begin{pgfscope}%
\pgfsetrectcap%
\pgfsetmiterjoin%
\pgfsetlinewidth{0.803000pt}%
\definecolor{currentstroke}{rgb}{0.000000,0.000000,0.000000}%
\pgfsetstrokecolor{currentstroke}%
\pgfsetdash{}{0pt}%
\pgfpathmoveto{\pgfqpoint{4.002500in}{0.499444in}}%
\pgfpathlineto{\pgfqpoint{4.002500in}{1.654444in}}%
\pgfusepath{stroke}%
\end{pgfscope}%
\begin{pgfscope}%
\pgfsetrectcap%
\pgfsetmiterjoin%
\pgfsetlinewidth{0.803000pt}%
\definecolor{currentstroke}{rgb}{0.000000,0.000000,0.000000}%
\pgfsetstrokecolor{currentstroke}%
\pgfsetdash{}{0pt}%
\pgfpathmoveto{\pgfqpoint{0.515000in}{0.499444in}}%
\pgfpathlineto{\pgfqpoint{4.002500in}{0.499444in}}%
\pgfusepath{stroke}%
\end{pgfscope}%
\begin{pgfscope}%
\pgfsetrectcap%
\pgfsetmiterjoin%
\pgfsetlinewidth{0.803000pt}%
\definecolor{currentstroke}{rgb}{0.000000,0.000000,0.000000}%
\pgfsetstrokecolor{currentstroke}%
\pgfsetdash{}{0pt}%
\pgfpathmoveto{\pgfqpoint{0.515000in}{1.654444in}}%
\pgfpathlineto{\pgfqpoint{4.002500in}{1.654444in}}%
\pgfusepath{stroke}%
\end{pgfscope}%
\begin{pgfscope}%
\pgfsetbuttcap%
\pgfsetmiterjoin%
\definecolor{currentfill}{rgb}{1.000000,1.000000,1.000000}%
\pgfsetfillcolor{currentfill}%
\pgfsetfillopacity{0.800000}%
\pgfsetlinewidth{1.003750pt}%
\definecolor{currentstroke}{rgb}{0.800000,0.800000,0.800000}%
\pgfsetstrokecolor{currentstroke}%
\pgfsetstrokeopacity{0.800000}%
\pgfsetdash{}{0pt}%
\pgfpathmoveto{\pgfqpoint{3.225556in}{1.154445in}}%
\pgfpathlineto{\pgfqpoint{3.905278in}{1.154445in}}%
\pgfpathquadraticcurveto{\pgfqpoint{3.933056in}{1.154445in}}{\pgfqpoint{3.933056in}{1.182222in}}%
\pgfpathlineto{\pgfqpoint{3.933056in}{1.557222in}}%
\pgfpathquadraticcurveto{\pgfqpoint{3.933056in}{1.585000in}}{\pgfqpoint{3.905278in}{1.585000in}}%
\pgfpathlineto{\pgfqpoint{3.225556in}{1.585000in}}%
\pgfpathquadraticcurveto{\pgfqpoint{3.197778in}{1.585000in}}{\pgfqpoint{3.197778in}{1.557222in}}%
\pgfpathlineto{\pgfqpoint{3.197778in}{1.182222in}}%
\pgfpathquadraticcurveto{\pgfqpoint{3.197778in}{1.154445in}}{\pgfqpoint{3.225556in}{1.154445in}}%
\pgfpathlineto{\pgfqpoint{3.225556in}{1.154445in}}%
\pgfpathclose%
\pgfusepath{stroke,fill}%
\end{pgfscope}%
\begin{pgfscope}%
\pgfsetbuttcap%
\pgfsetmiterjoin%
\pgfsetlinewidth{1.003750pt}%
\definecolor{currentstroke}{rgb}{0.000000,0.000000,0.000000}%
\pgfsetstrokecolor{currentstroke}%
\pgfsetdash{}{0pt}%
\pgfpathmoveto{\pgfqpoint{3.253334in}{1.432222in}}%
\pgfpathlineto{\pgfqpoint{3.531111in}{1.432222in}}%
\pgfpathlineto{\pgfqpoint{3.531111in}{1.529444in}}%
\pgfpathlineto{\pgfqpoint{3.253334in}{1.529444in}}%
\pgfpathlineto{\pgfqpoint{3.253334in}{1.432222in}}%
\pgfpathclose%
\pgfusepath{stroke}%
\end{pgfscope}%
\begin{pgfscope}%
\definecolor{textcolor}{rgb}{0.000000,0.000000,0.000000}%
\pgfsetstrokecolor{textcolor}%
\pgfsetfillcolor{textcolor}%
\pgftext[x=3.642223in,y=1.432222in,left,base]{\color{textcolor}\rmfamily\fontsize{10.000000}{12.000000}\selectfont Neg}%
\end{pgfscope}%
\begin{pgfscope}%
\pgfsetbuttcap%
\pgfsetmiterjoin%
\definecolor{currentfill}{rgb}{0.000000,0.000000,0.000000}%
\pgfsetfillcolor{currentfill}%
\pgfsetlinewidth{0.000000pt}%
\definecolor{currentstroke}{rgb}{0.000000,0.000000,0.000000}%
\pgfsetstrokecolor{currentstroke}%
\pgfsetstrokeopacity{0.000000}%
\pgfsetdash{}{0pt}%
\pgfpathmoveto{\pgfqpoint{3.253334in}{1.236944in}}%
\pgfpathlineto{\pgfqpoint{3.531111in}{1.236944in}}%
\pgfpathlineto{\pgfqpoint{3.531111in}{1.334167in}}%
\pgfpathlineto{\pgfqpoint{3.253334in}{1.334167in}}%
\pgfpathlineto{\pgfqpoint{3.253334in}{1.236944in}}%
\pgfpathclose%
\pgfusepath{fill}%
\end{pgfscope}%
\begin{pgfscope}%
\definecolor{textcolor}{rgb}{0.000000,0.000000,0.000000}%
\pgfsetstrokecolor{textcolor}%
\pgfsetfillcolor{textcolor}%
\pgftext[x=3.642223in,y=1.236944in,left,base]{\color{textcolor}\rmfamily\fontsize{10.000000}{12.000000}\selectfont Pos}%
\end{pgfscope}%
\end{pgfpicture}%
\makeatother%
\endgroup%
	
&
	\vskip 0pt
	\hfil ROC Curve
	
	%% Creator: Matplotlib, PGF backend
%%
%% To include the figure in your LaTeX document, write
%%   \input{<filename>.pgf}
%%
%% Make sure the required packages are loaded in your preamble
%%   \usepackage{pgf}
%%
%% Also ensure that all the required font packages are loaded; for instance,
%% the lmodern package is sometimes necessary when using math font.
%%   \usepackage{lmodern}
%%
%% Figures using additional raster images can only be included by \input if
%% they are in the same directory as the main LaTeX file. For loading figures
%% from other directories you can use the `import` package
%%   \usepackage{import}
%%
%% and then include the figures with
%%   \import{<path to file>}{<filename>.pgf}
%%
%% Matplotlib used the following preamble
%%   
%%   \usepackage{fontspec}
%%   \makeatletter\@ifpackageloaded{underscore}{}{\usepackage[strings]{underscore}}\makeatother
%%
\begingroup%
\makeatletter%
\begin{pgfpicture}%
\pgfpathrectangle{\pgfpointorigin}{\pgfqpoint{2.121861in}{1.654444in}}%
\pgfusepath{use as bounding box, clip}%
\begin{pgfscope}%
\pgfsetbuttcap%
\pgfsetmiterjoin%
\definecolor{currentfill}{rgb}{1.000000,1.000000,1.000000}%
\pgfsetfillcolor{currentfill}%
\pgfsetlinewidth{0.000000pt}%
\definecolor{currentstroke}{rgb}{1.000000,1.000000,1.000000}%
\pgfsetstrokecolor{currentstroke}%
\pgfsetdash{}{0pt}%
\pgfpathmoveto{\pgfqpoint{0.000000in}{0.000000in}}%
\pgfpathlineto{\pgfqpoint{2.121861in}{0.000000in}}%
\pgfpathlineto{\pgfqpoint{2.121861in}{1.654444in}}%
\pgfpathlineto{\pgfqpoint{0.000000in}{1.654444in}}%
\pgfpathlineto{\pgfqpoint{0.000000in}{0.000000in}}%
\pgfpathclose%
\pgfusepath{fill}%
\end{pgfscope}%
\begin{pgfscope}%
\pgfsetbuttcap%
\pgfsetmiterjoin%
\definecolor{currentfill}{rgb}{1.000000,1.000000,1.000000}%
\pgfsetfillcolor{currentfill}%
\pgfsetlinewidth{0.000000pt}%
\definecolor{currentstroke}{rgb}{0.000000,0.000000,0.000000}%
\pgfsetstrokecolor{currentstroke}%
\pgfsetstrokeopacity{0.000000}%
\pgfsetdash{}{0pt}%
\pgfpathmoveto{\pgfqpoint{0.503581in}{0.449444in}}%
\pgfpathlineto{\pgfqpoint{2.053581in}{0.449444in}}%
\pgfpathlineto{\pgfqpoint{2.053581in}{1.604444in}}%
\pgfpathlineto{\pgfqpoint{0.503581in}{1.604444in}}%
\pgfpathlineto{\pgfqpoint{0.503581in}{0.449444in}}%
\pgfpathclose%
\pgfusepath{fill}%
\end{pgfscope}%
\begin{pgfscope}%
\pgfsetbuttcap%
\pgfsetroundjoin%
\definecolor{currentfill}{rgb}{0.000000,0.000000,0.000000}%
\pgfsetfillcolor{currentfill}%
\pgfsetlinewidth{0.803000pt}%
\definecolor{currentstroke}{rgb}{0.000000,0.000000,0.000000}%
\pgfsetstrokecolor{currentstroke}%
\pgfsetdash{}{0pt}%
\pgfsys@defobject{currentmarker}{\pgfqpoint{0.000000in}{-0.048611in}}{\pgfqpoint{0.000000in}{0.000000in}}{%
\pgfpathmoveto{\pgfqpoint{0.000000in}{0.000000in}}%
\pgfpathlineto{\pgfqpoint{0.000000in}{-0.048611in}}%
\pgfusepath{stroke,fill}%
}%
\begin{pgfscope}%
\pgfsys@transformshift{0.574035in}{0.449444in}%
\pgfsys@useobject{currentmarker}{}%
\end{pgfscope}%
\end{pgfscope}%
\begin{pgfscope}%
\definecolor{textcolor}{rgb}{0.000000,0.000000,0.000000}%
\pgfsetstrokecolor{textcolor}%
\pgfsetfillcolor{textcolor}%
\pgftext[x=0.574035in,y=0.352222in,,top]{\color{textcolor}\rmfamily\fontsize{10.000000}{12.000000}\selectfont \(\displaystyle {0.0}\)}%
\end{pgfscope}%
\begin{pgfscope}%
\pgfsetbuttcap%
\pgfsetroundjoin%
\definecolor{currentfill}{rgb}{0.000000,0.000000,0.000000}%
\pgfsetfillcolor{currentfill}%
\pgfsetlinewidth{0.803000pt}%
\definecolor{currentstroke}{rgb}{0.000000,0.000000,0.000000}%
\pgfsetstrokecolor{currentstroke}%
\pgfsetdash{}{0pt}%
\pgfsys@defobject{currentmarker}{\pgfqpoint{0.000000in}{-0.048611in}}{\pgfqpoint{0.000000in}{0.000000in}}{%
\pgfpathmoveto{\pgfqpoint{0.000000in}{0.000000in}}%
\pgfpathlineto{\pgfqpoint{0.000000in}{-0.048611in}}%
\pgfusepath{stroke,fill}%
}%
\begin{pgfscope}%
\pgfsys@transformshift{1.278581in}{0.449444in}%
\pgfsys@useobject{currentmarker}{}%
\end{pgfscope}%
\end{pgfscope}%
\begin{pgfscope}%
\definecolor{textcolor}{rgb}{0.000000,0.000000,0.000000}%
\pgfsetstrokecolor{textcolor}%
\pgfsetfillcolor{textcolor}%
\pgftext[x=1.278581in,y=0.352222in,,top]{\color{textcolor}\rmfamily\fontsize{10.000000}{12.000000}\selectfont \(\displaystyle {0.5}\)}%
\end{pgfscope}%
\begin{pgfscope}%
\pgfsetbuttcap%
\pgfsetroundjoin%
\definecolor{currentfill}{rgb}{0.000000,0.000000,0.000000}%
\pgfsetfillcolor{currentfill}%
\pgfsetlinewidth{0.803000pt}%
\definecolor{currentstroke}{rgb}{0.000000,0.000000,0.000000}%
\pgfsetstrokecolor{currentstroke}%
\pgfsetdash{}{0pt}%
\pgfsys@defobject{currentmarker}{\pgfqpoint{0.000000in}{-0.048611in}}{\pgfqpoint{0.000000in}{0.000000in}}{%
\pgfpathmoveto{\pgfqpoint{0.000000in}{0.000000in}}%
\pgfpathlineto{\pgfqpoint{0.000000in}{-0.048611in}}%
\pgfusepath{stroke,fill}%
}%
\begin{pgfscope}%
\pgfsys@transformshift{1.983126in}{0.449444in}%
\pgfsys@useobject{currentmarker}{}%
\end{pgfscope}%
\end{pgfscope}%
\begin{pgfscope}%
\definecolor{textcolor}{rgb}{0.000000,0.000000,0.000000}%
\pgfsetstrokecolor{textcolor}%
\pgfsetfillcolor{textcolor}%
\pgftext[x=1.983126in,y=0.352222in,,top]{\color{textcolor}\rmfamily\fontsize{10.000000}{12.000000}\selectfont \(\displaystyle {1.0}\)}%
\end{pgfscope}%
\begin{pgfscope}%
\definecolor{textcolor}{rgb}{0.000000,0.000000,0.000000}%
\pgfsetstrokecolor{textcolor}%
\pgfsetfillcolor{textcolor}%
\pgftext[x=1.278581in,y=0.173333in,,top]{\color{textcolor}\rmfamily\fontsize{10.000000}{12.000000}\selectfont False positive rate}%
\end{pgfscope}%
\begin{pgfscope}%
\pgfsetbuttcap%
\pgfsetroundjoin%
\definecolor{currentfill}{rgb}{0.000000,0.000000,0.000000}%
\pgfsetfillcolor{currentfill}%
\pgfsetlinewidth{0.803000pt}%
\definecolor{currentstroke}{rgb}{0.000000,0.000000,0.000000}%
\pgfsetstrokecolor{currentstroke}%
\pgfsetdash{}{0pt}%
\pgfsys@defobject{currentmarker}{\pgfqpoint{-0.048611in}{0.000000in}}{\pgfqpoint{-0.000000in}{0.000000in}}{%
\pgfpathmoveto{\pgfqpoint{-0.000000in}{0.000000in}}%
\pgfpathlineto{\pgfqpoint{-0.048611in}{0.000000in}}%
\pgfusepath{stroke,fill}%
}%
\begin{pgfscope}%
\pgfsys@transformshift{0.503581in}{0.501944in}%
\pgfsys@useobject{currentmarker}{}%
\end{pgfscope}%
\end{pgfscope}%
\begin{pgfscope}%
\definecolor{textcolor}{rgb}{0.000000,0.000000,0.000000}%
\pgfsetstrokecolor{textcolor}%
\pgfsetfillcolor{textcolor}%
\pgftext[x=0.228889in, y=0.453750in, left, base]{\color{textcolor}\rmfamily\fontsize{10.000000}{12.000000}\selectfont \(\displaystyle {0.0}\)}%
\end{pgfscope}%
\begin{pgfscope}%
\pgfsetbuttcap%
\pgfsetroundjoin%
\definecolor{currentfill}{rgb}{0.000000,0.000000,0.000000}%
\pgfsetfillcolor{currentfill}%
\pgfsetlinewidth{0.803000pt}%
\definecolor{currentstroke}{rgb}{0.000000,0.000000,0.000000}%
\pgfsetstrokecolor{currentstroke}%
\pgfsetdash{}{0pt}%
\pgfsys@defobject{currentmarker}{\pgfqpoint{-0.048611in}{0.000000in}}{\pgfqpoint{-0.000000in}{0.000000in}}{%
\pgfpathmoveto{\pgfqpoint{-0.000000in}{0.000000in}}%
\pgfpathlineto{\pgfqpoint{-0.048611in}{0.000000in}}%
\pgfusepath{stroke,fill}%
}%
\begin{pgfscope}%
\pgfsys@transformshift{0.503581in}{1.026944in}%
\pgfsys@useobject{currentmarker}{}%
\end{pgfscope}%
\end{pgfscope}%
\begin{pgfscope}%
\definecolor{textcolor}{rgb}{0.000000,0.000000,0.000000}%
\pgfsetstrokecolor{textcolor}%
\pgfsetfillcolor{textcolor}%
\pgftext[x=0.228889in, y=0.978750in, left, base]{\color{textcolor}\rmfamily\fontsize{10.000000}{12.000000}\selectfont \(\displaystyle {0.5}\)}%
\end{pgfscope}%
\begin{pgfscope}%
\pgfsetbuttcap%
\pgfsetroundjoin%
\definecolor{currentfill}{rgb}{0.000000,0.000000,0.000000}%
\pgfsetfillcolor{currentfill}%
\pgfsetlinewidth{0.803000pt}%
\definecolor{currentstroke}{rgb}{0.000000,0.000000,0.000000}%
\pgfsetstrokecolor{currentstroke}%
\pgfsetdash{}{0pt}%
\pgfsys@defobject{currentmarker}{\pgfqpoint{-0.048611in}{0.000000in}}{\pgfqpoint{-0.000000in}{0.000000in}}{%
\pgfpathmoveto{\pgfqpoint{-0.000000in}{0.000000in}}%
\pgfpathlineto{\pgfqpoint{-0.048611in}{0.000000in}}%
\pgfusepath{stroke,fill}%
}%
\begin{pgfscope}%
\pgfsys@transformshift{0.503581in}{1.551944in}%
\pgfsys@useobject{currentmarker}{}%
\end{pgfscope}%
\end{pgfscope}%
\begin{pgfscope}%
\definecolor{textcolor}{rgb}{0.000000,0.000000,0.000000}%
\pgfsetstrokecolor{textcolor}%
\pgfsetfillcolor{textcolor}%
\pgftext[x=0.228889in, y=1.503750in, left, base]{\color{textcolor}\rmfamily\fontsize{10.000000}{12.000000}\selectfont \(\displaystyle {1.0}\)}%
\end{pgfscope}%
\begin{pgfscope}%
\definecolor{textcolor}{rgb}{0.000000,0.000000,0.000000}%
\pgfsetstrokecolor{textcolor}%
\pgfsetfillcolor{textcolor}%
\pgftext[x=0.173333in,y=1.026944in,,bottom,rotate=90.000000]{\color{textcolor}\rmfamily\fontsize{10.000000}{12.000000}\selectfont True positive rate}%
\end{pgfscope}%
\begin{pgfscope}%
\pgfpathrectangle{\pgfqpoint{0.503581in}{0.449444in}}{\pgfqpoint{1.550000in}{1.155000in}}%
\pgfusepath{clip}%
\pgfsetbuttcap%
\pgfsetroundjoin%
\pgfsetlinewidth{1.505625pt}%
\definecolor{currentstroke}{rgb}{0.000000,0.000000,0.000000}%
\pgfsetstrokecolor{currentstroke}%
\pgfsetdash{{5.550000pt}{2.400000pt}}{0.000000pt}%
\pgfpathmoveto{\pgfqpoint{0.574035in}{0.501944in}}%
\pgfpathlineto{\pgfqpoint{1.983126in}{1.551944in}}%
\pgfusepath{stroke}%
\end{pgfscope}%
\begin{pgfscope}%
\pgfpathrectangle{\pgfqpoint{0.503581in}{0.449444in}}{\pgfqpoint{1.550000in}{1.155000in}}%
\pgfusepath{clip}%
\pgfsetrectcap%
\pgfsetroundjoin%
\pgfsetlinewidth{1.505625pt}%
\definecolor{currentstroke}{rgb}{0.000000,0.000000,0.000000}%
\pgfsetstrokecolor{currentstroke}%
\pgfsetdash{}{0pt}%
\pgfpathmoveto{\pgfqpoint{0.574035in}{0.501944in}}%
\pgfpathlineto{\pgfqpoint{0.575145in}{0.511486in}}%
\pgfpathlineto{\pgfqpoint{0.575273in}{0.512565in}}%
\pgfpathlineto{\pgfqpoint{0.576383in}{0.519753in}}%
\pgfpathlineto{\pgfqpoint{0.576527in}{0.520803in}}%
\pgfpathlineto{\pgfqpoint{0.577637in}{0.527203in}}%
\pgfpathlineto{\pgfqpoint{0.577849in}{0.528312in}}%
\pgfpathlineto{\pgfqpoint{0.578959in}{0.534002in}}%
\pgfpathlineto{\pgfqpoint{0.579224in}{0.535110in}}%
\pgfpathlineto{\pgfqpoint{0.580322in}{0.540499in}}%
\pgfpathlineto{\pgfqpoint{0.580569in}{0.541608in}}%
\pgfpathlineto{\pgfqpoint{0.581676in}{0.547083in}}%
\pgfpathlineto{\pgfqpoint{0.581902in}{0.548173in}}%
\pgfpathlineto{\pgfqpoint{0.583012in}{0.553464in}}%
\pgfpathlineto{\pgfqpoint{0.583268in}{0.554573in}}%
\pgfpathlineto{\pgfqpoint{0.584373in}{0.559416in}}%
\pgfpathlineto{\pgfqpoint{0.584661in}{0.560506in}}%
\pgfpathlineto{\pgfqpoint{0.584661in}{0.560515in}}%
\pgfpathlineto{\pgfqpoint{0.585769in}{0.564980in}}%
\pgfpathlineto{\pgfqpoint{0.586002in}{0.566088in}}%
\pgfpathlineto{\pgfqpoint{0.587111in}{0.570300in}}%
\pgfpathlineto{\pgfqpoint{0.587360in}{0.571399in}}%
\pgfpathlineto{\pgfqpoint{0.588468in}{0.575892in}}%
\pgfpathlineto{\pgfqpoint{0.588717in}{0.577001in}}%
\pgfpathlineto{\pgfqpoint{0.589822in}{0.581602in}}%
\pgfpathlineto{\pgfqpoint{0.590166in}{0.582691in}}%
\pgfpathlineto{\pgfqpoint{0.591276in}{0.586951in}}%
\pgfpathlineto{\pgfqpoint{0.591546in}{0.588060in}}%
\pgfpathlineto{\pgfqpoint{0.592651in}{0.592767in}}%
\pgfpathlineto{\pgfqpoint{0.592998in}{0.593876in}}%
\pgfpathlineto{\pgfqpoint{0.594106in}{0.597708in}}%
\pgfpathlineto{\pgfqpoint{0.594341in}{0.598817in}}%
\pgfpathlineto{\pgfqpoint{0.595437in}{0.603291in}}%
\pgfpathlineto{\pgfqpoint{0.595753in}{0.604400in}}%
\pgfpathlineto{\pgfqpoint{0.596863in}{0.608300in}}%
\pgfpathlineto{\pgfqpoint{0.597179in}{0.609409in}}%
\pgfpathlineto{\pgfqpoint{0.598289in}{0.613562in}}%
\pgfpathlineto{\pgfqpoint{0.598578in}{0.614641in}}%
\pgfpathlineto{\pgfqpoint{0.599685in}{0.618269in}}%
\pgfpathlineto{\pgfqpoint{0.599985in}{0.619359in}}%
\pgfpathlineto{\pgfqpoint{0.601093in}{0.623084in}}%
\pgfpathlineto{\pgfqpoint{0.601451in}{0.624193in}}%
\pgfpathlineto{\pgfqpoint{0.602559in}{0.628248in}}%
\pgfpathlineto{\pgfqpoint{0.602959in}{0.629357in}}%
\pgfpathlineto{\pgfqpoint{0.604064in}{0.632654in}}%
\pgfpathlineto{\pgfqpoint{0.604460in}{0.633763in}}%
\pgfpathlineto{\pgfqpoint{0.605569in}{0.637595in}}%
\pgfpathlineto{\pgfqpoint{0.605907in}{0.638704in}}%
\pgfpathlineto{\pgfqpoint{0.607014in}{0.642089in}}%
\pgfpathlineto{\pgfqpoint{0.607359in}{0.643198in}}%
\pgfpathlineto{\pgfqpoint{0.608464in}{0.646660in}}%
\pgfpathlineto{\pgfqpoint{0.608822in}{0.647769in}}%
\pgfpathlineto{\pgfqpoint{0.609932in}{0.651484in}}%
\pgfpathlineto{\pgfqpoint{0.610362in}{0.652593in}}%
\pgfpathlineto{\pgfqpoint{0.611468in}{0.656581in}}%
\pgfpathlineto{\pgfqpoint{0.611840in}{0.657670in}}%
\pgfpathlineto{\pgfqpoint{0.612947in}{0.661687in}}%
\pgfpathlineto{\pgfqpoint{0.613415in}{0.662786in}}%
\pgfpathlineto{\pgfqpoint{0.614525in}{0.666453in}}%
\pgfpathlineto{\pgfqpoint{0.614886in}{0.667552in}}%
\pgfpathlineto{\pgfqpoint{0.615989in}{0.670762in}}%
\pgfpathlineto{\pgfqpoint{0.616375in}{0.671870in}}%
\pgfpathlineto{\pgfqpoint{0.617485in}{0.675090in}}%
\pgfpathlineto{\pgfqpoint{0.617866in}{0.676198in}}%
\pgfpathlineto{\pgfqpoint{0.618969in}{0.679428in}}%
\pgfpathlineto{\pgfqpoint{0.619355in}{0.680527in}}%
\pgfpathlineto{\pgfqpoint{0.620465in}{0.683532in}}%
\pgfpathlineto{\pgfqpoint{0.620844in}{0.684641in}}%
\pgfpathlineto{\pgfqpoint{0.621952in}{0.687948in}}%
\pgfpathlineto{\pgfqpoint{0.622327in}{0.689037in}}%
\pgfpathlineto{\pgfqpoint{0.623434in}{0.692169in}}%
\pgfpathlineto{\pgfqpoint{0.623774in}{0.693278in}}%
\pgfpathlineto{\pgfqpoint{0.624881in}{0.696662in}}%
\pgfpathlineto{\pgfqpoint{0.625363in}{0.697752in}}%
\pgfpathlineto{\pgfqpoint{0.626473in}{0.701234in}}%
\pgfpathlineto{\pgfqpoint{0.626833in}{0.702333in}}%
\pgfpathlineto{\pgfqpoint{0.627943in}{0.705163in}}%
\pgfpathlineto{\pgfqpoint{0.628318in}{0.706272in}}%
\pgfpathlineto{\pgfqpoint{0.629418in}{0.709238in}}%
\pgfpathlineto{\pgfqpoint{0.629872in}{0.710347in}}%
\pgfpathlineto{\pgfqpoint{0.630968in}{0.713265in}}%
\pgfpathlineto{\pgfqpoint{0.631459in}{0.714374in}}%
\pgfpathlineto{\pgfqpoint{0.632564in}{0.716834in}}%
\pgfpathlineto{\pgfqpoint{0.633139in}{0.717943in}}%
\pgfpathlineto{\pgfqpoint{0.634242in}{0.720735in}}%
\pgfpathlineto{\pgfqpoint{0.634667in}{0.721834in}}%
\pgfpathlineto{\pgfqpoint{0.635770in}{0.724878in}}%
\pgfpathlineto{\pgfqpoint{0.636268in}{0.725977in}}%
\pgfpathlineto{\pgfqpoint{0.637376in}{0.728681in}}%
\pgfpathlineto{\pgfqpoint{0.637857in}{0.729780in}}%
\pgfpathlineto{\pgfqpoint{0.638967in}{0.732620in}}%
\pgfpathlineto{\pgfqpoint{0.639374in}{0.733709in}}%
\pgfpathlineto{\pgfqpoint{0.640484in}{0.736404in}}%
\pgfpathlineto{\pgfqpoint{0.640931in}{0.737512in}}%
\pgfpathlineto{\pgfqpoint{0.642029in}{0.740382in}}%
\pgfpathlineto{\pgfqpoint{0.642476in}{0.741471in}}%
\pgfpathlineto{\pgfqpoint{0.643572in}{0.744039in}}%
\pgfpathlineto{\pgfqpoint{0.643584in}{0.744039in}}%
\pgfpathlineto{\pgfqpoint{0.644019in}{0.745138in}}%
\pgfpathlineto{\pgfqpoint{0.645128in}{0.747637in}}%
\pgfpathlineto{\pgfqpoint{0.645594in}{0.748736in}}%
\pgfpathlineto{\pgfqpoint{0.646701in}{0.751265in}}%
\pgfpathlineto{\pgfqpoint{0.647125in}{0.752374in}}%
\pgfpathlineto{\pgfqpoint{0.648235in}{0.755117in}}%
\pgfpathlineto{\pgfqpoint{0.648726in}{0.756226in}}%
\pgfpathlineto{\pgfqpoint{0.649833in}{0.758929in}}%
\pgfpathlineto{\pgfqpoint{0.650315in}{0.760009in}}%
\pgfpathlineto{\pgfqpoint{0.651425in}{0.762616in}}%
\pgfpathlineto{\pgfqpoint{0.651846in}{0.763705in}}%
\pgfpathlineto{\pgfqpoint{0.652956in}{0.766273in}}%
\pgfpathlineto{\pgfqpoint{0.653465in}{0.767381in}}%
\pgfpathlineto{\pgfqpoint{0.654573in}{0.769891in}}%
\pgfpathlineto{\pgfqpoint{0.655078in}{0.770990in}}%
\pgfpathlineto{\pgfqpoint{0.656173in}{0.773334in}}%
\pgfpathlineto{\pgfqpoint{0.656185in}{0.773334in}}%
\pgfpathlineto{\pgfqpoint{0.656655in}{0.774433in}}%
\pgfpathlineto{\pgfqpoint{0.657760in}{0.777283in}}%
\pgfpathlineto{\pgfqpoint{0.658309in}{0.778392in}}%
\pgfpathlineto{\pgfqpoint{0.659417in}{0.780424in}}%
\pgfpathlineto{\pgfqpoint{0.659894in}{0.781523in}}%
\pgfpathlineto{\pgfqpoint{0.661004in}{0.784169in}}%
\pgfpathlineto{\pgfqpoint{0.661502in}{0.785278in}}%
\pgfpathlineto{\pgfqpoint{0.662605in}{0.787379in}}%
\pgfpathlineto{\pgfqpoint{0.663114in}{0.788487in}}%
\pgfpathlineto{\pgfqpoint{0.664212in}{0.791074in}}%
\pgfpathlineto{\pgfqpoint{0.664850in}{0.792174in}}%
\pgfpathlineto{\pgfqpoint{0.665955in}{0.794333in}}%
\pgfpathlineto{\pgfqpoint{0.666434in}{0.795422in}}%
\pgfpathlineto{\pgfqpoint{0.667544in}{0.797815in}}%
\pgfpathlineto{\pgfqpoint{0.668268in}{0.798914in}}%
\pgfpathlineto{\pgfqpoint{0.669366in}{0.801248in}}%
\pgfpathlineto{\pgfqpoint{0.669964in}{0.802357in}}%
\pgfpathlineto{\pgfqpoint{0.671067in}{0.804925in}}%
\pgfpathlineto{\pgfqpoint{0.671653in}{0.806033in}}%
\pgfpathlineto{\pgfqpoint{0.672758in}{0.808426in}}%
\pgfpathlineto{\pgfqpoint{0.673310in}{0.809535in}}%
\pgfpathlineto{\pgfqpoint{0.674415in}{0.811927in}}%
\pgfpathlineto{\pgfqpoint{0.675032in}{0.813026in}}%
\pgfpathlineto{\pgfqpoint{0.676141in}{0.815195in}}%
\pgfpathlineto{\pgfqpoint{0.676635in}{0.816294in}}%
\pgfpathlineto{\pgfqpoint{0.677742in}{0.818590in}}%
\pgfpathlineto{\pgfqpoint{0.678291in}{0.819669in}}%
\pgfpathlineto{\pgfqpoint{0.679401in}{0.821984in}}%
\pgfpathlineto{\pgfqpoint{0.679997in}{0.823093in}}%
\pgfpathlineto{\pgfqpoint{0.681104in}{0.825427in}}%
\pgfpathlineto{\pgfqpoint{0.681621in}{0.826536in}}%
\pgfpathlineto{\pgfqpoint{0.682731in}{0.828598in}}%
\pgfpathlineto{\pgfqpoint{0.683266in}{0.829707in}}%
\pgfpathlineto{\pgfqpoint{0.684369in}{0.831730in}}%
\pgfpathlineto{\pgfqpoint{0.684983in}{0.832839in}}%
\pgfpathlineto{\pgfqpoint{0.686093in}{0.834920in}}%
\pgfpathlineto{\pgfqpoint{0.686607in}{0.836029in}}%
\pgfpathlineto{\pgfqpoint{0.687710in}{0.838188in}}%
\pgfpathlineto{\pgfqpoint{0.688368in}{0.839287in}}%
\pgfpathlineto{\pgfqpoint{0.689478in}{0.841252in}}%
\pgfpathlineto{\pgfqpoint{0.690102in}{0.842361in}}%
\pgfpathlineto{\pgfqpoint{0.691212in}{0.844423in}}%
\pgfpathlineto{\pgfqpoint{0.691996in}{0.845522in}}%
\pgfpathlineto{\pgfqpoint{0.693099in}{0.847681in}}%
\pgfpathlineto{\pgfqpoint{0.693657in}{0.848790in}}%
\pgfpathlineto{\pgfqpoint{0.694767in}{0.850560in}}%
\pgfpathlineto{\pgfqpoint{0.695363in}{0.851669in}}%
\pgfpathlineto{\pgfqpoint{0.696470in}{0.853507in}}%
\pgfpathlineto{\pgfqpoint{0.697108in}{0.854616in}}%
\pgfpathlineto{\pgfqpoint{0.698208in}{0.856464in}}%
\pgfpathlineto{\pgfqpoint{0.698936in}{0.857563in}}%
\pgfpathlineto{\pgfqpoint{0.700046in}{0.859644in}}%
\pgfpathlineto{\pgfqpoint{0.700840in}{0.860753in}}%
\pgfpathlineto{\pgfqpoint{0.701950in}{0.862620in}}%
\pgfpathlineto{\pgfqpoint{0.702657in}{0.863729in}}%
\pgfpathlineto{\pgfqpoint{0.703767in}{0.865538in}}%
\pgfpathlineto{\pgfqpoint{0.704353in}{0.866647in}}%
\pgfpathlineto{\pgfqpoint{0.705461in}{0.868816in}}%
\pgfpathlineto{\pgfqpoint{0.706138in}{0.869915in}}%
\pgfpathlineto{\pgfqpoint{0.707241in}{0.871792in}}%
\pgfpathlineto{\pgfqpoint{0.707724in}{0.872901in}}%
\pgfpathlineto{\pgfqpoint{0.708830in}{0.874729in}}%
\pgfpathlineto{\pgfqpoint{0.709397in}{0.875828in}}%
\pgfpathlineto{\pgfqpoint{0.710505in}{0.877540in}}%
\pgfpathlineto{\pgfqpoint{0.711070in}{0.878649in}}%
\pgfpathlineto{\pgfqpoint{0.712173in}{0.880662in}}%
\pgfpathlineto{\pgfqpoint{0.712899in}{0.881761in}}%
\pgfpathlineto{\pgfqpoint{0.714009in}{0.883775in}}%
\pgfpathlineto{\pgfqpoint{0.714735in}{0.884884in}}%
\pgfpathlineto{\pgfqpoint{0.715842in}{0.886712in}}%
\pgfpathlineto{\pgfqpoint{0.716603in}{0.887821in}}%
\pgfpathlineto{\pgfqpoint{0.717713in}{0.890058in}}%
\pgfpathlineto{\pgfqpoint{0.718369in}{0.891167in}}%
\pgfpathlineto{\pgfqpoint{0.719456in}{0.892820in}}%
\pgfpathlineto{\pgfqpoint{0.720207in}{0.893919in}}%
\pgfpathlineto{\pgfqpoint{0.721310in}{0.895602in}}%
\pgfpathlineto{\pgfqpoint{0.722087in}{0.896711in}}%
\pgfpathlineto{\pgfqpoint{0.723197in}{0.898617in}}%
\pgfpathlineto{\pgfqpoint{0.723777in}{0.899726in}}%
\pgfpathlineto{\pgfqpoint{0.724884in}{0.901564in}}%
\pgfpathlineto{\pgfqpoint{0.725664in}{0.902673in}}%
\pgfpathlineto{\pgfqpoint{0.726755in}{0.904414in}}%
\pgfpathlineto{\pgfqpoint{0.727539in}{0.905522in}}%
\pgfpathlineto{\pgfqpoint{0.728628in}{0.907419in}}%
\pgfpathlineto{\pgfqpoint{0.728632in}{0.907419in}}%
\pgfpathlineto{\pgfqpoint{0.729337in}{0.908528in}}%
\pgfpathlineto{\pgfqpoint{0.730445in}{0.910259in}}%
\pgfpathlineto{\pgfqpoint{0.731134in}{0.911368in}}%
\pgfpathlineto{\pgfqpoint{0.732244in}{0.912982in}}%
\pgfpathlineto{\pgfqpoint{0.732963in}{0.914091in}}%
\pgfpathlineto{\pgfqpoint{0.734072in}{0.915842in}}%
\pgfpathlineto{\pgfqpoint{0.734733in}{0.916951in}}%
\pgfpathlineto{\pgfqpoint{0.735841in}{0.918770in}}%
\pgfpathlineto{\pgfqpoint{0.736513in}{0.919849in}}%
\pgfpathlineto{\pgfqpoint{0.737623in}{0.921726in}}%
\pgfpathlineto{\pgfqpoint{0.738368in}{0.922835in}}%
\pgfpathlineto{\pgfqpoint{0.739473in}{0.924654in}}%
\pgfpathlineto{\pgfqpoint{0.740196in}{0.925763in}}%
\pgfpathlineto{\pgfqpoint{0.741304in}{0.927552in}}%
\pgfpathlineto{\pgfqpoint{0.742169in}{0.928661in}}%
\pgfpathlineto{\pgfqpoint{0.743277in}{0.930334in}}%
\pgfpathlineto{\pgfqpoint{0.743991in}{0.931433in}}%
\pgfpathlineto{\pgfqpoint{0.745099in}{0.932921in}}%
\pgfpathlineto{\pgfqpoint{0.745739in}{0.934020in}}%
\pgfpathlineto{\pgfqpoint{0.746839in}{0.935333in}}%
\pgfpathlineto{\pgfqpoint{0.747663in}{0.936432in}}%
\pgfpathlineto{\pgfqpoint{0.748756in}{0.937950in}}%
\pgfpathlineto{\pgfqpoint{0.749552in}{0.939058in}}%
\pgfpathlineto{\pgfqpoint{0.750657in}{0.940634in}}%
\pgfpathlineto{\pgfqpoint{0.751290in}{0.941743in}}%
\pgfpathlineto{\pgfqpoint{0.752379in}{0.943299in}}%
\pgfpathlineto{\pgfqpoint{0.753126in}{0.944408in}}%
\pgfpathlineto{\pgfqpoint{0.754236in}{0.946314in}}%
\pgfpathlineto{\pgfqpoint{0.754911in}{0.947423in}}%
\pgfpathlineto{\pgfqpoint{0.756018in}{0.949037in}}%
\pgfpathlineto{\pgfqpoint{0.756818in}{0.950146in}}%
\pgfpathlineto{\pgfqpoint{0.757921in}{0.951566in}}%
\pgfpathlineto{\pgfqpoint{0.758519in}{0.952675in}}%
\pgfpathlineto{\pgfqpoint{0.759622in}{0.954562in}}%
\pgfpathlineto{\pgfqpoint{0.760390in}{0.955671in}}%
\pgfpathlineto{\pgfqpoint{0.761432in}{0.956984in}}%
\pgfpathlineto{\pgfqpoint{0.762354in}{0.958093in}}%
\pgfpathlineto{\pgfqpoint{0.763461in}{0.959736in}}%
\pgfpathlineto{\pgfqpoint{0.764308in}{0.960826in}}%
\pgfpathlineto{\pgfqpoint{0.765418in}{0.962508in}}%
\pgfpathlineto{\pgfqpoint{0.766223in}{0.963617in}}%
\pgfpathlineto{\pgfqpoint{0.767333in}{0.964998in}}%
\pgfpathlineto{\pgfqpoint{0.768080in}{0.966107in}}%
\pgfpathlineto{\pgfqpoint{0.769190in}{0.967663in}}%
\pgfpathlineto{\pgfqpoint{0.770239in}{0.968772in}}%
\pgfpathlineto{\pgfqpoint{0.771349in}{0.970484in}}%
\pgfpathlineto{\pgfqpoint{0.772221in}{0.971573in}}%
\pgfpathlineto{\pgfqpoint{0.773310in}{0.972915in}}%
\pgfpathlineto{\pgfqpoint{0.774220in}{0.974024in}}%
\pgfpathlineto{\pgfqpoint{0.775328in}{0.975522in}}%
\pgfpathlineto{\pgfqpoint{0.776242in}{0.976631in}}%
\pgfpathlineto{\pgfqpoint{0.777345in}{0.978245in}}%
\pgfpathlineto{\pgfqpoint{0.778059in}{0.979354in}}%
\pgfpathlineto{\pgfqpoint{0.779162in}{0.980774in}}%
\pgfpathlineto{\pgfqpoint{0.780177in}{0.981873in}}%
\pgfpathlineto{\pgfqpoint{0.781286in}{0.983274in}}%
\pgfpathlineto{\pgfqpoint{0.782189in}{0.984382in}}%
\pgfpathlineto{\pgfqpoint{0.783294in}{0.985715in}}%
\pgfpathlineto{\pgfqpoint{0.784199in}{0.986824in}}%
\pgfpathlineto{\pgfqpoint{0.785300in}{0.988341in}}%
\pgfpathlineto{\pgfqpoint{0.786159in}{0.989450in}}%
\pgfpathlineto{\pgfqpoint{0.787266in}{0.991171in}}%
\pgfpathlineto{\pgfqpoint{0.788094in}{0.992280in}}%
\pgfpathlineto{\pgfqpoint{0.789204in}{0.993817in}}%
\pgfpathlineto{\pgfqpoint{0.790170in}{0.994926in}}%
\pgfpathlineto{\pgfqpoint{0.791266in}{0.996433in}}%
\pgfpathlineto{\pgfqpoint{0.792029in}{0.997542in}}%
\pgfpathlineto{\pgfqpoint{0.793139in}{0.998933in}}%
\pgfpathlineto{\pgfqpoint{0.794149in}{1.000042in}}%
\pgfpathlineto{\pgfqpoint{0.795242in}{1.001539in}}%
\pgfpathlineto{\pgfqpoint{0.796138in}{1.002648in}}%
\pgfpathlineto{\pgfqpoint{0.797231in}{1.004000in}}%
\pgfpathlineto{\pgfqpoint{0.797964in}{1.005070in}}%
\pgfpathlineto{\pgfqpoint{0.799051in}{1.006801in}}%
\pgfpathlineto{\pgfqpoint{0.799740in}{1.007910in}}%
\pgfpathlineto{\pgfqpoint{0.800850in}{1.009291in}}%
\pgfpathlineto{\pgfqpoint{0.801755in}{1.010400in}}%
\pgfpathlineto{\pgfqpoint{0.802853in}{1.011723in}}%
\pgfpathlineto{\pgfqpoint{0.803902in}{1.012831in}}%
\pgfpathlineto{\pgfqpoint{0.805010in}{1.014164in}}%
\pgfpathlineto{\pgfqpoint{0.805789in}{1.015273in}}%
\pgfpathlineto{\pgfqpoint{0.806894in}{1.016741in}}%
\pgfpathlineto{\pgfqpoint{0.807665in}{1.017850in}}%
\pgfpathlineto{\pgfqpoint{0.808772in}{1.019299in}}%
\pgfpathlineto{\pgfqpoint{0.809703in}{1.020408in}}%
\pgfpathlineto{\pgfqpoint{0.810813in}{1.021624in}}%
\pgfpathlineto{\pgfqpoint{0.811692in}{1.022733in}}%
\pgfpathlineto{\pgfqpoint{0.812781in}{1.024211in}}%
\pgfpathlineto{\pgfqpoint{0.813828in}{1.025320in}}%
\pgfpathlineto{\pgfqpoint{0.814936in}{1.026575in}}%
\pgfpathlineto{\pgfqpoint{0.815787in}{1.027683in}}%
\pgfpathlineto{\pgfqpoint{0.816871in}{1.028850in}}%
\pgfpathlineto{\pgfqpoint{0.817772in}{1.029959in}}%
\pgfpathlineto{\pgfqpoint{0.818882in}{1.031389in}}%
\pgfpathlineto{\pgfqpoint{0.819556in}{1.032498in}}%
\pgfpathlineto{\pgfqpoint{0.820659in}{1.033782in}}%
\pgfpathlineto{\pgfqpoint{0.821681in}{1.034890in}}%
\pgfpathlineto{\pgfqpoint{0.822791in}{1.036194in}}%
\pgfpathlineto{\pgfqpoint{0.823900in}{1.037293in}}%
\pgfpathlineto{\pgfqpoint{0.825010in}{1.038596in}}%
\pgfpathlineto{\pgfqpoint{0.825848in}{1.039705in}}%
\pgfpathlineto{\pgfqpoint{0.826958in}{1.041028in}}%
\pgfpathlineto{\pgfqpoint{0.827961in}{1.042127in}}%
\pgfpathlineto{\pgfqpoint{0.827961in}{1.042136in}}%
\pgfpathlineto{\pgfqpoint{0.829054in}{1.043323in}}%
\pgfpathlineto{\pgfqpoint{0.829941in}{1.044432in}}%
\pgfpathlineto{\pgfqpoint{0.831044in}{1.045706in}}%
\pgfpathlineto{\pgfqpoint{0.832188in}{1.046815in}}%
\pgfpathlineto{\pgfqpoint{0.833298in}{1.047953in}}%
\pgfpathlineto{\pgfqpoint{0.834333in}{1.049062in}}%
\pgfpathlineto{\pgfqpoint{0.835422in}{1.050511in}}%
\pgfpathlineto{\pgfqpoint{0.836444in}{1.051619in}}%
\pgfpathlineto{\pgfqpoint{0.837537in}{1.052942in}}%
\pgfpathlineto{\pgfqpoint{0.838373in}{1.054051in}}%
\pgfpathlineto{\pgfqpoint{0.839473in}{1.055296in}}%
\pgfpathlineto{\pgfqpoint{0.840332in}{1.056405in}}%
\pgfpathlineto{\pgfqpoint{0.841430in}{1.057494in}}%
\pgfpathlineto{\pgfqpoint{0.842323in}{1.058603in}}%
\pgfpathlineto{\pgfqpoint{0.843375in}{1.059712in}}%
\pgfpathlineto{\pgfqpoint{0.844331in}{1.060820in}}%
\pgfpathlineto{\pgfqpoint{0.845439in}{1.061978in}}%
\pgfpathlineto{\pgfqpoint{0.846449in}{1.063087in}}%
\pgfpathlineto{\pgfqpoint{0.847552in}{1.064176in}}%
\pgfpathlineto{\pgfqpoint{0.848678in}{1.065285in}}%
\pgfpathlineto{\pgfqpoint{0.849788in}{1.066510in}}%
\pgfpathlineto{\pgfqpoint{0.850800in}{1.067619in}}%
\pgfpathlineto{\pgfqpoint{0.851910in}{1.068883in}}%
\pgfpathlineto{\pgfqpoint{0.852901in}{1.069992in}}%
\pgfpathlineto{\pgfqpoint{0.854008in}{1.071120in}}%
\pgfpathlineto{\pgfqpoint{0.855048in}{1.072229in}}%
\pgfpathlineto{\pgfqpoint{0.856144in}{1.073396in}}%
\pgfpathlineto{\pgfqpoint{0.856154in}{1.073396in}}%
\pgfpathlineto{\pgfqpoint{0.857005in}{1.074505in}}%
\pgfpathlineto{\pgfqpoint{0.858094in}{1.075760in}}%
\pgfpathlineto{\pgfqpoint{0.859043in}{1.076869in}}%
\pgfpathlineto{\pgfqpoint{0.860146in}{1.078337in}}%
\pgfpathlineto{\pgfqpoint{0.861110in}{1.079446in}}%
\pgfpathlineto{\pgfqpoint{0.862215in}{1.080458in}}%
\pgfpathlineto{\pgfqpoint{0.863373in}{1.081537in}}%
\pgfpathlineto{\pgfqpoint{0.864476in}{1.082665in}}%
\pgfpathlineto{\pgfqpoint{0.865419in}{1.083755in}}%
\pgfpathlineto{\pgfqpoint{0.866521in}{1.085039in}}%
\pgfpathlineto{\pgfqpoint{0.867478in}{1.086147in}}%
\pgfpathlineto{\pgfqpoint{0.868588in}{1.087159in}}%
\pgfpathlineto{\pgfqpoint{0.869621in}{1.088268in}}%
\pgfpathlineto{\pgfqpoint{0.871059in}{1.089542in}}%
\pgfpathlineto{\pgfqpoint{0.872138in}{1.090651in}}%
\pgfpathlineto{\pgfqpoint{0.873232in}{1.091935in}}%
\pgfpathlineto{\pgfqpoint{0.874316in}{1.093034in}}%
\pgfpathlineto{\pgfqpoint{0.875421in}{1.094123in}}%
\pgfpathlineto{\pgfqpoint{0.876401in}{1.095232in}}%
\pgfpathlineto{\pgfqpoint{0.877511in}{1.096506in}}%
\pgfpathlineto{\pgfqpoint{0.878502in}{1.097615in}}%
\pgfpathlineto{\pgfqpoint{0.879609in}{1.098782in}}%
\pgfpathlineto{\pgfqpoint{0.880870in}{1.099891in}}%
\pgfpathlineto{\pgfqpoint{0.881976in}{1.101116in}}%
\pgfpathlineto{\pgfqpoint{0.882995in}{1.102225in}}%
\pgfpathlineto{\pgfqpoint{0.884105in}{1.103577in}}%
\pgfpathlineto{\pgfqpoint{0.885147in}{1.104686in}}%
\pgfpathlineto{\pgfqpoint{0.886252in}{1.105872in}}%
\pgfpathlineto{\pgfqpoint{0.887385in}{1.106981in}}%
\pgfpathlineto{\pgfqpoint{0.888488in}{1.107934in}}%
\pgfpathlineto{\pgfqpoint{0.889628in}{1.109043in}}%
\pgfpathlineto{\pgfqpoint{0.890733in}{1.110142in}}%
\pgfpathlineto{\pgfqpoint{0.891829in}{1.111251in}}%
\pgfpathlineto{\pgfqpoint{0.892925in}{1.112428in}}%
\pgfpathlineto{\pgfqpoint{0.894312in}{1.113536in}}%
\pgfpathlineto{\pgfqpoint{0.895415in}{1.114567in}}%
\pgfpathlineto{\pgfqpoint{0.896476in}{1.115676in}}%
\pgfpathlineto{\pgfqpoint{0.897581in}{1.116610in}}%
\pgfpathlineto{\pgfqpoint{0.898693in}{1.117719in}}%
\pgfpathlineto{\pgfqpoint{0.899801in}{1.118740in}}%
\pgfpathlineto{\pgfqpoint{0.901027in}{1.119839in}}%
\pgfpathlineto{\pgfqpoint{0.902137in}{1.120957in}}%
\pgfpathlineto{\pgfqpoint{0.903149in}{1.122057in}}%
\pgfpathlineto{\pgfqpoint{0.904249in}{1.123224in}}%
\pgfpathlineto{\pgfqpoint{0.905357in}{1.124332in}}%
\pgfpathlineto{\pgfqpoint{0.906462in}{1.125500in}}%
\pgfpathlineto{\pgfqpoint{0.907688in}{1.126608in}}%
\pgfpathlineto{\pgfqpoint{0.908777in}{1.127717in}}%
\pgfpathlineto{\pgfqpoint{0.908798in}{1.127717in}}%
\pgfpathlineto{\pgfqpoint{0.909952in}{1.128826in}}%
\pgfpathlineto{\pgfqpoint{0.911053in}{1.129828in}}%
\pgfpathlineto{\pgfqpoint{0.911946in}{1.130937in}}%
\pgfpathlineto{\pgfqpoint{0.913040in}{1.131890in}}%
\pgfpathlineto{\pgfqpoint{0.914271in}{1.132998in}}%
\pgfpathlineto{\pgfqpoint{0.915378in}{1.133981in}}%
\pgfpathlineto{\pgfqpoint{0.916595in}{1.135070in}}%
\pgfpathlineto{\pgfqpoint{0.916595in}{1.135080in}}%
\pgfpathlineto{\pgfqpoint{0.917703in}{1.136403in}}%
\pgfpathlineto{\pgfqpoint{0.918912in}{1.137502in}}%
\pgfpathlineto{\pgfqpoint{0.919987in}{1.138513in}}%
\pgfpathlineto{\pgfqpoint{0.921363in}{1.139622in}}%
\pgfpathlineto{\pgfqpoint{0.922456in}{1.140575in}}%
\pgfpathlineto{\pgfqpoint{0.922472in}{1.140575in}}%
\pgfpathlineto{\pgfqpoint{0.923445in}{1.141684in}}%
\pgfpathlineto{\pgfqpoint{0.924552in}{1.142832in}}%
\pgfpathlineto{\pgfqpoint{0.925904in}{1.143940in}}%
\pgfpathlineto{\pgfqpoint{0.927012in}{1.144913in}}%
\pgfpathlineto{\pgfqpoint{0.928173in}{1.146022in}}%
\pgfpathlineto{\pgfqpoint{0.929271in}{1.147024in}}%
\pgfpathlineto{\pgfqpoint{0.930379in}{1.148094in}}%
\pgfpathlineto{\pgfqpoint{0.931472in}{1.149202in}}%
\pgfpathlineto{\pgfqpoint{0.932710in}{1.150311in}}%
\pgfpathlineto{\pgfqpoint{0.933820in}{1.151371in}}%
\pgfpathlineto{\pgfqpoint{0.935041in}{1.152480in}}%
\pgfpathlineto{\pgfqpoint{0.936151in}{1.153540in}}%
\pgfpathlineto{\pgfqpoint{0.937361in}{1.154649in}}%
\pgfpathlineto{\pgfqpoint{0.938422in}{1.155447in}}%
\pgfpathlineto{\pgfqpoint{0.939762in}{1.156546in}}%
\pgfpathlineto{\pgfqpoint{0.940849in}{1.157518in}}%
\pgfpathlineto{\pgfqpoint{0.942080in}{1.158627in}}%
\pgfpathlineto{\pgfqpoint{0.943176in}{1.159512in}}%
\pgfpathlineto{\pgfqpoint{0.944283in}{1.160621in}}%
\pgfpathlineto{\pgfqpoint{0.945379in}{1.161778in}}%
\pgfpathlineto{\pgfqpoint{0.945391in}{1.161778in}}%
\pgfpathlineto{\pgfqpoint{0.946554in}{1.162887in}}%
\pgfpathlineto{\pgfqpoint{0.947631in}{1.163830in}}%
\pgfpathlineto{\pgfqpoint{0.948792in}{1.164939in}}%
\pgfpathlineto{\pgfqpoint{0.949898in}{1.166252in}}%
\pgfpathlineto{\pgfqpoint{0.949902in}{1.166252in}}%
\pgfpathlineto{\pgfqpoint{0.951321in}{1.167439in}}%
\pgfpathlineto{\pgfqpoint{0.952659in}{1.168528in}}%
\pgfpathlineto{\pgfqpoint{0.953767in}{1.169481in}}%
\pgfpathlineto{\pgfqpoint{0.954942in}{1.170590in}}%
\pgfpathlineto{\pgfqpoint{0.956052in}{1.171592in}}%
\pgfpathlineto{\pgfqpoint{0.957273in}{1.172701in}}%
\pgfpathlineto{\pgfqpoint{0.958383in}{1.173518in}}%
\pgfpathlineto{\pgfqpoint{0.959844in}{1.174627in}}%
\pgfpathlineto{\pgfqpoint{0.960945in}{1.175628in}}%
\pgfpathlineto{\pgfqpoint{0.962339in}{1.176727in}}%
\pgfpathlineto{\pgfqpoint{0.963441in}{1.177817in}}%
\pgfpathlineto{\pgfqpoint{0.964675in}{1.178896in}}%
\pgfpathlineto{\pgfqpoint{0.965754in}{1.180015in}}%
\pgfpathlineto{\pgfqpoint{0.967125in}{1.181124in}}%
\pgfpathlineto{\pgfqpoint{0.968225in}{1.182155in}}%
\pgfpathlineto{\pgfqpoint{0.969412in}{1.183244in}}%
\pgfpathlineto{\pgfqpoint{0.970517in}{1.184246in}}%
\pgfpathlineto{\pgfqpoint{0.972008in}{1.185355in}}%
\pgfpathlineto{\pgfqpoint{0.973051in}{1.186269in}}%
\pgfpathlineto{\pgfqpoint{0.973083in}{1.186269in}}%
\pgfpathlineto{\pgfqpoint{0.974561in}{1.187378in}}%
\pgfpathlineto{\pgfqpoint{0.975664in}{1.188263in}}%
\pgfpathlineto{\pgfqpoint{0.976925in}{1.189371in}}%
\pgfpathlineto{\pgfqpoint{0.978023in}{1.190432in}}%
\pgfpathlineto{\pgfqpoint{0.979403in}{1.191540in}}%
\pgfpathlineto{\pgfqpoint{0.980473in}{1.192474in}}%
\pgfpathlineto{\pgfqpoint{0.980499in}{1.192474in}}%
\pgfpathlineto{\pgfqpoint{0.981871in}{1.193583in}}%
\pgfpathlineto{\pgfqpoint{0.982970in}{1.194653in}}%
\pgfpathlineto{\pgfqpoint{0.984319in}{1.195762in}}%
\pgfpathlineto{\pgfqpoint{0.985408in}{1.196530in}}%
\pgfpathlineto{\pgfqpoint{0.986813in}{1.197639in}}%
\pgfpathlineto{\pgfqpoint{0.987914in}{1.198689in}}%
\pgfpathlineto{\pgfqpoint{0.989547in}{1.199788in}}%
\pgfpathlineto{\pgfqpoint{0.990657in}{1.200761in}}%
\pgfpathlineto{\pgfqpoint{0.992030in}{1.201870in}}%
\pgfpathlineto{\pgfqpoint{0.993121in}{1.202784in}}%
\pgfpathlineto{\pgfqpoint{0.994299in}{1.203893in}}%
\pgfpathlineto{\pgfqpoint{0.995373in}{1.204856in}}%
\pgfpathlineto{\pgfqpoint{0.996909in}{1.205964in}}%
\pgfpathlineto{\pgfqpoint{0.998010in}{1.206820in}}%
\pgfpathlineto{\pgfqpoint{0.999517in}{1.207929in}}%
\pgfpathlineto{\pgfqpoint{1.000625in}{1.208814in}}%
\pgfpathlineto{\pgfqpoint{1.002158in}{1.209923in}}%
\pgfpathlineto{\pgfqpoint{1.003231in}{1.210779in}}%
\pgfpathlineto{\pgfqpoint{1.004394in}{1.211888in}}%
\pgfpathlineto{\pgfqpoint{1.005502in}{1.212928in}}%
\pgfpathlineto{\pgfqpoint{1.006947in}{1.214037in}}%
\pgfpathlineto{\pgfqpoint{1.008056in}{1.215049in}}%
\pgfpathlineto{\pgfqpoint{1.009418in}{1.216157in}}%
\pgfpathlineto{\pgfqpoint{1.010518in}{1.216828in}}%
\pgfpathlineto{\pgfqpoint{1.012049in}{1.217937in}}%
\pgfpathlineto{\pgfqpoint{1.013152in}{1.218725in}}%
\pgfpathlineto{\pgfqpoint{1.014546in}{1.219834in}}%
\pgfpathlineto{\pgfqpoint{1.015628in}{1.220758in}}%
\pgfpathlineto{\pgfqpoint{1.017135in}{1.221867in}}%
\pgfpathlineto{\pgfqpoint{1.018243in}{1.222839in}}%
\pgfpathlineto{\pgfqpoint{1.019620in}{1.223948in}}%
\pgfpathlineto{\pgfqpoint{1.020714in}{1.224814in}}%
\pgfpathlineto{\pgfqpoint{1.022019in}{1.225922in}}%
\pgfpathlineto{\pgfqpoint{1.023120in}{1.226837in}}%
\pgfpathlineto{\pgfqpoint{1.024686in}{1.227936in}}%
\pgfpathlineto{\pgfqpoint{1.025788in}{1.228831in}}%
\pgfpathlineto{\pgfqpoint{1.027364in}{1.229939in}}%
\pgfpathlineto{\pgfqpoint{1.028474in}{1.230717in}}%
\pgfpathlineto{\pgfqpoint{1.029914in}{1.231826in}}%
\pgfpathlineto{\pgfqpoint{1.030991in}{1.232692in}}%
\pgfpathlineto{\pgfqpoint{1.032578in}{1.233801in}}%
\pgfpathlineto{\pgfqpoint{1.033625in}{1.234530in}}%
\pgfpathlineto{\pgfqpoint{1.034995in}{1.235639in}}%
\pgfpathlineto{\pgfqpoint{1.036103in}{1.236495in}}%
\pgfpathlineto{\pgfqpoint{1.037492in}{1.237604in}}%
\pgfpathlineto{\pgfqpoint{1.038592in}{1.238460in}}%
\pgfpathlineto{\pgfqpoint{1.040354in}{1.239568in}}%
\pgfpathlineto{\pgfqpoint{1.041457in}{1.240531in}}%
\pgfpathlineto{\pgfqpoint{1.042999in}{1.241630in}}%
\pgfpathlineto{\pgfqpoint{1.044104in}{1.242486in}}%
\pgfpathlineto{\pgfqpoint{1.045722in}{1.243595in}}%
\pgfpathlineto{\pgfqpoint{1.046822in}{1.244354in}}%
\pgfpathlineto{\pgfqpoint{1.048321in}{1.245462in}}%
\pgfpathlineto{\pgfqpoint{1.049356in}{1.246289in}}%
\pgfpathlineto{\pgfqpoint{1.050906in}{1.247398in}}%
\pgfpathlineto{\pgfqpoint{1.052006in}{1.248127in}}%
\pgfpathlineto{\pgfqpoint{1.053660in}{1.249236in}}%
\pgfpathlineto{\pgfqpoint{1.054768in}{1.249975in}}%
\pgfpathlineto{\pgfqpoint{1.056443in}{1.251084in}}%
\pgfpathlineto{\pgfqpoint{1.057527in}{1.251872in}}%
\pgfpathlineto{\pgfqpoint{1.057546in}{1.251872in}}%
\pgfpathlineto{\pgfqpoint{1.059186in}{1.252981in}}%
\pgfpathlineto{\pgfqpoint{1.060271in}{1.253817in}}%
\pgfpathlineto{\pgfqpoint{1.061850in}{1.254926in}}%
\pgfpathlineto{\pgfqpoint{1.062949in}{1.255626in}}%
\pgfpathlineto{\pgfqpoint{1.064615in}{1.256735in}}%
\pgfpathlineto{\pgfqpoint{1.065717in}{1.257474in}}%
\pgfpathlineto{\pgfqpoint{1.067246in}{1.258583in}}%
\pgfpathlineto{\pgfqpoint{1.068356in}{1.259205in}}%
\pgfpathlineto{\pgfqpoint{1.070180in}{1.260314in}}%
\pgfpathlineto{\pgfqpoint{1.071281in}{1.261005in}}%
\pgfpathlineto{\pgfqpoint{1.073263in}{1.262114in}}%
\pgfpathlineto{\pgfqpoint{1.074354in}{1.262804in}}%
\pgfpathlineto{\pgfqpoint{1.075611in}{1.263913in}}%
\pgfpathlineto{\pgfqpoint{1.076644in}{1.264672in}}%
\pgfpathlineto{\pgfqpoint{1.076693in}{1.264672in}}%
\pgfpathlineto{\pgfqpoint{1.078089in}{1.265780in}}%
\pgfpathlineto{\pgfqpoint{1.079199in}{1.266520in}}%
\pgfpathlineto{\pgfqpoint{1.080678in}{1.267628in}}%
\pgfpathlineto{\pgfqpoint{1.081765in}{1.268562in}}%
\pgfpathlineto{\pgfqpoint{1.083554in}{1.269661in}}%
\pgfpathlineto{\pgfqpoint{1.084664in}{1.270507in}}%
\pgfpathlineto{\pgfqpoint{1.086190in}{1.271616in}}%
\pgfpathlineto{\pgfqpoint{1.087277in}{1.272443in}}%
\pgfpathlineto{\pgfqpoint{1.088836in}{1.273552in}}%
\pgfpathlineto{\pgfqpoint{1.089943in}{1.274330in}}%
\pgfpathlineto{\pgfqpoint{1.091565in}{1.275429in}}%
\pgfpathlineto{\pgfqpoint{1.092656in}{1.276255in}}%
\pgfpathlineto{\pgfqpoint{1.094632in}{1.277364in}}%
\pgfpathlineto{\pgfqpoint{1.095665in}{1.278035in}}%
\pgfpathlineto{\pgfqpoint{1.097212in}{1.279144in}}%
\pgfpathlineto{\pgfqpoint{1.098310in}{1.279961in}}%
\pgfpathlineto{\pgfqpoint{1.100041in}{1.281060in}}%
\pgfpathlineto{\pgfqpoint{1.101151in}{1.281780in}}%
\pgfpathlineto{\pgfqpoint{1.102445in}{1.282889in}}%
\pgfpathlineto{\pgfqpoint{1.103534in}{1.283813in}}%
\pgfpathlineto{\pgfqpoint{1.105016in}{1.284921in}}%
\pgfpathlineto{\pgfqpoint{1.106105in}{1.285593in}}%
\pgfpathlineto{\pgfqpoint{1.107573in}{1.286701in}}%
\pgfpathlineto{\pgfqpoint{1.108576in}{1.287324in}}%
\pgfpathlineto{\pgfqpoint{1.110714in}{1.288433in}}%
\pgfpathlineto{\pgfqpoint{1.111819in}{1.289045in}}%
\pgfpathlineto{\pgfqpoint{1.113783in}{1.290154in}}%
\pgfpathlineto{\pgfqpoint{1.114888in}{1.290854in}}%
\pgfpathlineto{\pgfqpoint{1.116652in}{1.291963in}}%
\pgfpathlineto{\pgfqpoint{1.117759in}{1.292732in}}%
\pgfpathlineto{\pgfqpoint{1.119511in}{1.293840in}}%
\pgfpathlineto{\pgfqpoint{1.120593in}{1.294560in}}%
\pgfpathlineto{\pgfqpoint{1.122408in}{1.295669in}}%
\pgfpathlineto{\pgfqpoint{1.123502in}{1.296389in}}%
\pgfpathlineto{\pgfqpoint{1.125400in}{1.297497in}}%
\pgfpathlineto{\pgfqpoint{1.126503in}{1.298110in}}%
\pgfpathlineto{\pgfqpoint{1.128453in}{1.299219in}}%
\pgfpathlineto{\pgfqpoint{1.129535in}{1.299948in}}%
\pgfpathlineto{\pgfqpoint{1.129551in}{1.299948in}}%
\pgfpathlineto{\pgfqpoint{1.131227in}{1.301057in}}%
\pgfpathlineto{\pgfqpoint{1.132313in}{1.301621in}}%
\pgfpathlineto{\pgfqpoint{1.134198in}{1.302720in}}%
\pgfpathlineto{\pgfqpoint{1.135291in}{1.303411in}}%
\pgfpathlineto{\pgfqpoint{1.137241in}{1.304510in}}%
\pgfpathlineto{\pgfqpoint{1.138337in}{1.305123in}}%
\pgfpathlineto{\pgfqpoint{1.139796in}{1.306232in}}%
\pgfpathlineto{\pgfqpoint{1.140899in}{1.306883in}}%
\pgfpathlineto{\pgfqpoint{1.142809in}{1.307992in}}%
\pgfpathlineto{\pgfqpoint{1.143919in}{1.308653in}}%
\pgfpathlineto{\pgfqpoint{1.145738in}{1.309752in}}%
\pgfpathlineto{\pgfqpoint{1.146846in}{1.310433in}}%
\pgfpathlineto{\pgfqpoint{1.148637in}{1.311542in}}%
\pgfpathlineto{\pgfqpoint{1.149747in}{1.312252in}}%
\pgfpathlineto{\pgfqpoint{1.151753in}{1.313361in}}%
\pgfpathlineto{\pgfqpoint{1.152849in}{1.313886in}}%
\pgfpathlineto{\pgfqpoint{1.154745in}{1.314995in}}%
\pgfpathlineto{\pgfqpoint{1.155832in}{1.315540in}}%
\pgfpathlineto{\pgfqpoint{1.157756in}{1.316648in}}%
\pgfpathlineto{\pgfqpoint{1.158822in}{1.317271in}}%
\pgfpathlineto{\pgfqpoint{1.160511in}{1.318380in}}%
\pgfpathlineto{\pgfqpoint{1.161618in}{1.319090in}}%
\pgfpathlineto{\pgfqpoint{1.163531in}{1.320198in}}%
\pgfpathlineto{\pgfqpoint{1.164606in}{1.320928in}}%
\pgfpathlineto{\pgfqpoint{1.166181in}{1.322027in}}%
\pgfpathlineto{\pgfqpoint{1.167282in}{1.322562in}}%
\pgfpathlineto{\pgfqpoint{1.169231in}{1.323671in}}%
\pgfpathlineto{\pgfqpoint{1.170327in}{1.324283in}}%
\pgfpathlineto{\pgfqpoint{1.170341in}{1.324283in}}%
\pgfpathlineto{\pgfqpoint{1.172286in}{1.325392in}}%
\pgfpathlineto{\pgfqpoint{1.173396in}{1.325917in}}%
\pgfpathlineto{\pgfqpoint{1.175018in}{1.327026in}}%
\pgfpathlineto{\pgfqpoint{1.176118in}{1.327590in}}%
\pgfpathlineto{\pgfqpoint{1.178217in}{1.328699in}}%
\pgfpathlineto{\pgfqpoint{1.179327in}{1.329448in}}%
\pgfpathlineto{\pgfqpoint{1.181079in}{1.330557in}}%
\pgfpathlineto{\pgfqpoint{1.182170in}{1.331218in}}%
\pgfpathlineto{\pgfqpoint{1.183974in}{1.332327in}}%
\pgfpathlineto{\pgfqpoint{1.185083in}{1.333076in}}%
\pgfpathlineto{\pgfqpoint{1.187375in}{1.334185in}}%
\pgfpathlineto{\pgfqpoint{1.188485in}{1.334710in}}%
\pgfpathlineto{\pgfqpoint{1.190612in}{1.335819in}}%
\pgfpathlineto{\pgfqpoint{1.191719in}{1.336626in}}%
\pgfpathlineto{\pgfqpoint{1.193418in}{1.337735in}}%
\pgfpathlineto{\pgfqpoint{1.194523in}{1.338445in}}%
\pgfpathlineto{\pgfqpoint{1.196114in}{1.339553in}}%
\pgfpathlineto{\pgfqpoint{1.197213in}{1.340137in}}%
\pgfpathlineto{\pgfqpoint{1.199067in}{1.341246in}}%
\pgfpathlineto{\pgfqpoint{1.200151in}{1.341907in}}%
\pgfpathlineto{\pgfqpoint{1.202131in}{1.343016in}}%
\pgfpathlineto{\pgfqpoint{1.203237in}{1.343619in}}%
\pgfpathlineto{\pgfqpoint{1.205144in}{1.344728in}}%
\pgfpathlineto{\pgfqpoint{1.206238in}{1.345263in}}%
\pgfpathlineto{\pgfqpoint{1.208041in}{1.346372in}}%
\pgfpathlineto{\pgfqpoint{1.209149in}{1.347033in}}%
\pgfpathlineto{\pgfqpoint{1.211357in}{1.348142in}}%
\pgfpathlineto{\pgfqpoint{1.212697in}{1.349017in}}%
\pgfpathlineto{\pgfqpoint{1.214654in}{1.350126in}}%
\pgfpathlineto{\pgfqpoint{1.215757in}{1.350777in}}%
\pgfpathlineto{\pgfqpoint{1.217576in}{1.351886in}}%
\pgfpathlineto{\pgfqpoint{1.218679in}{1.352411in}}%
\pgfpathlineto{\pgfqpoint{1.221029in}{1.353520in}}%
\pgfpathlineto{\pgfqpoint{1.222123in}{1.354182in}}%
\pgfpathlineto{\pgfqpoint{1.224149in}{1.355290in}}%
\pgfpathlineto{\pgfqpoint{1.225254in}{1.355874in}}%
\pgfpathlineto{\pgfqpoint{1.227532in}{1.356983in}}%
\pgfpathlineto{\pgfqpoint{1.228628in}{1.357508in}}%
\pgfpathlineto{\pgfqpoint{1.230457in}{1.358617in}}%
\pgfpathlineto{\pgfqpoint{1.231546in}{1.359181in}}%
\pgfpathlineto{\pgfqpoint{1.233645in}{1.360290in}}%
\pgfpathlineto{\pgfqpoint{1.234754in}{1.360844in}}%
\pgfpathlineto{\pgfqpoint{1.237070in}{1.361953in}}%
\pgfpathlineto{\pgfqpoint{1.238165in}{1.362400in}}%
\pgfpathlineto{\pgfqpoint{1.240408in}{1.363509in}}%
\pgfpathlineto{\pgfqpoint{1.241509in}{1.364200in}}%
\pgfpathlineto{\pgfqpoint{1.241516in}{1.364200in}}%
\pgfpathlineto{\pgfqpoint{1.243468in}{1.365308in}}%
\pgfpathlineto{\pgfqpoint{1.244569in}{1.365863in}}%
\pgfpathlineto{\pgfqpoint{1.246867in}{1.366972in}}%
\pgfpathlineto{\pgfqpoint{1.247977in}{1.367623in}}%
\pgfpathlineto{\pgfqpoint{1.250232in}{1.368732in}}%
\pgfpathlineto{\pgfqpoint{1.251330in}{1.369335in}}%
\pgfpathlineto{\pgfqpoint{1.253291in}{1.370444in}}%
\pgfpathlineto{\pgfqpoint{1.254352in}{1.371115in}}%
\pgfpathlineto{\pgfqpoint{1.256854in}{1.372224in}}%
\pgfpathlineto{\pgfqpoint{1.257950in}{1.372778in}}%
\pgfpathlineto{\pgfqpoint{1.260241in}{1.373887in}}%
\pgfpathlineto{\pgfqpoint{1.261340in}{1.374509in}}%
\pgfpathlineto{\pgfqpoint{1.263629in}{1.375618in}}%
\pgfpathlineto{\pgfqpoint{1.264695in}{1.376250in}}%
\pgfpathlineto{\pgfqpoint{1.264702in}{1.376250in}}%
\pgfpathlineto{\pgfqpoint{1.267268in}{1.377359in}}%
\pgfpathlineto{\pgfqpoint{1.268378in}{1.377904in}}%
\pgfpathlineto{\pgfqpoint{1.270714in}{1.379013in}}%
\pgfpathlineto{\pgfqpoint{1.271817in}{1.379635in}}%
\pgfpathlineto{\pgfqpoint{1.273795in}{1.380744in}}%
\pgfpathlineto{\pgfqpoint{1.274902in}{1.381337in}}%
\pgfpathlineto{\pgfqpoint{1.277005in}{1.382446in}}%
\pgfpathlineto{\pgfqpoint{1.278108in}{1.383020in}}%
\pgfpathlineto{\pgfqpoint{1.280845in}{1.384129in}}%
\pgfpathlineto{\pgfqpoint{1.281920in}{1.384771in}}%
\pgfpathlineto{\pgfqpoint{1.284651in}{1.385879in}}%
\pgfpathlineto{\pgfqpoint{1.285717in}{1.386366in}}%
\pgfpathlineto{\pgfqpoint{1.287957in}{1.387474in}}%
\pgfpathlineto{\pgfqpoint{1.289000in}{1.387863in}}%
\pgfpathlineto{\pgfqpoint{1.291555in}{1.388972in}}%
\pgfpathlineto{\pgfqpoint{1.292583in}{1.389459in}}%
\pgfpathlineto{\pgfqpoint{1.292648in}{1.389459in}}%
\pgfpathlineto{\pgfqpoint{1.294726in}{1.390567in}}%
\pgfpathlineto{\pgfqpoint{1.295824in}{1.391093in}}%
\pgfpathlineto{\pgfqpoint{1.298339in}{1.392201in}}%
\pgfpathlineto{\pgfqpoint{1.299407in}{1.392678in}}%
\pgfpathlineto{\pgfqpoint{1.301857in}{1.393787in}}%
\pgfpathlineto{\pgfqpoint{1.302960in}{1.394244in}}%
\pgfpathlineto{\pgfqpoint{1.305543in}{1.395353in}}%
\pgfpathlineto{\pgfqpoint{1.306613in}{1.395751in}}%
\pgfpathlineto{\pgfqpoint{1.308993in}{1.396860in}}%
\pgfpathlineto{\pgfqpoint{1.310003in}{1.397288in}}%
\pgfpathlineto{\pgfqpoint{1.312681in}{1.398397in}}%
\pgfpathlineto{\pgfqpoint{1.313768in}{1.398942in}}%
\pgfpathlineto{\pgfqpoint{1.316076in}{1.400050in}}%
\pgfpathlineto{\pgfqpoint{1.317183in}{1.400614in}}%
\pgfpathlineto{\pgfqpoint{1.319869in}{1.401723in}}%
\pgfpathlineto{\pgfqpoint{1.320906in}{1.402093in}}%
\pgfpathlineto{\pgfqpoint{1.323919in}{1.403202in}}%
\pgfpathlineto{\pgfqpoint{1.325013in}{1.403824in}}%
\pgfpathlineto{\pgfqpoint{1.327928in}{1.404933in}}%
\pgfpathlineto{\pgfqpoint{1.328985in}{1.405390in}}%
\pgfpathlineto{\pgfqpoint{1.331991in}{1.406499in}}%
\pgfpathlineto{\pgfqpoint{1.333096in}{1.406898in}}%
\pgfpathlineto{\pgfqpoint{1.335446in}{1.407997in}}%
\pgfpathlineto{\pgfqpoint{1.336556in}{1.408493in}}%
\pgfpathlineto{\pgfqpoint{1.339369in}{1.409601in}}%
\pgfpathlineto{\pgfqpoint{1.340472in}{1.410097in}}%
\pgfpathlineto{\pgfqpoint{1.343241in}{1.411206in}}%
\pgfpathlineto{\pgfqpoint{1.344281in}{1.411508in}}%
\pgfpathlineto{\pgfqpoint{1.347117in}{1.412617in}}%
\pgfpathlineto{\pgfqpoint{1.348224in}{1.413045in}}%
\pgfpathlineto{\pgfqpoint{1.351282in}{1.414153in}}%
\pgfpathlineto{\pgfqpoint{1.352364in}{1.414640in}}%
\pgfpathlineto{\pgfqpoint{1.354423in}{1.415748in}}%
\pgfpathlineto{\pgfqpoint{1.355512in}{1.416108in}}%
\pgfpathlineto{\pgfqpoint{1.358315in}{1.417217in}}%
\pgfpathlineto{\pgfqpoint{1.359369in}{1.417664in}}%
\pgfpathlineto{\pgfqpoint{1.359414in}{1.417664in}}%
\pgfpathlineto{\pgfqpoint{1.362266in}{1.418764in}}%
\pgfpathlineto{\pgfqpoint{1.363364in}{1.419230in}}%
\pgfpathlineto{\pgfqpoint{1.365919in}{1.420339in}}%
\pgfpathlineto{\pgfqpoint{1.367020in}{1.420884in}}%
\pgfpathlineto{\pgfqpoint{1.369968in}{1.421993in}}%
\pgfpathlineto{\pgfqpoint{1.371073in}{1.422352in}}%
\pgfpathlineto{\pgfqpoint{1.373486in}{1.423461in}}%
\pgfpathlineto{\pgfqpoint{1.374556in}{1.423860in}}%
\pgfpathlineto{\pgfqpoint{1.377213in}{1.424969in}}%
\pgfpathlineto{\pgfqpoint{1.378309in}{1.425523in}}%
\pgfpathlineto{\pgfqpoint{1.381189in}{1.426632in}}%
\pgfpathlineto{\pgfqpoint{1.382288in}{1.427177in}}%
\pgfpathlineto{\pgfqpoint{1.384924in}{1.428285in}}%
\pgfpathlineto{\pgfqpoint{1.386034in}{1.428743in}}%
\pgfpathlineto{\pgfqpoint{1.388605in}{1.429851in}}%
\pgfpathlineto{\pgfqpoint{1.389708in}{1.430425in}}%
\pgfpathlineto{\pgfqpoint{1.392735in}{1.431534in}}%
\pgfpathlineto{\pgfqpoint{1.393819in}{1.432001in}}%
\pgfpathlineto{\pgfqpoint{1.396325in}{1.433110in}}%
\pgfpathlineto{\pgfqpoint{1.397281in}{1.433528in}}%
\pgfpathlineto{\pgfqpoint{1.400222in}{1.434637in}}%
\pgfpathlineto{\pgfqpoint{1.401274in}{1.435055in}}%
\pgfpathlineto{\pgfqpoint{1.404573in}{1.436164in}}%
\pgfpathlineto{\pgfqpoint{1.405676in}{1.436660in}}%
\pgfpathlineto{\pgfqpoint{1.408664in}{1.437759in}}%
\pgfpathlineto{\pgfqpoint{1.409743in}{1.438313in}}%
\pgfpathlineto{\pgfqpoint{1.412961in}{1.439422in}}%
\pgfpathlineto{\pgfqpoint{1.414015in}{1.439899in}}%
\pgfpathlineto{\pgfqpoint{1.414071in}{1.439899in}}%
\pgfpathlineto{\pgfqpoint{1.416609in}{1.441007in}}%
\pgfpathlineto{\pgfqpoint{1.417719in}{1.441484in}}%
\pgfpathlineto{\pgfqpoint{1.420981in}{1.442593in}}%
\pgfpathlineto{\pgfqpoint{1.422038in}{1.443001in}}%
\pgfpathlineto{\pgfqpoint{1.425493in}{1.444110in}}%
\pgfpathlineto{\pgfqpoint{1.426600in}{1.444596in}}%
\pgfpathlineto{\pgfqpoint{1.429916in}{1.445705in}}%
\pgfpathlineto{\pgfqpoint{1.430991in}{1.446123in}}%
\pgfpathlineto{\pgfqpoint{1.433839in}{1.447232in}}%
\pgfpathlineto{\pgfqpoint{1.434949in}{1.447582in}}%
\pgfpathlineto{\pgfqpoint{1.438101in}{1.448691in}}%
\pgfpathlineto{\pgfqpoint{1.439195in}{1.449041in}}%
\pgfpathlineto{\pgfqpoint{1.442248in}{1.450150in}}%
\pgfpathlineto{\pgfqpoint{1.443322in}{1.450471in}}%
\pgfpathlineto{\pgfqpoint{1.446671in}{1.451580in}}%
\pgfpathlineto{\pgfqpoint{1.447778in}{1.451920in}}%
\pgfpathlineto{\pgfqpoint{1.450884in}{1.453029in}}%
\pgfpathlineto{\pgfqpoint{1.451990in}{1.453457in}}%
\pgfpathlineto{\pgfqpoint{1.455559in}{1.454566in}}%
\pgfpathlineto{\pgfqpoint{1.456634in}{1.455081in}}%
\pgfpathlineto{\pgfqpoint{1.460070in}{1.456190in}}%
\pgfpathlineto{\pgfqpoint{1.461071in}{1.456530in}}%
\pgfpathlineto{\pgfqpoint{1.461155in}{1.456530in}}%
\pgfpathlineto{\pgfqpoint{1.464991in}{1.457629in}}%
\pgfpathlineto{\pgfqpoint{1.466071in}{1.458009in}}%
\pgfpathlineto{\pgfqpoint{1.469929in}{1.459117in}}%
\pgfpathlineto{\pgfqpoint{1.471022in}{1.459400in}}%
\pgfpathlineto{\pgfqpoint{1.475029in}{1.460508in}}%
\pgfpathlineto{\pgfqpoint{1.476118in}{1.460897in}}%
\pgfpathlineto{\pgfqpoint{1.479243in}{1.462006in}}%
\pgfpathlineto{\pgfqpoint{1.480336in}{1.462434in}}%
\pgfpathlineto{\pgfqpoint{1.483919in}{1.463543in}}%
\pgfpathlineto{\pgfqpoint{1.485024in}{1.463864in}}%
\pgfpathlineto{\pgfqpoint{1.488347in}{1.464973in}}%
\pgfpathlineto{\pgfqpoint{1.489578in}{1.465371in}}%
\pgfpathlineto{\pgfqpoint{1.492968in}{1.466480in}}%
\pgfpathlineto{\pgfqpoint{1.494066in}{1.466782in}}%
\pgfpathlineto{\pgfqpoint{1.497698in}{1.467890in}}%
\pgfpathlineto{\pgfqpoint{1.498801in}{1.468241in}}%
\pgfpathlineto{\pgfqpoint{1.502065in}{1.469349in}}%
\pgfpathlineto{\pgfqpoint{1.503068in}{1.469680in}}%
\pgfpathlineto{\pgfqpoint{1.507121in}{1.470789in}}%
\pgfpathlineto{\pgfqpoint{1.508227in}{1.471139in}}%
\pgfpathlineto{\pgfqpoint{1.512540in}{1.472248in}}%
\pgfpathlineto{\pgfqpoint{1.513590in}{1.472676in}}%
\pgfpathlineto{\pgfqpoint{1.517077in}{1.473785in}}%
\pgfpathlineto{\pgfqpoint{1.518078in}{1.474115in}}%
\pgfpathlineto{\pgfqpoint{1.521591in}{1.475214in}}%
\pgfpathlineto{\pgfqpoint{1.522687in}{1.475467in}}%
\pgfpathlineto{\pgfqpoint{1.527148in}{1.476576in}}%
\pgfpathlineto{\pgfqpoint{1.528241in}{1.476887in}}%
\pgfpathlineto{\pgfqpoint{1.528257in}{1.476887in}}%
\pgfpathlineto{\pgfqpoint{1.531678in}{1.477996in}}%
\pgfpathlineto{\pgfqpoint{1.532771in}{1.478259in}}%
\pgfpathlineto{\pgfqpoint{1.536599in}{1.479367in}}%
\pgfpathlineto{\pgfqpoint{1.537688in}{1.479698in}}%
\pgfpathlineto{\pgfqpoint{1.537704in}{1.479698in}}%
\pgfpathlineto{\pgfqpoint{1.542099in}{1.480807in}}%
\pgfpathlineto{\pgfqpoint{1.543186in}{1.481128in}}%
\pgfpathlineto{\pgfqpoint{1.546974in}{1.482237in}}%
\pgfpathlineto{\pgfqpoint{1.548058in}{1.482596in}}%
\pgfpathlineto{\pgfqpoint{1.552116in}{1.483705in}}%
\pgfpathlineto{\pgfqpoint{1.553184in}{1.484016in}}%
\pgfpathlineto{\pgfqpoint{1.557542in}{1.485125in}}%
\pgfpathlineto{\pgfqpoint{1.558617in}{1.485495in}}%
\pgfpathlineto{\pgfqpoint{1.562879in}{1.486604in}}%
\pgfpathlineto{\pgfqpoint{1.563954in}{1.486934in}}%
\pgfpathlineto{\pgfqpoint{1.563963in}{1.486934in}}%
\pgfpathlineto{\pgfqpoint{1.568191in}{1.488043in}}%
\pgfpathlineto{\pgfqpoint{1.569261in}{1.488364in}}%
\pgfpathlineto{\pgfqpoint{1.569299in}{1.488364in}}%
\pgfpathlineto{\pgfqpoint{1.573922in}{1.489473in}}%
\pgfpathlineto{\pgfqpoint{1.574901in}{1.489726in}}%
\pgfpathlineto{\pgfqpoint{1.575011in}{1.489726in}}%
\pgfpathlineto{\pgfqpoint{1.580083in}{1.490835in}}%
\pgfpathlineto{\pgfqpoint{1.581172in}{1.491117in}}%
\pgfpathlineto{\pgfqpoint{1.585595in}{1.492225in}}%
\pgfpathlineto{\pgfqpoint{1.586705in}{1.492517in}}%
\pgfpathlineto{\pgfqpoint{1.591982in}{1.493626in}}%
\pgfpathlineto{\pgfqpoint{1.593068in}{1.493889in}}%
\pgfpathlineto{\pgfqpoint{1.597343in}{1.494997in}}%
\pgfpathlineto{\pgfqpoint{1.598427in}{1.495221in}}%
\pgfpathlineto{\pgfqpoint{1.602475in}{1.496330in}}%
\pgfpathlineto{\pgfqpoint{1.603576in}{1.496651in}}%
\pgfpathlineto{\pgfqpoint{1.608346in}{1.497750in}}%
\pgfpathlineto{\pgfqpoint{1.609449in}{1.498051in}}%
\pgfpathlineto{\pgfqpoint{1.614081in}{1.499160in}}%
\pgfpathlineto{\pgfqpoint{1.615191in}{1.499374in}}%
\pgfpathlineto{\pgfqpoint{1.619589in}{1.500483in}}%
\pgfpathlineto{\pgfqpoint{1.620670in}{1.500755in}}%
\pgfpathlineto{\pgfqpoint{1.625103in}{1.501864in}}%
\pgfpathlineto{\pgfqpoint{1.626189in}{1.502117in}}%
\pgfpathlineto{\pgfqpoint{1.631010in}{1.503216in}}%
\pgfpathlineto{\pgfqpoint{1.632037in}{1.503459in}}%
\pgfpathlineto{\pgfqpoint{1.632106in}{1.503459in}}%
\pgfpathlineto{\pgfqpoint{1.636909in}{1.504568in}}%
\pgfpathlineto{\pgfqpoint{1.637993in}{1.504918in}}%
\pgfpathlineto{\pgfqpoint{1.642432in}{1.506027in}}%
\pgfpathlineto{\pgfqpoint{1.643535in}{1.506289in}}%
\pgfpathlineto{\pgfqpoint{1.648338in}{1.507398in}}%
\pgfpathlineto{\pgfqpoint{1.649424in}{1.507544in}}%
\pgfpathlineto{\pgfqpoint{1.654499in}{1.508653in}}%
\pgfpathlineto{\pgfqpoint{1.655590in}{1.508925in}}%
\pgfpathlineto{\pgfqpoint{1.660681in}{1.510034in}}%
\pgfpathlineto{\pgfqpoint{1.661716in}{1.510277in}}%
\pgfpathlineto{\pgfqpoint{1.661754in}{1.510277in}}%
\pgfpathlineto{\pgfqpoint{1.666374in}{1.511386in}}%
\pgfpathlineto{\pgfqpoint{1.667356in}{1.511590in}}%
\pgfpathlineto{\pgfqpoint{1.673052in}{1.512699in}}%
\pgfpathlineto{\pgfqpoint{1.673834in}{1.512874in}}%
\pgfpathlineto{\pgfqpoint{1.674122in}{1.512874in}}%
\pgfpathlineto{\pgfqpoint{1.679023in}{1.513983in}}%
\pgfpathlineto{\pgfqpoint{1.680121in}{1.514236in}}%
\pgfpathlineto{\pgfqpoint{1.685221in}{1.515345in}}%
\pgfpathlineto{\pgfqpoint{1.686280in}{1.515559in}}%
\pgfpathlineto{\pgfqpoint{1.691566in}{1.516667in}}%
\pgfpathlineto{\pgfqpoint{1.692566in}{1.516833in}}%
\pgfpathlineto{\pgfqpoint{1.698102in}{1.517941in}}%
\pgfpathlineto{\pgfqpoint{1.699193in}{1.518087in}}%
\pgfpathlineto{\pgfqpoint{1.704347in}{1.519186in}}%
\pgfpathlineto{\pgfqpoint{1.705373in}{1.519391in}}%
\pgfpathlineto{\pgfqpoint{1.712064in}{1.520499in}}%
\pgfpathlineto{\pgfqpoint{1.713093in}{1.520713in}}%
\pgfpathlineto{\pgfqpoint{1.719731in}{1.521822in}}%
\pgfpathlineto{\pgfqpoint{1.720804in}{1.522046in}}%
\pgfpathlineto{\pgfqpoint{1.727128in}{1.523155in}}%
\pgfpathlineto{\pgfqpoint{1.728079in}{1.523330in}}%
\pgfpathlineto{\pgfqpoint{1.728200in}{1.523330in}}%
\pgfpathlineto{\pgfqpoint{1.735860in}{1.524439in}}%
\pgfpathlineto{\pgfqpoint{1.736874in}{1.524614in}}%
\pgfpathlineto{\pgfqpoint{1.736919in}{1.524614in}}%
\pgfpathlineto{\pgfqpoint{1.743340in}{1.525722in}}%
\pgfpathlineto{\pgfqpoint{1.744415in}{1.525966in}}%
\pgfpathlineto{\pgfqpoint{1.750162in}{1.527074in}}%
\pgfpathlineto{\pgfqpoint{1.751212in}{1.527269in}}%
\pgfpathlineto{\pgfqpoint{1.759330in}{1.528378in}}%
\pgfpathlineto{\pgfqpoint{1.760300in}{1.528543in}}%
\pgfpathlineto{\pgfqpoint{1.760393in}{1.528543in}}%
\pgfpathlineto{\pgfqpoint{1.767715in}{1.529652in}}%
\pgfpathlineto{\pgfqpoint{1.768711in}{1.529905in}}%
\pgfpathlineto{\pgfqpoint{1.776091in}{1.531004in}}%
\pgfpathlineto{\pgfqpoint{1.777155in}{1.531188in}}%
\pgfpathlineto{\pgfqpoint{1.777192in}{1.531188in}}%
\pgfpathlineto{\pgfqpoint{1.785352in}{1.532297in}}%
\pgfpathlineto{\pgfqpoint{1.786457in}{1.532531in}}%
\pgfpathlineto{\pgfqpoint{1.792949in}{1.533639in}}%
\pgfpathlineto{\pgfqpoint{1.793998in}{1.533863in}}%
\pgfpathlineto{\pgfqpoint{1.802712in}{1.534972in}}%
\pgfpathlineto{\pgfqpoint{1.803663in}{1.535108in}}%
\pgfpathlineto{\pgfqpoint{1.803728in}{1.535108in}}%
\pgfpathlineto{\pgfqpoint{1.811614in}{1.536217in}}%
\pgfpathlineto{\pgfqpoint{1.812435in}{1.536334in}}%
\pgfpathlineto{\pgfqpoint{1.812668in}{1.536334in}}%
\pgfpathlineto{\pgfqpoint{1.820497in}{1.537442in}}%
\pgfpathlineto{\pgfqpoint{1.821600in}{1.537598in}}%
\pgfpathlineto{\pgfqpoint{1.830495in}{1.538707in}}%
\pgfpathlineto{\pgfqpoint{1.831563in}{1.538911in}}%
\pgfpathlineto{\pgfqpoint{1.840160in}{1.540020in}}%
\pgfpathlineto{\pgfqpoint{1.841231in}{1.540175in}}%
\pgfpathlineto{\pgfqpoint{1.841254in}{1.540175in}}%
\pgfpathlineto{\pgfqpoint{1.851040in}{1.541284in}}%
\pgfpathlineto{\pgfqpoint{1.852087in}{1.541372in}}%
\pgfpathlineto{\pgfqpoint{1.852101in}{1.541372in}}%
\pgfpathlineto{\pgfqpoint{1.861938in}{1.542481in}}%
\pgfpathlineto{\pgfqpoint{1.863013in}{1.542636in}}%
\pgfpathlineto{\pgfqpoint{1.873516in}{1.543745in}}%
\pgfpathlineto{\pgfqpoint{1.874312in}{1.543852in}}%
\pgfpathlineto{\pgfqpoint{1.874607in}{1.543852in}}%
\pgfpathlineto{\pgfqpoint{1.885224in}{1.544961in}}%
\pgfpathlineto{\pgfqpoint{1.886325in}{1.545116in}}%
\pgfpathlineto{\pgfqpoint{1.899150in}{1.546225in}}%
\pgfpathlineto{\pgfqpoint{1.900225in}{1.546361in}}%
\pgfpathlineto{\pgfqpoint{1.912191in}{1.547460in}}%
\pgfpathlineto{\pgfqpoint{1.913217in}{1.547616in}}%
\pgfpathlineto{\pgfqpoint{1.928104in}{1.548725in}}%
\pgfpathlineto{\pgfqpoint{1.929213in}{1.548842in}}%
\pgfpathlineto{\pgfqpoint{1.944886in}{1.549950in}}%
\pgfpathlineto{\pgfqpoint{1.945936in}{1.550057in}}%
\pgfpathlineto{\pgfqpoint{1.963612in}{1.551166in}}%
\pgfpathlineto{\pgfqpoint{1.964508in}{1.551234in}}%
\pgfpathlineto{\pgfqpoint{1.964703in}{1.551234in}}%
\pgfpathlineto{\pgfqpoint{1.983126in}{1.551944in}}%
\pgfpathlineto{\pgfqpoint{1.983126in}{1.551944in}}%
\pgfusepath{stroke}%
\end{pgfscope}%
\begin{pgfscope}%
\pgfsetrectcap%
\pgfsetmiterjoin%
\pgfsetlinewidth{0.803000pt}%
\definecolor{currentstroke}{rgb}{0.000000,0.000000,0.000000}%
\pgfsetstrokecolor{currentstroke}%
\pgfsetdash{}{0pt}%
\pgfpathmoveto{\pgfqpoint{0.503581in}{0.449444in}}%
\pgfpathlineto{\pgfqpoint{0.503581in}{1.604444in}}%
\pgfusepath{stroke}%
\end{pgfscope}%
\begin{pgfscope}%
\pgfsetrectcap%
\pgfsetmiterjoin%
\pgfsetlinewidth{0.803000pt}%
\definecolor{currentstroke}{rgb}{0.000000,0.000000,0.000000}%
\pgfsetstrokecolor{currentstroke}%
\pgfsetdash{}{0pt}%
\pgfpathmoveto{\pgfqpoint{2.053581in}{0.449444in}}%
\pgfpathlineto{\pgfqpoint{2.053581in}{1.604444in}}%
\pgfusepath{stroke}%
\end{pgfscope}%
\begin{pgfscope}%
\pgfsetrectcap%
\pgfsetmiterjoin%
\pgfsetlinewidth{0.803000pt}%
\definecolor{currentstroke}{rgb}{0.000000,0.000000,0.000000}%
\pgfsetstrokecolor{currentstroke}%
\pgfsetdash{}{0pt}%
\pgfpathmoveto{\pgfqpoint{0.503581in}{0.449444in}}%
\pgfpathlineto{\pgfqpoint{2.053581in}{0.449444in}}%
\pgfusepath{stroke}%
\end{pgfscope}%
\begin{pgfscope}%
\pgfsetrectcap%
\pgfsetmiterjoin%
\pgfsetlinewidth{0.803000pt}%
\definecolor{currentstroke}{rgb}{0.000000,0.000000,0.000000}%
\pgfsetstrokecolor{currentstroke}%
\pgfsetdash{}{0pt}%
\pgfpathmoveto{\pgfqpoint{0.503581in}{1.604444in}}%
\pgfpathlineto{\pgfqpoint{2.053581in}{1.604444in}}%
\pgfusepath{stroke}%
\end{pgfscope}%
\begin{pgfscope}%
\pgfsetbuttcap%
\pgfsetmiterjoin%
\definecolor{currentfill}{rgb}{1.000000,1.000000,1.000000}%
\pgfsetfillcolor{currentfill}%
\pgfsetfillopacity{0.800000}%
\pgfsetlinewidth{1.003750pt}%
\definecolor{currentstroke}{rgb}{0.800000,0.800000,0.800000}%
\pgfsetstrokecolor{currentstroke}%
\pgfsetstrokeopacity{0.800000}%
\pgfsetdash{}{0pt}%
\pgfpathmoveto{\pgfqpoint{0.782747in}{0.518889in}}%
\pgfpathlineto{\pgfqpoint{1.956358in}{0.518889in}}%
\pgfpathquadraticcurveto{\pgfqpoint{1.984136in}{0.518889in}}{\pgfqpoint{1.984136in}{0.546666in}}%
\pgfpathlineto{\pgfqpoint{1.984136in}{0.726388in}}%
\pgfpathquadraticcurveto{\pgfqpoint{1.984136in}{0.754166in}}{\pgfqpoint{1.956358in}{0.754166in}}%
\pgfpathlineto{\pgfqpoint{0.782747in}{0.754166in}}%
\pgfpathquadraticcurveto{\pgfqpoint{0.754970in}{0.754166in}}{\pgfqpoint{0.754970in}{0.726388in}}%
\pgfpathlineto{\pgfqpoint{0.754970in}{0.546666in}}%
\pgfpathquadraticcurveto{\pgfqpoint{0.754970in}{0.518889in}}{\pgfqpoint{0.782747in}{0.518889in}}%
\pgfpathlineto{\pgfqpoint{0.782747in}{0.518889in}}%
\pgfpathclose%
\pgfusepath{stroke,fill}%
\end{pgfscope}%
\begin{pgfscope}%
\pgfsetrectcap%
\pgfsetroundjoin%
\pgfsetlinewidth{1.505625pt}%
\definecolor{currentstroke}{rgb}{0.000000,0.000000,0.000000}%
\pgfsetstrokecolor{currentstroke}%
\pgfsetdash{}{0pt}%
\pgfpathmoveto{\pgfqpoint{0.810525in}{0.650000in}}%
\pgfpathlineto{\pgfqpoint{0.949414in}{0.650000in}}%
\pgfpathlineto{\pgfqpoint{1.088303in}{0.650000in}}%
\pgfusepath{stroke}%
\end{pgfscope}%
\begin{pgfscope}%
\definecolor{textcolor}{rgb}{0.000000,0.000000,0.000000}%
\pgfsetstrokecolor{textcolor}%
\pgfsetfillcolor{textcolor}%
\pgftext[x=1.199414in,y=0.601388in,left,base]{\color{textcolor}\rmfamily\fontsize{10.000000}{12.000000}\selectfont AUC=0.753}%
\end{pgfscope}%
\end{pgfpicture}%
\makeatother%
\endgroup%

\end{tabular}

The distribution has long tails, so we can make a more useful visualization by truncating the ends.  For this graph we mapped the 0.01 quantile to 0 and the 0.99 quantile to 1 leaving the center 98\% of the distribution and truncated the ends.  Our goal in clipping the tails is to make all of the models' results have approximately the same granularity when we choose the decision thresholds that give us the (politically) desired results.  


\

\verb|AdaBoost_5_Fold_Hard_Test_Transformed_98|

%
\noindent\begin{tabular}{@{\hspace{-6pt}}p{4.3in} @{\hspace{-6pt}}p{2.0in}}
	\vskip 0pt
	\hfil Raw Model Output
	
	%% Creator: Matplotlib, PGF backend
%%
%% To include the figure in your LaTeX document, write
%%   \input{<filename>.pgf}
%%
%% Make sure the required packages are loaded in your preamble
%%   \usepackage{pgf}
%%
%% Also ensure that all the required font packages are loaded; for instance,
%% the lmodern package is sometimes necessary when using math font.
%%   \usepackage{lmodern}
%%
%% Figures using additional raster images can only be included by \input if
%% they are in the same directory as the main LaTeX file. For loading figures
%% from other directories you can use the `import` package
%%   \usepackage{import}
%%
%% and then include the figures with
%%   \import{<path to file>}{<filename>.pgf}
%%
%% Matplotlib used the following preamble
%%   
%%   \usepackage{fontspec}
%%   \makeatletter\@ifpackageloaded{underscore}{}{\usepackage[strings]{underscore}}\makeatother
%%
\begingroup%
\makeatletter%
\begin{pgfpicture}%
\pgfpathrectangle{\pgfpointorigin}{\pgfqpoint{4.002500in}{1.686016in}}%
\pgfusepath{use as bounding box, clip}%
\begin{pgfscope}%
\pgfsetbuttcap%
\pgfsetmiterjoin%
\definecolor{currentfill}{rgb}{1.000000,1.000000,1.000000}%
\pgfsetfillcolor{currentfill}%
\pgfsetlinewidth{0.000000pt}%
\definecolor{currentstroke}{rgb}{1.000000,1.000000,1.000000}%
\pgfsetstrokecolor{currentstroke}%
\pgfsetdash{}{0pt}%
\pgfpathmoveto{\pgfqpoint{0.000000in}{0.000000in}}%
\pgfpathlineto{\pgfqpoint{4.002500in}{0.000000in}}%
\pgfpathlineto{\pgfqpoint{4.002500in}{1.686016in}}%
\pgfpathlineto{\pgfqpoint{0.000000in}{1.686016in}}%
\pgfpathlineto{\pgfqpoint{0.000000in}{0.000000in}}%
\pgfpathclose%
\pgfusepath{fill}%
\end{pgfscope}%
\begin{pgfscope}%
\pgfsetbuttcap%
\pgfsetmiterjoin%
\definecolor{currentfill}{rgb}{1.000000,1.000000,1.000000}%
\pgfsetfillcolor{currentfill}%
\pgfsetlinewidth{0.000000pt}%
\definecolor{currentstroke}{rgb}{0.000000,0.000000,0.000000}%
\pgfsetstrokecolor{currentstroke}%
\pgfsetstrokeopacity{0.000000}%
\pgfsetdash{}{0pt}%
\pgfpathmoveto{\pgfqpoint{0.465000in}{0.449444in}}%
\pgfpathlineto{\pgfqpoint{3.952500in}{0.449444in}}%
\pgfpathlineto{\pgfqpoint{3.952500in}{1.604444in}}%
\pgfpathlineto{\pgfqpoint{0.465000in}{1.604444in}}%
\pgfpathlineto{\pgfqpoint{0.465000in}{0.449444in}}%
\pgfpathclose%
\pgfusepath{fill}%
\end{pgfscope}%
\begin{pgfscope}%
\pgfpathrectangle{\pgfqpoint{0.465000in}{0.449444in}}{\pgfqpoint{3.487500in}{1.155000in}}%
\pgfusepath{clip}%
\pgfsetbuttcap%
\pgfsetmiterjoin%
\pgfsetlinewidth{1.003750pt}%
\definecolor{currentstroke}{rgb}{0.000000,0.000000,0.000000}%
\pgfsetstrokecolor{currentstroke}%
\pgfsetdash{}{0pt}%
\pgfpathmoveto{\pgfqpoint{0.560114in}{0.449444in}}%
\pgfpathlineto{\pgfqpoint{0.623523in}{0.449444in}}%
\pgfpathlineto{\pgfqpoint{0.623523in}{0.546074in}}%
\pgfpathlineto{\pgfqpoint{0.560114in}{0.546074in}}%
\pgfpathlineto{\pgfqpoint{0.560114in}{0.449444in}}%
\pgfpathclose%
\pgfusepath{stroke}%
\end{pgfscope}%
\begin{pgfscope}%
\pgfpathrectangle{\pgfqpoint{0.465000in}{0.449444in}}{\pgfqpoint{3.487500in}{1.155000in}}%
\pgfusepath{clip}%
\pgfsetbuttcap%
\pgfsetmiterjoin%
\pgfsetlinewidth{1.003750pt}%
\definecolor{currentstroke}{rgb}{0.000000,0.000000,0.000000}%
\pgfsetstrokecolor{currentstroke}%
\pgfsetdash{}{0pt}%
\pgfpathmoveto{\pgfqpoint{0.718637in}{0.449444in}}%
\pgfpathlineto{\pgfqpoint{0.782046in}{0.449444in}}%
\pgfpathlineto{\pgfqpoint{0.782046in}{0.583803in}}%
\pgfpathlineto{\pgfqpoint{0.718637in}{0.583803in}}%
\pgfpathlineto{\pgfqpoint{0.718637in}{0.449444in}}%
\pgfpathclose%
\pgfusepath{stroke}%
\end{pgfscope}%
\begin{pgfscope}%
\pgfpathrectangle{\pgfqpoint{0.465000in}{0.449444in}}{\pgfqpoint{3.487500in}{1.155000in}}%
\pgfusepath{clip}%
\pgfsetbuttcap%
\pgfsetmiterjoin%
\pgfsetlinewidth{1.003750pt}%
\definecolor{currentstroke}{rgb}{0.000000,0.000000,0.000000}%
\pgfsetstrokecolor{currentstroke}%
\pgfsetdash{}{0pt}%
\pgfpathmoveto{\pgfqpoint{0.877159in}{0.449444in}}%
\pgfpathlineto{\pgfqpoint{0.940568in}{0.449444in}}%
\pgfpathlineto{\pgfqpoint{0.940568in}{0.694280in}}%
\pgfpathlineto{\pgfqpoint{0.877159in}{0.694280in}}%
\pgfpathlineto{\pgfqpoint{0.877159in}{0.449444in}}%
\pgfpathclose%
\pgfusepath{stroke}%
\end{pgfscope}%
\begin{pgfscope}%
\pgfpathrectangle{\pgfqpoint{0.465000in}{0.449444in}}{\pgfqpoint{3.487500in}{1.155000in}}%
\pgfusepath{clip}%
\pgfsetbuttcap%
\pgfsetmiterjoin%
\pgfsetlinewidth{1.003750pt}%
\definecolor{currentstroke}{rgb}{0.000000,0.000000,0.000000}%
\pgfsetstrokecolor{currentstroke}%
\pgfsetdash{}{0pt}%
\pgfpathmoveto{\pgfqpoint{1.035682in}{0.449444in}}%
\pgfpathlineto{\pgfqpoint{1.099091in}{0.449444in}}%
\pgfpathlineto{\pgfqpoint{1.099091in}{0.851277in}}%
\pgfpathlineto{\pgfqpoint{1.035682in}{0.851277in}}%
\pgfpathlineto{\pgfqpoint{1.035682in}{0.449444in}}%
\pgfpathclose%
\pgfusepath{stroke}%
\end{pgfscope}%
\begin{pgfscope}%
\pgfpathrectangle{\pgfqpoint{0.465000in}{0.449444in}}{\pgfqpoint{3.487500in}{1.155000in}}%
\pgfusepath{clip}%
\pgfsetbuttcap%
\pgfsetmiterjoin%
\pgfsetlinewidth{1.003750pt}%
\definecolor{currentstroke}{rgb}{0.000000,0.000000,0.000000}%
\pgfsetstrokecolor{currentstroke}%
\pgfsetdash{}{0pt}%
\pgfpathmoveto{\pgfqpoint{1.194205in}{0.449444in}}%
\pgfpathlineto{\pgfqpoint{1.257614in}{0.449444in}}%
\pgfpathlineto{\pgfqpoint{1.257614in}{1.027849in}}%
\pgfpathlineto{\pgfqpoint{1.194205in}{1.027849in}}%
\pgfpathlineto{\pgfqpoint{1.194205in}{0.449444in}}%
\pgfpathclose%
\pgfusepath{stroke}%
\end{pgfscope}%
\begin{pgfscope}%
\pgfpathrectangle{\pgfqpoint{0.465000in}{0.449444in}}{\pgfqpoint{3.487500in}{1.155000in}}%
\pgfusepath{clip}%
\pgfsetbuttcap%
\pgfsetmiterjoin%
\pgfsetlinewidth{1.003750pt}%
\definecolor{currentstroke}{rgb}{0.000000,0.000000,0.000000}%
\pgfsetstrokecolor{currentstroke}%
\pgfsetdash{}{0pt}%
\pgfpathmoveto{\pgfqpoint{1.352728in}{0.449444in}}%
\pgfpathlineto{\pgfqpoint{1.416137in}{0.449444in}}%
\pgfpathlineto{\pgfqpoint{1.416137in}{1.217247in}}%
\pgfpathlineto{\pgfqpoint{1.352728in}{1.217247in}}%
\pgfpathlineto{\pgfqpoint{1.352728in}{0.449444in}}%
\pgfpathclose%
\pgfusepath{stroke}%
\end{pgfscope}%
\begin{pgfscope}%
\pgfpathrectangle{\pgfqpoint{0.465000in}{0.449444in}}{\pgfqpoint{3.487500in}{1.155000in}}%
\pgfusepath{clip}%
\pgfsetbuttcap%
\pgfsetmiterjoin%
\pgfsetlinewidth{1.003750pt}%
\definecolor{currentstroke}{rgb}{0.000000,0.000000,0.000000}%
\pgfsetstrokecolor{currentstroke}%
\pgfsetdash{}{0pt}%
\pgfpathmoveto{\pgfqpoint{1.511250in}{0.449444in}}%
\pgfpathlineto{\pgfqpoint{1.574659in}{0.449444in}}%
\pgfpathlineto{\pgfqpoint{1.574659in}{1.393963in}}%
\pgfpathlineto{\pgfqpoint{1.511250in}{1.393963in}}%
\pgfpathlineto{\pgfqpoint{1.511250in}{0.449444in}}%
\pgfpathclose%
\pgfusepath{stroke}%
\end{pgfscope}%
\begin{pgfscope}%
\pgfpathrectangle{\pgfqpoint{0.465000in}{0.449444in}}{\pgfqpoint{3.487500in}{1.155000in}}%
\pgfusepath{clip}%
\pgfsetbuttcap%
\pgfsetmiterjoin%
\pgfsetlinewidth{1.003750pt}%
\definecolor{currentstroke}{rgb}{0.000000,0.000000,0.000000}%
\pgfsetstrokecolor{currentstroke}%
\pgfsetdash{}{0pt}%
\pgfpathmoveto{\pgfqpoint{1.669773in}{0.449444in}}%
\pgfpathlineto{\pgfqpoint{1.733182in}{0.449444in}}%
\pgfpathlineto{\pgfqpoint{1.733182in}{1.510247in}}%
\pgfpathlineto{\pgfqpoint{1.669773in}{1.510247in}}%
\pgfpathlineto{\pgfqpoint{1.669773in}{0.449444in}}%
\pgfpathclose%
\pgfusepath{stroke}%
\end{pgfscope}%
\begin{pgfscope}%
\pgfpathrectangle{\pgfqpoint{0.465000in}{0.449444in}}{\pgfqpoint{3.487500in}{1.155000in}}%
\pgfusepath{clip}%
\pgfsetbuttcap%
\pgfsetmiterjoin%
\pgfsetlinewidth{1.003750pt}%
\definecolor{currentstroke}{rgb}{0.000000,0.000000,0.000000}%
\pgfsetstrokecolor{currentstroke}%
\pgfsetdash{}{0pt}%
\pgfpathmoveto{\pgfqpoint{1.828296in}{0.449444in}}%
\pgfpathlineto{\pgfqpoint{1.891705in}{0.449444in}}%
\pgfpathlineto{\pgfqpoint{1.891705in}{1.549444in}}%
\pgfpathlineto{\pgfqpoint{1.828296in}{1.549444in}}%
\pgfpathlineto{\pgfqpoint{1.828296in}{0.449444in}}%
\pgfpathclose%
\pgfusepath{stroke}%
\end{pgfscope}%
\begin{pgfscope}%
\pgfpathrectangle{\pgfqpoint{0.465000in}{0.449444in}}{\pgfqpoint{3.487500in}{1.155000in}}%
\pgfusepath{clip}%
\pgfsetbuttcap%
\pgfsetmiterjoin%
\pgfsetlinewidth{1.003750pt}%
\definecolor{currentstroke}{rgb}{0.000000,0.000000,0.000000}%
\pgfsetstrokecolor{currentstroke}%
\pgfsetdash{}{0pt}%
\pgfpathmoveto{\pgfqpoint{1.986818in}{0.449444in}}%
\pgfpathlineto{\pgfqpoint{2.050228in}{0.449444in}}%
\pgfpathlineto{\pgfqpoint{2.050228in}{1.494533in}}%
\pgfpathlineto{\pgfqpoint{1.986818in}{1.494533in}}%
\pgfpathlineto{\pgfqpoint{1.986818in}{0.449444in}}%
\pgfpathclose%
\pgfusepath{stroke}%
\end{pgfscope}%
\begin{pgfscope}%
\pgfpathrectangle{\pgfqpoint{0.465000in}{0.449444in}}{\pgfqpoint{3.487500in}{1.155000in}}%
\pgfusepath{clip}%
\pgfsetbuttcap%
\pgfsetmiterjoin%
\pgfsetlinewidth{1.003750pt}%
\definecolor{currentstroke}{rgb}{0.000000,0.000000,0.000000}%
\pgfsetstrokecolor{currentstroke}%
\pgfsetdash{}{0pt}%
\pgfpathmoveto{\pgfqpoint{2.145341in}{0.449444in}}%
\pgfpathlineto{\pgfqpoint{2.208750in}{0.449444in}}%
\pgfpathlineto{\pgfqpoint{2.208750in}{1.372410in}}%
\pgfpathlineto{\pgfqpoint{2.145341in}{1.372410in}}%
\pgfpathlineto{\pgfqpoint{2.145341in}{0.449444in}}%
\pgfpathclose%
\pgfusepath{stroke}%
\end{pgfscope}%
\begin{pgfscope}%
\pgfpathrectangle{\pgfqpoint{0.465000in}{0.449444in}}{\pgfqpoint{3.487500in}{1.155000in}}%
\pgfusepath{clip}%
\pgfsetbuttcap%
\pgfsetmiterjoin%
\pgfsetlinewidth{1.003750pt}%
\definecolor{currentstroke}{rgb}{0.000000,0.000000,0.000000}%
\pgfsetstrokecolor{currentstroke}%
\pgfsetdash{}{0pt}%
\pgfpathmoveto{\pgfqpoint{2.303864in}{0.449444in}}%
\pgfpathlineto{\pgfqpoint{2.367273in}{0.449444in}}%
\pgfpathlineto{\pgfqpoint{2.367273in}{1.196939in}}%
\pgfpathlineto{\pgfqpoint{2.303864in}{1.196939in}}%
\pgfpathlineto{\pgfqpoint{2.303864in}{0.449444in}}%
\pgfpathclose%
\pgfusepath{stroke}%
\end{pgfscope}%
\begin{pgfscope}%
\pgfpathrectangle{\pgfqpoint{0.465000in}{0.449444in}}{\pgfqpoint{3.487500in}{1.155000in}}%
\pgfusepath{clip}%
\pgfsetbuttcap%
\pgfsetmiterjoin%
\pgfsetlinewidth{1.003750pt}%
\definecolor{currentstroke}{rgb}{0.000000,0.000000,0.000000}%
\pgfsetstrokecolor{currentstroke}%
\pgfsetdash{}{0pt}%
\pgfpathmoveto{\pgfqpoint{2.462387in}{0.449444in}}%
\pgfpathlineto{\pgfqpoint{2.525796in}{0.449444in}}%
\pgfpathlineto{\pgfqpoint{2.525796in}{1.010604in}}%
\pgfpathlineto{\pgfqpoint{2.462387in}{1.010604in}}%
\pgfpathlineto{\pgfqpoint{2.462387in}{0.449444in}}%
\pgfpathclose%
\pgfusepath{stroke}%
\end{pgfscope}%
\begin{pgfscope}%
\pgfpathrectangle{\pgfqpoint{0.465000in}{0.449444in}}{\pgfqpoint{3.487500in}{1.155000in}}%
\pgfusepath{clip}%
\pgfsetbuttcap%
\pgfsetmiterjoin%
\pgfsetlinewidth{1.003750pt}%
\definecolor{currentstroke}{rgb}{0.000000,0.000000,0.000000}%
\pgfsetstrokecolor{currentstroke}%
\pgfsetdash{}{0pt}%
\pgfpathmoveto{\pgfqpoint{2.620909in}{0.449444in}}%
\pgfpathlineto{\pgfqpoint{2.684318in}{0.449444in}}%
\pgfpathlineto{\pgfqpoint{2.684318in}{0.845103in}}%
\pgfpathlineto{\pgfqpoint{2.620909in}{0.845103in}}%
\pgfpathlineto{\pgfqpoint{2.620909in}{0.449444in}}%
\pgfpathclose%
\pgfusepath{stroke}%
\end{pgfscope}%
\begin{pgfscope}%
\pgfpathrectangle{\pgfqpoint{0.465000in}{0.449444in}}{\pgfqpoint{3.487500in}{1.155000in}}%
\pgfusepath{clip}%
\pgfsetbuttcap%
\pgfsetmiterjoin%
\pgfsetlinewidth{1.003750pt}%
\definecolor{currentstroke}{rgb}{0.000000,0.000000,0.000000}%
\pgfsetstrokecolor{currentstroke}%
\pgfsetdash{}{0pt}%
\pgfpathmoveto{\pgfqpoint{2.779432in}{0.449444in}}%
\pgfpathlineto{\pgfqpoint{2.842841in}{0.449444in}}%
\pgfpathlineto{\pgfqpoint{2.842841in}{0.708303in}}%
\pgfpathlineto{\pgfqpoint{2.779432in}{0.708303in}}%
\pgfpathlineto{\pgfqpoint{2.779432in}{0.449444in}}%
\pgfpathclose%
\pgfusepath{stroke}%
\end{pgfscope}%
\begin{pgfscope}%
\pgfpathrectangle{\pgfqpoint{0.465000in}{0.449444in}}{\pgfqpoint{3.487500in}{1.155000in}}%
\pgfusepath{clip}%
\pgfsetbuttcap%
\pgfsetmiterjoin%
\pgfsetlinewidth{1.003750pt}%
\definecolor{currentstroke}{rgb}{0.000000,0.000000,0.000000}%
\pgfsetstrokecolor{currentstroke}%
\pgfsetdash{}{0pt}%
\pgfpathmoveto{\pgfqpoint{2.937955in}{0.449444in}}%
\pgfpathlineto{\pgfqpoint{3.001364in}{0.449444in}}%
\pgfpathlineto{\pgfqpoint{3.001364in}{0.618279in}}%
\pgfpathlineto{\pgfqpoint{2.937955in}{0.618279in}}%
\pgfpathlineto{\pgfqpoint{2.937955in}{0.449444in}}%
\pgfpathclose%
\pgfusepath{stroke}%
\end{pgfscope}%
\begin{pgfscope}%
\pgfpathrectangle{\pgfqpoint{0.465000in}{0.449444in}}{\pgfqpoint{3.487500in}{1.155000in}}%
\pgfusepath{clip}%
\pgfsetbuttcap%
\pgfsetmiterjoin%
\pgfsetlinewidth{1.003750pt}%
\definecolor{currentstroke}{rgb}{0.000000,0.000000,0.000000}%
\pgfsetstrokecolor{currentstroke}%
\pgfsetdash{}{0pt}%
\pgfpathmoveto{\pgfqpoint{3.096478in}{0.449444in}}%
\pgfpathlineto{\pgfqpoint{3.159887in}{0.449444in}}%
\pgfpathlineto{\pgfqpoint{3.159887in}{0.551274in}}%
\pgfpathlineto{\pgfqpoint{3.096478in}{0.551274in}}%
\pgfpathlineto{\pgfqpoint{3.096478in}{0.449444in}}%
\pgfpathclose%
\pgfusepath{stroke}%
\end{pgfscope}%
\begin{pgfscope}%
\pgfpathrectangle{\pgfqpoint{0.465000in}{0.449444in}}{\pgfqpoint{3.487500in}{1.155000in}}%
\pgfusepath{clip}%
\pgfsetbuttcap%
\pgfsetmiterjoin%
\pgfsetlinewidth{1.003750pt}%
\definecolor{currentstroke}{rgb}{0.000000,0.000000,0.000000}%
\pgfsetstrokecolor{currentstroke}%
\pgfsetdash{}{0pt}%
\pgfpathmoveto{\pgfqpoint{3.255000in}{0.449444in}}%
\pgfpathlineto{\pgfqpoint{3.318409in}{0.449444in}}%
\pgfpathlineto{\pgfqpoint{3.318409in}{0.509923in}}%
\pgfpathlineto{\pgfqpoint{3.255000in}{0.509923in}}%
\pgfpathlineto{\pgfqpoint{3.255000in}{0.449444in}}%
\pgfpathclose%
\pgfusepath{stroke}%
\end{pgfscope}%
\begin{pgfscope}%
\pgfpathrectangle{\pgfqpoint{0.465000in}{0.449444in}}{\pgfqpoint{3.487500in}{1.155000in}}%
\pgfusepath{clip}%
\pgfsetbuttcap%
\pgfsetmiterjoin%
\pgfsetlinewidth{1.003750pt}%
\definecolor{currentstroke}{rgb}{0.000000,0.000000,0.000000}%
\pgfsetstrokecolor{currentstroke}%
\pgfsetdash{}{0pt}%
\pgfpathmoveto{\pgfqpoint{3.413523in}{0.449444in}}%
\pgfpathlineto{\pgfqpoint{3.476932in}{0.449444in}}%
\pgfpathlineto{\pgfqpoint{3.476932in}{0.482803in}}%
\pgfpathlineto{\pgfqpoint{3.413523in}{0.482803in}}%
\pgfpathlineto{\pgfqpoint{3.413523in}{0.449444in}}%
\pgfpathclose%
\pgfusepath{stroke}%
\end{pgfscope}%
\begin{pgfscope}%
\pgfpathrectangle{\pgfqpoint{0.465000in}{0.449444in}}{\pgfqpoint{3.487500in}{1.155000in}}%
\pgfusepath{clip}%
\pgfsetbuttcap%
\pgfsetmiterjoin%
\pgfsetlinewidth{1.003750pt}%
\definecolor{currentstroke}{rgb}{0.000000,0.000000,0.000000}%
\pgfsetstrokecolor{currentstroke}%
\pgfsetdash{}{0pt}%
\pgfpathmoveto{\pgfqpoint{3.572046in}{0.449444in}}%
\pgfpathlineto{\pgfqpoint{3.635455in}{0.449444in}}%
\pgfpathlineto{\pgfqpoint{3.635455in}{0.468030in}}%
\pgfpathlineto{\pgfqpoint{3.572046in}{0.468030in}}%
\pgfpathlineto{\pgfqpoint{3.572046in}{0.449444in}}%
\pgfpathclose%
\pgfusepath{stroke}%
\end{pgfscope}%
\begin{pgfscope}%
\pgfpathrectangle{\pgfqpoint{0.465000in}{0.449444in}}{\pgfqpoint{3.487500in}{1.155000in}}%
\pgfusepath{clip}%
\pgfsetbuttcap%
\pgfsetmiterjoin%
\pgfsetlinewidth{1.003750pt}%
\definecolor{currentstroke}{rgb}{0.000000,0.000000,0.000000}%
\pgfsetstrokecolor{currentstroke}%
\pgfsetdash{}{0pt}%
\pgfpathmoveto{\pgfqpoint{3.730568in}{0.449444in}}%
\pgfpathlineto{\pgfqpoint{3.793978in}{0.449444in}}%
\pgfpathlineto{\pgfqpoint{3.793978in}{0.467456in}}%
\pgfpathlineto{\pgfqpoint{3.730568in}{0.467456in}}%
\pgfpathlineto{\pgfqpoint{3.730568in}{0.449444in}}%
\pgfpathclose%
\pgfusepath{stroke}%
\end{pgfscope}%
\begin{pgfscope}%
\pgfpathrectangle{\pgfqpoint{0.465000in}{0.449444in}}{\pgfqpoint{3.487500in}{1.155000in}}%
\pgfusepath{clip}%
\pgfsetbuttcap%
\pgfsetmiterjoin%
\definecolor{currentfill}{rgb}{0.000000,0.000000,0.000000}%
\pgfsetfillcolor{currentfill}%
\pgfsetlinewidth{0.000000pt}%
\definecolor{currentstroke}{rgb}{0.000000,0.000000,0.000000}%
\pgfsetstrokecolor{currentstroke}%
\pgfsetstrokeopacity{0.000000}%
\pgfsetdash{}{0pt}%
\pgfpathmoveto{\pgfqpoint{0.623523in}{0.449444in}}%
\pgfpathlineto{\pgfqpoint{0.686932in}{0.449444in}}%
\pgfpathlineto{\pgfqpoint{0.686932in}{0.450306in}}%
\pgfpathlineto{\pgfqpoint{0.623523in}{0.450306in}}%
\pgfpathlineto{\pgfqpoint{0.623523in}{0.449444in}}%
\pgfpathclose%
\pgfusepath{fill}%
\end{pgfscope}%
\begin{pgfscope}%
\pgfpathrectangle{\pgfqpoint{0.465000in}{0.449444in}}{\pgfqpoint{3.487500in}{1.155000in}}%
\pgfusepath{clip}%
\pgfsetbuttcap%
\pgfsetmiterjoin%
\definecolor{currentfill}{rgb}{0.000000,0.000000,0.000000}%
\pgfsetfillcolor{currentfill}%
\pgfsetlinewidth{0.000000pt}%
\definecolor{currentstroke}{rgb}{0.000000,0.000000,0.000000}%
\pgfsetstrokecolor{currentstroke}%
\pgfsetstrokeopacity{0.000000}%
\pgfsetdash{}{0pt}%
\pgfpathmoveto{\pgfqpoint{0.782046in}{0.449444in}}%
\pgfpathlineto{\pgfqpoint{0.845455in}{0.449444in}}%
\pgfpathlineto{\pgfqpoint{0.845455in}{0.451263in}}%
\pgfpathlineto{\pgfqpoint{0.782046in}{0.451263in}}%
\pgfpathlineto{\pgfqpoint{0.782046in}{0.449444in}}%
\pgfpathclose%
\pgfusepath{fill}%
\end{pgfscope}%
\begin{pgfscope}%
\pgfpathrectangle{\pgfqpoint{0.465000in}{0.449444in}}{\pgfqpoint{3.487500in}{1.155000in}}%
\pgfusepath{clip}%
\pgfsetbuttcap%
\pgfsetmiterjoin%
\definecolor{currentfill}{rgb}{0.000000,0.000000,0.000000}%
\pgfsetfillcolor{currentfill}%
\pgfsetlinewidth{0.000000pt}%
\definecolor{currentstroke}{rgb}{0.000000,0.000000,0.000000}%
\pgfsetstrokecolor{currentstroke}%
\pgfsetstrokeopacity{0.000000}%
\pgfsetdash{}{0pt}%
\pgfpathmoveto{\pgfqpoint{0.940568in}{0.449444in}}%
\pgfpathlineto{\pgfqpoint{1.003978in}{0.449444in}}%
\pgfpathlineto{\pgfqpoint{1.003978in}{0.453847in}}%
\pgfpathlineto{\pgfqpoint{0.940568in}{0.453847in}}%
\pgfpathlineto{\pgfqpoint{0.940568in}{0.449444in}}%
\pgfpathclose%
\pgfusepath{fill}%
\end{pgfscope}%
\begin{pgfscope}%
\pgfpathrectangle{\pgfqpoint{0.465000in}{0.449444in}}{\pgfqpoint{3.487500in}{1.155000in}}%
\pgfusepath{clip}%
\pgfsetbuttcap%
\pgfsetmiterjoin%
\definecolor{currentfill}{rgb}{0.000000,0.000000,0.000000}%
\pgfsetfillcolor{currentfill}%
\pgfsetlinewidth{0.000000pt}%
\definecolor{currentstroke}{rgb}{0.000000,0.000000,0.000000}%
\pgfsetstrokecolor{currentstroke}%
\pgfsetstrokeopacity{0.000000}%
\pgfsetdash{}{0pt}%
\pgfpathmoveto{\pgfqpoint{1.099091in}{0.449444in}}%
\pgfpathlineto{\pgfqpoint{1.162500in}{0.449444in}}%
\pgfpathlineto{\pgfqpoint{1.162500in}{0.459176in}}%
\pgfpathlineto{\pgfqpoint{1.099091in}{0.459176in}}%
\pgfpathlineto{\pgfqpoint{1.099091in}{0.449444in}}%
\pgfpathclose%
\pgfusepath{fill}%
\end{pgfscope}%
\begin{pgfscope}%
\pgfpathrectangle{\pgfqpoint{0.465000in}{0.449444in}}{\pgfqpoint{3.487500in}{1.155000in}}%
\pgfusepath{clip}%
\pgfsetbuttcap%
\pgfsetmiterjoin%
\definecolor{currentfill}{rgb}{0.000000,0.000000,0.000000}%
\pgfsetfillcolor{currentfill}%
\pgfsetlinewidth{0.000000pt}%
\definecolor{currentstroke}{rgb}{0.000000,0.000000,0.000000}%
\pgfsetstrokecolor{currentstroke}%
\pgfsetstrokeopacity{0.000000}%
\pgfsetdash{}{0pt}%
\pgfpathmoveto{\pgfqpoint{1.257614in}{0.449444in}}%
\pgfpathlineto{\pgfqpoint{1.321023in}{0.449444in}}%
\pgfpathlineto{\pgfqpoint{1.321023in}{0.468349in}}%
\pgfpathlineto{\pgfqpoint{1.257614in}{0.468349in}}%
\pgfpathlineto{\pgfqpoint{1.257614in}{0.449444in}}%
\pgfpathclose%
\pgfusepath{fill}%
\end{pgfscope}%
\begin{pgfscope}%
\pgfpathrectangle{\pgfqpoint{0.465000in}{0.449444in}}{\pgfqpoint{3.487500in}{1.155000in}}%
\pgfusepath{clip}%
\pgfsetbuttcap%
\pgfsetmiterjoin%
\definecolor{currentfill}{rgb}{0.000000,0.000000,0.000000}%
\pgfsetfillcolor{currentfill}%
\pgfsetlinewidth{0.000000pt}%
\definecolor{currentstroke}{rgb}{0.000000,0.000000,0.000000}%
\pgfsetstrokecolor{currentstroke}%
\pgfsetstrokeopacity{0.000000}%
\pgfsetdash{}{0pt}%
\pgfpathmoveto{\pgfqpoint{1.416137in}{0.449444in}}%
\pgfpathlineto{\pgfqpoint{1.479546in}{0.449444in}}%
\pgfpathlineto{\pgfqpoint{1.479546in}{0.483297in}}%
\pgfpathlineto{\pgfqpoint{1.416137in}{0.483297in}}%
\pgfpathlineto{\pgfqpoint{1.416137in}{0.449444in}}%
\pgfpathclose%
\pgfusepath{fill}%
\end{pgfscope}%
\begin{pgfscope}%
\pgfpathrectangle{\pgfqpoint{0.465000in}{0.449444in}}{\pgfqpoint{3.487500in}{1.155000in}}%
\pgfusepath{clip}%
\pgfsetbuttcap%
\pgfsetmiterjoin%
\definecolor{currentfill}{rgb}{0.000000,0.000000,0.000000}%
\pgfsetfillcolor{currentfill}%
\pgfsetlinewidth{0.000000pt}%
\definecolor{currentstroke}{rgb}{0.000000,0.000000,0.000000}%
\pgfsetstrokecolor{currentstroke}%
\pgfsetstrokeopacity{0.000000}%
\pgfsetdash{}{0pt}%
\pgfpathmoveto{\pgfqpoint{1.574659in}{0.449444in}}%
\pgfpathlineto{\pgfqpoint{1.638068in}{0.449444in}}%
\pgfpathlineto{\pgfqpoint{1.638068in}{0.506095in}}%
\pgfpathlineto{\pgfqpoint{1.574659in}{0.506095in}}%
\pgfpathlineto{\pgfqpoint{1.574659in}{0.449444in}}%
\pgfpathclose%
\pgfusepath{fill}%
\end{pgfscope}%
\begin{pgfscope}%
\pgfpathrectangle{\pgfqpoint{0.465000in}{0.449444in}}{\pgfqpoint{3.487500in}{1.155000in}}%
\pgfusepath{clip}%
\pgfsetbuttcap%
\pgfsetmiterjoin%
\definecolor{currentfill}{rgb}{0.000000,0.000000,0.000000}%
\pgfsetfillcolor{currentfill}%
\pgfsetlinewidth{0.000000pt}%
\definecolor{currentstroke}{rgb}{0.000000,0.000000,0.000000}%
\pgfsetstrokecolor{currentstroke}%
\pgfsetstrokeopacity{0.000000}%
\pgfsetdash{}{0pt}%
\pgfpathmoveto{\pgfqpoint{1.733182in}{0.449444in}}%
\pgfpathlineto{\pgfqpoint{1.796591in}{0.449444in}}%
\pgfpathlineto{\pgfqpoint{1.796591in}{0.538687in}}%
\pgfpathlineto{\pgfqpoint{1.733182in}{0.538687in}}%
\pgfpathlineto{\pgfqpoint{1.733182in}{0.449444in}}%
\pgfpathclose%
\pgfusepath{fill}%
\end{pgfscope}%
\begin{pgfscope}%
\pgfpathrectangle{\pgfqpoint{0.465000in}{0.449444in}}{\pgfqpoint{3.487500in}{1.155000in}}%
\pgfusepath{clip}%
\pgfsetbuttcap%
\pgfsetmiterjoin%
\definecolor{currentfill}{rgb}{0.000000,0.000000,0.000000}%
\pgfsetfillcolor{currentfill}%
\pgfsetlinewidth{0.000000pt}%
\definecolor{currentstroke}{rgb}{0.000000,0.000000,0.000000}%
\pgfsetstrokecolor{currentstroke}%
\pgfsetstrokeopacity{0.000000}%
\pgfsetdash{}{0pt}%
\pgfpathmoveto{\pgfqpoint{1.891705in}{0.449444in}}%
\pgfpathlineto{\pgfqpoint{1.955114in}{0.449444in}}%
\pgfpathlineto{\pgfqpoint{1.955114in}{0.573242in}}%
\pgfpathlineto{\pgfqpoint{1.891705in}{0.573242in}}%
\pgfpathlineto{\pgfqpoint{1.891705in}{0.449444in}}%
\pgfpathclose%
\pgfusepath{fill}%
\end{pgfscope}%
\begin{pgfscope}%
\pgfpathrectangle{\pgfqpoint{0.465000in}{0.449444in}}{\pgfqpoint{3.487500in}{1.155000in}}%
\pgfusepath{clip}%
\pgfsetbuttcap%
\pgfsetmiterjoin%
\definecolor{currentfill}{rgb}{0.000000,0.000000,0.000000}%
\pgfsetfillcolor{currentfill}%
\pgfsetlinewidth{0.000000pt}%
\definecolor{currentstroke}{rgb}{0.000000,0.000000,0.000000}%
\pgfsetstrokecolor{currentstroke}%
\pgfsetstrokeopacity{0.000000}%
\pgfsetdash{}{0pt}%
\pgfpathmoveto{\pgfqpoint{2.050228in}{0.449444in}}%
\pgfpathlineto{\pgfqpoint{2.113637in}{0.449444in}}%
\pgfpathlineto{\pgfqpoint{2.113637in}{0.606952in}}%
\pgfpathlineto{\pgfqpoint{2.050228in}{0.606952in}}%
\pgfpathlineto{\pgfqpoint{2.050228in}{0.449444in}}%
\pgfpathclose%
\pgfusepath{fill}%
\end{pgfscope}%
\begin{pgfscope}%
\pgfpathrectangle{\pgfqpoint{0.465000in}{0.449444in}}{\pgfqpoint{3.487500in}{1.155000in}}%
\pgfusepath{clip}%
\pgfsetbuttcap%
\pgfsetmiterjoin%
\definecolor{currentfill}{rgb}{0.000000,0.000000,0.000000}%
\pgfsetfillcolor{currentfill}%
\pgfsetlinewidth{0.000000pt}%
\definecolor{currentstroke}{rgb}{0.000000,0.000000,0.000000}%
\pgfsetstrokecolor{currentstroke}%
\pgfsetstrokeopacity{0.000000}%
\pgfsetdash{}{0pt}%
\pgfpathmoveto{\pgfqpoint{2.208750in}{0.449444in}}%
\pgfpathlineto{\pgfqpoint{2.272159in}{0.449444in}}%
\pgfpathlineto{\pgfqpoint{2.272159in}{0.632541in}}%
\pgfpathlineto{\pgfqpoint{2.208750in}{0.632541in}}%
\pgfpathlineto{\pgfqpoint{2.208750in}{0.449444in}}%
\pgfpathclose%
\pgfusepath{fill}%
\end{pgfscope}%
\begin{pgfscope}%
\pgfpathrectangle{\pgfqpoint{0.465000in}{0.449444in}}{\pgfqpoint{3.487500in}{1.155000in}}%
\pgfusepath{clip}%
\pgfsetbuttcap%
\pgfsetmiterjoin%
\definecolor{currentfill}{rgb}{0.000000,0.000000,0.000000}%
\pgfsetfillcolor{currentfill}%
\pgfsetlinewidth{0.000000pt}%
\definecolor{currentstroke}{rgb}{0.000000,0.000000,0.000000}%
\pgfsetstrokecolor{currentstroke}%
\pgfsetstrokeopacity{0.000000}%
\pgfsetdash{}{0pt}%
\pgfpathmoveto{\pgfqpoint{2.367273in}{0.449444in}}%
\pgfpathlineto{\pgfqpoint{2.430682in}{0.449444in}}%
\pgfpathlineto{\pgfqpoint{2.430682in}{0.641188in}}%
\pgfpathlineto{\pgfqpoint{2.367273in}{0.641188in}}%
\pgfpathlineto{\pgfqpoint{2.367273in}{0.449444in}}%
\pgfpathclose%
\pgfusepath{fill}%
\end{pgfscope}%
\begin{pgfscope}%
\pgfpathrectangle{\pgfqpoint{0.465000in}{0.449444in}}{\pgfqpoint{3.487500in}{1.155000in}}%
\pgfusepath{clip}%
\pgfsetbuttcap%
\pgfsetmiterjoin%
\definecolor{currentfill}{rgb}{0.000000,0.000000,0.000000}%
\pgfsetfillcolor{currentfill}%
\pgfsetlinewidth{0.000000pt}%
\definecolor{currentstroke}{rgb}{0.000000,0.000000,0.000000}%
\pgfsetstrokecolor{currentstroke}%
\pgfsetstrokeopacity{0.000000}%
\pgfsetdash{}{0pt}%
\pgfpathmoveto{\pgfqpoint{2.525796in}{0.449444in}}%
\pgfpathlineto{\pgfqpoint{2.589205in}{0.449444in}}%
\pgfpathlineto{\pgfqpoint{2.589205in}{0.636928in}}%
\pgfpathlineto{\pgfqpoint{2.525796in}{0.636928in}}%
\pgfpathlineto{\pgfqpoint{2.525796in}{0.449444in}}%
\pgfpathclose%
\pgfusepath{fill}%
\end{pgfscope}%
\begin{pgfscope}%
\pgfpathrectangle{\pgfqpoint{0.465000in}{0.449444in}}{\pgfqpoint{3.487500in}{1.155000in}}%
\pgfusepath{clip}%
\pgfsetbuttcap%
\pgfsetmiterjoin%
\definecolor{currentfill}{rgb}{0.000000,0.000000,0.000000}%
\pgfsetfillcolor{currentfill}%
\pgfsetlinewidth{0.000000pt}%
\definecolor{currentstroke}{rgb}{0.000000,0.000000,0.000000}%
\pgfsetstrokecolor{currentstroke}%
\pgfsetstrokeopacity{0.000000}%
\pgfsetdash{}{0pt}%
\pgfpathmoveto{\pgfqpoint{2.684318in}{0.449444in}}%
\pgfpathlineto{\pgfqpoint{2.747728in}{0.449444in}}%
\pgfpathlineto{\pgfqpoint{2.747728in}{0.618438in}}%
\pgfpathlineto{\pgfqpoint{2.684318in}{0.618438in}}%
\pgfpathlineto{\pgfqpoint{2.684318in}{0.449444in}}%
\pgfpathclose%
\pgfusepath{fill}%
\end{pgfscope}%
\begin{pgfscope}%
\pgfpathrectangle{\pgfqpoint{0.465000in}{0.449444in}}{\pgfqpoint{3.487500in}{1.155000in}}%
\pgfusepath{clip}%
\pgfsetbuttcap%
\pgfsetmiterjoin%
\definecolor{currentfill}{rgb}{0.000000,0.000000,0.000000}%
\pgfsetfillcolor{currentfill}%
\pgfsetlinewidth{0.000000pt}%
\definecolor{currentstroke}{rgb}{0.000000,0.000000,0.000000}%
\pgfsetstrokecolor{currentstroke}%
\pgfsetstrokeopacity{0.000000}%
\pgfsetdash{}{0pt}%
\pgfpathmoveto{\pgfqpoint{2.842841in}{0.449444in}}%
\pgfpathlineto{\pgfqpoint{2.906250in}{0.449444in}}%
\pgfpathlineto{\pgfqpoint{2.906250in}{0.591238in}}%
\pgfpathlineto{\pgfqpoint{2.842841in}{0.591238in}}%
\pgfpathlineto{\pgfqpoint{2.842841in}{0.449444in}}%
\pgfpathclose%
\pgfusepath{fill}%
\end{pgfscope}%
\begin{pgfscope}%
\pgfpathrectangle{\pgfqpoint{0.465000in}{0.449444in}}{\pgfqpoint{3.487500in}{1.155000in}}%
\pgfusepath{clip}%
\pgfsetbuttcap%
\pgfsetmiterjoin%
\definecolor{currentfill}{rgb}{0.000000,0.000000,0.000000}%
\pgfsetfillcolor{currentfill}%
\pgfsetlinewidth{0.000000pt}%
\definecolor{currentstroke}{rgb}{0.000000,0.000000,0.000000}%
\pgfsetstrokecolor{currentstroke}%
\pgfsetstrokeopacity{0.000000}%
\pgfsetdash{}{0pt}%
\pgfpathmoveto{\pgfqpoint{3.001364in}{0.449444in}}%
\pgfpathlineto{\pgfqpoint{3.064773in}{0.449444in}}%
\pgfpathlineto{\pgfqpoint{3.064773in}{0.567132in}}%
\pgfpathlineto{\pgfqpoint{3.001364in}{0.567132in}}%
\pgfpathlineto{\pgfqpoint{3.001364in}{0.449444in}}%
\pgfpathclose%
\pgfusepath{fill}%
\end{pgfscope}%
\begin{pgfscope}%
\pgfpathrectangle{\pgfqpoint{0.465000in}{0.449444in}}{\pgfqpoint{3.487500in}{1.155000in}}%
\pgfusepath{clip}%
\pgfsetbuttcap%
\pgfsetmiterjoin%
\definecolor{currentfill}{rgb}{0.000000,0.000000,0.000000}%
\pgfsetfillcolor{currentfill}%
\pgfsetlinewidth{0.000000pt}%
\definecolor{currentstroke}{rgb}{0.000000,0.000000,0.000000}%
\pgfsetstrokecolor{currentstroke}%
\pgfsetstrokeopacity{0.000000}%
\pgfsetdash{}{0pt}%
\pgfpathmoveto{\pgfqpoint{3.159887in}{0.449444in}}%
\pgfpathlineto{\pgfqpoint{3.223296in}{0.449444in}}%
\pgfpathlineto{\pgfqpoint{3.223296in}{0.533008in}}%
\pgfpathlineto{\pgfqpoint{3.159887in}{0.533008in}}%
\pgfpathlineto{\pgfqpoint{3.159887in}{0.449444in}}%
\pgfpathclose%
\pgfusepath{fill}%
\end{pgfscope}%
\begin{pgfscope}%
\pgfpathrectangle{\pgfqpoint{0.465000in}{0.449444in}}{\pgfqpoint{3.487500in}{1.155000in}}%
\pgfusepath{clip}%
\pgfsetbuttcap%
\pgfsetmiterjoin%
\definecolor{currentfill}{rgb}{0.000000,0.000000,0.000000}%
\pgfsetfillcolor{currentfill}%
\pgfsetlinewidth{0.000000pt}%
\definecolor{currentstroke}{rgb}{0.000000,0.000000,0.000000}%
\pgfsetstrokecolor{currentstroke}%
\pgfsetstrokeopacity{0.000000}%
\pgfsetdash{}{0pt}%
\pgfpathmoveto{\pgfqpoint{3.318409in}{0.449444in}}%
\pgfpathlineto{\pgfqpoint{3.381818in}{0.449444in}}%
\pgfpathlineto{\pgfqpoint{3.381818in}{0.507100in}}%
\pgfpathlineto{\pgfqpoint{3.318409in}{0.507100in}}%
\pgfpathlineto{\pgfqpoint{3.318409in}{0.449444in}}%
\pgfpathclose%
\pgfusepath{fill}%
\end{pgfscope}%
\begin{pgfscope}%
\pgfpathrectangle{\pgfqpoint{0.465000in}{0.449444in}}{\pgfqpoint{3.487500in}{1.155000in}}%
\pgfusepath{clip}%
\pgfsetbuttcap%
\pgfsetmiterjoin%
\definecolor{currentfill}{rgb}{0.000000,0.000000,0.000000}%
\pgfsetfillcolor{currentfill}%
\pgfsetlinewidth{0.000000pt}%
\definecolor{currentstroke}{rgb}{0.000000,0.000000,0.000000}%
\pgfsetstrokecolor{currentstroke}%
\pgfsetstrokeopacity{0.000000}%
\pgfsetdash{}{0pt}%
\pgfpathmoveto{\pgfqpoint{3.476932in}{0.449444in}}%
\pgfpathlineto{\pgfqpoint{3.540341in}{0.449444in}}%
\pgfpathlineto{\pgfqpoint{3.540341in}{0.487238in}}%
\pgfpathlineto{\pgfqpoint{3.476932in}{0.487238in}}%
\pgfpathlineto{\pgfqpoint{3.476932in}{0.449444in}}%
\pgfpathclose%
\pgfusepath{fill}%
\end{pgfscope}%
\begin{pgfscope}%
\pgfpathrectangle{\pgfqpoint{0.465000in}{0.449444in}}{\pgfqpoint{3.487500in}{1.155000in}}%
\pgfusepath{clip}%
\pgfsetbuttcap%
\pgfsetmiterjoin%
\definecolor{currentfill}{rgb}{0.000000,0.000000,0.000000}%
\pgfsetfillcolor{currentfill}%
\pgfsetlinewidth{0.000000pt}%
\definecolor{currentstroke}{rgb}{0.000000,0.000000,0.000000}%
\pgfsetstrokecolor{currentstroke}%
\pgfsetstrokeopacity{0.000000}%
\pgfsetdash{}{0pt}%
\pgfpathmoveto{\pgfqpoint{3.635455in}{0.449444in}}%
\pgfpathlineto{\pgfqpoint{3.698864in}{0.449444in}}%
\pgfpathlineto{\pgfqpoint{3.698864in}{0.472672in}}%
\pgfpathlineto{\pgfqpoint{3.635455in}{0.472672in}}%
\pgfpathlineto{\pgfqpoint{3.635455in}{0.449444in}}%
\pgfpathclose%
\pgfusepath{fill}%
\end{pgfscope}%
\begin{pgfscope}%
\pgfpathrectangle{\pgfqpoint{0.465000in}{0.449444in}}{\pgfqpoint{3.487500in}{1.155000in}}%
\pgfusepath{clip}%
\pgfsetbuttcap%
\pgfsetmiterjoin%
\definecolor{currentfill}{rgb}{0.000000,0.000000,0.000000}%
\pgfsetfillcolor{currentfill}%
\pgfsetlinewidth{0.000000pt}%
\definecolor{currentstroke}{rgb}{0.000000,0.000000,0.000000}%
\pgfsetstrokecolor{currentstroke}%
\pgfsetstrokeopacity{0.000000}%
\pgfsetdash{}{0pt}%
\pgfpathmoveto{\pgfqpoint{3.793978in}{0.449444in}}%
\pgfpathlineto{\pgfqpoint{3.857387in}{0.449444in}}%
\pgfpathlineto{\pgfqpoint{3.857387in}{0.481893in}}%
\pgfpathlineto{\pgfqpoint{3.793978in}{0.481893in}}%
\pgfpathlineto{\pgfqpoint{3.793978in}{0.449444in}}%
\pgfpathclose%
\pgfusepath{fill}%
\end{pgfscope}%
\begin{pgfscope}%
\pgfsetbuttcap%
\pgfsetroundjoin%
\definecolor{currentfill}{rgb}{0.000000,0.000000,0.000000}%
\pgfsetfillcolor{currentfill}%
\pgfsetlinewidth{0.803000pt}%
\definecolor{currentstroke}{rgb}{0.000000,0.000000,0.000000}%
\pgfsetstrokecolor{currentstroke}%
\pgfsetdash{}{0pt}%
\pgfsys@defobject{currentmarker}{\pgfqpoint{0.000000in}{-0.048611in}}{\pgfqpoint{0.000000in}{0.000000in}}{%
\pgfpathmoveto{\pgfqpoint{0.000000in}{0.000000in}}%
\pgfpathlineto{\pgfqpoint{0.000000in}{-0.048611in}}%
\pgfusepath{stroke,fill}%
}%
\begin{pgfscope}%
\pgfsys@transformshift{0.465000in}{0.449444in}%
\pgfsys@useobject{currentmarker}{}%
\end{pgfscope}%
\end{pgfscope}%
\begin{pgfscope}%
\pgfsetbuttcap%
\pgfsetroundjoin%
\definecolor{currentfill}{rgb}{0.000000,0.000000,0.000000}%
\pgfsetfillcolor{currentfill}%
\pgfsetlinewidth{0.803000pt}%
\definecolor{currentstroke}{rgb}{0.000000,0.000000,0.000000}%
\pgfsetstrokecolor{currentstroke}%
\pgfsetdash{}{0pt}%
\pgfsys@defobject{currentmarker}{\pgfqpoint{0.000000in}{-0.048611in}}{\pgfqpoint{0.000000in}{0.000000in}}{%
\pgfpathmoveto{\pgfqpoint{0.000000in}{0.000000in}}%
\pgfpathlineto{\pgfqpoint{0.000000in}{-0.048611in}}%
\pgfusepath{stroke,fill}%
}%
\begin{pgfscope}%
\pgfsys@transformshift{0.623523in}{0.449444in}%
\pgfsys@useobject{currentmarker}{}%
\end{pgfscope}%
\end{pgfscope}%
\begin{pgfscope}%
\definecolor{textcolor}{rgb}{0.000000,0.000000,0.000000}%
\pgfsetstrokecolor{textcolor}%
\pgfsetfillcolor{textcolor}%
\pgftext[x=0.623523in,y=0.352222in,,top]{\color{textcolor}\rmfamily\fontsize{10.000000}{12.000000}\selectfont 0.0}%
\end{pgfscope}%
\begin{pgfscope}%
\pgfsetbuttcap%
\pgfsetroundjoin%
\definecolor{currentfill}{rgb}{0.000000,0.000000,0.000000}%
\pgfsetfillcolor{currentfill}%
\pgfsetlinewidth{0.803000pt}%
\definecolor{currentstroke}{rgb}{0.000000,0.000000,0.000000}%
\pgfsetstrokecolor{currentstroke}%
\pgfsetdash{}{0pt}%
\pgfsys@defobject{currentmarker}{\pgfqpoint{0.000000in}{-0.048611in}}{\pgfqpoint{0.000000in}{0.000000in}}{%
\pgfpathmoveto{\pgfqpoint{0.000000in}{0.000000in}}%
\pgfpathlineto{\pgfqpoint{0.000000in}{-0.048611in}}%
\pgfusepath{stroke,fill}%
}%
\begin{pgfscope}%
\pgfsys@transformshift{0.782046in}{0.449444in}%
\pgfsys@useobject{currentmarker}{}%
\end{pgfscope}%
\end{pgfscope}%
\begin{pgfscope}%
\pgfsetbuttcap%
\pgfsetroundjoin%
\definecolor{currentfill}{rgb}{0.000000,0.000000,0.000000}%
\pgfsetfillcolor{currentfill}%
\pgfsetlinewidth{0.803000pt}%
\definecolor{currentstroke}{rgb}{0.000000,0.000000,0.000000}%
\pgfsetstrokecolor{currentstroke}%
\pgfsetdash{}{0pt}%
\pgfsys@defobject{currentmarker}{\pgfqpoint{0.000000in}{-0.048611in}}{\pgfqpoint{0.000000in}{0.000000in}}{%
\pgfpathmoveto{\pgfqpoint{0.000000in}{0.000000in}}%
\pgfpathlineto{\pgfqpoint{0.000000in}{-0.048611in}}%
\pgfusepath{stroke,fill}%
}%
\begin{pgfscope}%
\pgfsys@transformshift{0.940568in}{0.449444in}%
\pgfsys@useobject{currentmarker}{}%
\end{pgfscope}%
\end{pgfscope}%
\begin{pgfscope}%
\definecolor{textcolor}{rgb}{0.000000,0.000000,0.000000}%
\pgfsetstrokecolor{textcolor}%
\pgfsetfillcolor{textcolor}%
\pgftext[x=0.940568in,y=0.352222in,,top]{\color{textcolor}\rmfamily\fontsize{10.000000}{12.000000}\selectfont 0.1}%
\end{pgfscope}%
\begin{pgfscope}%
\pgfsetbuttcap%
\pgfsetroundjoin%
\definecolor{currentfill}{rgb}{0.000000,0.000000,0.000000}%
\pgfsetfillcolor{currentfill}%
\pgfsetlinewidth{0.803000pt}%
\definecolor{currentstroke}{rgb}{0.000000,0.000000,0.000000}%
\pgfsetstrokecolor{currentstroke}%
\pgfsetdash{}{0pt}%
\pgfsys@defobject{currentmarker}{\pgfqpoint{0.000000in}{-0.048611in}}{\pgfqpoint{0.000000in}{0.000000in}}{%
\pgfpathmoveto{\pgfqpoint{0.000000in}{0.000000in}}%
\pgfpathlineto{\pgfqpoint{0.000000in}{-0.048611in}}%
\pgfusepath{stroke,fill}%
}%
\begin{pgfscope}%
\pgfsys@transformshift{1.099091in}{0.449444in}%
\pgfsys@useobject{currentmarker}{}%
\end{pgfscope}%
\end{pgfscope}%
\begin{pgfscope}%
\pgfsetbuttcap%
\pgfsetroundjoin%
\definecolor{currentfill}{rgb}{0.000000,0.000000,0.000000}%
\pgfsetfillcolor{currentfill}%
\pgfsetlinewidth{0.803000pt}%
\definecolor{currentstroke}{rgb}{0.000000,0.000000,0.000000}%
\pgfsetstrokecolor{currentstroke}%
\pgfsetdash{}{0pt}%
\pgfsys@defobject{currentmarker}{\pgfqpoint{0.000000in}{-0.048611in}}{\pgfqpoint{0.000000in}{0.000000in}}{%
\pgfpathmoveto{\pgfqpoint{0.000000in}{0.000000in}}%
\pgfpathlineto{\pgfqpoint{0.000000in}{-0.048611in}}%
\pgfusepath{stroke,fill}%
}%
\begin{pgfscope}%
\pgfsys@transformshift{1.257614in}{0.449444in}%
\pgfsys@useobject{currentmarker}{}%
\end{pgfscope}%
\end{pgfscope}%
\begin{pgfscope}%
\definecolor{textcolor}{rgb}{0.000000,0.000000,0.000000}%
\pgfsetstrokecolor{textcolor}%
\pgfsetfillcolor{textcolor}%
\pgftext[x=1.257614in,y=0.352222in,,top]{\color{textcolor}\rmfamily\fontsize{10.000000}{12.000000}\selectfont 0.2}%
\end{pgfscope}%
\begin{pgfscope}%
\pgfsetbuttcap%
\pgfsetroundjoin%
\definecolor{currentfill}{rgb}{0.000000,0.000000,0.000000}%
\pgfsetfillcolor{currentfill}%
\pgfsetlinewidth{0.803000pt}%
\definecolor{currentstroke}{rgb}{0.000000,0.000000,0.000000}%
\pgfsetstrokecolor{currentstroke}%
\pgfsetdash{}{0pt}%
\pgfsys@defobject{currentmarker}{\pgfqpoint{0.000000in}{-0.048611in}}{\pgfqpoint{0.000000in}{0.000000in}}{%
\pgfpathmoveto{\pgfqpoint{0.000000in}{0.000000in}}%
\pgfpathlineto{\pgfqpoint{0.000000in}{-0.048611in}}%
\pgfusepath{stroke,fill}%
}%
\begin{pgfscope}%
\pgfsys@transformshift{1.416137in}{0.449444in}%
\pgfsys@useobject{currentmarker}{}%
\end{pgfscope}%
\end{pgfscope}%
\begin{pgfscope}%
\pgfsetbuttcap%
\pgfsetroundjoin%
\definecolor{currentfill}{rgb}{0.000000,0.000000,0.000000}%
\pgfsetfillcolor{currentfill}%
\pgfsetlinewidth{0.803000pt}%
\definecolor{currentstroke}{rgb}{0.000000,0.000000,0.000000}%
\pgfsetstrokecolor{currentstroke}%
\pgfsetdash{}{0pt}%
\pgfsys@defobject{currentmarker}{\pgfqpoint{0.000000in}{-0.048611in}}{\pgfqpoint{0.000000in}{0.000000in}}{%
\pgfpathmoveto{\pgfqpoint{0.000000in}{0.000000in}}%
\pgfpathlineto{\pgfqpoint{0.000000in}{-0.048611in}}%
\pgfusepath{stroke,fill}%
}%
\begin{pgfscope}%
\pgfsys@transformshift{1.574659in}{0.449444in}%
\pgfsys@useobject{currentmarker}{}%
\end{pgfscope}%
\end{pgfscope}%
\begin{pgfscope}%
\definecolor{textcolor}{rgb}{0.000000,0.000000,0.000000}%
\pgfsetstrokecolor{textcolor}%
\pgfsetfillcolor{textcolor}%
\pgftext[x=1.574659in,y=0.352222in,,top]{\color{textcolor}\rmfamily\fontsize{10.000000}{12.000000}\selectfont 0.3}%
\end{pgfscope}%
\begin{pgfscope}%
\pgfsetbuttcap%
\pgfsetroundjoin%
\definecolor{currentfill}{rgb}{0.000000,0.000000,0.000000}%
\pgfsetfillcolor{currentfill}%
\pgfsetlinewidth{0.803000pt}%
\definecolor{currentstroke}{rgb}{0.000000,0.000000,0.000000}%
\pgfsetstrokecolor{currentstroke}%
\pgfsetdash{}{0pt}%
\pgfsys@defobject{currentmarker}{\pgfqpoint{0.000000in}{-0.048611in}}{\pgfqpoint{0.000000in}{0.000000in}}{%
\pgfpathmoveto{\pgfqpoint{0.000000in}{0.000000in}}%
\pgfpathlineto{\pgfqpoint{0.000000in}{-0.048611in}}%
\pgfusepath{stroke,fill}%
}%
\begin{pgfscope}%
\pgfsys@transformshift{1.733182in}{0.449444in}%
\pgfsys@useobject{currentmarker}{}%
\end{pgfscope}%
\end{pgfscope}%
\begin{pgfscope}%
\pgfsetbuttcap%
\pgfsetroundjoin%
\definecolor{currentfill}{rgb}{0.000000,0.000000,0.000000}%
\pgfsetfillcolor{currentfill}%
\pgfsetlinewidth{0.803000pt}%
\definecolor{currentstroke}{rgb}{0.000000,0.000000,0.000000}%
\pgfsetstrokecolor{currentstroke}%
\pgfsetdash{}{0pt}%
\pgfsys@defobject{currentmarker}{\pgfqpoint{0.000000in}{-0.048611in}}{\pgfqpoint{0.000000in}{0.000000in}}{%
\pgfpathmoveto{\pgfqpoint{0.000000in}{0.000000in}}%
\pgfpathlineto{\pgfqpoint{0.000000in}{-0.048611in}}%
\pgfusepath{stroke,fill}%
}%
\begin{pgfscope}%
\pgfsys@transformshift{1.891705in}{0.449444in}%
\pgfsys@useobject{currentmarker}{}%
\end{pgfscope}%
\end{pgfscope}%
\begin{pgfscope}%
\definecolor{textcolor}{rgb}{0.000000,0.000000,0.000000}%
\pgfsetstrokecolor{textcolor}%
\pgfsetfillcolor{textcolor}%
\pgftext[x=1.891705in,y=0.352222in,,top]{\color{textcolor}\rmfamily\fontsize{10.000000}{12.000000}\selectfont 0.4}%
\end{pgfscope}%
\begin{pgfscope}%
\pgfsetbuttcap%
\pgfsetroundjoin%
\definecolor{currentfill}{rgb}{0.000000,0.000000,0.000000}%
\pgfsetfillcolor{currentfill}%
\pgfsetlinewidth{0.803000pt}%
\definecolor{currentstroke}{rgb}{0.000000,0.000000,0.000000}%
\pgfsetstrokecolor{currentstroke}%
\pgfsetdash{}{0pt}%
\pgfsys@defobject{currentmarker}{\pgfqpoint{0.000000in}{-0.048611in}}{\pgfqpoint{0.000000in}{0.000000in}}{%
\pgfpathmoveto{\pgfqpoint{0.000000in}{0.000000in}}%
\pgfpathlineto{\pgfqpoint{0.000000in}{-0.048611in}}%
\pgfusepath{stroke,fill}%
}%
\begin{pgfscope}%
\pgfsys@transformshift{2.050228in}{0.449444in}%
\pgfsys@useobject{currentmarker}{}%
\end{pgfscope}%
\end{pgfscope}%
\begin{pgfscope}%
\pgfsetbuttcap%
\pgfsetroundjoin%
\definecolor{currentfill}{rgb}{0.000000,0.000000,0.000000}%
\pgfsetfillcolor{currentfill}%
\pgfsetlinewidth{0.803000pt}%
\definecolor{currentstroke}{rgb}{0.000000,0.000000,0.000000}%
\pgfsetstrokecolor{currentstroke}%
\pgfsetdash{}{0pt}%
\pgfsys@defobject{currentmarker}{\pgfqpoint{0.000000in}{-0.048611in}}{\pgfqpoint{0.000000in}{0.000000in}}{%
\pgfpathmoveto{\pgfqpoint{0.000000in}{0.000000in}}%
\pgfpathlineto{\pgfqpoint{0.000000in}{-0.048611in}}%
\pgfusepath{stroke,fill}%
}%
\begin{pgfscope}%
\pgfsys@transformshift{2.208750in}{0.449444in}%
\pgfsys@useobject{currentmarker}{}%
\end{pgfscope}%
\end{pgfscope}%
\begin{pgfscope}%
\definecolor{textcolor}{rgb}{0.000000,0.000000,0.000000}%
\pgfsetstrokecolor{textcolor}%
\pgfsetfillcolor{textcolor}%
\pgftext[x=2.208750in,y=0.352222in,,top]{\color{textcolor}\rmfamily\fontsize{10.000000}{12.000000}\selectfont 0.5}%
\end{pgfscope}%
\begin{pgfscope}%
\pgfsetbuttcap%
\pgfsetroundjoin%
\definecolor{currentfill}{rgb}{0.000000,0.000000,0.000000}%
\pgfsetfillcolor{currentfill}%
\pgfsetlinewidth{0.803000pt}%
\definecolor{currentstroke}{rgb}{0.000000,0.000000,0.000000}%
\pgfsetstrokecolor{currentstroke}%
\pgfsetdash{}{0pt}%
\pgfsys@defobject{currentmarker}{\pgfqpoint{0.000000in}{-0.048611in}}{\pgfqpoint{0.000000in}{0.000000in}}{%
\pgfpathmoveto{\pgfqpoint{0.000000in}{0.000000in}}%
\pgfpathlineto{\pgfqpoint{0.000000in}{-0.048611in}}%
\pgfusepath{stroke,fill}%
}%
\begin{pgfscope}%
\pgfsys@transformshift{2.367273in}{0.449444in}%
\pgfsys@useobject{currentmarker}{}%
\end{pgfscope}%
\end{pgfscope}%
\begin{pgfscope}%
\pgfsetbuttcap%
\pgfsetroundjoin%
\definecolor{currentfill}{rgb}{0.000000,0.000000,0.000000}%
\pgfsetfillcolor{currentfill}%
\pgfsetlinewidth{0.803000pt}%
\definecolor{currentstroke}{rgb}{0.000000,0.000000,0.000000}%
\pgfsetstrokecolor{currentstroke}%
\pgfsetdash{}{0pt}%
\pgfsys@defobject{currentmarker}{\pgfqpoint{0.000000in}{-0.048611in}}{\pgfqpoint{0.000000in}{0.000000in}}{%
\pgfpathmoveto{\pgfqpoint{0.000000in}{0.000000in}}%
\pgfpathlineto{\pgfqpoint{0.000000in}{-0.048611in}}%
\pgfusepath{stroke,fill}%
}%
\begin{pgfscope}%
\pgfsys@transformshift{2.525796in}{0.449444in}%
\pgfsys@useobject{currentmarker}{}%
\end{pgfscope}%
\end{pgfscope}%
\begin{pgfscope}%
\definecolor{textcolor}{rgb}{0.000000,0.000000,0.000000}%
\pgfsetstrokecolor{textcolor}%
\pgfsetfillcolor{textcolor}%
\pgftext[x=2.525796in,y=0.352222in,,top]{\color{textcolor}\rmfamily\fontsize{10.000000}{12.000000}\selectfont 0.6}%
\end{pgfscope}%
\begin{pgfscope}%
\pgfsetbuttcap%
\pgfsetroundjoin%
\definecolor{currentfill}{rgb}{0.000000,0.000000,0.000000}%
\pgfsetfillcolor{currentfill}%
\pgfsetlinewidth{0.803000pt}%
\definecolor{currentstroke}{rgb}{0.000000,0.000000,0.000000}%
\pgfsetstrokecolor{currentstroke}%
\pgfsetdash{}{0pt}%
\pgfsys@defobject{currentmarker}{\pgfqpoint{0.000000in}{-0.048611in}}{\pgfqpoint{0.000000in}{0.000000in}}{%
\pgfpathmoveto{\pgfqpoint{0.000000in}{0.000000in}}%
\pgfpathlineto{\pgfqpoint{0.000000in}{-0.048611in}}%
\pgfusepath{stroke,fill}%
}%
\begin{pgfscope}%
\pgfsys@transformshift{2.684318in}{0.449444in}%
\pgfsys@useobject{currentmarker}{}%
\end{pgfscope}%
\end{pgfscope}%
\begin{pgfscope}%
\pgfsetbuttcap%
\pgfsetroundjoin%
\definecolor{currentfill}{rgb}{0.000000,0.000000,0.000000}%
\pgfsetfillcolor{currentfill}%
\pgfsetlinewidth{0.803000pt}%
\definecolor{currentstroke}{rgb}{0.000000,0.000000,0.000000}%
\pgfsetstrokecolor{currentstroke}%
\pgfsetdash{}{0pt}%
\pgfsys@defobject{currentmarker}{\pgfqpoint{0.000000in}{-0.048611in}}{\pgfqpoint{0.000000in}{0.000000in}}{%
\pgfpathmoveto{\pgfqpoint{0.000000in}{0.000000in}}%
\pgfpathlineto{\pgfqpoint{0.000000in}{-0.048611in}}%
\pgfusepath{stroke,fill}%
}%
\begin{pgfscope}%
\pgfsys@transformshift{2.842841in}{0.449444in}%
\pgfsys@useobject{currentmarker}{}%
\end{pgfscope}%
\end{pgfscope}%
\begin{pgfscope}%
\definecolor{textcolor}{rgb}{0.000000,0.000000,0.000000}%
\pgfsetstrokecolor{textcolor}%
\pgfsetfillcolor{textcolor}%
\pgftext[x=2.842841in,y=0.352222in,,top]{\color{textcolor}\rmfamily\fontsize{10.000000}{12.000000}\selectfont 0.7}%
\end{pgfscope}%
\begin{pgfscope}%
\pgfsetbuttcap%
\pgfsetroundjoin%
\definecolor{currentfill}{rgb}{0.000000,0.000000,0.000000}%
\pgfsetfillcolor{currentfill}%
\pgfsetlinewidth{0.803000pt}%
\definecolor{currentstroke}{rgb}{0.000000,0.000000,0.000000}%
\pgfsetstrokecolor{currentstroke}%
\pgfsetdash{}{0pt}%
\pgfsys@defobject{currentmarker}{\pgfqpoint{0.000000in}{-0.048611in}}{\pgfqpoint{0.000000in}{0.000000in}}{%
\pgfpathmoveto{\pgfqpoint{0.000000in}{0.000000in}}%
\pgfpathlineto{\pgfqpoint{0.000000in}{-0.048611in}}%
\pgfusepath{stroke,fill}%
}%
\begin{pgfscope}%
\pgfsys@transformshift{3.001364in}{0.449444in}%
\pgfsys@useobject{currentmarker}{}%
\end{pgfscope}%
\end{pgfscope}%
\begin{pgfscope}%
\pgfsetbuttcap%
\pgfsetroundjoin%
\definecolor{currentfill}{rgb}{0.000000,0.000000,0.000000}%
\pgfsetfillcolor{currentfill}%
\pgfsetlinewidth{0.803000pt}%
\definecolor{currentstroke}{rgb}{0.000000,0.000000,0.000000}%
\pgfsetstrokecolor{currentstroke}%
\pgfsetdash{}{0pt}%
\pgfsys@defobject{currentmarker}{\pgfqpoint{0.000000in}{-0.048611in}}{\pgfqpoint{0.000000in}{0.000000in}}{%
\pgfpathmoveto{\pgfqpoint{0.000000in}{0.000000in}}%
\pgfpathlineto{\pgfqpoint{0.000000in}{-0.048611in}}%
\pgfusepath{stroke,fill}%
}%
\begin{pgfscope}%
\pgfsys@transformshift{3.159887in}{0.449444in}%
\pgfsys@useobject{currentmarker}{}%
\end{pgfscope}%
\end{pgfscope}%
\begin{pgfscope}%
\definecolor{textcolor}{rgb}{0.000000,0.000000,0.000000}%
\pgfsetstrokecolor{textcolor}%
\pgfsetfillcolor{textcolor}%
\pgftext[x=3.159887in,y=0.352222in,,top]{\color{textcolor}\rmfamily\fontsize{10.000000}{12.000000}\selectfont 0.8}%
\end{pgfscope}%
\begin{pgfscope}%
\pgfsetbuttcap%
\pgfsetroundjoin%
\definecolor{currentfill}{rgb}{0.000000,0.000000,0.000000}%
\pgfsetfillcolor{currentfill}%
\pgfsetlinewidth{0.803000pt}%
\definecolor{currentstroke}{rgb}{0.000000,0.000000,0.000000}%
\pgfsetstrokecolor{currentstroke}%
\pgfsetdash{}{0pt}%
\pgfsys@defobject{currentmarker}{\pgfqpoint{0.000000in}{-0.048611in}}{\pgfqpoint{0.000000in}{0.000000in}}{%
\pgfpathmoveto{\pgfqpoint{0.000000in}{0.000000in}}%
\pgfpathlineto{\pgfqpoint{0.000000in}{-0.048611in}}%
\pgfusepath{stroke,fill}%
}%
\begin{pgfscope}%
\pgfsys@transformshift{3.318409in}{0.449444in}%
\pgfsys@useobject{currentmarker}{}%
\end{pgfscope}%
\end{pgfscope}%
\begin{pgfscope}%
\pgfsetbuttcap%
\pgfsetroundjoin%
\definecolor{currentfill}{rgb}{0.000000,0.000000,0.000000}%
\pgfsetfillcolor{currentfill}%
\pgfsetlinewidth{0.803000pt}%
\definecolor{currentstroke}{rgb}{0.000000,0.000000,0.000000}%
\pgfsetstrokecolor{currentstroke}%
\pgfsetdash{}{0pt}%
\pgfsys@defobject{currentmarker}{\pgfqpoint{0.000000in}{-0.048611in}}{\pgfqpoint{0.000000in}{0.000000in}}{%
\pgfpathmoveto{\pgfqpoint{0.000000in}{0.000000in}}%
\pgfpathlineto{\pgfqpoint{0.000000in}{-0.048611in}}%
\pgfusepath{stroke,fill}%
}%
\begin{pgfscope}%
\pgfsys@transformshift{3.476932in}{0.449444in}%
\pgfsys@useobject{currentmarker}{}%
\end{pgfscope}%
\end{pgfscope}%
\begin{pgfscope}%
\definecolor{textcolor}{rgb}{0.000000,0.000000,0.000000}%
\pgfsetstrokecolor{textcolor}%
\pgfsetfillcolor{textcolor}%
\pgftext[x=3.476932in,y=0.352222in,,top]{\color{textcolor}\rmfamily\fontsize{10.000000}{12.000000}\selectfont 0.9}%
\end{pgfscope}%
\begin{pgfscope}%
\pgfsetbuttcap%
\pgfsetroundjoin%
\definecolor{currentfill}{rgb}{0.000000,0.000000,0.000000}%
\pgfsetfillcolor{currentfill}%
\pgfsetlinewidth{0.803000pt}%
\definecolor{currentstroke}{rgb}{0.000000,0.000000,0.000000}%
\pgfsetstrokecolor{currentstroke}%
\pgfsetdash{}{0pt}%
\pgfsys@defobject{currentmarker}{\pgfqpoint{0.000000in}{-0.048611in}}{\pgfqpoint{0.000000in}{0.000000in}}{%
\pgfpathmoveto{\pgfqpoint{0.000000in}{0.000000in}}%
\pgfpathlineto{\pgfqpoint{0.000000in}{-0.048611in}}%
\pgfusepath{stroke,fill}%
}%
\begin{pgfscope}%
\pgfsys@transformshift{3.635455in}{0.449444in}%
\pgfsys@useobject{currentmarker}{}%
\end{pgfscope}%
\end{pgfscope}%
\begin{pgfscope}%
\pgfsetbuttcap%
\pgfsetroundjoin%
\definecolor{currentfill}{rgb}{0.000000,0.000000,0.000000}%
\pgfsetfillcolor{currentfill}%
\pgfsetlinewidth{0.803000pt}%
\definecolor{currentstroke}{rgb}{0.000000,0.000000,0.000000}%
\pgfsetstrokecolor{currentstroke}%
\pgfsetdash{}{0pt}%
\pgfsys@defobject{currentmarker}{\pgfqpoint{0.000000in}{-0.048611in}}{\pgfqpoint{0.000000in}{0.000000in}}{%
\pgfpathmoveto{\pgfqpoint{0.000000in}{0.000000in}}%
\pgfpathlineto{\pgfqpoint{0.000000in}{-0.048611in}}%
\pgfusepath{stroke,fill}%
}%
\begin{pgfscope}%
\pgfsys@transformshift{3.793978in}{0.449444in}%
\pgfsys@useobject{currentmarker}{}%
\end{pgfscope}%
\end{pgfscope}%
\begin{pgfscope}%
\definecolor{textcolor}{rgb}{0.000000,0.000000,0.000000}%
\pgfsetstrokecolor{textcolor}%
\pgfsetfillcolor{textcolor}%
\pgftext[x=3.793978in,y=0.352222in,,top]{\color{textcolor}\rmfamily\fontsize{10.000000}{12.000000}\selectfont 1.0}%
\end{pgfscope}%
\begin{pgfscope}%
\pgfsetbuttcap%
\pgfsetroundjoin%
\definecolor{currentfill}{rgb}{0.000000,0.000000,0.000000}%
\pgfsetfillcolor{currentfill}%
\pgfsetlinewidth{0.803000pt}%
\definecolor{currentstroke}{rgb}{0.000000,0.000000,0.000000}%
\pgfsetstrokecolor{currentstroke}%
\pgfsetdash{}{0pt}%
\pgfsys@defobject{currentmarker}{\pgfqpoint{0.000000in}{-0.048611in}}{\pgfqpoint{0.000000in}{0.000000in}}{%
\pgfpathmoveto{\pgfqpoint{0.000000in}{0.000000in}}%
\pgfpathlineto{\pgfqpoint{0.000000in}{-0.048611in}}%
\pgfusepath{stroke,fill}%
}%
\begin{pgfscope}%
\pgfsys@transformshift{3.952500in}{0.449444in}%
\pgfsys@useobject{currentmarker}{}%
\end{pgfscope}%
\end{pgfscope}%
\begin{pgfscope}%
\definecolor{textcolor}{rgb}{0.000000,0.000000,0.000000}%
\pgfsetstrokecolor{textcolor}%
\pgfsetfillcolor{textcolor}%
\pgftext[x=2.208750in,y=0.173333in,,top]{\color{textcolor}\rmfamily\fontsize{10.000000}{12.000000}\selectfont \(\displaystyle p\)}%
\end{pgfscope}%
\begin{pgfscope}%
\pgfsetbuttcap%
\pgfsetroundjoin%
\definecolor{currentfill}{rgb}{0.000000,0.000000,0.000000}%
\pgfsetfillcolor{currentfill}%
\pgfsetlinewidth{0.803000pt}%
\definecolor{currentstroke}{rgb}{0.000000,0.000000,0.000000}%
\pgfsetstrokecolor{currentstroke}%
\pgfsetdash{}{0pt}%
\pgfsys@defobject{currentmarker}{\pgfqpoint{-0.048611in}{0.000000in}}{\pgfqpoint{-0.000000in}{0.000000in}}{%
\pgfpathmoveto{\pgfqpoint{-0.000000in}{0.000000in}}%
\pgfpathlineto{\pgfqpoint{-0.048611in}{0.000000in}}%
\pgfusepath{stroke,fill}%
}%
\begin{pgfscope}%
\pgfsys@transformshift{0.465000in}{0.449444in}%
\pgfsys@useobject{currentmarker}{}%
\end{pgfscope}%
\end{pgfscope}%
\begin{pgfscope}%
\definecolor{textcolor}{rgb}{0.000000,0.000000,0.000000}%
\pgfsetstrokecolor{textcolor}%
\pgfsetfillcolor{textcolor}%
\pgftext[x=0.298333in, y=0.401250in, left, base]{\color{textcolor}\rmfamily\fontsize{10.000000}{12.000000}\selectfont \(\displaystyle {0}\)}%
\end{pgfscope}%
\begin{pgfscope}%
\pgfsetbuttcap%
\pgfsetroundjoin%
\definecolor{currentfill}{rgb}{0.000000,0.000000,0.000000}%
\pgfsetfillcolor{currentfill}%
\pgfsetlinewidth{0.803000pt}%
\definecolor{currentstroke}{rgb}{0.000000,0.000000,0.000000}%
\pgfsetstrokecolor{currentstroke}%
\pgfsetdash{}{0pt}%
\pgfsys@defobject{currentmarker}{\pgfqpoint{-0.048611in}{0.000000in}}{\pgfqpoint{-0.000000in}{0.000000in}}{%
\pgfpathmoveto{\pgfqpoint{-0.000000in}{0.000000in}}%
\pgfpathlineto{\pgfqpoint{-0.048611in}{0.000000in}}%
\pgfusepath{stroke,fill}%
}%
\begin{pgfscope}%
\pgfsys@transformshift{0.465000in}{1.018633in}%
\pgfsys@useobject{currentmarker}{}%
\end{pgfscope}%
\end{pgfscope}%
\begin{pgfscope}%
\definecolor{textcolor}{rgb}{0.000000,0.000000,0.000000}%
\pgfsetstrokecolor{textcolor}%
\pgfsetfillcolor{textcolor}%
\pgftext[x=0.298333in, y=0.970438in, left, base]{\color{textcolor}\rmfamily\fontsize{10.000000}{12.000000}\selectfont \(\displaystyle {5}\)}%
\end{pgfscope}%
\begin{pgfscope}%
\pgfsetbuttcap%
\pgfsetroundjoin%
\definecolor{currentfill}{rgb}{0.000000,0.000000,0.000000}%
\pgfsetfillcolor{currentfill}%
\pgfsetlinewidth{0.803000pt}%
\definecolor{currentstroke}{rgb}{0.000000,0.000000,0.000000}%
\pgfsetstrokecolor{currentstroke}%
\pgfsetdash{}{0pt}%
\pgfsys@defobject{currentmarker}{\pgfqpoint{-0.048611in}{0.000000in}}{\pgfqpoint{-0.000000in}{0.000000in}}{%
\pgfpathmoveto{\pgfqpoint{-0.000000in}{0.000000in}}%
\pgfpathlineto{\pgfqpoint{-0.048611in}{0.000000in}}%
\pgfusepath{stroke,fill}%
}%
\begin{pgfscope}%
\pgfsys@transformshift{0.465000in}{1.587822in}%
\pgfsys@useobject{currentmarker}{}%
\end{pgfscope}%
\end{pgfscope}%
\begin{pgfscope}%
\definecolor{textcolor}{rgb}{0.000000,0.000000,0.000000}%
\pgfsetstrokecolor{textcolor}%
\pgfsetfillcolor{textcolor}%
\pgftext[x=0.228889in, y=1.539627in, left, base]{\color{textcolor}\rmfamily\fontsize{10.000000}{12.000000}\selectfont \(\displaystyle {10}\)}%
\end{pgfscope}%
\begin{pgfscope}%
\definecolor{textcolor}{rgb}{0.000000,0.000000,0.000000}%
\pgfsetstrokecolor{textcolor}%
\pgfsetfillcolor{textcolor}%
\pgftext[x=0.173333in,y=1.026944in,,bottom,rotate=90.000000]{\color{textcolor}\rmfamily\fontsize{10.000000}{12.000000}\selectfont Percent of Data Set}%
\end{pgfscope}%
\begin{pgfscope}%
\pgfsetrectcap%
\pgfsetmiterjoin%
\pgfsetlinewidth{0.803000pt}%
\definecolor{currentstroke}{rgb}{0.000000,0.000000,0.000000}%
\pgfsetstrokecolor{currentstroke}%
\pgfsetdash{}{0pt}%
\pgfpathmoveto{\pgfqpoint{0.465000in}{0.449444in}}%
\pgfpathlineto{\pgfqpoint{0.465000in}{1.604444in}}%
\pgfusepath{stroke}%
\end{pgfscope}%
\begin{pgfscope}%
\pgfsetrectcap%
\pgfsetmiterjoin%
\pgfsetlinewidth{0.803000pt}%
\definecolor{currentstroke}{rgb}{0.000000,0.000000,0.000000}%
\pgfsetstrokecolor{currentstroke}%
\pgfsetdash{}{0pt}%
\pgfpathmoveto{\pgfqpoint{3.952500in}{0.449444in}}%
\pgfpathlineto{\pgfqpoint{3.952500in}{1.604444in}}%
\pgfusepath{stroke}%
\end{pgfscope}%
\begin{pgfscope}%
\pgfsetrectcap%
\pgfsetmiterjoin%
\pgfsetlinewidth{0.803000pt}%
\definecolor{currentstroke}{rgb}{0.000000,0.000000,0.000000}%
\pgfsetstrokecolor{currentstroke}%
\pgfsetdash{}{0pt}%
\pgfpathmoveto{\pgfqpoint{0.465000in}{0.449444in}}%
\pgfpathlineto{\pgfqpoint{3.952500in}{0.449444in}}%
\pgfusepath{stroke}%
\end{pgfscope}%
\begin{pgfscope}%
\pgfsetrectcap%
\pgfsetmiterjoin%
\pgfsetlinewidth{0.803000pt}%
\definecolor{currentstroke}{rgb}{0.000000,0.000000,0.000000}%
\pgfsetstrokecolor{currentstroke}%
\pgfsetdash{}{0pt}%
\pgfpathmoveto{\pgfqpoint{0.465000in}{1.604444in}}%
\pgfpathlineto{\pgfqpoint{3.952500in}{1.604444in}}%
\pgfusepath{stroke}%
\end{pgfscope}%
\begin{pgfscope}%
\pgfsetbuttcap%
\pgfsetmiterjoin%
\definecolor{currentfill}{rgb}{1.000000,1.000000,1.000000}%
\pgfsetfillcolor{currentfill}%
\pgfsetfillopacity{0.800000}%
\pgfsetlinewidth{1.003750pt}%
\definecolor{currentstroke}{rgb}{0.800000,0.800000,0.800000}%
\pgfsetstrokecolor{currentstroke}%
\pgfsetstrokeopacity{0.800000}%
\pgfsetdash{}{0pt}%
\pgfpathmoveto{\pgfqpoint{3.175556in}{1.104445in}}%
\pgfpathlineto{\pgfqpoint{3.855278in}{1.104445in}}%
\pgfpathquadraticcurveto{\pgfqpoint{3.883056in}{1.104445in}}{\pgfqpoint{3.883056in}{1.132222in}}%
\pgfpathlineto{\pgfqpoint{3.883056in}{1.507222in}}%
\pgfpathquadraticcurveto{\pgfqpoint{3.883056in}{1.535000in}}{\pgfqpoint{3.855278in}{1.535000in}}%
\pgfpathlineto{\pgfqpoint{3.175556in}{1.535000in}}%
\pgfpathquadraticcurveto{\pgfqpoint{3.147778in}{1.535000in}}{\pgfqpoint{3.147778in}{1.507222in}}%
\pgfpathlineto{\pgfqpoint{3.147778in}{1.132222in}}%
\pgfpathquadraticcurveto{\pgfqpoint{3.147778in}{1.104445in}}{\pgfqpoint{3.175556in}{1.104445in}}%
\pgfpathlineto{\pgfqpoint{3.175556in}{1.104445in}}%
\pgfpathclose%
\pgfusepath{stroke,fill}%
\end{pgfscope}%
\begin{pgfscope}%
\pgfsetbuttcap%
\pgfsetmiterjoin%
\pgfsetlinewidth{1.003750pt}%
\definecolor{currentstroke}{rgb}{0.000000,0.000000,0.000000}%
\pgfsetstrokecolor{currentstroke}%
\pgfsetdash{}{0pt}%
\pgfpathmoveto{\pgfqpoint{3.203334in}{1.382222in}}%
\pgfpathlineto{\pgfqpoint{3.481111in}{1.382222in}}%
\pgfpathlineto{\pgfqpoint{3.481111in}{1.479444in}}%
\pgfpathlineto{\pgfqpoint{3.203334in}{1.479444in}}%
\pgfpathlineto{\pgfqpoint{3.203334in}{1.382222in}}%
\pgfpathclose%
\pgfusepath{stroke}%
\end{pgfscope}%
\begin{pgfscope}%
\definecolor{textcolor}{rgb}{0.000000,0.000000,0.000000}%
\pgfsetstrokecolor{textcolor}%
\pgfsetfillcolor{textcolor}%
\pgftext[x=3.592223in,y=1.382222in,left,base]{\color{textcolor}\rmfamily\fontsize{10.000000}{12.000000}\selectfont Neg}%
\end{pgfscope}%
\begin{pgfscope}%
\pgfsetbuttcap%
\pgfsetmiterjoin%
\definecolor{currentfill}{rgb}{0.000000,0.000000,0.000000}%
\pgfsetfillcolor{currentfill}%
\pgfsetlinewidth{0.000000pt}%
\definecolor{currentstroke}{rgb}{0.000000,0.000000,0.000000}%
\pgfsetstrokecolor{currentstroke}%
\pgfsetstrokeopacity{0.000000}%
\pgfsetdash{}{0pt}%
\pgfpathmoveto{\pgfqpoint{3.203334in}{1.186944in}}%
\pgfpathlineto{\pgfqpoint{3.481111in}{1.186944in}}%
\pgfpathlineto{\pgfqpoint{3.481111in}{1.284167in}}%
\pgfpathlineto{\pgfqpoint{3.203334in}{1.284167in}}%
\pgfpathlineto{\pgfqpoint{3.203334in}{1.186944in}}%
\pgfpathclose%
\pgfusepath{fill}%
\end{pgfscope}%
\begin{pgfscope}%
\definecolor{textcolor}{rgb}{0.000000,0.000000,0.000000}%
\pgfsetstrokecolor{textcolor}%
\pgfsetfillcolor{textcolor}%
\pgftext[x=3.592223in,y=1.186944in,left,base]{\color{textcolor}\rmfamily\fontsize{10.000000}{12.000000}\selectfont Pos}%
\end{pgfscope}%
\end{pgfpicture}%
\makeatother%
\endgroup%
	
&
	\vskip 0pt
	\hfil ROC Curve
	
	%% Creator: Matplotlib, PGF backend
%%
%% To include the figure in your LaTeX document, write
%%   \input{<filename>.pgf}
%%
%% Make sure the required packages are loaded in your preamble
%%   \usepackage{pgf}
%%
%% Also ensure that all the required font packages are loaded; for instance,
%% the lmodern package is sometimes necessary when using math font.
%%   \usepackage{lmodern}
%%
%% Figures using additional raster images can only be included by \input if
%% they are in the same directory as the main LaTeX file. For loading figures
%% from other directories you can use the `import` package
%%   \usepackage{import}
%%
%% and then include the figures with
%%   \import{<path to file>}{<filename>.pgf}
%%
%% Matplotlib used the following preamble
%%   
%%   \usepackage{fontspec}
%%   \makeatletter\@ifpackageloaded{underscore}{}{\usepackage[strings]{underscore}}\makeatother
%%
\begingroup%
\makeatletter%
\begin{pgfpicture}%
\pgfpathrectangle{\pgfpointorigin}{\pgfqpoint{2.221861in}{1.754444in}}%
\pgfusepath{use as bounding box, clip}%
\begin{pgfscope}%
\pgfsetbuttcap%
\pgfsetmiterjoin%
\definecolor{currentfill}{rgb}{1.000000,1.000000,1.000000}%
\pgfsetfillcolor{currentfill}%
\pgfsetlinewidth{0.000000pt}%
\definecolor{currentstroke}{rgb}{1.000000,1.000000,1.000000}%
\pgfsetstrokecolor{currentstroke}%
\pgfsetdash{}{0pt}%
\pgfpathmoveto{\pgfqpoint{0.000000in}{0.000000in}}%
\pgfpathlineto{\pgfqpoint{2.221861in}{0.000000in}}%
\pgfpathlineto{\pgfqpoint{2.221861in}{1.754444in}}%
\pgfpathlineto{\pgfqpoint{0.000000in}{1.754444in}}%
\pgfpathlineto{\pgfqpoint{0.000000in}{0.000000in}}%
\pgfpathclose%
\pgfusepath{fill}%
\end{pgfscope}%
\begin{pgfscope}%
\pgfsetbuttcap%
\pgfsetmiterjoin%
\definecolor{currentfill}{rgb}{1.000000,1.000000,1.000000}%
\pgfsetfillcolor{currentfill}%
\pgfsetlinewidth{0.000000pt}%
\definecolor{currentstroke}{rgb}{0.000000,0.000000,0.000000}%
\pgfsetstrokecolor{currentstroke}%
\pgfsetstrokeopacity{0.000000}%
\pgfsetdash{}{0pt}%
\pgfpathmoveto{\pgfqpoint{0.553581in}{0.499444in}}%
\pgfpathlineto{\pgfqpoint{2.103581in}{0.499444in}}%
\pgfpathlineto{\pgfqpoint{2.103581in}{1.654444in}}%
\pgfpathlineto{\pgfqpoint{0.553581in}{1.654444in}}%
\pgfpathlineto{\pgfqpoint{0.553581in}{0.499444in}}%
\pgfpathclose%
\pgfusepath{fill}%
\end{pgfscope}%
\begin{pgfscope}%
\pgfsetbuttcap%
\pgfsetroundjoin%
\definecolor{currentfill}{rgb}{0.000000,0.000000,0.000000}%
\pgfsetfillcolor{currentfill}%
\pgfsetlinewidth{0.803000pt}%
\definecolor{currentstroke}{rgb}{0.000000,0.000000,0.000000}%
\pgfsetstrokecolor{currentstroke}%
\pgfsetdash{}{0pt}%
\pgfsys@defobject{currentmarker}{\pgfqpoint{0.000000in}{-0.048611in}}{\pgfqpoint{0.000000in}{0.000000in}}{%
\pgfpathmoveto{\pgfqpoint{0.000000in}{0.000000in}}%
\pgfpathlineto{\pgfqpoint{0.000000in}{-0.048611in}}%
\pgfusepath{stroke,fill}%
}%
\begin{pgfscope}%
\pgfsys@transformshift{0.624035in}{0.499444in}%
\pgfsys@useobject{currentmarker}{}%
\end{pgfscope}%
\end{pgfscope}%
\begin{pgfscope}%
\definecolor{textcolor}{rgb}{0.000000,0.000000,0.000000}%
\pgfsetstrokecolor{textcolor}%
\pgfsetfillcolor{textcolor}%
\pgftext[x=0.624035in,y=0.402222in,,top]{\color{textcolor}\rmfamily\fontsize{10.000000}{12.000000}\selectfont \(\displaystyle {0.0}\)}%
\end{pgfscope}%
\begin{pgfscope}%
\pgfsetbuttcap%
\pgfsetroundjoin%
\definecolor{currentfill}{rgb}{0.000000,0.000000,0.000000}%
\pgfsetfillcolor{currentfill}%
\pgfsetlinewidth{0.803000pt}%
\definecolor{currentstroke}{rgb}{0.000000,0.000000,0.000000}%
\pgfsetstrokecolor{currentstroke}%
\pgfsetdash{}{0pt}%
\pgfsys@defobject{currentmarker}{\pgfqpoint{0.000000in}{-0.048611in}}{\pgfqpoint{0.000000in}{0.000000in}}{%
\pgfpathmoveto{\pgfqpoint{0.000000in}{0.000000in}}%
\pgfpathlineto{\pgfqpoint{0.000000in}{-0.048611in}}%
\pgfusepath{stroke,fill}%
}%
\begin{pgfscope}%
\pgfsys@transformshift{1.328581in}{0.499444in}%
\pgfsys@useobject{currentmarker}{}%
\end{pgfscope}%
\end{pgfscope}%
\begin{pgfscope}%
\definecolor{textcolor}{rgb}{0.000000,0.000000,0.000000}%
\pgfsetstrokecolor{textcolor}%
\pgfsetfillcolor{textcolor}%
\pgftext[x=1.328581in,y=0.402222in,,top]{\color{textcolor}\rmfamily\fontsize{10.000000}{12.000000}\selectfont \(\displaystyle {0.5}\)}%
\end{pgfscope}%
\begin{pgfscope}%
\pgfsetbuttcap%
\pgfsetroundjoin%
\definecolor{currentfill}{rgb}{0.000000,0.000000,0.000000}%
\pgfsetfillcolor{currentfill}%
\pgfsetlinewidth{0.803000pt}%
\definecolor{currentstroke}{rgb}{0.000000,0.000000,0.000000}%
\pgfsetstrokecolor{currentstroke}%
\pgfsetdash{}{0pt}%
\pgfsys@defobject{currentmarker}{\pgfqpoint{0.000000in}{-0.048611in}}{\pgfqpoint{0.000000in}{0.000000in}}{%
\pgfpathmoveto{\pgfqpoint{0.000000in}{0.000000in}}%
\pgfpathlineto{\pgfqpoint{0.000000in}{-0.048611in}}%
\pgfusepath{stroke,fill}%
}%
\begin{pgfscope}%
\pgfsys@transformshift{2.033126in}{0.499444in}%
\pgfsys@useobject{currentmarker}{}%
\end{pgfscope}%
\end{pgfscope}%
\begin{pgfscope}%
\definecolor{textcolor}{rgb}{0.000000,0.000000,0.000000}%
\pgfsetstrokecolor{textcolor}%
\pgfsetfillcolor{textcolor}%
\pgftext[x=2.033126in,y=0.402222in,,top]{\color{textcolor}\rmfamily\fontsize{10.000000}{12.000000}\selectfont \(\displaystyle {1.0}\)}%
\end{pgfscope}%
\begin{pgfscope}%
\definecolor{textcolor}{rgb}{0.000000,0.000000,0.000000}%
\pgfsetstrokecolor{textcolor}%
\pgfsetfillcolor{textcolor}%
\pgftext[x=1.328581in,y=0.223333in,,top]{\color{textcolor}\rmfamily\fontsize{10.000000}{12.000000}\selectfont False positive rate}%
\end{pgfscope}%
\begin{pgfscope}%
\pgfsetbuttcap%
\pgfsetroundjoin%
\definecolor{currentfill}{rgb}{0.000000,0.000000,0.000000}%
\pgfsetfillcolor{currentfill}%
\pgfsetlinewidth{0.803000pt}%
\definecolor{currentstroke}{rgb}{0.000000,0.000000,0.000000}%
\pgfsetstrokecolor{currentstroke}%
\pgfsetdash{}{0pt}%
\pgfsys@defobject{currentmarker}{\pgfqpoint{-0.048611in}{0.000000in}}{\pgfqpoint{-0.000000in}{0.000000in}}{%
\pgfpathmoveto{\pgfqpoint{-0.000000in}{0.000000in}}%
\pgfpathlineto{\pgfqpoint{-0.048611in}{0.000000in}}%
\pgfusepath{stroke,fill}%
}%
\begin{pgfscope}%
\pgfsys@transformshift{0.553581in}{0.551944in}%
\pgfsys@useobject{currentmarker}{}%
\end{pgfscope}%
\end{pgfscope}%
\begin{pgfscope}%
\definecolor{textcolor}{rgb}{0.000000,0.000000,0.000000}%
\pgfsetstrokecolor{textcolor}%
\pgfsetfillcolor{textcolor}%
\pgftext[x=0.278889in, y=0.503750in, left, base]{\color{textcolor}\rmfamily\fontsize{10.000000}{12.000000}\selectfont \(\displaystyle {0.0}\)}%
\end{pgfscope}%
\begin{pgfscope}%
\pgfsetbuttcap%
\pgfsetroundjoin%
\definecolor{currentfill}{rgb}{0.000000,0.000000,0.000000}%
\pgfsetfillcolor{currentfill}%
\pgfsetlinewidth{0.803000pt}%
\definecolor{currentstroke}{rgb}{0.000000,0.000000,0.000000}%
\pgfsetstrokecolor{currentstroke}%
\pgfsetdash{}{0pt}%
\pgfsys@defobject{currentmarker}{\pgfqpoint{-0.048611in}{0.000000in}}{\pgfqpoint{-0.000000in}{0.000000in}}{%
\pgfpathmoveto{\pgfqpoint{-0.000000in}{0.000000in}}%
\pgfpathlineto{\pgfqpoint{-0.048611in}{0.000000in}}%
\pgfusepath{stroke,fill}%
}%
\begin{pgfscope}%
\pgfsys@transformshift{0.553581in}{1.076944in}%
\pgfsys@useobject{currentmarker}{}%
\end{pgfscope}%
\end{pgfscope}%
\begin{pgfscope}%
\definecolor{textcolor}{rgb}{0.000000,0.000000,0.000000}%
\pgfsetstrokecolor{textcolor}%
\pgfsetfillcolor{textcolor}%
\pgftext[x=0.278889in, y=1.028750in, left, base]{\color{textcolor}\rmfamily\fontsize{10.000000}{12.000000}\selectfont \(\displaystyle {0.5}\)}%
\end{pgfscope}%
\begin{pgfscope}%
\pgfsetbuttcap%
\pgfsetroundjoin%
\definecolor{currentfill}{rgb}{0.000000,0.000000,0.000000}%
\pgfsetfillcolor{currentfill}%
\pgfsetlinewidth{0.803000pt}%
\definecolor{currentstroke}{rgb}{0.000000,0.000000,0.000000}%
\pgfsetstrokecolor{currentstroke}%
\pgfsetdash{}{0pt}%
\pgfsys@defobject{currentmarker}{\pgfqpoint{-0.048611in}{0.000000in}}{\pgfqpoint{-0.000000in}{0.000000in}}{%
\pgfpathmoveto{\pgfqpoint{-0.000000in}{0.000000in}}%
\pgfpathlineto{\pgfqpoint{-0.048611in}{0.000000in}}%
\pgfusepath{stroke,fill}%
}%
\begin{pgfscope}%
\pgfsys@transformshift{0.553581in}{1.601944in}%
\pgfsys@useobject{currentmarker}{}%
\end{pgfscope}%
\end{pgfscope}%
\begin{pgfscope}%
\definecolor{textcolor}{rgb}{0.000000,0.000000,0.000000}%
\pgfsetstrokecolor{textcolor}%
\pgfsetfillcolor{textcolor}%
\pgftext[x=0.278889in, y=1.553750in, left, base]{\color{textcolor}\rmfamily\fontsize{10.000000}{12.000000}\selectfont \(\displaystyle {1.0}\)}%
\end{pgfscope}%
\begin{pgfscope}%
\definecolor{textcolor}{rgb}{0.000000,0.000000,0.000000}%
\pgfsetstrokecolor{textcolor}%
\pgfsetfillcolor{textcolor}%
\pgftext[x=0.223333in,y=1.076944in,,bottom,rotate=90.000000]{\color{textcolor}\rmfamily\fontsize{10.000000}{12.000000}\selectfont True positive rate}%
\end{pgfscope}%
\begin{pgfscope}%
\pgfpathrectangle{\pgfqpoint{0.553581in}{0.499444in}}{\pgfqpoint{1.550000in}{1.155000in}}%
\pgfusepath{clip}%
\pgfsetbuttcap%
\pgfsetroundjoin%
\pgfsetlinewidth{1.505625pt}%
\definecolor{currentstroke}{rgb}{0.000000,0.000000,0.000000}%
\pgfsetstrokecolor{currentstroke}%
\pgfsetdash{{5.550000pt}{2.400000pt}}{0.000000pt}%
\pgfpathmoveto{\pgfqpoint{0.624035in}{0.551944in}}%
\pgfpathlineto{\pgfqpoint{2.033126in}{1.601944in}}%
\pgfusepath{stroke}%
\end{pgfscope}%
\begin{pgfscope}%
\pgfpathrectangle{\pgfqpoint{0.553581in}{0.499444in}}{\pgfqpoint{1.550000in}{1.155000in}}%
\pgfusepath{clip}%
\pgfsetrectcap%
\pgfsetroundjoin%
\pgfsetlinewidth{1.505625pt}%
\definecolor{currentstroke}{rgb}{0.000000,0.000000,0.000000}%
\pgfsetstrokecolor{currentstroke}%
\pgfsetdash{}{0pt}%
\pgfpathmoveto{\pgfqpoint{0.624035in}{0.551944in}}%
\pgfpathlineto{\pgfqpoint{0.628828in}{0.583321in}}%
\pgfpathlineto{\pgfqpoint{0.629052in}{0.584420in}}%
\pgfpathlineto{\pgfqpoint{0.630162in}{0.589701in}}%
\pgfpathlineto{\pgfqpoint{0.630406in}{0.590810in}}%
\pgfpathlineto{\pgfqpoint{0.631511in}{0.596422in}}%
\pgfpathlineto{\pgfqpoint{0.631758in}{0.597443in}}%
\pgfpathlineto{\pgfqpoint{0.632865in}{0.602851in}}%
\pgfpathlineto{\pgfqpoint{0.633140in}{0.603960in}}%
\pgfpathlineto{\pgfqpoint{0.634250in}{0.608959in}}%
\pgfpathlineto{\pgfqpoint{0.634568in}{0.610048in}}%
\pgfpathlineto{\pgfqpoint{0.635662in}{0.614659in}}%
\pgfpathlineto{\pgfqpoint{0.635676in}{0.614659in}}%
\pgfpathlineto{\pgfqpoint{0.635939in}{0.615767in}}%
\pgfpathlineto{\pgfqpoint{0.637049in}{0.619969in}}%
\pgfpathlineto{\pgfqpoint{0.637270in}{0.621058in}}%
\pgfpathlineto{\pgfqpoint{0.638380in}{0.625581in}}%
\pgfpathlineto{\pgfqpoint{0.638645in}{0.626690in}}%
\pgfpathlineto{\pgfqpoint{0.639755in}{0.631261in}}%
\pgfpathlineto{\pgfqpoint{0.640104in}{0.632370in}}%
\pgfpathlineto{\pgfqpoint{0.641213in}{0.636659in}}%
\pgfpathlineto{\pgfqpoint{0.641481in}{0.637758in}}%
\pgfpathlineto{\pgfqpoint{0.642589in}{0.642602in}}%
\pgfpathlineto{\pgfqpoint{0.642945in}{0.643711in}}%
\pgfpathlineto{\pgfqpoint{0.644054in}{0.647553in}}%
\pgfpathlineto{\pgfqpoint{0.644320in}{0.648661in}}%
\pgfpathlineto{\pgfqpoint{0.645423in}{0.653223in}}%
\pgfpathlineto{\pgfqpoint{0.645427in}{0.653223in}}%
\pgfpathlineto{\pgfqpoint{0.645744in}{0.654332in}}%
\pgfpathlineto{\pgfqpoint{0.646849in}{0.658281in}}%
\pgfpathlineto{\pgfqpoint{0.647172in}{0.659380in}}%
\pgfpathlineto{\pgfqpoint{0.648280in}{0.663523in}}%
\pgfpathlineto{\pgfqpoint{0.648575in}{0.664622in}}%
\pgfpathlineto{\pgfqpoint{0.649685in}{0.668269in}}%
\pgfpathlineto{\pgfqpoint{0.649985in}{0.669359in}}%
\pgfpathlineto{\pgfqpoint{0.651093in}{0.673084in}}%
\pgfpathlineto{\pgfqpoint{0.651451in}{0.674193in}}%
\pgfpathlineto{\pgfqpoint{0.652559in}{0.678248in}}%
\pgfpathlineto{\pgfqpoint{0.652959in}{0.679357in}}%
\pgfpathlineto{\pgfqpoint{0.654064in}{0.682654in}}%
\pgfpathlineto{\pgfqpoint{0.654460in}{0.683763in}}%
\pgfpathlineto{\pgfqpoint{0.655569in}{0.687595in}}%
\pgfpathlineto{\pgfqpoint{0.655907in}{0.688704in}}%
\pgfpathlineto{\pgfqpoint{0.657014in}{0.692089in}}%
\pgfpathlineto{\pgfqpoint{0.657359in}{0.693198in}}%
\pgfpathlineto{\pgfqpoint{0.658464in}{0.696660in}}%
\pgfpathlineto{\pgfqpoint{0.658822in}{0.697769in}}%
\pgfpathlineto{\pgfqpoint{0.659932in}{0.701484in}}%
\pgfpathlineto{\pgfqpoint{0.660362in}{0.702593in}}%
\pgfpathlineto{\pgfqpoint{0.661468in}{0.706581in}}%
\pgfpathlineto{\pgfqpoint{0.661840in}{0.707670in}}%
\pgfpathlineto{\pgfqpoint{0.662947in}{0.711687in}}%
\pgfpathlineto{\pgfqpoint{0.663415in}{0.712786in}}%
\pgfpathlineto{\pgfqpoint{0.664525in}{0.716453in}}%
\pgfpathlineto{\pgfqpoint{0.664886in}{0.717552in}}%
\pgfpathlineto{\pgfqpoint{0.665989in}{0.720762in}}%
\pgfpathlineto{\pgfqpoint{0.666375in}{0.721870in}}%
\pgfpathlineto{\pgfqpoint{0.667485in}{0.725090in}}%
\pgfpathlineto{\pgfqpoint{0.667866in}{0.726198in}}%
\pgfpathlineto{\pgfqpoint{0.668969in}{0.729428in}}%
\pgfpathlineto{\pgfqpoint{0.669355in}{0.730527in}}%
\pgfpathlineto{\pgfqpoint{0.670465in}{0.733532in}}%
\pgfpathlineto{\pgfqpoint{0.670844in}{0.734641in}}%
\pgfpathlineto{\pgfqpoint{0.671952in}{0.737948in}}%
\pgfpathlineto{\pgfqpoint{0.672327in}{0.739037in}}%
\pgfpathlineto{\pgfqpoint{0.673434in}{0.742169in}}%
\pgfpathlineto{\pgfqpoint{0.673774in}{0.743278in}}%
\pgfpathlineto{\pgfqpoint{0.674881in}{0.746662in}}%
\pgfpathlineto{\pgfqpoint{0.675363in}{0.747752in}}%
\pgfpathlineto{\pgfqpoint{0.676473in}{0.751234in}}%
\pgfpathlineto{\pgfqpoint{0.676833in}{0.752333in}}%
\pgfpathlineto{\pgfqpoint{0.677943in}{0.755163in}}%
\pgfpathlineto{\pgfqpoint{0.678318in}{0.756272in}}%
\pgfpathlineto{\pgfqpoint{0.679418in}{0.759238in}}%
\pgfpathlineto{\pgfqpoint{0.679872in}{0.760347in}}%
\pgfpathlineto{\pgfqpoint{0.680968in}{0.763265in}}%
\pgfpathlineto{\pgfqpoint{0.681459in}{0.764374in}}%
\pgfpathlineto{\pgfqpoint{0.682564in}{0.766834in}}%
\pgfpathlineto{\pgfqpoint{0.683139in}{0.767943in}}%
\pgfpathlineto{\pgfqpoint{0.684242in}{0.770735in}}%
\pgfpathlineto{\pgfqpoint{0.684667in}{0.771834in}}%
\pgfpathlineto{\pgfqpoint{0.685770in}{0.774878in}}%
\pgfpathlineto{\pgfqpoint{0.686268in}{0.775977in}}%
\pgfpathlineto{\pgfqpoint{0.687376in}{0.778681in}}%
\pgfpathlineto{\pgfqpoint{0.687857in}{0.779780in}}%
\pgfpathlineto{\pgfqpoint{0.688967in}{0.782620in}}%
\pgfpathlineto{\pgfqpoint{0.689374in}{0.783709in}}%
\pgfpathlineto{\pgfqpoint{0.690484in}{0.786404in}}%
\pgfpathlineto{\pgfqpoint{0.690931in}{0.787512in}}%
\pgfpathlineto{\pgfqpoint{0.692029in}{0.790382in}}%
\pgfpathlineto{\pgfqpoint{0.692476in}{0.791471in}}%
\pgfpathlineto{\pgfqpoint{0.693572in}{0.794039in}}%
\pgfpathlineto{\pgfqpoint{0.693584in}{0.794039in}}%
\pgfpathlineto{\pgfqpoint{0.694019in}{0.795138in}}%
\pgfpathlineto{\pgfqpoint{0.695128in}{0.797637in}}%
\pgfpathlineto{\pgfqpoint{0.695594in}{0.798736in}}%
\pgfpathlineto{\pgfqpoint{0.696701in}{0.801265in}}%
\pgfpathlineto{\pgfqpoint{0.697125in}{0.802374in}}%
\pgfpathlineto{\pgfqpoint{0.698235in}{0.805117in}}%
\pgfpathlineto{\pgfqpoint{0.698726in}{0.806226in}}%
\pgfpathlineto{\pgfqpoint{0.699833in}{0.808929in}}%
\pgfpathlineto{\pgfqpoint{0.700315in}{0.810009in}}%
\pgfpathlineto{\pgfqpoint{0.701425in}{0.812616in}}%
\pgfpathlineto{\pgfqpoint{0.701846in}{0.813705in}}%
\pgfpathlineto{\pgfqpoint{0.702956in}{0.816273in}}%
\pgfpathlineto{\pgfqpoint{0.703465in}{0.817381in}}%
\pgfpathlineto{\pgfqpoint{0.704573in}{0.819891in}}%
\pgfpathlineto{\pgfqpoint{0.705078in}{0.820990in}}%
\pgfpathlineto{\pgfqpoint{0.706173in}{0.823334in}}%
\pgfpathlineto{\pgfqpoint{0.706185in}{0.823334in}}%
\pgfpathlineto{\pgfqpoint{0.706655in}{0.824433in}}%
\pgfpathlineto{\pgfqpoint{0.707760in}{0.827283in}}%
\pgfpathlineto{\pgfqpoint{0.708309in}{0.828392in}}%
\pgfpathlineto{\pgfqpoint{0.709417in}{0.830424in}}%
\pgfpathlineto{\pgfqpoint{0.709894in}{0.831523in}}%
\pgfpathlineto{\pgfqpoint{0.711004in}{0.834169in}}%
\pgfpathlineto{\pgfqpoint{0.711502in}{0.835278in}}%
\pgfpathlineto{\pgfqpoint{0.712605in}{0.837379in}}%
\pgfpathlineto{\pgfqpoint{0.713114in}{0.838487in}}%
\pgfpathlineto{\pgfqpoint{0.714212in}{0.841074in}}%
\pgfpathlineto{\pgfqpoint{0.714850in}{0.842174in}}%
\pgfpathlineto{\pgfqpoint{0.715955in}{0.844333in}}%
\pgfpathlineto{\pgfqpoint{0.716434in}{0.845422in}}%
\pgfpathlineto{\pgfqpoint{0.717544in}{0.847815in}}%
\pgfpathlineto{\pgfqpoint{0.718268in}{0.848914in}}%
\pgfpathlineto{\pgfqpoint{0.719366in}{0.851248in}}%
\pgfpathlineto{\pgfqpoint{0.719964in}{0.852357in}}%
\pgfpathlineto{\pgfqpoint{0.721067in}{0.854925in}}%
\pgfpathlineto{\pgfqpoint{0.721653in}{0.856033in}}%
\pgfpathlineto{\pgfqpoint{0.722758in}{0.858426in}}%
\pgfpathlineto{\pgfqpoint{0.723310in}{0.859535in}}%
\pgfpathlineto{\pgfqpoint{0.724415in}{0.861927in}}%
\pgfpathlineto{\pgfqpoint{0.725032in}{0.863026in}}%
\pgfpathlineto{\pgfqpoint{0.726141in}{0.865195in}}%
\pgfpathlineto{\pgfqpoint{0.726635in}{0.866294in}}%
\pgfpathlineto{\pgfqpoint{0.727742in}{0.868590in}}%
\pgfpathlineto{\pgfqpoint{0.728291in}{0.869669in}}%
\pgfpathlineto{\pgfqpoint{0.729401in}{0.871984in}}%
\pgfpathlineto{\pgfqpoint{0.729997in}{0.873093in}}%
\pgfpathlineto{\pgfqpoint{0.731104in}{0.875427in}}%
\pgfpathlineto{\pgfqpoint{0.731621in}{0.876536in}}%
\pgfpathlineto{\pgfqpoint{0.732731in}{0.878598in}}%
\pgfpathlineto{\pgfqpoint{0.733266in}{0.879707in}}%
\pgfpathlineto{\pgfqpoint{0.734369in}{0.881730in}}%
\pgfpathlineto{\pgfqpoint{0.734983in}{0.882839in}}%
\pgfpathlineto{\pgfqpoint{0.736093in}{0.884920in}}%
\pgfpathlineto{\pgfqpoint{0.736607in}{0.886029in}}%
\pgfpathlineto{\pgfqpoint{0.737710in}{0.888188in}}%
\pgfpathlineto{\pgfqpoint{0.738368in}{0.889287in}}%
\pgfpathlineto{\pgfqpoint{0.739478in}{0.891252in}}%
\pgfpathlineto{\pgfqpoint{0.740102in}{0.892361in}}%
\pgfpathlineto{\pgfqpoint{0.741212in}{0.894423in}}%
\pgfpathlineto{\pgfqpoint{0.741996in}{0.895522in}}%
\pgfpathlineto{\pgfqpoint{0.743099in}{0.897681in}}%
\pgfpathlineto{\pgfqpoint{0.743657in}{0.898790in}}%
\pgfpathlineto{\pgfqpoint{0.744767in}{0.900560in}}%
\pgfpathlineto{\pgfqpoint{0.745363in}{0.901669in}}%
\pgfpathlineto{\pgfqpoint{0.746470in}{0.903507in}}%
\pgfpathlineto{\pgfqpoint{0.747108in}{0.904616in}}%
\pgfpathlineto{\pgfqpoint{0.748208in}{0.906464in}}%
\pgfpathlineto{\pgfqpoint{0.748936in}{0.907563in}}%
\pgfpathlineto{\pgfqpoint{0.750046in}{0.909644in}}%
\pgfpathlineto{\pgfqpoint{0.750840in}{0.910753in}}%
\pgfpathlineto{\pgfqpoint{0.751950in}{0.912620in}}%
\pgfpathlineto{\pgfqpoint{0.752657in}{0.913729in}}%
\pgfpathlineto{\pgfqpoint{0.753767in}{0.915538in}}%
\pgfpathlineto{\pgfqpoint{0.754353in}{0.916647in}}%
\pgfpathlineto{\pgfqpoint{0.755461in}{0.918816in}}%
\pgfpathlineto{\pgfqpoint{0.756138in}{0.919915in}}%
\pgfpathlineto{\pgfqpoint{0.757241in}{0.921792in}}%
\pgfpathlineto{\pgfqpoint{0.757724in}{0.922901in}}%
\pgfpathlineto{\pgfqpoint{0.758830in}{0.924729in}}%
\pgfpathlineto{\pgfqpoint{0.759397in}{0.925828in}}%
\pgfpathlineto{\pgfqpoint{0.760505in}{0.927540in}}%
\pgfpathlineto{\pgfqpoint{0.761070in}{0.928649in}}%
\pgfpathlineto{\pgfqpoint{0.762173in}{0.930662in}}%
\pgfpathlineto{\pgfqpoint{0.762899in}{0.931761in}}%
\pgfpathlineto{\pgfqpoint{0.764009in}{0.933775in}}%
\pgfpathlineto{\pgfqpoint{0.764735in}{0.934884in}}%
\pgfpathlineto{\pgfqpoint{0.765842in}{0.936712in}}%
\pgfpathlineto{\pgfqpoint{0.766603in}{0.937821in}}%
\pgfpathlineto{\pgfqpoint{0.767713in}{0.940058in}}%
\pgfpathlineto{\pgfqpoint{0.768369in}{0.941167in}}%
\pgfpathlineto{\pgfqpoint{0.769456in}{0.942820in}}%
\pgfpathlineto{\pgfqpoint{0.770207in}{0.943919in}}%
\pgfpathlineto{\pgfqpoint{0.771310in}{0.945602in}}%
\pgfpathlineto{\pgfqpoint{0.772087in}{0.946711in}}%
\pgfpathlineto{\pgfqpoint{0.773197in}{0.948617in}}%
\pgfpathlineto{\pgfqpoint{0.773777in}{0.949726in}}%
\pgfpathlineto{\pgfqpoint{0.774884in}{0.951564in}}%
\pgfpathlineto{\pgfqpoint{0.775664in}{0.952673in}}%
\pgfpathlineto{\pgfqpoint{0.776755in}{0.954414in}}%
\pgfpathlineto{\pgfqpoint{0.777539in}{0.955522in}}%
\pgfpathlineto{\pgfqpoint{0.778628in}{0.957419in}}%
\pgfpathlineto{\pgfqpoint{0.778632in}{0.957419in}}%
\pgfpathlineto{\pgfqpoint{0.779337in}{0.958528in}}%
\pgfpathlineto{\pgfqpoint{0.780445in}{0.960259in}}%
\pgfpathlineto{\pgfqpoint{0.781134in}{0.961368in}}%
\pgfpathlineto{\pgfqpoint{0.782244in}{0.962982in}}%
\pgfpathlineto{\pgfqpoint{0.782963in}{0.964091in}}%
\pgfpathlineto{\pgfqpoint{0.784072in}{0.965842in}}%
\pgfpathlineto{\pgfqpoint{0.784733in}{0.966951in}}%
\pgfpathlineto{\pgfqpoint{0.785841in}{0.968770in}}%
\pgfpathlineto{\pgfqpoint{0.786513in}{0.969849in}}%
\pgfpathlineto{\pgfqpoint{0.787623in}{0.971726in}}%
\pgfpathlineto{\pgfqpoint{0.788368in}{0.972835in}}%
\pgfpathlineto{\pgfqpoint{0.789473in}{0.974654in}}%
\pgfpathlineto{\pgfqpoint{0.790196in}{0.975763in}}%
\pgfpathlineto{\pgfqpoint{0.791304in}{0.977552in}}%
\pgfpathlineto{\pgfqpoint{0.792169in}{0.978661in}}%
\pgfpathlineto{\pgfqpoint{0.793277in}{0.980334in}}%
\pgfpathlineto{\pgfqpoint{0.793991in}{0.981433in}}%
\pgfpathlineto{\pgfqpoint{0.795099in}{0.982921in}}%
\pgfpathlineto{\pgfqpoint{0.795739in}{0.984020in}}%
\pgfpathlineto{\pgfqpoint{0.796839in}{0.985333in}}%
\pgfpathlineto{\pgfqpoint{0.797663in}{0.986432in}}%
\pgfpathlineto{\pgfqpoint{0.798756in}{0.987950in}}%
\pgfpathlineto{\pgfqpoint{0.799552in}{0.989058in}}%
\pgfpathlineto{\pgfqpoint{0.800657in}{0.990634in}}%
\pgfpathlineto{\pgfqpoint{0.801290in}{0.991743in}}%
\pgfpathlineto{\pgfqpoint{0.802379in}{0.993299in}}%
\pgfpathlineto{\pgfqpoint{0.803126in}{0.994408in}}%
\pgfpathlineto{\pgfqpoint{0.804236in}{0.996314in}}%
\pgfpathlineto{\pgfqpoint{0.804911in}{0.997423in}}%
\pgfpathlineto{\pgfqpoint{0.806018in}{0.999037in}}%
\pgfpathlineto{\pgfqpoint{0.806818in}{1.000146in}}%
\pgfpathlineto{\pgfqpoint{0.807921in}{1.001566in}}%
\pgfpathlineto{\pgfqpoint{0.808519in}{1.002675in}}%
\pgfpathlineto{\pgfqpoint{0.809622in}{1.004562in}}%
\pgfpathlineto{\pgfqpoint{0.810390in}{1.005671in}}%
\pgfpathlineto{\pgfqpoint{0.811432in}{1.006984in}}%
\pgfpathlineto{\pgfqpoint{0.812354in}{1.008093in}}%
\pgfpathlineto{\pgfqpoint{0.813461in}{1.009736in}}%
\pgfpathlineto{\pgfqpoint{0.814308in}{1.010826in}}%
\pgfpathlineto{\pgfqpoint{0.815418in}{1.012508in}}%
\pgfpathlineto{\pgfqpoint{0.816223in}{1.013617in}}%
\pgfpathlineto{\pgfqpoint{0.817333in}{1.014998in}}%
\pgfpathlineto{\pgfqpoint{0.818080in}{1.016107in}}%
\pgfpathlineto{\pgfqpoint{0.819190in}{1.017663in}}%
\pgfpathlineto{\pgfqpoint{0.820239in}{1.018772in}}%
\pgfpathlineto{\pgfqpoint{0.821349in}{1.020484in}}%
\pgfpathlineto{\pgfqpoint{0.822221in}{1.021573in}}%
\pgfpathlineto{\pgfqpoint{0.823310in}{1.022915in}}%
\pgfpathlineto{\pgfqpoint{0.824220in}{1.024024in}}%
\pgfpathlineto{\pgfqpoint{0.825328in}{1.025522in}}%
\pgfpathlineto{\pgfqpoint{0.826242in}{1.026631in}}%
\pgfpathlineto{\pgfqpoint{0.827345in}{1.028245in}}%
\pgfpathlineto{\pgfqpoint{0.828059in}{1.029354in}}%
\pgfpathlineto{\pgfqpoint{0.829162in}{1.030774in}}%
\pgfpathlineto{\pgfqpoint{0.830177in}{1.031873in}}%
\pgfpathlineto{\pgfqpoint{0.831286in}{1.033274in}}%
\pgfpathlineto{\pgfqpoint{0.832189in}{1.034382in}}%
\pgfpathlineto{\pgfqpoint{0.833294in}{1.035715in}}%
\pgfpathlineto{\pgfqpoint{0.834199in}{1.036824in}}%
\pgfpathlineto{\pgfqpoint{0.835300in}{1.038341in}}%
\pgfpathlineto{\pgfqpoint{0.836159in}{1.039450in}}%
\pgfpathlineto{\pgfqpoint{0.837266in}{1.041171in}}%
\pgfpathlineto{\pgfqpoint{0.838094in}{1.042280in}}%
\pgfpathlineto{\pgfqpoint{0.839204in}{1.043817in}}%
\pgfpathlineto{\pgfqpoint{0.840170in}{1.044926in}}%
\pgfpathlineto{\pgfqpoint{0.841266in}{1.046433in}}%
\pgfpathlineto{\pgfqpoint{0.842029in}{1.047542in}}%
\pgfpathlineto{\pgfqpoint{0.843139in}{1.048933in}}%
\pgfpathlineto{\pgfqpoint{0.844149in}{1.050042in}}%
\pgfpathlineto{\pgfqpoint{0.845242in}{1.051539in}}%
\pgfpathlineto{\pgfqpoint{0.846138in}{1.052648in}}%
\pgfpathlineto{\pgfqpoint{0.847231in}{1.054000in}}%
\pgfpathlineto{\pgfqpoint{0.847964in}{1.055070in}}%
\pgfpathlineto{\pgfqpoint{0.849051in}{1.056801in}}%
\pgfpathlineto{\pgfqpoint{0.849740in}{1.057910in}}%
\pgfpathlineto{\pgfqpoint{0.850850in}{1.059291in}}%
\pgfpathlineto{\pgfqpoint{0.851755in}{1.060400in}}%
\pgfpathlineto{\pgfqpoint{0.852853in}{1.061723in}}%
\pgfpathlineto{\pgfqpoint{0.853902in}{1.062831in}}%
\pgfpathlineto{\pgfqpoint{0.855010in}{1.064164in}}%
\pgfpathlineto{\pgfqpoint{0.855789in}{1.065273in}}%
\pgfpathlineto{\pgfqpoint{0.856894in}{1.066741in}}%
\pgfpathlineto{\pgfqpoint{0.857665in}{1.067850in}}%
\pgfpathlineto{\pgfqpoint{0.858772in}{1.069299in}}%
\pgfpathlineto{\pgfqpoint{0.859703in}{1.070408in}}%
\pgfpathlineto{\pgfqpoint{0.860813in}{1.071624in}}%
\pgfpathlineto{\pgfqpoint{0.861692in}{1.072733in}}%
\pgfpathlineto{\pgfqpoint{0.862781in}{1.074211in}}%
\pgfpathlineto{\pgfqpoint{0.863828in}{1.075320in}}%
\pgfpathlineto{\pgfqpoint{0.864936in}{1.076575in}}%
\pgfpathlineto{\pgfqpoint{0.865787in}{1.077683in}}%
\pgfpathlineto{\pgfqpoint{0.866871in}{1.078850in}}%
\pgfpathlineto{\pgfqpoint{0.867772in}{1.079959in}}%
\pgfpathlineto{\pgfqpoint{0.868882in}{1.081389in}}%
\pgfpathlineto{\pgfqpoint{0.869556in}{1.082498in}}%
\pgfpathlineto{\pgfqpoint{0.870659in}{1.083782in}}%
\pgfpathlineto{\pgfqpoint{0.871681in}{1.084890in}}%
\pgfpathlineto{\pgfqpoint{0.872791in}{1.086194in}}%
\pgfpathlineto{\pgfqpoint{0.873900in}{1.087293in}}%
\pgfpathlineto{\pgfqpoint{0.875010in}{1.088596in}}%
\pgfpathlineto{\pgfqpoint{0.875848in}{1.089705in}}%
\pgfpathlineto{\pgfqpoint{0.876958in}{1.091028in}}%
\pgfpathlineto{\pgfqpoint{0.877961in}{1.092127in}}%
\pgfpathlineto{\pgfqpoint{0.877961in}{1.092136in}}%
\pgfpathlineto{\pgfqpoint{0.879054in}{1.093323in}}%
\pgfpathlineto{\pgfqpoint{0.879941in}{1.094432in}}%
\pgfpathlineto{\pgfqpoint{0.881044in}{1.095706in}}%
\pgfpathlineto{\pgfqpoint{0.882188in}{1.096815in}}%
\pgfpathlineto{\pgfqpoint{0.883298in}{1.097953in}}%
\pgfpathlineto{\pgfqpoint{0.884333in}{1.099062in}}%
\pgfpathlineto{\pgfqpoint{0.885422in}{1.100511in}}%
\pgfpathlineto{\pgfqpoint{0.886444in}{1.101619in}}%
\pgfpathlineto{\pgfqpoint{0.887537in}{1.102942in}}%
\pgfpathlineto{\pgfqpoint{0.888373in}{1.104051in}}%
\pgfpathlineto{\pgfqpoint{0.889473in}{1.105296in}}%
\pgfpathlineto{\pgfqpoint{0.890332in}{1.106405in}}%
\pgfpathlineto{\pgfqpoint{0.891430in}{1.107494in}}%
\pgfpathlineto{\pgfqpoint{0.892323in}{1.108603in}}%
\pgfpathlineto{\pgfqpoint{0.893375in}{1.109712in}}%
\pgfpathlineto{\pgfqpoint{0.894331in}{1.110820in}}%
\pgfpathlineto{\pgfqpoint{0.895439in}{1.111978in}}%
\pgfpathlineto{\pgfqpoint{0.896449in}{1.113087in}}%
\pgfpathlineto{\pgfqpoint{0.897552in}{1.114176in}}%
\pgfpathlineto{\pgfqpoint{0.898678in}{1.115285in}}%
\pgfpathlineto{\pgfqpoint{0.899788in}{1.116510in}}%
\pgfpathlineto{\pgfqpoint{0.900800in}{1.117619in}}%
\pgfpathlineto{\pgfqpoint{0.901910in}{1.118883in}}%
\pgfpathlineto{\pgfqpoint{0.902901in}{1.119992in}}%
\pgfpathlineto{\pgfqpoint{0.904008in}{1.121120in}}%
\pgfpathlineto{\pgfqpoint{0.905048in}{1.122229in}}%
\pgfpathlineto{\pgfqpoint{0.906144in}{1.123396in}}%
\pgfpathlineto{\pgfqpoint{0.906154in}{1.123396in}}%
\pgfpathlineto{\pgfqpoint{0.907005in}{1.124505in}}%
\pgfpathlineto{\pgfqpoint{0.908094in}{1.125760in}}%
\pgfpathlineto{\pgfqpoint{0.909043in}{1.126869in}}%
\pgfpathlineto{\pgfqpoint{0.910146in}{1.128337in}}%
\pgfpathlineto{\pgfqpoint{0.911110in}{1.129446in}}%
\pgfpathlineto{\pgfqpoint{0.912215in}{1.130458in}}%
\pgfpathlineto{\pgfqpoint{0.913373in}{1.131537in}}%
\pgfpathlineto{\pgfqpoint{0.914476in}{1.132665in}}%
\pgfpathlineto{\pgfqpoint{0.915419in}{1.133755in}}%
\pgfpathlineto{\pgfqpoint{0.916521in}{1.135039in}}%
\pgfpathlineto{\pgfqpoint{0.917478in}{1.136147in}}%
\pgfpathlineto{\pgfqpoint{0.918588in}{1.137159in}}%
\pgfpathlineto{\pgfqpoint{0.919621in}{1.138268in}}%
\pgfpathlineto{\pgfqpoint{0.921059in}{1.139542in}}%
\pgfpathlineto{\pgfqpoint{0.922138in}{1.140651in}}%
\pgfpathlineto{\pgfqpoint{0.923232in}{1.141935in}}%
\pgfpathlineto{\pgfqpoint{0.924316in}{1.143034in}}%
\pgfpathlineto{\pgfqpoint{0.925421in}{1.144123in}}%
\pgfpathlineto{\pgfqpoint{0.926401in}{1.145232in}}%
\pgfpathlineto{\pgfqpoint{0.927511in}{1.146506in}}%
\pgfpathlineto{\pgfqpoint{0.928502in}{1.147615in}}%
\pgfpathlineto{\pgfqpoint{0.929609in}{1.148782in}}%
\pgfpathlineto{\pgfqpoint{0.930870in}{1.149891in}}%
\pgfpathlineto{\pgfqpoint{0.931976in}{1.151116in}}%
\pgfpathlineto{\pgfqpoint{0.932995in}{1.152225in}}%
\pgfpathlineto{\pgfqpoint{0.934105in}{1.153577in}}%
\pgfpathlineto{\pgfqpoint{0.935147in}{1.154686in}}%
\pgfpathlineto{\pgfqpoint{0.936252in}{1.155872in}}%
\pgfpathlineto{\pgfqpoint{0.937385in}{1.156981in}}%
\pgfpathlineto{\pgfqpoint{0.938488in}{1.157934in}}%
\pgfpathlineto{\pgfqpoint{0.939628in}{1.159043in}}%
\pgfpathlineto{\pgfqpoint{0.940733in}{1.160142in}}%
\pgfpathlineto{\pgfqpoint{0.941829in}{1.161251in}}%
\pgfpathlineto{\pgfqpoint{0.942925in}{1.162428in}}%
\pgfpathlineto{\pgfqpoint{0.944312in}{1.163536in}}%
\pgfpathlineto{\pgfqpoint{0.945415in}{1.164567in}}%
\pgfpathlineto{\pgfqpoint{0.946476in}{1.165676in}}%
\pgfpathlineto{\pgfqpoint{0.947581in}{1.166610in}}%
\pgfpathlineto{\pgfqpoint{0.948693in}{1.167719in}}%
\pgfpathlineto{\pgfqpoint{0.949801in}{1.168740in}}%
\pgfpathlineto{\pgfqpoint{0.951027in}{1.169839in}}%
\pgfpathlineto{\pgfqpoint{0.952137in}{1.170957in}}%
\pgfpathlineto{\pgfqpoint{0.953149in}{1.172057in}}%
\pgfpathlineto{\pgfqpoint{0.954249in}{1.173224in}}%
\pgfpathlineto{\pgfqpoint{0.955357in}{1.174332in}}%
\pgfpathlineto{\pgfqpoint{0.956462in}{1.175500in}}%
\pgfpathlineto{\pgfqpoint{0.957688in}{1.176608in}}%
\pgfpathlineto{\pgfqpoint{0.958777in}{1.177717in}}%
\pgfpathlineto{\pgfqpoint{0.958798in}{1.177717in}}%
\pgfpathlineto{\pgfqpoint{0.959952in}{1.178826in}}%
\pgfpathlineto{\pgfqpoint{0.961053in}{1.179828in}}%
\pgfpathlineto{\pgfqpoint{0.961946in}{1.180937in}}%
\pgfpathlineto{\pgfqpoint{0.963040in}{1.181890in}}%
\pgfpathlineto{\pgfqpoint{0.964271in}{1.182998in}}%
\pgfpathlineto{\pgfqpoint{0.965378in}{1.183981in}}%
\pgfpathlineto{\pgfqpoint{0.966595in}{1.185070in}}%
\pgfpathlineto{\pgfqpoint{0.966595in}{1.185080in}}%
\pgfpathlineto{\pgfqpoint{0.967703in}{1.186403in}}%
\pgfpathlineto{\pgfqpoint{0.968912in}{1.187502in}}%
\pgfpathlineto{\pgfqpoint{0.969987in}{1.188513in}}%
\pgfpathlineto{\pgfqpoint{0.971363in}{1.189622in}}%
\pgfpathlineto{\pgfqpoint{0.972456in}{1.190575in}}%
\pgfpathlineto{\pgfqpoint{0.972472in}{1.190575in}}%
\pgfpathlineto{\pgfqpoint{0.973445in}{1.191684in}}%
\pgfpathlineto{\pgfqpoint{0.974552in}{1.192832in}}%
\pgfpathlineto{\pgfqpoint{0.975904in}{1.193940in}}%
\pgfpathlineto{\pgfqpoint{0.977012in}{1.194913in}}%
\pgfpathlineto{\pgfqpoint{0.978173in}{1.196022in}}%
\pgfpathlineto{\pgfqpoint{0.979271in}{1.197024in}}%
\pgfpathlineto{\pgfqpoint{0.980379in}{1.198094in}}%
\pgfpathlineto{\pgfqpoint{0.981472in}{1.199202in}}%
\pgfpathlineto{\pgfqpoint{0.982710in}{1.200311in}}%
\pgfpathlineto{\pgfqpoint{0.983820in}{1.201371in}}%
\pgfpathlineto{\pgfqpoint{0.985041in}{1.202480in}}%
\pgfpathlineto{\pgfqpoint{0.986151in}{1.203540in}}%
\pgfpathlineto{\pgfqpoint{0.987361in}{1.204649in}}%
\pgfpathlineto{\pgfqpoint{0.988422in}{1.205447in}}%
\pgfpathlineto{\pgfqpoint{0.989762in}{1.206546in}}%
\pgfpathlineto{\pgfqpoint{0.990849in}{1.207518in}}%
\pgfpathlineto{\pgfqpoint{0.992080in}{1.208627in}}%
\pgfpathlineto{\pgfqpoint{0.993176in}{1.209512in}}%
\pgfpathlineto{\pgfqpoint{0.994283in}{1.210621in}}%
\pgfpathlineto{\pgfqpoint{0.995379in}{1.211778in}}%
\pgfpathlineto{\pgfqpoint{0.995391in}{1.211778in}}%
\pgfpathlineto{\pgfqpoint{0.996554in}{1.212887in}}%
\pgfpathlineto{\pgfqpoint{0.997631in}{1.213830in}}%
\pgfpathlineto{\pgfqpoint{0.998792in}{1.214939in}}%
\pgfpathlineto{\pgfqpoint{0.999898in}{1.216252in}}%
\pgfpathlineto{\pgfqpoint{0.999902in}{1.216252in}}%
\pgfpathlineto{\pgfqpoint{1.001321in}{1.217439in}}%
\pgfpathlineto{\pgfqpoint{1.002659in}{1.218528in}}%
\pgfpathlineto{\pgfqpoint{1.003767in}{1.219481in}}%
\pgfpathlineto{\pgfqpoint{1.004942in}{1.220590in}}%
\pgfpathlineto{\pgfqpoint{1.006052in}{1.221592in}}%
\pgfpathlineto{\pgfqpoint{1.007273in}{1.222701in}}%
\pgfpathlineto{\pgfqpoint{1.008383in}{1.223518in}}%
\pgfpathlineto{\pgfqpoint{1.009844in}{1.224627in}}%
\pgfpathlineto{\pgfqpoint{1.010945in}{1.225628in}}%
\pgfpathlineto{\pgfqpoint{1.012339in}{1.226727in}}%
\pgfpathlineto{\pgfqpoint{1.013441in}{1.227817in}}%
\pgfpathlineto{\pgfqpoint{1.014675in}{1.228896in}}%
\pgfpathlineto{\pgfqpoint{1.015754in}{1.230015in}}%
\pgfpathlineto{\pgfqpoint{1.017125in}{1.231124in}}%
\pgfpathlineto{\pgfqpoint{1.018225in}{1.232155in}}%
\pgfpathlineto{\pgfqpoint{1.019412in}{1.233244in}}%
\pgfpathlineto{\pgfqpoint{1.020517in}{1.234246in}}%
\pgfpathlineto{\pgfqpoint{1.022008in}{1.235355in}}%
\pgfpathlineto{\pgfqpoint{1.023051in}{1.236269in}}%
\pgfpathlineto{\pgfqpoint{1.023083in}{1.236269in}}%
\pgfpathlineto{\pgfqpoint{1.024561in}{1.237378in}}%
\pgfpathlineto{\pgfqpoint{1.025664in}{1.238263in}}%
\pgfpathlineto{\pgfqpoint{1.026925in}{1.239371in}}%
\pgfpathlineto{\pgfqpoint{1.028023in}{1.240432in}}%
\pgfpathlineto{\pgfqpoint{1.029403in}{1.241540in}}%
\pgfpathlineto{\pgfqpoint{1.030473in}{1.242474in}}%
\pgfpathlineto{\pgfqpoint{1.030499in}{1.242474in}}%
\pgfpathlineto{\pgfqpoint{1.031871in}{1.243583in}}%
\pgfpathlineto{\pgfqpoint{1.032970in}{1.244653in}}%
\pgfpathlineto{\pgfqpoint{1.034319in}{1.245762in}}%
\pgfpathlineto{\pgfqpoint{1.035408in}{1.246530in}}%
\pgfpathlineto{\pgfqpoint{1.036813in}{1.247639in}}%
\pgfpathlineto{\pgfqpoint{1.037914in}{1.248689in}}%
\pgfpathlineto{\pgfqpoint{1.039547in}{1.249788in}}%
\pgfpathlineto{\pgfqpoint{1.040657in}{1.250761in}}%
\pgfpathlineto{\pgfqpoint{1.042030in}{1.251870in}}%
\pgfpathlineto{\pgfqpoint{1.043121in}{1.252784in}}%
\pgfpathlineto{\pgfqpoint{1.044299in}{1.253893in}}%
\pgfpathlineto{\pgfqpoint{1.045373in}{1.254856in}}%
\pgfpathlineto{\pgfqpoint{1.046909in}{1.255964in}}%
\pgfpathlineto{\pgfqpoint{1.048010in}{1.256820in}}%
\pgfpathlineto{\pgfqpoint{1.049517in}{1.257929in}}%
\pgfpathlineto{\pgfqpoint{1.050625in}{1.258814in}}%
\pgfpathlineto{\pgfqpoint{1.052158in}{1.259923in}}%
\pgfpathlineto{\pgfqpoint{1.053231in}{1.260779in}}%
\pgfpathlineto{\pgfqpoint{1.054394in}{1.261888in}}%
\pgfpathlineto{\pgfqpoint{1.055502in}{1.262928in}}%
\pgfpathlineto{\pgfqpoint{1.056947in}{1.264037in}}%
\pgfpathlineto{\pgfqpoint{1.058056in}{1.265049in}}%
\pgfpathlineto{\pgfqpoint{1.059418in}{1.266157in}}%
\pgfpathlineto{\pgfqpoint{1.060518in}{1.266828in}}%
\pgfpathlineto{\pgfqpoint{1.062049in}{1.267937in}}%
\pgfpathlineto{\pgfqpoint{1.063152in}{1.268725in}}%
\pgfpathlineto{\pgfqpoint{1.064546in}{1.269834in}}%
\pgfpathlineto{\pgfqpoint{1.065628in}{1.270758in}}%
\pgfpathlineto{\pgfqpoint{1.067135in}{1.271867in}}%
\pgfpathlineto{\pgfqpoint{1.068243in}{1.272839in}}%
\pgfpathlineto{\pgfqpoint{1.069620in}{1.273948in}}%
\pgfpathlineto{\pgfqpoint{1.070714in}{1.274814in}}%
\pgfpathlineto{\pgfqpoint{1.072019in}{1.275922in}}%
\pgfpathlineto{\pgfqpoint{1.073120in}{1.276837in}}%
\pgfpathlineto{\pgfqpoint{1.074686in}{1.277936in}}%
\pgfpathlineto{\pgfqpoint{1.075788in}{1.278831in}}%
\pgfpathlineto{\pgfqpoint{1.077364in}{1.279939in}}%
\pgfpathlineto{\pgfqpoint{1.078474in}{1.280717in}}%
\pgfpathlineto{\pgfqpoint{1.079914in}{1.281826in}}%
\pgfpathlineto{\pgfqpoint{1.080991in}{1.282692in}}%
\pgfpathlineto{\pgfqpoint{1.082578in}{1.283801in}}%
\pgfpathlineto{\pgfqpoint{1.083625in}{1.284530in}}%
\pgfpathlineto{\pgfqpoint{1.084995in}{1.285639in}}%
\pgfpathlineto{\pgfqpoint{1.086103in}{1.286495in}}%
\pgfpathlineto{\pgfqpoint{1.087492in}{1.287604in}}%
\pgfpathlineto{\pgfqpoint{1.088592in}{1.288460in}}%
\pgfpathlineto{\pgfqpoint{1.090354in}{1.289568in}}%
\pgfpathlineto{\pgfqpoint{1.091457in}{1.290531in}}%
\pgfpathlineto{\pgfqpoint{1.092999in}{1.291630in}}%
\pgfpathlineto{\pgfqpoint{1.094104in}{1.292486in}}%
\pgfpathlineto{\pgfqpoint{1.095722in}{1.293595in}}%
\pgfpathlineto{\pgfqpoint{1.096822in}{1.294354in}}%
\pgfpathlineto{\pgfqpoint{1.098321in}{1.295462in}}%
\pgfpathlineto{\pgfqpoint{1.099356in}{1.296289in}}%
\pgfpathlineto{\pgfqpoint{1.100906in}{1.297398in}}%
\pgfpathlineto{\pgfqpoint{1.102006in}{1.298127in}}%
\pgfpathlineto{\pgfqpoint{1.103660in}{1.299236in}}%
\pgfpathlineto{\pgfqpoint{1.104768in}{1.299975in}}%
\pgfpathlineto{\pgfqpoint{1.106443in}{1.301084in}}%
\pgfpathlineto{\pgfqpoint{1.107527in}{1.301872in}}%
\pgfpathlineto{\pgfqpoint{1.107546in}{1.301872in}}%
\pgfpathlineto{\pgfqpoint{1.109186in}{1.302981in}}%
\pgfpathlineto{\pgfqpoint{1.110271in}{1.303817in}}%
\pgfpathlineto{\pgfqpoint{1.111850in}{1.304926in}}%
\pgfpathlineto{\pgfqpoint{1.112949in}{1.305626in}}%
\pgfpathlineto{\pgfqpoint{1.114615in}{1.306735in}}%
\pgfpathlineto{\pgfqpoint{1.115717in}{1.307474in}}%
\pgfpathlineto{\pgfqpoint{1.117246in}{1.308583in}}%
\pgfpathlineto{\pgfqpoint{1.118356in}{1.309205in}}%
\pgfpathlineto{\pgfqpoint{1.120180in}{1.310314in}}%
\pgfpathlineto{\pgfqpoint{1.121281in}{1.311005in}}%
\pgfpathlineto{\pgfqpoint{1.123263in}{1.312114in}}%
\pgfpathlineto{\pgfqpoint{1.124354in}{1.312804in}}%
\pgfpathlineto{\pgfqpoint{1.125611in}{1.313913in}}%
\pgfpathlineto{\pgfqpoint{1.126644in}{1.314672in}}%
\pgfpathlineto{\pgfqpoint{1.126693in}{1.314672in}}%
\pgfpathlineto{\pgfqpoint{1.128089in}{1.315780in}}%
\pgfpathlineto{\pgfqpoint{1.129199in}{1.316520in}}%
\pgfpathlineto{\pgfqpoint{1.130678in}{1.317628in}}%
\pgfpathlineto{\pgfqpoint{1.131765in}{1.318562in}}%
\pgfpathlineto{\pgfqpoint{1.133554in}{1.319661in}}%
\pgfpathlineto{\pgfqpoint{1.134664in}{1.320507in}}%
\pgfpathlineto{\pgfqpoint{1.136190in}{1.321616in}}%
\pgfpathlineto{\pgfqpoint{1.137277in}{1.322443in}}%
\pgfpathlineto{\pgfqpoint{1.138836in}{1.323552in}}%
\pgfpathlineto{\pgfqpoint{1.139943in}{1.324330in}}%
\pgfpathlineto{\pgfqpoint{1.141565in}{1.325429in}}%
\pgfpathlineto{\pgfqpoint{1.142656in}{1.326255in}}%
\pgfpathlineto{\pgfqpoint{1.144632in}{1.327364in}}%
\pgfpathlineto{\pgfqpoint{1.145665in}{1.328035in}}%
\pgfpathlineto{\pgfqpoint{1.147212in}{1.329144in}}%
\pgfpathlineto{\pgfqpoint{1.148310in}{1.329961in}}%
\pgfpathlineto{\pgfqpoint{1.150041in}{1.331060in}}%
\pgfpathlineto{\pgfqpoint{1.151151in}{1.331780in}}%
\pgfpathlineto{\pgfqpoint{1.152445in}{1.332889in}}%
\pgfpathlineto{\pgfqpoint{1.153534in}{1.333813in}}%
\pgfpathlineto{\pgfqpoint{1.155016in}{1.334921in}}%
\pgfpathlineto{\pgfqpoint{1.156105in}{1.335593in}}%
\pgfpathlineto{\pgfqpoint{1.157573in}{1.336701in}}%
\pgfpathlineto{\pgfqpoint{1.158576in}{1.337324in}}%
\pgfpathlineto{\pgfqpoint{1.160714in}{1.338433in}}%
\pgfpathlineto{\pgfqpoint{1.161819in}{1.339045in}}%
\pgfpathlineto{\pgfqpoint{1.163783in}{1.340154in}}%
\pgfpathlineto{\pgfqpoint{1.164888in}{1.340854in}}%
\pgfpathlineto{\pgfqpoint{1.166652in}{1.341963in}}%
\pgfpathlineto{\pgfqpoint{1.167759in}{1.342732in}}%
\pgfpathlineto{\pgfqpoint{1.169511in}{1.343840in}}%
\pgfpathlineto{\pgfqpoint{1.170593in}{1.344560in}}%
\pgfpathlineto{\pgfqpoint{1.172408in}{1.345669in}}%
\pgfpathlineto{\pgfqpoint{1.173502in}{1.346389in}}%
\pgfpathlineto{\pgfqpoint{1.175400in}{1.347497in}}%
\pgfpathlineto{\pgfqpoint{1.176503in}{1.348110in}}%
\pgfpathlineto{\pgfqpoint{1.178453in}{1.349219in}}%
\pgfpathlineto{\pgfqpoint{1.179535in}{1.349948in}}%
\pgfpathlineto{\pgfqpoint{1.179551in}{1.349948in}}%
\pgfpathlineto{\pgfqpoint{1.181227in}{1.351057in}}%
\pgfpathlineto{\pgfqpoint{1.182313in}{1.351621in}}%
\pgfpathlineto{\pgfqpoint{1.184198in}{1.352720in}}%
\pgfpathlineto{\pgfqpoint{1.185291in}{1.353411in}}%
\pgfpathlineto{\pgfqpoint{1.187241in}{1.354510in}}%
\pgfpathlineto{\pgfqpoint{1.188337in}{1.355123in}}%
\pgfpathlineto{\pgfqpoint{1.189796in}{1.356232in}}%
\pgfpathlineto{\pgfqpoint{1.190899in}{1.356883in}}%
\pgfpathlineto{\pgfqpoint{1.192809in}{1.357992in}}%
\pgfpathlineto{\pgfqpoint{1.193919in}{1.358653in}}%
\pgfpathlineto{\pgfqpoint{1.195738in}{1.359752in}}%
\pgfpathlineto{\pgfqpoint{1.196846in}{1.360433in}}%
\pgfpathlineto{\pgfqpoint{1.198637in}{1.361542in}}%
\pgfpathlineto{\pgfqpoint{1.199747in}{1.362252in}}%
\pgfpathlineto{\pgfqpoint{1.201753in}{1.363361in}}%
\pgfpathlineto{\pgfqpoint{1.202849in}{1.363886in}}%
\pgfpathlineto{\pgfqpoint{1.204745in}{1.364995in}}%
\pgfpathlineto{\pgfqpoint{1.205832in}{1.365540in}}%
\pgfpathlineto{\pgfqpoint{1.207756in}{1.366648in}}%
\pgfpathlineto{\pgfqpoint{1.208822in}{1.367271in}}%
\pgfpathlineto{\pgfqpoint{1.210511in}{1.368380in}}%
\pgfpathlineto{\pgfqpoint{1.211618in}{1.369090in}}%
\pgfpathlineto{\pgfqpoint{1.213531in}{1.370198in}}%
\pgfpathlineto{\pgfqpoint{1.214606in}{1.370928in}}%
\pgfpathlineto{\pgfqpoint{1.216181in}{1.372027in}}%
\pgfpathlineto{\pgfqpoint{1.217282in}{1.372562in}}%
\pgfpathlineto{\pgfqpoint{1.219231in}{1.373671in}}%
\pgfpathlineto{\pgfqpoint{1.220327in}{1.374283in}}%
\pgfpathlineto{\pgfqpoint{1.220341in}{1.374283in}}%
\pgfpathlineto{\pgfqpoint{1.222286in}{1.375392in}}%
\pgfpathlineto{\pgfqpoint{1.223396in}{1.375917in}}%
\pgfpathlineto{\pgfqpoint{1.225018in}{1.377026in}}%
\pgfpathlineto{\pgfqpoint{1.226118in}{1.377590in}}%
\pgfpathlineto{\pgfqpoint{1.228217in}{1.378699in}}%
\pgfpathlineto{\pgfqpoint{1.229327in}{1.379448in}}%
\pgfpathlineto{\pgfqpoint{1.231079in}{1.380557in}}%
\pgfpathlineto{\pgfqpoint{1.232170in}{1.381218in}}%
\pgfpathlineto{\pgfqpoint{1.233974in}{1.382327in}}%
\pgfpathlineto{\pgfqpoint{1.235083in}{1.383076in}}%
\pgfpathlineto{\pgfqpoint{1.237375in}{1.384185in}}%
\pgfpathlineto{\pgfqpoint{1.238485in}{1.384710in}}%
\pgfpathlineto{\pgfqpoint{1.240612in}{1.385819in}}%
\pgfpathlineto{\pgfqpoint{1.241719in}{1.386626in}}%
\pgfpathlineto{\pgfqpoint{1.243418in}{1.387735in}}%
\pgfpathlineto{\pgfqpoint{1.244523in}{1.388445in}}%
\pgfpathlineto{\pgfqpoint{1.246114in}{1.389553in}}%
\pgfpathlineto{\pgfqpoint{1.247213in}{1.390137in}}%
\pgfpathlineto{\pgfqpoint{1.249067in}{1.391246in}}%
\pgfpathlineto{\pgfqpoint{1.250151in}{1.391907in}}%
\pgfpathlineto{\pgfqpoint{1.252131in}{1.393016in}}%
\pgfpathlineto{\pgfqpoint{1.253237in}{1.393619in}}%
\pgfpathlineto{\pgfqpoint{1.255144in}{1.394728in}}%
\pgfpathlineto{\pgfqpoint{1.256238in}{1.395263in}}%
\pgfpathlineto{\pgfqpoint{1.258041in}{1.396372in}}%
\pgfpathlineto{\pgfqpoint{1.259149in}{1.397033in}}%
\pgfpathlineto{\pgfqpoint{1.261357in}{1.398142in}}%
\pgfpathlineto{\pgfqpoint{1.262697in}{1.399017in}}%
\pgfpathlineto{\pgfqpoint{1.264654in}{1.400126in}}%
\pgfpathlineto{\pgfqpoint{1.265757in}{1.400777in}}%
\pgfpathlineto{\pgfqpoint{1.267576in}{1.401886in}}%
\pgfpathlineto{\pgfqpoint{1.268679in}{1.402411in}}%
\pgfpathlineto{\pgfqpoint{1.271029in}{1.403520in}}%
\pgfpathlineto{\pgfqpoint{1.272123in}{1.404182in}}%
\pgfpathlineto{\pgfqpoint{1.274149in}{1.405290in}}%
\pgfpathlineto{\pgfqpoint{1.275254in}{1.405874in}}%
\pgfpathlineto{\pgfqpoint{1.277532in}{1.406983in}}%
\pgfpathlineto{\pgfqpoint{1.278628in}{1.407508in}}%
\pgfpathlineto{\pgfqpoint{1.280457in}{1.408617in}}%
\pgfpathlineto{\pgfqpoint{1.281546in}{1.409181in}}%
\pgfpathlineto{\pgfqpoint{1.283645in}{1.410290in}}%
\pgfpathlineto{\pgfqpoint{1.284754in}{1.410844in}}%
\pgfpathlineto{\pgfqpoint{1.287070in}{1.411953in}}%
\pgfpathlineto{\pgfqpoint{1.288165in}{1.412400in}}%
\pgfpathlineto{\pgfqpoint{1.290408in}{1.413509in}}%
\pgfpathlineto{\pgfqpoint{1.291509in}{1.414200in}}%
\pgfpathlineto{\pgfqpoint{1.291516in}{1.414200in}}%
\pgfpathlineto{\pgfqpoint{1.293468in}{1.415308in}}%
\pgfpathlineto{\pgfqpoint{1.294569in}{1.415863in}}%
\pgfpathlineto{\pgfqpoint{1.296867in}{1.416972in}}%
\pgfpathlineto{\pgfqpoint{1.297977in}{1.417623in}}%
\pgfpathlineto{\pgfqpoint{1.300232in}{1.418732in}}%
\pgfpathlineto{\pgfqpoint{1.301330in}{1.419335in}}%
\pgfpathlineto{\pgfqpoint{1.303291in}{1.420444in}}%
\pgfpathlineto{\pgfqpoint{1.304352in}{1.421115in}}%
\pgfpathlineto{\pgfqpoint{1.306854in}{1.422224in}}%
\pgfpathlineto{\pgfqpoint{1.307950in}{1.422778in}}%
\pgfpathlineto{\pgfqpoint{1.310241in}{1.423887in}}%
\pgfpathlineto{\pgfqpoint{1.311340in}{1.424509in}}%
\pgfpathlineto{\pgfqpoint{1.313629in}{1.425618in}}%
\pgfpathlineto{\pgfqpoint{1.314695in}{1.426250in}}%
\pgfpathlineto{\pgfqpoint{1.314702in}{1.426250in}}%
\pgfpathlineto{\pgfqpoint{1.317268in}{1.427359in}}%
\pgfpathlineto{\pgfqpoint{1.318378in}{1.427904in}}%
\pgfpathlineto{\pgfqpoint{1.320714in}{1.429013in}}%
\pgfpathlineto{\pgfqpoint{1.321817in}{1.429635in}}%
\pgfpathlineto{\pgfqpoint{1.323795in}{1.430744in}}%
\pgfpathlineto{\pgfqpoint{1.324902in}{1.431337in}}%
\pgfpathlineto{\pgfqpoint{1.327005in}{1.432446in}}%
\pgfpathlineto{\pgfqpoint{1.328108in}{1.433020in}}%
\pgfpathlineto{\pgfqpoint{1.330845in}{1.434129in}}%
\pgfpathlineto{\pgfqpoint{1.331920in}{1.434771in}}%
\pgfpathlineto{\pgfqpoint{1.334651in}{1.435879in}}%
\pgfpathlineto{\pgfqpoint{1.335717in}{1.436366in}}%
\pgfpathlineto{\pgfqpoint{1.337957in}{1.437474in}}%
\pgfpathlineto{\pgfqpoint{1.339000in}{1.437863in}}%
\pgfpathlineto{\pgfqpoint{1.341555in}{1.438972in}}%
\pgfpathlineto{\pgfqpoint{1.342583in}{1.439459in}}%
\pgfpathlineto{\pgfqpoint{1.342648in}{1.439459in}}%
\pgfpathlineto{\pgfqpoint{1.344726in}{1.440567in}}%
\pgfpathlineto{\pgfqpoint{1.345824in}{1.441093in}}%
\pgfpathlineto{\pgfqpoint{1.348339in}{1.442201in}}%
\pgfpathlineto{\pgfqpoint{1.349407in}{1.442678in}}%
\pgfpathlineto{\pgfqpoint{1.351857in}{1.443787in}}%
\pgfpathlineto{\pgfqpoint{1.352960in}{1.444244in}}%
\pgfpathlineto{\pgfqpoint{1.355543in}{1.445353in}}%
\pgfpathlineto{\pgfqpoint{1.356613in}{1.445751in}}%
\pgfpathlineto{\pgfqpoint{1.358993in}{1.446860in}}%
\pgfpathlineto{\pgfqpoint{1.360003in}{1.447288in}}%
\pgfpathlineto{\pgfqpoint{1.362681in}{1.448397in}}%
\pgfpathlineto{\pgfqpoint{1.363768in}{1.448942in}}%
\pgfpathlineto{\pgfqpoint{1.366076in}{1.450050in}}%
\pgfpathlineto{\pgfqpoint{1.367183in}{1.450614in}}%
\pgfpathlineto{\pgfqpoint{1.369869in}{1.451723in}}%
\pgfpathlineto{\pgfqpoint{1.370906in}{1.452093in}}%
\pgfpathlineto{\pgfqpoint{1.373919in}{1.453202in}}%
\pgfpathlineto{\pgfqpoint{1.375013in}{1.453824in}}%
\pgfpathlineto{\pgfqpoint{1.377928in}{1.454933in}}%
\pgfpathlineto{\pgfqpoint{1.378985in}{1.455390in}}%
\pgfpathlineto{\pgfqpoint{1.381991in}{1.456499in}}%
\pgfpathlineto{\pgfqpoint{1.383096in}{1.456898in}}%
\pgfpathlineto{\pgfqpoint{1.385446in}{1.457997in}}%
\pgfpathlineto{\pgfqpoint{1.386556in}{1.458493in}}%
\pgfpathlineto{\pgfqpoint{1.389369in}{1.459601in}}%
\pgfpathlineto{\pgfqpoint{1.390472in}{1.460097in}}%
\pgfpathlineto{\pgfqpoint{1.393241in}{1.461206in}}%
\pgfpathlineto{\pgfqpoint{1.394281in}{1.461508in}}%
\pgfpathlineto{\pgfqpoint{1.397117in}{1.462617in}}%
\pgfpathlineto{\pgfqpoint{1.398224in}{1.463045in}}%
\pgfpathlineto{\pgfqpoint{1.401282in}{1.464153in}}%
\pgfpathlineto{\pgfqpoint{1.402364in}{1.464640in}}%
\pgfpathlineto{\pgfqpoint{1.404423in}{1.465748in}}%
\pgfpathlineto{\pgfqpoint{1.405512in}{1.466108in}}%
\pgfpathlineto{\pgfqpoint{1.408315in}{1.467217in}}%
\pgfpathlineto{\pgfqpoint{1.409369in}{1.467664in}}%
\pgfpathlineto{\pgfqpoint{1.409414in}{1.467664in}}%
\pgfpathlineto{\pgfqpoint{1.412266in}{1.468764in}}%
\pgfpathlineto{\pgfqpoint{1.413364in}{1.469230in}}%
\pgfpathlineto{\pgfqpoint{1.415919in}{1.470339in}}%
\pgfpathlineto{\pgfqpoint{1.417020in}{1.470884in}}%
\pgfpathlineto{\pgfqpoint{1.419968in}{1.471993in}}%
\pgfpathlineto{\pgfqpoint{1.421073in}{1.472352in}}%
\pgfpathlineto{\pgfqpoint{1.423486in}{1.473461in}}%
\pgfpathlineto{\pgfqpoint{1.424556in}{1.473860in}}%
\pgfpathlineto{\pgfqpoint{1.427213in}{1.474969in}}%
\pgfpathlineto{\pgfqpoint{1.428309in}{1.475523in}}%
\pgfpathlineto{\pgfqpoint{1.431189in}{1.476632in}}%
\pgfpathlineto{\pgfqpoint{1.432288in}{1.477177in}}%
\pgfpathlineto{\pgfqpoint{1.434924in}{1.478285in}}%
\pgfpathlineto{\pgfqpoint{1.436034in}{1.478743in}}%
\pgfpathlineto{\pgfqpoint{1.438605in}{1.479851in}}%
\pgfpathlineto{\pgfqpoint{1.439708in}{1.480425in}}%
\pgfpathlineto{\pgfqpoint{1.442735in}{1.481534in}}%
\pgfpathlineto{\pgfqpoint{1.443819in}{1.482001in}}%
\pgfpathlineto{\pgfqpoint{1.446325in}{1.483110in}}%
\pgfpathlineto{\pgfqpoint{1.447281in}{1.483528in}}%
\pgfpathlineto{\pgfqpoint{1.450222in}{1.484637in}}%
\pgfpathlineto{\pgfqpoint{1.451274in}{1.485055in}}%
\pgfpathlineto{\pgfqpoint{1.454573in}{1.486164in}}%
\pgfpathlineto{\pgfqpoint{1.455676in}{1.486660in}}%
\pgfpathlineto{\pgfqpoint{1.458664in}{1.487759in}}%
\pgfpathlineto{\pgfqpoint{1.459743in}{1.488313in}}%
\pgfpathlineto{\pgfqpoint{1.462961in}{1.489422in}}%
\pgfpathlineto{\pgfqpoint{1.464015in}{1.489899in}}%
\pgfpathlineto{\pgfqpoint{1.464071in}{1.489899in}}%
\pgfpathlineto{\pgfqpoint{1.466609in}{1.491007in}}%
\pgfpathlineto{\pgfqpoint{1.467719in}{1.491484in}}%
\pgfpathlineto{\pgfqpoint{1.470981in}{1.492593in}}%
\pgfpathlineto{\pgfqpoint{1.472038in}{1.493001in}}%
\pgfpathlineto{\pgfqpoint{1.475493in}{1.494110in}}%
\pgfpathlineto{\pgfqpoint{1.476600in}{1.494596in}}%
\pgfpathlineto{\pgfqpoint{1.479916in}{1.495705in}}%
\pgfpathlineto{\pgfqpoint{1.480991in}{1.496123in}}%
\pgfpathlineto{\pgfqpoint{1.483839in}{1.497232in}}%
\pgfpathlineto{\pgfqpoint{1.484949in}{1.497582in}}%
\pgfpathlineto{\pgfqpoint{1.488101in}{1.498691in}}%
\pgfpathlineto{\pgfqpoint{1.489195in}{1.499041in}}%
\pgfpathlineto{\pgfqpoint{1.492248in}{1.500150in}}%
\pgfpathlineto{\pgfqpoint{1.493322in}{1.500471in}}%
\pgfpathlineto{\pgfqpoint{1.496671in}{1.501580in}}%
\pgfpathlineto{\pgfqpoint{1.497778in}{1.501920in}}%
\pgfpathlineto{\pgfqpoint{1.500884in}{1.503029in}}%
\pgfpathlineto{\pgfqpoint{1.501990in}{1.503457in}}%
\pgfpathlineto{\pgfqpoint{1.505559in}{1.504566in}}%
\pgfpathlineto{\pgfqpoint{1.506634in}{1.505081in}}%
\pgfpathlineto{\pgfqpoint{1.510070in}{1.506190in}}%
\pgfpathlineto{\pgfqpoint{1.511071in}{1.506530in}}%
\pgfpathlineto{\pgfqpoint{1.511155in}{1.506530in}}%
\pgfpathlineto{\pgfqpoint{1.514991in}{1.507629in}}%
\pgfpathlineto{\pgfqpoint{1.516071in}{1.508009in}}%
\pgfpathlineto{\pgfqpoint{1.519929in}{1.509117in}}%
\pgfpathlineto{\pgfqpoint{1.521022in}{1.509400in}}%
\pgfpathlineto{\pgfqpoint{1.525029in}{1.510508in}}%
\pgfpathlineto{\pgfqpoint{1.526118in}{1.510897in}}%
\pgfpathlineto{\pgfqpoint{1.529243in}{1.512006in}}%
\pgfpathlineto{\pgfqpoint{1.530336in}{1.512434in}}%
\pgfpathlineto{\pgfqpoint{1.533919in}{1.513543in}}%
\pgfpathlineto{\pgfqpoint{1.535024in}{1.513864in}}%
\pgfpathlineto{\pgfqpoint{1.538347in}{1.514973in}}%
\pgfpathlineto{\pgfqpoint{1.539578in}{1.515371in}}%
\pgfpathlineto{\pgfqpoint{1.542968in}{1.516480in}}%
\pgfpathlineto{\pgfqpoint{1.544066in}{1.516782in}}%
\pgfpathlineto{\pgfqpoint{1.547698in}{1.517890in}}%
\pgfpathlineto{\pgfqpoint{1.548801in}{1.518241in}}%
\pgfpathlineto{\pgfqpoint{1.552065in}{1.519349in}}%
\pgfpathlineto{\pgfqpoint{1.553068in}{1.519680in}}%
\pgfpathlineto{\pgfqpoint{1.557121in}{1.520789in}}%
\pgfpathlineto{\pgfqpoint{1.558227in}{1.521139in}}%
\pgfpathlineto{\pgfqpoint{1.562540in}{1.522248in}}%
\pgfpathlineto{\pgfqpoint{1.563590in}{1.522676in}}%
\pgfpathlineto{\pgfqpoint{1.567077in}{1.523785in}}%
\pgfpathlineto{\pgfqpoint{1.568078in}{1.524115in}}%
\pgfpathlineto{\pgfqpoint{1.571591in}{1.525214in}}%
\pgfpathlineto{\pgfqpoint{1.572687in}{1.525467in}}%
\pgfpathlineto{\pgfqpoint{1.577148in}{1.526576in}}%
\pgfpathlineto{\pgfqpoint{1.578241in}{1.526887in}}%
\pgfpathlineto{\pgfqpoint{1.578257in}{1.526887in}}%
\pgfpathlineto{\pgfqpoint{1.581678in}{1.527996in}}%
\pgfpathlineto{\pgfqpoint{1.582771in}{1.528259in}}%
\pgfpathlineto{\pgfqpoint{1.586599in}{1.529367in}}%
\pgfpathlineto{\pgfqpoint{1.587688in}{1.529698in}}%
\pgfpathlineto{\pgfqpoint{1.587704in}{1.529698in}}%
\pgfpathlineto{\pgfqpoint{1.592099in}{1.530807in}}%
\pgfpathlineto{\pgfqpoint{1.593186in}{1.531128in}}%
\pgfpathlineto{\pgfqpoint{1.596974in}{1.532237in}}%
\pgfpathlineto{\pgfqpoint{1.598058in}{1.532596in}}%
\pgfpathlineto{\pgfqpoint{1.602116in}{1.533705in}}%
\pgfpathlineto{\pgfqpoint{1.603184in}{1.534016in}}%
\pgfpathlineto{\pgfqpoint{1.607542in}{1.535125in}}%
\pgfpathlineto{\pgfqpoint{1.608617in}{1.535495in}}%
\pgfpathlineto{\pgfqpoint{1.612879in}{1.536604in}}%
\pgfpathlineto{\pgfqpoint{1.613954in}{1.536934in}}%
\pgfpathlineto{\pgfqpoint{1.613963in}{1.536934in}}%
\pgfpathlineto{\pgfqpoint{1.618191in}{1.538043in}}%
\pgfpathlineto{\pgfqpoint{1.619261in}{1.538364in}}%
\pgfpathlineto{\pgfqpoint{1.619299in}{1.538364in}}%
\pgfpathlineto{\pgfqpoint{1.623922in}{1.539473in}}%
\pgfpathlineto{\pgfqpoint{1.624901in}{1.539726in}}%
\pgfpathlineto{\pgfqpoint{1.625011in}{1.539726in}}%
\pgfpathlineto{\pgfqpoint{1.630083in}{1.540835in}}%
\pgfpathlineto{\pgfqpoint{1.631172in}{1.541117in}}%
\pgfpathlineto{\pgfqpoint{1.635595in}{1.542225in}}%
\pgfpathlineto{\pgfqpoint{1.636705in}{1.542517in}}%
\pgfpathlineto{\pgfqpoint{1.641982in}{1.543626in}}%
\pgfpathlineto{\pgfqpoint{1.643068in}{1.543889in}}%
\pgfpathlineto{\pgfqpoint{1.647343in}{1.544997in}}%
\pgfpathlineto{\pgfqpoint{1.648427in}{1.545221in}}%
\pgfpathlineto{\pgfqpoint{1.652475in}{1.546330in}}%
\pgfpathlineto{\pgfqpoint{1.653576in}{1.546651in}}%
\pgfpathlineto{\pgfqpoint{1.658346in}{1.547750in}}%
\pgfpathlineto{\pgfqpoint{1.659449in}{1.548051in}}%
\pgfpathlineto{\pgfqpoint{1.664081in}{1.549160in}}%
\pgfpathlineto{\pgfqpoint{1.665191in}{1.549374in}}%
\pgfpathlineto{\pgfqpoint{1.669589in}{1.550483in}}%
\pgfpathlineto{\pgfqpoint{1.670670in}{1.550755in}}%
\pgfpathlineto{\pgfqpoint{1.675103in}{1.551864in}}%
\pgfpathlineto{\pgfqpoint{1.676189in}{1.552117in}}%
\pgfpathlineto{\pgfqpoint{1.681010in}{1.553216in}}%
\pgfpathlineto{\pgfqpoint{1.682037in}{1.553459in}}%
\pgfpathlineto{\pgfqpoint{1.682106in}{1.553459in}}%
\pgfpathlineto{\pgfqpoint{1.686909in}{1.554568in}}%
\pgfpathlineto{\pgfqpoint{1.687993in}{1.554918in}}%
\pgfpathlineto{\pgfqpoint{1.692432in}{1.556027in}}%
\pgfpathlineto{\pgfqpoint{1.693535in}{1.556289in}}%
\pgfpathlineto{\pgfqpoint{1.698338in}{1.557398in}}%
\pgfpathlineto{\pgfqpoint{1.699424in}{1.557544in}}%
\pgfpathlineto{\pgfqpoint{1.704499in}{1.558653in}}%
\pgfpathlineto{\pgfqpoint{1.705590in}{1.558925in}}%
\pgfpathlineto{\pgfqpoint{1.710681in}{1.560034in}}%
\pgfpathlineto{\pgfqpoint{1.711716in}{1.560277in}}%
\pgfpathlineto{\pgfqpoint{1.711754in}{1.560277in}}%
\pgfpathlineto{\pgfqpoint{1.716374in}{1.561386in}}%
\pgfpathlineto{\pgfqpoint{1.717356in}{1.561590in}}%
\pgfpathlineto{\pgfqpoint{1.723052in}{1.562699in}}%
\pgfpathlineto{\pgfqpoint{1.723834in}{1.562874in}}%
\pgfpathlineto{\pgfqpoint{1.724122in}{1.562874in}}%
\pgfpathlineto{\pgfqpoint{1.729023in}{1.563983in}}%
\pgfpathlineto{\pgfqpoint{1.730121in}{1.564236in}}%
\pgfpathlineto{\pgfqpoint{1.735221in}{1.565345in}}%
\pgfpathlineto{\pgfqpoint{1.736280in}{1.565559in}}%
\pgfpathlineto{\pgfqpoint{1.741566in}{1.566667in}}%
\pgfpathlineto{\pgfqpoint{1.742566in}{1.566833in}}%
\pgfpathlineto{\pgfqpoint{1.748102in}{1.567941in}}%
\pgfpathlineto{\pgfqpoint{1.749193in}{1.568087in}}%
\pgfpathlineto{\pgfqpoint{1.754347in}{1.569186in}}%
\pgfpathlineto{\pgfqpoint{1.755373in}{1.569391in}}%
\pgfpathlineto{\pgfqpoint{1.762064in}{1.570499in}}%
\pgfpathlineto{\pgfqpoint{1.763093in}{1.570713in}}%
\pgfpathlineto{\pgfqpoint{1.769731in}{1.571822in}}%
\pgfpathlineto{\pgfqpoint{1.770804in}{1.572046in}}%
\pgfpathlineto{\pgfqpoint{1.777128in}{1.573155in}}%
\pgfpathlineto{\pgfqpoint{1.778079in}{1.573330in}}%
\pgfpathlineto{\pgfqpoint{1.778200in}{1.573330in}}%
\pgfpathlineto{\pgfqpoint{1.785860in}{1.574439in}}%
\pgfpathlineto{\pgfqpoint{1.786874in}{1.574614in}}%
\pgfpathlineto{\pgfqpoint{1.786919in}{1.574614in}}%
\pgfpathlineto{\pgfqpoint{1.793340in}{1.575722in}}%
\pgfpathlineto{\pgfqpoint{1.794415in}{1.575966in}}%
\pgfpathlineto{\pgfqpoint{1.800162in}{1.577074in}}%
\pgfpathlineto{\pgfqpoint{1.801212in}{1.577269in}}%
\pgfpathlineto{\pgfqpoint{1.809330in}{1.578378in}}%
\pgfpathlineto{\pgfqpoint{1.810300in}{1.578543in}}%
\pgfpathlineto{\pgfqpoint{1.810393in}{1.578543in}}%
\pgfpathlineto{\pgfqpoint{1.817715in}{1.579652in}}%
\pgfpathlineto{\pgfqpoint{1.818711in}{1.579905in}}%
\pgfpathlineto{\pgfqpoint{1.826091in}{1.581004in}}%
\pgfpathlineto{\pgfqpoint{1.827155in}{1.581188in}}%
\pgfpathlineto{\pgfqpoint{1.827192in}{1.581188in}}%
\pgfpathlineto{\pgfqpoint{1.835352in}{1.582297in}}%
\pgfpathlineto{\pgfqpoint{1.836457in}{1.582531in}}%
\pgfpathlineto{\pgfqpoint{1.842949in}{1.583639in}}%
\pgfpathlineto{\pgfqpoint{1.843998in}{1.583863in}}%
\pgfpathlineto{\pgfqpoint{1.852712in}{1.584972in}}%
\pgfpathlineto{\pgfqpoint{1.853663in}{1.585108in}}%
\pgfpathlineto{\pgfqpoint{1.853728in}{1.585108in}}%
\pgfpathlineto{\pgfqpoint{1.861614in}{1.586217in}}%
\pgfpathlineto{\pgfqpoint{1.862435in}{1.586334in}}%
\pgfpathlineto{\pgfqpoint{1.862668in}{1.586334in}}%
\pgfpathlineto{\pgfqpoint{1.870497in}{1.587442in}}%
\pgfpathlineto{\pgfqpoint{1.871600in}{1.587598in}}%
\pgfpathlineto{\pgfqpoint{1.880495in}{1.588707in}}%
\pgfpathlineto{\pgfqpoint{1.881563in}{1.588911in}}%
\pgfpathlineto{\pgfqpoint{1.890160in}{1.590020in}}%
\pgfpathlineto{\pgfqpoint{1.891231in}{1.590175in}}%
\pgfpathlineto{\pgfqpoint{1.891254in}{1.590175in}}%
\pgfpathlineto{\pgfqpoint{1.901040in}{1.591284in}}%
\pgfpathlineto{\pgfqpoint{1.902087in}{1.591372in}}%
\pgfpathlineto{\pgfqpoint{1.902101in}{1.591372in}}%
\pgfpathlineto{\pgfqpoint{1.911938in}{1.592481in}}%
\pgfpathlineto{\pgfqpoint{1.913013in}{1.592636in}}%
\pgfpathlineto{\pgfqpoint{1.923516in}{1.593745in}}%
\pgfpathlineto{\pgfqpoint{1.924312in}{1.593852in}}%
\pgfpathlineto{\pgfqpoint{1.924607in}{1.593852in}}%
\pgfpathlineto{\pgfqpoint{1.935224in}{1.594961in}}%
\pgfpathlineto{\pgfqpoint{1.936325in}{1.595116in}}%
\pgfpathlineto{\pgfqpoint{1.949150in}{1.596225in}}%
\pgfpathlineto{\pgfqpoint{1.950225in}{1.596361in}}%
\pgfpathlineto{\pgfqpoint{1.962191in}{1.597460in}}%
\pgfpathlineto{\pgfqpoint{1.963217in}{1.597616in}}%
\pgfpathlineto{\pgfqpoint{1.978104in}{1.598725in}}%
\pgfpathlineto{\pgfqpoint{1.979213in}{1.598842in}}%
\pgfpathlineto{\pgfqpoint{1.994886in}{1.599950in}}%
\pgfpathlineto{\pgfqpoint{1.995936in}{1.600057in}}%
\pgfpathlineto{\pgfqpoint{2.013612in}{1.601166in}}%
\pgfpathlineto{\pgfqpoint{2.014508in}{1.601234in}}%
\pgfpathlineto{\pgfqpoint{2.014703in}{1.601234in}}%
\pgfpathlineto{\pgfqpoint{2.033126in}{1.601944in}}%
\pgfpathlineto{\pgfqpoint{2.033126in}{1.601944in}}%
\pgfusepath{stroke}%
\end{pgfscope}%
\begin{pgfscope}%
\pgfsetrectcap%
\pgfsetmiterjoin%
\pgfsetlinewidth{0.803000pt}%
\definecolor{currentstroke}{rgb}{0.000000,0.000000,0.000000}%
\pgfsetstrokecolor{currentstroke}%
\pgfsetdash{}{0pt}%
\pgfpathmoveto{\pgfqpoint{0.553581in}{0.499444in}}%
\pgfpathlineto{\pgfqpoint{0.553581in}{1.654444in}}%
\pgfusepath{stroke}%
\end{pgfscope}%
\begin{pgfscope}%
\pgfsetrectcap%
\pgfsetmiterjoin%
\pgfsetlinewidth{0.803000pt}%
\definecolor{currentstroke}{rgb}{0.000000,0.000000,0.000000}%
\pgfsetstrokecolor{currentstroke}%
\pgfsetdash{}{0pt}%
\pgfpathmoveto{\pgfqpoint{2.103581in}{0.499444in}}%
\pgfpathlineto{\pgfqpoint{2.103581in}{1.654444in}}%
\pgfusepath{stroke}%
\end{pgfscope}%
\begin{pgfscope}%
\pgfsetrectcap%
\pgfsetmiterjoin%
\pgfsetlinewidth{0.803000pt}%
\definecolor{currentstroke}{rgb}{0.000000,0.000000,0.000000}%
\pgfsetstrokecolor{currentstroke}%
\pgfsetdash{}{0pt}%
\pgfpathmoveto{\pgfqpoint{0.553581in}{0.499444in}}%
\pgfpathlineto{\pgfqpoint{2.103581in}{0.499444in}}%
\pgfusepath{stroke}%
\end{pgfscope}%
\begin{pgfscope}%
\pgfsetrectcap%
\pgfsetmiterjoin%
\pgfsetlinewidth{0.803000pt}%
\definecolor{currentstroke}{rgb}{0.000000,0.000000,0.000000}%
\pgfsetstrokecolor{currentstroke}%
\pgfsetdash{}{0pt}%
\pgfpathmoveto{\pgfqpoint{0.553581in}{1.654444in}}%
\pgfpathlineto{\pgfqpoint{2.103581in}{1.654444in}}%
\pgfusepath{stroke}%
\end{pgfscope}%
\begin{pgfscope}%
\pgfsetbuttcap%
\pgfsetmiterjoin%
\definecolor{currentfill}{rgb}{1.000000,1.000000,1.000000}%
\pgfsetfillcolor{currentfill}%
\pgfsetfillopacity{0.800000}%
\pgfsetlinewidth{1.003750pt}%
\definecolor{currentstroke}{rgb}{0.800000,0.800000,0.800000}%
\pgfsetstrokecolor{currentstroke}%
\pgfsetstrokeopacity{0.800000}%
\pgfsetdash{}{0pt}%
\pgfpathmoveto{\pgfqpoint{0.832747in}{0.568889in}}%
\pgfpathlineto{\pgfqpoint{2.006358in}{0.568889in}}%
\pgfpathquadraticcurveto{\pgfqpoint{2.034136in}{0.568889in}}{\pgfqpoint{2.034136in}{0.596666in}}%
\pgfpathlineto{\pgfqpoint{2.034136in}{0.776388in}}%
\pgfpathquadraticcurveto{\pgfqpoint{2.034136in}{0.804166in}}{\pgfqpoint{2.006358in}{0.804166in}}%
\pgfpathlineto{\pgfqpoint{0.832747in}{0.804166in}}%
\pgfpathquadraticcurveto{\pgfqpoint{0.804970in}{0.804166in}}{\pgfqpoint{0.804970in}{0.776388in}}%
\pgfpathlineto{\pgfqpoint{0.804970in}{0.596666in}}%
\pgfpathquadraticcurveto{\pgfqpoint{0.804970in}{0.568889in}}{\pgfqpoint{0.832747in}{0.568889in}}%
\pgfpathlineto{\pgfqpoint{0.832747in}{0.568889in}}%
\pgfpathclose%
\pgfusepath{stroke,fill}%
\end{pgfscope}%
\begin{pgfscope}%
\pgfsetrectcap%
\pgfsetroundjoin%
\pgfsetlinewidth{1.505625pt}%
\definecolor{currentstroke}{rgb}{0.000000,0.000000,0.000000}%
\pgfsetstrokecolor{currentstroke}%
\pgfsetdash{}{0pt}%
\pgfpathmoveto{\pgfqpoint{0.860525in}{0.700000in}}%
\pgfpathlineto{\pgfqpoint{0.999414in}{0.700000in}}%
\pgfpathlineto{\pgfqpoint{1.138303in}{0.700000in}}%
\pgfusepath{stroke}%
\end{pgfscope}%
\begin{pgfscope}%
\definecolor{textcolor}{rgb}{0.000000,0.000000,0.000000}%
\pgfsetstrokecolor{textcolor}%
\pgfsetfillcolor{textcolor}%
\pgftext[x=1.249414in,y=0.651388in,left,base]{\color{textcolor}\rmfamily\fontsize{10.000000}{12.000000}\selectfont AUC=0.753}%
\end{pgfscope}%
\end{pgfpicture}%
\makeatother%
\endgroup%

\end{tabular}

\



The model below is as effective at separating the two classes ($\text{ROC}=0.778$), but the distribution is skewed to the left.  Its results were nearly continuous, with the 214,070 samples returning 210,157 unique values of $p$, so we can fine tune the decision threshold.  

\

%
\verb|KBFC_5_Fold_alpha_0_5_gamma_0_0_Hard_Test|

\

\noindent\begin{tabular}{@{\hspace{-6pt}}p{4.3in} @{\hspace{-6pt}}p{2.0in}}
	\vskip 0pt
	\hfil Raw Model Output
	
	%% Creator: Matplotlib, PGF backend
%%
%% To include the figure in your LaTeX document, write
%%   \input{<filename>.pgf}
%%
%% Make sure the required packages are loaded in your preamble
%%   \usepackage{pgf}
%%
%% Also ensure that all the required font packages are loaded; for instance,
%% the lmodern package is sometimes necessary when using math font.
%%   \usepackage{lmodern}
%%
%% Figures using additional raster images can only be included by \input if
%% they are in the same directory as the main LaTeX file. For loading figures
%% from other directories you can use the `import` package
%%   \usepackage{import}
%%
%% and then include the figures with
%%   \import{<path to file>}{<filename>.pgf}
%%
%% Matplotlib used the following preamble
%%   
%%   \usepackage{fontspec}
%%   \makeatletter\@ifpackageloaded{underscore}{}{\usepackage[strings]{underscore}}\makeatother
%%
\begingroup%
\makeatletter%
\begin{pgfpicture}%
\pgfpathrectangle{\pgfpointorigin}{\pgfqpoint{4.002500in}{1.654444in}}%
\pgfusepath{use as bounding box, clip}%
\begin{pgfscope}%
\pgfsetbuttcap%
\pgfsetmiterjoin%
\definecolor{currentfill}{rgb}{1.000000,1.000000,1.000000}%
\pgfsetfillcolor{currentfill}%
\pgfsetlinewidth{0.000000pt}%
\definecolor{currentstroke}{rgb}{1.000000,1.000000,1.000000}%
\pgfsetstrokecolor{currentstroke}%
\pgfsetdash{}{0pt}%
\pgfpathmoveto{\pgfqpoint{0.000000in}{0.000000in}}%
\pgfpathlineto{\pgfqpoint{4.002500in}{0.000000in}}%
\pgfpathlineto{\pgfqpoint{4.002500in}{1.654444in}}%
\pgfpathlineto{\pgfqpoint{0.000000in}{1.654444in}}%
\pgfpathlineto{\pgfqpoint{0.000000in}{0.000000in}}%
\pgfpathclose%
\pgfusepath{fill}%
\end{pgfscope}%
\begin{pgfscope}%
\pgfsetbuttcap%
\pgfsetmiterjoin%
\definecolor{currentfill}{rgb}{1.000000,1.000000,1.000000}%
\pgfsetfillcolor{currentfill}%
\pgfsetlinewidth{0.000000pt}%
\definecolor{currentstroke}{rgb}{0.000000,0.000000,0.000000}%
\pgfsetstrokecolor{currentstroke}%
\pgfsetstrokeopacity{0.000000}%
\pgfsetdash{}{0pt}%
\pgfpathmoveto{\pgfqpoint{0.465000in}{0.449444in}}%
\pgfpathlineto{\pgfqpoint{3.952500in}{0.449444in}}%
\pgfpathlineto{\pgfqpoint{3.952500in}{1.604444in}}%
\pgfpathlineto{\pgfqpoint{0.465000in}{1.604444in}}%
\pgfpathlineto{\pgfqpoint{0.465000in}{0.449444in}}%
\pgfpathclose%
\pgfusepath{fill}%
\end{pgfscope}%
\begin{pgfscope}%
\pgfpathrectangle{\pgfqpoint{0.465000in}{0.449444in}}{\pgfqpoint{3.487500in}{1.155000in}}%
\pgfusepath{clip}%
\pgfsetbuttcap%
\pgfsetmiterjoin%
\pgfsetlinewidth{1.003750pt}%
\definecolor{currentstroke}{rgb}{0.000000,0.000000,0.000000}%
\pgfsetstrokecolor{currentstroke}%
\pgfsetdash{}{0pt}%
\pgfpathmoveto{\pgfqpoint{0.560114in}{0.449444in}}%
\pgfpathlineto{\pgfqpoint{0.623523in}{0.449444in}}%
\pgfpathlineto{\pgfqpoint{0.623523in}{0.449444in}}%
\pgfpathlineto{\pgfqpoint{0.560114in}{0.449444in}}%
\pgfpathlineto{\pgfqpoint{0.560114in}{0.449444in}}%
\pgfpathclose%
\pgfusepath{stroke}%
\end{pgfscope}%
\begin{pgfscope}%
\pgfpathrectangle{\pgfqpoint{0.465000in}{0.449444in}}{\pgfqpoint{3.487500in}{1.155000in}}%
\pgfusepath{clip}%
\pgfsetbuttcap%
\pgfsetmiterjoin%
\pgfsetlinewidth{1.003750pt}%
\definecolor{currentstroke}{rgb}{0.000000,0.000000,0.000000}%
\pgfsetstrokecolor{currentstroke}%
\pgfsetdash{}{0pt}%
\pgfpathmoveto{\pgfqpoint{0.718637in}{0.449444in}}%
\pgfpathlineto{\pgfqpoint{0.782046in}{0.449444in}}%
\pgfpathlineto{\pgfqpoint{0.782046in}{1.549444in}}%
\pgfpathlineto{\pgfqpoint{0.718637in}{1.549444in}}%
\pgfpathlineto{\pgfqpoint{0.718637in}{0.449444in}}%
\pgfpathclose%
\pgfusepath{stroke}%
\end{pgfscope}%
\begin{pgfscope}%
\pgfpathrectangle{\pgfqpoint{0.465000in}{0.449444in}}{\pgfqpoint{3.487500in}{1.155000in}}%
\pgfusepath{clip}%
\pgfsetbuttcap%
\pgfsetmiterjoin%
\pgfsetlinewidth{1.003750pt}%
\definecolor{currentstroke}{rgb}{0.000000,0.000000,0.000000}%
\pgfsetstrokecolor{currentstroke}%
\pgfsetdash{}{0pt}%
\pgfpathmoveto{\pgfqpoint{0.877159in}{0.449444in}}%
\pgfpathlineto{\pgfqpoint{0.940568in}{0.449444in}}%
\pgfpathlineto{\pgfqpoint{0.940568in}{1.286987in}}%
\pgfpathlineto{\pgfqpoint{0.877159in}{1.286987in}}%
\pgfpathlineto{\pgfqpoint{0.877159in}{0.449444in}}%
\pgfpathclose%
\pgfusepath{stroke}%
\end{pgfscope}%
\begin{pgfscope}%
\pgfpathrectangle{\pgfqpoint{0.465000in}{0.449444in}}{\pgfqpoint{3.487500in}{1.155000in}}%
\pgfusepath{clip}%
\pgfsetbuttcap%
\pgfsetmiterjoin%
\pgfsetlinewidth{1.003750pt}%
\definecolor{currentstroke}{rgb}{0.000000,0.000000,0.000000}%
\pgfsetstrokecolor{currentstroke}%
\pgfsetdash{}{0pt}%
\pgfpathmoveto{\pgfqpoint{1.035682in}{0.449444in}}%
\pgfpathlineto{\pgfqpoint{1.099091in}{0.449444in}}%
\pgfpathlineto{\pgfqpoint{1.099091in}{0.976501in}}%
\pgfpathlineto{\pgfqpoint{1.035682in}{0.976501in}}%
\pgfpathlineto{\pgfqpoint{1.035682in}{0.449444in}}%
\pgfpathclose%
\pgfusepath{stroke}%
\end{pgfscope}%
\begin{pgfscope}%
\pgfpathrectangle{\pgfqpoint{0.465000in}{0.449444in}}{\pgfqpoint{3.487500in}{1.155000in}}%
\pgfusepath{clip}%
\pgfsetbuttcap%
\pgfsetmiterjoin%
\pgfsetlinewidth{1.003750pt}%
\definecolor{currentstroke}{rgb}{0.000000,0.000000,0.000000}%
\pgfsetstrokecolor{currentstroke}%
\pgfsetdash{}{0pt}%
\pgfpathmoveto{\pgfqpoint{1.194205in}{0.449444in}}%
\pgfpathlineto{\pgfqpoint{1.257614in}{0.449444in}}%
\pgfpathlineto{\pgfqpoint{1.257614in}{0.790998in}}%
\pgfpathlineto{\pgfqpoint{1.194205in}{0.790998in}}%
\pgfpathlineto{\pgfqpoint{1.194205in}{0.449444in}}%
\pgfpathclose%
\pgfusepath{stroke}%
\end{pgfscope}%
\begin{pgfscope}%
\pgfpathrectangle{\pgfqpoint{0.465000in}{0.449444in}}{\pgfqpoint{3.487500in}{1.155000in}}%
\pgfusepath{clip}%
\pgfsetbuttcap%
\pgfsetmiterjoin%
\pgfsetlinewidth{1.003750pt}%
\definecolor{currentstroke}{rgb}{0.000000,0.000000,0.000000}%
\pgfsetstrokecolor{currentstroke}%
\pgfsetdash{}{0pt}%
\pgfpathmoveto{\pgfqpoint{1.352728in}{0.449444in}}%
\pgfpathlineto{\pgfqpoint{1.416137in}{0.449444in}}%
\pgfpathlineto{\pgfqpoint{1.416137in}{0.673699in}}%
\pgfpathlineto{\pgfqpoint{1.352728in}{0.673699in}}%
\pgfpathlineto{\pgfqpoint{1.352728in}{0.449444in}}%
\pgfpathclose%
\pgfusepath{stroke}%
\end{pgfscope}%
\begin{pgfscope}%
\pgfpathrectangle{\pgfqpoint{0.465000in}{0.449444in}}{\pgfqpoint{3.487500in}{1.155000in}}%
\pgfusepath{clip}%
\pgfsetbuttcap%
\pgfsetmiterjoin%
\pgfsetlinewidth{1.003750pt}%
\definecolor{currentstroke}{rgb}{0.000000,0.000000,0.000000}%
\pgfsetstrokecolor{currentstroke}%
\pgfsetdash{}{0pt}%
\pgfpathmoveto{\pgfqpoint{1.511250in}{0.449444in}}%
\pgfpathlineto{\pgfqpoint{1.574659in}{0.449444in}}%
\pgfpathlineto{\pgfqpoint{1.574659in}{0.598778in}}%
\pgfpathlineto{\pgfqpoint{1.511250in}{0.598778in}}%
\pgfpathlineto{\pgfqpoint{1.511250in}{0.449444in}}%
\pgfpathclose%
\pgfusepath{stroke}%
\end{pgfscope}%
\begin{pgfscope}%
\pgfpathrectangle{\pgfqpoint{0.465000in}{0.449444in}}{\pgfqpoint{3.487500in}{1.155000in}}%
\pgfusepath{clip}%
\pgfsetbuttcap%
\pgfsetmiterjoin%
\pgfsetlinewidth{1.003750pt}%
\definecolor{currentstroke}{rgb}{0.000000,0.000000,0.000000}%
\pgfsetstrokecolor{currentstroke}%
\pgfsetdash{}{0pt}%
\pgfpathmoveto{\pgfqpoint{1.669773in}{0.449444in}}%
\pgfpathlineto{\pgfqpoint{1.733182in}{0.449444in}}%
\pgfpathlineto{\pgfqpoint{1.733182in}{0.550100in}}%
\pgfpathlineto{\pgfqpoint{1.669773in}{0.550100in}}%
\pgfpathlineto{\pgfqpoint{1.669773in}{0.449444in}}%
\pgfpathclose%
\pgfusepath{stroke}%
\end{pgfscope}%
\begin{pgfscope}%
\pgfpathrectangle{\pgfqpoint{0.465000in}{0.449444in}}{\pgfqpoint{3.487500in}{1.155000in}}%
\pgfusepath{clip}%
\pgfsetbuttcap%
\pgfsetmiterjoin%
\pgfsetlinewidth{1.003750pt}%
\definecolor{currentstroke}{rgb}{0.000000,0.000000,0.000000}%
\pgfsetstrokecolor{currentstroke}%
\pgfsetdash{}{0pt}%
\pgfpathmoveto{\pgfqpoint{1.828296in}{0.449444in}}%
\pgfpathlineto{\pgfqpoint{1.891705in}{0.449444in}}%
\pgfpathlineto{\pgfqpoint{1.891705in}{0.517758in}}%
\pgfpathlineto{\pgfqpoint{1.828296in}{0.517758in}}%
\pgfpathlineto{\pgfqpoint{1.828296in}{0.449444in}}%
\pgfpathclose%
\pgfusepath{stroke}%
\end{pgfscope}%
\begin{pgfscope}%
\pgfpathrectangle{\pgfqpoint{0.465000in}{0.449444in}}{\pgfqpoint{3.487500in}{1.155000in}}%
\pgfusepath{clip}%
\pgfsetbuttcap%
\pgfsetmiterjoin%
\pgfsetlinewidth{1.003750pt}%
\definecolor{currentstroke}{rgb}{0.000000,0.000000,0.000000}%
\pgfsetstrokecolor{currentstroke}%
\pgfsetdash{}{0pt}%
\pgfpathmoveto{\pgfqpoint{1.986818in}{0.449444in}}%
\pgfpathlineto{\pgfqpoint{2.050228in}{0.449444in}}%
\pgfpathlineto{\pgfqpoint{2.050228in}{0.497658in}}%
\pgfpathlineto{\pgfqpoint{1.986818in}{0.497658in}}%
\pgfpathlineto{\pgfqpoint{1.986818in}{0.449444in}}%
\pgfpathclose%
\pgfusepath{stroke}%
\end{pgfscope}%
\begin{pgfscope}%
\pgfpathrectangle{\pgfqpoint{0.465000in}{0.449444in}}{\pgfqpoint{3.487500in}{1.155000in}}%
\pgfusepath{clip}%
\pgfsetbuttcap%
\pgfsetmiterjoin%
\pgfsetlinewidth{1.003750pt}%
\definecolor{currentstroke}{rgb}{0.000000,0.000000,0.000000}%
\pgfsetstrokecolor{currentstroke}%
\pgfsetdash{}{0pt}%
\pgfpathmoveto{\pgfqpoint{2.145341in}{0.449444in}}%
\pgfpathlineto{\pgfqpoint{2.208750in}{0.449444in}}%
\pgfpathlineto{\pgfqpoint{2.208750in}{0.483923in}}%
\pgfpathlineto{\pgfqpoint{2.145341in}{0.483923in}}%
\pgfpathlineto{\pgfqpoint{2.145341in}{0.449444in}}%
\pgfpathclose%
\pgfusepath{stroke}%
\end{pgfscope}%
\begin{pgfscope}%
\pgfpathrectangle{\pgfqpoint{0.465000in}{0.449444in}}{\pgfqpoint{3.487500in}{1.155000in}}%
\pgfusepath{clip}%
\pgfsetbuttcap%
\pgfsetmiterjoin%
\pgfsetlinewidth{1.003750pt}%
\definecolor{currentstroke}{rgb}{0.000000,0.000000,0.000000}%
\pgfsetstrokecolor{currentstroke}%
\pgfsetdash{}{0pt}%
\pgfpathmoveto{\pgfqpoint{2.303864in}{0.449444in}}%
\pgfpathlineto{\pgfqpoint{2.367273in}{0.449444in}}%
\pgfpathlineto{\pgfqpoint{2.367273in}{0.473974in}}%
\pgfpathlineto{\pgfqpoint{2.303864in}{0.473974in}}%
\pgfpathlineto{\pgfqpoint{2.303864in}{0.449444in}}%
\pgfpathclose%
\pgfusepath{stroke}%
\end{pgfscope}%
\begin{pgfscope}%
\pgfpathrectangle{\pgfqpoint{0.465000in}{0.449444in}}{\pgfqpoint{3.487500in}{1.155000in}}%
\pgfusepath{clip}%
\pgfsetbuttcap%
\pgfsetmiterjoin%
\pgfsetlinewidth{1.003750pt}%
\definecolor{currentstroke}{rgb}{0.000000,0.000000,0.000000}%
\pgfsetstrokecolor{currentstroke}%
\pgfsetdash{}{0pt}%
\pgfpathmoveto{\pgfqpoint{2.462387in}{0.449444in}}%
\pgfpathlineto{\pgfqpoint{2.525796in}{0.449444in}}%
\pgfpathlineto{\pgfqpoint{2.525796in}{0.466510in}}%
\pgfpathlineto{\pgfqpoint{2.462387in}{0.466510in}}%
\pgfpathlineto{\pgfqpoint{2.462387in}{0.449444in}}%
\pgfpathclose%
\pgfusepath{stroke}%
\end{pgfscope}%
\begin{pgfscope}%
\pgfpathrectangle{\pgfqpoint{0.465000in}{0.449444in}}{\pgfqpoint{3.487500in}{1.155000in}}%
\pgfusepath{clip}%
\pgfsetbuttcap%
\pgfsetmiterjoin%
\pgfsetlinewidth{1.003750pt}%
\definecolor{currentstroke}{rgb}{0.000000,0.000000,0.000000}%
\pgfsetstrokecolor{currentstroke}%
\pgfsetdash{}{0pt}%
\pgfpathmoveto{\pgfqpoint{2.620909in}{0.449444in}}%
\pgfpathlineto{\pgfqpoint{2.684318in}{0.449444in}}%
\pgfpathlineto{\pgfqpoint{2.684318in}{0.461176in}}%
\pgfpathlineto{\pgfqpoint{2.620909in}{0.461176in}}%
\pgfpathlineto{\pgfqpoint{2.620909in}{0.449444in}}%
\pgfpathclose%
\pgfusepath{stroke}%
\end{pgfscope}%
\begin{pgfscope}%
\pgfpathrectangle{\pgfqpoint{0.465000in}{0.449444in}}{\pgfqpoint{3.487500in}{1.155000in}}%
\pgfusepath{clip}%
\pgfsetbuttcap%
\pgfsetmiterjoin%
\pgfsetlinewidth{1.003750pt}%
\definecolor{currentstroke}{rgb}{0.000000,0.000000,0.000000}%
\pgfsetstrokecolor{currentstroke}%
\pgfsetdash{}{0pt}%
\pgfpathmoveto{\pgfqpoint{2.779432in}{0.449444in}}%
\pgfpathlineto{\pgfqpoint{2.842841in}{0.449444in}}%
\pgfpathlineto{\pgfqpoint{2.842841in}{0.458553in}}%
\pgfpathlineto{\pgfqpoint{2.779432in}{0.458553in}}%
\pgfpathlineto{\pgfqpoint{2.779432in}{0.449444in}}%
\pgfpathclose%
\pgfusepath{stroke}%
\end{pgfscope}%
\begin{pgfscope}%
\pgfpathrectangle{\pgfqpoint{0.465000in}{0.449444in}}{\pgfqpoint{3.487500in}{1.155000in}}%
\pgfusepath{clip}%
\pgfsetbuttcap%
\pgfsetmiterjoin%
\pgfsetlinewidth{1.003750pt}%
\definecolor{currentstroke}{rgb}{0.000000,0.000000,0.000000}%
\pgfsetstrokecolor{currentstroke}%
\pgfsetdash{}{0pt}%
\pgfpathmoveto{\pgfqpoint{2.937955in}{0.449444in}}%
\pgfpathlineto{\pgfqpoint{3.001364in}{0.449444in}}%
\pgfpathlineto{\pgfqpoint{3.001364in}{0.455102in}}%
\pgfpathlineto{\pgfqpoint{2.937955in}{0.455102in}}%
\pgfpathlineto{\pgfqpoint{2.937955in}{0.449444in}}%
\pgfpathclose%
\pgfusepath{stroke}%
\end{pgfscope}%
\begin{pgfscope}%
\pgfpathrectangle{\pgfqpoint{0.465000in}{0.449444in}}{\pgfqpoint{3.487500in}{1.155000in}}%
\pgfusepath{clip}%
\pgfsetbuttcap%
\pgfsetmiterjoin%
\pgfsetlinewidth{1.003750pt}%
\definecolor{currentstroke}{rgb}{0.000000,0.000000,0.000000}%
\pgfsetstrokecolor{currentstroke}%
\pgfsetdash{}{0pt}%
\pgfpathmoveto{\pgfqpoint{3.096478in}{0.449444in}}%
\pgfpathlineto{\pgfqpoint{3.159887in}{0.449444in}}%
\pgfpathlineto{\pgfqpoint{3.159887in}{0.453434in}}%
\pgfpathlineto{\pgfqpoint{3.096478in}{0.453434in}}%
\pgfpathlineto{\pgfqpoint{3.096478in}{0.449444in}}%
\pgfpathclose%
\pgfusepath{stroke}%
\end{pgfscope}%
\begin{pgfscope}%
\pgfpathrectangle{\pgfqpoint{0.465000in}{0.449444in}}{\pgfqpoint{3.487500in}{1.155000in}}%
\pgfusepath{clip}%
\pgfsetbuttcap%
\pgfsetmiterjoin%
\pgfsetlinewidth{1.003750pt}%
\definecolor{currentstroke}{rgb}{0.000000,0.000000,0.000000}%
\pgfsetstrokecolor{currentstroke}%
\pgfsetdash{}{0pt}%
\pgfpathmoveto{\pgfqpoint{3.255000in}{0.449444in}}%
\pgfpathlineto{\pgfqpoint{3.318409in}{0.449444in}}%
\pgfpathlineto{\pgfqpoint{3.318409in}{0.451778in}}%
\pgfpathlineto{\pgfqpoint{3.255000in}{0.451778in}}%
\pgfpathlineto{\pgfqpoint{3.255000in}{0.449444in}}%
\pgfpathclose%
\pgfusepath{stroke}%
\end{pgfscope}%
\begin{pgfscope}%
\pgfpathrectangle{\pgfqpoint{0.465000in}{0.449444in}}{\pgfqpoint{3.487500in}{1.155000in}}%
\pgfusepath{clip}%
\pgfsetbuttcap%
\pgfsetmiterjoin%
\pgfsetlinewidth{1.003750pt}%
\definecolor{currentstroke}{rgb}{0.000000,0.000000,0.000000}%
\pgfsetstrokecolor{currentstroke}%
\pgfsetdash{}{0pt}%
\pgfpathmoveto{\pgfqpoint{3.413523in}{0.449444in}}%
\pgfpathlineto{\pgfqpoint{3.476932in}{0.449444in}}%
\pgfpathlineto{\pgfqpoint{3.476932in}{0.450400in}}%
\pgfpathlineto{\pgfqpoint{3.413523in}{0.450400in}}%
\pgfpathlineto{\pgfqpoint{3.413523in}{0.449444in}}%
\pgfpathclose%
\pgfusepath{stroke}%
\end{pgfscope}%
\begin{pgfscope}%
\pgfpathrectangle{\pgfqpoint{0.465000in}{0.449444in}}{\pgfqpoint{3.487500in}{1.155000in}}%
\pgfusepath{clip}%
\pgfsetbuttcap%
\pgfsetmiterjoin%
\pgfsetlinewidth{1.003750pt}%
\definecolor{currentstroke}{rgb}{0.000000,0.000000,0.000000}%
\pgfsetstrokecolor{currentstroke}%
\pgfsetdash{}{0pt}%
\pgfpathmoveto{\pgfqpoint{3.572046in}{0.449444in}}%
\pgfpathlineto{\pgfqpoint{3.635455in}{0.449444in}}%
\pgfpathlineto{\pgfqpoint{3.635455in}{0.449618in}}%
\pgfpathlineto{\pgfqpoint{3.572046in}{0.449618in}}%
\pgfpathlineto{\pgfqpoint{3.572046in}{0.449444in}}%
\pgfpathclose%
\pgfusepath{stroke}%
\end{pgfscope}%
\begin{pgfscope}%
\pgfpathrectangle{\pgfqpoint{0.465000in}{0.449444in}}{\pgfqpoint{3.487500in}{1.155000in}}%
\pgfusepath{clip}%
\pgfsetbuttcap%
\pgfsetmiterjoin%
\pgfsetlinewidth{1.003750pt}%
\definecolor{currentstroke}{rgb}{0.000000,0.000000,0.000000}%
\pgfsetstrokecolor{currentstroke}%
\pgfsetdash{}{0pt}%
\pgfpathmoveto{\pgfqpoint{3.730568in}{0.449444in}}%
\pgfpathlineto{\pgfqpoint{3.793978in}{0.449444in}}%
\pgfpathlineto{\pgfqpoint{3.793978in}{0.449467in}}%
\pgfpathlineto{\pgfqpoint{3.730568in}{0.449467in}}%
\pgfpathlineto{\pgfqpoint{3.730568in}{0.449444in}}%
\pgfpathclose%
\pgfusepath{stroke}%
\end{pgfscope}%
\begin{pgfscope}%
\pgfpathrectangle{\pgfqpoint{0.465000in}{0.449444in}}{\pgfqpoint{3.487500in}{1.155000in}}%
\pgfusepath{clip}%
\pgfsetbuttcap%
\pgfsetmiterjoin%
\definecolor{currentfill}{rgb}{0.000000,0.000000,0.000000}%
\pgfsetfillcolor{currentfill}%
\pgfsetlinewidth{0.000000pt}%
\definecolor{currentstroke}{rgb}{0.000000,0.000000,0.000000}%
\pgfsetstrokecolor{currentstroke}%
\pgfsetstrokeopacity{0.000000}%
\pgfsetdash{}{0pt}%
\pgfpathmoveto{\pgfqpoint{0.623523in}{0.449444in}}%
\pgfpathlineto{\pgfqpoint{0.686932in}{0.449444in}}%
\pgfpathlineto{\pgfqpoint{0.686932in}{0.449444in}}%
\pgfpathlineto{\pgfqpoint{0.623523in}{0.449444in}}%
\pgfpathlineto{\pgfqpoint{0.623523in}{0.449444in}}%
\pgfpathclose%
\pgfusepath{fill}%
\end{pgfscope}%
\begin{pgfscope}%
\pgfpathrectangle{\pgfqpoint{0.465000in}{0.449444in}}{\pgfqpoint{3.487500in}{1.155000in}}%
\pgfusepath{clip}%
\pgfsetbuttcap%
\pgfsetmiterjoin%
\definecolor{currentfill}{rgb}{0.000000,0.000000,0.000000}%
\pgfsetfillcolor{currentfill}%
\pgfsetlinewidth{0.000000pt}%
\definecolor{currentstroke}{rgb}{0.000000,0.000000,0.000000}%
\pgfsetstrokecolor{currentstroke}%
\pgfsetstrokeopacity{0.000000}%
\pgfsetdash{}{0pt}%
\pgfpathmoveto{\pgfqpoint{0.782046in}{0.449444in}}%
\pgfpathlineto{\pgfqpoint{0.845455in}{0.449444in}}%
\pgfpathlineto{\pgfqpoint{0.845455in}{0.483900in}}%
\pgfpathlineto{\pgfqpoint{0.782046in}{0.483900in}}%
\pgfpathlineto{\pgfqpoint{0.782046in}{0.449444in}}%
\pgfpathclose%
\pgfusepath{fill}%
\end{pgfscope}%
\begin{pgfscope}%
\pgfpathrectangle{\pgfqpoint{0.465000in}{0.449444in}}{\pgfqpoint{3.487500in}{1.155000in}}%
\pgfusepath{clip}%
\pgfsetbuttcap%
\pgfsetmiterjoin%
\definecolor{currentfill}{rgb}{0.000000,0.000000,0.000000}%
\pgfsetfillcolor{currentfill}%
\pgfsetlinewidth{0.000000pt}%
\definecolor{currentstroke}{rgb}{0.000000,0.000000,0.000000}%
\pgfsetstrokecolor{currentstroke}%
\pgfsetstrokeopacity{0.000000}%
\pgfsetdash{}{0pt}%
\pgfpathmoveto{\pgfqpoint{0.940568in}{0.449444in}}%
\pgfpathlineto{\pgfqpoint{1.003978in}{0.449444in}}%
\pgfpathlineto{\pgfqpoint{1.003978in}{0.521129in}}%
\pgfpathlineto{\pgfqpoint{0.940568in}{0.521129in}}%
\pgfpathlineto{\pgfqpoint{0.940568in}{0.449444in}}%
\pgfpathclose%
\pgfusepath{fill}%
\end{pgfscope}%
\begin{pgfscope}%
\pgfpathrectangle{\pgfqpoint{0.465000in}{0.449444in}}{\pgfqpoint{3.487500in}{1.155000in}}%
\pgfusepath{clip}%
\pgfsetbuttcap%
\pgfsetmiterjoin%
\definecolor{currentfill}{rgb}{0.000000,0.000000,0.000000}%
\pgfsetfillcolor{currentfill}%
\pgfsetlinewidth{0.000000pt}%
\definecolor{currentstroke}{rgb}{0.000000,0.000000,0.000000}%
\pgfsetstrokecolor{currentstroke}%
\pgfsetstrokeopacity{0.000000}%
\pgfsetdash{}{0pt}%
\pgfpathmoveto{\pgfqpoint{1.099091in}{0.449444in}}%
\pgfpathlineto{\pgfqpoint{1.162500in}{0.449444in}}%
\pgfpathlineto{\pgfqpoint{1.162500in}{0.526665in}}%
\pgfpathlineto{\pgfqpoint{1.099091in}{0.526665in}}%
\pgfpathlineto{\pgfqpoint{1.099091in}{0.449444in}}%
\pgfpathclose%
\pgfusepath{fill}%
\end{pgfscope}%
\begin{pgfscope}%
\pgfpathrectangle{\pgfqpoint{0.465000in}{0.449444in}}{\pgfqpoint{3.487500in}{1.155000in}}%
\pgfusepath{clip}%
\pgfsetbuttcap%
\pgfsetmiterjoin%
\definecolor{currentfill}{rgb}{0.000000,0.000000,0.000000}%
\pgfsetfillcolor{currentfill}%
\pgfsetlinewidth{0.000000pt}%
\definecolor{currentstroke}{rgb}{0.000000,0.000000,0.000000}%
\pgfsetstrokecolor{currentstroke}%
\pgfsetstrokeopacity{0.000000}%
\pgfsetdash{}{0pt}%
\pgfpathmoveto{\pgfqpoint{1.257614in}{0.449444in}}%
\pgfpathlineto{\pgfqpoint{1.321023in}{0.449444in}}%
\pgfpathlineto{\pgfqpoint{1.321023in}{0.522125in}}%
\pgfpathlineto{\pgfqpoint{1.257614in}{0.522125in}}%
\pgfpathlineto{\pgfqpoint{1.257614in}{0.449444in}}%
\pgfpathclose%
\pgfusepath{fill}%
\end{pgfscope}%
\begin{pgfscope}%
\pgfpathrectangle{\pgfqpoint{0.465000in}{0.449444in}}{\pgfqpoint{3.487500in}{1.155000in}}%
\pgfusepath{clip}%
\pgfsetbuttcap%
\pgfsetmiterjoin%
\definecolor{currentfill}{rgb}{0.000000,0.000000,0.000000}%
\pgfsetfillcolor{currentfill}%
\pgfsetlinewidth{0.000000pt}%
\definecolor{currentstroke}{rgb}{0.000000,0.000000,0.000000}%
\pgfsetstrokecolor{currentstroke}%
\pgfsetstrokeopacity{0.000000}%
\pgfsetdash{}{0pt}%
\pgfpathmoveto{\pgfqpoint{1.416137in}{0.449444in}}%
\pgfpathlineto{\pgfqpoint{1.479546in}{0.449444in}}%
\pgfpathlineto{\pgfqpoint{1.479546in}{0.513230in}}%
\pgfpathlineto{\pgfqpoint{1.416137in}{0.513230in}}%
\pgfpathlineto{\pgfqpoint{1.416137in}{0.449444in}}%
\pgfpathclose%
\pgfusepath{fill}%
\end{pgfscope}%
\begin{pgfscope}%
\pgfpathrectangle{\pgfqpoint{0.465000in}{0.449444in}}{\pgfqpoint{3.487500in}{1.155000in}}%
\pgfusepath{clip}%
\pgfsetbuttcap%
\pgfsetmiterjoin%
\definecolor{currentfill}{rgb}{0.000000,0.000000,0.000000}%
\pgfsetfillcolor{currentfill}%
\pgfsetlinewidth{0.000000pt}%
\definecolor{currentstroke}{rgb}{0.000000,0.000000,0.000000}%
\pgfsetstrokecolor{currentstroke}%
\pgfsetstrokeopacity{0.000000}%
\pgfsetdash{}{0pt}%
\pgfpathmoveto{\pgfqpoint{1.574659in}{0.449444in}}%
\pgfpathlineto{\pgfqpoint{1.638068in}{0.449444in}}%
\pgfpathlineto{\pgfqpoint{1.638068in}{0.503200in}}%
\pgfpathlineto{\pgfqpoint{1.574659in}{0.503200in}}%
\pgfpathlineto{\pgfqpoint{1.574659in}{0.449444in}}%
\pgfpathclose%
\pgfusepath{fill}%
\end{pgfscope}%
\begin{pgfscope}%
\pgfpathrectangle{\pgfqpoint{0.465000in}{0.449444in}}{\pgfqpoint{3.487500in}{1.155000in}}%
\pgfusepath{clip}%
\pgfsetbuttcap%
\pgfsetmiterjoin%
\definecolor{currentfill}{rgb}{0.000000,0.000000,0.000000}%
\pgfsetfillcolor{currentfill}%
\pgfsetlinewidth{0.000000pt}%
\definecolor{currentstroke}{rgb}{0.000000,0.000000,0.000000}%
\pgfsetstrokecolor{currentstroke}%
\pgfsetstrokeopacity{0.000000}%
\pgfsetdash{}{0pt}%
\pgfpathmoveto{\pgfqpoint{1.733182in}{0.449444in}}%
\pgfpathlineto{\pgfqpoint{1.796591in}{0.449444in}}%
\pgfpathlineto{\pgfqpoint{1.796591in}{0.493698in}}%
\pgfpathlineto{\pgfqpoint{1.733182in}{0.493698in}}%
\pgfpathlineto{\pgfqpoint{1.733182in}{0.449444in}}%
\pgfpathclose%
\pgfusepath{fill}%
\end{pgfscope}%
\begin{pgfscope}%
\pgfpathrectangle{\pgfqpoint{0.465000in}{0.449444in}}{\pgfqpoint{3.487500in}{1.155000in}}%
\pgfusepath{clip}%
\pgfsetbuttcap%
\pgfsetmiterjoin%
\definecolor{currentfill}{rgb}{0.000000,0.000000,0.000000}%
\pgfsetfillcolor{currentfill}%
\pgfsetlinewidth{0.000000pt}%
\definecolor{currentstroke}{rgb}{0.000000,0.000000,0.000000}%
\pgfsetstrokecolor{currentstroke}%
\pgfsetstrokeopacity{0.000000}%
\pgfsetdash{}{0pt}%
\pgfpathmoveto{\pgfqpoint{1.891705in}{0.449444in}}%
\pgfpathlineto{\pgfqpoint{1.955114in}{0.449444in}}%
\pgfpathlineto{\pgfqpoint{1.955114in}{0.487287in}}%
\pgfpathlineto{\pgfqpoint{1.891705in}{0.487287in}}%
\pgfpathlineto{\pgfqpoint{1.891705in}{0.449444in}}%
\pgfpathclose%
\pgfusepath{fill}%
\end{pgfscope}%
\begin{pgfscope}%
\pgfpathrectangle{\pgfqpoint{0.465000in}{0.449444in}}{\pgfqpoint{3.487500in}{1.155000in}}%
\pgfusepath{clip}%
\pgfsetbuttcap%
\pgfsetmiterjoin%
\definecolor{currentfill}{rgb}{0.000000,0.000000,0.000000}%
\pgfsetfillcolor{currentfill}%
\pgfsetlinewidth{0.000000pt}%
\definecolor{currentstroke}{rgb}{0.000000,0.000000,0.000000}%
\pgfsetstrokecolor{currentstroke}%
\pgfsetstrokeopacity{0.000000}%
\pgfsetdash{}{0pt}%
\pgfpathmoveto{\pgfqpoint{2.050228in}{0.449444in}}%
\pgfpathlineto{\pgfqpoint{2.113637in}{0.449444in}}%
\pgfpathlineto{\pgfqpoint{2.113637in}{0.481543in}}%
\pgfpathlineto{\pgfqpoint{2.050228in}{0.481543in}}%
\pgfpathlineto{\pgfqpoint{2.050228in}{0.449444in}}%
\pgfpathclose%
\pgfusepath{fill}%
\end{pgfscope}%
\begin{pgfscope}%
\pgfpathrectangle{\pgfqpoint{0.465000in}{0.449444in}}{\pgfqpoint{3.487500in}{1.155000in}}%
\pgfusepath{clip}%
\pgfsetbuttcap%
\pgfsetmiterjoin%
\definecolor{currentfill}{rgb}{0.000000,0.000000,0.000000}%
\pgfsetfillcolor{currentfill}%
\pgfsetlinewidth{0.000000pt}%
\definecolor{currentstroke}{rgb}{0.000000,0.000000,0.000000}%
\pgfsetstrokecolor{currentstroke}%
\pgfsetstrokeopacity{0.000000}%
\pgfsetdash{}{0pt}%
\pgfpathmoveto{\pgfqpoint{2.208750in}{0.449444in}}%
\pgfpathlineto{\pgfqpoint{2.272159in}{0.449444in}}%
\pgfpathlineto{\pgfqpoint{2.272159in}{0.477188in}}%
\pgfpathlineto{\pgfqpoint{2.208750in}{0.477188in}}%
\pgfpathlineto{\pgfqpoint{2.208750in}{0.449444in}}%
\pgfpathclose%
\pgfusepath{fill}%
\end{pgfscope}%
\begin{pgfscope}%
\pgfpathrectangle{\pgfqpoint{0.465000in}{0.449444in}}{\pgfqpoint{3.487500in}{1.155000in}}%
\pgfusepath{clip}%
\pgfsetbuttcap%
\pgfsetmiterjoin%
\definecolor{currentfill}{rgb}{0.000000,0.000000,0.000000}%
\pgfsetfillcolor{currentfill}%
\pgfsetlinewidth{0.000000pt}%
\definecolor{currentstroke}{rgb}{0.000000,0.000000,0.000000}%
\pgfsetstrokecolor{currentstroke}%
\pgfsetstrokeopacity{0.000000}%
\pgfsetdash{}{0pt}%
\pgfpathmoveto{\pgfqpoint{2.367273in}{0.449444in}}%
\pgfpathlineto{\pgfqpoint{2.430682in}{0.449444in}}%
\pgfpathlineto{\pgfqpoint{2.430682in}{0.472851in}}%
\pgfpathlineto{\pgfqpoint{2.367273in}{0.472851in}}%
\pgfpathlineto{\pgfqpoint{2.367273in}{0.449444in}}%
\pgfpathclose%
\pgfusepath{fill}%
\end{pgfscope}%
\begin{pgfscope}%
\pgfpathrectangle{\pgfqpoint{0.465000in}{0.449444in}}{\pgfqpoint{3.487500in}{1.155000in}}%
\pgfusepath{clip}%
\pgfsetbuttcap%
\pgfsetmiterjoin%
\definecolor{currentfill}{rgb}{0.000000,0.000000,0.000000}%
\pgfsetfillcolor{currentfill}%
\pgfsetlinewidth{0.000000pt}%
\definecolor{currentstroke}{rgb}{0.000000,0.000000,0.000000}%
\pgfsetstrokecolor{currentstroke}%
\pgfsetstrokeopacity{0.000000}%
\pgfsetdash{}{0pt}%
\pgfpathmoveto{\pgfqpoint{2.525796in}{0.449444in}}%
\pgfpathlineto{\pgfqpoint{2.589205in}{0.449444in}}%
\pgfpathlineto{\pgfqpoint{2.589205in}{0.470141in}}%
\pgfpathlineto{\pgfqpoint{2.525796in}{0.470141in}}%
\pgfpathlineto{\pgfqpoint{2.525796in}{0.449444in}}%
\pgfpathclose%
\pgfusepath{fill}%
\end{pgfscope}%
\begin{pgfscope}%
\pgfpathrectangle{\pgfqpoint{0.465000in}{0.449444in}}{\pgfqpoint{3.487500in}{1.155000in}}%
\pgfusepath{clip}%
\pgfsetbuttcap%
\pgfsetmiterjoin%
\definecolor{currentfill}{rgb}{0.000000,0.000000,0.000000}%
\pgfsetfillcolor{currentfill}%
\pgfsetlinewidth{0.000000pt}%
\definecolor{currentstroke}{rgb}{0.000000,0.000000,0.000000}%
\pgfsetstrokecolor{currentstroke}%
\pgfsetstrokeopacity{0.000000}%
\pgfsetdash{}{0pt}%
\pgfpathmoveto{\pgfqpoint{2.684318in}{0.449444in}}%
\pgfpathlineto{\pgfqpoint{2.747728in}{0.449444in}}%
\pgfpathlineto{\pgfqpoint{2.747728in}{0.466405in}}%
\pgfpathlineto{\pgfqpoint{2.684318in}{0.466405in}}%
\pgfpathlineto{\pgfqpoint{2.684318in}{0.449444in}}%
\pgfpathclose%
\pgfusepath{fill}%
\end{pgfscope}%
\begin{pgfscope}%
\pgfpathrectangle{\pgfqpoint{0.465000in}{0.449444in}}{\pgfqpoint{3.487500in}{1.155000in}}%
\pgfusepath{clip}%
\pgfsetbuttcap%
\pgfsetmiterjoin%
\definecolor{currentfill}{rgb}{0.000000,0.000000,0.000000}%
\pgfsetfillcolor{currentfill}%
\pgfsetlinewidth{0.000000pt}%
\definecolor{currentstroke}{rgb}{0.000000,0.000000,0.000000}%
\pgfsetstrokecolor{currentstroke}%
\pgfsetstrokeopacity{0.000000}%
\pgfsetdash{}{0pt}%
\pgfpathmoveto{\pgfqpoint{2.842841in}{0.449444in}}%
\pgfpathlineto{\pgfqpoint{2.906250in}{0.449444in}}%
\pgfpathlineto{\pgfqpoint{2.906250in}{0.464008in}}%
\pgfpathlineto{\pgfqpoint{2.842841in}{0.464008in}}%
\pgfpathlineto{\pgfqpoint{2.842841in}{0.449444in}}%
\pgfpathclose%
\pgfusepath{fill}%
\end{pgfscope}%
\begin{pgfscope}%
\pgfpathrectangle{\pgfqpoint{0.465000in}{0.449444in}}{\pgfqpoint{3.487500in}{1.155000in}}%
\pgfusepath{clip}%
\pgfsetbuttcap%
\pgfsetmiterjoin%
\definecolor{currentfill}{rgb}{0.000000,0.000000,0.000000}%
\pgfsetfillcolor{currentfill}%
\pgfsetlinewidth{0.000000pt}%
\definecolor{currentstroke}{rgb}{0.000000,0.000000,0.000000}%
\pgfsetstrokecolor{currentstroke}%
\pgfsetstrokeopacity{0.000000}%
\pgfsetdash{}{0pt}%
\pgfpathmoveto{\pgfqpoint{3.001364in}{0.449444in}}%
\pgfpathlineto{\pgfqpoint{3.064773in}{0.449444in}}%
\pgfpathlineto{\pgfqpoint{3.064773in}{0.462178in}}%
\pgfpathlineto{\pgfqpoint{3.001364in}{0.462178in}}%
\pgfpathlineto{\pgfqpoint{3.001364in}{0.449444in}}%
\pgfpathclose%
\pgfusepath{fill}%
\end{pgfscope}%
\begin{pgfscope}%
\pgfpathrectangle{\pgfqpoint{0.465000in}{0.449444in}}{\pgfqpoint{3.487500in}{1.155000in}}%
\pgfusepath{clip}%
\pgfsetbuttcap%
\pgfsetmiterjoin%
\definecolor{currentfill}{rgb}{0.000000,0.000000,0.000000}%
\pgfsetfillcolor{currentfill}%
\pgfsetlinewidth{0.000000pt}%
\definecolor{currentstroke}{rgb}{0.000000,0.000000,0.000000}%
\pgfsetstrokecolor{currentstroke}%
\pgfsetstrokeopacity{0.000000}%
\pgfsetdash{}{0pt}%
\pgfpathmoveto{\pgfqpoint{3.159887in}{0.449444in}}%
\pgfpathlineto{\pgfqpoint{3.223296in}{0.449444in}}%
\pgfpathlineto{\pgfqpoint{3.223296in}{0.459433in}}%
\pgfpathlineto{\pgfqpoint{3.159887in}{0.459433in}}%
\pgfpathlineto{\pgfqpoint{3.159887in}{0.449444in}}%
\pgfpathclose%
\pgfusepath{fill}%
\end{pgfscope}%
\begin{pgfscope}%
\pgfpathrectangle{\pgfqpoint{0.465000in}{0.449444in}}{\pgfqpoint{3.487500in}{1.155000in}}%
\pgfusepath{clip}%
\pgfsetbuttcap%
\pgfsetmiterjoin%
\definecolor{currentfill}{rgb}{0.000000,0.000000,0.000000}%
\pgfsetfillcolor{currentfill}%
\pgfsetlinewidth{0.000000pt}%
\definecolor{currentstroke}{rgb}{0.000000,0.000000,0.000000}%
\pgfsetstrokecolor{currentstroke}%
\pgfsetstrokeopacity{0.000000}%
\pgfsetdash{}{0pt}%
\pgfpathmoveto{\pgfqpoint{3.318409in}{0.449444in}}%
\pgfpathlineto{\pgfqpoint{3.381818in}{0.449444in}}%
\pgfpathlineto{\pgfqpoint{3.381818in}{0.456219in}}%
\pgfpathlineto{\pgfqpoint{3.318409in}{0.456219in}}%
\pgfpathlineto{\pgfqpoint{3.318409in}{0.449444in}}%
\pgfpathclose%
\pgfusepath{fill}%
\end{pgfscope}%
\begin{pgfscope}%
\pgfpathrectangle{\pgfqpoint{0.465000in}{0.449444in}}{\pgfqpoint{3.487500in}{1.155000in}}%
\pgfusepath{clip}%
\pgfsetbuttcap%
\pgfsetmiterjoin%
\definecolor{currentfill}{rgb}{0.000000,0.000000,0.000000}%
\pgfsetfillcolor{currentfill}%
\pgfsetlinewidth{0.000000pt}%
\definecolor{currentstroke}{rgb}{0.000000,0.000000,0.000000}%
\pgfsetstrokecolor{currentstroke}%
\pgfsetstrokeopacity{0.000000}%
\pgfsetdash{}{0pt}%
\pgfpathmoveto{\pgfqpoint{3.476932in}{0.449444in}}%
\pgfpathlineto{\pgfqpoint{3.540341in}{0.449444in}}%
\pgfpathlineto{\pgfqpoint{3.540341in}{0.452757in}}%
\pgfpathlineto{\pgfqpoint{3.476932in}{0.452757in}}%
\pgfpathlineto{\pgfqpoint{3.476932in}{0.449444in}}%
\pgfpathclose%
\pgfusepath{fill}%
\end{pgfscope}%
\begin{pgfscope}%
\pgfpathrectangle{\pgfqpoint{0.465000in}{0.449444in}}{\pgfqpoint{3.487500in}{1.155000in}}%
\pgfusepath{clip}%
\pgfsetbuttcap%
\pgfsetmiterjoin%
\definecolor{currentfill}{rgb}{0.000000,0.000000,0.000000}%
\pgfsetfillcolor{currentfill}%
\pgfsetlinewidth{0.000000pt}%
\definecolor{currentstroke}{rgb}{0.000000,0.000000,0.000000}%
\pgfsetstrokecolor{currentstroke}%
\pgfsetstrokeopacity{0.000000}%
\pgfsetdash{}{0pt}%
\pgfpathmoveto{\pgfqpoint{3.635455in}{0.449444in}}%
\pgfpathlineto{\pgfqpoint{3.698864in}{0.449444in}}%
\pgfpathlineto{\pgfqpoint{3.698864in}{0.450330in}}%
\pgfpathlineto{\pgfqpoint{3.635455in}{0.450330in}}%
\pgfpathlineto{\pgfqpoint{3.635455in}{0.449444in}}%
\pgfpathclose%
\pgfusepath{fill}%
\end{pgfscope}%
\begin{pgfscope}%
\pgfpathrectangle{\pgfqpoint{0.465000in}{0.449444in}}{\pgfqpoint{3.487500in}{1.155000in}}%
\pgfusepath{clip}%
\pgfsetbuttcap%
\pgfsetmiterjoin%
\definecolor{currentfill}{rgb}{0.000000,0.000000,0.000000}%
\pgfsetfillcolor{currentfill}%
\pgfsetlinewidth{0.000000pt}%
\definecolor{currentstroke}{rgb}{0.000000,0.000000,0.000000}%
\pgfsetstrokecolor{currentstroke}%
\pgfsetstrokeopacity{0.000000}%
\pgfsetdash{}{0pt}%
\pgfpathmoveto{\pgfqpoint{3.793978in}{0.449444in}}%
\pgfpathlineto{\pgfqpoint{3.857387in}{0.449444in}}%
\pgfpathlineto{\pgfqpoint{3.857387in}{0.449751in}}%
\pgfpathlineto{\pgfqpoint{3.793978in}{0.449751in}}%
\pgfpathlineto{\pgfqpoint{3.793978in}{0.449444in}}%
\pgfpathclose%
\pgfusepath{fill}%
\end{pgfscope}%
\begin{pgfscope}%
\pgfsetbuttcap%
\pgfsetroundjoin%
\definecolor{currentfill}{rgb}{0.000000,0.000000,0.000000}%
\pgfsetfillcolor{currentfill}%
\pgfsetlinewidth{0.803000pt}%
\definecolor{currentstroke}{rgb}{0.000000,0.000000,0.000000}%
\pgfsetstrokecolor{currentstroke}%
\pgfsetdash{}{0pt}%
\pgfsys@defobject{currentmarker}{\pgfqpoint{0.000000in}{-0.048611in}}{\pgfqpoint{0.000000in}{0.000000in}}{%
\pgfpathmoveto{\pgfqpoint{0.000000in}{0.000000in}}%
\pgfpathlineto{\pgfqpoint{0.000000in}{-0.048611in}}%
\pgfusepath{stroke,fill}%
}%
\begin{pgfscope}%
\pgfsys@transformshift{0.465000in}{0.449444in}%
\pgfsys@useobject{currentmarker}{}%
\end{pgfscope}%
\end{pgfscope}%
\begin{pgfscope}%
\pgfsetbuttcap%
\pgfsetroundjoin%
\definecolor{currentfill}{rgb}{0.000000,0.000000,0.000000}%
\pgfsetfillcolor{currentfill}%
\pgfsetlinewidth{0.803000pt}%
\definecolor{currentstroke}{rgb}{0.000000,0.000000,0.000000}%
\pgfsetstrokecolor{currentstroke}%
\pgfsetdash{}{0pt}%
\pgfsys@defobject{currentmarker}{\pgfqpoint{0.000000in}{-0.048611in}}{\pgfqpoint{0.000000in}{0.000000in}}{%
\pgfpathmoveto{\pgfqpoint{0.000000in}{0.000000in}}%
\pgfpathlineto{\pgfqpoint{0.000000in}{-0.048611in}}%
\pgfusepath{stroke,fill}%
}%
\begin{pgfscope}%
\pgfsys@transformshift{0.623523in}{0.449444in}%
\pgfsys@useobject{currentmarker}{}%
\end{pgfscope}%
\end{pgfscope}%
\begin{pgfscope}%
\definecolor{textcolor}{rgb}{0.000000,0.000000,0.000000}%
\pgfsetstrokecolor{textcolor}%
\pgfsetfillcolor{textcolor}%
\pgftext[x=0.623523in,y=0.352222in,,top]{\color{textcolor}\rmfamily\fontsize{10.000000}{12.000000}\selectfont 0.0}%
\end{pgfscope}%
\begin{pgfscope}%
\pgfsetbuttcap%
\pgfsetroundjoin%
\definecolor{currentfill}{rgb}{0.000000,0.000000,0.000000}%
\pgfsetfillcolor{currentfill}%
\pgfsetlinewidth{0.803000pt}%
\definecolor{currentstroke}{rgb}{0.000000,0.000000,0.000000}%
\pgfsetstrokecolor{currentstroke}%
\pgfsetdash{}{0pt}%
\pgfsys@defobject{currentmarker}{\pgfqpoint{0.000000in}{-0.048611in}}{\pgfqpoint{0.000000in}{0.000000in}}{%
\pgfpathmoveto{\pgfqpoint{0.000000in}{0.000000in}}%
\pgfpathlineto{\pgfqpoint{0.000000in}{-0.048611in}}%
\pgfusepath{stroke,fill}%
}%
\begin{pgfscope}%
\pgfsys@transformshift{0.782046in}{0.449444in}%
\pgfsys@useobject{currentmarker}{}%
\end{pgfscope}%
\end{pgfscope}%
\begin{pgfscope}%
\pgfsetbuttcap%
\pgfsetroundjoin%
\definecolor{currentfill}{rgb}{0.000000,0.000000,0.000000}%
\pgfsetfillcolor{currentfill}%
\pgfsetlinewidth{0.803000pt}%
\definecolor{currentstroke}{rgb}{0.000000,0.000000,0.000000}%
\pgfsetstrokecolor{currentstroke}%
\pgfsetdash{}{0pt}%
\pgfsys@defobject{currentmarker}{\pgfqpoint{0.000000in}{-0.048611in}}{\pgfqpoint{0.000000in}{0.000000in}}{%
\pgfpathmoveto{\pgfqpoint{0.000000in}{0.000000in}}%
\pgfpathlineto{\pgfqpoint{0.000000in}{-0.048611in}}%
\pgfusepath{stroke,fill}%
}%
\begin{pgfscope}%
\pgfsys@transformshift{0.940568in}{0.449444in}%
\pgfsys@useobject{currentmarker}{}%
\end{pgfscope}%
\end{pgfscope}%
\begin{pgfscope}%
\definecolor{textcolor}{rgb}{0.000000,0.000000,0.000000}%
\pgfsetstrokecolor{textcolor}%
\pgfsetfillcolor{textcolor}%
\pgftext[x=0.940568in,y=0.352222in,,top]{\color{textcolor}\rmfamily\fontsize{10.000000}{12.000000}\selectfont 0.1}%
\end{pgfscope}%
\begin{pgfscope}%
\pgfsetbuttcap%
\pgfsetroundjoin%
\definecolor{currentfill}{rgb}{0.000000,0.000000,0.000000}%
\pgfsetfillcolor{currentfill}%
\pgfsetlinewidth{0.803000pt}%
\definecolor{currentstroke}{rgb}{0.000000,0.000000,0.000000}%
\pgfsetstrokecolor{currentstroke}%
\pgfsetdash{}{0pt}%
\pgfsys@defobject{currentmarker}{\pgfqpoint{0.000000in}{-0.048611in}}{\pgfqpoint{0.000000in}{0.000000in}}{%
\pgfpathmoveto{\pgfqpoint{0.000000in}{0.000000in}}%
\pgfpathlineto{\pgfqpoint{0.000000in}{-0.048611in}}%
\pgfusepath{stroke,fill}%
}%
\begin{pgfscope}%
\pgfsys@transformshift{1.099091in}{0.449444in}%
\pgfsys@useobject{currentmarker}{}%
\end{pgfscope}%
\end{pgfscope}%
\begin{pgfscope}%
\pgfsetbuttcap%
\pgfsetroundjoin%
\definecolor{currentfill}{rgb}{0.000000,0.000000,0.000000}%
\pgfsetfillcolor{currentfill}%
\pgfsetlinewidth{0.803000pt}%
\definecolor{currentstroke}{rgb}{0.000000,0.000000,0.000000}%
\pgfsetstrokecolor{currentstroke}%
\pgfsetdash{}{0pt}%
\pgfsys@defobject{currentmarker}{\pgfqpoint{0.000000in}{-0.048611in}}{\pgfqpoint{0.000000in}{0.000000in}}{%
\pgfpathmoveto{\pgfqpoint{0.000000in}{0.000000in}}%
\pgfpathlineto{\pgfqpoint{0.000000in}{-0.048611in}}%
\pgfusepath{stroke,fill}%
}%
\begin{pgfscope}%
\pgfsys@transformshift{1.257614in}{0.449444in}%
\pgfsys@useobject{currentmarker}{}%
\end{pgfscope}%
\end{pgfscope}%
\begin{pgfscope}%
\definecolor{textcolor}{rgb}{0.000000,0.000000,0.000000}%
\pgfsetstrokecolor{textcolor}%
\pgfsetfillcolor{textcolor}%
\pgftext[x=1.257614in,y=0.352222in,,top]{\color{textcolor}\rmfamily\fontsize{10.000000}{12.000000}\selectfont 0.2}%
\end{pgfscope}%
\begin{pgfscope}%
\pgfsetbuttcap%
\pgfsetroundjoin%
\definecolor{currentfill}{rgb}{0.000000,0.000000,0.000000}%
\pgfsetfillcolor{currentfill}%
\pgfsetlinewidth{0.803000pt}%
\definecolor{currentstroke}{rgb}{0.000000,0.000000,0.000000}%
\pgfsetstrokecolor{currentstroke}%
\pgfsetdash{}{0pt}%
\pgfsys@defobject{currentmarker}{\pgfqpoint{0.000000in}{-0.048611in}}{\pgfqpoint{0.000000in}{0.000000in}}{%
\pgfpathmoveto{\pgfqpoint{0.000000in}{0.000000in}}%
\pgfpathlineto{\pgfqpoint{0.000000in}{-0.048611in}}%
\pgfusepath{stroke,fill}%
}%
\begin{pgfscope}%
\pgfsys@transformshift{1.416137in}{0.449444in}%
\pgfsys@useobject{currentmarker}{}%
\end{pgfscope}%
\end{pgfscope}%
\begin{pgfscope}%
\pgfsetbuttcap%
\pgfsetroundjoin%
\definecolor{currentfill}{rgb}{0.000000,0.000000,0.000000}%
\pgfsetfillcolor{currentfill}%
\pgfsetlinewidth{0.803000pt}%
\definecolor{currentstroke}{rgb}{0.000000,0.000000,0.000000}%
\pgfsetstrokecolor{currentstroke}%
\pgfsetdash{}{0pt}%
\pgfsys@defobject{currentmarker}{\pgfqpoint{0.000000in}{-0.048611in}}{\pgfqpoint{0.000000in}{0.000000in}}{%
\pgfpathmoveto{\pgfqpoint{0.000000in}{0.000000in}}%
\pgfpathlineto{\pgfqpoint{0.000000in}{-0.048611in}}%
\pgfusepath{stroke,fill}%
}%
\begin{pgfscope}%
\pgfsys@transformshift{1.574659in}{0.449444in}%
\pgfsys@useobject{currentmarker}{}%
\end{pgfscope}%
\end{pgfscope}%
\begin{pgfscope}%
\definecolor{textcolor}{rgb}{0.000000,0.000000,0.000000}%
\pgfsetstrokecolor{textcolor}%
\pgfsetfillcolor{textcolor}%
\pgftext[x=1.574659in,y=0.352222in,,top]{\color{textcolor}\rmfamily\fontsize{10.000000}{12.000000}\selectfont 0.3}%
\end{pgfscope}%
\begin{pgfscope}%
\pgfsetbuttcap%
\pgfsetroundjoin%
\definecolor{currentfill}{rgb}{0.000000,0.000000,0.000000}%
\pgfsetfillcolor{currentfill}%
\pgfsetlinewidth{0.803000pt}%
\definecolor{currentstroke}{rgb}{0.000000,0.000000,0.000000}%
\pgfsetstrokecolor{currentstroke}%
\pgfsetdash{}{0pt}%
\pgfsys@defobject{currentmarker}{\pgfqpoint{0.000000in}{-0.048611in}}{\pgfqpoint{0.000000in}{0.000000in}}{%
\pgfpathmoveto{\pgfqpoint{0.000000in}{0.000000in}}%
\pgfpathlineto{\pgfqpoint{0.000000in}{-0.048611in}}%
\pgfusepath{stroke,fill}%
}%
\begin{pgfscope}%
\pgfsys@transformshift{1.733182in}{0.449444in}%
\pgfsys@useobject{currentmarker}{}%
\end{pgfscope}%
\end{pgfscope}%
\begin{pgfscope}%
\pgfsetbuttcap%
\pgfsetroundjoin%
\definecolor{currentfill}{rgb}{0.000000,0.000000,0.000000}%
\pgfsetfillcolor{currentfill}%
\pgfsetlinewidth{0.803000pt}%
\definecolor{currentstroke}{rgb}{0.000000,0.000000,0.000000}%
\pgfsetstrokecolor{currentstroke}%
\pgfsetdash{}{0pt}%
\pgfsys@defobject{currentmarker}{\pgfqpoint{0.000000in}{-0.048611in}}{\pgfqpoint{0.000000in}{0.000000in}}{%
\pgfpathmoveto{\pgfqpoint{0.000000in}{0.000000in}}%
\pgfpathlineto{\pgfqpoint{0.000000in}{-0.048611in}}%
\pgfusepath{stroke,fill}%
}%
\begin{pgfscope}%
\pgfsys@transformshift{1.891705in}{0.449444in}%
\pgfsys@useobject{currentmarker}{}%
\end{pgfscope}%
\end{pgfscope}%
\begin{pgfscope}%
\definecolor{textcolor}{rgb}{0.000000,0.000000,0.000000}%
\pgfsetstrokecolor{textcolor}%
\pgfsetfillcolor{textcolor}%
\pgftext[x=1.891705in,y=0.352222in,,top]{\color{textcolor}\rmfamily\fontsize{10.000000}{12.000000}\selectfont 0.4}%
\end{pgfscope}%
\begin{pgfscope}%
\pgfsetbuttcap%
\pgfsetroundjoin%
\definecolor{currentfill}{rgb}{0.000000,0.000000,0.000000}%
\pgfsetfillcolor{currentfill}%
\pgfsetlinewidth{0.803000pt}%
\definecolor{currentstroke}{rgb}{0.000000,0.000000,0.000000}%
\pgfsetstrokecolor{currentstroke}%
\pgfsetdash{}{0pt}%
\pgfsys@defobject{currentmarker}{\pgfqpoint{0.000000in}{-0.048611in}}{\pgfqpoint{0.000000in}{0.000000in}}{%
\pgfpathmoveto{\pgfqpoint{0.000000in}{0.000000in}}%
\pgfpathlineto{\pgfqpoint{0.000000in}{-0.048611in}}%
\pgfusepath{stroke,fill}%
}%
\begin{pgfscope}%
\pgfsys@transformshift{2.050228in}{0.449444in}%
\pgfsys@useobject{currentmarker}{}%
\end{pgfscope}%
\end{pgfscope}%
\begin{pgfscope}%
\pgfsetbuttcap%
\pgfsetroundjoin%
\definecolor{currentfill}{rgb}{0.000000,0.000000,0.000000}%
\pgfsetfillcolor{currentfill}%
\pgfsetlinewidth{0.803000pt}%
\definecolor{currentstroke}{rgb}{0.000000,0.000000,0.000000}%
\pgfsetstrokecolor{currentstroke}%
\pgfsetdash{}{0pt}%
\pgfsys@defobject{currentmarker}{\pgfqpoint{0.000000in}{-0.048611in}}{\pgfqpoint{0.000000in}{0.000000in}}{%
\pgfpathmoveto{\pgfqpoint{0.000000in}{0.000000in}}%
\pgfpathlineto{\pgfqpoint{0.000000in}{-0.048611in}}%
\pgfusepath{stroke,fill}%
}%
\begin{pgfscope}%
\pgfsys@transformshift{2.208750in}{0.449444in}%
\pgfsys@useobject{currentmarker}{}%
\end{pgfscope}%
\end{pgfscope}%
\begin{pgfscope}%
\definecolor{textcolor}{rgb}{0.000000,0.000000,0.000000}%
\pgfsetstrokecolor{textcolor}%
\pgfsetfillcolor{textcolor}%
\pgftext[x=2.208750in,y=0.352222in,,top]{\color{textcolor}\rmfamily\fontsize{10.000000}{12.000000}\selectfont 0.5}%
\end{pgfscope}%
\begin{pgfscope}%
\pgfsetbuttcap%
\pgfsetroundjoin%
\definecolor{currentfill}{rgb}{0.000000,0.000000,0.000000}%
\pgfsetfillcolor{currentfill}%
\pgfsetlinewidth{0.803000pt}%
\definecolor{currentstroke}{rgb}{0.000000,0.000000,0.000000}%
\pgfsetstrokecolor{currentstroke}%
\pgfsetdash{}{0pt}%
\pgfsys@defobject{currentmarker}{\pgfqpoint{0.000000in}{-0.048611in}}{\pgfqpoint{0.000000in}{0.000000in}}{%
\pgfpathmoveto{\pgfqpoint{0.000000in}{0.000000in}}%
\pgfpathlineto{\pgfqpoint{0.000000in}{-0.048611in}}%
\pgfusepath{stroke,fill}%
}%
\begin{pgfscope}%
\pgfsys@transformshift{2.367273in}{0.449444in}%
\pgfsys@useobject{currentmarker}{}%
\end{pgfscope}%
\end{pgfscope}%
\begin{pgfscope}%
\pgfsetbuttcap%
\pgfsetroundjoin%
\definecolor{currentfill}{rgb}{0.000000,0.000000,0.000000}%
\pgfsetfillcolor{currentfill}%
\pgfsetlinewidth{0.803000pt}%
\definecolor{currentstroke}{rgb}{0.000000,0.000000,0.000000}%
\pgfsetstrokecolor{currentstroke}%
\pgfsetdash{}{0pt}%
\pgfsys@defobject{currentmarker}{\pgfqpoint{0.000000in}{-0.048611in}}{\pgfqpoint{0.000000in}{0.000000in}}{%
\pgfpathmoveto{\pgfqpoint{0.000000in}{0.000000in}}%
\pgfpathlineto{\pgfqpoint{0.000000in}{-0.048611in}}%
\pgfusepath{stroke,fill}%
}%
\begin{pgfscope}%
\pgfsys@transformshift{2.525796in}{0.449444in}%
\pgfsys@useobject{currentmarker}{}%
\end{pgfscope}%
\end{pgfscope}%
\begin{pgfscope}%
\definecolor{textcolor}{rgb}{0.000000,0.000000,0.000000}%
\pgfsetstrokecolor{textcolor}%
\pgfsetfillcolor{textcolor}%
\pgftext[x=2.525796in,y=0.352222in,,top]{\color{textcolor}\rmfamily\fontsize{10.000000}{12.000000}\selectfont 0.6}%
\end{pgfscope}%
\begin{pgfscope}%
\pgfsetbuttcap%
\pgfsetroundjoin%
\definecolor{currentfill}{rgb}{0.000000,0.000000,0.000000}%
\pgfsetfillcolor{currentfill}%
\pgfsetlinewidth{0.803000pt}%
\definecolor{currentstroke}{rgb}{0.000000,0.000000,0.000000}%
\pgfsetstrokecolor{currentstroke}%
\pgfsetdash{}{0pt}%
\pgfsys@defobject{currentmarker}{\pgfqpoint{0.000000in}{-0.048611in}}{\pgfqpoint{0.000000in}{0.000000in}}{%
\pgfpathmoveto{\pgfqpoint{0.000000in}{0.000000in}}%
\pgfpathlineto{\pgfqpoint{0.000000in}{-0.048611in}}%
\pgfusepath{stroke,fill}%
}%
\begin{pgfscope}%
\pgfsys@transformshift{2.684318in}{0.449444in}%
\pgfsys@useobject{currentmarker}{}%
\end{pgfscope}%
\end{pgfscope}%
\begin{pgfscope}%
\pgfsetbuttcap%
\pgfsetroundjoin%
\definecolor{currentfill}{rgb}{0.000000,0.000000,0.000000}%
\pgfsetfillcolor{currentfill}%
\pgfsetlinewidth{0.803000pt}%
\definecolor{currentstroke}{rgb}{0.000000,0.000000,0.000000}%
\pgfsetstrokecolor{currentstroke}%
\pgfsetdash{}{0pt}%
\pgfsys@defobject{currentmarker}{\pgfqpoint{0.000000in}{-0.048611in}}{\pgfqpoint{0.000000in}{0.000000in}}{%
\pgfpathmoveto{\pgfqpoint{0.000000in}{0.000000in}}%
\pgfpathlineto{\pgfqpoint{0.000000in}{-0.048611in}}%
\pgfusepath{stroke,fill}%
}%
\begin{pgfscope}%
\pgfsys@transformshift{2.842841in}{0.449444in}%
\pgfsys@useobject{currentmarker}{}%
\end{pgfscope}%
\end{pgfscope}%
\begin{pgfscope}%
\definecolor{textcolor}{rgb}{0.000000,0.000000,0.000000}%
\pgfsetstrokecolor{textcolor}%
\pgfsetfillcolor{textcolor}%
\pgftext[x=2.842841in,y=0.352222in,,top]{\color{textcolor}\rmfamily\fontsize{10.000000}{12.000000}\selectfont 0.7}%
\end{pgfscope}%
\begin{pgfscope}%
\pgfsetbuttcap%
\pgfsetroundjoin%
\definecolor{currentfill}{rgb}{0.000000,0.000000,0.000000}%
\pgfsetfillcolor{currentfill}%
\pgfsetlinewidth{0.803000pt}%
\definecolor{currentstroke}{rgb}{0.000000,0.000000,0.000000}%
\pgfsetstrokecolor{currentstroke}%
\pgfsetdash{}{0pt}%
\pgfsys@defobject{currentmarker}{\pgfqpoint{0.000000in}{-0.048611in}}{\pgfqpoint{0.000000in}{0.000000in}}{%
\pgfpathmoveto{\pgfqpoint{0.000000in}{0.000000in}}%
\pgfpathlineto{\pgfqpoint{0.000000in}{-0.048611in}}%
\pgfusepath{stroke,fill}%
}%
\begin{pgfscope}%
\pgfsys@transformshift{3.001364in}{0.449444in}%
\pgfsys@useobject{currentmarker}{}%
\end{pgfscope}%
\end{pgfscope}%
\begin{pgfscope}%
\pgfsetbuttcap%
\pgfsetroundjoin%
\definecolor{currentfill}{rgb}{0.000000,0.000000,0.000000}%
\pgfsetfillcolor{currentfill}%
\pgfsetlinewidth{0.803000pt}%
\definecolor{currentstroke}{rgb}{0.000000,0.000000,0.000000}%
\pgfsetstrokecolor{currentstroke}%
\pgfsetdash{}{0pt}%
\pgfsys@defobject{currentmarker}{\pgfqpoint{0.000000in}{-0.048611in}}{\pgfqpoint{0.000000in}{0.000000in}}{%
\pgfpathmoveto{\pgfqpoint{0.000000in}{0.000000in}}%
\pgfpathlineto{\pgfqpoint{0.000000in}{-0.048611in}}%
\pgfusepath{stroke,fill}%
}%
\begin{pgfscope}%
\pgfsys@transformshift{3.159887in}{0.449444in}%
\pgfsys@useobject{currentmarker}{}%
\end{pgfscope}%
\end{pgfscope}%
\begin{pgfscope}%
\definecolor{textcolor}{rgb}{0.000000,0.000000,0.000000}%
\pgfsetstrokecolor{textcolor}%
\pgfsetfillcolor{textcolor}%
\pgftext[x=3.159887in,y=0.352222in,,top]{\color{textcolor}\rmfamily\fontsize{10.000000}{12.000000}\selectfont 0.8}%
\end{pgfscope}%
\begin{pgfscope}%
\pgfsetbuttcap%
\pgfsetroundjoin%
\definecolor{currentfill}{rgb}{0.000000,0.000000,0.000000}%
\pgfsetfillcolor{currentfill}%
\pgfsetlinewidth{0.803000pt}%
\definecolor{currentstroke}{rgb}{0.000000,0.000000,0.000000}%
\pgfsetstrokecolor{currentstroke}%
\pgfsetdash{}{0pt}%
\pgfsys@defobject{currentmarker}{\pgfqpoint{0.000000in}{-0.048611in}}{\pgfqpoint{0.000000in}{0.000000in}}{%
\pgfpathmoveto{\pgfqpoint{0.000000in}{0.000000in}}%
\pgfpathlineto{\pgfqpoint{0.000000in}{-0.048611in}}%
\pgfusepath{stroke,fill}%
}%
\begin{pgfscope}%
\pgfsys@transformshift{3.318409in}{0.449444in}%
\pgfsys@useobject{currentmarker}{}%
\end{pgfscope}%
\end{pgfscope}%
\begin{pgfscope}%
\pgfsetbuttcap%
\pgfsetroundjoin%
\definecolor{currentfill}{rgb}{0.000000,0.000000,0.000000}%
\pgfsetfillcolor{currentfill}%
\pgfsetlinewidth{0.803000pt}%
\definecolor{currentstroke}{rgb}{0.000000,0.000000,0.000000}%
\pgfsetstrokecolor{currentstroke}%
\pgfsetdash{}{0pt}%
\pgfsys@defobject{currentmarker}{\pgfqpoint{0.000000in}{-0.048611in}}{\pgfqpoint{0.000000in}{0.000000in}}{%
\pgfpathmoveto{\pgfqpoint{0.000000in}{0.000000in}}%
\pgfpathlineto{\pgfqpoint{0.000000in}{-0.048611in}}%
\pgfusepath{stroke,fill}%
}%
\begin{pgfscope}%
\pgfsys@transformshift{3.476932in}{0.449444in}%
\pgfsys@useobject{currentmarker}{}%
\end{pgfscope}%
\end{pgfscope}%
\begin{pgfscope}%
\definecolor{textcolor}{rgb}{0.000000,0.000000,0.000000}%
\pgfsetstrokecolor{textcolor}%
\pgfsetfillcolor{textcolor}%
\pgftext[x=3.476932in,y=0.352222in,,top]{\color{textcolor}\rmfamily\fontsize{10.000000}{12.000000}\selectfont 0.9}%
\end{pgfscope}%
\begin{pgfscope}%
\pgfsetbuttcap%
\pgfsetroundjoin%
\definecolor{currentfill}{rgb}{0.000000,0.000000,0.000000}%
\pgfsetfillcolor{currentfill}%
\pgfsetlinewidth{0.803000pt}%
\definecolor{currentstroke}{rgb}{0.000000,0.000000,0.000000}%
\pgfsetstrokecolor{currentstroke}%
\pgfsetdash{}{0pt}%
\pgfsys@defobject{currentmarker}{\pgfqpoint{0.000000in}{-0.048611in}}{\pgfqpoint{0.000000in}{0.000000in}}{%
\pgfpathmoveto{\pgfqpoint{0.000000in}{0.000000in}}%
\pgfpathlineto{\pgfqpoint{0.000000in}{-0.048611in}}%
\pgfusepath{stroke,fill}%
}%
\begin{pgfscope}%
\pgfsys@transformshift{3.635455in}{0.449444in}%
\pgfsys@useobject{currentmarker}{}%
\end{pgfscope}%
\end{pgfscope}%
\begin{pgfscope}%
\pgfsetbuttcap%
\pgfsetroundjoin%
\definecolor{currentfill}{rgb}{0.000000,0.000000,0.000000}%
\pgfsetfillcolor{currentfill}%
\pgfsetlinewidth{0.803000pt}%
\definecolor{currentstroke}{rgb}{0.000000,0.000000,0.000000}%
\pgfsetstrokecolor{currentstroke}%
\pgfsetdash{}{0pt}%
\pgfsys@defobject{currentmarker}{\pgfqpoint{0.000000in}{-0.048611in}}{\pgfqpoint{0.000000in}{0.000000in}}{%
\pgfpathmoveto{\pgfqpoint{0.000000in}{0.000000in}}%
\pgfpathlineto{\pgfqpoint{0.000000in}{-0.048611in}}%
\pgfusepath{stroke,fill}%
}%
\begin{pgfscope}%
\pgfsys@transformshift{3.793978in}{0.449444in}%
\pgfsys@useobject{currentmarker}{}%
\end{pgfscope}%
\end{pgfscope}%
\begin{pgfscope}%
\definecolor{textcolor}{rgb}{0.000000,0.000000,0.000000}%
\pgfsetstrokecolor{textcolor}%
\pgfsetfillcolor{textcolor}%
\pgftext[x=3.793978in,y=0.352222in,,top]{\color{textcolor}\rmfamily\fontsize{10.000000}{12.000000}\selectfont 1.0}%
\end{pgfscope}%
\begin{pgfscope}%
\pgfsetbuttcap%
\pgfsetroundjoin%
\definecolor{currentfill}{rgb}{0.000000,0.000000,0.000000}%
\pgfsetfillcolor{currentfill}%
\pgfsetlinewidth{0.803000pt}%
\definecolor{currentstroke}{rgb}{0.000000,0.000000,0.000000}%
\pgfsetstrokecolor{currentstroke}%
\pgfsetdash{}{0pt}%
\pgfsys@defobject{currentmarker}{\pgfqpoint{0.000000in}{-0.048611in}}{\pgfqpoint{0.000000in}{0.000000in}}{%
\pgfpathmoveto{\pgfqpoint{0.000000in}{0.000000in}}%
\pgfpathlineto{\pgfqpoint{0.000000in}{-0.048611in}}%
\pgfusepath{stroke,fill}%
}%
\begin{pgfscope}%
\pgfsys@transformshift{3.952500in}{0.449444in}%
\pgfsys@useobject{currentmarker}{}%
\end{pgfscope}%
\end{pgfscope}%
\begin{pgfscope}%
\definecolor{textcolor}{rgb}{0.000000,0.000000,0.000000}%
\pgfsetstrokecolor{textcolor}%
\pgfsetfillcolor{textcolor}%
\pgftext[x=2.208750in,y=0.173333in,,top]{\color{textcolor}\rmfamily\fontsize{10.000000}{12.000000}\selectfont \(\displaystyle p\)}%
\end{pgfscope}%
\begin{pgfscope}%
\pgfsetbuttcap%
\pgfsetroundjoin%
\definecolor{currentfill}{rgb}{0.000000,0.000000,0.000000}%
\pgfsetfillcolor{currentfill}%
\pgfsetlinewidth{0.803000pt}%
\definecolor{currentstroke}{rgb}{0.000000,0.000000,0.000000}%
\pgfsetstrokecolor{currentstroke}%
\pgfsetdash{}{0pt}%
\pgfsys@defobject{currentmarker}{\pgfqpoint{-0.048611in}{0.000000in}}{\pgfqpoint{-0.000000in}{0.000000in}}{%
\pgfpathmoveto{\pgfqpoint{-0.000000in}{0.000000in}}%
\pgfpathlineto{\pgfqpoint{-0.048611in}{0.000000in}}%
\pgfusepath{stroke,fill}%
}%
\begin{pgfscope}%
\pgfsys@transformshift{0.465000in}{0.449444in}%
\pgfsys@useobject{currentmarker}{}%
\end{pgfscope}%
\end{pgfscope}%
\begin{pgfscope}%
\definecolor{textcolor}{rgb}{0.000000,0.000000,0.000000}%
\pgfsetstrokecolor{textcolor}%
\pgfsetfillcolor{textcolor}%
\pgftext[x=0.298333in, y=0.401250in, left, base]{\color{textcolor}\rmfamily\fontsize{10.000000}{12.000000}\selectfont \(\displaystyle {0}\)}%
\end{pgfscope}%
\begin{pgfscope}%
\pgfsetbuttcap%
\pgfsetroundjoin%
\definecolor{currentfill}{rgb}{0.000000,0.000000,0.000000}%
\pgfsetfillcolor{currentfill}%
\pgfsetlinewidth{0.803000pt}%
\definecolor{currentstroke}{rgb}{0.000000,0.000000,0.000000}%
\pgfsetstrokecolor{currentstroke}%
\pgfsetdash{}{0pt}%
\pgfsys@defobject{currentmarker}{\pgfqpoint{-0.048611in}{0.000000in}}{\pgfqpoint{-0.000000in}{0.000000in}}{%
\pgfpathmoveto{\pgfqpoint{-0.000000in}{0.000000in}}%
\pgfpathlineto{\pgfqpoint{-0.048611in}{0.000000in}}%
\pgfusepath{stroke,fill}%
}%
\begin{pgfscope}%
\pgfsys@transformshift{0.465000in}{0.862657in}%
\pgfsys@useobject{currentmarker}{}%
\end{pgfscope}%
\end{pgfscope}%
\begin{pgfscope}%
\definecolor{textcolor}{rgb}{0.000000,0.000000,0.000000}%
\pgfsetstrokecolor{textcolor}%
\pgfsetfillcolor{textcolor}%
\pgftext[x=0.228889in, y=0.814463in, left, base]{\color{textcolor}\rmfamily\fontsize{10.000000}{12.000000}\selectfont \(\displaystyle {10}\)}%
\end{pgfscope}%
\begin{pgfscope}%
\pgfsetbuttcap%
\pgfsetroundjoin%
\definecolor{currentfill}{rgb}{0.000000,0.000000,0.000000}%
\pgfsetfillcolor{currentfill}%
\pgfsetlinewidth{0.803000pt}%
\definecolor{currentstroke}{rgb}{0.000000,0.000000,0.000000}%
\pgfsetstrokecolor{currentstroke}%
\pgfsetdash{}{0pt}%
\pgfsys@defobject{currentmarker}{\pgfqpoint{-0.048611in}{0.000000in}}{\pgfqpoint{-0.000000in}{0.000000in}}{%
\pgfpathmoveto{\pgfqpoint{-0.000000in}{0.000000in}}%
\pgfpathlineto{\pgfqpoint{-0.048611in}{0.000000in}}%
\pgfusepath{stroke,fill}%
}%
\begin{pgfscope}%
\pgfsys@transformshift{0.465000in}{1.275870in}%
\pgfsys@useobject{currentmarker}{}%
\end{pgfscope}%
\end{pgfscope}%
\begin{pgfscope}%
\definecolor{textcolor}{rgb}{0.000000,0.000000,0.000000}%
\pgfsetstrokecolor{textcolor}%
\pgfsetfillcolor{textcolor}%
\pgftext[x=0.228889in, y=1.227675in, left, base]{\color{textcolor}\rmfamily\fontsize{10.000000}{12.000000}\selectfont \(\displaystyle {20}\)}%
\end{pgfscope}%
\begin{pgfscope}%
\definecolor{textcolor}{rgb}{0.000000,0.000000,0.000000}%
\pgfsetstrokecolor{textcolor}%
\pgfsetfillcolor{textcolor}%
\pgftext[x=0.173333in,y=1.026944in,,bottom,rotate=90.000000]{\color{textcolor}\rmfamily\fontsize{10.000000}{12.000000}\selectfont Percent of Data Set}%
\end{pgfscope}%
\begin{pgfscope}%
\pgfsetrectcap%
\pgfsetmiterjoin%
\pgfsetlinewidth{0.803000pt}%
\definecolor{currentstroke}{rgb}{0.000000,0.000000,0.000000}%
\pgfsetstrokecolor{currentstroke}%
\pgfsetdash{}{0pt}%
\pgfpathmoveto{\pgfqpoint{0.465000in}{0.449444in}}%
\pgfpathlineto{\pgfqpoint{0.465000in}{1.604444in}}%
\pgfusepath{stroke}%
\end{pgfscope}%
\begin{pgfscope}%
\pgfsetrectcap%
\pgfsetmiterjoin%
\pgfsetlinewidth{0.803000pt}%
\definecolor{currentstroke}{rgb}{0.000000,0.000000,0.000000}%
\pgfsetstrokecolor{currentstroke}%
\pgfsetdash{}{0pt}%
\pgfpathmoveto{\pgfqpoint{3.952500in}{0.449444in}}%
\pgfpathlineto{\pgfqpoint{3.952500in}{1.604444in}}%
\pgfusepath{stroke}%
\end{pgfscope}%
\begin{pgfscope}%
\pgfsetrectcap%
\pgfsetmiterjoin%
\pgfsetlinewidth{0.803000pt}%
\definecolor{currentstroke}{rgb}{0.000000,0.000000,0.000000}%
\pgfsetstrokecolor{currentstroke}%
\pgfsetdash{}{0pt}%
\pgfpathmoveto{\pgfqpoint{0.465000in}{0.449444in}}%
\pgfpathlineto{\pgfqpoint{3.952500in}{0.449444in}}%
\pgfusepath{stroke}%
\end{pgfscope}%
\begin{pgfscope}%
\pgfsetrectcap%
\pgfsetmiterjoin%
\pgfsetlinewidth{0.803000pt}%
\definecolor{currentstroke}{rgb}{0.000000,0.000000,0.000000}%
\pgfsetstrokecolor{currentstroke}%
\pgfsetdash{}{0pt}%
\pgfpathmoveto{\pgfqpoint{0.465000in}{1.604444in}}%
\pgfpathlineto{\pgfqpoint{3.952500in}{1.604444in}}%
\pgfusepath{stroke}%
\end{pgfscope}%
\begin{pgfscope}%
\pgfsetbuttcap%
\pgfsetmiterjoin%
\definecolor{currentfill}{rgb}{1.000000,1.000000,1.000000}%
\pgfsetfillcolor{currentfill}%
\pgfsetfillopacity{0.800000}%
\pgfsetlinewidth{1.003750pt}%
\definecolor{currentstroke}{rgb}{0.800000,0.800000,0.800000}%
\pgfsetstrokecolor{currentstroke}%
\pgfsetstrokeopacity{0.800000}%
\pgfsetdash{}{0pt}%
\pgfpathmoveto{\pgfqpoint{3.175556in}{1.104445in}}%
\pgfpathlineto{\pgfqpoint{3.855278in}{1.104445in}}%
\pgfpathquadraticcurveto{\pgfqpoint{3.883056in}{1.104445in}}{\pgfqpoint{3.883056in}{1.132222in}}%
\pgfpathlineto{\pgfqpoint{3.883056in}{1.507222in}}%
\pgfpathquadraticcurveto{\pgfqpoint{3.883056in}{1.535000in}}{\pgfqpoint{3.855278in}{1.535000in}}%
\pgfpathlineto{\pgfqpoint{3.175556in}{1.535000in}}%
\pgfpathquadraticcurveto{\pgfqpoint{3.147778in}{1.535000in}}{\pgfqpoint{3.147778in}{1.507222in}}%
\pgfpathlineto{\pgfqpoint{3.147778in}{1.132222in}}%
\pgfpathquadraticcurveto{\pgfqpoint{3.147778in}{1.104445in}}{\pgfqpoint{3.175556in}{1.104445in}}%
\pgfpathlineto{\pgfqpoint{3.175556in}{1.104445in}}%
\pgfpathclose%
\pgfusepath{stroke,fill}%
\end{pgfscope}%
\begin{pgfscope}%
\pgfsetbuttcap%
\pgfsetmiterjoin%
\pgfsetlinewidth{1.003750pt}%
\definecolor{currentstroke}{rgb}{0.000000,0.000000,0.000000}%
\pgfsetstrokecolor{currentstroke}%
\pgfsetdash{}{0pt}%
\pgfpathmoveto{\pgfqpoint{3.203334in}{1.382222in}}%
\pgfpathlineto{\pgfqpoint{3.481111in}{1.382222in}}%
\pgfpathlineto{\pgfqpoint{3.481111in}{1.479444in}}%
\pgfpathlineto{\pgfqpoint{3.203334in}{1.479444in}}%
\pgfpathlineto{\pgfqpoint{3.203334in}{1.382222in}}%
\pgfpathclose%
\pgfusepath{stroke}%
\end{pgfscope}%
\begin{pgfscope}%
\definecolor{textcolor}{rgb}{0.000000,0.000000,0.000000}%
\pgfsetstrokecolor{textcolor}%
\pgfsetfillcolor{textcolor}%
\pgftext[x=3.592223in,y=1.382222in,left,base]{\color{textcolor}\rmfamily\fontsize{10.000000}{12.000000}\selectfont Neg}%
\end{pgfscope}%
\begin{pgfscope}%
\pgfsetbuttcap%
\pgfsetmiterjoin%
\definecolor{currentfill}{rgb}{0.000000,0.000000,0.000000}%
\pgfsetfillcolor{currentfill}%
\pgfsetlinewidth{0.000000pt}%
\definecolor{currentstroke}{rgb}{0.000000,0.000000,0.000000}%
\pgfsetstrokecolor{currentstroke}%
\pgfsetstrokeopacity{0.000000}%
\pgfsetdash{}{0pt}%
\pgfpathmoveto{\pgfqpoint{3.203334in}{1.186944in}}%
\pgfpathlineto{\pgfqpoint{3.481111in}{1.186944in}}%
\pgfpathlineto{\pgfqpoint{3.481111in}{1.284167in}}%
\pgfpathlineto{\pgfqpoint{3.203334in}{1.284167in}}%
\pgfpathlineto{\pgfqpoint{3.203334in}{1.186944in}}%
\pgfpathclose%
\pgfusepath{fill}%
\end{pgfscope}%
\begin{pgfscope}%
\definecolor{textcolor}{rgb}{0.000000,0.000000,0.000000}%
\pgfsetstrokecolor{textcolor}%
\pgfsetfillcolor{textcolor}%
\pgftext[x=3.592223in,y=1.186944in,left,base]{\color{textcolor}\rmfamily\fontsize{10.000000}{12.000000}\selectfont Pos}%
\end{pgfscope}%
\end{pgfpicture}%
\makeatother%
\endgroup%
	
&
	\vskip 0pt
	\hfil ROC Curve
	
	%% Creator: Matplotlib, PGF backend
%%
%% To include the figure in your LaTeX document, write
%%   \input{<filename>.pgf}
%%
%% Make sure the required packages are loaded in your preamble
%%   \usepackage{pgf}
%%
%% Also ensure that all the required font packages are loaded; for instance,
%% the lmodern package is sometimes necessary when using math font.
%%   \usepackage{lmodern}
%%
%% Figures using additional raster images can only be included by \input if
%% they are in the same directory as the main LaTeX file. For loading figures
%% from other directories you can use the `import` package
%%   \usepackage{import}
%%
%% and then include the figures with
%%   \import{<path to file>}{<filename>.pgf}
%%
%% Matplotlib used the following preamble
%%   
%%   \usepackage{fontspec}
%%   \makeatletter\@ifpackageloaded{underscore}{}{\usepackage[strings]{underscore}}\makeatother
%%
\begingroup%
\makeatletter%
\begin{pgfpicture}%
\pgfpathrectangle{\pgfpointorigin}{\pgfqpoint{2.121861in}{1.654444in}}%
\pgfusepath{use as bounding box, clip}%
\begin{pgfscope}%
\pgfsetbuttcap%
\pgfsetmiterjoin%
\definecolor{currentfill}{rgb}{1.000000,1.000000,1.000000}%
\pgfsetfillcolor{currentfill}%
\pgfsetlinewidth{0.000000pt}%
\definecolor{currentstroke}{rgb}{1.000000,1.000000,1.000000}%
\pgfsetstrokecolor{currentstroke}%
\pgfsetdash{}{0pt}%
\pgfpathmoveto{\pgfqpoint{0.000000in}{0.000000in}}%
\pgfpathlineto{\pgfqpoint{2.121861in}{0.000000in}}%
\pgfpathlineto{\pgfqpoint{2.121861in}{1.654444in}}%
\pgfpathlineto{\pgfqpoint{0.000000in}{1.654444in}}%
\pgfpathlineto{\pgfqpoint{0.000000in}{0.000000in}}%
\pgfpathclose%
\pgfusepath{fill}%
\end{pgfscope}%
\begin{pgfscope}%
\pgfsetbuttcap%
\pgfsetmiterjoin%
\definecolor{currentfill}{rgb}{1.000000,1.000000,1.000000}%
\pgfsetfillcolor{currentfill}%
\pgfsetlinewidth{0.000000pt}%
\definecolor{currentstroke}{rgb}{0.000000,0.000000,0.000000}%
\pgfsetstrokecolor{currentstroke}%
\pgfsetstrokeopacity{0.000000}%
\pgfsetdash{}{0pt}%
\pgfpathmoveto{\pgfqpoint{0.503581in}{0.449444in}}%
\pgfpathlineto{\pgfqpoint{2.053581in}{0.449444in}}%
\pgfpathlineto{\pgfqpoint{2.053581in}{1.604444in}}%
\pgfpathlineto{\pgfqpoint{0.503581in}{1.604444in}}%
\pgfpathlineto{\pgfqpoint{0.503581in}{0.449444in}}%
\pgfpathclose%
\pgfusepath{fill}%
\end{pgfscope}%
\begin{pgfscope}%
\pgfsetbuttcap%
\pgfsetroundjoin%
\definecolor{currentfill}{rgb}{0.000000,0.000000,0.000000}%
\pgfsetfillcolor{currentfill}%
\pgfsetlinewidth{0.803000pt}%
\definecolor{currentstroke}{rgb}{0.000000,0.000000,0.000000}%
\pgfsetstrokecolor{currentstroke}%
\pgfsetdash{}{0pt}%
\pgfsys@defobject{currentmarker}{\pgfqpoint{0.000000in}{-0.048611in}}{\pgfqpoint{0.000000in}{0.000000in}}{%
\pgfpathmoveto{\pgfqpoint{0.000000in}{0.000000in}}%
\pgfpathlineto{\pgfqpoint{0.000000in}{-0.048611in}}%
\pgfusepath{stroke,fill}%
}%
\begin{pgfscope}%
\pgfsys@transformshift{0.574035in}{0.449444in}%
\pgfsys@useobject{currentmarker}{}%
\end{pgfscope}%
\end{pgfscope}%
\begin{pgfscope}%
\definecolor{textcolor}{rgb}{0.000000,0.000000,0.000000}%
\pgfsetstrokecolor{textcolor}%
\pgfsetfillcolor{textcolor}%
\pgftext[x=0.574035in,y=0.352222in,,top]{\color{textcolor}\rmfamily\fontsize{10.000000}{12.000000}\selectfont \(\displaystyle {0.0}\)}%
\end{pgfscope}%
\begin{pgfscope}%
\pgfsetbuttcap%
\pgfsetroundjoin%
\definecolor{currentfill}{rgb}{0.000000,0.000000,0.000000}%
\pgfsetfillcolor{currentfill}%
\pgfsetlinewidth{0.803000pt}%
\definecolor{currentstroke}{rgb}{0.000000,0.000000,0.000000}%
\pgfsetstrokecolor{currentstroke}%
\pgfsetdash{}{0pt}%
\pgfsys@defobject{currentmarker}{\pgfqpoint{0.000000in}{-0.048611in}}{\pgfqpoint{0.000000in}{0.000000in}}{%
\pgfpathmoveto{\pgfqpoint{0.000000in}{0.000000in}}%
\pgfpathlineto{\pgfqpoint{0.000000in}{-0.048611in}}%
\pgfusepath{stroke,fill}%
}%
\begin{pgfscope}%
\pgfsys@transformshift{1.278581in}{0.449444in}%
\pgfsys@useobject{currentmarker}{}%
\end{pgfscope}%
\end{pgfscope}%
\begin{pgfscope}%
\definecolor{textcolor}{rgb}{0.000000,0.000000,0.000000}%
\pgfsetstrokecolor{textcolor}%
\pgfsetfillcolor{textcolor}%
\pgftext[x=1.278581in,y=0.352222in,,top]{\color{textcolor}\rmfamily\fontsize{10.000000}{12.000000}\selectfont \(\displaystyle {0.5}\)}%
\end{pgfscope}%
\begin{pgfscope}%
\pgfsetbuttcap%
\pgfsetroundjoin%
\definecolor{currentfill}{rgb}{0.000000,0.000000,0.000000}%
\pgfsetfillcolor{currentfill}%
\pgfsetlinewidth{0.803000pt}%
\definecolor{currentstroke}{rgb}{0.000000,0.000000,0.000000}%
\pgfsetstrokecolor{currentstroke}%
\pgfsetdash{}{0pt}%
\pgfsys@defobject{currentmarker}{\pgfqpoint{0.000000in}{-0.048611in}}{\pgfqpoint{0.000000in}{0.000000in}}{%
\pgfpathmoveto{\pgfqpoint{0.000000in}{0.000000in}}%
\pgfpathlineto{\pgfqpoint{0.000000in}{-0.048611in}}%
\pgfusepath{stroke,fill}%
}%
\begin{pgfscope}%
\pgfsys@transformshift{1.983126in}{0.449444in}%
\pgfsys@useobject{currentmarker}{}%
\end{pgfscope}%
\end{pgfscope}%
\begin{pgfscope}%
\definecolor{textcolor}{rgb}{0.000000,0.000000,0.000000}%
\pgfsetstrokecolor{textcolor}%
\pgfsetfillcolor{textcolor}%
\pgftext[x=1.983126in,y=0.352222in,,top]{\color{textcolor}\rmfamily\fontsize{10.000000}{12.000000}\selectfont \(\displaystyle {1.0}\)}%
\end{pgfscope}%
\begin{pgfscope}%
\definecolor{textcolor}{rgb}{0.000000,0.000000,0.000000}%
\pgfsetstrokecolor{textcolor}%
\pgfsetfillcolor{textcolor}%
\pgftext[x=1.278581in,y=0.173333in,,top]{\color{textcolor}\rmfamily\fontsize{10.000000}{12.000000}\selectfont False positive rate}%
\end{pgfscope}%
\begin{pgfscope}%
\pgfsetbuttcap%
\pgfsetroundjoin%
\definecolor{currentfill}{rgb}{0.000000,0.000000,0.000000}%
\pgfsetfillcolor{currentfill}%
\pgfsetlinewidth{0.803000pt}%
\definecolor{currentstroke}{rgb}{0.000000,0.000000,0.000000}%
\pgfsetstrokecolor{currentstroke}%
\pgfsetdash{}{0pt}%
\pgfsys@defobject{currentmarker}{\pgfqpoint{-0.048611in}{0.000000in}}{\pgfqpoint{-0.000000in}{0.000000in}}{%
\pgfpathmoveto{\pgfqpoint{-0.000000in}{0.000000in}}%
\pgfpathlineto{\pgfqpoint{-0.048611in}{0.000000in}}%
\pgfusepath{stroke,fill}%
}%
\begin{pgfscope}%
\pgfsys@transformshift{0.503581in}{0.501944in}%
\pgfsys@useobject{currentmarker}{}%
\end{pgfscope}%
\end{pgfscope}%
\begin{pgfscope}%
\definecolor{textcolor}{rgb}{0.000000,0.000000,0.000000}%
\pgfsetstrokecolor{textcolor}%
\pgfsetfillcolor{textcolor}%
\pgftext[x=0.228889in, y=0.453750in, left, base]{\color{textcolor}\rmfamily\fontsize{10.000000}{12.000000}\selectfont \(\displaystyle {0.0}\)}%
\end{pgfscope}%
\begin{pgfscope}%
\pgfsetbuttcap%
\pgfsetroundjoin%
\definecolor{currentfill}{rgb}{0.000000,0.000000,0.000000}%
\pgfsetfillcolor{currentfill}%
\pgfsetlinewidth{0.803000pt}%
\definecolor{currentstroke}{rgb}{0.000000,0.000000,0.000000}%
\pgfsetstrokecolor{currentstroke}%
\pgfsetdash{}{0pt}%
\pgfsys@defobject{currentmarker}{\pgfqpoint{-0.048611in}{0.000000in}}{\pgfqpoint{-0.000000in}{0.000000in}}{%
\pgfpathmoveto{\pgfqpoint{-0.000000in}{0.000000in}}%
\pgfpathlineto{\pgfqpoint{-0.048611in}{0.000000in}}%
\pgfusepath{stroke,fill}%
}%
\begin{pgfscope}%
\pgfsys@transformshift{0.503581in}{1.026944in}%
\pgfsys@useobject{currentmarker}{}%
\end{pgfscope}%
\end{pgfscope}%
\begin{pgfscope}%
\definecolor{textcolor}{rgb}{0.000000,0.000000,0.000000}%
\pgfsetstrokecolor{textcolor}%
\pgfsetfillcolor{textcolor}%
\pgftext[x=0.228889in, y=0.978750in, left, base]{\color{textcolor}\rmfamily\fontsize{10.000000}{12.000000}\selectfont \(\displaystyle {0.5}\)}%
\end{pgfscope}%
\begin{pgfscope}%
\pgfsetbuttcap%
\pgfsetroundjoin%
\definecolor{currentfill}{rgb}{0.000000,0.000000,0.000000}%
\pgfsetfillcolor{currentfill}%
\pgfsetlinewidth{0.803000pt}%
\definecolor{currentstroke}{rgb}{0.000000,0.000000,0.000000}%
\pgfsetstrokecolor{currentstroke}%
\pgfsetdash{}{0pt}%
\pgfsys@defobject{currentmarker}{\pgfqpoint{-0.048611in}{0.000000in}}{\pgfqpoint{-0.000000in}{0.000000in}}{%
\pgfpathmoveto{\pgfqpoint{-0.000000in}{0.000000in}}%
\pgfpathlineto{\pgfqpoint{-0.048611in}{0.000000in}}%
\pgfusepath{stroke,fill}%
}%
\begin{pgfscope}%
\pgfsys@transformshift{0.503581in}{1.551944in}%
\pgfsys@useobject{currentmarker}{}%
\end{pgfscope}%
\end{pgfscope}%
\begin{pgfscope}%
\definecolor{textcolor}{rgb}{0.000000,0.000000,0.000000}%
\pgfsetstrokecolor{textcolor}%
\pgfsetfillcolor{textcolor}%
\pgftext[x=0.228889in, y=1.503750in, left, base]{\color{textcolor}\rmfamily\fontsize{10.000000}{12.000000}\selectfont \(\displaystyle {1.0}\)}%
\end{pgfscope}%
\begin{pgfscope}%
\definecolor{textcolor}{rgb}{0.000000,0.000000,0.000000}%
\pgfsetstrokecolor{textcolor}%
\pgfsetfillcolor{textcolor}%
\pgftext[x=0.173333in,y=1.026944in,,bottom,rotate=90.000000]{\color{textcolor}\rmfamily\fontsize{10.000000}{12.000000}\selectfont True positive rate}%
\end{pgfscope}%
\begin{pgfscope}%
\pgfpathrectangle{\pgfqpoint{0.503581in}{0.449444in}}{\pgfqpoint{1.550000in}{1.155000in}}%
\pgfusepath{clip}%
\pgfsetbuttcap%
\pgfsetroundjoin%
\pgfsetlinewidth{1.505625pt}%
\definecolor{currentstroke}{rgb}{0.000000,0.000000,0.000000}%
\pgfsetstrokecolor{currentstroke}%
\pgfsetdash{{5.550000pt}{2.400000pt}}{0.000000pt}%
\pgfpathmoveto{\pgfqpoint{0.574035in}{0.501944in}}%
\pgfpathlineto{\pgfqpoint{1.983126in}{1.551944in}}%
\pgfusepath{stroke}%
\end{pgfscope}%
\begin{pgfscope}%
\pgfpathrectangle{\pgfqpoint{0.503581in}{0.449444in}}{\pgfqpoint{1.550000in}{1.155000in}}%
\pgfusepath{clip}%
\pgfsetrectcap%
\pgfsetroundjoin%
\pgfsetlinewidth{1.505625pt}%
\definecolor{currentstroke}{rgb}{0.000000,0.000000,0.000000}%
\pgfsetstrokecolor{currentstroke}%
\pgfsetdash{}{0pt}%
\pgfpathmoveto{\pgfqpoint{0.574035in}{0.501944in}}%
\pgfpathlineto{\pgfqpoint{0.575145in}{0.517710in}}%
\pgfpathlineto{\pgfqpoint{0.575257in}{0.518819in}}%
\pgfpathlineto{\pgfqpoint{0.576367in}{0.531084in}}%
\pgfpathlineto{\pgfqpoint{0.576478in}{0.532183in}}%
\pgfpathlineto{\pgfqpoint{0.577588in}{0.542551in}}%
\pgfpathlineto{\pgfqpoint{0.577732in}{0.543660in}}%
\pgfpathlineto{\pgfqpoint{0.578842in}{0.554738in}}%
\pgfpathlineto{\pgfqpoint{0.578947in}{0.555798in}}%
\pgfpathlineto{\pgfqpoint{0.580057in}{0.564668in}}%
\pgfpathlineto{\pgfqpoint{0.580234in}{0.565767in}}%
\pgfpathlineto{\pgfqpoint{0.581341in}{0.573072in}}%
\pgfpathlineto{\pgfqpoint{0.581485in}{0.574151in}}%
\pgfpathlineto{\pgfqpoint{0.582595in}{0.581563in}}%
\pgfpathlineto{\pgfqpoint{0.582812in}{0.582672in}}%
\pgfpathlineto{\pgfqpoint{0.583917in}{0.589888in}}%
\pgfpathlineto{\pgfqpoint{0.584117in}{0.590997in}}%
\pgfpathlineto{\pgfqpoint{0.585227in}{0.597368in}}%
\pgfpathlineto{\pgfqpoint{0.585392in}{0.598467in}}%
\pgfpathlineto{\pgfqpoint{0.586502in}{0.605567in}}%
\pgfpathlineto{\pgfqpoint{0.586679in}{0.606656in}}%
\pgfpathlineto{\pgfqpoint{0.587786in}{0.612541in}}%
\pgfpathlineto{\pgfqpoint{0.587951in}{0.613649in}}%
\pgfpathlineto{\pgfqpoint{0.589061in}{0.619320in}}%
\pgfpathlineto{\pgfqpoint{0.589275in}{0.620429in}}%
\pgfpathlineto{\pgfqpoint{0.590383in}{0.626109in}}%
\pgfpathlineto{\pgfqpoint{0.590555in}{0.627159in}}%
\pgfpathlineto{\pgfqpoint{0.591665in}{0.632975in}}%
\pgfpathlineto{\pgfqpoint{0.591942in}{0.634074in}}%
\pgfpathlineto{\pgfqpoint{0.593049in}{0.639599in}}%
\pgfpathlineto{\pgfqpoint{0.593291in}{0.640708in}}%
\pgfpathlineto{\pgfqpoint{0.594401in}{0.646125in}}%
\pgfpathlineto{\pgfqpoint{0.594617in}{0.647185in}}%
\pgfpathlineto{\pgfqpoint{0.595727in}{0.651971in}}%
\pgfpathlineto{\pgfqpoint{0.595941in}{0.653070in}}%
\pgfpathlineto{\pgfqpoint{0.597051in}{0.657651in}}%
\pgfpathlineto{\pgfqpoint{0.597391in}{0.658759in}}%
\pgfpathlineto{\pgfqpoint{0.598501in}{0.663652in}}%
\pgfpathlineto{\pgfqpoint{0.598808in}{0.664760in}}%
\pgfpathlineto{\pgfqpoint{0.599918in}{0.669079in}}%
\pgfpathlineto{\pgfqpoint{0.600190in}{0.670168in}}%
\pgfpathlineto{\pgfqpoint{0.601298in}{0.674438in}}%
\pgfpathlineto{\pgfqpoint{0.601300in}{0.674438in}}%
\pgfpathlineto{\pgfqpoint{0.601630in}{0.675527in}}%
\pgfpathlineto{\pgfqpoint{0.602740in}{0.679787in}}%
\pgfpathlineto{\pgfqpoint{0.603036in}{0.680887in}}%
\pgfpathlineto{\pgfqpoint{0.604145in}{0.685137in}}%
\pgfpathlineto{\pgfqpoint{0.604413in}{0.686197in}}%
\pgfpathlineto{\pgfqpoint{0.605523in}{0.690272in}}%
\pgfpathlineto{\pgfqpoint{0.605865in}{0.691371in}}%
\pgfpathlineto{\pgfqpoint{0.606970in}{0.695203in}}%
\pgfpathlineto{\pgfqpoint{0.607296in}{0.696312in}}%
\pgfpathlineto{\pgfqpoint{0.608403in}{0.700572in}}%
\pgfpathlineto{\pgfqpoint{0.608713in}{0.701681in}}%
\pgfpathlineto{\pgfqpoint{0.609818in}{0.705134in}}%
\pgfpathlineto{\pgfqpoint{0.610188in}{0.706223in}}%
\pgfpathlineto{\pgfqpoint{0.611295in}{0.710026in}}%
\pgfpathlineto{\pgfqpoint{0.611686in}{0.711135in}}%
\pgfpathlineto{\pgfqpoint{0.612794in}{0.714802in}}%
\pgfpathlineto{\pgfqpoint{0.613120in}{0.715911in}}%
\pgfpathlineto{\pgfqpoint{0.614227in}{0.719597in}}%
\pgfpathlineto{\pgfqpoint{0.614574in}{0.720706in}}%
\pgfpathlineto{\pgfqpoint{0.615672in}{0.723798in}}%
\pgfpathlineto{\pgfqpoint{0.615993in}{0.724897in}}%
\pgfpathlineto{\pgfqpoint{0.617096in}{0.728486in}}%
\pgfpathlineto{\pgfqpoint{0.617103in}{0.728486in}}%
\pgfpathlineto{\pgfqpoint{0.617438in}{0.729586in}}%
\pgfpathlineto{\pgfqpoint{0.618548in}{0.733534in}}%
\pgfpathlineto{\pgfqpoint{0.618999in}{0.734643in}}%
\pgfpathlineto{\pgfqpoint{0.620109in}{0.738154in}}%
\pgfpathlineto{\pgfqpoint{0.620491in}{0.739224in}}%
\pgfpathlineto{\pgfqpoint{0.621589in}{0.742346in}}%
\pgfpathlineto{\pgfqpoint{0.621973in}{0.743455in}}%
\pgfpathlineto{\pgfqpoint{0.623078in}{0.746927in}}%
\pgfpathlineto{\pgfqpoint{0.623481in}{0.748036in}}%
\pgfpathlineto{\pgfqpoint{0.624590in}{0.751703in}}%
\pgfpathlineto{\pgfqpoint{0.624965in}{0.752782in}}%
\pgfpathlineto{\pgfqpoint{0.626075in}{0.755593in}}%
\pgfpathlineto{\pgfqpoint{0.626577in}{0.756692in}}%
\pgfpathlineto{\pgfqpoint{0.627683in}{0.759562in}}%
\pgfpathlineto{\pgfqpoint{0.628048in}{0.760612in}}%
\pgfpathlineto{\pgfqpoint{0.629158in}{0.763802in}}%
\pgfpathlineto{\pgfqpoint{0.629498in}{0.764911in}}%
\pgfpathlineto{\pgfqpoint{0.630607in}{0.767946in}}%
\pgfpathlineto{\pgfqpoint{0.631105in}{0.769054in}}%
\pgfpathlineto{\pgfqpoint{0.632213in}{0.772021in}}%
\pgfpathlineto{\pgfqpoint{0.632604in}{0.773130in}}%
\pgfpathlineto{\pgfqpoint{0.633707in}{0.776271in}}%
\pgfpathlineto{\pgfqpoint{0.634186in}{0.777380in}}%
\pgfpathlineto{\pgfqpoint{0.635293in}{0.780502in}}%
\pgfpathlineto{\pgfqpoint{0.635749in}{0.781611in}}%
\pgfpathlineto{\pgfqpoint{0.636838in}{0.784636in}}%
\pgfpathlineto{\pgfqpoint{0.636843in}{0.784636in}}%
\pgfpathlineto{\pgfqpoint{0.637380in}{0.785745in}}%
\pgfpathlineto{\pgfqpoint{0.638490in}{0.788721in}}%
\pgfpathlineto{\pgfqpoint{0.638923in}{0.789830in}}%
\pgfpathlineto{\pgfqpoint{0.640033in}{0.792621in}}%
\pgfpathlineto{\pgfqpoint{0.640547in}{0.793720in}}%
\pgfpathlineto{\pgfqpoint{0.641655in}{0.796443in}}%
\pgfpathlineto{\pgfqpoint{0.642064in}{0.797552in}}%
\pgfpathlineto{\pgfqpoint{0.643162in}{0.800237in}}%
\pgfpathlineto{\pgfqpoint{0.643674in}{0.801345in}}%
\pgfpathlineto{\pgfqpoint{0.644782in}{0.804302in}}%
\pgfpathlineto{\pgfqpoint{0.645252in}{0.805411in}}%
\pgfpathlineto{\pgfqpoint{0.646355in}{0.807803in}}%
\pgfpathlineto{\pgfqpoint{0.646967in}{0.808912in}}%
\pgfpathlineto{\pgfqpoint{0.648076in}{0.811354in}}%
\pgfpathlineto{\pgfqpoint{0.648602in}{0.812462in}}%
\pgfpathlineto{\pgfqpoint{0.649712in}{0.815186in}}%
\pgfpathlineto{\pgfqpoint{0.650150in}{0.816294in}}%
\pgfpathlineto{\pgfqpoint{0.651259in}{0.819057in}}%
\pgfpathlineto{\pgfqpoint{0.651829in}{0.820165in}}%
\pgfpathlineto{\pgfqpoint{0.652939in}{0.823142in}}%
\pgfpathlineto{\pgfqpoint{0.653507in}{0.824212in}}%
\pgfpathlineto{\pgfqpoint{0.654617in}{0.826682in}}%
\pgfpathlineto{\pgfqpoint{0.655036in}{0.827791in}}%
\pgfpathlineto{\pgfqpoint{0.656143in}{0.830008in}}%
\pgfpathlineto{\pgfqpoint{0.656616in}{0.831098in}}%
\pgfpathlineto{\pgfqpoint{0.657718in}{0.833597in}}%
\pgfpathlineto{\pgfqpoint{0.658154in}{0.834706in}}%
\pgfpathlineto{\pgfqpoint{0.659259in}{0.837060in}}%
\pgfpathlineto{\pgfqpoint{0.659771in}{0.838169in}}%
\pgfpathlineto{\pgfqpoint{0.660878in}{0.840766in}}%
\pgfpathlineto{\pgfqpoint{0.661339in}{0.841865in}}%
\pgfpathlineto{\pgfqpoint{0.662444in}{0.844345in}}%
\pgfpathlineto{\pgfqpoint{0.662928in}{0.845444in}}%
\pgfpathlineto{\pgfqpoint{0.664028in}{0.847846in}}%
\pgfpathlineto{\pgfqpoint{0.664557in}{0.848955in}}%
\pgfpathlineto{\pgfqpoint{0.665664in}{0.851435in}}%
\pgfpathlineto{\pgfqpoint{0.666225in}{0.852544in}}%
\pgfpathlineto{\pgfqpoint{0.667335in}{0.854946in}}%
\pgfpathlineto{\pgfqpoint{0.667923in}{0.856055in}}%
\pgfpathlineto{\pgfqpoint{0.669033in}{0.858321in}}%
\pgfpathlineto{\pgfqpoint{0.669515in}{0.859430in}}%
\pgfpathlineto{\pgfqpoint{0.670622in}{0.861560in}}%
\pgfpathlineto{\pgfqpoint{0.671272in}{0.862669in}}%
\pgfpathlineto{\pgfqpoint{0.672381in}{0.865003in}}%
\pgfpathlineto{\pgfqpoint{0.673061in}{0.866112in}}%
\pgfpathlineto{\pgfqpoint{0.674159in}{0.868427in}}%
\pgfpathlineto{\pgfqpoint{0.674755in}{0.869536in}}%
\pgfpathlineto{\pgfqpoint{0.675865in}{0.871559in}}%
\pgfpathlineto{\pgfqpoint{0.676507in}{0.872667in}}%
\pgfpathlineto{\pgfqpoint{0.677612in}{0.874681in}}%
\pgfpathlineto{\pgfqpoint{0.678203in}{0.875790in}}%
\pgfpathlineto{\pgfqpoint{0.679313in}{0.877939in}}%
\pgfpathlineto{\pgfqpoint{0.679878in}{0.879038in}}%
\pgfpathlineto{\pgfqpoint{0.680988in}{0.881324in}}%
\pgfpathlineto{\pgfqpoint{0.681637in}{0.882433in}}%
\pgfpathlineto{\pgfqpoint{0.682747in}{0.884329in}}%
\pgfpathlineto{\pgfqpoint{0.683317in}{0.885438in}}%
\pgfpathlineto{\pgfqpoint{0.684425in}{0.887665in}}%
\pgfpathlineto{\pgfqpoint{0.685076in}{0.888774in}}%
\pgfpathlineto{\pgfqpoint{0.686184in}{0.890787in}}%
\pgfpathlineto{\pgfqpoint{0.686768in}{0.891867in}}%
\pgfpathlineto{\pgfqpoint{0.687873in}{0.893977in}}%
\pgfpathlineto{\pgfqpoint{0.688482in}{0.895057in}}%
\pgfpathlineto{\pgfqpoint{0.689588in}{0.897090in}}%
\pgfpathlineto{\pgfqpoint{0.690265in}{0.898199in}}%
\pgfpathlineto{\pgfqpoint{0.691375in}{0.900056in}}%
\pgfpathlineto{\pgfqpoint{0.691961in}{0.901165in}}%
\pgfpathlineto{\pgfqpoint{0.693071in}{0.903052in}}%
\pgfpathlineto{\pgfqpoint{0.693627in}{0.904161in}}%
\pgfpathlineto{\pgfqpoint{0.694734in}{0.906155in}}%
\pgfpathlineto{\pgfqpoint{0.695272in}{0.907254in}}%
\pgfpathlineto{\pgfqpoint{0.696370in}{0.909510in}}%
\pgfpathlineto{\pgfqpoint{0.697140in}{0.910619in}}%
\pgfpathlineto{\pgfqpoint{0.698250in}{0.912379in}}%
\pgfpathlineto{\pgfqpoint{0.698918in}{0.913488in}}%
\pgfpathlineto{\pgfqpoint{0.700028in}{0.915317in}}%
\pgfpathlineto{\pgfqpoint{0.700714in}{0.916416in}}%
\pgfpathlineto{\pgfqpoint{0.701824in}{0.918390in}}%
\pgfpathlineto{\pgfqpoint{0.702545in}{0.919499in}}%
\pgfpathlineto{\pgfqpoint{0.703653in}{0.921396in}}%
\pgfpathlineto{\pgfqpoint{0.704372in}{0.922504in}}%
\pgfpathlineto{\pgfqpoint{0.705482in}{0.924401in}}%
\pgfpathlineto{\pgfqpoint{0.706177in}{0.925510in}}%
\pgfpathlineto{\pgfqpoint{0.707282in}{0.927562in}}%
\pgfpathlineto{\pgfqpoint{0.707852in}{0.928671in}}%
\pgfpathlineto{\pgfqpoint{0.708918in}{0.930528in}}%
\pgfpathlineto{\pgfqpoint{0.709602in}{0.931637in}}%
\pgfpathlineto{\pgfqpoint{0.710703in}{0.933485in}}%
\pgfpathlineto{\pgfqpoint{0.711466in}{0.934594in}}%
\pgfpathlineto{\pgfqpoint{0.712576in}{0.936510in}}%
\pgfpathlineto{\pgfqpoint{0.713302in}{0.937609in}}%
\pgfpathlineto{\pgfqpoint{0.714407in}{0.939827in}}%
\pgfpathlineto{\pgfqpoint{0.715065in}{0.940936in}}%
\pgfpathlineto{\pgfqpoint{0.716161in}{0.942618in}}%
\pgfpathlineto{\pgfqpoint{0.716173in}{0.942618in}}%
\pgfpathlineto{\pgfqpoint{0.716950in}{0.943727in}}%
\pgfpathlineto{\pgfqpoint{0.718053in}{0.945614in}}%
\pgfpathlineto{\pgfqpoint{0.718816in}{0.946723in}}%
\pgfpathlineto{\pgfqpoint{0.719919in}{0.948532in}}%
\pgfpathlineto{\pgfqpoint{0.720649in}{0.949621in}}%
\pgfpathlineto{\pgfqpoint{0.721759in}{0.951450in}}%
\pgfpathlineto{\pgfqpoint{0.722390in}{0.952558in}}%
\pgfpathlineto{\pgfqpoint{0.723493in}{0.954465in}}%
\pgfpathlineto{\pgfqpoint{0.724198in}{0.955573in}}%
\pgfpathlineto{\pgfqpoint{0.725303in}{0.957548in}}%
\pgfpathlineto{\pgfqpoint{0.725966in}{0.958657in}}%
\pgfpathlineto{\pgfqpoint{0.727074in}{0.960310in}}%
\pgfpathlineto{\pgfqpoint{0.727804in}{0.961419in}}%
\pgfpathlineto{\pgfqpoint{0.728898in}{0.963218in}}%
\pgfpathlineto{\pgfqpoint{0.729740in}{0.964327in}}%
\pgfpathlineto{\pgfqpoint{0.730838in}{0.965903in}}%
\pgfpathlineto{\pgfqpoint{0.731741in}{0.967011in}}%
\pgfpathlineto{\pgfqpoint{0.732849in}{0.968645in}}%
\pgfpathlineto{\pgfqpoint{0.733721in}{0.969725in}}%
\pgfpathlineto{\pgfqpoint{0.734824in}{0.971417in}}%
\pgfpathlineto{\pgfqpoint{0.735561in}{0.972526in}}%
\pgfpathlineto{\pgfqpoint{0.736643in}{0.974180in}}%
\pgfpathlineto{\pgfqpoint{0.737409in}{0.975288in}}%
\pgfpathlineto{\pgfqpoint{0.738505in}{0.976777in}}%
\pgfpathlineto{\pgfqpoint{0.739282in}{0.977885in}}%
\pgfpathlineto{\pgfqpoint{0.740389in}{0.979733in}}%
\pgfpathlineto{\pgfqpoint{0.741085in}{0.980832in}}%
\pgfpathlineto{\pgfqpoint{0.742188in}{0.982243in}}%
\pgfpathlineto{\pgfqpoint{0.742888in}{0.983351in}}%
\pgfpathlineto{\pgfqpoint{0.743998in}{0.985005in}}%
\pgfpathlineto{\pgfqpoint{0.744673in}{0.986114in}}%
\pgfpathlineto{\pgfqpoint{0.746318in}{0.988253in}}%
\pgfpathlineto{\pgfqpoint{0.747009in}{0.989362in}}%
\pgfpathlineto{\pgfqpoint{0.748114in}{0.990918in}}%
\pgfpathlineto{\pgfqpoint{0.748947in}{0.992027in}}%
\pgfpathlineto{\pgfqpoint{0.750052in}{0.993661in}}%
\pgfpathlineto{\pgfqpoint{0.750853in}{0.994770in}}%
\pgfpathlineto{\pgfqpoint{0.751963in}{0.996433in}}%
\pgfpathlineto{\pgfqpoint{0.752775in}{0.997542in}}%
\pgfpathlineto{\pgfqpoint{0.753875in}{0.998933in}}%
\pgfpathlineto{\pgfqpoint{0.753884in}{0.998933in}}%
\pgfpathlineto{\pgfqpoint{0.754792in}{1.000042in}}%
\pgfpathlineto{\pgfqpoint{0.755902in}{1.001676in}}%
\pgfpathlineto{\pgfqpoint{0.756656in}{1.002784in}}%
\pgfpathlineto{\pgfqpoint{0.757758in}{1.004263in}}%
\pgfpathlineto{\pgfqpoint{0.758561in}{1.005371in}}%
\pgfpathlineto{\pgfqpoint{0.759664in}{1.006937in}}%
\pgfpathlineto{\pgfqpoint{0.760651in}{1.008046in}}%
\pgfpathlineto{\pgfqpoint{0.762331in}{1.010351in}}%
\pgfpathlineto{\pgfqpoint{0.763094in}{1.011460in}}%
\pgfpathlineto{\pgfqpoint{0.764180in}{1.013133in}}%
\pgfpathlineto{\pgfqpoint{0.765083in}{1.014242in}}%
\pgfpathlineto{\pgfqpoint{0.766179in}{1.015876in}}%
\pgfpathlineto{\pgfqpoint{0.766872in}{1.016985in}}%
\pgfpathlineto{\pgfqpoint{0.767982in}{1.018716in}}%
\pgfpathlineto{\pgfqpoint{0.768966in}{1.019805in}}%
\pgfpathlineto{\pgfqpoint{0.770069in}{1.020953in}}%
\pgfpathlineto{\pgfqpoint{0.770895in}{1.022062in}}%
\pgfpathlineto{\pgfqpoint{0.771998in}{1.023871in}}%
\pgfpathlineto{\pgfqpoint{0.772919in}{1.024979in}}%
\pgfpathlineto{\pgfqpoint{0.774015in}{1.026390in}}%
\pgfpathlineto{\pgfqpoint{0.774958in}{1.027499in}}%
\pgfpathlineto{\pgfqpoint{0.776065in}{1.028880in}}%
\pgfpathlineto{\pgfqpoint{0.776898in}{1.029988in}}%
\pgfpathlineto{\pgfqpoint{0.777992in}{1.031613in}}%
\pgfpathlineto{\pgfqpoint{0.778925in}{1.032722in}}%
\pgfpathlineto{\pgfqpoint{0.780032in}{1.034278in}}%
\pgfpathlineto{\pgfqpoint{0.780951in}{1.035386in}}%
\pgfpathlineto{\pgfqpoint{0.782054in}{1.037059in}}%
\pgfpathlineto{\pgfqpoint{0.782876in}{1.038168in}}%
\pgfpathlineto{\pgfqpoint{0.783985in}{1.039783in}}%
\pgfpathlineto{\pgfqpoint{0.784928in}{1.040892in}}%
\pgfpathlineto{\pgfqpoint{0.785993in}{1.042195in}}%
\pgfpathlineto{\pgfqpoint{0.786889in}{1.043304in}}%
\pgfpathlineto{\pgfqpoint{0.787994in}{1.044763in}}%
\pgfpathlineto{\pgfqpoint{0.787999in}{1.044763in}}%
\pgfpathlineto{\pgfqpoint{0.788737in}{1.045871in}}%
\pgfpathlineto{\pgfqpoint{0.789832in}{1.047369in}}%
\pgfpathlineto{\pgfqpoint{0.790563in}{1.048478in}}%
\pgfpathlineto{\pgfqpoint{0.791650in}{1.049801in}}%
\pgfpathlineto{\pgfqpoint{0.792334in}{1.050909in}}%
\pgfpathlineto{\pgfqpoint{0.793441in}{1.052359in}}%
\pgfpathlineto{\pgfqpoint{0.794318in}{1.053467in}}%
\pgfpathlineto{\pgfqpoint{0.795426in}{1.054800in}}%
\pgfpathlineto{\pgfqpoint{0.796345in}{1.055909in}}%
\pgfpathlineto{\pgfqpoint{0.797448in}{1.056988in}}%
\pgfpathlineto{\pgfqpoint{0.798332in}{1.058058in}}%
\pgfpathlineto{\pgfqpoint{0.799442in}{1.059556in}}%
\pgfpathlineto{\pgfqpoint{0.800480in}{1.060665in}}%
\pgfpathlineto{\pgfqpoint{0.801589in}{1.061813in}}%
\pgfpathlineto{\pgfqpoint{0.802418in}{1.062921in}}%
\pgfpathlineto{\pgfqpoint{0.803523in}{1.064166in}}%
\pgfpathlineto{\pgfqpoint{0.804496in}{1.065275in}}%
\pgfpathlineto{\pgfqpoint{0.805596in}{1.066539in}}%
\pgfpathlineto{\pgfqpoint{0.806571in}{1.067639in}}%
\pgfpathlineto{\pgfqpoint{0.807671in}{1.068874in}}%
\pgfpathlineto{\pgfqpoint{0.808586in}{1.069983in}}%
\pgfpathlineto{\pgfqpoint{0.809689in}{1.071373in}}%
\pgfpathlineto{\pgfqpoint{0.810680in}{1.072482in}}%
\pgfpathlineto{\pgfqpoint{0.811787in}{1.073951in}}%
\pgfpathlineto{\pgfqpoint{0.812662in}{1.075050in}}%
\pgfpathlineto{\pgfqpoint{0.813768in}{1.076538in}}%
\pgfpathlineto{\pgfqpoint{0.814710in}{1.077647in}}%
\pgfpathlineto{\pgfqpoint{0.815808in}{1.078901in}}%
\pgfpathlineto{\pgfqpoint{0.816962in}{1.080010in}}%
\pgfpathlineto{\pgfqpoint{0.818067in}{1.081489in}}%
\pgfpathlineto{\pgfqpoint{0.818070in}{1.081489in}}%
\pgfpathlineto{\pgfqpoint{0.819005in}{1.082597in}}%
\pgfpathlineto{\pgfqpoint{0.820115in}{1.083920in}}%
\pgfpathlineto{\pgfqpoint{0.821018in}{1.085019in}}%
\pgfpathlineto{\pgfqpoint{0.822125in}{1.086546in}}%
\pgfpathlineto{\pgfqpoint{0.823086in}{1.087655in}}%
\pgfpathlineto{\pgfqpoint{0.824194in}{1.088822in}}%
\pgfpathlineto{\pgfqpoint{0.825327in}{1.089931in}}%
\pgfpathlineto{\pgfqpoint{0.826418in}{1.091205in}}%
\pgfpathlineto{\pgfqpoint{0.827588in}{1.092314in}}%
\pgfpathlineto{\pgfqpoint{0.828694in}{1.093676in}}%
\pgfpathlineto{\pgfqpoint{0.829787in}{1.094784in}}%
\pgfpathlineto{\pgfqpoint{0.830892in}{1.096010in}}%
\pgfpathlineto{\pgfqpoint{0.831779in}{1.097119in}}%
\pgfpathlineto{\pgfqpoint{0.832882in}{1.098373in}}%
\pgfpathlineto{\pgfqpoint{0.834038in}{1.099472in}}%
\pgfpathlineto{\pgfqpoint{0.835134in}{1.100523in}}%
\pgfpathlineto{\pgfqpoint{0.836183in}{1.101632in}}%
\pgfpathlineto{\pgfqpoint{0.837263in}{1.102847in}}%
\pgfpathlineto{\pgfqpoint{0.838568in}{1.103956in}}%
\pgfpathlineto{\pgfqpoint{0.839671in}{1.105036in}}%
\pgfpathlineto{\pgfqpoint{0.840774in}{1.106144in}}%
\pgfpathlineto{\pgfqpoint{0.841881in}{1.107467in}}%
\pgfpathlineto{\pgfqpoint{0.842984in}{1.108576in}}%
\pgfpathlineto{\pgfqpoint{0.844089in}{1.109753in}}%
\pgfpathlineto{\pgfqpoint{0.845230in}{1.110862in}}%
\pgfpathlineto{\pgfqpoint{0.846335in}{1.112097in}}%
\pgfpathlineto{\pgfqpoint{0.847268in}{1.113206in}}%
\pgfpathlineto{\pgfqpoint{0.848357in}{1.114606in}}%
\pgfpathlineto{\pgfqpoint{0.849518in}{1.115715in}}%
\pgfpathlineto{\pgfqpoint{0.850628in}{1.116941in}}%
\pgfpathlineto{\pgfqpoint{0.851768in}{1.118049in}}%
\pgfpathlineto{\pgfqpoint{0.852866in}{1.119080in}}%
\pgfpathlineto{\pgfqpoint{0.854027in}{1.120160in}}%
\pgfpathlineto{\pgfqpoint{0.855132in}{1.121317in}}%
\pgfpathlineto{\pgfqpoint{0.856188in}{1.122426in}}%
\pgfpathlineto{\pgfqpoint{0.857287in}{1.123506in}}%
\pgfpathlineto{\pgfqpoint{0.858357in}{1.124615in}}%
\pgfpathlineto{\pgfqpoint{0.859467in}{1.125762in}}%
\pgfpathlineto{\pgfqpoint{0.860386in}{1.126871in}}%
\pgfpathlineto{\pgfqpoint{0.861486in}{1.127921in}}%
\pgfpathlineto{\pgfqpoint{0.862494in}{1.129030in}}%
\pgfpathlineto{\pgfqpoint{0.863592in}{1.130129in}}%
\pgfpathlineto{\pgfqpoint{0.864635in}{1.131238in}}%
\pgfpathlineto{\pgfqpoint{0.865744in}{1.132493in}}%
\pgfpathlineto{\pgfqpoint{0.866971in}{1.133602in}}%
\pgfpathlineto{\pgfqpoint{0.868069in}{1.134808in}}%
\pgfpathlineto{\pgfqpoint{0.869507in}{1.135916in}}%
\pgfpathlineto{\pgfqpoint{0.871028in}{1.137521in}}%
\pgfpathlineto{\pgfqpoint{0.872148in}{1.138630in}}%
\pgfpathlineto{\pgfqpoint{0.873246in}{1.139641in}}%
\pgfpathlineto{\pgfqpoint{0.874207in}{1.140741in}}%
\pgfpathlineto{\pgfqpoint{0.875310in}{1.141742in}}%
\pgfpathlineto{\pgfqpoint{0.876636in}{1.142851in}}%
\pgfpathlineto{\pgfqpoint{0.877736in}{1.144018in}}%
\pgfpathlineto{\pgfqpoint{0.878781in}{1.145127in}}%
\pgfpathlineto{\pgfqpoint{0.879872in}{1.146314in}}%
\pgfpathlineto{\pgfqpoint{0.881084in}{1.147422in}}%
\pgfpathlineto{\pgfqpoint{0.882173in}{1.148249in}}%
\pgfpathlineto{\pgfqpoint{0.882194in}{1.148249in}}%
\pgfpathlineto{\pgfqpoint{0.883455in}{1.149348in}}%
\pgfpathlineto{\pgfqpoint{0.884558in}{1.150515in}}%
\pgfpathlineto{\pgfqpoint{0.885647in}{1.151614in}}%
\pgfpathlineto{\pgfqpoint{0.886745in}{1.152772in}}%
\pgfpathlineto{\pgfqpoint{0.888053in}{1.153871in}}%
\pgfpathlineto{\pgfqpoint{0.889163in}{1.154892in}}%
\pgfpathlineto{\pgfqpoint{0.890350in}{1.155991in}}%
\pgfpathlineto{\pgfqpoint{0.891448in}{1.156925in}}%
\pgfpathlineto{\pgfqpoint{0.892700in}{1.158034in}}%
\pgfpathlineto{\pgfqpoint{0.893802in}{1.159016in}}%
\pgfpathlineto{\pgfqpoint{0.894675in}{1.160125in}}%
\pgfpathlineto{\pgfqpoint{0.895778in}{1.161360in}}%
\pgfpathlineto{\pgfqpoint{0.896934in}{1.162469in}}%
\pgfpathlineto{\pgfqpoint{0.898030in}{1.163548in}}%
\pgfpathlineto{\pgfqpoint{0.899380in}{1.164657in}}%
\pgfpathlineto{\pgfqpoint{0.900468in}{1.165649in}}%
\pgfpathlineto{\pgfqpoint{0.901509in}{1.166748in}}%
\pgfpathlineto{\pgfqpoint{0.901509in}{1.166758in}}%
\pgfpathlineto{\pgfqpoint{0.902604in}{1.167702in}}%
\pgfpathlineto{\pgfqpoint{0.904056in}{1.168810in}}%
\pgfpathlineto{\pgfqpoint{0.905155in}{1.169890in}}%
\pgfpathlineto{\pgfqpoint{0.906325in}{1.170999in}}%
\pgfpathlineto{\pgfqpoint{0.907435in}{1.171835in}}%
\pgfpathlineto{\pgfqpoint{0.908659in}{1.172915in}}%
\pgfpathlineto{\pgfqpoint{0.909752in}{1.174121in}}%
\pgfpathlineto{\pgfqpoint{0.910964in}{1.175230in}}%
\pgfpathlineto{\pgfqpoint{0.912070in}{1.176183in}}%
\pgfpathlineto{\pgfqpoint{0.913203in}{1.177282in}}%
\pgfpathlineto{\pgfqpoint{0.914306in}{1.178099in}}%
\pgfpathlineto{\pgfqpoint{0.915436in}{1.179188in}}%
\pgfpathlineto{\pgfqpoint{0.916544in}{1.180307in}}%
\pgfpathlineto{\pgfqpoint{0.917849in}{1.181415in}}%
\pgfpathlineto{\pgfqpoint{0.918947in}{1.182310in}}%
\pgfpathlineto{\pgfqpoint{0.920171in}{1.183419in}}%
\pgfpathlineto{\pgfqpoint{0.921258in}{1.184518in}}%
\pgfpathlineto{\pgfqpoint{0.922505in}{1.185617in}}%
\pgfpathlineto{\pgfqpoint{0.923608in}{1.186638in}}%
\pgfpathlineto{\pgfqpoint{0.924913in}{1.187747in}}%
\pgfpathlineto{\pgfqpoint{0.926018in}{1.188661in}}%
\pgfpathlineto{\pgfqpoint{0.927028in}{1.189770in}}%
\pgfpathlineto{\pgfqpoint{0.928126in}{1.190830in}}%
\pgfpathlineto{\pgfqpoint{0.929397in}{1.191939in}}%
\pgfpathlineto{\pgfqpoint{0.930504in}{1.192892in}}%
\pgfpathlineto{\pgfqpoint{0.931968in}{1.194001in}}%
\pgfpathlineto{\pgfqpoint{0.933073in}{1.194993in}}%
\pgfpathlineto{\pgfqpoint{0.934187in}{1.196102in}}%
\pgfpathlineto{\pgfqpoint{0.935293in}{1.197172in}}%
\pgfpathlineto{\pgfqpoint{0.935297in}{1.197172in}}%
\pgfpathlineto{\pgfqpoint{0.936519in}{1.198281in}}%
\pgfpathlineto{\pgfqpoint{0.937619in}{1.199156in}}%
\pgfpathlineto{\pgfqpoint{0.937626in}{1.199156in}}%
\pgfpathlineto{\pgfqpoint{0.938934in}{1.200265in}}%
\pgfpathlineto{\pgfqpoint{0.940037in}{1.201140in}}%
\pgfpathlineto{\pgfqpoint{0.941803in}{1.202249in}}%
\pgfpathlineto{\pgfqpoint{0.942910in}{1.203212in}}%
\pgfpathlineto{\pgfqpoint{0.944262in}{1.204321in}}%
\pgfpathlineto{\pgfqpoint{0.945363in}{1.205264in}}%
\pgfpathlineto{\pgfqpoint{0.946668in}{1.206373in}}%
\pgfpathlineto{\pgfqpoint{0.947771in}{1.207219in}}%
\pgfpathlineto{\pgfqpoint{0.949088in}{1.208328in}}%
\pgfpathlineto{\pgfqpoint{0.950195in}{1.209261in}}%
\pgfpathlineto{\pgfqpoint{0.951563in}{1.210370in}}%
\pgfpathlineto{\pgfqpoint{0.952673in}{1.211372in}}%
\pgfpathlineto{\pgfqpoint{0.953976in}{1.212481in}}%
\pgfpathlineto{\pgfqpoint{0.955079in}{1.213492in}}%
\pgfpathlineto{\pgfqpoint{0.956252in}{1.214601in}}%
\pgfpathlineto{\pgfqpoint{0.957352in}{1.215515in}}%
\pgfpathlineto{\pgfqpoint{0.958713in}{1.216624in}}%
\pgfpathlineto{\pgfqpoint{0.959823in}{1.217363in}}%
\pgfpathlineto{\pgfqpoint{0.961196in}{1.218472in}}%
\pgfpathlineto{\pgfqpoint{0.962299in}{1.219338in}}%
\pgfpathlineto{\pgfqpoint{0.963676in}{1.220447in}}%
\pgfpathlineto{\pgfqpoint{0.964786in}{1.221410in}}%
\pgfpathlineto{\pgfqpoint{0.966247in}{1.222518in}}%
\pgfpathlineto{\pgfqpoint{0.967311in}{1.223491in}}%
\pgfpathlineto{\pgfqpoint{0.968665in}{1.224600in}}%
\pgfpathlineto{\pgfqpoint{0.969775in}{1.225281in}}%
\pgfpathlineto{\pgfqpoint{0.971080in}{1.226389in}}%
\pgfpathlineto{\pgfqpoint{0.972171in}{1.227187in}}%
\pgfpathlineto{\pgfqpoint{0.972185in}{1.227187in}}%
\pgfpathlineto{\pgfqpoint{0.973488in}{1.228296in}}%
\pgfpathlineto{\pgfqpoint{0.974593in}{1.229327in}}%
\pgfpathlineto{\pgfqpoint{0.976329in}{1.230435in}}%
\pgfpathlineto{\pgfqpoint{0.977432in}{1.231408in}}%
\pgfpathlineto{\pgfqpoint{0.978837in}{1.232517in}}%
\pgfpathlineto{\pgfqpoint{0.979931in}{1.233499in}}%
\pgfpathlineto{\pgfqpoint{0.979938in}{1.233499in}}%
\pgfpathlineto{\pgfqpoint{0.981443in}{1.234608in}}%
\pgfpathlineto{\pgfqpoint{0.982546in}{1.235600in}}%
\pgfpathlineto{\pgfqpoint{0.983821in}{1.236709in}}%
\pgfpathlineto{\pgfqpoint{0.984931in}{1.237681in}}%
\pgfpathlineto{\pgfqpoint{0.986164in}{1.238790in}}%
\pgfpathlineto{\pgfqpoint{0.987262in}{1.239724in}}%
\pgfpathlineto{\pgfqpoint{0.987272in}{1.239724in}}%
\pgfpathlineto{\pgfqpoint{0.988821in}{1.240833in}}%
\pgfpathlineto{\pgfqpoint{0.989924in}{1.241708in}}%
\pgfpathlineto{\pgfqpoint{0.991264in}{1.242817in}}%
\pgfpathlineto{\pgfqpoint{0.992367in}{1.243585in}}%
\pgfpathlineto{\pgfqpoint{0.992372in}{1.243585in}}%
\pgfpathlineto{\pgfqpoint{0.993845in}{1.244694in}}%
\pgfpathlineto{\pgfqpoint{0.994950in}{1.245637in}}%
\pgfpathlineto{\pgfqpoint{0.996593in}{1.246746in}}%
\pgfpathlineto{\pgfqpoint{0.997677in}{1.247544in}}%
\pgfpathlineto{\pgfqpoint{0.999257in}{1.248653in}}%
\pgfpathlineto{\pgfqpoint{1.000346in}{1.249586in}}%
\pgfpathlineto{\pgfqpoint{1.002067in}{1.250695in}}%
\pgfpathlineto{\pgfqpoint{1.003177in}{1.251463in}}%
\pgfpathlineto{\pgfqpoint{1.004664in}{1.252572in}}%
\pgfpathlineto{\pgfqpoint{1.005765in}{1.253302in}}%
\pgfpathlineto{\pgfqpoint{1.007517in}{1.254401in}}%
\pgfpathlineto{\pgfqpoint{1.008617in}{1.255276in}}%
\pgfpathlineto{\pgfqpoint{1.008622in}{1.255276in}}%
\pgfpathlineto{\pgfqpoint{1.009913in}{1.256385in}}%
\pgfpathlineto{\pgfqpoint{1.011023in}{1.257153in}}%
\pgfpathlineto{\pgfqpoint{1.012605in}{1.258262in}}%
\pgfpathlineto{\pgfqpoint{1.013685in}{1.258991in}}%
\pgfpathlineto{\pgfqpoint{1.015051in}{1.260100in}}%
\pgfpathlineto{\pgfqpoint{1.016121in}{1.261034in}}%
\pgfpathlineto{\pgfqpoint{1.017712in}{1.262143in}}%
\pgfpathlineto{\pgfqpoint{1.018820in}{1.263067in}}%
\pgfpathlineto{\pgfqpoint{1.020497in}{1.264166in}}%
\pgfpathlineto{\pgfqpoint{1.021603in}{1.264905in}}%
\pgfpathlineto{\pgfqpoint{1.023192in}{1.266014in}}%
\pgfpathlineto{\pgfqpoint{1.024250in}{1.266870in}}%
\pgfpathlineto{\pgfqpoint{1.024297in}{1.266870in}}%
\pgfpathlineto{\pgfqpoint{1.025684in}{1.267969in}}%
\pgfpathlineto{\pgfqpoint{1.026752in}{1.268727in}}%
\pgfpathlineto{\pgfqpoint{1.028657in}{1.269836in}}%
\pgfpathlineto{\pgfqpoint{1.029756in}{1.270498in}}%
\pgfpathlineto{\pgfqpoint{1.031370in}{1.271606in}}%
\pgfpathlineto{\pgfqpoint{1.032436in}{1.272307in}}%
\pgfpathlineto{\pgfqpoint{1.034023in}{1.273415in}}%
\pgfpathlineto{\pgfqpoint{1.035112in}{1.274310in}}%
\pgfpathlineto{\pgfqpoint{1.036961in}{1.275419in}}%
\pgfpathlineto{\pgfqpoint{1.038067in}{1.276362in}}%
\pgfpathlineto{\pgfqpoint{1.039958in}{1.277471in}}%
\pgfpathlineto{\pgfqpoint{1.041061in}{1.278123in}}%
\pgfpathlineto{\pgfqpoint{1.042397in}{1.279232in}}%
\pgfpathlineto{\pgfqpoint{1.043497in}{1.280019in}}%
\pgfpathlineto{\pgfqpoint{1.045324in}{1.281128in}}%
\pgfpathlineto{\pgfqpoint{1.046434in}{1.281819in}}%
\pgfpathlineto{\pgfqpoint{1.048062in}{1.282928in}}%
\pgfpathlineto{\pgfqpoint{1.049137in}{1.283647in}}%
\pgfpathlineto{\pgfqpoint{1.050757in}{1.284756in}}%
\pgfpathlineto{\pgfqpoint{1.051855in}{1.285593in}}%
\pgfpathlineto{\pgfqpoint{1.053346in}{1.286701in}}%
\pgfpathlineto{\pgfqpoint{1.054451in}{1.287596in}}%
\pgfpathlineto{\pgfqpoint{1.056062in}{1.288705in}}%
\pgfpathlineto{\pgfqpoint{1.057167in}{1.289610in}}%
\pgfpathlineto{\pgfqpoint{1.057171in}{1.289610in}}%
\pgfpathlineto{\pgfqpoint{1.059028in}{1.290718in}}%
\pgfpathlineto{\pgfqpoint{1.060136in}{1.291477in}}%
\pgfpathlineto{\pgfqpoint{1.061706in}{1.292586in}}%
\pgfpathlineto{\pgfqpoint{1.062809in}{1.293519in}}%
\pgfpathlineto{\pgfqpoint{1.064405in}{1.294618in}}%
\pgfpathlineto{\pgfqpoint{1.065506in}{1.295377in}}%
\pgfpathlineto{\pgfqpoint{1.067153in}{1.296486in}}%
\pgfpathlineto{\pgfqpoint{1.068251in}{1.297108in}}%
\pgfpathlineto{\pgfqpoint{1.070034in}{1.298217in}}%
\pgfpathlineto{\pgfqpoint{1.071136in}{1.298879in}}%
\pgfpathlineto{\pgfqpoint{1.073151in}{1.299987in}}%
\pgfpathlineto{\pgfqpoint{1.074245in}{1.300736in}}%
\pgfpathlineto{\pgfqpoint{1.075904in}{1.301845in}}%
\pgfpathlineto{\pgfqpoint{1.077007in}{1.302438in}}%
\pgfpathlineto{\pgfqpoint{1.078882in}{1.303547in}}%
\pgfpathlineto{\pgfqpoint{1.079985in}{1.304179in}}%
\pgfpathlineto{\pgfqpoint{1.081786in}{1.305278in}}%
\pgfpathlineto{\pgfqpoint{1.082845in}{1.305940in}}%
\pgfpathlineto{\pgfqpoint{1.082875in}{1.305940in}}%
\pgfpathlineto{\pgfqpoint{1.084743in}{1.307049in}}%
\pgfpathlineto{\pgfqpoint{1.085823in}{1.307681in}}%
\pgfpathlineto{\pgfqpoint{1.085853in}{1.307681in}}%
\pgfpathlineto{\pgfqpoint{1.087891in}{1.308790in}}%
\pgfpathlineto{\pgfqpoint{1.088980in}{1.309470in}}%
\pgfpathlineto{\pgfqpoint{1.090604in}{1.310579in}}%
\pgfpathlineto{\pgfqpoint{1.091709in}{1.311124in}}%
\pgfpathlineto{\pgfqpoint{1.093531in}{1.312233in}}%
\pgfpathlineto{\pgfqpoint{1.094639in}{1.312777in}}%
\pgfpathlineto{\pgfqpoint{1.096407in}{1.313886in}}%
\pgfpathlineto{\pgfqpoint{1.097494in}{1.314499in}}%
\pgfpathlineto{\pgfqpoint{1.099327in}{1.315608in}}%
\pgfpathlineto{\pgfqpoint{1.100725in}{1.316677in}}%
\pgfpathlineto{\pgfqpoint{1.102368in}{1.317786in}}%
\pgfpathlineto{\pgfqpoint{1.103455in}{1.318467in}}%
\pgfpathlineto{\pgfqpoint{1.105542in}{1.319566in}}%
\pgfpathlineto{\pgfqpoint{1.106638in}{1.320325in}}%
\pgfpathlineto{\pgfqpoint{1.106652in}{1.320325in}}%
\pgfpathlineto{\pgfqpoint{1.108415in}{1.321434in}}%
\pgfpathlineto{\pgfqpoint{1.109523in}{1.322085in}}%
\pgfpathlineto{\pgfqpoint{1.111405in}{1.323194in}}%
\pgfpathlineto{\pgfqpoint{1.112506in}{1.324060in}}%
\pgfpathlineto{\pgfqpoint{1.114113in}{1.325149in}}%
\pgfpathlineto{\pgfqpoint{1.114113in}{1.325159in}}%
\pgfpathlineto{\pgfqpoint{1.115209in}{1.325859in}}%
\pgfpathlineto{\pgfqpoint{1.117031in}{1.326968in}}%
\pgfpathlineto{\pgfqpoint{1.118106in}{1.327629in}}%
\pgfpathlineto{\pgfqpoint{1.118139in}{1.327629in}}%
\pgfpathlineto{\pgfqpoint{1.120040in}{1.328738in}}%
\pgfpathlineto{\pgfqpoint{1.121112in}{1.329322in}}%
\pgfpathlineto{\pgfqpoint{1.121145in}{1.329322in}}%
\pgfpathlineto{\pgfqpoint{1.122892in}{1.330430in}}%
\pgfpathlineto{\pgfqpoint{1.123993in}{1.331131in}}%
\pgfpathlineto{\pgfqpoint{1.125926in}{1.332239in}}%
\pgfpathlineto{\pgfqpoint{1.127022in}{1.332901in}}%
\pgfpathlineto{\pgfqpoint{1.127036in}{1.332901in}}%
\pgfpathlineto{\pgfqpoint{1.128860in}{1.334010in}}%
\pgfpathlineto{\pgfqpoint{1.130168in}{1.334700in}}%
\pgfpathlineto{\pgfqpoint{1.131922in}{1.335809in}}%
\pgfpathlineto{\pgfqpoint{1.133030in}{1.336480in}}%
\pgfpathlineto{\pgfqpoint{1.135033in}{1.337589in}}%
\pgfpathlineto{\pgfqpoint{1.136136in}{1.338095in}}%
\pgfpathlineto{\pgfqpoint{1.137923in}{1.339203in}}%
\pgfpathlineto{\pgfqpoint{1.139009in}{1.339933in}}%
\pgfpathlineto{\pgfqpoint{1.139033in}{1.339933in}}%
\pgfpathlineto{\pgfqpoint{1.140380in}{1.340789in}}%
\pgfpathlineto{\pgfqpoint{1.142588in}{1.341897in}}%
\pgfpathlineto{\pgfqpoint{1.143691in}{1.342481in}}%
\pgfpathlineto{\pgfqpoint{1.145608in}{1.343590in}}%
\pgfpathlineto{\pgfqpoint{1.146711in}{1.344310in}}%
\pgfpathlineto{\pgfqpoint{1.148586in}{1.345389in}}%
\pgfpathlineto{\pgfqpoint{1.149696in}{1.346031in}}%
\pgfpathlineto{\pgfqpoint{1.152216in}{1.347130in}}%
\pgfpathlineto{\pgfqpoint{1.153307in}{1.347762in}}%
\pgfpathlineto{\pgfqpoint{1.155390in}{1.348871in}}%
\pgfpathlineto{\pgfqpoint{1.156474in}{1.349406in}}%
\pgfpathlineto{\pgfqpoint{1.156490in}{1.349406in}}%
\pgfpathlineto{\pgfqpoint{1.158794in}{1.350505in}}%
\pgfpathlineto{\pgfqpoint{1.159899in}{1.351167in}}%
\pgfpathlineto{\pgfqpoint{1.162153in}{1.352275in}}%
\pgfpathlineto{\pgfqpoint{1.163252in}{1.352849in}}%
\pgfpathlineto{\pgfqpoint{1.165632in}{1.353958in}}%
\pgfpathlineto{\pgfqpoint{1.166723in}{1.354610in}}%
\pgfpathlineto{\pgfqpoint{1.166742in}{1.354610in}}%
\pgfpathlineto{\pgfqpoint{1.169164in}{1.355718in}}%
\pgfpathlineto{\pgfqpoint{1.170274in}{1.356244in}}%
\pgfpathlineto{\pgfqpoint{1.172310in}{1.357352in}}%
\pgfpathlineto{\pgfqpoint{1.173403in}{1.357936in}}%
\pgfpathlineto{\pgfqpoint{1.173419in}{1.357936in}}%
\pgfpathlineto{\pgfqpoint{1.175609in}{1.359045in}}%
\pgfpathlineto{\pgfqpoint{1.176856in}{1.359551in}}%
\pgfpathlineto{\pgfqpoint{1.178857in}{1.360659in}}%
\pgfpathlineto{\pgfqpoint{1.179937in}{1.361253in}}%
\pgfpathlineto{\pgfqpoint{1.182142in}{1.362361in}}%
\pgfpathlineto{\pgfqpoint{1.183227in}{1.362838in}}%
\pgfpathlineto{\pgfqpoint{1.185272in}{1.363947in}}%
\pgfpathlineto{\pgfqpoint{1.186335in}{1.364491in}}%
\pgfpathlineto{\pgfqpoint{1.186365in}{1.364491in}}%
\pgfpathlineto{\pgfqpoint{1.189053in}{1.365600in}}%
\pgfpathlineto{\pgfqpoint{1.190128in}{1.366281in}}%
\pgfpathlineto{\pgfqpoint{1.192354in}{1.367390in}}%
\pgfpathlineto{\pgfqpoint{1.193460in}{1.367905in}}%
\pgfpathlineto{\pgfqpoint{1.195582in}{1.369014in}}%
\pgfpathlineto{\pgfqpoint{1.196684in}{1.369656in}}%
\pgfpathlineto{\pgfqpoint{1.198897in}{1.370745in}}%
\pgfpathlineto{\pgfqpoint{1.200000in}{1.371397in}}%
\pgfpathlineto{\pgfqpoint{1.200005in}{1.371397in}}%
\pgfpathlineto{\pgfqpoint{1.202094in}{1.372506in}}%
\pgfpathlineto{\pgfqpoint{1.203174in}{1.373128in}}%
\pgfpathlineto{\pgfqpoint{1.206052in}{1.374237in}}%
\pgfpathlineto{\pgfqpoint{1.207152in}{1.374898in}}%
\pgfpathlineto{\pgfqpoint{1.209796in}{1.376007in}}%
\pgfpathlineto{\pgfqpoint{1.210905in}{1.376532in}}%
\pgfpathlineto{\pgfqpoint{1.212846in}{1.377641in}}%
\pgfpathlineto{\pgfqpoint{1.213951in}{1.378244in}}%
\pgfpathlineto{\pgfqpoint{1.216171in}{1.379353in}}%
\pgfpathlineto{\pgfqpoint{1.217274in}{1.379791in}}%
\pgfpathlineto{\pgfqpoint{1.220077in}{1.380899in}}%
\pgfpathlineto{\pgfqpoint{1.221299in}{1.381434in}}%
\pgfpathlineto{\pgfqpoint{1.224000in}{1.382543in}}%
\pgfpathlineto{\pgfqpoint{1.225075in}{1.383020in}}%
\pgfpathlineto{\pgfqpoint{1.227130in}{1.384129in}}%
\pgfpathlineto{\pgfqpoint{1.228240in}{1.384527in}}%
\pgfpathlineto{\pgfqpoint{1.230701in}{1.385636in}}%
\pgfpathlineto{\pgfqpoint{1.231811in}{1.386259in}}%
\pgfpathlineto{\pgfqpoint{1.234170in}{1.387367in}}%
\pgfpathlineto{\pgfqpoint{1.235271in}{1.388087in}}%
\pgfpathlineto{\pgfqpoint{1.235278in}{1.388087in}}%
\pgfpathlineto{\pgfqpoint{1.237286in}{1.389196in}}%
\pgfpathlineto{\pgfqpoint{1.238379in}{1.389760in}}%
\pgfpathlineto{\pgfqpoint{1.241204in}{1.390869in}}%
\pgfpathlineto{\pgfqpoint{1.242307in}{1.391569in}}%
\pgfpathlineto{\pgfqpoint{1.244962in}{1.392678in}}%
\pgfpathlineto{\pgfqpoint{1.246058in}{1.393242in}}%
\pgfpathlineto{\pgfqpoint{1.246069in}{1.393242in}}%
\pgfpathlineto{\pgfqpoint{1.248340in}{1.394341in}}%
\pgfpathlineto{\pgfqpoint{1.249443in}{1.394857in}}%
\pgfpathlineto{\pgfqpoint{1.251586in}{1.395956in}}%
\pgfpathlineto{\pgfqpoint{1.252687in}{1.396452in}}%
\pgfpathlineto{\pgfqpoint{1.255095in}{1.397560in}}%
\pgfpathlineto{\pgfqpoint{1.256193in}{1.398163in}}%
\pgfpathlineto{\pgfqpoint{1.256202in}{1.398163in}}%
\pgfpathlineto{\pgfqpoint{1.258566in}{1.399272in}}%
\pgfpathlineto{\pgfqpoint{1.259671in}{1.399866in}}%
\pgfpathlineto{\pgfqpoint{1.259676in}{1.399866in}}%
\pgfpathlineto{\pgfqpoint{1.262584in}{1.400974in}}%
\pgfpathlineto{\pgfqpoint{1.263657in}{1.401500in}}%
\pgfpathlineto{\pgfqpoint{1.263683in}{1.401500in}}%
\pgfpathlineto{\pgfqpoint{1.266428in}{1.402608in}}%
\pgfpathlineto{\pgfqpoint{1.267524in}{1.403065in}}%
\pgfpathlineto{\pgfqpoint{1.269618in}{1.404174in}}%
\pgfpathlineto{\pgfqpoint{1.270723in}{1.404680in}}%
\pgfpathlineto{\pgfqpoint{1.273220in}{1.405789in}}%
\pgfpathlineto{\pgfqpoint{1.274295in}{1.406353in}}%
\pgfpathlineto{\pgfqpoint{1.276857in}{1.407462in}}%
\pgfpathlineto{\pgfqpoint{1.277929in}{1.407822in}}%
\pgfpathlineto{\pgfqpoint{1.280684in}{1.408930in}}%
\pgfpathlineto{\pgfqpoint{1.281731in}{1.409387in}}%
\pgfpathlineto{\pgfqpoint{1.281754in}{1.409387in}}%
\pgfpathlineto{\pgfqpoint{1.284251in}{1.410496in}}%
\pgfpathlineto{\pgfqpoint{1.285358in}{1.410905in}}%
\pgfpathlineto{\pgfqpoint{1.287655in}{1.412014in}}%
\pgfpathlineto{\pgfqpoint{1.288928in}{1.412558in}}%
\pgfpathlineto{\pgfqpoint{1.291862in}{1.413667in}}%
\pgfpathlineto{\pgfqpoint{1.292932in}{1.414095in}}%
\pgfpathlineto{\pgfqpoint{1.296245in}{1.415204in}}%
\pgfpathlineto{\pgfqpoint{1.297313in}{1.415748in}}%
\pgfpathlineto{\pgfqpoint{1.300026in}{1.416857in}}%
\pgfpathlineto{\pgfqpoint{1.301117in}{1.417266in}}%
\pgfpathlineto{\pgfqpoint{1.301134in}{1.417266in}}%
\pgfpathlineto{\pgfqpoint{1.303840in}{1.418374in}}%
\pgfpathlineto{\pgfqpoint{1.305084in}{1.418861in}}%
\pgfpathlineto{\pgfqpoint{1.307958in}{1.419960in}}%
\pgfpathlineto{\pgfqpoint{1.309061in}{1.420398in}}%
\pgfpathlineto{\pgfqpoint{1.312711in}{1.421506in}}%
\pgfpathlineto{\pgfqpoint{1.313782in}{1.422002in}}%
\pgfpathlineto{\pgfqpoint{1.317451in}{1.423111in}}%
\pgfpathlineto{\pgfqpoint{1.318545in}{1.423559in}}%
\pgfpathlineto{\pgfqpoint{1.318559in}{1.423559in}}%
\pgfpathlineto{\pgfqpoint{1.321616in}{1.424667in}}%
\pgfpathlineto{\pgfqpoint{1.322670in}{1.425066in}}%
\pgfpathlineto{\pgfqpoint{1.322726in}{1.425066in}}%
\pgfpathlineto{\pgfqpoint{1.325757in}{1.426175in}}%
\pgfpathlineto{\pgfqpoint{1.326821in}{1.426486in}}%
\pgfpathlineto{\pgfqpoint{1.326858in}{1.426486in}}%
\pgfpathlineto{\pgfqpoint{1.329701in}{1.427595in}}%
\pgfpathlineto{\pgfqpoint{1.330806in}{1.428140in}}%
\pgfpathlineto{\pgfqpoint{1.330811in}{1.428140in}}%
\pgfpathlineto{\pgfqpoint{1.333354in}{1.429248in}}%
\pgfpathlineto{\pgfqpoint{1.334406in}{1.429715in}}%
\pgfpathlineto{\pgfqpoint{1.334436in}{1.429715in}}%
\pgfpathlineto{\pgfqpoint{1.337107in}{1.430824in}}%
\pgfpathlineto{\pgfqpoint{1.338210in}{1.431281in}}%
\pgfpathlineto{\pgfqpoint{1.340811in}{1.432390in}}%
\pgfpathlineto{\pgfqpoint{1.341861in}{1.432750in}}%
\pgfpathlineto{\pgfqpoint{1.344667in}{1.433859in}}%
\pgfpathlineto{\pgfqpoint{1.345725in}{1.434364in}}%
\pgfpathlineto{\pgfqpoint{1.345767in}{1.434364in}}%
\pgfpathlineto{\pgfqpoint{1.348818in}{1.435473in}}%
\pgfpathlineto{\pgfqpoint{1.349914in}{1.435872in}}%
\pgfpathlineto{\pgfqpoint{1.352782in}{1.436981in}}%
\pgfpathlineto{\pgfqpoint{1.353876in}{1.437263in}}%
\pgfpathlineto{\pgfqpoint{1.356528in}{1.438372in}}%
\pgfpathlineto{\pgfqpoint{1.357624in}{1.438799in}}%
\pgfpathlineto{\pgfqpoint{1.360758in}{1.439908in}}%
\pgfpathlineto{\pgfqpoint{1.361864in}{1.440307in}}%
\pgfpathlineto{\pgfqpoint{1.365475in}{1.441416in}}%
\pgfpathlineto{\pgfqpoint{1.366538in}{1.441795in}}%
\pgfpathlineto{\pgfqpoint{1.366580in}{1.441795in}}%
\pgfpathlineto{\pgfqpoint{1.369789in}{1.442904in}}%
\pgfpathlineto{\pgfqpoint{1.370891in}{1.443332in}}%
\pgfpathlineto{\pgfqpoint{1.373481in}{1.444441in}}%
\pgfpathlineto{\pgfqpoint{1.374579in}{1.444908in}}%
\pgfpathlineto{\pgfqpoint{1.377457in}{1.446016in}}%
\pgfpathlineto{\pgfqpoint{1.378563in}{1.446444in}}%
\pgfpathlineto{\pgfqpoint{1.381701in}{1.447553in}}%
\pgfpathlineto{\pgfqpoint{1.382790in}{1.447884in}}%
\pgfpathlineto{\pgfqpoint{1.385971in}{1.448993in}}%
\pgfpathlineto{\pgfqpoint{1.387062in}{1.449527in}}%
\pgfpathlineto{\pgfqpoint{1.390350in}{1.450636in}}%
\pgfpathlineto{\pgfqpoint{1.391457in}{1.451006in}}%
\pgfpathlineto{\pgfqpoint{1.394049in}{1.452115in}}%
\pgfpathlineto{\pgfqpoint{1.395145in}{1.452484in}}%
\pgfpathlineto{\pgfqpoint{1.398912in}{1.453593in}}%
\pgfpathlineto{\pgfqpoint{1.399887in}{1.453875in}}%
\pgfpathlineto{\pgfqpoint{1.399985in}{1.453875in}}%
\pgfpathlineto{\pgfqpoint{1.404085in}{1.454984in}}%
\pgfpathlineto{\pgfqpoint{1.405194in}{1.455344in}}%
\pgfpathlineto{\pgfqpoint{1.408561in}{1.456452in}}%
\pgfpathlineto{\pgfqpoint{1.409666in}{1.456705in}}%
\pgfpathlineto{\pgfqpoint{1.409671in}{1.456705in}}%
\pgfpathlineto{\pgfqpoint{1.413122in}{1.457814in}}%
\pgfpathlineto{\pgfqpoint{1.414073in}{1.458164in}}%
\pgfpathlineto{\pgfqpoint{1.414231in}{1.458164in}}%
\pgfpathlineto{\pgfqpoint{1.417312in}{1.459273in}}%
\pgfpathlineto{\pgfqpoint{1.418364in}{1.459584in}}%
\pgfpathlineto{\pgfqpoint{1.418410in}{1.459584in}}%
\pgfpathlineto{\pgfqpoint{1.422058in}{1.460693in}}%
\pgfpathlineto{\pgfqpoint{1.423275in}{1.461014in}}%
\pgfpathlineto{\pgfqpoint{1.426993in}{1.462113in}}%
\pgfpathlineto{\pgfqpoint{1.428031in}{1.462415in}}%
\pgfpathlineto{\pgfqpoint{1.432159in}{1.463523in}}%
\pgfpathlineto{\pgfqpoint{1.433259in}{1.463844in}}%
\pgfpathlineto{\pgfqpoint{1.433264in}{1.463844in}}%
\pgfpathlineto{\pgfqpoint{1.436426in}{1.464943in}}%
\pgfpathlineto{\pgfqpoint{1.437536in}{1.465362in}}%
\pgfpathlineto{\pgfqpoint{1.441259in}{1.466470in}}%
\pgfpathlineto{\pgfqpoint{1.442450in}{1.466791in}}%
\pgfpathlineto{\pgfqpoint{1.445752in}{1.467900in}}%
\pgfpathlineto{\pgfqpoint{1.446852in}{1.468211in}}%
\pgfpathlineto{\pgfqpoint{1.450377in}{1.469320in}}%
\pgfpathlineto{\pgfqpoint{1.451433in}{1.469661in}}%
\pgfpathlineto{\pgfqpoint{1.451480in}{1.469661in}}%
\pgfpathlineto{\pgfqpoint{1.455098in}{1.470769in}}%
\pgfpathlineto{\pgfqpoint{1.456189in}{1.471032in}}%
\pgfpathlineto{\pgfqpoint{1.459184in}{1.472131in}}%
\pgfpathlineto{\pgfqpoint{1.460275in}{1.472374in}}%
\pgfpathlineto{\pgfqpoint{1.464319in}{1.473483in}}%
\pgfpathlineto{\pgfqpoint{1.465419in}{1.473814in}}%
\pgfpathlineto{\pgfqpoint{1.468779in}{1.474913in}}%
\pgfpathlineto{\pgfqpoint{1.469817in}{1.475321in}}%
\pgfpathlineto{\pgfqpoint{1.469880in}{1.475321in}}%
\pgfpathlineto{\pgfqpoint{1.474028in}{1.476430in}}%
\pgfpathlineto{\pgfqpoint{1.475115in}{1.476907in}}%
\pgfpathlineto{\pgfqpoint{1.475134in}{1.476907in}}%
\pgfpathlineto{\pgfqpoint{1.480252in}{1.478015in}}%
\pgfpathlineto{\pgfqpoint{1.481353in}{1.478414in}}%
\pgfpathlineto{\pgfqpoint{1.485471in}{1.479523in}}%
\pgfpathlineto{\pgfqpoint{1.486576in}{1.479825in}}%
\pgfpathlineto{\pgfqpoint{1.490555in}{1.480933in}}%
\pgfpathlineto{\pgfqpoint{1.491635in}{1.481225in}}%
\pgfpathlineto{\pgfqpoint{1.495888in}{1.482334in}}%
\pgfpathlineto{\pgfqpoint{1.496881in}{1.482558in}}%
\pgfpathlineto{\pgfqpoint{1.496989in}{1.482558in}}%
\pgfpathlineto{\pgfqpoint{1.501267in}{1.483666in}}%
\pgfpathlineto{\pgfqpoint{1.502242in}{1.483939in}}%
\pgfpathlineto{\pgfqpoint{1.502307in}{1.483939in}}%
\pgfpathlineto{\pgfqpoint{1.506547in}{1.485047in}}%
\pgfpathlineto{\pgfqpoint{1.507591in}{1.485291in}}%
\pgfpathlineto{\pgfqpoint{1.507619in}{1.485291in}}%
\pgfpathlineto{\pgfqpoint{1.512312in}{1.486399in}}%
\pgfpathlineto{\pgfqpoint{1.513418in}{1.486691in}}%
\pgfpathlineto{\pgfqpoint{1.513422in}{1.486691in}}%
\pgfpathlineto{\pgfqpoint{1.517382in}{1.487800in}}%
\pgfpathlineto{\pgfqpoint{1.518599in}{1.488218in}}%
\pgfpathlineto{\pgfqpoint{1.523141in}{1.489327in}}%
\pgfpathlineto{\pgfqpoint{1.524230in}{1.489629in}}%
\pgfpathlineto{\pgfqpoint{1.528611in}{1.490737in}}%
\pgfpathlineto{\pgfqpoint{1.529702in}{1.491097in}}%
\pgfpathlineto{\pgfqpoint{1.529719in}{1.491097in}}%
\pgfpathlineto{\pgfqpoint{1.534277in}{1.492206in}}%
\pgfpathlineto{\pgfqpoint{1.535384in}{1.492546in}}%
\pgfpathlineto{\pgfqpoint{1.539591in}{1.493655in}}%
\pgfpathlineto{\pgfqpoint{1.540694in}{1.493996in}}%
\pgfpathlineto{\pgfqpoint{1.540698in}{1.493996in}}%
\pgfpathlineto{\pgfqpoint{1.546201in}{1.495104in}}%
\pgfpathlineto{\pgfqpoint{1.547283in}{1.495377in}}%
\pgfpathlineto{\pgfqpoint{1.547299in}{1.495377in}}%
\pgfpathlineto{\pgfqpoint{1.551241in}{1.496485in}}%
\pgfpathlineto{\pgfqpoint{1.552337in}{1.496699in}}%
\pgfpathlineto{\pgfqpoint{1.552351in}{1.496699in}}%
\pgfpathlineto{\pgfqpoint{1.556906in}{1.497808in}}%
\pgfpathlineto{\pgfqpoint{1.558007in}{1.498071in}}%
\pgfpathlineto{\pgfqpoint{1.563580in}{1.499180in}}%
\pgfpathlineto{\pgfqpoint{1.564647in}{1.499530in}}%
\pgfpathlineto{\pgfqpoint{1.564668in}{1.499530in}}%
\pgfpathlineto{\pgfqpoint{1.569417in}{1.500639in}}%
\pgfpathlineto{\pgfqpoint{1.570392in}{1.500843in}}%
\pgfpathlineto{\pgfqpoint{1.570471in}{1.500843in}}%
\pgfpathlineto{\pgfqpoint{1.575350in}{1.501952in}}%
\pgfpathlineto{\pgfqpoint{1.576535in}{1.502166in}}%
\pgfpathlineto{\pgfqpoint{1.581795in}{1.503274in}}%
\pgfpathlineto{\pgfqpoint{1.582880in}{1.503537in}}%
\pgfpathlineto{\pgfqpoint{1.588434in}{1.504646in}}%
\pgfpathlineto{\pgfqpoint{1.589532in}{1.504908in}}%
\pgfpathlineto{\pgfqpoint{1.594367in}{1.506007in}}%
\pgfpathlineto{\pgfqpoint{1.595446in}{1.506328in}}%
\pgfpathlineto{\pgfqpoint{1.600619in}{1.507437in}}%
\pgfpathlineto{\pgfqpoint{1.601996in}{1.507797in}}%
\pgfpathlineto{\pgfqpoint{1.607976in}{1.508906in}}%
\pgfpathlineto{\pgfqpoint{1.609046in}{1.509227in}}%
\pgfpathlineto{\pgfqpoint{1.609067in}{1.509227in}}%
\pgfpathlineto{\pgfqpoint{1.614165in}{1.510336in}}%
\pgfpathlineto{\pgfqpoint{1.615256in}{1.510579in}}%
\pgfpathlineto{\pgfqpoint{1.620543in}{1.511687in}}%
\pgfpathlineto{\pgfqpoint{1.621597in}{1.511931in}}%
\pgfpathlineto{\pgfqpoint{1.621622in}{1.511931in}}%
\pgfpathlineto{\pgfqpoint{1.626532in}{1.513039in}}%
\pgfpathlineto{\pgfqpoint{1.627637in}{1.513321in}}%
\pgfpathlineto{\pgfqpoint{1.633633in}{1.514430in}}%
\pgfpathlineto{\pgfqpoint{1.634643in}{1.514654in}}%
\pgfpathlineto{\pgfqpoint{1.634731in}{1.514654in}}%
\pgfpathlineto{\pgfqpoint{1.640383in}{1.515763in}}%
\pgfpathlineto{\pgfqpoint{1.641492in}{1.515918in}}%
\pgfpathlineto{\pgfqpoint{1.646956in}{1.517027in}}%
\pgfpathlineto{\pgfqpoint{1.648061in}{1.517231in}}%
\pgfpathlineto{\pgfqpoint{1.654920in}{1.518340in}}%
\pgfpathlineto{\pgfqpoint{1.656025in}{1.518535in}}%
\pgfpathlineto{\pgfqpoint{1.662235in}{1.519644in}}%
\pgfpathlineto{\pgfqpoint{1.663326in}{1.519838in}}%
\pgfpathlineto{\pgfqpoint{1.663331in}{1.519838in}}%
\pgfpathlineto{\pgfqpoint{1.669481in}{1.520947in}}%
\pgfpathlineto{\pgfqpoint{1.670574in}{1.521190in}}%
\pgfpathlineto{\pgfqpoint{1.677987in}{1.522299in}}%
\pgfpathlineto{\pgfqpoint{1.679074in}{1.522474in}}%
\pgfpathlineto{\pgfqpoint{1.686824in}{1.523583in}}%
\pgfpathlineto{\pgfqpoint{1.687887in}{1.523855in}}%
\pgfpathlineto{\pgfqpoint{1.695075in}{1.524964in}}%
\pgfpathlineto{\pgfqpoint{1.696129in}{1.525119in}}%
\pgfpathlineto{\pgfqpoint{1.696164in}{1.525119in}}%
\pgfpathlineto{\pgfqpoint{1.703309in}{1.526228in}}%
\pgfpathlineto{\pgfqpoint{1.704412in}{1.526374in}}%
\pgfpathlineto{\pgfqpoint{1.713330in}{1.527483in}}%
\pgfpathlineto{\pgfqpoint{1.714438in}{1.527677in}}%
\pgfpathlineto{\pgfqpoint{1.722267in}{1.528786in}}%
\pgfpathlineto{\pgfqpoint{1.723372in}{1.528951in}}%
\pgfpathlineto{\pgfqpoint{1.730592in}{1.530060in}}%
\pgfpathlineto{\pgfqpoint{1.731665in}{1.530187in}}%
\pgfpathlineto{\pgfqpoint{1.738324in}{1.531295in}}%
\pgfpathlineto{\pgfqpoint{1.739385in}{1.531402in}}%
\pgfpathlineto{\pgfqpoint{1.746970in}{1.532511in}}%
\pgfpathlineto{\pgfqpoint{1.748040in}{1.532686in}}%
\pgfpathlineto{\pgfqpoint{1.755637in}{1.533795in}}%
\pgfpathlineto{\pgfqpoint{1.756675in}{1.533960in}}%
\pgfpathlineto{\pgfqpoint{1.756738in}{1.533960in}}%
\pgfpathlineto{\pgfqpoint{1.765202in}{1.535069in}}%
\pgfpathlineto{\pgfqpoint{1.766305in}{1.535244in}}%
\pgfpathlineto{\pgfqpoint{1.776382in}{1.536353in}}%
\pgfpathlineto{\pgfqpoint{1.777422in}{1.536538in}}%
\pgfpathlineto{\pgfqpoint{1.777434in}{1.536538in}}%
\pgfpathlineto{\pgfqpoint{1.785789in}{1.537647in}}%
\pgfpathlineto{\pgfqpoint{1.786857in}{1.537831in}}%
\pgfpathlineto{\pgfqpoint{1.798442in}{1.538940in}}%
\pgfpathlineto{\pgfqpoint{1.799440in}{1.539047in}}%
\pgfpathlineto{\pgfqpoint{1.809659in}{1.540156in}}%
\pgfpathlineto{\pgfqpoint{1.810480in}{1.540312in}}%
\pgfpathlineto{\pgfqpoint{1.810518in}{1.540312in}}%
\pgfpathlineto{\pgfqpoint{1.823031in}{1.541420in}}%
\pgfpathlineto{\pgfqpoint{1.823994in}{1.541576in}}%
\pgfpathlineto{\pgfqpoint{1.824045in}{1.541576in}}%
\pgfpathlineto{\pgfqpoint{1.833713in}{1.542685in}}%
\pgfpathlineto{\pgfqpoint{1.834769in}{1.542753in}}%
\pgfpathlineto{\pgfqpoint{1.848029in}{1.543862in}}%
\pgfpathlineto{\pgfqpoint{1.848858in}{1.543969in}}%
\pgfpathlineto{\pgfqpoint{1.849137in}{1.543969in}}%
\pgfpathlineto{\pgfqpoint{1.862867in}{1.545077in}}%
\pgfpathlineto{\pgfqpoint{1.863767in}{1.545165in}}%
\pgfpathlineto{\pgfqpoint{1.863816in}{1.545165in}}%
\pgfpathlineto{\pgfqpoint{1.877283in}{1.546274in}}%
\pgfpathlineto{\pgfqpoint{1.878284in}{1.546381in}}%
\pgfpathlineto{\pgfqpoint{1.878391in}{1.546381in}}%
\pgfpathlineto{\pgfqpoint{1.892381in}{1.547490in}}%
\pgfpathlineto{\pgfqpoint{1.893475in}{1.547597in}}%
\pgfpathlineto{\pgfqpoint{1.909006in}{1.548705in}}%
\pgfpathlineto{\pgfqpoint{1.910113in}{1.548783in}}%
\pgfpathlineto{\pgfqpoint{1.928334in}{1.549882in}}%
\pgfpathlineto{\pgfqpoint{1.929395in}{1.549941in}}%
\pgfpathlineto{\pgfqpoint{1.929425in}{1.549941in}}%
\pgfpathlineto{\pgfqpoint{1.957297in}{1.551049in}}%
\pgfpathlineto{\pgfqpoint{1.958235in}{1.551147in}}%
\pgfpathlineto{\pgfqpoint{1.958388in}{1.551147in}}%
\pgfpathlineto{\pgfqpoint{1.983126in}{1.551944in}}%
\pgfpathlineto{\pgfqpoint{1.983126in}{1.551944in}}%
\pgfusepath{stroke}%
\end{pgfscope}%
\begin{pgfscope}%
\pgfsetrectcap%
\pgfsetmiterjoin%
\pgfsetlinewidth{0.803000pt}%
\definecolor{currentstroke}{rgb}{0.000000,0.000000,0.000000}%
\pgfsetstrokecolor{currentstroke}%
\pgfsetdash{}{0pt}%
\pgfpathmoveto{\pgfqpoint{0.503581in}{0.449444in}}%
\pgfpathlineto{\pgfqpoint{0.503581in}{1.604444in}}%
\pgfusepath{stroke}%
\end{pgfscope}%
\begin{pgfscope}%
\pgfsetrectcap%
\pgfsetmiterjoin%
\pgfsetlinewidth{0.803000pt}%
\definecolor{currentstroke}{rgb}{0.000000,0.000000,0.000000}%
\pgfsetstrokecolor{currentstroke}%
\pgfsetdash{}{0pt}%
\pgfpathmoveto{\pgfqpoint{2.053581in}{0.449444in}}%
\pgfpathlineto{\pgfqpoint{2.053581in}{1.604444in}}%
\pgfusepath{stroke}%
\end{pgfscope}%
\begin{pgfscope}%
\pgfsetrectcap%
\pgfsetmiterjoin%
\pgfsetlinewidth{0.803000pt}%
\definecolor{currentstroke}{rgb}{0.000000,0.000000,0.000000}%
\pgfsetstrokecolor{currentstroke}%
\pgfsetdash{}{0pt}%
\pgfpathmoveto{\pgfqpoint{0.503581in}{0.449444in}}%
\pgfpathlineto{\pgfqpoint{2.053581in}{0.449444in}}%
\pgfusepath{stroke}%
\end{pgfscope}%
\begin{pgfscope}%
\pgfsetrectcap%
\pgfsetmiterjoin%
\pgfsetlinewidth{0.803000pt}%
\definecolor{currentstroke}{rgb}{0.000000,0.000000,0.000000}%
\pgfsetstrokecolor{currentstroke}%
\pgfsetdash{}{0pt}%
\pgfpathmoveto{\pgfqpoint{0.503581in}{1.604444in}}%
\pgfpathlineto{\pgfqpoint{2.053581in}{1.604444in}}%
\pgfusepath{stroke}%
\end{pgfscope}%
\begin{pgfscope}%
\pgfsetbuttcap%
\pgfsetmiterjoin%
\definecolor{currentfill}{rgb}{1.000000,1.000000,1.000000}%
\pgfsetfillcolor{currentfill}%
\pgfsetfillopacity{0.800000}%
\pgfsetlinewidth{1.003750pt}%
\definecolor{currentstroke}{rgb}{0.800000,0.800000,0.800000}%
\pgfsetstrokecolor{currentstroke}%
\pgfsetstrokeopacity{0.800000}%
\pgfsetdash{}{0pt}%
\pgfpathmoveto{\pgfqpoint{0.782747in}{0.518889in}}%
\pgfpathlineto{\pgfqpoint{1.956358in}{0.518889in}}%
\pgfpathquadraticcurveto{\pgfqpoint{1.984136in}{0.518889in}}{\pgfqpoint{1.984136in}{0.546666in}}%
\pgfpathlineto{\pgfqpoint{1.984136in}{0.726388in}}%
\pgfpathquadraticcurveto{\pgfqpoint{1.984136in}{0.754166in}}{\pgfqpoint{1.956358in}{0.754166in}}%
\pgfpathlineto{\pgfqpoint{0.782747in}{0.754166in}}%
\pgfpathquadraticcurveto{\pgfqpoint{0.754970in}{0.754166in}}{\pgfqpoint{0.754970in}{0.726388in}}%
\pgfpathlineto{\pgfqpoint{0.754970in}{0.546666in}}%
\pgfpathquadraticcurveto{\pgfqpoint{0.754970in}{0.518889in}}{\pgfqpoint{0.782747in}{0.518889in}}%
\pgfpathlineto{\pgfqpoint{0.782747in}{0.518889in}}%
\pgfpathclose%
\pgfusepath{stroke,fill}%
\end{pgfscope}%
\begin{pgfscope}%
\pgfsetrectcap%
\pgfsetroundjoin%
\pgfsetlinewidth{1.505625pt}%
\definecolor{currentstroke}{rgb}{0.000000,0.000000,0.000000}%
\pgfsetstrokecolor{currentstroke}%
\pgfsetdash{}{0pt}%
\pgfpathmoveto{\pgfqpoint{0.810525in}{0.650000in}}%
\pgfpathlineto{\pgfqpoint{0.949414in}{0.650000in}}%
\pgfpathlineto{\pgfqpoint{1.088303in}{0.650000in}}%
\pgfusepath{stroke}%
\end{pgfscope}%
\begin{pgfscope}%
\definecolor{textcolor}{rgb}{0.000000,0.000000,0.000000}%
\pgfsetstrokecolor{textcolor}%
\pgfsetfillcolor{textcolor}%
\pgftext[x=1.199414in,y=0.601388in,left,base]{\color{textcolor}\rmfamily\fontsize{10.000000}{12.000000}\selectfont AUC=0.778}%
\end{pgfscope}%
\end{pgfpicture}%
\makeatother%
\endgroup%

\end{tabular}

\

The second method we will use to modify the model outputs' distribution is to employ class weights in the model building process.  Here we employed class weights proportional to the class imbalance.  The motivation behind class weights is to better separate the positive and negative classes, but note that the area under the ROC curve does not change.  We have not investigated whether the model using class weights does a better job at separating the classes in some intervals, but overall the effect is negligible.  One effect using class weights did have here is shifting the distribution.  


\

\verb|KBFC_5_Fold_alpha_balanced_gamma_0_0_Hard|



\noindent\begin{tabular}{@{\hspace{-6pt}}p{4.3in} @{\hspace{-6pt}}p{2.0in}}
	\vskip 0pt
	\hfil Raw Model Output
	
	%% Creator: Matplotlib, PGF backend
%%
%% To include the figure in your LaTeX document, write
%%   \input{<filename>.pgf}
%%
%% Make sure the required packages are loaded in your preamble
%%   \usepackage{pgf}
%%
%% Also ensure that all the required font packages are loaded; for instance,
%% the lmodern package is sometimes necessary when using math font.
%%   \usepackage{lmodern}
%%
%% Figures using additional raster images can only be included by \input if
%% they are in the same directory as the main LaTeX file. For loading figures
%% from other directories you can use the `import` package
%%   \usepackage{import}
%%
%% and then include the figures with
%%   \import{<path to file>}{<filename>.pgf}
%%
%% Matplotlib used the following preamble
%%   
%%   \usepackage{fontspec}
%%   \makeatletter\@ifpackageloaded{underscore}{}{\usepackage[strings]{underscore}}\makeatother
%%
\begingroup%
\makeatletter%
\begin{pgfpicture}%
\pgfpathrectangle{\pgfpointorigin}{\pgfqpoint{4.033056in}{1.754444in}}%
\pgfusepath{use as bounding box, clip}%
\begin{pgfscope}%
\pgfsetbuttcap%
\pgfsetmiterjoin%
\definecolor{currentfill}{rgb}{1.000000,1.000000,1.000000}%
\pgfsetfillcolor{currentfill}%
\pgfsetlinewidth{0.000000pt}%
\definecolor{currentstroke}{rgb}{1.000000,1.000000,1.000000}%
\pgfsetstrokecolor{currentstroke}%
\pgfsetdash{}{0pt}%
\pgfpathmoveto{\pgfqpoint{0.000000in}{0.000000in}}%
\pgfpathlineto{\pgfqpoint{4.033056in}{0.000000in}}%
\pgfpathlineto{\pgfqpoint{4.033056in}{1.754444in}}%
\pgfpathlineto{\pgfqpoint{0.000000in}{1.754444in}}%
\pgfpathlineto{\pgfqpoint{0.000000in}{0.000000in}}%
\pgfpathclose%
\pgfusepath{fill}%
\end{pgfscope}%
\begin{pgfscope}%
\pgfsetbuttcap%
\pgfsetmiterjoin%
\definecolor{currentfill}{rgb}{1.000000,1.000000,1.000000}%
\pgfsetfillcolor{currentfill}%
\pgfsetlinewidth{0.000000pt}%
\definecolor{currentstroke}{rgb}{0.000000,0.000000,0.000000}%
\pgfsetstrokecolor{currentstroke}%
\pgfsetstrokeopacity{0.000000}%
\pgfsetdash{}{0pt}%
\pgfpathmoveto{\pgfqpoint{0.445556in}{0.499444in}}%
\pgfpathlineto{\pgfqpoint{3.933056in}{0.499444in}}%
\pgfpathlineto{\pgfqpoint{3.933056in}{1.654444in}}%
\pgfpathlineto{\pgfqpoint{0.445556in}{1.654444in}}%
\pgfpathlineto{\pgfqpoint{0.445556in}{0.499444in}}%
\pgfpathclose%
\pgfusepath{fill}%
\end{pgfscope}%
\begin{pgfscope}%
\pgfpathrectangle{\pgfqpoint{0.445556in}{0.499444in}}{\pgfqpoint{3.487500in}{1.155000in}}%
\pgfusepath{clip}%
\pgfsetbuttcap%
\pgfsetmiterjoin%
\pgfsetlinewidth{1.003750pt}%
\definecolor{currentstroke}{rgb}{0.000000,0.000000,0.000000}%
\pgfsetstrokecolor{currentstroke}%
\pgfsetdash{}{0pt}%
\pgfpathmoveto{\pgfqpoint{0.540669in}{0.499444in}}%
\pgfpathlineto{\pgfqpoint{0.604078in}{0.499444in}}%
\pgfpathlineto{\pgfqpoint{0.604078in}{0.499444in}}%
\pgfpathlineto{\pgfqpoint{0.540669in}{0.499444in}}%
\pgfpathlineto{\pgfqpoint{0.540669in}{0.499444in}}%
\pgfpathclose%
\pgfusepath{stroke}%
\end{pgfscope}%
\begin{pgfscope}%
\pgfpathrectangle{\pgfqpoint{0.445556in}{0.499444in}}{\pgfqpoint{3.487500in}{1.155000in}}%
\pgfusepath{clip}%
\pgfsetbuttcap%
\pgfsetmiterjoin%
\pgfsetlinewidth{1.003750pt}%
\definecolor{currentstroke}{rgb}{0.000000,0.000000,0.000000}%
\pgfsetstrokecolor{currentstroke}%
\pgfsetdash{}{0pt}%
\pgfpathmoveto{\pgfqpoint{0.699192in}{0.499444in}}%
\pgfpathlineto{\pgfqpoint{0.762601in}{0.499444in}}%
\pgfpathlineto{\pgfqpoint{0.762601in}{1.087901in}}%
\pgfpathlineto{\pgfqpoint{0.699192in}{1.087901in}}%
\pgfpathlineto{\pgfqpoint{0.699192in}{0.499444in}}%
\pgfpathclose%
\pgfusepath{stroke}%
\end{pgfscope}%
\begin{pgfscope}%
\pgfpathrectangle{\pgfqpoint{0.445556in}{0.499444in}}{\pgfqpoint{3.487500in}{1.155000in}}%
\pgfusepath{clip}%
\pgfsetbuttcap%
\pgfsetmiterjoin%
\pgfsetlinewidth{1.003750pt}%
\definecolor{currentstroke}{rgb}{0.000000,0.000000,0.000000}%
\pgfsetstrokecolor{currentstroke}%
\pgfsetdash{}{0pt}%
\pgfpathmoveto{\pgfqpoint{0.857715in}{0.499444in}}%
\pgfpathlineto{\pgfqpoint{0.921124in}{0.499444in}}%
\pgfpathlineto{\pgfqpoint{0.921124in}{1.467920in}}%
\pgfpathlineto{\pgfqpoint{0.857715in}{1.467920in}}%
\pgfpathlineto{\pgfqpoint{0.857715in}{0.499444in}}%
\pgfpathclose%
\pgfusepath{stroke}%
\end{pgfscope}%
\begin{pgfscope}%
\pgfpathrectangle{\pgfqpoint{0.445556in}{0.499444in}}{\pgfqpoint{3.487500in}{1.155000in}}%
\pgfusepath{clip}%
\pgfsetbuttcap%
\pgfsetmiterjoin%
\pgfsetlinewidth{1.003750pt}%
\definecolor{currentstroke}{rgb}{0.000000,0.000000,0.000000}%
\pgfsetstrokecolor{currentstroke}%
\pgfsetdash{}{0pt}%
\pgfpathmoveto{\pgfqpoint{1.016238in}{0.499444in}}%
\pgfpathlineto{\pgfqpoint{1.079647in}{0.499444in}}%
\pgfpathlineto{\pgfqpoint{1.079647in}{1.575191in}}%
\pgfpathlineto{\pgfqpoint{1.016238in}{1.575191in}}%
\pgfpathlineto{\pgfqpoint{1.016238in}{0.499444in}}%
\pgfpathclose%
\pgfusepath{stroke}%
\end{pgfscope}%
\begin{pgfscope}%
\pgfpathrectangle{\pgfqpoint{0.445556in}{0.499444in}}{\pgfqpoint{3.487500in}{1.155000in}}%
\pgfusepath{clip}%
\pgfsetbuttcap%
\pgfsetmiterjoin%
\pgfsetlinewidth{1.003750pt}%
\definecolor{currentstroke}{rgb}{0.000000,0.000000,0.000000}%
\pgfsetstrokecolor{currentstroke}%
\pgfsetdash{}{0pt}%
\pgfpathmoveto{\pgfqpoint{1.174760in}{0.499444in}}%
\pgfpathlineto{\pgfqpoint{1.238169in}{0.499444in}}%
\pgfpathlineto{\pgfqpoint{1.238169in}{1.595893in}}%
\pgfpathlineto{\pgfqpoint{1.174760in}{1.595893in}}%
\pgfpathlineto{\pgfqpoint{1.174760in}{0.499444in}}%
\pgfpathclose%
\pgfusepath{stroke}%
\end{pgfscope}%
\begin{pgfscope}%
\pgfpathrectangle{\pgfqpoint{0.445556in}{0.499444in}}{\pgfqpoint{3.487500in}{1.155000in}}%
\pgfusepath{clip}%
\pgfsetbuttcap%
\pgfsetmiterjoin%
\pgfsetlinewidth{1.003750pt}%
\definecolor{currentstroke}{rgb}{0.000000,0.000000,0.000000}%
\pgfsetstrokecolor{currentstroke}%
\pgfsetdash{}{0pt}%
\pgfpathmoveto{\pgfqpoint{1.333283in}{0.499444in}}%
\pgfpathlineto{\pgfqpoint{1.396692in}{0.499444in}}%
\pgfpathlineto{\pgfqpoint{1.396692in}{1.599444in}}%
\pgfpathlineto{\pgfqpoint{1.333283in}{1.599444in}}%
\pgfpathlineto{\pgfqpoint{1.333283in}{0.499444in}}%
\pgfpathclose%
\pgfusepath{stroke}%
\end{pgfscope}%
\begin{pgfscope}%
\pgfpathrectangle{\pgfqpoint{0.445556in}{0.499444in}}{\pgfqpoint{3.487500in}{1.155000in}}%
\pgfusepath{clip}%
\pgfsetbuttcap%
\pgfsetmiterjoin%
\pgfsetlinewidth{1.003750pt}%
\definecolor{currentstroke}{rgb}{0.000000,0.000000,0.000000}%
\pgfsetstrokecolor{currentstroke}%
\pgfsetdash{}{0pt}%
\pgfpathmoveto{\pgfqpoint{1.491806in}{0.499444in}}%
\pgfpathlineto{\pgfqpoint{1.555215in}{0.499444in}}%
\pgfpathlineto{\pgfqpoint{1.555215in}{1.571988in}}%
\pgfpathlineto{\pgfqpoint{1.491806in}{1.571988in}}%
\pgfpathlineto{\pgfqpoint{1.491806in}{0.499444in}}%
\pgfpathclose%
\pgfusepath{stroke}%
\end{pgfscope}%
\begin{pgfscope}%
\pgfpathrectangle{\pgfqpoint{0.445556in}{0.499444in}}{\pgfqpoint{3.487500in}{1.155000in}}%
\pgfusepath{clip}%
\pgfsetbuttcap%
\pgfsetmiterjoin%
\pgfsetlinewidth{1.003750pt}%
\definecolor{currentstroke}{rgb}{0.000000,0.000000,0.000000}%
\pgfsetstrokecolor{currentstroke}%
\pgfsetdash{}{0pt}%
\pgfpathmoveto{\pgfqpoint{1.650328in}{0.499444in}}%
\pgfpathlineto{\pgfqpoint{1.713738in}{0.499444in}}%
\pgfpathlineto{\pgfqpoint{1.713738in}{1.539102in}}%
\pgfpathlineto{\pgfqpoint{1.650328in}{1.539102in}}%
\pgfpathlineto{\pgfqpoint{1.650328in}{0.499444in}}%
\pgfpathclose%
\pgfusepath{stroke}%
\end{pgfscope}%
\begin{pgfscope}%
\pgfpathrectangle{\pgfqpoint{0.445556in}{0.499444in}}{\pgfqpoint{3.487500in}{1.155000in}}%
\pgfusepath{clip}%
\pgfsetbuttcap%
\pgfsetmiterjoin%
\pgfsetlinewidth{1.003750pt}%
\definecolor{currentstroke}{rgb}{0.000000,0.000000,0.000000}%
\pgfsetstrokecolor{currentstroke}%
\pgfsetdash{}{0pt}%
\pgfpathmoveto{\pgfqpoint{1.808851in}{0.499444in}}%
\pgfpathlineto{\pgfqpoint{1.872260in}{0.499444in}}%
\pgfpathlineto{\pgfqpoint{1.872260in}{1.497442in}}%
\pgfpathlineto{\pgfqpoint{1.808851in}{1.497442in}}%
\pgfpathlineto{\pgfqpoint{1.808851in}{0.499444in}}%
\pgfpathclose%
\pgfusepath{stroke}%
\end{pgfscope}%
\begin{pgfscope}%
\pgfpathrectangle{\pgfqpoint{0.445556in}{0.499444in}}{\pgfqpoint{3.487500in}{1.155000in}}%
\pgfusepath{clip}%
\pgfsetbuttcap%
\pgfsetmiterjoin%
\pgfsetlinewidth{1.003750pt}%
\definecolor{currentstroke}{rgb}{0.000000,0.000000,0.000000}%
\pgfsetstrokecolor{currentstroke}%
\pgfsetdash{}{0pt}%
\pgfpathmoveto{\pgfqpoint{1.967374in}{0.499444in}}%
\pgfpathlineto{\pgfqpoint{2.030783in}{0.499444in}}%
\pgfpathlineto{\pgfqpoint{2.030783in}{1.456293in}}%
\pgfpathlineto{\pgfqpoint{1.967374in}{1.456293in}}%
\pgfpathlineto{\pgfqpoint{1.967374in}{0.499444in}}%
\pgfpathclose%
\pgfusepath{stroke}%
\end{pgfscope}%
\begin{pgfscope}%
\pgfpathrectangle{\pgfqpoint{0.445556in}{0.499444in}}{\pgfqpoint{3.487500in}{1.155000in}}%
\pgfusepath{clip}%
\pgfsetbuttcap%
\pgfsetmiterjoin%
\pgfsetlinewidth{1.003750pt}%
\definecolor{currentstroke}{rgb}{0.000000,0.000000,0.000000}%
\pgfsetstrokecolor{currentstroke}%
\pgfsetdash{}{0pt}%
\pgfpathmoveto{\pgfqpoint{2.125897in}{0.499444in}}%
\pgfpathlineto{\pgfqpoint{2.189306in}{0.499444in}}%
\pgfpathlineto{\pgfqpoint{2.189306in}{1.389823in}}%
\pgfpathlineto{\pgfqpoint{2.125897in}{1.389823in}}%
\pgfpathlineto{\pgfqpoint{2.125897in}{0.499444in}}%
\pgfpathclose%
\pgfusepath{stroke}%
\end{pgfscope}%
\begin{pgfscope}%
\pgfpathrectangle{\pgfqpoint{0.445556in}{0.499444in}}{\pgfqpoint{3.487500in}{1.155000in}}%
\pgfusepath{clip}%
\pgfsetbuttcap%
\pgfsetmiterjoin%
\pgfsetlinewidth{1.003750pt}%
\definecolor{currentstroke}{rgb}{0.000000,0.000000,0.000000}%
\pgfsetstrokecolor{currentstroke}%
\pgfsetdash{}{0pt}%
\pgfpathmoveto{\pgfqpoint{2.284419in}{0.499444in}}%
\pgfpathlineto{\pgfqpoint{2.347828in}{0.499444in}}%
\pgfpathlineto{\pgfqpoint{2.347828in}{1.327833in}}%
\pgfpathlineto{\pgfqpoint{2.284419in}{1.327833in}}%
\pgfpathlineto{\pgfqpoint{2.284419in}{0.499444in}}%
\pgfpathclose%
\pgfusepath{stroke}%
\end{pgfscope}%
\begin{pgfscope}%
\pgfpathrectangle{\pgfqpoint{0.445556in}{0.499444in}}{\pgfqpoint{3.487500in}{1.155000in}}%
\pgfusepath{clip}%
\pgfsetbuttcap%
\pgfsetmiterjoin%
\pgfsetlinewidth{1.003750pt}%
\definecolor{currentstroke}{rgb}{0.000000,0.000000,0.000000}%
\pgfsetstrokecolor{currentstroke}%
\pgfsetdash{}{0pt}%
\pgfpathmoveto{\pgfqpoint{2.442942in}{0.499444in}}%
\pgfpathlineto{\pgfqpoint{2.506351in}{0.499444in}}%
\pgfpathlineto{\pgfqpoint{2.506351in}{1.255189in}}%
\pgfpathlineto{\pgfqpoint{2.442942in}{1.255189in}}%
\pgfpathlineto{\pgfqpoint{2.442942in}{0.499444in}}%
\pgfpathclose%
\pgfusepath{stroke}%
\end{pgfscope}%
\begin{pgfscope}%
\pgfpathrectangle{\pgfqpoint{0.445556in}{0.499444in}}{\pgfqpoint{3.487500in}{1.155000in}}%
\pgfusepath{clip}%
\pgfsetbuttcap%
\pgfsetmiterjoin%
\pgfsetlinewidth{1.003750pt}%
\definecolor{currentstroke}{rgb}{0.000000,0.000000,0.000000}%
\pgfsetstrokecolor{currentstroke}%
\pgfsetdash{}{0pt}%
\pgfpathmoveto{\pgfqpoint{2.601465in}{0.499444in}}%
\pgfpathlineto{\pgfqpoint{2.664874in}{0.499444in}}%
\pgfpathlineto{\pgfqpoint{2.664874in}{1.170083in}}%
\pgfpathlineto{\pgfqpoint{2.601465in}{1.170083in}}%
\pgfpathlineto{\pgfqpoint{2.601465in}{0.499444in}}%
\pgfpathclose%
\pgfusepath{stroke}%
\end{pgfscope}%
\begin{pgfscope}%
\pgfpathrectangle{\pgfqpoint{0.445556in}{0.499444in}}{\pgfqpoint{3.487500in}{1.155000in}}%
\pgfusepath{clip}%
\pgfsetbuttcap%
\pgfsetmiterjoin%
\pgfsetlinewidth{1.003750pt}%
\definecolor{currentstroke}{rgb}{0.000000,0.000000,0.000000}%
\pgfsetstrokecolor{currentstroke}%
\pgfsetdash{}{0pt}%
\pgfpathmoveto{\pgfqpoint{2.759988in}{0.499444in}}%
\pgfpathlineto{\pgfqpoint{2.823397in}{0.499444in}}%
\pgfpathlineto{\pgfqpoint{2.823397in}{1.075345in}}%
\pgfpathlineto{\pgfqpoint{2.759988in}{1.075345in}}%
\pgfpathlineto{\pgfqpoint{2.759988in}{0.499444in}}%
\pgfpathclose%
\pgfusepath{stroke}%
\end{pgfscope}%
\begin{pgfscope}%
\pgfpathrectangle{\pgfqpoint{0.445556in}{0.499444in}}{\pgfqpoint{3.487500in}{1.155000in}}%
\pgfusepath{clip}%
\pgfsetbuttcap%
\pgfsetmiterjoin%
\pgfsetlinewidth{1.003750pt}%
\definecolor{currentstroke}{rgb}{0.000000,0.000000,0.000000}%
\pgfsetstrokecolor{currentstroke}%
\pgfsetdash{}{0pt}%
\pgfpathmoveto{\pgfqpoint{2.918510in}{0.499444in}}%
\pgfpathlineto{\pgfqpoint{2.981919in}{0.499444in}}%
\pgfpathlineto{\pgfqpoint{2.981919in}{0.976754in}}%
\pgfpathlineto{\pgfqpoint{2.918510in}{0.976754in}}%
\pgfpathlineto{\pgfqpoint{2.918510in}{0.499444in}}%
\pgfpathclose%
\pgfusepath{stroke}%
\end{pgfscope}%
\begin{pgfscope}%
\pgfpathrectangle{\pgfqpoint{0.445556in}{0.499444in}}{\pgfqpoint{3.487500in}{1.155000in}}%
\pgfusepath{clip}%
\pgfsetbuttcap%
\pgfsetmiterjoin%
\pgfsetlinewidth{1.003750pt}%
\definecolor{currentstroke}{rgb}{0.000000,0.000000,0.000000}%
\pgfsetstrokecolor{currentstroke}%
\pgfsetdash{}{0pt}%
\pgfpathmoveto{\pgfqpoint{3.077033in}{0.499444in}}%
\pgfpathlineto{\pgfqpoint{3.140442in}{0.499444in}}%
\pgfpathlineto{\pgfqpoint{3.140442in}{0.870598in}}%
\pgfpathlineto{\pgfqpoint{3.077033in}{0.870598in}}%
\pgfpathlineto{\pgfqpoint{3.077033in}{0.499444in}}%
\pgfpathclose%
\pgfusepath{stroke}%
\end{pgfscope}%
\begin{pgfscope}%
\pgfpathrectangle{\pgfqpoint{0.445556in}{0.499444in}}{\pgfqpoint{3.487500in}{1.155000in}}%
\pgfusepath{clip}%
\pgfsetbuttcap%
\pgfsetmiterjoin%
\pgfsetlinewidth{1.003750pt}%
\definecolor{currentstroke}{rgb}{0.000000,0.000000,0.000000}%
\pgfsetstrokecolor{currentstroke}%
\pgfsetdash{}{0pt}%
\pgfpathmoveto{\pgfqpoint{3.235556in}{0.499444in}}%
\pgfpathlineto{\pgfqpoint{3.298965in}{0.499444in}}%
\pgfpathlineto{\pgfqpoint{3.298965in}{0.770406in}}%
\pgfpathlineto{\pgfqpoint{3.235556in}{0.770406in}}%
\pgfpathlineto{\pgfqpoint{3.235556in}{0.499444in}}%
\pgfpathclose%
\pgfusepath{stroke}%
\end{pgfscope}%
\begin{pgfscope}%
\pgfpathrectangle{\pgfqpoint{0.445556in}{0.499444in}}{\pgfqpoint{3.487500in}{1.155000in}}%
\pgfusepath{clip}%
\pgfsetbuttcap%
\pgfsetmiterjoin%
\pgfsetlinewidth{1.003750pt}%
\definecolor{currentstroke}{rgb}{0.000000,0.000000,0.000000}%
\pgfsetstrokecolor{currentstroke}%
\pgfsetdash{}{0pt}%
\pgfpathmoveto{\pgfqpoint{3.394078in}{0.499444in}}%
\pgfpathlineto{\pgfqpoint{3.457488in}{0.499444in}}%
\pgfpathlineto{\pgfqpoint{3.457488in}{0.679683in}}%
\pgfpathlineto{\pgfqpoint{3.394078in}{0.679683in}}%
\pgfpathlineto{\pgfqpoint{3.394078in}{0.499444in}}%
\pgfpathclose%
\pgfusepath{stroke}%
\end{pgfscope}%
\begin{pgfscope}%
\pgfpathrectangle{\pgfqpoint{0.445556in}{0.499444in}}{\pgfqpoint{3.487500in}{1.155000in}}%
\pgfusepath{clip}%
\pgfsetbuttcap%
\pgfsetmiterjoin%
\pgfsetlinewidth{1.003750pt}%
\definecolor{currentstroke}{rgb}{0.000000,0.000000,0.000000}%
\pgfsetstrokecolor{currentstroke}%
\pgfsetdash{}{0pt}%
\pgfpathmoveto{\pgfqpoint{3.552601in}{0.499444in}}%
\pgfpathlineto{\pgfqpoint{3.616010in}{0.499444in}}%
\pgfpathlineto{\pgfqpoint{3.616010in}{0.606645in}}%
\pgfpathlineto{\pgfqpoint{3.552601in}{0.606645in}}%
\pgfpathlineto{\pgfqpoint{3.552601in}{0.499444in}}%
\pgfpathclose%
\pgfusepath{stroke}%
\end{pgfscope}%
\begin{pgfscope}%
\pgfpathrectangle{\pgfqpoint{0.445556in}{0.499444in}}{\pgfqpoint{3.487500in}{1.155000in}}%
\pgfusepath{clip}%
\pgfsetbuttcap%
\pgfsetmiterjoin%
\pgfsetlinewidth{1.003750pt}%
\definecolor{currentstroke}{rgb}{0.000000,0.000000,0.000000}%
\pgfsetstrokecolor{currentstroke}%
\pgfsetdash{}{0pt}%
\pgfpathmoveto{\pgfqpoint{3.711124in}{0.499444in}}%
\pgfpathlineto{\pgfqpoint{3.774533in}{0.499444in}}%
\pgfpathlineto{\pgfqpoint{3.774533in}{0.530776in}}%
\pgfpathlineto{\pgfqpoint{3.711124in}{0.530776in}}%
\pgfpathlineto{\pgfqpoint{3.711124in}{0.499444in}}%
\pgfpathclose%
\pgfusepath{stroke}%
\end{pgfscope}%
\begin{pgfscope}%
\pgfpathrectangle{\pgfqpoint{0.445556in}{0.499444in}}{\pgfqpoint{3.487500in}{1.155000in}}%
\pgfusepath{clip}%
\pgfsetbuttcap%
\pgfsetmiterjoin%
\definecolor{currentfill}{rgb}{0.000000,0.000000,0.000000}%
\pgfsetfillcolor{currentfill}%
\pgfsetlinewidth{0.000000pt}%
\definecolor{currentstroke}{rgb}{0.000000,0.000000,0.000000}%
\pgfsetstrokecolor{currentstroke}%
\pgfsetstrokeopacity{0.000000}%
\pgfsetdash{}{0pt}%
\pgfpathmoveto{\pgfqpoint{0.604078in}{0.499444in}}%
\pgfpathlineto{\pgfqpoint{0.667488in}{0.499444in}}%
\pgfpathlineto{\pgfqpoint{0.667488in}{0.499444in}}%
\pgfpathlineto{\pgfqpoint{0.604078in}{0.499444in}}%
\pgfpathlineto{\pgfqpoint{0.604078in}{0.499444in}}%
\pgfpathclose%
\pgfusepath{fill}%
\end{pgfscope}%
\begin{pgfscope}%
\pgfpathrectangle{\pgfqpoint{0.445556in}{0.499444in}}{\pgfqpoint{3.487500in}{1.155000in}}%
\pgfusepath{clip}%
\pgfsetbuttcap%
\pgfsetmiterjoin%
\definecolor{currentfill}{rgb}{0.000000,0.000000,0.000000}%
\pgfsetfillcolor{currentfill}%
\pgfsetlinewidth{0.000000pt}%
\definecolor{currentstroke}{rgb}{0.000000,0.000000,0.000000}%
\pgfsetstrokecolor{currentstroke}%
\pgfsetstrokeopacity{0.000000}%
\pgfsetdash{}{0pt}%
\pgfpathmoveto{\pgfqpoint{0.762601in}{0.499444in}}%
\pgfpathlineto{\pgfqpoint{0.826010in}{0.499444in}}%
\pgfpathlineto{\pgfqpoint{0.826010in}{0.505084in}}%
\pgfpathlineto{\pgfqpoint{0.762601in}{0.505084in}}%
\pgfpathlineto{\pgfqpoint{0.762601in}{0.499444in}}%
\pgfpathclose%
\pgfusepath{fill}%
\end{pgfscope}%
\begin{pgfscope}%
\pgfpathrectangle{\pgfqpoint{0.445556in}{0.499444in}}{\pgfqpoint{3.487500in}{1.155000in}}%
\pgfusepath{clip}%
\pgfsetbuttcap%
\pgfsetmiterjoin%
\definecolor{currentfill}{rgb}{0.000000,0.000000,0.000000}%
\pgfsetfillcolor{currentfill}%
\pgfsetlinewidth{0.000000pt}%
\definecolor{currentstroke}{rgb}{0.000000,0.000000,0.000000}%
\pgfsetstrokecolor{currentstroke}%
\pgfsetstrokeopacity{0.000000}%
\pgfsetdash{}{0pt}%
\pgfpathmoveto{\pgfqpoint{0.921124in}{0.499444in}}%
\pgfpathlineto{\pgfqpoint{0.984533in}{0.499444in}}%
\pgfpathlineto{\pgfqpoint{0.984533in}{0.517106in}}%
\pgfpathlineto{\pgfqpoint{0.921124in}{0.517106in}}%
\pgfpathlineto{\pgfqpoint{0.921124in}{0.499444in}}%
\pgfpathclose%
\pgfusepath{fill}%
\end{pgfscope}%
\begin{pgfscope}%
\pgfpathrectangle{\pgfqpoint{0.445556in}{0.499444in}}{\pgfqpoint{3.487500in}{1.155000in}}%
\pgfusepath{clip}%
\pgfsetbuttcap%
\pgfsetmiterjoin%
\definecolor{currentfill}{rgb}{0.000000,0.000000,0.000000}%
\pgfsetfillcolor{currentfill}%
\pgfsetlinewidth{0.000000pt}%
\definecolor{currentstroke}{rgb}{0.000000,0.000000,0.000000}%
\pgfsetstrokecolor{currentstroke}%
\pgfsetstrokeopacity{0.000000}%
\pgfsetdash{}{0pt}%
\pgfpathmoveto{\pgfqpoint{1.079647in}{0.499444in}}%
\pgfpathlineto{\pgfqpoint{1.143056in}{0.499444in}}%
\pgfpathlineto{\pgfqpoint{1.143056in}{0.532377in}}%
\pgfpathlineto{\pgfqpoint{1.079647in}{0.532377in}}%
\pgfpathlineto{\pgfqpoint{1.079647in}{0.499444in}}%
\pgfpathclose%
\pgfusepath{fill}%
\end{pgfscope}%
\begin{pgfscope}%
\pgfpathrectangle{\pgfqpoint{0.445556in}{0.499444in}}{\pgfqpoint{3.487500in}{1.155000in}}%
\pgfusepath{clip}%
\pgfsetbuttcap%
\pgfsetmiterjoin%
\definecolor{currentfill}{rgb}{0.000000,0.000000,0.000000}%
\pgfsetfillcolor{currentfill}%
\pgfsetlinewidth{0.000000pt}%
\definecolor{currentstroke}{rgb}{0.000000,0.000000,0.000000}%
\pgfsetstrokecolor{currentstroke}%
\pgfsetstrokeopacity{0.000000}%
\pgfsetdash{}{0pt}%
\pgfpathmoveto{\pgfqpoint{1.238169in}{0.499444in}}%
\pgfpathlineto{\pgfqpoint{1.301578in}{0.499444in}}%
\pgfpathlineto{\pgfqpoint{1.301578in}{0.545189in}}%
\pgfpathlineto{\pgfqpoint{1.238169in}{0.545189in}}%
\pgfpathlineto{\pgfqpoint{1.238169in}{0.499444in}}%
\pgfpathclose%
\pgfusepath{fill}%
\end{pgfscope}%
\begin{pgfscope}%
\pgfpathrectangle{\pgfqpoint{0.445556in}{0.499444in}}{\pgfqpoint{3.487500in}{1.155000in}}%
\pgfusepath{clip}%
\pgfsetbuttcap%
\pgfsetmiterjoin%
\definecolor{currentfill}{rgb}{0.000000,0.000000,0.000000}%
\pgfsetfillcolor{currentfill}%
\pgfsetlinewidth{0.000000pt}%
\definecolor{currentstroke}{rgb}{0.000000,0.000000,0.000000}%
\pgfsetstrokecolor{currentstroke}%
\pgfsetstrokeopacity{0.000000}%
\pgfsetdash{}{0pt}%
\pgfpathmoveto{\pgfqpoint{1.396692in}{0.499444in}}%
\pgfpathlineto{\pgfqpoint{1.460101in}{0.499444in}}%
\pgfpathlineto{\pgfqpoint{1.460101in}{0.558882in}}%
\pgfpathlineto{\pgfqpoint{1.396692in}{0.558882in}}%
\pgfpathlineto{\pgfqpoint{1.396692in}{0.499444in}}%
\pgfpathclose%
\pgfusepath{fill}%
\end{pgfscope}%
\begin{pgfscope}%
\pgfpathrectangle{\pgfqpoint{0.445556in}{0.499444in}}{\pgfqpoint{3.487500in}{1.155000in}}%
\pgfusepath{clip}%
\pgfsetbuttcap%
\pgfsetmiterjoin%
\definecolor{currentfill}{rgb}{0.000000,0.000000,0.000000}%
\pgfsetfillcolor{currentfill}%
\pgfsetlinewidth{0.000000pt}%
\definecolor{currentstroke}{rgb}{0.000000,0.000000,0.000000}%
\pgfsetstrokecolor{currentstroke}%
\pgfsetstrokeopacity{0.000000}%
\pgfsetdash{}{0pt}%
\pgfpathmoveto{\pgfqpoint{1.555215in}{0.499444in}}%
\pgfpathlineto{\pgfqpoint{1.618624in}{0.499444in}}%
\pgfpathlineto{\pgfqpoint{1.618624in}{0.577193in}}%
\pgfpathlineto{\pgfqpoint{1.555215in}{0.577193in}}%
\pgfpathlineto{\pgfqpoint{1.555215in}{0.499444in}}%
\pgfpathclose%
\pgfusepath{fill}%
\end{pgfscope}%
\begin{pgfscope}%
\pgfpathrectangle{\pgfqpoint{0.445556in}{0.499444in}}{\pgfqpoint{3.487500in}{1.155000in}}%
\pgfusepath{clip}%
\pgfsetbuttcap%
\pgfsetmiterjoin%
\definecolor{currentfill}{rgb}{0.000000,0.000000,0.000000}%
\pgfsetfillcolor{currentfill}%
\pgfsetlinewidth{0.000000pt}%
\definecolor{currentstroke}{rgb}{0.000000,0.000000,0.000000}%
\pgfsetstrokecolor{currentstroke}%
\pgfsetstrokeopacity{0.000000}%
\pgfsetdash{}{0pt}%
\pgfpathmoveto{\pgfqpoint{1.713738in}{0.499444in}}%
\pgfpathlineto{\pgfqpoint{1.777147in}{0.499444in}}%
\pgfpathlineto{\pgfqpoint{1.777147in}{0.594670in}}%
\pgfpathlineto{\pgfqpoint{1.713738in}{0.594670in}}%
\pgfpathlineto{\pgfqpoint{1.713738in}{0.499444in}}%
\pgfpathclose%
\pgfusepath{fill}%
\end{pgfscope}%
\begin{pgfscope}%
\pgfpathrectangle{\pgfqpoint{0.445556in}{0.499444in}}{\pgfqpoint{3.487500in}{1.155000in}}%
\pgfusepath{clip}%
\pgfsetbuttcap%
\pgfsetmiterjoin%
\definecolor{currentfill}{rgb}{0.000000,0.000000,0.000000}%
\pgfsetfillcolor{currentfill}%
\pgfsetlinewidth{0.000000pt}%
\definecolor{currentstroke}{rgb}{0.000000,0.000000,0.000000}%
\pgfsetstrokecolor{currentstroke}%
\pgfsetstrokeopacity{0.000000}%
\pgfsetdash{}{0pt}%
\pgfpathmoveto{\pgfqpoint{1.872260in}{0.499444in}}%
\pgfpathlineto{\pgfqpoint{1.935669in}{0.499444in}}%
\pgfpathlineto{\pgfqpoint{1.935669in}{0.607991in}}%
\pgfpathlineto{\pgfqpoint{1.872260in}{0.607991in}}%
\pgfpathlineto{\pgfqpoint{1.872260in}{0.499444in}}%
\pgfpathclose%
\pgfusepath{fill}%
\end{pgfscope}%
\begin{pgfscope}%
\pgfpathrectangle{\pgfqpoint{0.445556in}{0.499444in}}{\pgfqpoint{3.487500in}{1.155000in}}%
\pgfusepath{clip}%
\pgfsetbuttcap%
\pgfsetmiterjoin%
\definecolor{currentfill}{rgb}{0.000000,0.000000,0.000000}%
\pgfsetfillcolor{currentfill}%
\pgfsetlinewidth{0.000000pt}%
\definecolor{currentstroke}{rgb}{0.000000,0.000000,0.000000}%
\pgfsetstrokecolor{currentstroke}%
\pgfsetstrokeopacity{0.000000}%
\pgfsetdash{}{0pt}%
\pgfpathmoveto{\pgfqpoint{2.030783in}{0.499444in}}%
\pgfpathlineto{\pgfqpoint{2.094192in}{0.499444in}}%
\pgfpathlineto{\pgfqpoint{2.094192in}{0.628740in}}%
\pgfpathlineto{\pgfqpoint{2.030783in}{0.628740in}}%
\pgfpathlineto{\pgfqpoint{2.030783in}{0.499444in}}%
\pgfpathclose%
\pgfusepath{fill}%
\end{pgfscope}%
\begin{pgfscope}%
\pgfpathrectangle{\pgfqpoint{0.445556in}{0.499444in}}{\pgfqpoint{3.487500in}{1.155000in}}%
\pgfusepath{clip}%
\pgfsetbuttcap%
\pgfsetmiterjoin%
\definecolor{currentfill}{rgb}{0.000000,0.000000,0.000000}%
\pgfsetfillcolor{currentfill}%
\pgfsetlinewidth{0.000000pt}%
\definecolor{currentstroke}{rgb}{0.000000,0.000000,0.000000}%
\pgfsetstrokecolor{currentstroke}%
\pgfsetstrokeopacity{0.000000}%
\pgfsetdash{}{0pt}%
\pgfpathmoveto{\pgfqpoint{2.189306in}{0.499444in}}%
\pgfpathlineto{\pgfqpoint{2.252715in}{0.499444in}}%
\pgfpathlineto{\pgfqpoint{2.252715in}{0.645613in}}%
\pgfpathlineto{\pgfqpoint{2.189306in}{0.645613in}}%
\pgfpathlineto{\pgfqpoint{2.189306in}{0.499444in}}%
\pgfpathclose%
\pgfusepath{fill}%
\end{pgfscope}%
\begin{pgfscope}%
\pgfpathrectangle{\pgfqpoint{0.445556in}{0.499444in}}{\pgfqpoint{3.487500in}{1.155000in}}%
\pgfusepath{clip}%
\pgfsetbuttcap%
\pgfsetmiterjoin%
\definecolor{currentfill}{rgb}{0.000000,0.000000,0.000000}%
\pgfsetfillcolor{currentfill}%
\pgfsetlinewidth{0.000000pt}%
\definecolor{currentstroke}{rgb}{0.000000,0.000000,0.000000}%
\pgfsetstrokecolor{currentstroke}%
\pgfsetstrokeopacity{0.000000}%
\pgfsetdash{}{0pt}%
\pgfpathmoveto{\pgfqpoint{2.347828in}{0.499444in}}%
\pgfpathlineto{\pgfqpoint{2.411238in}{0.499444in}}%
\pgfpathlineto{\pgfqpoint{2.411238in}{0.665154in}}%
\pgfpathlineto{\pgfqpoint{2.347828in}{0.665154in}}%
\pgfpathlineto{\pgfqpoint{2.347828in}{0.499444in}}%
\pgfpathclose%
\pgfusepath{fill}%
\end{pgfscope}%
\begin{pgfscope}%
\pgfpathrectangle{\pgfqpoint{0.445556in}{0.499444in}}{\pgfqpoint{3.487500in}{1.155000in}}%
\pgfusepath{clip}%
\pgfsetbuttcap%
\pgfsetmiterjoin%
\definecolor{currentfill}{rgb}{0.000000,0.000000,0.000000}%
\pgfsetfillcolor{currentfill}%
\pgfsetlinewidth{0.000000pt}%
\definecolor{currentstroke}{rgb}{0.000000,0.000000,0.000000}%
\pgfsetstrokecolor{currentstroke}%
\pgfsetstrokeopacity{0.000000}%
\pgfsetdash{}{0pt}%
\pgfpathmoveto{\pgfqpoint{2.506351in}{0.499444in}}%
\pgfpathlineto{\pgfqpoint{2.569760in}{0.499444in}}%
\pgfpathlineto{\pgfqpoint{2.569760in}{0.680008in}}%
\pgfpathlineto{\pgfqpoint{2.506351in}{0.680008in}}%
\pgfpathlineto{\pgfqpoint{2.506351in}{0.499444in}}%
\pgfpathclose%
\pgfusepath{fill}%
\end{pgfscope}%
\begin{pgfscope}%
\pgfpathrectangle{\pgfqpoint{0.445556in}{0.499444in}}{\pgfqpoint{3.487500in}{1.155000in}}%
\pgfusepath{clip}%
\pgfsetbuttcap%
\pgfsetmiterjoin%
\definecolor{currentfill}{rgb}{0.000000,0.000000,0.000000}%
\pgfsetfillcolor{currentfill}%
\pgfsetlinewidth{0.000000pt}%
\definecolor{currentstroke}{rgb}{0.000000,0.000000,0.000000}%
\pgfsetstrokecolor{currentstroke}%
\pgfsetstrokeopacity{0.000000}%
\pgfsetdash{}{0pt}%
\pgfpathmoveto{\pgfqpoint{2.664874in}{0.499444in}}%
\pgfpathlineto{\pgfqpoint{2.728283in}{0.499444in}}%
\pgfpathlineto{\pgfqpoint{2.728283in}{0.690429in}}%
\pgfpathlineto{\pgfqpoint{2.664874in}{0.690429in}}%
\pgfpathlineto{\pgfqpoint{2.664874in}{0.499444in}}%
\pgfpathclose%
\pgfusepath{fill}%
\end{pgfscope}%
\begin{pgfscope}%
\pgfpathrectangle{\pgfqpoint{0.445556in}{0.499444in}}{\pgfqpoint{3.487500in}{1.155000in}}%
\pgfusepath{clip}%
\pgfsetbuttcap%
\pgfsetmiterjoin%
\definecolor{currentfill}{rgb}{0.000000,0.000000,0.000000}%
\pgfsetfillcolor{currentfill}%
\pgfsetlinewidth{0.000000pt}%
\definecolor{currentstroke}{rgb}{0.000000,0.000000,0.000000}%
\pgfsetstrokecolor{currentstroke}%
\pgfsetstrokeopacity{0.000000}%
\pgfsetdash{}{0pt}%
\pgfpathmoveto{\pgfqpoint{2.823397in}{0.499444in}}%
\pgfpathlineto{\pgfqpoint{2.886806in}{0.499444in}}%
\pgfpathlineto{\pgfqpoint{2.886806in}{0.696509in}}%
\pgfpathlineto{\pgfqpoint{2.823397in}{0.696509in}}%
\pgfpathlineto{\pgfqpoint{2.823397in}{0.499444in}}%
\pgfpathclose%
\pgfusepath{fill}%
\end{pgfscope}%
\begin{pgfscope}%
\pgfpathrectangle{\pgfqpoint{0.445556in}{0.499444in}}{\pgfqpoint{3.487500in}{1.155000in}}%
\pgfusepath{clip}%
\pgfsetbuttcap%
\pgfsetmiterjoin%
\definecolor{currentfill}{rgb}{0.000000,0.000000,0.000000}%
\pgfsetfillcolor{currentfill}%
\pgfsetlinewidth{0.000000pt}%
\definecolor{currentstroke}{rgb}{0.000000,0.000000,0.000000}%
\pgfsetstrokecolor{currentstroke}%
\pgfsetstrokeopacity{0.000000}%
\pgfsetdash{}{0pt}%
\pgfpathmoveto{\pgfqpoint{2.981919in}{0.499444in}}%
\pgfpathlineto{\pgfqpoint{3.045328in}{0.499444in}}%
\pgfpathlineto{\pgfqpoint{3.045328in}{0.704377in}}%
\pgfpathlineto{\pgfqpoint{2.981919in}{0.704377in}}%
\pgfpathlineto{\pgfqpoint{2.981919in}{0.499444in}}%
\pgfpathclose%
\pgfusepath{fill}%
\end{pgfscope}%
\begin{pgfscope}%
\pgfpathrectangle{\pgfqpoint{0.445556in}{0.499444in}}{\pgfqpoint{3.487500in}{1.155000in}}%
\pgfusepath{clip}%
\pgfsetbuttcap%
\pgfsetmiterjoin%
\definecolor{currentfill}{rgb}{0.000000,0.000000,0.000000}%
\pgfsetfillcolor{currentfill}%
\pgfsetlinewidth{0.000000pt}%
\definecolor{currentstroke}{rgb}{0.000000,0.000000,0.000000}%
\pgfsetstrokecolor{currentstroke}%
\pgfsetstrokeopacity{0.000000}%
\pgfsetdash{}{0pt}%
\pgfpathmoveto{\pgfqpoint{3.140442in}{0.499444in}}%
\pgfpathlineto{\pgfqpoint{3.203851in}{0.499444in}}%
\pgfpathlineto{\pgfqpoint{3.203851in}{0.701708in}}%
\pgfpathlineto{\pgfqpoint{3.140442in}{0.701708in}}%
\pgfpathlineto{\pgfqpoint{3.140442in}{0.499444in}}%
\pgfpathclose%
\pgfusepath{fill}%
\end{pgfscope}%
\begin{pgfscope}%
\pgfpathrectangle{\pgfqpoint{0.445556in}{0.499444in}}{\pgfqpoint{3.487500in}{1.155000in}}%
\pgfusepath{clip}%
\pgfsetbuttcap%
\pgfsetmiterjoin%
\definecolor{currentfill}{rgb}{0.000000,0.000000,0.000000}%
\pgfsetfillcolor{currentfill}%
\pgfsetlinewidth{0.000000pt}%
\definecolor{currentstroke}{rgb}{0.000000,0.000000,0.000000}%
\pgfsetstrokecolor{currentstroke}%
\pgfsetstrokeopacity{0.000000}%
\pgfsetdash{}{0pt}%
\pgfpathmoveto{\pgfqpoint{3.298965in}{0.499444in}}%
\pgfpathlineto{\pgfqpoint{3.362374in}{0.499444in}}%
\pgfpathlineto{\pgfqpoint{3.362374in}{0.696370in}}%
\pgfpathlineto{\pgfqpoint{3.298965in}{0.696370in}}%
\pgfpathlineto{\pgfqpoint{3.298965in}{0.499444in}}%
\pgfpathclose%
\pgfusepath{fill}%
\end{pgfscope}%
\begin{pgfscope}%
\pgfpathrectangle{\pgfqpoint{0.445556in}{0.499444in}}{\pgfqpoint{3.487500in}{1.155000in}}%
\pgfusepath{clip}%
\pgfsetbuttcap%
\pgfsetmiterjoin%
\definecolor{currentfill}{rgb}{0.000000,0.000000,0.000000}%
\pgfsetfillcolor{currentfill}%
\pgfsetlinewidth{0.000000pt}%
\definecolor{currentstroke}{rgb}{0.000000,0.000000,0.000000}%
\pgfsetstrokecolor{currentstroke}%
\pgfsetstrokeopacity{0.000000}%
\pgfsetdash{}{0pt}%
\pgfpathmoveto{\pgfqpoint{3.457488in}{0.499444in}}%
\pgfpathlineto{\pgfqpoint{3.520897in}{0.499444in}}%
\pgfpathlineto{\pgfqpoint{3.520897in}{0.684302in}}%
\pgfpathlineto{\pgfqpoint{3.457488in}{0.684302in}}%
\pgfpathlineto{\pgfqpoint{3.457488in}{0.499444in}}%
\pgfpathclose%
\pgfusepath{fill}%
\end{pgfscope}%
\begin{pgfscope}%
\pgfpathrectangle{\pgfqpoint{0.445556in}{0.499444in}}{\pgfqpoint{3.487500in}{1.155000in}}%
\pgfusepath{clip}%
\pgfsetbuttcap%
\pgfsetmiterjoin%
\definecolor{currentfill}{rgb}{0.000000,0.000000,0.000000}%
\pgfsetfillcolor{currentfill}%
\pgfsetlinewidth{0.000000pt}%
\definecolor{currentstroke}{rgb}{0.000000,0.000000,0.000000}%
\pgfsetstrokecolor{currentstroke}%
\pgfsetstrokeopacity{0.000000}%
\pgfsetdash{}{0pt}%
\pgfpathmoveto{\pgfqpoint{3.616010in}{0.499444in}}%
\pgfpathlineto{\pgfqpoint{3.679419in}{0.499444in}}%
\pgfpathlineto{\pgfqpoint{3.679419in}{0.675923in}}%
\pgfpathlineto{\pgfqpoint{3.616010in}{0.675923in}}%
\pgfpathlineto{\pgfqpoint{3.616010in}{0.499444in}}%
\pgfpathclose%
\pgfusepath{fill}%
\end{pgfscope}%
\begin{pgfscope}%
\pgfpathrectangle{\pgfqpoint{0.445556in}{0.499444in}}{\pgfqpoint{3.487500in}{1.155000in}}%
\pgfusepath{clip}%
\pgfsetbuttcap%
\pgfsetmiterjoin%
\definecolor{currentfill}{rgb}{0.000000,0.000000,0.000000}%
\pgfsetfillcolor{currentfill}%
\pgfsetlinewidth{0.000000pt}%
\definecolor{currentstroke}{rgb}{0.000000,0.000000,0.000000}%
\pgfsetstrokecolor{currentstroke}%
\pgfsetstrokeopacity{0.000000}%
\pgfsetdash{}{0pt}%
\pgfpathmoveto{\pgfqpoint{3.774533in}{0.499444in}}%
\pgfpathlineto{\pgfqpoint{3.837942in}{0.499444in}}%
\pgfpathlineto{\pgfqpoint{3.837942in}{0.586779in}}%
\pgfpathlineto{\pgfqpoint{3.774533in}{0.586779in}}%
\pgfpathlineto{\pgfqpoint{3.774533in}{0.499444in}}%
\pgfpathclose%
\pgfusepath{fill}%
\end{pgfscope}%
\begin{pgfscope}%
\pgfsetbuttcap%
\pgfsetroundjoin%
\definecolor{currentfill}{rgb}{0.000000,0.000000,0.000000}%
\pgfsetfillcolor{currentfill}%
\pgfsetlinewidth{0.803000pt}%
\definecolor{currentstroke}{rgb}{0.000000,0.000000,0.000000}%
\pgfsetstrokecolor{currentstroke}%
\pgfsetdash{}{0pt}%
\pgfsys@defobject{currentmarker}{\pgfqpoint{0.000000in}{-0.048611in}}{\pgfqpoint{0.000000in}{0.000000in}}{%
\pgfpathmoveto{\pgfqpoint{0.000000in}{0.000000in}}%
\pgfpathlineto{\pgfqpoint{0.000000in}{-0.048611in}}%
\pgfusepath{stroke,fill}%
}%
\begin{pgfscope}%
\pgfsys@transformshift{0.445556in}{0.499444in}%
\pgfsys@useobject{currentmarker}{}%
\end{pgfscope}%
\end{pgfscope}%
\begin{pgfscope}%
\pgfsetbuttcap%
\pgfsetroundjoin%
\definecolor{currentfill}{rgb}{0.000000,0.000000,0.000000}%
\pgfsetfillcolor{currentfill}%
\pgfsetlinewidth{0.803000pt}%
\definecolor{currentstroke}{rgb}{0.000000,0.000000,0.000000}%
\pgfsetstrokecolor{currentstroke}%
\pgfsetdash{}{0pt}%
\pgfsys@defobject{currentmarker}{\pgfqpoint{0.000000in}{-0.048611in}}{\pgfqpoint{0.000000in}{0.000000in}}{%
\pgfpathmoveto{\pgfqpoint{0.000000in}{0.000000in}}%
\pgfpathlineto{\pgfqpoint{0.000000in}{-0.048611in}}%
\pgfusepath{stroke,fill}%
}%
\begin{pgfscope}%
\pgfsys@transformshift{0.604078in}{0.499444in}%
\pgfsys@useobject{currentmarker}{}%
\end{pgfscope}%
\end{pgfscope}%
\begin{pgfscope}%
\definecolor{textcolor}{rgb}{0.000000,0.000000,0.000000}%
\pgfsetstrokecolor{textcolor}%
\pgfsetfillcolor{textcolor}%
\pgftext[x=0.604078in,y=0.402222in,,top]{\color{textcolor}\rmfamily\fontsize{10.000000}{12.000000}\selectfont 0.0}%
\end{pgfscope}%
\begin{pgfscope}%
\pgfsetbuttcap%
\pgfsetroundjoin%
\definecolor{currentfill}{rgb}{0.000000,0.000000,0.000000}%
\pgfsetfillcolor{currentfill}%
\pgfsetlinewidth{0.803000pt}%
\definecolor{currentstroke}{rgb}{0.000000,0.000000,0.000000}%
\pgfsetstrokecolor{currentstroke}%
\pgfsetdash{}{0pt}%
\pgfsys@defobject{currentmarker}{\pgfqpoint{0.000000in}{-0.048611in}}{\pgfqpoint{0.000000in}{0.000000in}}{%
\pgfpathmoveto{\pgfqpoint{0.000000in}{0.000000in}}%
\pgfpathlineto{\pgfqpoint{0.000000in}{-0.048611in}}%
\pgfusepath{stroke,fill}%
}%
\begin{pgfscope}%
\pgfsys@transformshift{0.762601in}{0.499444in}%
\pgfsys@useobject{currentmarker}{}%
\end{pgfscope}%
\end{pgfscope}%
\begin{pgfscope}%
\pgfsetbuttcap%
\pgfsetroundjoin%
\definecolor{currentfill}{rgb}{0.000000,0.000000,0.000000}%
\pgfsetfillcolor{currentfill}%
\pgfsetlinewidth{0.803000pt}%
\definecolor{currentstroke}{rgb}{0.000000,0.000000,0.000000}%
\pgfsetstrokecolor{currentstroke}%
\pgfsetdash{}{0pt}%
\pgfsys@defobject{currentmarker}{\pgfqpoint{0.000000in}{-0.048611in}}{\pgfqpoint{0.000000in}{0.000000in}}{%
\pgfpathmoveto{\pgfqpoint{0.000000in}{0.000000in}}%
\pgfpathlineto{\pgfqpoint{0.000000in}{-0.048611in}}%
\pgfusepath{stroke,fill}%
}%
\begin{pgfscope}%
\pgfsys@transformshift{0.921124in}{0.499444in}%
\pgfsys@useobject{currentmarker}{}%
\end{pgfscope}%
\end{pgfscope}%
\begin{pgfscope}%
\definecolor{textcolor}{rgb}{0.000000,0.000000,0.000000}%
\pgfsetstrokecolor{textcolor}%
\pgfsetfillcolor{textcolor}%
\pgftext[x=0.921124in,y=0.402222in,,top]{\color{textcolor}\rmfamily\fontsize{10.000000}{12.000000}\selectfont 0.1}%
\end{pgfscope}%
\begin{pgfscope}%
\pgfsetbuttcap%
\pgfsetroundjoin%
\definecolor{currentfill}{rgb}{0.000000,0.000000,0.000000}%
\pgfsetfillcolor{currentfill}%
\pgfsetlinewidth{0.803000pt}%
\definecolor{currentstroke}{rgb}{0.000000,0.000000,0.000000}%
\pgfsetstrokecolor{currentstroke}%
\pgfsetdash{}{0pt}%
\pgfsys@defobject{currentmarker}{\pgfqpoint{0.000000in}{-0.048611in}}{\pgfqpoint{0.000000in}{0.000000in}}{%
\pgfpathmoveto{\pgfqpoint{0.000000in}{0.000000in}}%
\pgfpathlineto{\pgfqpoint{0.000000in}{-0.048611in}}%
\pgfusepath{stroke,fill}%
}%
\begin{pgfscope}%
\pgfsys@transformshift{1.079647in}{0.499444in}%
\pgfsys@useobject{currentmarker}{}%
\end{pgfscope}%
\end{pgfscope}%
\begin{pgfscope}%
\pgfsetbuttcap%
\pgfsetroundjoin%
\definecolor{currentfill}{rgb}{0.000000,0.000000,0.000000}%
\pgfsetfillcolor{currentfill}%
\pgfsetlinewidth{0.803000pt}%
\definecolor{currentstroke}{rgb}{0.000000,0.000000,0.000000}%
\pgfsetstrokecolor{currentstroke}%
\pgfsetdash{}{0pt}%
\pgfsys@defobject{currentmarker}{\pgfqpoint{0.000000in}{-0.048611in}}{\pgfqpoint{0.000000in}{0.000000in}}{%
\pgfpathmoveto{\pgfqpoint{0.000000in}{0.000000in}}%
\pgfpathlineto{\pgfqpoint{0.000000in}{-0.048611in}}%
\pgfusepath{stroke,fill}%
}%
\begin{pgfscope}%
\pgfsys@transformshift{1.238169in}{0.499444in}%
\pgfsys@useobject{currentmarker}{}%
\end{pgfscope}%
\end{pgfscope}%
\begin{pgfscope}%
\definecolor{textcolor}{rgb}{0.000000,0.000000,0.000000}%
\pgfsetstrokecolor{textcolor}%
\pgfsetfillcolor{textcolor}%
\pgftext[x=1.238169in,y=0.402222in,,top]{\color{textcolor}\rmfamily\fontsize{10.000000}{12.000000}\selectfont 0.2}%
\end{pgfscope}%
\begin{pgfscope}%
\pgfsetbuttcap%
\pgfsetroundjoin%
\definecolor{currentfill}{rgb}{0.000000,0.000000,0.000000}%
\pgfsetfillcolor{currentfill}%
\pgfsetlinewidth{0.803000pt}%
\definecolor{currentstroke}{rgb}{0.000000,0.000000,0.000000}%
\pgfsetstrokecolor{currentstroke}%
\pgfsetdash{}{0pt}%
\pgfsys@defobject{currentmarker}{\pgfqpoint{0.000000in}{-0.048611in}}{\pgfqpoint{0.000000in}{0.000000in}}{%
\pgfpathmoveto{\pgfqpoint{0.000000in}{0.000000in}}%
\pgfpathlineto{\pgfqpoint{0.000000in}{-0.048611in}}%
\pgfusepath{stroke,fill}%
}%
\begin{pgfscope}%
\pgfsys@transformshift{1.396692in}{0.499444in}%
\pgfsys@useobject{currentmarker}{}%
\end{pgfscope}%
\end{pgfscope}%
\begin{pgfscope}%
\pgfsetbuttcap%
\pgfsetroundjoin%
\definecolor{currentfill}{rgb}{0.000000,0.000000,0.000000}%
\pgfsetfillcolor{currentfill}%
\pgfsetlinewidth{0.803000pt}%
\definecolor{currentstroke}{rgb}{0.000000,0.000000,0.000000}%
\pgfsetstrokecolor{currentstroke}%
\pgfsetdash{}{0pt}%
\pgfsys@defobject{currentmarker}{\pgfqpoint{0.000000in}{-0.048611in}}{\pgfqpoint{0.000000in}{0.000000in}}{%
\pgfpathmoveto{\pgfqpoint{0.000000in}{0.000000in}}%
\pgfpathlineto{\pgfqpoint{0.000000in}{-0.048611in}}%
\pgfusepath{stroke,fill}%
}%
\begin{pgfscope}%
\pgfsys@transformshift{1.555215in}{0.499444in}%
\pgfsys@useobject{currentmarker}{}%
\end{pgfscope}%
\end{pgfscope}%
\begin{pgfscope}%
\definecolor{textcolor}{rgb}{0.000000,0.000000,0.000000}%
\pgfsetstrokecolor{textcolor}%
\pgfsetfillcolor{textcolor}%
\pgftext[x=1.555215in,y=0.402222in,,top]{\color{textcolor}\rmfamily\fontsize{10.000000}{12.000000}\selectfont 0.3}%
\end{pgfscope}%
\begin{pgfscope}%
\pgfsetbuttcap%
\pgfsetroundjoin%
\definecolor{currentfill}{rgb}{0.000000,0.000000,0.000000}%
\pgfsetfillcolor{currentfill}%
\pgfsetlinewidth{0.803000pt}%
\definecolor{currentstroke}{rgb}{0.000000,0.000000,0.000000}%
\pgfsetstrokecolor{currentstroke}%
\pgfsetdash{}{0pt}%
\pgfsys@defobject{currentmarker}{\pgfqpoint{0.000000in}{-0.048611in}}{\pgfqpoint{0.000000in}{0.000000in}}{%
\pgfpathmoveto{\pgfqpoint{0.000000in}{0.000000in}}%
\pgfpathlineto{\pgfqpoint{0.000000in}{-0.048611in}}%
\pgfusepath{stroke,fill}%
}%
\begin{pgfscope}%
\pgfsys@transformshift{1.713738in}{0.499444in}%
\pgfsys@useobject{currentmarker}{}%
\end{pgfscope}%
\end{pgfscope}%
\begin{pgfscope}%
\pgfsetbuttcap%
\pgfsetroundjoin%
\definecolor{currentfill}{rgb}{0.000000,0.000000,0.000000}%
\pgfsetfillcolor{currentfill}%
\pgfsetlinewidth{0.803000pt}%
\definecolor{currentstroke}{rgb}{0.000000,0.000000,0.000000}%
\pgfsetstrokecolor{currentstroke}%
\pgfsetdash{}{0pt}%
\pgfsys@defobject{currentmarker}{\pgfqpoint{0.000000in}{-0.048611in}}{\pgfqpoint{0.000000in}{0.000000in}}{%
\pgfpathmoveto{\pgfqpoint{0.000000in}{0.000000in}}%
\pgfpathlineto{\pgfqpoint{0.000000in}{-0.048611in}}%
\pgfusepath{stroke,fill}%
}%
\begin{pgfscope}%
\pgfsys@transformshift{1.872260in}{0.499444in}%
\pgfsys@useobject{currentmarker}{}%
\end{pgfscope}%
\end{pgfscope}%
\begin{pgfscope}%
\definecolor{textcolor}{rgb}{0.000000,0.000000,0.000000}%
\pgfsetstrokecolor{textcolor}%
\pgfsetfillcolor{textcolor}%
\pgftext[x=1.872260in,y=0.402222in,,top]{\color{textcolor}\rmfamily\fontsize{10.000000}{12.000000}\selectfont 0.4}%
\end{pgfscope}%
\begin{pgfscope}%
\pgfsetbuttcap%
\pgfsetroundjoin%
\definecolor{currentfill}{rgb}{0.000000,0.000000,0.000000}%
\pgfsetfillcolor{currentfill}%
\pgfsetlinewidth{0.803000pt}%
\definecolor{currentstroke}{rgb}{0.000000,0.000000,0.000000}%
\pgfsetstrokecolor{currentstroke}%
\pgfsetdash{}{0pt}%
\pgfsys@defobject{currentmarker}{\pgfqpoint{0.000000in}{-0.048611in}}{\pgfqpoint{0.000000in}{0.000000in}}{%
\pgfpathmoveto{\pgfqpoint{0.000000in}{0.000000in}}%
\pgfpathlineto{\pgfqpoint{0.000000in}{-0.048611in}}%
\pgfusepath{stroke,fill}%
}%
\begin{pgfscope}%
\pgfsys@transformshift{2.030783in}{0.499444in}%
\pgfsys@useobject{currentmarker}{}%
\end{pgfscope}%
\end{pgfscope}%
\begin{pgfscope}%
\pgfsetbuttcap%
\pgfsetroundjoin%
\definecolor{currentfill}{rgb}{0.000000,0.000000,0.000000}%
\pgfsetfillcolor{currentfill}%
\pgfsetlinewidth{0.803000pt}%
\definecolor{currentstroke}{rgb}{0.000000,0.000000,0.000000}%
\pgfsetstrokecolor{currentstroke}%
\pgfsetdash{}{0pt}%
\pgfsys@defobject{currentmarker}{\pgfqpoint{0.000000in}{-0.048611in}}{\pgfqpoint{0.000000in}{0.000000in}}{%
\pgfpathmoveto{\pgfqpoint{0.000000in}{0.000000in}}%
\pgfpathlineto{\pgfqpoint{0.000000in}{-0.048611in}}%
\pgfusepath{stroke,fill}%
}%
\begin{pgfscope}%
\pgfsys@transformshift{2.189306in}{0.499444in}%
\pgfsys@useobject{currentmarker}{}%
\end{pgfscope}%
\end{pgfscope}%
\begin{pgfscope}%
\definecolor{textcolor}{rgb}{0.000000,0.000000,0.000000}%
\pgfsetstrokecolor{textcolor}%
\pgfsetfillcolor{textcolor}%
\pgftext[x=2.189306in,y=0.402222in,,top]{\color{textcolor}\rmfamily\fontsize{10.000000}{12.000000}\selectfont 0.5}%
\end{pgfscope}%
\begin{pgfscope}%
\pgfsetbuttcap%
\pgfsetroundjoin%
\definecolor{currentfill}{rgb}{0.000000,0.000000,0.000000}%
\pgfsetfillcolor{currentfill}%
\pgfsetlinewidth{0.803000pt}%
\definecolor{currentstroke}{rgb}{0.000000,0.000000,0.000000}%
\pgfsetstrokecolor{currentstroke}%
\pgfsetdash{}{0pt}%
\pgfsys@defobject{currentmarker}{\pgfqpoint{0.000000in}{-0.048611in}}{\pgfqpoint{0.000000in}{0.000000in}}{%
\pgfpathmoveto{\pgfqpoint{0.000000in}{0.000000in}}%
\pgfpathlineto{\pgfqpoint{0.000000in}{-0.048611in}}%
\pgfusepath{stroke,fill}%
}%
\begin{pgfscope}%
\pgfsys@transformshift{2.347828in}{0.499444in}%
\pgfsys@useobject{currentmarker}{}%
\end{pgfscope}%
\end{pgfscope}%
\begin{pgfscope}%
\pgfsetbuttcap%
\pgfsetroundjoin%
\definecolor{currentfill}{rgb}{0.000000,0.000000,0.000000}%
\pgfsetfillcolor{currentfill}%
\pgfsetlinewidth{0.803000pt}%
\definecolor{currentstroke}{rgb}{0.000000,0.000000,0.000000}%
\pgfsetstrokecolor{currentstroke}%
\pgfsetdash{}{0pt}%
\pgfsys@defobject{currentmarker}{\pgfqpoint{0.000000in}{-0.048611in}}{\pgfqpoint{0.000000in}{0.000000in}}{%
\pgfpathmoveto{\pgfqpoint{0.000000in}{0.000000in}}%
\pgfpathlineto{\pgfqpoint{0.000000in}{-0.048611in}}%
\pgfusepath{stroke,fill}%
}%
\begin{pgfscope}%
\pgfsys@transformshift{2.506351in}{0.499444in}%
\pgfsys@useobject{currentmarker}{}%
\end{pgfscope}%
\end{pgfscope}%
\begin{pgfscope}%
\definecolor{textcolor}{rgb}{0.000000,0.000000,0.000000}%
\pgfsetstrokecolor{textcolor}%
\pgfsetfillcolor{textcolor}%
\pgftext[x=2.506351in,y=0.402222in,,top]{\color{textcolor}\rmfamily\fontsize{10.000000}{12.000000}\selectfont 0.6}%
\end{pgfscope}%
\begin{pgfscope}%
\pgfsetbuttcap%
\pgfsetroundjoin%
\definecolor{currentfill}{rgb}{0.000000,0.000000,0.000000}%
\pgfsetfillcolor{currentfill}%
\pgfsetlinewidth{0.803000pt}%
\definecolor{currentstroke}{rgb}{0.000000,0.000000,0.000000}%
\pgfsetstrokecolor{currentstroke}%
\pgfsetdash{}{0pt}%
\pgfsys@defobject{currentmarker}{\pgfqpoint{0.000000in}{-0.048611in}}{\pgfqpoint{0.000000in}{0.000000in}}{%
\pgfpathmoveto{\pgfqpoint{0.000000in}{0.000000in}}%
\pgfpathlineto{\pgfqpoint{0.000000in}{-0.048611in}}%
\pgfusepath{stroke,fill}%
}%
\begin{pgfscope}%
\pgfsys@transformshift{2.664874in}{0.499444in}%
\pgfsys@useobject{currentmarker}{}%
\end{pgfscope}%
\end{pgfscope}%
\begin{pgfscope}%
\pgfsetbuttcap%
\pgfsetroundjoin%
\definecolor{currentfill}{rgb}{0.000000,0.000000,0.000000}%
\pgfsetfillcolor{currentfill}%
\pgfsetlinewidth{0.803000pt}%
\definecolor{currentstroke}{rgb}{0.000000,0.000000,0.000000}%
\pgfsetstrokecolor{currentstroke}%
\pgfsetdash{}{0pt}%
\pgfsys@defobject{currentmarker}{\pgfqpoint{0.000000in}{-0.048611in}}{\pgfqpoint{0.000000in}{0.000000in}}{%
\pgfpathmoveto{\pgfqpoint{0.000000in}{0.000000in}}%
\pgfpathlineto{\pgfqpoint{0.000000in}{-0.048611in}}%
\pgfusepath{stroke,fill}%
}%
\begin{pgfscope}%
\pgfsys@transformshift{2.823397in}{0.499444in}%
\pgfsys@useobject{currentmarker}{}%
\end{pgfscope}%
\end{pgfscope}%
\begin{pgfscope}%
\definecolor{textcolor}{rgb}{0.000000,0.000000,0.000000}%
\pgfsetstrokecolor{textcolor}%
\pgfsetfillcolor{textcolor}%
\pgftext[x=2.823397in,y=0.402222in,,top]{\color{textcolor}\rmfamily\fontsize{10.000000}{12.000000}\selectfont 0.7}%
\end{pgfscope}%
\begin{pgfscope}%
\pgfsetbuttcap%
\pgfsetroundjoin%
\definecolor{currentfill}{rgb}{0.000000,0.000000,0.000000}%
\pgfsetfillcolor{currentfill}%
\pgfsetlinewidth{0.803000pt}%
\definecolor{currentstroke}{rgb}{0.000000,0.000000,0.000000}%
\pgfsetstrokecolor{currentstroke}%
\pgfsetdash{}{0pt}%
\pgfsys@defobject{currentmarker}{\pgfqpoint{0.000000in}{-0.048611in}}{\pgfqpoint{0.000000in}{0.000000in}}{%
\pgfpathmoveto{\pgfqpoint{0.000000in}{0.000000in}}%
\pgfpathlineto{\pgfqpoint{0.000000in}{-0.048611in}}%
\pgfusepath{stroke,fill}%
}%
\begin{pgfscope}%
\pgfsys@transformshift{2.981919in}{0.499444in}%
\pgfsys@useobject{currentmarker}{}%
\end{pgfscope}%
\end{pgfscope}%
\begin{pgfscope}%
\pgfsetbuttcap%
\pgfsetroundjoin%
\definecolor{currentfill}{rgb}{0.000000,0.000000,0.000000}%
\pgfsetfillcolor{currentfill}%
\pgfsetlinewidth{0.803000pt}%
\definecolor{currentstroke}{rgb}{0.000000,0.000000,0.000000}%
\pgfsetstrokecolor{currentstroke}%
\pgfsetdash{}{0pt}%
\pgfsys@defobject{currentmarker}{\pgfqpoint{0.000000in}{-0.048611in}}{\pgfqpoint{0.000000in}{0.000000in}}{%
\pgfpathmoveto{\pgfqpoint{0.000000in}{0.000000in}}%
\pgfpathlineto{\pgfqpoint{0.000000in}{-0.048611in}}%
\pgfusepath{stroke,fill}%
}%
\begin{pgfscope}%
\pgfsys@transformshift{3.140442in}{0.499444in}%
\pgfsys@useobject{currentmarker}{}%
\end{pgfscope}%
\end{pgfscope}%
\begin{pgfscope}%
\definecolor{textcolor}{rgb}{0.000000,0.000000,0.000000}%
\pgfsetstrokecolor{textcolor}%
\pgfsetfillcolor{textcolor}%
\pgftext[x=3.140442in,y=0.402222in,,top]{\color{textcolor}\rmfamily\fontsize{10.000000}{12.000000}\selectfont 0.8}%
\end{pgfscope}%
\begin{pgfscope}%
\pgfsetbuttcap%
\pgfsetroundjoin%
\definecolor{currentfill}{rgb}{0.000000,0.000000,0.000000}%
\pgfsetfillcolor{currentfill}%
\pgfsetlinewidth{0.803000pt}%
\definecolor{currentstroke}{rgb}{0.000000,0.000000,0.000000}%
\pgfsetstrokecolor{currentstroke}%
\pgfsetdash{}{0pt}%
\pgfsys@defobject{currentmarker}{\pgfqpoint{0.000000in}{-0.048611in}}{\pgfqpoint{0.000000in}{0.000000in}}{%
\pgfpathmoveto{\pgfqpoint{0.000000in}{0.000000in}}%
\pgfpathlineto{\pgfqpoint{0.000000in}{-0.048611in}}%
\pgfusepath{stroke,fill}%
}%
\begin{pgfscope}%
\pgfsys@transformshift{3.298965in}{0.499444in}%
\pgfsys@useobject{currentmarker}{}%
\end{pgfscope}%
\end{pgfscope}%
\begin{pgfscope}%
\pgfsetbuttcap%
\pgfsetroundjoin%
\definecolor{currentfill}{rgb}{0.000000,0.000000,0.000000}%
\pgfsetfillcolor{currentfill}%
\pgfsetlinewidth{0.803000pt}%
\definecolor{currentstroke}{rgb}{0.000000,0.000000,0.000000}%
\pgfsetstrokecolor{currentstroke}%
\pgfsetdash{}{0pt}%
\pgfsys@defobject{currentmarker}{\pgfqpoint{0.000000in}{-0.048611in}}{\pgfqpoint{0.000000in}{0.000000in}}{%
\pgfpathmoveto{\pgfqpoint{0.000000in}{0.000000in}}%
\pgfpathlineto{\pgfqpoint{0.000000in}{-0.048611in}}%
\pgfusepath{stroke,fill}%
}%
\begin{pgfscope}%
\pgfsys@transformshift{3.457488in}{0.499444in}%
\pgfsys@useobject{currentmarker}{}%
\end{pgfscope}%
\end{pgfscope}%
\begin{pgfscope}%
\definecolor{textcolor}{rgb}{0.000000,0.000000,0.000000}%
\pgfsetstrokecolor{textcolor}%
\pgfsetfillcolor{textcolor}%
\pgftext[x=3.457488in,y=0.402222in,,top]{\color{textcolor}\rmfamily\fontsize{10.000000}{12.000000}\selectfont 0.9}%
\end{pgfscope}%
\begin{pgfscope}%
\pgfsetbuttcap%
\pgfsetroundjoin%
\definecolor{currentfill}{rgb}{0.000000,0.000000,0.000000}%
\pgfsetfillcolor{currentfill}%
\pgfsetlinewidth{0.803000pt}%
\definecolor{currentstroke}{rgb}{0.000000,0.000000,0.000000}%
\pgfsetstrokecolor{currentstroke}%
\pgfsetdash{}{0pt}%
\pgfsys@defobject{currentmarker}{\pgfqpoint{0.000000in}{-0.048611in}}{\pgfqpoint{0.000000in}{0.000000in}}{%
\pgfpathmoveto{\pgfqpoint{0.000000in}{0.000000in}}%
\pgfpathlineto{\pgfqpoint{0.000000in}{-0.048611in}}%
\pgfusepath{stroke,fill}%
}%
\begin{pgfscope}%
\pgfsys@transformshift{3.616010in}{0.499444in}%
\pgfsys@useobject{currentmarker}{}%
\end{pgfscope}%
\end{pgfscope}%
\begin{pgfscope}%
\pgfsetbuttcap%
\pgfsetroundjoin%
\definecolor{currentfill}{rgb}{0.000000,0.000000,0.000000}%
\pgfsetfillcolor{currentfill}%
\pgfsetlinewidth{0.803000pt}%
\definecolor{currentstroke}{rgb}{0.000000,0.000000,0.000000}%
\pgfsetstrokecolor{currentstroke}%
\pgfsetdash{}{0pt}%
\pgfsys@defobject{currentmarker}{\pgfqpoint{0.000000in}{-0.048611in}}{\pgfqpoint{0.000000in}{0.000000in}}{%
\pgfpathmoveto{\pgfqpoint{0.000000in}{0.000000in}}%
\pgfpathlineto{\pgfqpoint{0.000000in}{-0.048611in}}%
\pgfusepath{stroke,fill}%
}%
\begin{pgfscope}%
\pgfsys@transformshift{3.774533in}{0.499444in}%
\pgfsys@useobject{currentmarker}{}%
\end{pgfscope}%
\end{pgfscope}%
\begin{pgfscope}%
\definecolor{textcolor}{rgb}{0.000000,0.000000,0.000000}%
\pgfsetstrokecolor{textcolor}%
\pgfsetfillcolor{textcolor}%
\pgftext[x=3.774533in,y=0.402222in,,top]{\color{textcolor}\rmfamily\fontsize{10.000000}{12.000000}\selectfont 1.0}%
\end{pgfscope}%
\begin{pgfscope}%
\pgfsetbuttcap%
\pgfsetroundjoin%
\definecolor{currentfill}{rgb}{0.000000,0.000000,0.000000}%
\pgfsetfillcolor{currentfill}%
\pgfsetlinewidth{0.803000pt}%
\definecolor{currentstroke}{rgb}{0.000000,0.000000,0.000000}%
\pgfsetstrokecolor{currentstroke}%
\pgfsetdash{}{0pt}%
\pgfsys@defobject{currentmarker}{\pgfqpoint{0.000000in}{-0.048611in}}{\pgfqpoint{0.000000in}{0.000000in}}{%
\pgfpathmoveto{\pgfqpoint{0.000000in}{0.000000in}}%
\pgfpathlineto{\pgfqpoint{0.000000in}{-0.048611in}}%
\pgfusepath{stroke,fill}%
}%
\begin{pgfscope}%
\pgfsys@transformshift{3.933056in}{0.499444in}%
\pgfsys@useobject{currentmarker}{}%
\end{pgfscope}%
\end{pgfscope}%
\begin{pgfscope}%
\definecolor{textcolor}{rgb}{0.000000,0.000000,0.000000}%
\pgfsetstrokecolor{textcolor}%
\pgfsetfillcolor{textcolor}%
\pgftext[x=2.189306in,y=0.223333in,,top]{\color{textcolor}\rmfamily\fontsize{10.000000}{12.000000}\selectfont \(\displaystyle p\)}%
\end{pgfscope}%
\begin{pgfscope}%
\pgfsetbuttcap%
\pgfsetroundjoin%
\definecolor{currentfill}{rgb}{0.000000,0.000000,0.000000}%
\pgfsetfillcolor{currentfill}%
\pgfsetlinewidth{0.803000pt}%
\definecolor{currentstroke}{rgb}{0.000000,0.000000,0.000000}%
\pgfsetstrokecolor{currentstroke}%
\pgfsetdash{}{0pt}%
\pgfsys@defobject{currentmarker}{\pgfqpoint{-0.048611in}{0.000000in}}{\pgfqpoint{-0.000000in}{0.000000in}}{%
\pgfpathmoveto{\pgfqpoint{-0.000000in}{0.000000in}}%
\pgfpathlineto{\pgfqpoint{-0.048611in}{0.000000in}}%
\pgfusepath{stroke,fill}%
}%
\begin{pgfscope}%
\pgfsys@transformshift{0.445556in}{0.499444in}%
\pgfsys@useobject{currentmarker}{}%
\end{pgfscope}%
\end{pgfscope}%
\begin{pgfscope}%
\definecolor{textcolor}{rgb}{0.000000,0.000000,0.000000}%
\pgfsetstrokecolor{textcolor}%
\pgfsetfillcolor{textcolor}%
\pgftext[x=0.278889in, y=0.451250in, left, base]{\color{textcolor}\rmfamily\fontsize{10.000000}{12.000000}\selectfont \(\displaystyle {0}\)}%
\end{pgfscope}%
\begin{pgfscope}%
\pgfsetbuttcap%
\pgfsetroundjoin%
\definecolor{currentfill}{rgb}{0.000000,0.000000,0.000000}%
\pgfsetfillcolor{currentfill}%
\pgfsetlinewidth{0.803000pt}%
\definecolor{currentstroke}{rgb}{0.000000,0.000000,0.000000}%
\pgfsetstrokecolor{currentstroke}%
\pgfsetdash{}{0pt}%
\pgfsys@defobject{currentmarker}{\pgfqpoint{-0.048611in}{0.000000in}}{\pgfqpoint{-0.000000in}{0.000000in}}{%
\pgfpathmoveto{\pgfqpoint{-0.000000in}{0.000000in}}%
\pgfpathlineto{\pgfqpoint{-0.048611in}{0.000000in}}%
\pgfusepath{stroke,fill}%
}%
\begin{pgfscope}%
\pgfsys@transformshift{0.445556in}{0.830663in}%
\pgfsys@useobject{currentmarker}{}%
\end{pgfscope}%
\end{pgfscope}%
\begin{pgfscope}%
\definecolor{textcolor}{rgb}{0.000000,0.000000,0.000000}%
\pgfsetstrokecolor{textcolor}%
\pgfsetfillcolor{textcolor}%
\pgftext[x=0.278889in, y=0.782469in, left, base]{\color{textcolor}\rmfamily\fontsize{10.000000}{12.000000}\selectfont \(\displaystyle {2}\)}%
\end{pgfscope}%
\begin{pgfscope}%
\pgfsetbuttcap%
\pgfsetroundjoin%
\definecolor{currentfill}{rgb}{0.000000,0.000000,0.000000}%
\pgfsetfillcolor{currentfill}%
\pgfsetlinewidth{0.803000pt}%
\definecolor{currentstroke}{rgb}{0.000000,0.000000,0.000000}%
\pgfsetstrokecolor{currentstroke}%
\pgfsetdash{}{0pt}%
\pgfsys@defobject{currentmarker}{\pgfqpoint{-0.048611in}{0.000000in}}{\pgfqpoint{-0.000000in}{0.000000in}}{%
\pgfpathmoveto{\pgfqpoint{-0.000000in}{0.000000in}}%
\pgfpathlineto{\pgfqpoint{-0.048611in}{0.000000in}}%
\pgfusepath{stroke,fill}%
}%
\begin{pgfscope}%
\pgfsys@transformshift{0.445556in}{1.161882in}%
\pgfsys@useobject{currentmarker}{}%
\end{pgfscope}%
\end{pgfscope}%
\begin{pgfscope}%
\definecolor{textcolor}{rgb}{0.000000,0.000000,0.000000}%
\pgfsetstrokecolor{textcolor}%
\pgfsetfillcolor{textcolor}%
\pgftext[x=0.278889in, y=1.113688in, left, base]{\color{textcolor}\rmfamily\fontsize{10.000000}{12.000000}\selectfont \(\displaystyle {4}\)}%
\end{pgfscope}%
\begin{pgfscope}%
\pgfsetbuttcap%
\pgfsetroundjoin%
\definecolor{currentfill}{rgb}{0.000000,0.000000,0.000000}%
\pgfsetfillcolor{currentfill}%
\pgfsetlinewidth{0.803000pt}%
\definecolor{currentstroke}{rgb}{0.000000,0.000000,0.000000}%
\pgfsetstrokecolor{currentstroke}%
\pgfsetdash{}{0pt}%
\pgfsys@defobject{currentmarker}{\pgfqpoint{-0.048611in}{0.000000in}}{\pgfqpoint{-0.000000in}{0.000000in}}{%
\pgfpathmoveto{\pgfqpoint{-0.000000in}{0.000000in}}%
\pgfpathlineto{\pgfqpoint{-0.048611in}{0.000000in}}%
\pgfusepath{stroke,fill}%
}%
\begin{pgfscope}%
\pgfsys@transformshift{0.445556in}{1.493101in}%
\pgfsys@useobject{currentmarker}{}%
\end{pgfscope}%
\end{pgfscope}%
\begin{pgfscope}%
\definecolor{textcolor}{rgb}{0.000000,0.000000,0.000000}%
\pgfsetstrokecolor{textcolor}%
\pgfsetfillcolor{textcolor}%
\pgftext[x=0.278889in, y=1.444906in, left, base]{\color{textcolor}\rmfamily\fontsize{10.000000}{12.000000}\selectfont \(\displaystyle {6}\)}%
\end{pgfscope}%
\begin{pgfscope}%
\definecolor{textcolor}{rgb}{0.000000,0.000000,0.000000}%
\pgfsetstrokecolor{textcolor}%
\pgfsetfillcolor{textcolor}%
\pgftext[x=0.223333in,y=1.076944in,,bottom,rotate=90.000000]{\color{textcolor}\rmfamily\fontsize{10.000000}{12.000000}\selectfont Percent of Data Set}%
\end{pgfscope}%
\begin{pgfscope}%
\pgfsetrectcap%
\pgfsetmiterjoin%
\pgfsetlinewidth{0.803000pt}%
\definecolor{currentstroke}{rgb}{0.000000,0.000000,0.000000}%
\pgfsetstrokecolor{currentstroke}%
\pgfsetdash{}{0pt}%
\pgfpathmoveto{\pgfqpoint{0.445556in}{0.499444in}}%
\pgfpathlineto{\pgfqpoint{0.445556in}{1.654444in}}%
\pgfusepath{stroke}%
\end{pgfscope}%
\begin{pgfscope}%
\pgfsetrectcap%
\pgfsetmiterjoin%
\pgfsetlinewidth{0.803000pt}%
\definecolor{currentstroke}{rgb}{0.000000,0.000000,0.000000}%
\pgfsetstrokecolor{currentstroke}%
\pgfsetdash{}{0pt}%
\pgfpathmoveto{\pgfqpoint{3.933056in}{0.499444in}}%
\pgfpathlineto{\pgfqpoint{3.933056in}{1.654444in}}%
\pgfusepath{stroke}%
\end{pgfscope}%
\begin{pgfscope}%
\pgfsetrectcap%
\pgfsetmiterjoin%
\pgfsetlinewidth{0.803000pt}%
\definecolor{currentstroke}{rgb}{0.000000,0.000000,0.000000}%
\pgfsetstrokecolor{currentstroke}%
\pgfsetdash{}{0pt}%
\pgfpathmoveto{\pgfqpoint{0.445556in}{0.499444in}}%
\pgfpathlineto{\pgfqpoint{3.933056in}{0.499444in}}%
\pgfusepath{stroke}%
\end{pgfscope}%
\begin{pgfscope}%
\pgfsetrectcap%
\pgfsetmiterjoin%
\pgfsetlinewidth{0.803000pt}%
\definecolor{currentstroke}{rgb}{0.000000,0.000000,0.000000}%
\pgfsetstrokecolor{currentstroke}%
\pgfsetdash{}{0pt}%
\pgfpathmoveto{\pgfqpoint{0.445556in}{1.654444in}}%
\pgfpathlineto{\pgfqpoint{3.933056in}{1.654444in}}%
\pgfusepath{stroke}%
\end{pgfscope}%
\begin{pgfscope}%
\pgfsetbuttcap%
\pgfsetmiterjoin%
\definecolor{currentfill}{rgb}{1.000000,1.000000,1.000000}%
\pgfsetfillcolor{currentfill}%
\pgfsetfillopacity{0.800000}%
\pgfsetlinewidth{1.003750pt}%
\definecolor{currentstroke}{rgb}{0.800000,0.800000,0.800000}%
\pgfsetstrokecolor{currentstroke}%
\pgfsetstrokeopacity{0.800000}%
\pgfsetdash{}{0pt}%
\pgfpathmoveto{\pgfqpoint{3.156111in}{1.154445in}}%
\pgfpathlineto{\pgfqpoint{3.835833in}{1.154445in}}%
\pgfpathquadraticcurveto{\pgfqpoint{3.863611in}{1.154445in}}{\pgfqpoint{3.863611in}{1.182222in}}%
\pgfpathlineto{\pgfqpoint{3.863611in}{1.557222in}}%
\pgfpathquadraticcurveto{\pgfqpoint{3.863611in}{1.585000in}}{\pgfqpoint{3.835833in}{1.585000in}}%
\pgfpathlineto{\pgfqpoint{3.156111in}{1.585000in}}%
\pgfpathquadraticcurveto{\pgfqpoint{3.128333in}{1.585000in}}{\pgfqpoint{3.128333in}{1.557222in}}%
\pgfpathlineto{\pgfqpoint{3.128333in}{1.182222in}}%
\pgfpathquadraticcurveto{\pgfqpoint{3.128333in}{1.154445in}}{\pgfqpoint{3.156111in}{1.154445in}}%
\pgfpathlineto{\pgfqpoint{3.156111in}{1.154445in}}%
\pgfpathclose%
\pgfusepath{stroke,fill}%
\end{pgfscope}%
\begin{pgfscope}%
\pgfsetbuttcap%
\pgfsetmiterjoin%
\pgfsetlinewidth{1.003750pt}%
\definecolor{currentstroke}{rgb}{0.000000,0.000000,0.000000}%
\pgfsetstrokecolor{currentstroke}%
\pgfsetdash{}{0pt}%
\pgfpathmoveto{\pgfqpoint{3.183889in}{1.432222in}}%
\pgfpathlineto{\pgfqpoint{3.461667in}{1.432222in}}%
\pgfpathlineto{\pgfqpoint{3.461667in}{1.529444in}}%
\pgfpathlineto{\pgfqpoint{3.183889in}{1.529444in}}%
\pgfpathlineto{\pgfqpoint{3.183889in}{1.432222in}}%
\pgfpathclose%
\pgfusepath{stroke}%
\end{pgfscope}%
\begin{pgfscope}%
\definecolor{textcolor}{rgb}{0.000000,0.000000,0.000000}%
\pgfsetstrokecolor{textcolor}%
\pgfsetfillcolor{textcolor}%
\pgftext[x=3.572778in,y=1.432222in,left,base]{\color{textcolor}\rmfamily\fontsize{10.000000}{12.000000}\selectfont Neg}%
\end{pgfscope}%
\begin{pgfscope}%
\pgfsetbuttcap%
\pgfsetmiterjoin%
\definecolor{currentfill}{rgb}{0.000000,0.000000,0.000000}%
\pgfsetfillcolor{currentfill}%
\pgfsetlinewidth{0.000000pt}%
\definecolor{currentstroke}{rgb}{0.000000,0.000000,0.000000}%
\pgfsetstrokecolor{currentstroke}%
\pgfsetstrokeopacity{0.000000}%
\pgfsetdash{}{0pt}%
\pgfpathmoveto{\pgfqpoint{3.183889in}{1.236944in}}%
\pgfpathlineto{\pgfqpoint{3.461667in}{1.236944in}}%
\pgfpathlineto{\pgfqpoint{3.461667in}{1.334167in}}%
\pgfpathlineto{\pgfqpoint{3.183889in}{1.334167in}}%
\pgfpathlineto{\pgfqpoint{3.183889in}{1.236944in}}%
\pgfpathclose%
\pgfusepath{fill}%
\end{pgfscope}%
\begin{pgfscope}%
\definecolor{textcolor}{rgb}{0.000000,0.000000,0.000000}%
\pgfsetstrokecolor{textcolor}%
\pgfsetfillcolor{textcolor}%
\pgftext[x=3.572778in,y=1.236944in,left,base]{\color{textcolor}\rmfamily\fontsize{10.000000}{12.000000}\selectfont Pos}%
\end{pgfscope}%
\end{pgfpicture}%
\makeatother%
\endgroup%
	
&
	\vskip 0pt
	\hfil ROC Curve
	
	%% Creator: Matplotlib, PGF backend
%%
%% To include the figure in your LaTeX document, write
%%   \input{<filename>.pgf}
%%
%% Make sure the required packages are loaded in your preamble
%%   \usepackage{pgf}
%%
%% Also ensure that all the required font packages are loaded; for instance,
%% the lmodern package is sometimes necessary when using math font.
%%   \usepackage{lmodern}
%%
%% Figures using additional raster images can only be included by \input if
%% they are in the same directory as the main LaTeX file. For loading figures
%% from other directories you can use the `import` package
%%   \usepackage{import}
%%
%% and then include the figures with
%%   \import{<path to file>}{<filename>.pgf}
%%
%% Matplotlib used the following preamble
%%   
%%   \usepackage{fontspec}
%%   \makeatletter\@ifpackageloaded{underscore}{}{\usepackage[strings]{underscore}}\makeatother
%%
\begingroup%
\makeatletter%
\begin{pgfpicture}%
\pgfpathrectangle{\pgfpointorigin}{\pgfqpoint{2.121861in}{1.654444in}}%
\pgfusepath{use as bounding box, clip}%
\begin{pgfscope}%
\pgfsetbuttcap%
\pgfsetmiterjoin%
\definecolor{currentfill}{rgb}{1.000000,1.000000,1.000000}%
\pgfsetfillcolor{currentfill}%
\pgfsetlinewidth{0.000000pt}%
\definecolor{currentstroke}{rgb}{1.000000,1.000000,1.000000}%
\pgfsetstrokecolor{currentstroke}%
\pgfsetdash{}{0pt}%
\pgfpathmoveto{\pgfqpoint{0.000000in}{0.000000in}}%
\pgfpathlineto{\pgfqpoint{2.121861in}{0.000000in}}%
\pgfpathlineto{\pgfqpoint{2.121861in}{1.654444in}}%
\pgfpathlineto{\pgfqpoint{0.000000in}{1.654444in}}%
\pgfpathlineto{\pgfqpoint{0.000000in}{0.000000in}}%
\pgfpathclose%
\pgfusepath{fill}%
\end{pgfscope}%
\begin{pgfscope}%
\pgfsetbuttcap%
\pgfsetmiterjoin%
\definecolor{currentfill}{rgb}{1.000000,1.000000,1.000000}%
\pgfsetfillcolor{currentfill}%
\pgfsetlinewidth{0.000000pt}%
\definecolor{currentstroke}{rgb}{0.000000,0.000000,0.000000}%
\pgfsetstrokecolor{currentstroke}%
\pgfsetstrokeopacity{0.000000}%
\pgfsetdash{}{0pt}%
\pgfpathmoveto{\pgfqpoint{0.503581in}{0.449444in}}%
\pgfpathlineto{\pgfqpoint{2.053581in}{0.449444in}}%
\pgfpathlineto{\pgfqpoint{2.053581in}{1.604444in}}%
\pgfpathlineto{\pgfqpoint{0.503581in}{1.604444in}}%
\pgfpathlineto{\pgfqpoint{0.503581in}{0.449444in}}%
\pgfpathclose%
\pgfusepath{fill}%
\end{pgfscope}%
\begin{pgfscope}%
\pgfsetbuttcap%
\pgfsetroundjoin%
\definecolor{currentfill}{rgb}{0.000000,0.000000,0.000000}%
\pgfsetfillcolor{currentfill}%
\pgfsetlinewidth{0.803000pt}%
\definecolor{currentstroke}{rgb}{0.000000,0.000000,0.000000}%
\pgfsetstrokecolor{currentstroke}%
\pgfsetdash{}{0pt}%
\pgfsys@defobject{currentmarker}{\pgfqpoint{0.000000in}{-0.048611in}}{\pgfqpoint{0.000000in}{0.000000in}}{%
\pgfpathmoveto{\pgfqpoint{0.000000in}{0.000000in}}%
\pgfpathlineto{\pgfqpoint{0.000000in}{-0.048611in}}%
\pgfusepath{stroke,fill}%
}%
\begin{pgfscope}%
\pgfsys@transformshift{0.574035in}{0.449444in}%
\pgfsys@useobject{currentmarker}{}%
\end{pgfscope}%
\end{pgfscope}%
\begin{pgfscope}%
\definecolor{textcolor}{rgb}{0.000000,0.000000,0.000000}%
\pgfsetstrokecolor{textcolor}%
\pgfsetfillcolor{textcolor}%
\pgftext[x=0.574035in,y=0.352222in,,top]{\color{textcolor}\rmfamily\fontsize{10.000000}{12.000000}\selectfont \(\displaystyle {0.0}\)}%
\end{pgfscope}%
\begin{pgfscope}%
\pgfsetbuttcap%
\pgfsetroundjoin%
\definecolor{currentfill}{rgb}{0.000000,0.000000,0.000000}%
\pgfsetfillcolor{currentfill}%
\pgfsetlinewidth{0.803000pt}%
\definecolor{currentstroke}{rgb}{0.000000,0.000000,0.000000}%
\pgfsetstrokecolor{currentstroke}%
\pgfsetdash{}{0pt}%
\pgfsys@defobject{currentmarker}{\pgfqpoint{0.000000in}{-0.048611in}}{\pgfqpoint{0.000000in}{0.000000in}}{%
\pgfpathmoveto{\pgfqpoint{0.000000in}{0.000000in}}%
\pgfpathlineto{\pgfqpoint{0.000000in}{-0.048611in}}%
\pgfusepath{stroke,fill}%
}%
\begin{pgfscope}%
\pgfsys@transformshift{1.278581in}{0.449444in}%
\pgfsys@useobject{currentmarker}{}%
\end{pgfscope}%
\end{pgfscope}%
\begin{pgfscope}%
\definecolor{textcolor}{rgb}{0.000000,0.000000,0.000000}%
\pgfsetstrokecolor{textcolor}%
\pgfsetfillcolor{textcolor}%
\pgftext[x=1.278581in,y=0.352222in,,top]{\color{textcolor}\rmfamily\fontsize{10.000000}{12.000000}\selectfont \(\displaystyle {0.5}\)}%
\end{pgfscope}%
\begin{pgfscope}%
\pgfsetbuttcap%
\pgfsetroundjoin%
\definecolor{currentfill}{rgb}{0.000000,0.000000,0.000000}%
\pgfsetfillcolor{currentfill}%
\pgfsetlinewidth{0.803000pt}%
\definecolor{currentstroke}{rgb}{0.000000,0.000000,0.000000}%
\pgfsetstrokecolor{currentstroke}%
\pgfsetdash{}{0pt}%
\pgfsys@defobject{currentmarker}{\pgfqpoint{0.000000in}{-0.048611in}}{\pgfqpoint{0.000000in}{0.000000in}}{%
\pgfpathmoveto{\pgfqpoint{0.000000in}{0.000000in}}%
\pgfpathlineto{\pgfqpoint{0.000000in}{-0.048611in}}%
\pgfusepath{stroke,fill}%
}%
\begin{pgfscope}%
\pgfsys@transformshift{1.983126in}{0.449444in}%
\pgfsys@useobject{currentmarker}{}%
\end{pgfscope}%
\end{pgfscope}%
\begin{pgfscope}%
\definecolor{textcolor}{rgb}{0.000000,0.000000,0.000000}%
\pgfsetstrokecolor{textcolor}%
\pgfsetfillcolor{textcolor}%
\pgftext[x=1.983126in,y=0.352222in,,top]{\color{textcolor}\rmfamily\fontsize{10.000000}{12.000000}\selectfont \(\displaystyle {1.0}\)}%
\end{pgfscope}%
\begin{pgfscope}%
\definecolor{textcolor}{rgb}{0.000000,0.000000,0.000000}%
\pgfsetstrokecolor{textcolor}%
\pgfsetfillcolor{textcolor}%
\pgftext[x=1.278581in,y=0.173333in,,top]{\color{textcolor}\rmfamily\fontsize{10.000000}{12.000000}\selectfont False positive rate}%
\end{pgfscope}%
\begin{pgfscope}%
\pgfsetbuttcap%
\pgfsetroundjoin%
\definecolor{currentfill}{rgb}{0.000000,0.000000,0.000000}%
\pgfsetfillcolor{currentfill}%
\pgfsetlinewidth{0.803000pt}%
\definecolor{currentstroke}{rgb}{0.000000,0.000000,0.000000}%
\pgfsetstrokecolor{currentstroke}%
\pgfsetdash{}{0pt}%
\pgfsys@defobject{currentmarker}{\pgfqpoint{-0.048611in}{0.000000in}}{\pgfqpoint{-0.000000in}{0.000000in}}{%
\pgfpathmoveto{\pgfqpoint{-0.000000in}{0.000000in}}%
\pgfpathlineto{\pgfqpoint{-0.048611in}{0.000000in}}%
\pgfusepath{stroke,fill}%
}%
\begin{pgfscope}%
\pgfsys@transformshift{0.503581in}{0.501944in}%
\pgfsys@useobject{currentmarker}{}%
\end{pgfscope}%
\end{pgfscope}%
\begin{pgfscope}%
\definecolor{textcolor}{rgb}{0.000000,0.000000,0.000000}%
\pgfsetstrokecolor{textcolor}%
\pgfsetfillcolor{textcolor}%
\pgftext[x=0.228889in, y=0.453750in, left, base]{\color{textcolor}\rmfamily\fontsize{10.000000}{12.000000}\selectfont \(\displaystyle {0.0}\)}%
\end{pgfscope}%
\begin{pgfscope}%
\pgfsetbuttcap%
\pgfsetroundjoin%
\definecolor{currentfill}{rgb}{0.000000,0.000000,0.000000}%
\pgfsetfillcolor{currentfill}%
\pgfsetlinewidth{0.803000pt}%
\definecolor{currentstroke}{rgb}{0.000000,0.000000,0.000000}%
\pgfsetstrokecolor{currentstroke}%
\pgfsetdash{}{0pt}%
\pgfsys@defobject{currentmarker}{\pgfqpoint{-0.048611in}{0.000000in}}{\pgfqpoint{-0.000000in}{0.000000in}}{%
\pgfpathmoveto{\pgfqpoint{-0.000000in}{0.000000in}}%
\pgfpathlineto{\pgfqpoint{-0.048611in}{0.000000in}}%
\pgfusepath{stroke,fill}%
}%
\begin{pgfscope}%
\pgfsys@transformshift{0.503581in}{1.026944in}%
\pgfsys@useobject{currentmarker}{}%
\end{pgfscope}%
\end{pgfscope}%
\begin{pgfscope}%
\definecolor{textcolor}{rgb}{0.000000,0.000000,0.000000}%
\pgfsetstrokecolor{textcolor}%
\pgfsetfillcolor{textcolor}%
\pgftext[x=0.228889in, y=0.978750in, left, base]{\color{textcolor}\rmfamily\fontsize{10.000000}{12.000000}\selectfont \(\displaystyle {0.5}\)}%
\end{pgfscope}%
\begin{pgfscope}%
\pgfsetbuttcap%
\pgfsetroundjoin%
\definecolor{currentfill}{rgb}{0.000000,0.000000,0.000000}%
\pgfsetfillcolor{currentfill}%
\pgfsetlinewidth{0.803000pt}%
\definecolor{currentstroke}{rgb}{0.000000,0.000000,0.000000}%
\pgfsetstrokecolor{currentstroke}%
\pgfsetdash{}{0pt}%
\pgfsys@defobject{currentmarker}{\pgfqpoint{-0.048611in}{0.000000in}}{\pgfqpoint{-0.000000in}{0.000000in}}{%
\pgfpathmoveto{\pgfqpoint{-0.000000in}{0.000000in}}%
\pgfpathlineto{\pgfqpoint{-0.048611in}{0.000000in}}%
\pgfusepath{stroke,fill}%
}%
\begin{pgfscope}%
\pgfsys@transformshift{0.503581in}{1.551944in}%
\pgfsys@useobject{currentmarker}{}%
\end{pgfscope}%
\end{pgfscope}%
\begin{pgfscope}%
\definecolor{textcolor}{rgb}{0.000000,0.000000,0.000000}%
\pgfsetstrokecolor{textcolor}%
\pgfsetfillcolor{textcolor}%
\pgftext[x=0.228889in, y=1.503750in, left, base]{\color{textcolor}\rmfamily\fontsize{10.000000}{12.000000}\selectfont \(\displaystyle {1.0}\)}%
\end{pgfscope}%
\begin{pgfscope}%
\definecolor{textcolor}{rgb}{0.000000,0.000000,0.000000}%
\pgfsetstrokecolor{textcolor}%
\pgfsetfillcolor{textcolor}%
\pgftext[x=0.173333in,y=1.026944in,,bottom,rotate=90.000000]{\color{textcolor}\rmfamily\fontsize{10.000000}{12.000000}\selectfont True positive rate}%
\end{pgfscope}%
\begin{pgfscope}%
\pgfpathrectangle{\pgfqpoint{0.503581in}{0.449444in}}{\pgfqpoint{1.550000in}{1.155000in}}%
\pgfusepath{clip}%
\pgfsetbuttcap%
\pgfsetroundjoin%
\pgfsetlinewidth{1.505625pt}%
\definecolor{currentstroke}{rgb}{0.000000,0.000000,0.000000}%
\pgfsetstrokecolor{currentstroke}%
\pgfsetdash{{5.550000pt}{2.400000pt}}{0.000000pt}%
\pgfpathmoveto{\pgfqpoint{0.574035in}{0.501944in}}%
\pgfpathlineto{\pgfqpoint{1.983126in}{1.551944in}}%
\pgfusepath{stroke}%
\end{pgfscope}%
\begin{pgfscope}%
\pgfpathrectangle{\pgfqpoint{0.503581in}{0.449444in}}{\pgfqpoint{1.550000in}{1.155000in}}%
\pgfusepath{clip}%
\pgfsetrectcap%
\pgfsetroundjoin%
\pgfsetlinewidth{1.505625pt}%
\definecolor{currentstroke}{rgb}{0.000000,0.000000,0.000000}%
\pgfsetstrokecolor{currentstroke}%
\pgfsetdash{}{0pt}%
\pgfpathmoveto{\pgfqpoint{0.574035in}{0.501944in}}%
\pgfpathlineto{\pgfqpoint{0.575145in}{0.517068in}}%
\pgfpathlineto{\pgfqpoint{0.575254in}{0.518167in}}%
\pgfpathlineto{\pgfqpoint{0.576360in}{0.529781in}}%
\pgfpathlineto{\pgfqpoint{0.576434in}{0.530870in}}%
\pgfpathlineto{\pgfqpoint{0.577544in}{0.542074in}}%
\pgfpathlineto{\pgfqpoint{0.577665in}{0.543183in}}%
\pgfpathlineto{\pgfqpoint{0.578775in}{0.552832in}}%
\pgfpathlineto{\pgfqpoint{0.578921in}{0.553940in}}%
\pgfpathlineto{\pgfqpoint{0.580027in}{0.562704in}}%
\pgfpathlineto{\pgfqpoint{0.580224in}{0.563812in}}%
\pgfpathlineto{\pgfqpoint{0.581334in}{0.572410in}}%
\pgfpathlineto{\pgfqpoint{0.581518in}{0.573519in}}%
\pgfpathlineto{\pgfqpoint{0.582621in}{0.581290in}}%
\pgfpathlineto{\pgfqpoint{0.582779in}{0.582351in}}%
\pgfpathlineto{\pgfqpoint{0.582779in}{0.582399in}}%
\pgfpathlineto{\pgfqpoint{0.585567in}{0.598798in}}%
\pgfpathlineto{\pgfqpoint{0.586676in}{0.604760in}}%
\pgfpathlineto{\pgfqpoint{0.586842in}{0.605859in}}%
\pgfpathlineto{\pgfqpoint{0.587951in}{0.611344in}}%
\pgfpathlineto{\pgfqpoint{0.588205in}{0.612395in}}%
\pgfpathlineto{\pgfqpoint{0.589306in}{0.618016in}}%
\pgfpathlineto{\pgfqpoint{0.589522in}{0.619125in}}%
\pgfpathlineto{\pgfqpoint{0.590632in}{0.625107in}}%
\pgfpathlineto{\pgfqpoint{0.590899in}{0.626177in}}%
\pgfpathlineto{\pgfqpoint{0.592009in}{0.631944in}}%
\pgfpathlineto{\pgfqpoint{0.592237in}{0.633043in}}%
\pgfpathlineto{\pgfqpoint{0.593345in}{0.637945in}}%
\pgfpathlineto{\pgfqpoint{0.593582in}{0.639044in}}%
\pgfpathlineto{\pgfqpoint{0.594692in}{0.644433in}}%
\pgfpathlineto{\pgfqpoint{0.594918in}{0.645512in}}%
\pgfpathlineto{\pgfqpoint{0.596027in}{0.650356in}}%
\pgfpathlineto{\pgfqpoint{0.596267in}{0.651465in}}%
\pgfpathlineto{\pgfqpoint{0.597375in}{0.655812in}}%
\pgfpathlineto{\pgfqpoint{0.597377in}{0.655812in}}%
\pgfpathlineto{\pgfqpoint{0.607885in}{0.694999in}}%
\pgfpathlineto{\pgfqpoint{0.608257in}{0.696089in}}%
\pgfpathlineto{\pgfqpoint{0.609362in}{0.699580in}}%
\pgfpathlineto{\pgfqpoint{0.609692in}{0.700689in}}%
\pgfpathlineto{\pgfqpoint{0.610791in}{0.704881in}}%
\pgfpathlineto{\pgfqpoint{0.611095in}{0.705970in}}%
\pgfpathlineto{\pgfqpoint{0.612205in}{0.709705in}}%
\pgfpathlineto{\pgfqpoint{0.612654in}{0.710804in}}%
\pgfpathlineto{\pgfqpoint{0.613759in}{0.714636in}}%
\pgfpathlineto{\pgfqpoint{0.614120in}{0.715735in}}%
\pgfpathlineto{\pgfqpoint{0.615230in}{0.719645in}}%
\pgfpathlineto{\pgfqpoint{0.615623in}{0.720754in}}%
\pgfpathlineto{\pgfqpoint{0.616733in}{0.724460in}}%
\pgfpathlineto{\pgfqpoint{0.617117in}{0.725569in}}%
\pgfpathlineto{\pgfqpoint{0.618227in}{0.728837in}}%
\pgfpathlineto{\pgfqpoint{0.618653in}{0.729945in}}%
\pgfpathlineto{\pgfqpoint{0.619762in}{0.733087in}}%
\pgfpathlineto{\pgfqpoint{0.620177in}{0.734186in}}%
\pgfpathlineto{\pgfqpoint{0.621284in}{0.737230in}}%
\pgfpathlineto{\pgfqpoint{0.621682in}{0.738329in}}%
\pgfpathlineto{\pgfqpoint{0.622792in}{0.742093in}}%
\pgfpathlineto{\pgfqpoint{0.623153in}{0.743192in}}%
\pgfpathlineto{\pgfqpoint{0.624260in}{0.746772in}}%
\pgfpathlineto{\pgfqpoint{0.624716in}{0.747871in}}%
\pgfpathlineto{\pgfqpoint{0.625821in}{0.751285in}}%
\pgfpathlineto{\pgfqpoint{0.625826in}{0.751285in}}%
\pgfpathlineto{\pgfqpoint{0.629342in}{0.761235in}}%
\pgfpathlineto{\pgfqpoint{0.629670in}{0.762343in}}%
\pgfpathlineto{\pgfqpoint{0.630780in}{0.765164in}}%
\pgfpathlineto{\pgfqpoint{0.631147in}{0.766273in}}%
\pgfpathlineto{\pgfqpoint{0.632257in}{0.769113in}}%
\pgfpathlineto{\pgfqpoint{0.632718in}{0.770222in}}%
\pgfpathlineto{\pgfqpoint{0.633828in}{0.772838in}}%
\pgfpathlineto{\pgfqpoint{0.634193in}{0.773937in}}%
\pgfpathlineto{\pgfqpoint{0.635296in}{0.776563in}}%
\pgfpathlineto{\pgfqpoint{0.635747in}{0.777662in}}%
\pgfpathlineto{\pgfqpoint{0.636845in}{0.780658in}}%
\pgfpathlineto{\pgfqpoint{0.637262in}{0.781767in}}%
\pgfpathlineto{\pgfqpoint{0.638372in}{0.784801in}}%
\pgfpathlineto{\pgfqpoint{0.638865in}{0.785910in}}%
\pgfpathlineto{\pgfqpoint{0.641729in}{0.793127in}}%
\pgfpathlineto{\pgfqpoint{0.642239in}{0.794216in}}%
\pgfpathlineto{\pgfqpoint{0.643335in}{0.796959in}}%
\pgfpathlineto{\pgfqpoint{0.643826in}{0.798068in}}%
\pgfpathlineto{\pgfqpoint{0.644931in}{0.800674in}}%
\pgfpathlineto{\pgfqpoint{0.644935in}{0.800674in}}%
\pgfpathlineto{\pgfqpoint{0.645452in}{0.801783in}}%
\pgfpathlineto{\pgfqpoint{0.646559in}{0.803932in}}%
\pgfpathlineto{\pgfqpoint{0.647095in}{0.805041in}}%
\pgfpathlineto{\pgfqpoint{0.648200in}{0.807949in}}%
\pgfpathlineto{\pgfqpoint{0.648774in}{0.809058in}}%
\pgfpathlineto{\pgfqpoint{0.649884in}{0.811733in}}%
\pgfpathlineto{\pgfqpoint{0.650338in}{0.812832in}}%
\pgfpathlineto{\pgfqpoint{0.651448in}{0.815468in}}%
\pgfpathlineto{\pgfqpoint{0.651950in}{0.816577in}}%
\pgfpathlineto{\pgfqpoint{0.653058in}{0.818950in}}%
\pgfpathlineto{\pgfqpoint{0.653593in}{0.820058in}}%
\pgfpathlineto{\pgfqpoint{0.654691in}{0.822179in}}%
\pgfpathlineto{\pgfqpoint{0.654696in}{0.822179in}}%
\pgfpathlineto{\pgfqpoint{0.655180in}{0.823288in}}%
\pgfpathlineto{\pgfqpoint{0.656287in}{0.826147in}}%
\pgfpathlineto{\pgfqpoint{0.656744in}{0.827256in}}%
\pgfpathlineto{\pgfqpoint{0.657853in}{0.829473in}}%
\pgfpathlineto{\pgfqpoint{0.658377in}{0.830582in}}%
\pgfpathlineto{\pgfqpoint{0.659487in}{0.832771in}}%
\pgfpathlineto{\pgfqpoint{0.659938in}{0.833879in}}%
\pgfpathlineto{\pgfqpoint{0.661041in}{0.836185in}}%
\pgfpathlineto{\pgfqpoint{0.661530in}{0.837293in}}%
\pgfpathlineto{\pgfqpoint{0.662637in}{0.839783in}}%
\pgfpathlineto{\pgfqpoint{0.663140in}{0.840892in}}%
\pgfpathlineto{\pgfqpoint{0.664247in}{0.843450in}}%
\pgfpathlineto{\pgfqpoint{0.664785in}{0.844559in}}%
\pgfpathlineto{\pgfqpoint{0.665860in}{0.846796in}}%
\pgfpathlineto{\pgfqpoint{0.665895in}{0.846796in}}%
\pgfpathlineto{\pgfqpoint{0.668214in}{0.851727in}}%
\pgfpathlineto{\pgfqpoint{0.668673in}{0.852836in}}%
\pgfpathlineto{\pgfqpoint{0.669778in}{0.855141in}}%
\pgfpathlineto{\pgfqpoint{0.670301in}{0.856250in}}%
\pgfpathlineto{\pgfqpoint{0.671409in}{0.858389in}}%
\pgfpathlineto{\pgfqpoint{0.671942in}{0.859498in}}%
\pgfpathlineto{\pgfqpoint{0.673049in}{0.861803in}}%
\pgfpathlineto{\pgfqpoint{0.673659in}{0.862912in}}%
\pgfpathlineto{\pgfqpoint{0.674764in}{0.864955in}}%
\pgfpathlineto{\pgfqpoint{0.675313in}{0.866063in}}%
\pgfpathlineto{\pgfqpoint{0.676423in}{0.868077in}}%
\pgfpathlineto{\pgfqpoint{0.677077in}{0.869185in}}%
\pgfpathlineto{\pgfqpoint{0.678182in}{0.871199in}}%
\pgfpathlineto{\pgfqpoint{0.678782in}{0.872308in}}%
\pgfpathlineto{\pgfqpoint{0.679890in}{0.874914in}}%
\pgfpathlineto{\pgfqpoint{0.680658in}{0.876004in}}%
\pgfpathlineto{\pgfqpoint{0.681767in}{0.878425in}}%
\pgfpathlineto{\pgfqpoint{0.682326in}{0.879534in}}%
\pgfpathlineto{\pgfqpoint{0.683436in}{0.881577in}}%
\pgfpathlineto{\pgfqpoint{0.683924in}{0.882685in}}%
\pgfpathlineto{\pgfqpoint{0.685034in}{0.884942in}}%
\pgfpathlineto{\pgfqpoint{0.685567in}{0.886041in}}%
\pgfpathlineto{\pgfqpoint{0.686668in}{0.888190in}}%
\pgfpathlineto{\pgfqpoint{0.687368in}{0.889299in}}%
\pgfpathlineto{\pgfqpoint{0.688468in}{0.891332in}}%
\pgfpathlineto{\pgfqpoint{0.689153in}{0.892441in}}%
\pgfpathlineto{\pgfqpoint{0.690255in}{0.894337in}}%
\pgfpathlineto{\pgfqpoint{0.690923in}{0.895446in}}%
\pgfpathlineto{\pgfqpoint{0.692026in}{0.897683in}}%
\pgfpathlineto{\pgfqpoint{0.692622in}{0.898792in}}%
\pgfpathlineto{\pgfqpoint{0.693722in}{0.900679in}}%
\pgfpathlineto{\pgfqpoint{0.694330in}{0.901788in}}%
\pgfpathlineto{\pgfqpoint{0.695439in}{0.903986in}}%
\pgfpathlineto{\pgfqpoint{0.695912in}{0.905095in}}%
\pgfpathlineto{\pgfqpoint{0.697022in}{0.906952in}}%
\pgfpathlineto{\pgfqpoint{0.697661in}{0.908061in}}%
\pgfpathlineto{\pgfqpoint{0.698760in}{0.910629in}}%
\pgfpathlineto{\pgfqpoint{0.699390in}{0.911738in}}%
\pgfpathlineto{\pgfqpoint{0.700500in}{0.913585in}}%
\pgfpathlineto{\pgfqpoint{0.701149in}{0.914694in}}%
\pgfpathlineto{\pgfqpoint{0.702259in}{0.916591in}}%
\pgfpathlineto{\pgfqpoint{0.702785in}{0.917700in}}%
\pgfpathlineto{\pgfqpoint{0.703895in}{0.919791in}}%
\pgfpathlineto{\pgfqpoint{0.704595in}{0.920900in}}%
\pgfpathlineto{\pgfqpoint{0.705705in}{0.922592in}}%
\pgfpathlineto{\pgfqpoint{0.706426in}{0.923691in}}%
\pgfpathlineto{\pgfqpoint{0.707524in}{0.925549in}}%
\pgfpathlineto{\pgfqpoint{0.708220in}{0.926657in}}%
\pgfpathlineto{\pgfqpoint{0.709328in}{0.928165in}}%
\pgfpathlineto{\pgfqpoint{0.709981in}{0.929274in}}%
\pgfpathlineto{\pgfqpoint{0.711091in}{0.931044in}}%
\pgfpathlineto{\pgfqpoint{0.711657in}{0.932153in}}%
\pgfpathlineto{\pgfqpoint{0.712764in}{0.934020in}}%
\pgfpathlineto{\pgfqpoint{0.713427in}{0.935129in}}%
\pgfpathlineto{\pgfqpoint{0.714519in}{0.936977in}}%
\pgfpathlineto{\pgfqpoint{0.715170in}{0.938086in}}%
\pgfpathlineto{\pgfqpoint{0.716254in}{0.939798in}}%
\pgfpathlineto{\pgfqpoint{0.717055in}{0.940906in}}%
\pgfpathlineto{\pgfqpoint{0.718146in}{0.942579in}}%
\pgfpathlineto{\pgfqpoint{0.718839in}{0.943688in}}%
\pgfpathlineto{\pgfqpoint{0.719949in}{0.945546in}}%
\pgfpathlineto{\pgfqpoint{0.720677in}{0.946655in}}%
\pgfpathlineto{\pgfqpoint{0.721783in}{0.948386in}}%
\pgfpathlineto{\pgfqpoint{0.722432in}{0.949485in}}%
\pgfpathlineto{\pgfqpoint{0.723535in}{0.951206in}}%
\pgfpathlineto{\pgfqpoint{0.724277in}{0.952315in}}%
\pgfpathlineto{\pgfqpoint{0.725387in}{0.953862in}}%
\pgfpathlineto{\pgfqpoint{0.726129in}{0.954970in}}%
\pgfpathlineto{\pgfqpoint{0.727234in}{0.956818in}}%
\pgfpathlineto{\pgfqpoint{0.727976in}{0.957917in}}%
\pgfpathlineto{\pgfqpoint{0.729086in}{0.959785in}}%
\pgfpathlineto{\pgfqpoint{0.729921in}{0.960894in}}%
\pgfpathlineto{\pgfqpoint{0.731029in}{0.962674in}}%
\pgfpathlineto{\pgfqpoint{0.731722in}{0.963782in}}%
\pgfpathlineto{\pgfqpoint{0.732828in}{0.965582in}}%
\pgfpathlineto{\pgfqpoint{0.733675in}{0.966690in}}%
\pgfpathlineto{\pgfqpoint{0.734784in}{0.968324in}}%
\pgfpathlineto{\pgfqpoint{0.735555in}{0.969433in}}%
\pgfpathlineto{\pgfqpoint{0.736655in}{0.970960in}}%
\pgfpathlineto{\pgfqpoint{0.737379in}{0.972069in}}%
\pgfpathlineto{\pgfqpoint{0.738486in}{0.973606in}}%
\pgfpathlineto{\pgfqpoint{0.738489in}{0.973606in}}%
\pgfpathlineto{\pgfqpoint{0.740250in}{0.976212in}}%
\pgfpathlineto{\pgfqpoint{0.741113in}{0.977321in}}%
\pgfpathlineto{\pgfqpoint{0.742223in}{0.978566in}}%
\pgfpathlineto{\pgfqpoint{0.742965in}{0.979675in}}%
\pgfpathlineto{\pgfqpoint{0.744070in}{0.981221in}}%
\pgfpathlineto{\pgfqpoint{0.745001in}{0.982330in}}%
\pgfpathlineto{\pgfqpoint{0.746111in}{0.983984in}}%
\pgfpathlineto{\pgfqpoint{0.746907in}{0.985092in}}%
\pgfpathlineto{\pgfqpoint{0.748012in}{0.986921in}}%
\pgfpathlineto{\pgfqpoint{0.748668in}{0.988030in}}%
\pgfpathlineto{\pgfqpoint{0.749768in}{0.989897in}}%
\pgfpathlineto{\pgfqpoint{0.749775in}{0.989897in}}%
\pgfpathlineto{\pgfqpoint{0.750564in}{0.991006in}}%
\pgfpathlineto{\pgfqpoint{0.751648in}{0.992513in}}%
\pgfpathlineto{\pgfqpoint{0.752319in}{0.993622in}}%
\pgfpathlineto{\pgfqpoint{0.753426in}{0.995081in}}%
\pgfpathlineto{\pgfqpoint{0.754231in}{0.996190in}}%
\pgfpathlineto{\pgfqpoint{0.755341in}{0.997775in}}%
\pgfpathlineto{\pgfqpoint{0.756088in}{0.998884in}}%
\pgfpathlineto{\pgfqpoint{0.757193in}{1.000314in}}%
\pgfpathlineto{\pgfqpoint{0.757891in}{1.001423in}}%
\pgfpathlineto{\pgfqpoint{0.759001in}{1.002930in}}%
\pgfpathlineto{\pgfqpoint{0.759894in}{1.004039in}}%
\pgfpathlineto{\pgfqpoint{0.760997in}{1.005410in}}%
\pgfpathlineto{\pgfqpoint{0.761002in}{1.005410in}}%
\pgfpathlineto{\pgfqpoint{0.761819in}{1.006500in}}%
\pgfpathlineto{\pgfqpoint{0.762917in}{1.007968in}}%
\pgfpathlineto{\pgfqpoint{0.763782in}{1.009067in}}%
\pgfpathlineto{\pgfqpoint{0.764888in}{1.010721in}}%
\pgfpathlineto{\pgfqpoint{0.764892in}{1.010721in}}%
\pgfpathlineto{\pgfqpoint{0.766770in}{1.013814in}}%
\pgfpathlineto{\pgfqpoint{0.767647in}{1.014923in}}%
\pgfpathlineto{\pgfqpoint{0.768755in}{1.016411in}}%
\pgfpathlineto{\pgfqpoint{0.769543in}{1.017519in}}%
\pgfpathlineto{\pgfqpoint{0.770644in}{1.018969in}}%
\pgfpathlineto{\pgfqpoint{0.771521in}{1.020077in}}%
\pgfpathlineto{\pgfqpoint{0.772631in}{1.021780in}}%
\pgfpathlineto{\pgfqpoint{0.773348in}{1.022888in}}%
\pgfpathlineto{\pgfqpoint{0.774455in}{1.024503in}}%
\pgfpathlineto{\pgfqpoint{0.775258in}{1.025612in}}%
\pgfpathlineto{\pgfqpoint{0.776970in}{1.028033in}}%
\pgfpathlineto{\pgfqpoint{0.777957in}{1.029142in}}%
\pgfpathlineto{\pgfqpoint{0.779062in}{1.030543in}}%
\pgfpathlineto{\pgfqpoint{0.779967in}{1.031652in}}%
\pgfpathlineto{\pgfqpoint{0.781065in}{1.033149in}}%
\pgfpathlineto{\pgfqpoint{0.782124in}{1.034258in}}%
\pgfpathlineto{\pgfqpoint{0.783232in}{1.035600in}}%
\pgfpathlineto{\pgfqpoint{0.783862in}{1.036709in}}%
\pgfpathlineto{\pgfqpoint{0.784965in}{1.038236in}}%
\pgfpathlineto{\pgfqpoint{0.785826in}{1.039345in}}%
\pgfpathlineto{\pgfqpoint{0.787378in}{1.041125in}}%
\pgfpathlineto{\pgfqpoint{0.788029in}{1.042234in}}%
\pgfpathlineto{\pgfqpoint{0.789730in}{1.044587in}}%
\pgfpathlineto{\pgfqpoint{0.790863in}{1.045696in}}%
\pgfpathlineto{\pgfqpoint{0.791943in}{1.047097in}}%
\pgfpathlineto{\pgfqpoint{0.792799in}{1.048206in}}%
\pgfpathlineto{\pgfqpoint{0.793883in}{1.049577in}}%
\pgfpathlineto{\pgfqpoint{0.793904in}{1.049577in}}%
\pgfpathlineto{\pgfqpoint{0.794881in}{1.050686in}}%
\pgfpathlineto{\pgfqpoint{0.795989in}{1.052028in}}%
\pgfpathlineto{\pgfqpoint{0.795991in}{1.052028in}}%
\pgfpathlineto{\pgfqpoint{0.797632in}{1.054168in}}%
\pgfpathlineto{\pgfqpoint{0.798614in}{1.055277in}}%
\pgfpathlineto{\pgfqpoint{0.799723in}{1.056570in}}%
\pgfpathlineto{\pgfqpoint{0.800722in}{1.057679in}}%
\pgfpathlineto{\pgfqpoint{0.801831in}{1.059157in}}%
\pgfpathlineto{\pgfqpoint{0.802788in}{1.060266in}}%
\pgfpathlineto{\pgfqpoint{0.803893in}{1.061715in}}%
\pgfpathlineto{\pgfqpoint{0.804919in}{1.062814in}}%
\pgfpathlineto{\pgfqpoint{0.806015in}{1.064088in}}%
\pgfpathlineto{\pgfqpoint{0.806848in}{1.065188in}}%
\pgfpathlineto{\pgfqpoint{0.807955in}{1.066977in}}%
\pgfpathlineto{\pgfqpoint{0.808723in}{1.068086in}}%
\pgfpathlineto{\pgfqpoint{0.809793in}{1.069525in}}%
\pgfpathlineto{\pgfqpoint{0.809831in}{1.069525in}}%
\pgfpathlineto{\pgfqpoint{0.811001in}{1.070624in}}%
\pgfpathlineto{\pgfqpoint{0.812085in}{1.071860in}}%
\pgfpathlineto{\pgfqpoint{0.812958in}{1.072959in}}%
\pgfpathlineto{\pgfqpoint{0.812958in}{1.072968in}}%
\pgfpathlineto{\pgfqpoint{0.814068in}{1.074223in}}%
\pgfpathlineto{\pgfqpoint{0.815015in}{1.075332in}}%
\pgfpathlineto{\pgfqpoint{0.816125in}{1.076509in}}%
\pgfpathlineto{\pgfqpoint{0.817130in}{1.077618in}}%
\pgfpathlineto{\pgfqpoint{0.818235in}{1.078746in}}%
\pgfpathlineto{\pgfqpoint{0.819240in}{1.079845in}}%
\pgfpathlineto{\pgfqpoint{0.820341in}{1.081080in}}%
\pgfpathlineto{\pgfqpoint{0.821092in}{1.082189in}}%
\pgfpathlineto{\pgfqpoint{0.822202in}{1.083249in}}%
\pgfpathlineto{\pgfqpoint{0.823114in}{1.084358in}}%
\pgfpathlineto{\pgfqpoint{0.824224in}{1.085749in}}%
\pgfpathlineto{\pgfqpoint{0.825203in}{1.086857in}}%
\pgfpathlineto{\pgfqpoint{0.826304in}{1.087986in}}%
\pgfpathlineto{\pgfqpoint{0.827363in}{1.089094in}}%
\pgfpathlineto{\pgfqpoint{0.828472in}{1.090408in}}%
\pgfpathlineto{\pgfqpoint{0.829480in}{1.091516in}}%
\pgfpathlineto{\pgfqpoint{0.830587in}{1.092713in}}%
\pgfpathlineto{\pgfqpoint{0.831646in}{1.093821in}}%
\pgfpathlineto{\pgfqpoint{0.832749in}{1.094862in}}%
\pgfpathlineto{\pgfqpoint{0.833603in}{1.095971in}}%
\pgfpathlineto{\pgfqpoint{0.834706in}{1.097119in}}%
\pgfpathlineto{\pgfqpoint{0.834710in}{1.097119in}}%
\pgfpathlineto{\pgfqpoint{0.835711in}{1.098227in}}%
\pgfpathlineto{\pgfqpoint{0.836818in}{1.099472in}}%
\pgfpathlineto{\pgfqpoint{0.837726in}{1.100581in}}%
\pgfpathlineto{\pgfqpoint{0.838833in}{1.102011in}}%
\pgfpathlineto{\pgfqpoint{0.839713in}{1.103120in}}%
\pgfpathlineto{\pgfqpoint{0.840820in}{1.104365in}}%
\pgfpathlineto{\pgfqpoint{0.841625in}{1.105464in}}%
\pgfpathlineto{\pgfqpoint{0.842721in}{1.106436in}}%
\pgfpathlineto{\pgfqpoint{0.844010in}{1.107545in}}%
\pgfpathlineto{\pgfqpoint{0.845092in}{1.108683in}}%
\pgfpathlineto{\pgfqpoint{0.846177in}{1.109782in}}%
\pgfpathlineto{\pgfqpoint{0.847265in}{1.110881in}}%
\pgfpathlineto{\pgfqpoint{0.848264in}{1.111990in}}%
\pgfpathlineto{\pgfqpoint{0.849350in}{1.113186in}}%
\pgfpathlineto{\pgfqpoint{0.849362in}{1.113186in}}%
\pgfpathlineto{\pgfqpoint{0.850313in}{1.114295in}}%
\pgfpathlineto{\pgfqpoint{0.851391in}{1.115618in}}%
\pgfpathlineto{\pgfqpoint{0.851423in}{1.115618in}}%
\pgfpathlineto{\pgfqpoint{0.852440in}{1.116727in}}%
\pgfpathlineto{\pgfqpoint{0.853550in}{1.117981in}}%
\pgfpathlineto{\pgfqpoint{0.854609in}{1.119090in}}%
\pgfpathlineto{\pgfqpoint{0.855718in}{1.120413in}}%
\pgfpathlineto{\pgfqpoint{0.856831in}{1.121522in}}%
\pgfpathlineto{\pgfqpoint{0.857934in}{1.122864in}}%
\pgfpathlineto{\pgfqpoint{0.858750in}{1.123973in}}%
\pgfpathlineto{\pgfqpoint{0.859855in}{1.125295in}}%
\pgfpathlineto{\pgfqpoint{0.860956in}{1.126404in}}%
\pgfpathlineto{\pgfqpoint{0.862054in}{1.127562in}}%
\pgfpathlineto{\pgfqpoint{0.863313in}{1.128670in}}%
\pgfpathlineto{\pgfqpoint{0.864409in}{1.129876in}}%
\pgfpathlineto{\pgfqpoint{0.865395in}{1.130975in}}%
\pgfpathlineto{\pgfqpoint{0.866501in}{1.132298in}}%
\pgfpathlineto{\pgfqpoint{0.867652in}{1.133397in}}%
\pgfpathlineto{\pgfqpoint{0.868762in}{1.134671in}}%
\pgfpathlineto{\pgfqpoint{0.869853in}{1.135770in}}%
\pgfpathlineto{\pgfqpoint{0.870933in}{1.136870in}}%
\pgfpathlineto{\pgfqpoint{0.871950in}{1.137978in}}%
\pgfpathlineto{\pgfqpoint{0.873057in}{1.139145in}}%
\pgfpathlineto{\pgfqpoint{0.874235in}{1.140254in}}%
\pgfpathlineto{\pgfqpoint{0.875337in}{1.141392in}}%
\pgfpathlineto{\pgfqpoint{0.876373in}{1.142501in}}%
\pgfpathlineto{\pgfqpoint{0.877471in}{1.143853in}}%
\pgfpathlineto{\pgfqpoint{0.877476in}{1.143853in}}%
\pgfpathlineto{\pgfqpoint{0.878600in}{1.144962in}}%
\pgfpathlineto{\pgfqpoint{0.879702in}{1.146012in}}%
\pgfpathlineto{\pgfqpoint{0.879707in}{1.146012in}}%
\pgfpathlineto{\pgfqpoint{0.880866in}{1.147121in}}%
\pgfpathlineto{\pgfqpoint{0.881973in}{1.148132in}}%
\pgfpathlineto{\pgfqpoint{0.883246in}{1.149241in}}%
\pgfpathlineto{\pgfqpoint{0.884356in}{1.150331in}}%
\pgfpathlineto{\pgfqpoint{0.885542in}{1.151420in}}%
\pgfpathlineto{\pgfqpoint{0.886641in}{1.152548in}}%
\pgfpathlineto{\pgfqpoint{0.886650in}{1.152548in}}%
\pgfpathlineto{\pgfqpoint{0.887958in}{1.153657in}}%
\pgfpathlineto{\pgfqpoint{0.889067in}{1.154727in}}%
\pgfpathlineto{\pgfqpoint{0.890219in}{1.155816in}}%
\pgfpathlineto{\pgfqpoint{0.890219in}{1.155826in}}%
\pgfpathlineto{\pgfqpoint{0.891322in}{1.156954in}}%
\pgfpathlineto{\pgfqpoint{0.892441in}{1.158043in}}%
\pgfpathlineto{\pgfqpoint{0.893544in}{1.159172in}}%
\pgfpathlineto{\pgfqpoint{0.894905in}{1.160280in}}%
\pgfpathlineto{\pgfqpoint{0.896015in}{1.161321in}}%
\pgfpathlineto{\pgfqpoint{0.897134in}{1.162430in}}%
\pgfpathlineto{\pgfqpoint{0.898235in}{1.163480in}}%
\pgfpathlineto{\pgfqpoint{0.898244in}{1.163480in}}%
\pgfpathlineto{\pgfqpoint{0.899342in}{1.164589in}}%
\pgfpathlineto{\pgfqpoint{0.900450in}{1.165747in}}%
\pgfpathlineto{\pgfqpoint{0.901499in}{1.166846in}}%
\pgfpathlineto{\pgfqpoint{0.902609in}{1.167750in}}%
\pgfpathlineto{\pgfqpoint{0.903775in}{1.168849in}}%
\pgfpathlineto{\pgfqpoint{0.904854in}{1.169841in}}%
\pgfpathlineto{\pgfqpoint{0.906136in}{1.170950in}}%
\pgfpathlineto{\pgfqpoint{0.907204in}{1.172127in}}%
\pgfpathlineto{\pgfqpoint{0.907246in}{1.172127in}}%
\pgfpathlineto{\pgfqpoint{0.908740in}{1.173654in}}%
\pgfpathlineto{\pgfqpoint{0.909957in}{1.174763in}}%
\pgfpathlineto{\pgfqpoint{0.911043in}{1.175842in}}%
\pgfpathlineto{\pgfqpoint{0.912325in}{1.176951in}}%
\pgfpathlineto{\pgfqpoint{0.913431in}{1.177924in}}%
\pgfpathlineto{\pgfqpoint{0.913435in}{1.177924in}}%
\pgfpathlineto{\pgfqpoint{0.914664in}{1.179285in}}%
\pgfpathlineto{\pgfqpoint{0.916053in}{1.180394in}}%
\pgfpathlineto{\pgfqpoint{0.917156in}{1.181522in}}%
\pgfpathlineto{\pgfqpoint{0.918375in}{1.182621in}}%
\pgfpathlineto{\pgfqpoint{0.919462in}{1.183575in}}%
\pgfpathlineto{\pgfqpoint{0.920467in}{1.184683in}}%
\pgfpathlineto{\pgfqpoint{0.921572in}{1.185724in}}%
\pgfpathlineto{\pgfqpoint{0.923047in}{1.186833in}}%
\pgfpathlineto{\pgfqpoint{0.924529in}{1.188282in}}%
\pgfpathlineto{\pgfqpoint{0.925944in}{1.189391in}}%
\pgfpathlineto{\pgfqpoint{0.927347in}{1.190519in}}%
\pgfpathlineto{\pgfqpoint{0.928466in}{1.191628in}}%
\pgfpathlineto{\pgfqpoint{0.929574in}{1.192659in}}%
\pgfpathlineto{\pgfqpoint{0.931219in}{1.193768in}}%
\pgfpathlineto{\pgfqpoint{0.932326in}{1.194799in}}%
\pgfpathlineto{\pgfqpoint{0.933569in}{1.195898in}}%
\pgfpathlineto{\pgfqpoint{0.933569in}{1.195907in}}%
\pgfpathlineto{\pgfqpoint{0.935053in}{1.197357in}}%
\pgfpathlineto{\pgfqpoint{0.936184in}{1.198465in}}%
\pgfpathlineto{\pgfqpoint{0.937284in}{1.199467in}}%
\pgfpathlineto{\pgfqpoint{0.938455in}{1.200576in}}%
\pgfpathlineto{\pgfqpoint{0.939532in}{1.201432in}}%
\pgfpathlineto{\pgfqpoint{0.940905in}{1.202541in}}%
\pgfpathlineto{\pgfqpoint{0.942008in}{1.203494in}}%
\pgfpathlineto{\pgfqpoint{0.943385in}{1.204593in}}%
\pgfpathlineto{\pgfqpoint{0.944493in}{1.205566in}}%
\pgfpathlineto{\pgfqpoint{0.945579in}{1.206674in}}%
\pgfpathlineto{\pgfqpoint{0.946663in}{1.207676in}}%
\pgfpathlineto{\pgfqpoint{0.947943in}{1.208785in}}%
\pgfpathlineto{\pgfqpoint{0.949051in}{1.209709in}}%
\pgfpathlineto{\pgfqpoint{0.950151in}{1.210818in}}%
\pgfpathlineto{\pgfqpoint{0.951252in}{1.211693in}}%
\pgfpathlineto{\pgfqpoint{0.952727in}{1.212802in}}%
\pgfpathlineto{\pgfqpoint{0.953834in}{1.213580in}}%
\pgfpathlineto{\pgfqpoint{0.954960in}{1.214689in}}%
\pgfpathlineto{\pgfqpoint{0.956068in}{1.215700in}}%
\pgfpathlineto{\pgfqpoint{0.957327in}{1.216809in}}%
\pgfpathlineto{\pgfqpoint{0.958437in}{1.217782in}}%
\pgfpathlineto{\pgfqpoint{0.959802in}{1.218881in}}%
\pgfpathlineto{\pgfqpoint{0.960901in}{1.219669in}}%
\pgfpathlineto{\pgfqpoint{0.962266in}{1.220777in}}%
\pgfpathlineto{\pgfqpoint{0.963360in}{1.221653in}}%
\pgfpathlineto{\pgfqpoint{0.964740in}{1.222761in}}%
\pgfpathlineto{\pgfqpoint{0.965843in}{1.223870in}}%
\pgfpathlineto{\pgfqpoint{0.967080in}{1.224979in}}%
\pgfpathlineto{\pgfqpoint{0.968186in}{1.225835in}}%
\pgfpathlineto{\pgfqpoint{0.969565in}{1.226944in}}%
\pgfpathlineto{\pgfqpoint{0.970675in}{1.227722in}}%
\pgfpathlineto{\pgfqpoint{0.972022in}{1.228811in}}%
\pgfpathlineto{\pgfqpoint{0.973123in}{1.229852in}}%
\pgfpathlineto{\pgfqpoint{0.974310in}{1.230961in}}%
\pgfpathlineto{\pgfqpoint{0.975410in}{1.231943in}}%
\pgfpathlineto{\pgfqpoint{0.977102in}{1.233052in}}%
\pgfpathlineto{\pgfqpoint{0.978191in}{1.234015in}}%
\pgfpathlineto{\pgfqpoint{0.979791in}{1.235123in}}%
\pgfpathlineto{\pgfqpoint{0.980890in}{1.236047in}}%
\pgfpathlineto{\pgfqpoint{0.982244in}{1.237156in}}%
\pgfpathlineto{\pgfqpoint{0.983347in}{1.238090in}}%
\pgfpathlineto{\pgfqpoint{0.984961in}{1.239199in}}%
\pgfpathlineto{\pgfqpoint{0.986067in}{1.240045in}}%
\pgfpathlineto{\pgfqpoint{0.987542in}{1.241154in}}%
\pgfpathlineto{\pgfqpoint{0.988649in}{1.241922in}}%
\pgfpathlineto{\pgfqpoint{0.989910in}{1.243031in}}%
\pgfpathlineto{\pgfqpoint{0.990995in}{1.244120in}}%
\pgfpathlineto{\pgfqpoint{0.991011in}{1.244120in}}%
\pgfpathlineto{\pgfqpoint{0.992188in}{1.245229in}}%
\pgfpathlineto{\pgfqpoint{0.993291in}{1.246095in}}%
\pgfpathlineto{\pgfqpoint{0.993298in}{1.246095in}}%
\pgfpathlineto{\pgfqpoint{0.994673in}{1.247203in}}%
\pgfpathlineto{\pgfqpoint{0.995783in}{1.248166in}}%
\pgfpathlineto{\pgfqpoint{0.996953in}{1.249275in}}%
\pgfpathlineto{\pgfqpoint{0.998052in}{1.250238in}}%
\pgfpathlineto{\pgfqpoint{0.998059in}{1.250238in}}%
\pgfpathlineto{\pgfqpoint{0.999592in}{1.251347in}}%
\pgfpathlineto{\pgfqpoint{1.000697in}{1.252047in}}%
\pgfpathlineto{\pgfqpoint{1.002072in}{1.253156in}}%
\pgfpathlineto{\pgfqpoint{1.003147in}{1.253924in}}%
\pgfpathlineto{\pgfqpoint{1.003180in}{1.253924in}}%
\pgfpathlineto{\pgfqpoint{1.004618in}{1.255033in}}%
\pgfpathlineto{\pgfqpoint{1.005723in}{1.255821in}}%
\pgfpathlineto{\pgfqpoint{1.007414in}{1.256930in}}%
\pgfpathlineto{\pgfqpoint{1.008785in}{1.257892in}}%
\pgfpathlineto{\pgfqpoint{1.010085in}{1.258991in}}%
\pgfpathlineto{\pgfqpoint{1.011156in}{1.259886in}}%
\pgfpathlineto{\pgfqpoint{1.011177in}{1.259886in}}%
\pgfpathlineto{\pgfqpoint{1.013057in}{1.260995in}}%
\pgfpathlineto{\pgfqpoint{1.014164in}{1.261705in}}%
\pgfpathlineto{\pgfqpoint{1.015544in}{1.262814in}}%
\pgfpathlineto{\pgfqpoint{1.016635in}{1.263786in}}%
\pgfpathlineto{\pgfqpoint{1.017992in}{1.264895in}}%
\pgfpathlineto{\pgfqpoint{1.019087in}{1.265712in}}%
\pgfpathlineto{\pgfqpoint{1.020646in}{1.266821in}}%
\pgfpathlineto{\pgfqpoint{1.021749in}{1.267453in}}%
\pgfpathlineto{\pgfqpoint{1.023536in}{1.268562in}}%
\pgfpathlineto{\pgfqpoint{1.024644in}{1.269253in}}%
\pgfpathlineto{\pgfqpoint{1.024646in}{1.269253in}}%
\pgfpathlineto{\pgfqpoint{1.026133in}{1.270361in}}%
\pgfpathlineto{\pgfqpoint{1.027243in}{1.271227in}}%
\pgfpathlineto{\pgfqpoint{1.028695in}{1.272336in}}%
\pgfpathlineto{\pgfqpoint{1.029802in}{1.273065in}}%
\pgfpathlineto{\pgfqpoint{1.031210in}{1.274164in}}%
\pgfpathlineto{\pgfqpoint{1.032308in}{1.274962in}}%
\pgfpathlineto{\pgfqpoint{1.033769in}{1.276071in}}%
\pgfpathlineto{\pgfqpoint{1.034874in}{1.276820in}}%
\pgfpathlineto{\pgfqpoint{1.036631in}{1.277919in}}%
\pgfpathlineto{\pgfqpoint{1.037741in}{1.278502in}}%
\pgfpathlineto{\pgfqpoint{1.039142in}{1.279611in}}%
\pgfpathlineto{\pgfqpoint{1.040251in}{1.280399in}}%
\pgfpathlineto{\pgfqpoint{1.041887in}{1.281508in}}%
\pgfpathlineto{\pgfqpoint{1.042995in}{1.282266in}}%
\pgfpathlineto{\pgfqpoint{1.044514in}{1.283375in}}%
\pgfpathlineto{\pgfqpoint{1.045573in}{1.284056in}}%
\pgfpathlineto{\pgfqpoint{1.047334in}{1.285165in}}%
\pgfpathlineto{\pgfqpoint{1.048388in}{1.285923in}}%
\pgfpathlineto{\pgfqpoint{1.048411in}{1.285923in}}%
\pgfpathlineto{\pgfqpoint{1.050035in}{1.287032in}}%
\pgfpathlineto{\pgfqpoint{1.051134in}{1.287723in}}%
\pgfpathlineto{\pgfqpoint{1.051145in}{1.287723in}}%
\pgfpathlineto{\pgfqpoint{1.052981in}{1.288831in}}%
\pgfpathlineto{\pgfqpoint{1.054068in}{1.289580in}}%
\pgfpathlineto{\pgfqpoint{1.054086in}{1.289580in}}%
\pgfpathlineto{\pgfqpoint{1.055955in}{1.290689in}}%
\pgfpathlineto{\pgfqpoint{1.057043in}{1.291380in}}%
\pgfpathlineto{\pgfqpoint{1.057060in}{1.291380in}}%
\pgfpathlineto{\pgfqpoint{1.059042in}{1.292488in}}%
\pgfpathlineto{\pgfqpoint{1.060103in}{1.293169in}}%
\pgfpathlineto{\pgfqpoint{1.061581in}{1.294278in}}%
\pgfpathlineto{\pgfqpoint{1.062639in}{1.295105in}}%
\pgfpathlineto{\pgfqpoint{1.064394in}{1.296214in}}%
\pgfpathlineto{\pgfqpoint{1.065501in}{1.297011in}}%
\pgfpathlineto{\pgfqpoint{1.067197in}{1.298120in}}%
\pgfpathlineto{\pgfqpoint{1.068272in}{1.299024in}}%
\pgfpathlineto{\pgfqpoint{1.070199in}{1.300133in}}%
\pgfpathlineto{\pgfqpoint{1.071306in}{1.300970in}}%
\pgfpathlineto{\pgfqpoint{1.073177in}{1.302078in}}%
\pgfpathlineto{\pgfqpoint{1.074287in}{1.302730in}}%
\pgfpathlineto{\pgfqpoint{1.075864in}{1.303839in}}%
\pgfpathlineto{\pgfqpoint{1.076956in}{1.304393in}}%
\pgfpathlineto{\pgfqpoint{1.078640in}{1.305502in}}%
\pgfpathlineto{\pgfqpoint{1.079727in}{1.306270in}}%
\pgfpathlineto{\pgfqpoint{1.081597in}{1.307379in}}%
\pgfpathlineto{\pgfqpoint{1.082658in}{1.307943in}}%
\pgfpathlineto{\pgfqpoint{1.084278in}{1.309052in}}%
\pgfpathlineto{\pgfqpoint{1.085343in}{1.309616in}}%
\pgfpathlineto{\pgfqpoint{1.087114in}{1.310725in}}%
\pgfpathlineto{\pgfqpoint{1.088478in}{1.311678in}}%
\pgfpathlineto{\pgfqpoint{1.090576in}{1.312787in}}%
\pgfpathlineto{\pgfqpoint{1.091684in}{1.313555in}}%
\pgfpathlineto{\pgfqpoint{1.093154in}{1.314664in}}%
\pgfpathlineto{\pgfqpoint{1.094252in}{1.315403in}}%
\pgfpathlineto{\pgfqpoint{1.096153in}{1.316512in}}%
\pgfpathlineto{\pgfqpoint{1.097238in}{1.317144in}}%
\pgfpathlineto{\pgfqpoint{1.099115in}{1.318243in}}%
\pgfpathlineto{\pgfqpoint{1.100218in}{1.318983in}}%
\pgfpathlineto{\pgfqpoint{1.102305in}{1.320091in}}%
\pgfpathlineto{\pgfqpoint{1.103385in}{1.320724in}}%
\pgfpathlineto{\pgfqpoint{1.105395in}{1.321832in}}%
\pgfpathlineto{\pgfqpoint{1.106496in}{1.322494in}}%
\pgfpathlineto{\pgfqpoint{1.108576in}{1.323603in}}%
\pgfpathlineto{\pgfqpoint{1.109639in}{1.324137in}}%
\pgfpathlineto{\pgfqpoint{1.111396in}{1.325246in}}%
\pgfpathlineto{\pgfqpoint{1.112506in}{1.325771in}}%
\pgfpathlineto{\pgfqpoint{1.114435in}{1.326880in}}%
\pgfpathlineto{\pgfqpoint{1.115540in}{1.327697in}}%
\pgfpathlineto{\pgfqpoint{1.117557in}{1.328806in}}%
\pgfpathlineto{\pgfqpoint{1.118648in}{1.329448in}}%
\pgfpathlineto{\pgfqpoint{1.120654in}{1.330547in}}%
\pgfpathlineto{\pgfqpoint{1.121754in}{1.331160in}}%
\pgfpathlineto{\pgfqpoint{1.123769in}{1.332269in}}%
\pgfpathlineto{\pgfqpoint{1.124847in}{1.332774in}}%
\pgfpathlineto{\pgfqpoint{1.124865in}{1.332774in}}%
\pgfpathlineto{\pgfqpoint{1.126908in}{1.333883in}}%
\pgfpathlineto{\pgfqpoint{1.128018in}{1.334603in}}%
\pgfpathlineto{\pgfqpoint{1.129775in}{1.335712in}}%
\pgfpathlineto{\pgfqpoint{1.130882in}{1.336412in}}%
\pgfpathlineto{\pgfqpoint{1.132806in}{1.337521in}}%
\pgfpathlineto{\pgfqpoint{1.133907in}{1.338202in}}%
\pgfpathlineto{\pgfqpoint{1.135999in}{1.339310in}}%
\pgfpathlineto{\pgfqpoint{1.137099in}{1.339855in}}%
\pgfpathlineto{\pgfqpoint{1.139205in}{1.340964in}}%
\pgfpathlineto{\pgfqpoint{1.140315in}{1.341654in}}%
\pgfpathlineto{\pgfqpoint{1.142348in}{1.342763in}}%
\pgfpathlineto{\pgfqpoint{1.143395in}{1.343269in}}%
\pgfpathlineto{\pgfqpoint{1.145843in}{1.344358in}}%
\pgfpathlineto{\pgfqpoint{1.146939in}{1.345049in}}%
\pgfpathlineto{\pgfqpoint{1.148828in}{1.346158in}}%
\pgfpathlineto{\pgfqpoint{1.149929in}{1.346848in}}%
\pgfpathlineto{\pgfqpoint{1.149938in}{1.346848in}}%
\pgfpathlineto{\pgfqpoint{1.151262in}{1.347646in}}%
\pgfpathlineto{\pgfqpoint{1.152900in}{1.348754in}}%
\pgfpathlineto{\pgfqpoint{1.153991in}{1.349270in}}%
\pgfpathlineto{\pgfqpoint{1.155892in}{1.350379in}}%
\pgfpathlineto{\pgfqpoint{1.156997in}{1.350953in}}%
\pgfpathlineto{\pgfqpoint{1.159352in}{1.352061in}}%
\pgfpathlineto{\pgfqpoint{1.160450in}{1.352499in}}%
\pgfpathlineto{\pgfqpoint{1.160462in}{1.352499in}}%
\pgfpathlineto{\pgfqpoint{1.162777in}{1.353598in}}%
\pgfpathlineto{\pgfqpoint{1.163838in}{1.354230in}}%
\pgfpathlineto{\pgfqpoint{1.163866in}{1.354230in}}%
\pgfpathlineto{\pgfqpoint{1.165848in}{1.355339in}}%
\pgfpathlineto{\pgfqpoint{1.167170in}{1.356098in}}%
\pgfpathlineto{\pgfqpoint{1.169373in}{1.357206in}}%
\pgfpathlineto{\pgfqpoint{1.170474in}{1.357810in}}%
\pgfpathlineto{\pgfqpoint{1.172545in}{1.358918in}}%
\pgfpathlineto{\pgfqpoint{1.173648in}{1.359434in}}%
\pgfpathlineto{\pgfqpoint{1.175625in}{1.360543in}}%
\pgfpathlineto{\pgfqpoint{1.176630in}{1.361058in}}%
\pgfpathlineto{\pgfqpoint{1.178876in}{1.362167in}}%
\pgfpathlineto{\pgfqpoint{1.179967in}{1.362673in}}%
\pgfpathlineto{\pgfqpoint{1.181779in}{1.363781in}}%
\pgfpathlineto{\pgfqpoint{1.182750in}{1.364307in}}%
\pgfpathlineto{\pgfqpoint{1.182840in}{1.364307in}}%
\pgfpathlineto{\pgfqpoint{1.185118in}{1.365415in}}%
\pgfpathlineto{\pgfqpoint{1.186223in}{1.365902in}}%
\pgfpathlineto{\pgfqpoint{1.188387in}{1.367010in}}%
\pgfpathlineto{\pgfqpoint{1.189486in}{1.367448in}}%
\pgfpathlineto{\pgfqpoint{1.191512in}{1.368557in}}%
\pgfpathlineto{\pgfqpoint{1.192613in}{1.369248in}}%
\pgfpathlineto{\pgfqpoint{1.195226in}{1.370347in}}%
\pgfpathlineto{\pgfqpoint{1.196331in}{1.370988in}}%
\pgfpathlineto{\pgfqpoint{1.199118in}{1.372097in}}%
\pgfpathlineto{\pgfqpoint{1.200212in}{1.372535in}}%
\pgfpathlineto{\pgfqpoint{1.202441in}{1.373644in}}%
\pgfpathlineto{\pgfqpoint{1.203548in}{1.374188in}}%
\pgfpathlineto{\pgfqpoint{1.205784in}{1.375297in}}%
\pgfpathlineto{\pgfqpoint{1.206885in}{1.375871in}}%
\pgfpathlineto{\pgfqpoint{1.209630in}{1.376980in}}%
\pgfpathlineto{\pgfqpoint{1.210933in}{1.377661in}}%
\pgfpathlineto{\pgfqpoint{1.213374in}{1.378769in}}%
\pgfpathlineto{\pgfqpoint{1.214440in}{1.379236in}}%
\pgfpathlineto{\pgfqpoint{1.214482in}{1.379236in}}%
\pgfpathlineto{\pgfqpoint{1.216434in}{1.380345in}}%
\pgfpathlineto{\pgfqpoint{1.217520in}{1.381006in}}%
\pgfpathlineto{\pgfqpoint{1.220271in}{1.382115in}}%
\pgfpathlineto{\pgfqpoint{1.221380in}{1.382738in}}%
\pgfpathlineto{\pgfqpoint{1.223823in}{1.383837in}}%
\pgfpathlineto{\pgfqpoint{1.224919in}{1.384381in}}%
\pgfpathlineto{\pgfqpoint{1.227490in}{1.385490in}}%
\pgfpathlineto{\pgfqpoint{1.228570in}{1.386006in}}%
\pgfpathlineto{\pgfqpoint{1.230850in}{1.387115in}}%
\pgfpathlineto{\pgfqpoint{1.231955in}{1.387649in}}%
\pgfpathlineto{\pgfqpoint{1.231960in}{1.387649in}}%
\pgfpathlineto{\pgfqpoint{1.234664in}{1.388749in}}%
\pgfpathlineto{\pgfqpoint{1.235769in}{1.389303in}}%
\pgfpathlineto{\pgfqpoint{1.238305in}{1.390412in}}%
\pgfpathlineto{\pgfqpoint{1.239403in}{1.390976in}}%
\pgfpathlineto{\pgfqpoint{1.241672in}{1.392085in}}%
\pgfpathlineto{\pgfqpoint{1.242924in}{1.392590in}}%
\pgfpathlineto{\pgfqpoint{1.245039in}{1.393699in}}%
\pgfpathlineto{\pgfqpoint{1.246116in}{1.394234in}}%
\pgfpathlineto{\pgfqpoint{1.248612in}{1.395343in}}%
\pgfpathlineto{\pgfqpoint{1.249720in}{1.395858in}}%
\pgfpathlineto{\pgfqpoint{1.249722in}{1.395858in}}%
\pgfpathlineto{\pgfqpoint{1.252033in}{1.396967in}}%
\pgfpathlineto{\pgfqpoint{1.253138in}{1.397385in}}%
\pgfpathlineto{\pgfqpoint{1.255881in}{1.398484in}}%
\pgfpathlineto{\pgfqpoint{1.256954in}{1.399058in}}%
\pgfpathlineto{\pgfqpoint{1.259764in}{1.400157in}}%
\pgfpathlineto{\pgfqpoint{1.260839in}{1.400858in}}%
\pgfpathlineto{\pgfqpoint{1.263164in}{1.401957in}}%
\pgfpathlineto{\pgfqpoint{1.264185in}{1.402462in}}%
\pgfpathlineto{\pgfqpoint{1.266505in}{1.403571in}}%
\pgfpathlineto{\pgfqpoint{1.267610in}{1.404096in}}%
\pgfpathlineto{\pgfqpoint{1.270477in}{1.405195in}}%
\pgfpathlineto{\pgfqpoint{1.271570in}{1.405653in}}%
\pgfpathlineto{\pgfqpoint{1.273997in}{1.406761in}}%
\pgfpathlineto{\pgfqpoint{1.275105in}{1.407335in}}%
\pgfpathlineto{\pgfqpoint{1.277871in}{1.408444in}}%
\pgfpathlineto{\pgfqpoint{1.278965in}{1.408833in}}%
\pgfpathlineto{\pgfqpoint{1.278976in}{1.408833in}}%
\pgfpathlineto{\pgfqpoint{1.281477in}{1.409942in}}%
\pgfpathlineto{\pgfqpoint{1.282580in}{1.410448in}}%
\pgfpathlineto{\pgfqpoint{1.285556in}{1.411556in}}%
\pgfpathlineto{\pgfqpoint{1.286654in}{1.411994in}}%
\pgfpathlineto{\pgfqpoint{1.289156in}{1.413093in}}%
\pgfpathlineto{\pgfqpoint{1.290231in}{1.413521in}}%
\pgfpathlineto{\pgfqpoint{1.293421in}{1.414630in}}%
\pgfpathlineto{\pgfqpoint{1.294530in}{1.415048in}}%
\pgfpathlineto{\pgfqpoint{1.297516in}{1.416157in}}%
\pgfpathlineto{\pgfqpoint{1.298565in}{1.416546in}}%
\pgfpathlineto{\pgfqpoint{1.298605in}{1.416546in}}%
\pgfpathlineto{\pgfqpoint{1.301355in}{1.417655in}}%
\pgfpathlineto{\pgfqpoint{1.302425in}{1.418054in}}%
\pgfpathlineto{\pgfqpoint{1.302453in}{1.418054in}}%
\pgfpathlineto{\pgfqpoint{1.305708in}{1.419162in}}%
\pgfpathlineto{\pgfqpoint{1.306816in}{1.419600in}}%
\pgfpathlineto{\pgfqpoint{1.309217in}{1.420699in}}%
\pgfpathlineto{\pgfqpoint{1.310313in}{1.421098in}}%
\pgfpathlineto{\pgfqpoint{1.312944in}{1.422207in}}%
\pgfpathlineto{\pgfqpoint{1.314042in}{1.422664in}}%
\pgfpathlineto{\pgfqpoint{1.317153in}{1.423763in}}%
\pgfpathlineto{\pgfqpoint{1.318247in}{1.424181in}}%
\pgfpathlineto{\pgfqpoint{1.321728in}{1.425290in}}%
\pgfpathlineto{\pgfqpoint{1.322823in}{1.425708in}}%
\pgfpathlineto{\pgfqpoint{1.325636in}{1.426817in}}%
\pgfpathlineto{\pgfqpoint{1.326716in}{1.427235in}}%
\pgfpathlineto{\pgfqpoint{1.329962in}{1.428344in}}%
\pgfpathlineto{\pgfqpoint{1.331041in}{1.428811in}}%
\pgfpathlineto{\pgfqpoint{1.331051in}{1.428811in}}%
\pgfpathlineto{\pgfqpoint{1.334243in}{1.429919in}}%
\pgfpathlineto{\pgfqpoint{1.335327in}{1.430328in}}%
\pgfpathlineto{\pgfqpoint{1.337747in}{1.431437in}}%
\pgfpathlineto{\pgfqpoint{1.338845in}{1.431855in}}%
\pgfpathlineto{\pgfqpoint{1.341495in}{1.432964in}}%
\pgfpathlineto{\pgfqpoint{1.342561in}{1.433353in}}%
\pgfpathlineto{\pgfqpoint{1.345893in}{1.434452in}}%
\pgfpathlineto{\pgfqpoint{1.346989in}{1.434948in}}%
\pgfpathlineto{\pgfqpoint{1.350323in}{1.436057in}}%
\pgfpathlineto{\pgfqpoint{1.351384in}{1.436465in}}%
\pgfpathlineto{\pgfqpoint{1.354418in}{1.437574in}}%
\pgfpathlineto{\pgfqpoint{1.355491in}{1.437924in}}%
\pgfpathlineto{\pgfqpoint{1.355502in}{1.437924in}}%
\pgfpathlineto{\pgfqpoint{1.357955in}{1.439033in}}%
\pgfpathlineto{\pgfqpoint{1.359025in}{1.439412in}}%
\pgfpathlineto{\pgfqpoint{1.361810in}{1.440521in}}%
\pgfpathlineto{\pgfqpoint{1.362801in}{1.440842in}}%
\pgfpathlineto{\pgfqpoint{1.365998in}{1.441951in}}%
\pgfpathlineto{\pgfqpoint{1.367108in}{1.442379in}}%
\pgfpathlineto{\pgfqpoint{1.370033in}{1.443487in}}%
\pgfpathlineto{\pgfqpoint{1.371110in}{1.443964in}}%
\pgfpathlineto{\pgfqpoint{1.373825in}{1.445073in}}%
\pgfpathlineto{\pgfqpoint{1.374905in}{1.445452in}}%
\pgfpathlineto{\pgfqpoint{1.377595in}{1.446561in}}%
\pgfpathlineto{\pgfqpoint{1.378618in}{1.446901in}}%
\pgfpathlineto{\pgfqpoint{1.382057in}{1.448010in}}%
\pgfpathlineto{\pgfqpoint{1.383158in}{1.448302in}}%
\pgfpathlineto{\pgfqpoint{1.386932in}{1.449411in}}%
\pgfpathlineto{\pgfqpoint{1.388028in}{1.449722in}}%
\pgfpathlineto{\pgfqpoint{1.391320in}{1.450831in}}%
\pgfpathlineto{\pgfqpoint{1.392418in}{1.451220in}}%
\pgfpathlineto{\pgfqpoint{1.395876in}{1.452319in}}%
\pgfpathlineto{\pgfqpoint{1.396981in}{1.452669in}}%
\pgfpathlineto{\pgfqpoint{1.396983in}{1.452669in}}%
\pgfpathlineto{\pgfqpoint{1.400145in}{1.453778in}}%
\pgfpathlineto{\pgfqpoint{1.401181in}{1.454089in}}%
\pgfpathlineto{\pgfqpoint{1.404741in}{1.455198in}}%
\pgfpathlineto{\pgfqpoint{1.405839in}{1.455616in}}%
\pgfpathlineto{\pgfqpoint{1.409338in}{1.456725in}}%
\pgfpathlineto{\pgfqpoint{1.410448in}{1.457085in}}%
\pgfpathlineto{\pgfqpoint{1.413861in}{1.458193in}}%
\pgfpathlineto{\pgfqpoint{1.414955in}{1.458592in}}%
\pgfpathlineto{\pgfqpoint{1.418513in}{1.459701in}}%
\pgfpathlineto{\pgfqpoint{1.419622in}{1.459983in}}%
\pgfpathlineto{\pgfqpoint{1.422871in}{1.461092in}}%
\pgfpathlineto{\pgfqpoint{1.423894in}{1.461316in}}%
\pgfpathlineto{\pgfqpoint{1.427508in}{1.462424in}}%
\pgfpathlineto{\pgfqpoint{1.428611in}{1.462823in}}%
\pgfpathlineto{\pgfqpoint{1.432471in}{1.463932in}}%
\pgfpathlineto{\pgfqpoint{1.433578in}{1.464224in}}%
\pgfpathlineto{\pgfqpoint{1.436833in}{1.465332in}}%
\pgfpathlineto{\pgfqpoint{1.437897in}{1.465624in}}%
\pgfpathlineto{\pgfqpoint{1.437936in}{1.465624in}}%
\pgfpathlineto{\pgfqpoint{1.441470in}{1.466733in}}%
\pgfpathlineto{\pgfqpoint{1.442557in}{1.467112in}}%
\pgfpathlineto{\pgfqpoint{1.445393in}{1.468221in}}%
\pgfpathlineto{\pgfqpoint{1.446480in}{1.468610in}}%
\pgfpathlineto{\pgfqpoint{1.450359in}{1.469719in}}%
\pgfpathlineto{\pgfqpoint{1.451392in}{1.470050in}}%
\pgfpathlineto{\pgfqpoint{1.455461in}{1.471158in}}%
\pgfpathlineto{\pgfqpoint{1.456524in}{1.471489in}}%
\pgfpathlineto{\pgfqpoint{1.460187in}{1.472598in}}%
\pgfpathlineto{\pgfqpoint{1.461273in}{1.472968in}}%
\pgfpathlineto{\pgfqpoint{1.465284in}{1.474076in}}%
\pgfpathlineto{\pgfqpoint{1.466334in}{1.474378in}}%
\pgfpathlineto{\pgfqpoint{1.466392in}{1.474378in}}%
\pgfpathlineto{\pgfqpoint{1.470494in}{1.475487in}}%
\pgfpathlineto{\pgfqpoint{1.471532in}{1.475740in}}%
\pgfpathlineto{\pgfqpoint{1.476008in}{1.476848in}}%
\pgfpathlineto{\pgfqpoint{1.477088in}{1.477101in}}%
\pgfpathlineto{\pgfqpoint{1.481104in}{1.478210in}}%
\pgfpathlineto{\pgfqpoint{1.482193in}{1.478511in}}%
\pgfpathlineto{\pgfqpoint{1.485925in}{1.479620in}}%
\pgfpathlineto{\pgfqpoint{1.487137in}{1.480029in}}%
\pgfpathlineto{\pgfqpoint{1.491204in}{1.481138in}}%
\pgfpathlineto{\pgfqpoint{1.492223in}{1.481361in}}%
\pgfpathlineto{\pgfqpoint{1.492275in}{1.481361in}}%
\pgfpathlineto{\pgfqpoint{1.496816in}{1.482470in}}%
\pgfpathlineto{\pgfqpoint{1.497873in}{1.482742in}}%
\pgfpathlineto{\pgfqpoint{1.497901in}{1.482742in}}%
\pgfpathlineto{\pgfqpoint{1.501803in}{1.483851in}}%
\pgfpathlineto{\pgfqpoint{1.502901in}{1.484240in}}%
\pgfpathlineto{\pgfqpoint{1.502910in}{1.484240in}}%
\pgfpathlineto{\pgfqpoint{1.506505in}{1.485349in}}%
\pgfpathlineto{\pgfqpoint{1.507608in}{1.485660in}}%
\pgfpathlineto{\pgfqpoint{1.512073in}{1.486769in}}%
\pgfpathlineto{\pgfqpoint{1.513157in}{1.486944in}}%
\pgfpathlineto{\pgfqpoint{1.516994in}{1.488053in}}%
\pgfpathlineto{\pgfqpoint{1.518097in}{1.488422in}}%
\pgfpathlineto{\pgfqpoint{1.521833in}{1.489531in}}%
\pgfpathlineto{\pgfqpoint{1.522920in}{1.489813in}}%
\pgfpathlineto{\pgfqpoint{1.527364in}{1.490903in}}%
\pgfpathlineto{\pgfqpoint{1.528448in}{1.491263in}}%
\pgfpathlineto{\pgfqpoint{1.532697in}{1.492371in}}%
\pgfpathlineto{\pgfqpoint{1.533807in}{1.492634in}}%
\pgfpathlineto{\pgfqpoint{1.538507in}{1.493743in}}%
\pgfpathlineto{\pgfqpoint{1.539605in}{1.494044in}}%
\pgfpathlineto{\pgfqpoint{1.544140in}{1.495153in}}%
\pgfpathlineto{\pgfqpoint{1.545238in}{1.495503in}}%
\pgfpathlineto{\pgfqpoint{1.549454in}{1.496612in}}%
\pgfpathlineto{\pgfqpoint{1.550559in}{1.496923in}}%
\pgfpathlineto{\pgfqpoint{1.555506in}{1.498032in}}%
\pgfpathlineto{\pgfqpoint{1.556581in}{1.498256in}}%
\pgfpathlineto{\pgfqpoint{1.560934in}{1.499364in}}%
\pgfpathlineto{\pgfqpoint{1.562014in}{1.499608in}}%
\pgfpathlineto{\pgfqpoint{1.567109in}{1.500716in}}%
\pgfpathlineto{\pgfqpoint{1.568182in}{1.500989in}}%
\pgfpathlineto{\pgfqpoint{1.568219in}{1.500989in}}%
\pgfpathlineto{\pgfqpoint{1.572938in}{1.502097in}}%
\pgfpathlineto{\pgfqpoint{1.574045in}{1.502331in}}%
\pgfpathlineto{\pgfqpoint{1.578857in}{1.503440in}}%
\pgfpathlineto{\pgfqpoint{1.579860in}{1.503741in}}%
\pgfpathlineto{\pgfqpoint{1.579869in}{1.503741in}}%
\pgfpathlineto{\pgfqpoint{1.585206in}{1.504850in}}%
\pgfpathlineto{\pgfqpoint{1.586244in}{1.505015in}}%
\pgfpathlineto{\pgfqpoint{1.586295in}{1.505015in}}%
\pgfpathlineto{\pgfqpoint{1.591705in}{1.506124in}}%
\pgfpathlineto{\pgfqpoint{1.592789in}{1.506328in}}%
\pgfpathlineto{\pgfqpoint{1.598057in}{1.507427in}}%
\pgfpathlineto{\pgfqpoint{1.599155in}{1.507602in}}%
\pgfpathlineto{\pgfqpoint{1.604914in}{1.508711in}}%
\pgfpathlineto{\pgfqpoint{1.605859in}{1.508906in}}%
\pgfpathlineto{\pgfqpoint{1.606012in}{1.508906in}}%
\pgfpathlineto{\pgfqpoint{1.611333in}{1.510015in}}%
\pgfpathlineto{\pgfqpoint{1.612413in}{1.510248in}}%
\pgfpathlineto{\pgfqpoint{1.612432in}{1.510248in}}%
\pgfpathlineto{\pgfqpoint{1.617762in}{1.511357in}}%
\pgfpathlineto{\pgfqpoint{1.618844in}{1.511590in}}%
\pgfpathlineto{\pgfqpoint{1.618863in}{1.511590in}}%
\pgfpathlineto{\pgfqpoint{1.624798in}{1.512699in}}%
\pgfpathlineto{\pgfqpoint{1.625899in}{1.512903in}}%
\pgfpathlineto{\pgfqpoint{1.630843in}{1.514012in}}%
\pgfpathlineto{\pgfqpoint{1.631948in}{1.514245in}}%
\pgfpathlineto{\pgfqpoint{1.637984in}{1.515354in}}%
\pgfpathlineto{\pgfqpoint{1.639082in}{1.515568in}}%
\pgfpathlineto{\pgfqpoint{1.645573in}{1.516677in}}%
\pgfpathlineto{\pgfqpoint{1.646660in}{1.516891in}}%
\pgfpathlineto{\pgfqpoint{1.652123in}{1.518000in}}%
\pgfpathlineto{\pgfqpoint{1.653219in}{1.518243in}}%
\pgfpathlineto{\pgfqpoint{1.659950in}{1.519352in}}%
\pgfpathlineto{\pgfqpoint{1.661058in}{1.519634in}}%
\pgfpathlineto{\pgfqpoint{1.668152in}{1.520743in}}%
\pgfpathlineto{\pgfqpoint{1.669241in}{1.520947in}}%
\pgfpathlineto{\pgfqpoint{1.669248in}{1.520947in}}%
\pgfpathlineto{\pgfqpoint{1.675202in}{1.522056in}}%
\pgfpathlineto{\pgfqpoint{1.676268in}{1.522192in}}%
\pgfpathlineto{\pgfqpoint{1.676307in}{1.522192in}}%
\pgfpathlineto{\pgfqpoint{1.683813in}{1.523301in}}%
\pgfpathlineto{\pgfqpoint{1.684865in}{1.523446in}}%
\pgfpathlineto{\pgfqpoint{1.684870in}{1.523446in}}%
\pgfpathlineto{\pgfqpoint{1.692753in}{1.524555in}}%
\pgfpathlineto{\pgfqpoint{1.693732in}{1.524740in}}%
\pgfpathlineto{\pgfqpoint{1.693853in}{1.524740in}}%
\pgfpathlineto{\pgfqpoint{1.701368in}{1.525849in}}%
\pgfpathlineto{\pgfqpoint{1.702434in}{1.526034in}}%
\pgfpathlineto{\pgfqpoint{1.709568in}{1.527142in}}%
\pgfpathlineto{\pgfqpoint{1.710629in}{1.527376in}}%
\pgfpathlineto{\pgfqpoint{1.710645in}{1.527376in}}%
\pgfpathlineto{\pgfqpoint{1.717442in}{1.528485in}}%
\pgfpathlineto{\pgfqpoint{1.718551in}{1.528708in}}%
\pgfpathlineto{\pgfqpoint{1.725222in}{1.529817in}}%
\pgfpathlineto{\pgfqpoint{1.726274in}{1.529934in}}%
\pgfpathlineto{\pgfqpoint{1.734273in}{1.531043in}}%
\pgfpathlineto{\pgfqpoint{1.735367in}{1.531198in}}%
\pgfpathlineto{\pgfqpoint{1.744580in}{1.532307in}}%
\pgfpathlineto{\pgfqpoint{1.745465in}{1.532424in}}%
\pgfpathlineto{\pgfqpoint{1.745662in}{1.532424in}}%
\pgfpathlineto{\pgfqpoint{1.753280in}{1.533523in}}%
\pgfpathlineto{\pgfqpoint{1.754332in}{1.533746in}}%
\pgfpathlineto{\pgfqpoint{1.754353in}{1.533746in}}%
\pgfpathlineto{\pgfqpoint{1.763606in}{1.534855in}}%
\pgfpathlineto{\pgfqpoint{1.764542in}{1.534943in}}%
\pgfpathlineto{\pgfqpoint{1.764702in}{1.534943in}}%
\pgfpathlineto{\pgfqpoint{1.772908in}{1.536052in}}%
\pgfpathlineto{\pgfqpoint{1.773965in}{1.536168in}}%
\pgfpathlineto{\pgfqpoint{1.782553in}{1.537277in}}%
\pgfpathlineto{\pgfqpoint{1.783525in}{1.537433in}}%
\pgfpathlineto{\pgfqpoint{1.794298in}{1.538541in}}%
\pgfpathlineto{\pgfqpoint{1.795387in}{1.538736in}}%
\pgfpathlineto{\pgfqpoint{1.805522in}{1.539845in}}%
\pgfpathlineto{\pgfqpoint{1.806518in}{1.539991in}}%
\pgfpathlineto{\pgfqpoint{1.806630in}{1.539991in}}%
\pgfpathlineto{\pgfqpoint{1.817486in}{1.541099in}}%
\pgfpathlineto{\pgfqpoint{1.818468in}{1.541206in}}%
\pgfpathlineto{\pgfqpoint{1.818498in}{1.541206in}}%
\pgfpathlineto{\pgfqpoint{1.830451in}{1.542315in}}%
\pgfpathlineto{\pgfqpoint{1.831493in}{1.542403in}}%
\pgfpathlineto{\pgfqpoint{1.843539in}{1.543512in}}%
\pgfpathlineto{\pgfqpoint{1.844511in}{1.543619in}}%
\pgfpathlineto{\pgfqpoint{1.844597in}{1.543619in}}%
\pgfpathlineto{\pgfqpoint{1.855328in}{1.544727in}}%
\pgfpathlineto{\pgfqpoint{1.856433in}{1.544805in}}%
\pgfpathlineto{\pgfqpoint{1.869789in}{1.545914in}}%
\pgfpathlineto{\pgfqpoint{1.870617in}{1.546021in}}%
\pgfpathlineto{\pgfqpoint{1.870729in}{1.546021in}}%
\pgfpathlineto{\pgfqpoint{1.887214in}{1.547130in}}%
\pgfpathlineto{\pgfqpoint{1.888086in}{1.547227in}}%
\pgfpathlineto{\pgfqpoint{1.903636in}{1.548336in}}%
\pgfpathlineto{\pgfqpoint{1.904457in}{1.548384in}}%
\pgfpathlineto{\pgfqpoint{1.904541in}{1.548384in}}%
\pgfpathlineto{\pgfqpoint{1.924404in}{1.549493in}}%
\pgfpathlineto{\pgfqpoint{1.925363in}{1.549542in}}%
\pgfpathlineto{\pgfqpoint{1.925437in}{1.549542in}}%
\pgfpathlineto{\pgfqpoint{1.948293in}{1.550651in}}%
\pgfpathlineto{\pgfqpoint{1.949165in}{1.550728in}}%
\pgfpathlineto{\pgfqpoint{1.949335in}{1.550728in}}%
\pgfpathlineto{\pgfqpoint{1.979615in}{1.551837in}}%
\pgfpathlineto{\pgfqpoint{1.980592in}{1.551896in}}%
\pgfpathlineto{\pgfqpoint{1.983126in}{1.551944in}}%
\pgfpathlineto{\pgfqpoint{1.983126in}{1.551944in}}%
\pgfusepath{stroke}%
\end{pgfscope}%
\begin{pgfscope}%
\pgfsetrectcap%
\pgfsetmiterjoin%
\pgfsetlinewidth{0.803000pt}%
\definecolor{currentstroke}{rgb}{0.000000,0.000000,0.000000}%
\pgfsetstrokecolor{currentstroke}%
\pgfsetdash{}{0pt}%
\pgfpathmoveto{\pgfqpoint{0.503581in}{0.449444in}}%
\pgfpathlineto{\pgfqpoint{0.503581in}{1.604444in}}%
\pgfusepath{stroke}%
\end{pgfscope}%
\begin{pgfscope}%
\pgfsetrectcap%
\pgfsetmiterjoin%
\pgfsetlinewidth{0.803000pt}%
\definecolor{currentstroke}{rgb}{0.000000,0.000000,0.000000}%
\pgfsetstrokecolor{currentstroke}%
\pgfsetdash{}{0pt}%
\pgfpathmoveto{\pgfqpoint{2.053581in}{0.449444in}}%
\pgfpathlineto{\pgfqpoint{2.053581in}{1.604444in}}%
\pgfusepath{stroke}%
\end{pgfscope}%
\begin{pgfscope}%
\pgfsetrectcap%
\pgfsetmiterjoin%
\pgfsetlinewidth{0.803000pt}%
\definecolor{currentstroke}{rgb}{0.000000,0.000000,0.000000}%
\pgfsetstrokecolor{currentstroke}%
\pgfsetdash{}{0pt}%
\pgfpathmoveto{\pgfqpoint{0.503581in}{0.449444in}}%
\pgfpathlineto{\pgfqpoint{2.053581in}{0.449444in}}%
\pgfusepath{stroke}%
\end{pgfscope}%
\begin{pgfscope}%
\pgfsetrectcap%
\pgfsetmiterjoin%
\pgfsetlinewidth{0.803000pt}%
\definecolor{currentstroke}{rgb}{0.000000,0.000000,0.000000}%
\pgfsetstrokecolor{currentstroke}%
\pgfsetdash{}{0pt}%
\pgfpathmoveto{\pgfqpoint{0.503581in}{1.604444in}}%
\pgfpathlineto{\pgfqpoint{2.053581in}{1.604444in}}%
\pgfusepath{stroke}%
\end{pgfscope}%
\begin{pgfscope}%
\pgfsetbuttcap%
\pgfsetmiterjoin%
\definecolor{currentfill}{rgb}{1.000000,1.000000,1.000000}%
\pgfsetfillcolor{currentfill}%
\pgfsetfillopacity{0.800000}%
\pgfsetlinewidth{1.003750pt}%
\definecolor{currentstroke}{rgb}{0.800000,0.800000,0.800000}%
\pgfsetstrokecolor{currentstroke}%
\pgfsetstrokeopacity{0.800000}%
\pgfsetdash{}{0pt}%
\pgfpathmoveto{\pgfqpoint{0.782747in}{0.518889in}}%
\pgfpathlineto{\pgfqpoint{1.956358in}{0.518889in}}%
\pgfpathquadraticcurveto{\pgfqpoint{1.984136in}{0.518889in}}{\pgfqpoint{1.984136in}{0.546666in}}%
\pgfpathlineto{\pgfqpoint{1.984136in}{0.726388in}}%
\pgfpathquadraticcurveto{\pgfqpoint{1.984136in}{0.754166in}}{\pgfqpoint{1.956358in}{0.754166in}}%
\pgfpathlineto{\pgfqpoint{0.782747in}{0.754166in}}%
\pgfpathquadraticcurveto{\pgfqpoint{0.754970in}{0.754166in}}{\pgfqpoint{0.754970in}{0.726388in}}%
\pgfpathlineto{\pgfqpoint{0.754970in}{0.546666in}}%
\pgfpathquadraticcurveto{\pgfqpoint{0.754970in}{0.518889in}}{\pgfqpoint{0.782747in}{0.518889in}}%
\pgfpathlineto{\pgfqpoint{0.782747in}{0.518889in}}%
\pgfpathclose%
\pgfusepath{stroke,fill}%
\end{pgfscope}%
\begin{pgfscope}%
\pgfsetrectcap%
\pgfsetroundjoin%
\pgfsetlinewidth{1.505625pt}%
\definecolor{currentstroke}{rgb}{0.000000,0.000000,0.000000}%
\pgfsetstrokecolor{currentstroke}%
\pgfsetdash{}{0pt}%
\pgfpathmoveto{\pgfqpoint{0.810525in}{0.650000in}}%
\pgfpathlineto{\pgfqpoint{0.949414in}{0.650000in}}%
\pgfpathlineto{\pgfqpoint{1.088303in}{0.650000in}}%
\pgfusepath{stroke}%
\end{pgfscope}%
\begin{pgfscope}%
\definecolor{textcolor}{rgb}{0.000000,0.000000,0.000000}%
\pgfsetstrokecolor{textcolor}%
\pgfsetfillcolor{textcolor}%
\pgftext[x=1.199414in,y=0.601388in,left,base]{\color{textcolor}\rmfamily\fontsize{10.000000}{12.000000}\selectfont AUC=0.778}%
\end{pgfscope}%
\end{pgfpicture}%
\makeatother%
\endgroup%

\end{tabular}

\



%
\verb|Bagging_Hard_Tomek_0_v1_Test|

\

This model returned 217 different values, but most of them were rare.  Taking out the 5\% of the data set with the least frequent values, 95\% of the samples had only 10 values of $p$.  It may be a useful model, but we will not be able to fine tune the decision threshold.  

\noindent\begin{tabular}{@{\hspace{-6pt}}p{4.3in} @{\hspace{-6pt}}p{2.0in}}
	\vskip 0pt
	\hfil Raw Model Output
	
	%% Creator: Matplotlib, PGF backend
%%
%% To include the figure in your LaTeX document, write
%%   \input{<filename>.pgf}
%%
%% Make sure the required packages are loaded in your preamble
%%   \usepackage{pgf}
%%
%% Also ensure that all the required font packages are loaded; for instance,
%% the lmodern package is sometimes necessary when using math font.
%%   \usepackage{lmodern}
%%
%% Figures using additional raster images can only be included by \input if
%% they are in the same directory as the main LaTeX file. For loading figures
%% from other directories you can use the `import` package
%%   \usepackage{import}
%%
%% and then include the figures with
%%   \import{<path to file>}{<filename>.pgf}
%%
%% Matplotlib used the following preamble
%%   
%%   \usepackage{fontspec}
%%   \makeatletter\@ifpackageloaded{underscore}{}{\usepackage[strings]{underscore}}\makeatother
%%
\begingroup%
\makeatletter%
\begin{pgfpicture}%
\pgfpathrectangle{\pgfpointorigin}{\pgfqpoint{4.102500in}{1.754444in}}%
\pgfusepath{use as bounding box, clip}%
\begin{pgfscope}%
\pgfsetbuttcap%
\pgfsetmiterjoin%
\definecolor{currentfill}{rgb}{1.000000,1.000000,1.000000}%
\pgfsetfillcolor{currentfill}%
\pgfsetlinewidth{0.000000pt}%
\definecolor{currentstroke}{rgb}{1.000000,1.000000,1.000000}%
\pgfsetstrokecolor{currentstroke}%
\pgfsetdash{}{0pt}%
\pgfpathmoveto{\pgfqpoint{0.000000in}{0.000000in}}%
\pgfpathlineto{\pgfqpoint{4.102500in}{0.000000in}}%
\pgfpathlineto{\pgfqpoint{4.102500in}{1.754444in}}%
\pgfpathlineto{\pgfqpoint{0.000000in}{1.754444in}}%
\pgfpathlineto{\pgfqpoint{0.000000in}{0.000000in}}%
\pgfpathclose%
\pgfusepath{fill}%
\end{pgfscope}%
\begin{pgfscope}%
\pgfsetbuttcap%
\pgfsetmiterjoin%
\definecolor{currentfill}{rgb}{1.000000,1.000000,1.000000}%
\pgfsetfillcolor{currentfill}%
\pgfsetlinewidth{0.000000pt}%
\definecolor{currentstroke}{rgb}{0.000000,0.000000,0.000000}%
\pgfsetstrokecolor{currentstroke}%
\pgfsetstrokeopacity{0.000000}%
\pgfsetdash{}{0pt}%
\pgfpathmoveto{\pgfqpoint{0.515000in}{0.499444in}}%
\pgfpathlineto{\pgfqpoint{4.002500in}{0.499444in}}%
\pgfpathlineto{\pgfqpoint{4.002500in}{1.654444in}}%
\pgfpathlineto{\pgfqpoint{0.515000in}{1.654444in}}%
\pgfpathlineto{\pgfqpoint{0.515000in}{0.499444in}}%
\pgfpathclose%
\pgfusepath{fill}%
\end{pgfscope}%
\begin{pgfscope}%
\pgfpathrectangle{\pgfqpoint{0.515000in}{0.499444in}}{\pgfqpoint{3.487500in}{1.155000in}}%
\pgfusepath{clip}%
\pgfsetbuttcap%
\pgfsetmiterjoin%
\pgfsetlinewidth{1.003750pt}%
\definecolor{currentstroke}{rgb}{0.000000,0.000000,0.000000}%
\pgfsetstrokecolor{currentstroke}%
\pgfsetdash{}{0pt}%
\pgfpathmoveto{\pgfqpoint{0.610114in}{0.499444in}}%
\pgfpathlineto{\pgfqpoint{0.673523in}{0.499444in}}%
\pgfpathlineto{\pgfqpoint{0.673523in}{1.344992in}}%
\pgfpathlineto{\pgfqpoint{0.610114in}{1.344992in}}%
\pgfpathlineto{\pgfqpoint{0.610114in}{0.499444in}}%
\pgfpathclose%
\pgfusepath{stroke}%
\end{pgfscope}%
\begin{pgfscope}%
\pgfpathrectangle{\pgfqpoint{0.515000in}{0.499444in}}{\pgfqpoint{3.487500in}{1.155000in}}%
\pgfusepath{clip}%
\pgfsetbuttcap%
\pgfsetmiterjoin%
\pgfsetlinewidth{1.003750pt}%
\definecolor{currentstroke}{rgb}{0.000000,0.000000,0.000000}%
\pgfsetstrokecolor{currentstroke}%
\pgfsetdash{}{0pt}%
\pgfpathmoveto{\pgfqpoint{0.768637in}{0.499444in}}%
\pgfpathlineto{\pgfqpoint{0.832046in}{0.499444in}}%
\pgfpathlineto{\pgfqpoint{0.832046in}{0.502259in}}%
\pgfpathlineto{\pgfqpoint{0.768637in}{0.502259in}}%
\pgfpathlineto{\pgfqpoint{0.768637in}{0.499444in}}%
\pgfpathclose%
\pgfusepath{stroke}%
\end{pgfscope}%
\begin{pgfscope}%
\pgfpathrectangle{\pgfqpoint{0.515000in}{0.499444in}}{\pgfqpoint{3.487500in}{1.155000in}}%
\pgfusepath{clip}%
\pgfsetbuttcap%
\pgfsetmiterjoin%
\pgfsetlinewidth{1.003750pt}%
\definecolor{currentstroke}{rgb}{0.000000,0.000000,0.000000}%
\pgfsetstrokecolor{currentstroke}%
\pgfsetdash{}{0pt}%
\pgfpathmoveto{\pgfqpoint{0.927159in}{0.499444in}}%
\pgfpathlineto{\pgfqpoint{0.990568in}{0.499444in}}%
\pgfpathlineto{\pgfqpoint{0.990568in}{1.573377in}}%
\pgfpathlineto{\pgfqpoint{0.927159in}{1.573377in}}%
\pgfpathlineto{\pgfqpoint{0.927159in}{0.499444in}}%
\pgfpathclose%
\pgfusepath{stroke}%
\end{pgfscope}%
\begin{pgfscope}%
\pgfpathrectangle{\pgfqpoint{0.515000in}{0.499444in}}{\pgfqpoint{3.487500in}{1.155000in}}%
\pgfusepath{clip}%
\pgfsetbuttcap%
\pgfsetmiterjoin%
\pgfsetlinewidth{1.003750pt}%
\definecolor{currentstroke}{rgb}{0.000000,0.000000,0.000000}%
\pgfsetstrokecolor{currentstroke}%
\pgfsetdash{}{0pt}%
\pgfpathmoveto{\pgfqpoint{1.085682in}{0.499444in}}%
\pgfpathlineto{\pgfqpoint{1.149091in}{0.499444in}}%
\pgfpathlineto{\pgfqpoint{1.149091in}{0.504232in}}%
\pgfpathlineto{\pgfqpoint{1.085682in}{0.504232in}}%
\pgfpathlineto{\pgfqpoint{1.085682in}{0.499444in}}%
\pgfpathclose%
\pgfusepath{stroke}%
\end{pgfscope}%
\begin{pgfscope}%
\pgfpathrectangle{\pgfqpoint{0.515000in}{0.499444in}}{\pgfqpoint{3.487500in}{1.155000in}}%
\pgfusepath{clip}%
\pgfsetbuttcap%
\pgfsetmiterjoin%
\pgfsetlinewidth{1.003750pt}%
\definecolor{currentstroke}{rgb}{0.000000,0.000000,0.000000}%
\pgfsetstrokecolor{currentstroke}%
\pgfsetdash{}{0pt}%
\pgfpathmoveto{\pgfqpoint{1.244205in}{0.499444in}}%
\pgfpathlineto{\pgfqpoint{1.307614in}{0.499444in}}%
\pgfpathlineto{\pgfqpoint{1.307614in}{1.599444in}}%
\pgfpathlineto{\pgfqpoint{1.244205in}{1.599444in}}%
\pgfpathlineto{\pgfqpoint{1.244205in}{0.499444in}}%
\pgfpathclose%
\pgfusepath{stroke}%
\end{pgfscope}%
\begin{pgfscope}%
\pgfpathrectangle{\pgfqpoint{0.515000in}{0.499444in}}{\pgfqpoint{3.487500in}{1.155000in}}%
\pgfusepath{clip}%
\pgfsetbuttcap%
\pgfsetmiterjoin%
\pgfsetlinewidth{1.003750pt}%
\definecolor{currentstroke}{rgb}{0.000000,0.000000,0.000000}%
\pgfsetstrokecolor{currentstroke}%
\pgfsetdash{}{0pt}%
\pgfpathmoveto{\pgfqpoint{1.402728in}{0.499444in}}%
\pgfpathlineto{\pgfqpoint{1.466137in}{0.499444in}}%
\pgfpathlineto{\pgfqpoint{1.466137in}{0.505795in}}%
\pgfpathlineto{\pgfqpoint{1.402728in}{0.505795in}}%
\pgfpathlineto{\pgfqpoint{1.402728in}{0.499444in}}%
\pgfpathclose%
\pgfusepath{stroke}%
\end{pgfscope}%
\begin{pgfscope}%
\pgfpathrectangle{\pgfqpoint{0.515000in}{0.499444in}}{\pgfqpoint{3.487500in}{1.155000in}}%
\pgfusepath{clip}%
\pgfsetbuttcap%
\pgfsetmiterjoin%
\pgfsetlinewidth{1.003750pt}%
\definecolor{currentstroke}{rgb}{0.000000,0.000000,0.000000}%
\pgfsetstrokecolor{currentstroke}%
\pgfsetdash{}{0pt}%
\pgfpathmoveto{\pgfqpoint{1.561250in}{0.499444in}}%
\pgfpathlineto{\pgfqpoint{1.624659in}{0.499444in}}%
\pgfpathlineto{\pgfqpoint{1.624659in}{1.518561in}}%
\pgfpathlineto{\pgfqpoint{1.561250in}{1.518561in}}%
\pgfpathlineto{\pgfqpoint{1.561250in}{0.499444in}}%
\pgfpathclose%
\pgfusepath{stroke}%
\end{pgfscope}%
\begin{pgfscope}%
\pgfpathrectangle{\pgfqpoint{0.515000in}{0.499444in}}{\pgfqpoint{3.487500in}{1.155000in}}%
\pgfusepath{clip}%
\pgfsetbuttcap%
\pgfsetmiterjoin%
\pgfsetlinewidth{1.003750pt}%
\definecolor{currentstroke}{rgb}{0.000000,0.000000,0.000000}%
\pgfsetstrokecolor{currentstroke}%
\pgfsetdash{}{0pt}%
\pgfpathmoveto{\pgfqpoint{1.719773in}{0.499444in}}%
\pgfpathlineto{\pgfqpoint{1.783182in}{0.499444in}}%
\pgfpathlineto{\pgfqpoint{1.783182in}{0.506116in}}%
\pgfpathlineto{\pgfqpoint{1.719773in}{0.506116in}}%
\pgfpathlineto{\pgfqpoint{1.719773in}{0.499444in}}%
\pgfpathclose%
\pgfusepath{stroke}%
\end{pgfscope}%
\begin{pgfscope}%
\pgfpathrectangle{\pgfqpoint{0.515000in}{0.499444in}}{\pgfqpoint{3.487500in}{1.155000in}}%
\pgfusepath{clip}%
\pgfsetbuttcap%
\pgfsetmiterjoin%
\pgfsetlinewidth{1.003750pt}%
\definecolor{currentstroke}{rgb}{0.000000,0.000000,0.000000}%
\pgfsetstrokecolor{currentstroke}%
\pgfsetdash{}{0pt}%
\pgfpathmoveto{\pgfqpoint{1.878296in}{0.499444in}}%
\pgfpathlineto{\pgfqpoint{1.941705in}{0.499444in}}%
\pgfpathlineto{\pgfqpoint{1.941705in}{1.359111in}}%
\pgfpathlineto{\pgfqpoint{1.878296in}{1.359111in}}%
\pgfpathlineto{\pgfqpoint{1.878296in}{0.499444in}}%
\pgfpathclose%
\pgfusepath{stroke}%
\end{pgfscope}%
\begin{pgfscope}%
\pgfpathrectangle{\pgfqpoint{0.515000in}{0.499444in}}{\pgfqpoint{3.487500in}{1.155000in}}%
\pgfusepath{clip}%
\pgfsetbuttcap%
\pgfsetmiterjoin%
\pgfsetlinewidth{1.003750pt}%
\definecolor{currentstroke}{rgb}{0.000000,0.000000,0.000000}%
\pgfsetstrokecolor{currentstroke}%
\pgfsetdash{}{0pt}%
\pgfpathmoveto{\pgfqpoint{2.036818in}{0.499444in}}%
\pgfpathlineto{\pgfqpoint{2.100228in}{0.499444in}}%
\pgfpathlineto{\pgfqpoint{2.100228in}{0.505196in}}%
\pgfpathlineto{\pgfqpoint{2.036818in}{0.505196in}}%
\pgfpathlineto{\pgfqpoint{2.036818in}{0.499444in}}%
\pgfpathclose%
\pgfusepath{stroke}%
\end{pgfscope}%
\begin{pgfscope}%
\pgfpathrectangle{\pgfqpoint{0.515000in}{0.499444in}}{\pgfqpoint{3.487500in}{1.155000in}}%
\pgfusepath{clip}%
\pgfsetbuttcap%
\pgfsetmiterjoin%
\pgfsetlinewidth{1.003750pt}%
\definecolor{currentstroke}{rgb}{0.000000,0.000000,0.000000}%
\pgfsetstrokecolor{currentstroke}%
\pgfsetdash{}{0pt}%
\pgfpathmoveto{\pgfqpoint{2.195341in}{0.499444in}}%
\pgfpathlineto{\pgfqpoint{2.258750in}{0.499444in}}%
\pgfpathlineto{\pgfqpoint{2.258750in}{1.171711in}}%
\pgfpathlineto{\pgfqpoint{2.195341in}{1.171711in}}%
\pgfpathlineto{\pgfqpoint{2.195341in}{0.499444in}}%
\pgfpathclose%
\pgfusepath{stroke}%
\end{pgfscope}%
\begin{pgfscope}%
\pgfpathrectangle{\pgfqpoint{0.515000in}{0.499444in}}{\pgfqpoint{3.487500in}{1.155000in}}%
\pgfusepath{clip}%
\pgfsetbuttcap%
\pgfsetmiterjoin%
\pgfsetlinewidth{1.003750pt}%
\definecolor{currentstroke}{rgb}{0.000000,0.000000,0.000000}%
\pgfsetstrokecolor{currentstroke}%
\pgfsetdash{}{0pt}%
\pgfpathmoveto{\pgfqpoint{2.353864in}{0.499444in}}%
\pgfpathlineto{\pgfqpoint{2.417273in}{0.499444in}}%
\pgfpathlineto{\pgfqpoint{2.417273in}{0.504343in}}%
\pgfpathlineto{\pgfqpoint{2.353864in}{0.504343in}}%
\pgfpathlineto{\pgfqpoint{2.353864in}{0.499444in}}%
\pgfpathclose%
\pgfusepath{stroke}%
\end{pgfscope}%
\begin{pgfscope}%
\pgfpathrectangle{\pgfqpoint{0.515000in}{0.499444in}}{\pgfqpoint{3.487500in}{1.155000in}}%
\pgfusepath{clip}%
\pgfsetbuttcap%
\pgfsetmiterjoin%
\pgfsetlinewidth{1.003750pt}%
\definecolor{currentstroke}{rgb}{0.000000,0.000000,0.000000}%
\pgfsetstrokecolor{currentstroke}%
\pgfsetdash{}{0pt}%
\pgfpathmoveto{\pgfqpoint{2.512387in}{0.499444in}}%
\pgfpathlineto{\pgfqpoint{2.575796in}{0.499444in}}%
\pgfpathlineto{\pgfqpoint{2.575796in}{0.972972in}}%
\pgfpathlineto{\pgfqpoint{2.512387in}{0.972972in}}%
\pgfpathlineto{\pgfqpoint{2.512387in}{0.499444in}}%
\pgfpathclose%
\pgfusepath{stroke}%
\end{pgfscope}%
\begin{pgfscope}%
\pgfpathrectangle{\pgfqpoint{0.515000in}{0.499444in}}{\pgfqpoint{3.487500in}{1.155000in}}%
\pgfusepath{clip}%
\pgfsetbuttcap%
\pgfsetmiterjoin%
\pgfsetlinewidth{1.003750pt}%
\definecolor{currentstroke}{rgb}{0.000000,0.000000,0.000000}%
\pgfsetstrokecolor{currentstroke}%
\pgfsetdash{}{0pt}%
\pgfpathmoveto{\pgfqpoint{2.670909in}{0.499444in}}%
\pgfpathlineto{\pgfqpoint{2.734318in}{0.499444in}}%
\pgfpathlineto{\pgfqpoint{2.734318in}{0.503489in}}%
\pgfpathlineto{\pgfqpoint{2.670909in}{0.503489in}}%
\pgfpathlineto{\pgfqpoint{2.670909in}{0.499444in}}%
\pgfpathclose%
\pgfusepath{stroke}%
\end{pgfscope}%
\begin{pgfscope}%
\pgfpathrectangle{\pgfqpoint{0.515000in}{0.499444in}}{\pgfqpoint{3.487500in}{1.155000in}}%
\pgfusepath{clip}%
\pgfsetbuttcap%
\pgfsetmiterjoin%
\pgfsetlinewidth{1.003750pt}%
\definecolor{currentstroke}{rgb}{0.000000,0.000000,0.000000}%
\pgfsetstrokecolor{currentstroke}%
\pgfsetdash{}{0pt}%
\pgfpathmoveto{\pgfqpoint{2.829432in}{0.499444in}}%
\pgfpathlineto{\pgfqpoint{2.892841in}{0.499444in}}%
\pgfpathlineto{\pgfqpoint{2.892841in}{0.811794in}}%
\pgfpathlineto{\pgfqpoint{2.829432in}{0.811794in}}%
\pgfpathlineto{\pgfqpoint{2.829432in}{0.499444in}}%
\pgfpathclose%
\pgfusepath{stroke}%
\end{pgfscope}%
\begin{pgfscope}%
\pgfpathrectangle{\pgfqpoint{0.515000in}{0.499444in}}{\pgfqpoint{3.487500in}{1.155000in}}%
\pgfusepath{clip}%
\pgfsetbuttcap%
\pgfsetmiterjoin%
\pgfsetlinewidth{1.003750pt}%
\definecolor{currentstroke}{rgb}{0.000000,0.000000,0.000000}%
\pgfsetstrokecolor{currentstroke}%
\pgfsetdash{}{0pt}%
\pgfpathmoveto{\pgfqpoint{2.987955in}{0.499444in}}%
\pgfpathlineto{\pgfqpoint{3.051364in}{0.499444in}}%
\pgfpathlineto{\pgfqpoint{3.051364in}{0.501838in}}%
\pgfpathlineto{\pgfqpoint{2.987955in}{0.501838in}}%
\pgfpathlineto{\pgfqpoint{2.987955in}{0.499444in}}%
\pgfpathclose%
\pgfusepath{stroke}%
\end{pgfscope}%
\begin{pgfscope}%
\pgfpathrectangle{\pgfqpoint{0.515000in}{0.499444in}}{\pgfqpoint{3.487500in}{1.155000in}}%
\pgfusepath{clip}%
\pgfsetbuttcap%
\pgfsetmiterjoin%
\pgfsetlinewidth{1.003750pt}%
\definecolor{currentstroke}{rgb}{0.000000,0.000000,0.000000}%
\pgfsetstrokecolor{currentstroke}%
\pgfsetdash{}{0pt}%
\pgfpathmoveto{\pgfqpoint{3.146478in}{0.499444in}}%
\pgfpathlineto{\pgfqpoint{3.209887in}{0.499444in}}%
\pgfpathlineto{\pgfqpoint{3.209887in}{0.683021in}}%
\pgfpathlineto{\pgfqpoint{3.146478in}{0.683021in}}%
\pgfpathlineto{\pgfqpoint{3.146478in}{0.499444in}}%
\pgfpathclose%
\pgfusepath{stroke}%
\end{pgfscope}%
\begin{pgfscope}%
\pgfpathrectangle{\pgfqpoint{0.515000in}{0.499444in}}{\pgfqpoint{3.487500in}{1.155000in}}%
\pgfusepath{clip}%
\pgfsetbuttcap%
\pgfsetmiterjoin%
\pgfsetlinewidth{1.003750pt}%
\definecolor{currentstroke}{rgb}{0.000000,0.000000,0.000000}%
\pgfsetstrokecolor{currentstroke}%
\pgfsetdash{}{0pt}%
\pgfpathmoveto{\pgfqpoint{3.305000in}{0.499444in}}%
\pgfpathlineto{\pgfqpoint{3.368409in}{0.499444in}}%
\pgfpathlineto{\pgfqpoint{3.368409in}{0.501084in}}%
\pgfpathlineto{\pgfqpoint{3.305000in}{0.501084in}}%
\pgfpathlineto{\pgfqpoint{3.305000in}{0.499444in}}%
\pgfpathclose%
\pgfusepath{stroke}%
\end{pgfscope}%
\begin{pgfscope}%
\pgfpathrectangle{\pgfqpoint{0.515000in}{0.499444in}}{\pgfqpoint{3.487500in}{1.155000in}}%
\pgfusepath{clip}%
\pgfsetbuttcap%
\pgfsetmiterjoin%
\pgfsetlinewidth{1.003750pt}%
\definecolor{currentstroke}{rgb}{0.000000,0.000000,0.000000}%
\pgfsetstrokecolor{currentstroke}%
\pgfsetdash{}{0pt}%
\pgfpathmoveto{\pgfqpoint{3.463523in}{0.499444in}}%
\pgfpathlineto{\pgfqpoint{3.526932in}{0.499444in}}%
\pgfpathlineto{\pgfqpoint{3.526932in}{0.595533in}}%
\pgfpathlineto{\pgfqpoint{3.463523in}{0.595533in}}%
\pgfpathlineto{\pgfqpoint{3.463523in}{0.499444in}}%
\pgfpathclose%
\pgfusepath{stroke}%
\end{pgfscope}%
\begin{pgfscope}%
\pgfpathrectangle{\pgfqpoint{0.515000in}{0.499444in}}{\pgfqpoint{3.487500in}{1.155000in}}%
\pgfusepath{clip}%
\pgfsetbuttcap%
\pgfsetmiterjoin%
\pgfsetlinewidth{1.003750pt}%
\definecolor{currentstroke}{rgb}{0.000000,0.000000,0.000000}%
\pgfsetstrokecolor{currentstroke}%
\pgfsetdash{}{0pt}%
\pgfpathmoveto{\pgfqpoint{3.622046in}{0.499444in}}%
\pgfpathlineto{\pgfqpoint{3.685455in}{0.499444in}}%
\pgfpathlineto{\pgfqpoint{3.685455in}{0.499899in}}%
\pgfpathlineto{\pgfqpoint{3.622046in}{0.499899in}}%
\pgfpathlineto{\pgfqpoint{3.622046in}{0.499444in}}%
\pgfpathclose%
\pgfusepath{stroke}%
\end{pgfscope}%
\begin{pgfscope}%
\pgfpathrectangle{\pgfqpoint{0.515000in}{0.499444in}}{\pgfqpoint{3.487500in}{1.155000in}}%
\pgfusepath{clip}%
\pgfsetbuttcap%
\pgfsetmiterjoin%
\pgfsetlinewidth{1.003750pt}%
\definecolor{currentstroke}{rgb}{0.000000,0.000000,0.000000}%
\pgfsetstrokecolor{currentstroke}%
\pgfsetdash{}{0pt}%
\pgfpathmoveto{\pgfqpoint{3.780568in}{0.499444in}}%
\pgfpathlineto{\pgfqpoint{3.843978in}{0.499444in}}%
\pgfpathlineto{\pgfqpoint{3.843978in}{0.535475in}}%
\pgfpathlineto{\pgfqpoint{3.780568in}{0.535475in}}%
\pgfpathlineto{\pgfqpoint{3.780568in}{0.499444in}}%
\pgfpathclose%
\pgfusepath{stroke}%
\end{pgfscope}%
\begin{pgfscope}%
\pgfpathrectangle{\pgfqpoint{0.515000in}{0.499444in}}{\pgfqpoint{3.487500in}{1.155000in}}%
\pgfusepath{clip}%
\pgfsetbuttcap%
\pgfsetmiterjoin%
\definecolor{currentfill}{rgb}{0.000000,0.000000,0.000000}%
\pgfsetfillcolor{currentfill}%
\pgfsetlinewidth{0.000000pt}%
\definecolor{currentstroke}{rgb}{0.000000,0.000000,0.000000}%
\pgfsetstrokecolor{currentstroke}%
\pgfsetstrokeopacity{0.000000}%
\pgfsetdash{}{0pt}%
\pgfpathmoveto{\pgfqpoint{0.673523in}{0.499444in}}%
\pgfpathlineto{\pgfqpoint{0.736932in}{0.499444in}}%
\pgfpathlineto{\pgfqpoint{0.736932in}{0.522907in}}%
\pgfpathlineto{\pgfqpoint{0.673523in}{0.522907in}}%
\pgfpathlineto{\pgfqpoint{0.673523in}{0.499444in}}%
\pgfpathclose%
\pgfusepath{fill}%
\end{pgfscope}%
\begin{pgfscope}%
\pgfpathrectangle{\pgfqpoint{0.515000in}{0.499444in}}{\pgfqpoint{3.487500in}{1.155000in}}%
\pgfusepath{clip}%
\pgfsetbuttcap%
\pgfsetmiterjoin%
\definecolor{currentfill}{rgb}{0.000000,0.000000,0.000000}%
\pgfsetfillcolor{currentfill}%
\pgfsetlinewidth{0.000000pt}%
\definecolor{currentstroke}{rgb}{0.000000,0.000000,0.000000}%
\pgfsetstrokecolor{currentstroke}%
\pgfsetstrokeopacity{0.000000}%
\pgfsetdash{}{0pt}%
\pgfpathmoveto{\pgfqpoint{0.832046in}{0.499444in}}%
\pgfpathlineto{\pgfqpoint{0.895455in}{0.499444in}}%
\pgfpathlineto{\pgfqpoint{0.895455in}{0.499522in}}%
\pgfpathlineto{\pgfqpoint{0.832046in}{0.499522in}}%
\pgfpathlineto{\pgfqpoint{0.832046in}{0.499444in}}%
\pgfpathclose%
\pgfusepath{fill}%
\end{pgfscope}%
\begin{pgfscope}%
\pgfpathrectangle{\pgfqpoint{0.515000in}{0.499444in}}{\pgfqpoint{3.487500in}{1.155000in}}%
\pgfusepath{clip}%
\pgfsetbuttcap%
\pgfsetmiterjoin%
\definecolor{currentfill}{rgb}{0.000000,0.000000,0.000000}%
\pgfsetfillcolor{currentfill}%
\pgfsetlinewidth{0.000000pt}%
\definecolor{currentstroke}{rgb}{0.000000,0.000000,0.000000}%
\pgfsetstrokecolor{currentstroke}%
\pgfsetstrokeopacity{0.000000}%
\pgfsetdash{}{0pt}%
\pgfpathmoveto{\pgfqpoint{0.990568in}{0.499444in}}%
\pgfpathlineto{\pgfqpoint{1.053978in}{0.499444in}}%
\pgfpathlineto{\pgfqpoint{1.053978in}{0.550481in}}%
\pgfpathlineto{\pgfqpoint{0.990568in}{0.550481in}}%
\pgfpathlineto{\pgfqpoint{0.990568in}{0.499444in}}%
\pgfpathclose%
\pgfusepath{fill}%
\end{pgfscope}%
\begin{pgfscope}%
\pgfpathrectangle{\pgfqpoint{0.515000in}{0.499444in}}{\pgfqpoint{3.487500in}{1.155000in}}%
\pgfusepath{clip}%
\pgfsetbuttcap%
\pgfsetmiterjoin%
\definecolor{currentfill}{rgb}{0.000000,0.000000,0.000000}%
\pgfsetfillcolor{currentfill}%
\pgfsetlinewidth{0.000000pt}%
\definecolor{currentstroke}{rgb}{0.000000,0.000000,0.000000}%
\pgfsetstrokecolor{currentstroke}%
\pgfsetstrokeopacity{0.000000}%
\pgfsetdash{}{0pt}%
\pgfpathmoveto{\pgfqpoint{1.149091in}{0.499444in}}%
\pgfpathlineto{\pgfqpoint{1.212500in}{0.499444in}}%
\pgfpathlineto{\pgfqpoint{1.212500in}{0.499677in}}%
\pgfpathlineto{\pgfqpoint{1.149091in}{0.499677in}}%
\pgfpathlineto{\pgfqpoint{1.149091in}{0.499444in}}%
\pgfpathclose%
\pgfusepath{fill}%
\end{pgfscope}%
\begin{pgfscope}%
\pgfpathrectangle{\pgfqpoint{0.515000in}{0.499444in}}{\pgfqpoint{3.487500in}{1.155000in}}%
\pgfusepath{clip}%
\pgfsetbuttcap%
\pgfsetmiterjoin%
\definecolor{currentfill}{rgb}{0.000000,0.000000,0.000000}%
\pgfsetfillcolor{currentfill}%
\pgfsetlinewidth{0.000000pt}%
\definecolor{currentstroke}{rgb}{0.000000,0.000000,0.000000}%
\pgfsetstrokecolor{currentstroke}%
\pgfsetstrokeopacity{0.000000}%
\pgfsetdash{}{0pt}%
\pgfpathmoveto{\pgfqpoint{1.307614in}{0.499444in}}%
\pgfpathlineto{\pgfqpoint{1.371023in}{0.499444in}}%
\pgfpathlineto{\pgfqpoint{1.371023in}{0.581247in}}%
\pgfpathlineto{\pgfqpoint{1.307614in}{0.581247in}}%
\pgfpathlineto{\pgfqpoint{1.307614in}{0.499444in}}%
\pgfpathclose%
\pgfusepath{fill}%
\end{pgfscope}%
\begin{pgfscope}%
\pgfpathrectangle{\pgfqpoint{0.515000in}{0.499444in}}{\pgfqpoint{3.487500in}{1.155000in}}%
\pgfusepath{clip}%
\pgfsetbuttcap%
\pgfsetmiterjoin%
\definecolor{currentfill}{rgb}{0.000000,0.000000,0.000000}%
\pgfsetfillcolor{currentfill}%
\pgfsetlinewidth{0.000000pt}%
\definecolor{currentstroke}{rgb}{0.000000,0.000000,0.000000}%
\pgfsetstrokecolor{currentstroke}%
\pgfsetstrokeopacity{0.000000}%
\pgfsetdash{}{0pt}%
\pgfpathmoveto{\pgfqpoint{1.466137in}{0.499444in}}%
\pgfpathlineto{\pgfqpoint{1.529546in}{0.499444in}}%
\pgfpathlineto{\pgfqpoint{1.529546in}{0.500009in}}%
\pgfpathlineto{\pgfqpoint{1.466137in}{0.500009in}}%
\pgfpathlineto{\pgfqpoint{1.466137in}{0.499444in}}%
\pgfpathclose%
\pgfusepath{fill}%
\end{pgfscope}%
\begin{pgfscope}%
\pgfpathrectangle{\pgfqpoint{0.515000in}{0.499444in}}{\pgfqpoint{3.487500in}{1.155000in}}%
\pgfusepath{clip}%
\pgfsetbuttcap%
\pgfsetmiterjoin%
\definecolor{currentfill}{rgb}{0.000000,0.000000,0.000000}%
\pgfsetfillcolor{currentfill}%
\pgfsetlinewidth{0.000000pt}%
\definecolor{currentstroke}{rgb}{0.000000,0.000000,0.000000}%
\pgfsetstrokecolor{currentstroke}%
\pgfsetstrokeopacity{0.000000}%
\pgfsetdash{}{0pt}%
\pgfpathmoveto{\pgfqpoint{1.624659in}{0.499444in}}%
\pgfpathlineto{\pgfqpoint{1.688068in}{0.499444in}}%
\pgfpathlineto{\pgfqpoint{1.688068in}{0.610439in}}%
\pgfpathlineto{\pgfqpoint{1.624659in}{0.610439in}}%
\pgfpathlineto{\pgfqpoint{1.624659in}{0.499444in}}%
\pgfpathclose%
\pgfusepath{fill}%
\end{pgfscope}%
\begin{pgfscope}%
\pgfpathrectangle{\pgfqpoint{0.515000in}{0.499444in}}{\pgfqpoint{3.487500in}{1.155000in}}%
\pgfusepath{clip}%
\pgfsetbuttcap%
\pgfsetmiterjoin%
\definecolor{currentfill}{rgb}{0.000000,0.000000,0.000000}%
\pgfsetfillcolor{currentfill}%
\pgfsetlinewidth{0.000000pt}%
\definecolor{currentstroke}{rgb}{0.000000,0.000000,0.000000}%
\pgfsetstrokecolor{currentstroke}%
\pgfsetstrokeopacity{0.000000}%
\pgfsetdash{}{0pt}%
\pgfpathmoveto{\pgfqpoint{1.783182in}{0.499444in}}%
\pgfpathlineto{\pgfqpoint{1.846591in}{0.499444in}}%
\pgfpathlineto{\pgfqpoint{1.846591in}{0.500220in}}%
\pgfpathlineto{\pgfqpoint{1.783182in}{0.500220in}}%
\pgfpathlineto{\pgfqpoint{1.783182in}{0.499444in}}%
\pgfpathclose%
\pgfusepath{fill}%
\end{pgfscope}%
\begin{pgfscope}%
\pgfpathrectangle{\pgfqpoint{0.515000in}{0.499444in}}{\pgfqpoint{3.487500in}{1.155000in}}%
\pgfusepath{clip}%
\pgfsetbuttcap%
\pgfsetmiterjoin%
\definecolor{currentfill}{rgb}{0.000000,0.000000,0.000000}%
\pgfsetfillcolor{currentfill}%
\pgfsetlinewidth{0.000000pt}%
\definecolor{currentstroke}{rgb}{0.000000,0.000000,0.000000}%
\pgfsetstrokecolor{currentstroke}%
\pgfsetstrokeopacity{0.000000}%
\pgfsetdash{}{0pt}%
\pgfpathmoveto{\pgfqpoint{1.941705in}{0.499444in}}%
\pgfpathlineto{\pgfqpoint{2.005114in}{0.499444in}}%
\pgfpathlineto{\pgfqpoint{2.005114in}{0.632904in}}%
\pgfpathlineto{\pgfqpoint{1.941705in}{0.632904in}}%
\pgfpathlineto{\pgfqpoint{1.941705in}{0.499444in}}%
\pgfpathclose%
\pgfusepath{fill}%
\end{pgfscope}%
\begin{pgfscope}%
\pgfpathrectangle{\pgfqpoint{0.515000in}{0.499444in}}{\pgfqpoint{3.487500in}{1.155000in}}%
\pgfusepath{clip}%
\pgfsetbuttcap%
\pgfsetmiterjoin%
\definecolor{currentfill}{rgb}{0.000000,0.000000,0.000000}%
\pgfsetfillcolor{currentfill}%
\pgfsetlinewidth{0.000000pt}%
\definecolor{currentstroke}{rgb}{0.000000,0.000000,0.000000}%
\pgfsetstrokecolor{currentstroke}%
\pgfsetstrokeopacity{0.000000}%
\pgfsetdash{}{0pt}%
\pgfpathmoveto{\pgfqpoint{2.100228in}{0.499444in}}%
\pgfpathlineto{\pgfqpoint{2.163637in}{0.499444in}}%
\pgfpathlineto{\pgfqpoint{2.163637in}{0.500497in}}%
\pgfpathlineto{\pgfqpoint{2.100228in}{0.500497in}}%
\pgfpathlineto{\pgfqpoint{2.100228in}{0.499444in}}%
\pgfpathclose%
\pgfusepath{fill}%
\end{pgfscope}%
\begin{pgfscope}%
\pgfpathrectangle{\pgfqpoint{0.515000in}{0.499444in}}{\pgfqpoint{3.487500in}{1.155000in}}%
\pgfusepath{clip}%
\pgfsetbuttcap%
\pgfsetmiterjoin%
\definecolor{currentfill}{rgb}{0.000000,0.000000,0.000000}%
\pgfsetfillcolor{currentfill}%
\pgfsetlinewidth{0.000000pt}%
\definecolor{currentstroke}{rgb}{0.000000,0.000000,0.000000}%
\pgfsetstrokecolor{currentstroke}%
\pgfsetstrokeopacity{0.000000}%
\pgfsetdash{}{0pt}%
\pgfpathmoveto{\pgfqpoint{2.258750in}{0.499444in}}%
\pgfpathlineto{\pgfqpoint{2.322159in}{0.499444in}}%
\pgfpathlineto{\pgfqpoint{2.322159in}{0.649640in}}%
\pgfpathlineto{\pgfqpoint{2.258750in}{0.649640in}}%
\pgfpathlineto{\pgfqpoint{2.258750in}{0.499444in}}%
\pgfpathclose%
\pgfusepath{fill}%
\end{pgfscope}%
\begin{pgfscope}%
\pgfpathrectangle{\pgfqpoint{0.515000in}{0.499444in}}{\pgfqpoint{3.487500in}{1.155000in}}%
\pgfusepath{clip}%
\pgfsetbuttcap%
\pgfsetmiterjoin%
\definecolor{currentfill}{rgb}{0.000000,0.000000,0.000000}%
\pgfsetfillcolor{currentfill}%
\pgfsetlinewidth{0.000000pt}%
\definecolor{currentstroke}{rgb}{0.000000,0.000000,0.000000}%
\pgfsetstrokecolor{currentstroke}%
\pgfsetstrokeopacity{0.000000}%
\pgfsetdash{}{0pt}%
\pgfpathmoveto{\pgfqpoint{2.417273in}{0.499444in}}%
\pgfpathlineto{\pgfqpoint{2.480682in}{0.499444in}}%
\pgfpathlineto{\pgfqpoint{2.480682in}{0.500841in}}%
\pgfpathlineto{\pgfqpoint{2.417273in}{0.500841in}}%
\pgfpathlineto{\pgfqpoint{2.417273in}{0.499444in}}%
\pgfpathclose%
\pgfusepath{fill}%
\end{pgfscope}%
\begin{pgfscope}%
\pgfpathrectangle{\pgfqpoint{0.515000in}{0.499444in}}{\pgfqpoint{3.487500in}{1.155000in}}%
\pgfusepath{clip}%
\pgfsetbuttcap%
\pgfsetmiterjoin%
\definecolor{currentfill}{rgb}{0.000000,0.000000,0.000000}%
\pgfsetfillcolor{currentfill}%
\pgfsetlinewidth{0.000000pt}%
\definecolor{currentstroke}{rgb}{0.000000,0.000000,0.000000}%
\pgfsetstrokecolor{currentstroke}%
\pgfsetstrokeopacity{0.000000}%
\pgfsetdash{}{0pt}%
\pgfpathmoveto{\pgfqpoint{2.575796in}{0.499444in}}%
\pgfpathlineto{\pgfqpoint{2.639205in}{0.499444in}}%
\pgfpathlineto{\pgfqpoint{2.639205in}{0.653175in}}%
\pgfpathlineto{\pgfqpoint{2.575796in}{0.653175in}}%
\pgfpathlineto{\pgfqpoint{2.575796in}{0.499444in}}%
\pgfpathclose%
\pgfusepath{fill}%
\end{pgfscope}%
\begin{pgfscope}%
\pgfpathrectangle{\pgfqpoint{0.515000in}{0.499444in}}{\pgfqpoint{3.487500in}{1.155000in}}%
\pgfusepath{clip}%
\pgfsetbuttcap%
\pgfsetmiterjoin%
\definecolor{currentfill}{rgb}{0.000000,0.000000,0.000000}%
\pgfsetfillcolor{currentfill}%
\pgfsetlinewidth{0.000000pt}%
\definecolor{currentstroke}{rgb}{0.000000,0.000000,0.000000}%
\pgfsetstrokecolor{currentstroke}%
\pgfsetstrokeopacity{0.000000}%
\pgfsetdash{}{0pt}%
\pgfpathmoveto{\pgfqpoint{2.734318in}{0.499444in}}%
\pgfpathlineto{\pgfqpoint{2.797728in}{0.499444in}}%
\pgfpathlineto{\pgfqpoint{2.797728in}{0.500996in}}%
\pgfpathlineto{\pgfqpoint{2.734318in}{0.500996in}}%
\pgfpathlineto{\pgfqpoint{2.734318in}{0.499444in}}%
\pgfpathclose%
\pgfusepath{fill}%
\end{pgfscope}%
\begin{pgfscope}%
\pgfpathrectangle{\pgfqpoint{0.515000in}{0.499444in}}{\pgfqpoint{3.487500in}{1.155000in}}%
\pgfusepath{clip}%
\pgfsetbuttcap%
\pgfsetmiterjoin%
\definecolor{currentfill}{rgb}{0.000000,0.000000,0.000000}%
\pgfsetfillcolor{currentfill}%
\pgfsetlinewidth{0.000000pt}%
\definecolor{currentstroke}{rgb}{0.000000,0.000000,0.000000}%
\pgfsetstrokecolor{currentstroke}%
\pgfsetstrokeopacity{0.000000}%
\pgfsetdash{}{0pt}%
\pgfpathmoveto{\pgfqpoint{2.892841in}{0.499444in}}%
\pgfpathlineto{\pgfqpoint{2.956250in}{0.499444in}}%
\pgfpathlineto{\pgfqpoint{2.956250in}{0.651058in}}%
\pgfpathlineto{\pgfqpoint{2.892841in}{0.651058in}}%
\pgfpathlineto{\pgfqpoint{2.892841in}{0.499444in}}%
\pgfpathclose%
\pgfusepath{fill}%
\end{pgfscope}%
\begin{pgfscope}%
\pgfpathrectangle{\pgfqpoint{0.515000in}{0.499444in}}{\pgfqpoint{3.487500in}{1.155000in}}%
\pgfusepath{clip}%
\pgfsetbuttcap%
\pgfsetmiterjoin%
\definecolor{currentfill}{rgb}{0.000000,0.000000,0.000000}%
\pgfsetfillcolor{currentfill}%
\pgfsetlinewidth{0.000000pt}%
\definecolor{currentstroke}{rgb}{0.000000,0.000000,0.000000}%
\pgfsetstrokecolor{currentstroke}%
\pgfsetstrokeopacity{0.000000}%
\pgfsetdash{}{0pt}%
\pgfpathmoveto{\pgfqpoint{3.051364in}{0.499444in}}%
\pgfpathlineto{\pgfqpoint{3.114773in}{0.499444in}}%
\pgfpathlineto{\pgfqpoint{3.114773in}{0.501040in}}%
\pgfpathlineto{\pgfqpoint{3.051364in}{0.501040in}}%
\pgfpathlineto{\pgfqpoint{3.051364in}{0.499444in}}%
\pgfpathclose%
\pgfusepath{fill}%
\end{pgfscope}%
\begin{pgfscope}%
\pgfpathrectangle{\pgfqpoint{0.515000in}{0.499444in}}{\pgfqpoint{3.487500in}{1.155000in}}%
\pgfusepath{clip}%
\pgfsetbuttcap%
\pgfsetmiterjoin%
\definecolor{currentfill}{rgb}{0.000000,0.000000,0.000000}%
\pgfsetfillcolor{currentfill}%
\pgfsetlinewidth{0.000000pt}%
\definecolor{currentstroke}{rgb}{0.000000,0.000000,0.000000}%
\pgfsetstrokecolor{currentstroke}%
\pgfsetstrokeopacity{0.000000}%
\pgfsetdash{}{0pt}%
\pgfpathmoveto{\pgfqpoint{3.209887in}{0.499444in}}%
\pgfpathlineto{\pgfqpoint{3.273296in}{0.499444in}}%
\pgfpathlineto{\pgfqpoint{3.273296in}{0.639599in}}%
\pgfpathlineto{\pgfqpoint{3.209887in}{0.639599in}}%
\pgfpathlineto{\pgfqpoint{3.209887in}{0.499444in}}%
\pgfpathclose%
\pgfusepath{fill}%
\end{pgfscope}%
\begin{pgfscope}%
\pgfpathrectangle{\pgfqpoint{0.515000in}{0.499444in}}{\pgfqpoint{3.487500in}{1.155000in}}%
\pgfusepath{clip}%
\pgfsetbuttcap%
\pgfsetmiterjoin%
\definecolor{currentfill}{rgb}{0.000000,0.000000,0.000000}%
\pgfsetfillcolor{currentfill}%
\pgfsetlinewidth{0.000000pt}%
\definecolor{currentstroke}{rgb}{0.000000,0.000000,0.000000}%
\pgfsetstrokecolor{currentstroke}%
\pgfsetstrokeopacity{0.000000}%
\pgfsetdash{}{0pt}%
\pgfpathmoveto{\pgfqpoint{3.368409in}{0.499444in}}%
\pgfpathlineto{\pgfqpoint{3.431818in}{0.499444in}}%
\pgfpathlineto{\pgfqpoint{3.431818in}{0.500486in}}%
\pgfpathlineto{\pgfqpoint{3.368409in}{0.500486in}}%
\pgfpathlineto{\pgfqpoint{3.368409in}{0.499444in}}%
\pgfpathclose%
\pgfusepath{fill}%
\end{pgfscope}%
\begin{pgfscope}%
\pgfpathrectangle{\pgfqpoint{0.515000in}{0.499444in}}{\pgfqpoint{3.487500in}{1.155000in}}%
\pgfusepath{clip}%
\pgfsetbuttcap%
\pgfsetmiterjoin%
\definecolor{currentfill}{rgb}{0.000000,0.000000,0.000000}%
\pgfsetfillcolor{currentfill}%
\pgfsetlinewidth{0.000000pt}%
\definecolor{currentstroke}{rgb}{0.000000,0.000000,0.000000}%
\pgfsetstrokecolor{currentstroke}%
\pgfsetstrokeopacity{0.000000}%
\pgfsetdash{}{0pt}%
\pgfpathmoveto{\pgfqpoint{3.526932in}{0.499444in}}%
\pgfpathlineto{\pgfqpoint{3.590341in}{0.499444in}}%
\pgfpathlineto{\pgfqpoint{3.590341in}{0.617012in}}%
\pgfpathlineto{\pgfqpoint{3.526932in}{0.617012in}}%
\pgfpathlineto{\pgfqpoint{3.526932in}{0.499444in}}%
\pgfpathclose%
\pgfusepath{fill}%
\end{pgfscope}%
\begin{pgfscope}%
\pgfpathrectangle{\pgfqpoint{0.515000in}{0.499444in}}{\pgfqpoint{3.487500in}{1.155000in}}%
\pgfusepath{clip}%
\pgfsetbuttcap%
\pgfsetmiterjoin%
\definecolor{currentfill}{rgb}{0.000000,0.000000,0.000000}%
\pgfsetfillcolor{currentfill}%
\pgfsetlinewidth{0.000000pt}%
\definecolor{currentstroke}{rgb}{0.000000,0.000000,0.000000}%
\pgfsetstrokecolor{currentstroke}%
\pgfsetstrokeopacity{0.000000}%
\pgfsetdash{}{0pt}%
\pgfpathmoveto{\pgfqpoint{3.685455in}{0.499444in}}%
\pgfpathlineto{\pgfqpoint{3.748864in}{0.499444in}}%
\pgfpathlineto{\pgfqpoint{3.748864in}{0.500198in}}%
\pgfpathlineto{\pgfqpoint{3.685455in}{0.500198in}}%
\pgfpathlineto{\pgfqpoint{3.685455in}{0.499444in}}%
\pgfpathclose%
\pgfusepath{fill}%
\end{pgfscope}%
\begin{pgfscope}%
\pgfpathrectangle{\pgfqpoint{0.515000in}{0.499444in}}{\pgfqpoint{3.487500in}{1.155000in}}%
\pgfusepath{clip}%
\pgfsetbuttcap%
\pgfsetmiterjoin%
\definecolor{currentfill}{rgb}{0.000000,0.000000,0.000000}%
\pgfsetfillcolor{currentfill}%
\pgfsetlinewidth{0.000000pt}%
\definecolor{currentstroke}{rgb}{0.000000,0.000000,0.000000}%
\pgfsetstrokecolor{currentstroke}%
\pgfsetstrokeopacity{0.000000}%
\pgfsetdash{}{0pt}%
\pgfpathmoveto{\pgfqpoint{3.843978in}{0.499444in}}%
\pgfpathlineto{\pgfqpoint{3.907387in}{0.499444in}}%
\pgfpathlineto{\pgfqpoint{3.907387in}{0.572846in}}%
\pgfpathlineto{\pgfqpoint{3.843978in}{0.572846in}}%
\pgfpathlineto{\pgfqpoint{3.843978in}{0.499444in}}%
\pgfpathclose%
\pgfusepath{fill}%
\end{pgfscope}%
\begin{pgfscope}%
\pgfsetbuttcap%
\pgfsetroundjoin%
\definecolor{currentfill}{rgb}{0.000000,0.000000,0.000000}%
\pgfsetfillcolor{currentfill}%
\pgfsetlinewidth{0.803000pt}%
\definecolor{currentstroke}{rgb}{0.000000,0.000000,0.000000}%
\pgfsetstrokecolor{currentstroke}%
\pgfsetdash{}{0pt}%
\pgfsys@defobject{currentmarker}{\pgfqpoint{0.000000in}{-0.048611in}}{\pgfqpoint{0.000000in}{0.000000in}}{%
\pgfpathmoveto{\pgfqpoint{0.000000in}{0.000000in}}%
\pgfpathlineto{\pgfqpoint{0.000000in}{-0.048611in}}%
\pgfusepath{stroke,fill}%
}%
\begin{pgfscope}%
\pgfsys@transformshift{0.515000in}{0.499444in}%
\pgfsys@useobject{currentmarker}{}%
\end{pgfscope}%
\end{pgfscope}%
\begin{pgfscope}%
\pgfsetbuttcap%
\pgfsetroundjoin%
\definecolor{currentfill}{rgb}{0.000000,0.000000,0.000000}%
\pgfsetfillcolor{currentfill}%
\pgfsetlinewidth{0.803000pt}%
\definecolor{currentstroke}{rgb}{0.000000,0.000000,0.000000}%
\pgfsetstrokecolor{currentstroke}%
\pgfsetdash{}{0pt}%
\pgfsys@defobject{currentmarker}{\pgfqpoint{0.000000in}{-0.048611in}}{\pgfqpoint{0.000000in}{0.000000in}}{%
\pgfpathmoveto{\pgfqpoint{0.000000in}{0.000000in}}%
\pgfpathlineto{\pgfqpoint{0.000000in}{-0.048611in}}%
\pgfusepath{stroke,fill}%
}%
\begin{pgfscope}%
\pgfsys@transformshift{0.673523in}{0.499444in}%
\pgfsys@useobject{currentmarker}{}%
\end{pgfscope}%
\end{pgfscope}%
\begin{pgfscope}%
\definecolor{textcolor}{rgb}{0.000000,0.000000,0.000000}%
\pgfsetstrokecolor{textcolor}%
\pgfsetfillcolor{textcolor}%
\pgftext[x=0.673523in,y=0.402222in,,top]{\color{textcolor}\rmfamily\fontsize{10.000000}{12.000000}\selectfont 0.0}%
\end{pgfscope}%
\begin{pgfscope}%
\pgfsetbuttcap%
\pgfsetroundjoin%
\definecolor{currentfill}{rgb}{0.000000,0.000000,0.000000}%
\pgfsetfillcolor{currentfill}%
\pgfsetlinewidth{0.803000pt}%
\definecolor{currentstroke}{rgb}{0.000000,0.000000,0.000000}%
\pgfsetstrokecolor{currentstroke}%
\pgfsetdash{}{0pt}%
\pgfsys@defobject{currentmarker}{\pgfqpoint{0.000000in}{-0.048611in}}{\pgfqpoint{0.000000in}{0.000000in}}{%
\pgfpathmoveto{\pgfqpoint{0.000000in}{0.000000in}}%
\pgfpathlineto{\pgfqpoint{0.000000in}{-0.048611in}}%
\pgfusepath{stroke,fill}%
}%
\begin{pgfscope}%
\pgfsys@transformshift{0.832046in}{0.499444in}%
\pgfsys@useobject{currentmarker}{}%
\end{pgfscope}%
\end{pgfscope}%
\begin{pgfscope}%
\pgfsetbuttcap%
\pgfsetroundjoin%
\definecolor{currentfill}{rgb}{0.000000,0.000000,0.000000}%
\pgfsetfillcolor{currentfill}%
\pgfsetlinewidth{0.803000pt}%
\definecolor{currentstroke}{rgb}{0.000000,0.000000,0.000000}%
\pgfsetstrokecolor{currentstroke}%
\pgfsetdash{}{0pt}%
\pgfsys@defobject{currentmarker}{\pgfqpoint{0.000000in}{-0.048611in}}{\pgfqpoint{0.000000in}{0.000000in}}{%
\pgfpathmoveto{\pgfqpoint{0.000000in}{0.000000in}}%
\pgfpathlineto{\pgfqpoint{0.000000in}{-0.048611in}}%
\pgfusepath{stroke,fill}%
}%
\begin{pgfscope}%
\pgfsys@transformshift{0.990568in}{0.499444in}%
\pgfsys@useobject{currentmarker}{}%
\end{pgfscope}%
\end{pgfscope}%
\begin{pgfscope}%
\definecolor{textcolor}{rgb}{0.000000,0.000000,0.000000}%
\pgfsetstrokecolor{textcolor}%
\pgfsetfillcolor{textcolor}%
\pgftext[x=0.990568in,y=0.402222in,,top]{\color{textcolor}\rmfamily\fontsize{10.000000}{12.000000}\selectfont 0.1}%
\end{pgfscope}%
\begin{pgfscope}%
\pgfsetbuttcap%
\pgfsetroundjoin%
\definecolor{currentfill}{rgb}{0.000000,0.000000,0.000000}%
\pgfsetfillcolor{currentfill}%
\pgfsetlinewidth{0.803000pt}%
\definecolor{currentstroke}{rgb}{0.000000,0.000000,0.000000}%
\pgfsetstrokecolor{currentstroke}%
\pgfsetdash{}{0pt}%
\pgfsys@defobject{currentmarker}{\pgfqpoint{0.000000in}{-0.048611in}}{\pgfqpoint{0.000000in}{0.000000in}}{%
\pgfpathmoveto{\pgfqpoint{0.000000in}{0.000000in}}%
\pgfpathlineto{\pgfqpoint{0.000000in}{-0.048611in}}%
\pgfusepath{stroke,fill}%
}%
\begin{pgfscope}%
\pgfsys@transformshift{1.149091in}{0.499444in}%
\pgfsys@useobject{currentmarker}{}%
\end{pgfscope}%
\end{pgfscope}%
\begin{pgfscope}%
\pgfsetbuttcap%
\pgfsetroundjoin%
\definecolor{currentfill}{rgb}{0.000000,0.000000,0.000000}%
\pgfsetfillcolor{currentfill}%
\pgfsetlinewidth{0.803000pt}%
\definecolor{currentstroke}{rgb}{0.000000,0.000000,0.000000}%
\pgfsetstrokecolor{currentstroke}%
\pgfsetdash{}{0pt}%
\pgfsys@defobject{currentmarker}{\pgfqpoint{0.000000in}{-0.048611in}}{\pgfqpoint{0.000000in}{0.000000in}}{%
\pgfpathmoveto{\pgfqpoint{0.000000in}{0.000000in}}%
\pgfpathlineto{\pgfqpoint{0.000000in}{-0.048611in}}%
\pgfusepath{stroke,fill}%
}%
\begin{pgfscope}%
\pgfsys@transformshift{1.307614in}{0.499444in}%
\pgfsys@useobject{currentmarker}{}%
\end{pgfscope}%
\end{pgfscope}%
\begin{pgfscope}%
\definecolor{textcolor}{rgb}{0.000000,0.000000,0.000000}%
\pgfsetstrokecolor{textcolor}%
\pgfsetfillcolor{textcolor}%
\pgftext[x=1.307614in,y=0.402222in,,top]{\color{textcolor}\rmfamily\fontsize{10.000000}{12.000000}\selectfont 0.2}%
\end{pgfscope}%
\begin{pgfscope}%
\pgfsetbuttcap%
\pgfsetroundjoin%
\definecolor{currentfill}{rgb}{0.000000,0.000000,0.000000}%
\pgfsetfillcolor{currentfill}%
\pgfsetlinewidth{0.803000pt}%
\definecolor{currentstroke}{rgb}{0.000000,0.000000,0.000000}%
\pgfsetstrokecolor{currentstroke}%
\pgfsetdash{}{0pt}%
\pgfsys@defobject{currentmarker}{\pgfqpoint{0.000000in}{-0.048611in}}{\pgfqpoint{0.000000in}{0.000000in}}{%
\pgfpathmoveto{\pgfqpoint{0.000000in}{0.000000in}}%
\pgfpathlineto{\pgfqpoint{0.000000in}{-0.048611in}}%
\pgfusepath{stroke,fill}%
}%
\begin{pgfscope}%
\pgfsys@transformshift{1.466137in}{0.499444in}%
\pgfsys@useobject{currentmarker}{}%
\end{pgfscope}%
\end{pgfscope}%
\begin{pgfscope}%
\pgfsetbuttcap%
\pgfsetroundjoin%
\definecolor{currentfill}{rgb}{0.000000,0.000000,0.000000}%
\pgfsetfillcolor{currentfill}%
\pgfsetlinewidth{0.803000pt}%
\definecolor{currentstroke}{rgb}{0.000000,0.000000,0.000000}%
\pgfsetstrokecolor{currentstroke}%
\pgfsetdash{}{0pt}%
\pgfsys@defobject{currentmarker}{\pgfqpoint{0.000000in}{-0.048611in}}{\pgfqpoint{0.000000in}{0.000000in}}{%
\pgfpathmoveto{\pgfqpoint{0.000000in}{0.000000in}}%
\pgfpathlineto{\pgfqpoint{0.000000in}{-0.048611in}}%
\pgfusepath{stroke,fill}%
}%
\begin{pgfscope}%
\pgfsys@transformshift{1.624659in}{0.499444in}%
\pgfsys@useobject{currentmarker}{}%
\end{pgfscope}%
\end{pgfscope}%
\begin{pgfscope}%
\definecolor{textcolor}{rgb}{0.000000,0.000000,0.000000}%
\pgfsetstrokecolor{textcolor}%
\pgfsetfillcolor{textcolor}%
\pgftext[x=1.624659in,y=0.402222in,,top]{\color{textcolor}\rmfamily\fontsize{10.000000}{12.000000}\selectfont 0.3}%
\end{pgfscope}%
\begin{pgfscope}%
\pgfsetbuttcap%
\pgfsetroundjoin%
\definecolor{currentfill}{rgb}{0.000000,0.000000,0.000000}%
\pgfsetfillcolor{currentfill}%
\pgfsetlinewidth{0.803000pt}%
\definecolor{currentstroke}{rgb}{0.000000,0.000000,0.000000}%
\pgfsetstrokecolor{currentstroke}%
\pgfsetdash{}{0pt}%
\pgfsys@defobject{currentmarker}{\pgfqpoint{0.000000in}{-0.048611in}}{\pgfqpoint{0.000000in}{0.000000in}}{%
\pgfpathmoveto{\pgfqpoint{0.000000in}{0.000000in}}%
\pgfpathlineto{\pgfqpoint{0.000000in}{-0.048611in}}%
\pgfusepath{stroke,fill}%
}%
\begin{pgfscope}%
\pgfsys@transformshift{1.783182in}{0.499444in}%
\pgfsys@useobject{currentmarker}{}%
\end{pgfscope}%
\end{pgfscope}%
\begin{pgfscope}%
\pgfsetbuttcap%
\pgfsetroundjoin%
\definecolor{currentfill}{rgb}{0.000000,0.000000,0.000000}%
\pgfsetfillcolor{currentfill}%
\pgfsetlinewidth{0.803000pt}%
\definecolor{currentstroke}{rgb}{0.000000,0.000000,0.000000}%
\pgfsetstrokecolor{currentstroke}%
\pgfsetdash{}{0pt}%
\pgfsys@defobject{currentmarker}{\pgfqpoint{0.000000in}{-0.048611in}}{\pgfqpoint{0.000000in}{0.000000in}}{%
\pgfpathmoveto{\pgfqpoint{0.000000in}{0.000000in}}%
\pgfpathlineto{\pgfqpoint{0.000000in}{-0.048611in}}%
\pgfusepath{stroke,fill}%
}%
\begin{pgfscope}%
\pgfsys@transformshift{1.941705in}{0.499444in}%
\pgfsys@useobject{currentmarker}{}%
\end{pgfscope}%
\end{pgfscope}%
\begin{pgfscope}%
\definecolor{textcolor}{rgb}{0.000000,0.000000,0.000000}%
\pgfsetstrokecolor{textcolor}%
\pgfsetfillcolor{textcolor}%
\pgftext[x=1.941705in,y=0.402222in,,top]{\color{textcolor}\rmfamily\fontsize{10.000000}{12.000000}\selectfont 0.4}%
\end{pgfscope}%
\begin{pgfscope}%
\pgfsetbuttcap%
\pgfsetroundjoin%
\definecolor{currentfill}{rgb}{0.000000,0.000000,0.000000}%
\pgfsetfillcolor{currentfill}%
\pgfsetlinewidth{0.803000pt}%
\definecolor{currentstroke}{rgb}{0.000000,0.000000,0.000000}%
\pgfsetstrokecolor{currentstroke}%
\pgfsetdash{}{0pt}%
\pgfsys@defobject{currentmarker}{\pgfqpoint{0.000000in}{-0.048611in}}{\pgfqpoint{0.000000in}{0.000000in}}{%
\pgfpathmoveto{\pgfqpoint{0.000000in}{0.000000in}}%
\pgfpathlineto{\pgfqpoint{0.000000in}{-0.048611in}}%
\pgfusepath{stroke,fill}%
}%
\begin{pgfscope}%
\pgfsys@transformshift{2.100228in}{0.499444in}%
\pgfsys@useobject{currentmarker}{}%
\end{pgfscope}%
\end{pgfscope}%
\begin{pgfscope}%
\pgfsetbuttcap%
\pgfsetroundjoin%
\definecolor{currentfill}{rgb}{0.000000,0.000000,0.000000}%
\pgfsetfillcolor{currentfill}%
\pgfsetlinewidth{0.803000pt}%
\definecolor{currentstroke}{rgb}{0.000000,0.000000,0.000000}%
\pgfsetstrokecolor{currentstroke}%
\pgfsetdash{}{0pt}%
\pgfsys@defobject{currentmarker}{\pgfqpoint{0.000000in}{-0.048611in}}{\pgfqpoint{0.000000in}{0.000000in}}{%
\pgfpathmoveto{\pgfqpoint{0.000000in}{0.000000in}}%
\pgfpathlineto{\pgfqpoint{0.000000in}{-0.048611in}}%
\pgfusepath{stroke,fill}%
}%
\begin{pgfscope}%
\pgfsys@transformshift{2.258750in}{0.499444in}%
\pgfsys@useobject{currentmarker}{}%
\end{pgfscope}%
\end{pgfscope}%
\begin{pgfscope}%
\definecolor{textcolor}{rgb}{0.000000,0.000000,0.000000}%
\pgfsetstrokecolor{textcolor}%
\pgfsetfillcolor{textcolor}%
\pgftext[x=2.258750in,y=0.402222in,,top]{\color{textcolor}\rmfamily\fontsize{10.000000}{12.000000}\selectfont 0.5}%
\end{pgfscope}%
\begin{pgfscope}%
\pgfsetbuttcap%
\pgfsetroundjoin%
\definecolor{currentfill}{rgb}{0.000000,0.000000,0.000000}%
\pgfsetfillcolor{currentfill}%
\pgfsetlinewidth{0.803000pt}%
\definecolor{currentstroke}{rgb}{0.000000,0.000000,0.000000}%
\pgfsetstrokecolor{currentstroke}%
\pgfsetdash{}{0pt}%
\pgfsys@defobject{currentmarker}{\pgfqpoint{0.000000in}{-0.048611in}}{\pgfqpoint{0.000000in}{0.000000in}}{%
\pgfpathmoveto{\pgfqpoint{0.000000in}{0.000000in}}%
\pgfpathlineto{\pgfqpoint{0.000000in}{-0.048611in}}%
\pgfusepath{stroke,fill}%
}%
\begin{pgfscope}%
\pgfsys@transformshift{2.417273in}{0.499444in}%
\pgfsys@useobject{currentmarker}{}%
\end{pgfscope}%
\end{pgfscope}%
\begin{pgfscope}%
\pgfsetbuttcap%
\pgfsetroundjoin%
\definecolor{currentfill}{rgb}{0.000000,0.000000,0.000000}%
\pgfsetfillcolor{currentfill}%
\pgfsetlinewidth{0.803000pt}%
\definecolor{currentstroke}{rgb}{0.000000,0.000000,0.000000}%
\pgfsetstrokecolor{currentstroke}%
\pgfsetdash{}{0pt}%
\pgfsys@defobject{currentmarker}{\pgfqpoint{0.000000in}{-0.048611in}}{\pgfqpoint{0.000000in}{0.000000in}}{%
\pgfpathmoveto{\pgfqpoint{0.000000in}{0.000000in}}%
\pgfpathlineto{\pgfqpoint{0.000000in}{-0.048611in}}%
\pgfusepath{stroke,fill}%
}%
\begin{pgfscope}%
\pgfsys@transformshift{2.575796in}{0.499444in}%
\pgfsys@useobject{currentmarker}{}%
\end{pgfscope}%
\end{pgfscope}%
\begin{pgfscope}%
\definecolor{textcolor}{rgb}{0.000000,0.000000,0.000000}%
\pgfsetstrokecolor{textcolor}%
\pgfsetfillcolor{textcolor}%
\pgftext[x=2.575796in,y=0.402222in,,top]{\color{textcolor}\rmfamily\fontsize{10.000000}{12.000000}\selectfont 0.6}%
\end{pgfscope}%
\begin{pgfscope}%
\pgfsetbuttcap%
\pgfsetroundjoin%
\definecolor{currentfill}{rgb}{0.000000,0.000000,0.000000}%
\pgfsetfillcolor{currentfill}%
\pgfsetlinewidth{0.803000pt}%
\definecolor{currentstroke}{rgb}{0.000000,0.000000,0.000000}%
\pgfsetstrokecolor{currentstroke}%
\pgfsetdash{}{0pt}%
\pgfsys@defobject{currentmarker}{\pgfqpoint{0.000000in}{-0.048611in}}{\pgfqpoint{0.000000in}{0.000000in}}{%
\pgfpathmoveto{\pgfqpoint{0.000000in}{0.000000in}}%
\pgfpathlineto{\pgfqpoint{0.000000in}{-0.048611in}}%
\pgfusepath{stroke,fill}%
}%
\begin{pgfscope}%
\pgfsys@transformshift{2.734318in}{0.499444in}%
\pgfsys@useobject{currentmarker}{}%
\end{pgfscope}%
\end{pgfscope}%
\begin{pgfscope}%
\pgfsetbuttcap%
\pgfsetroundjoin%
\definecolor{currentfill}{rgb}{0.000000,0.000000,0.000000}%
\pgfsetfillcolor{currentfill}%
\pgfsetlinewidth{0.803000pt}%
\definecolor{currentstroke}{rgb}{0.000000,0.000000,0.000000}%
\pgfsetstrokecolor{currentstroke}%
\pgfsetdash{}{0pt}%
\pgfsys@defobject{currentmarker}{\pgfqpoint{0.000000in}{-0.048611in}}{\pgfqpoint{0.000000in}{0.000000in}}{%
\pgfpathmoveto{\pgfqpoint{0.000000in}{0.000000in}}%
\pgfpathlineto{\pgfqpoint{0.000000in}{-0.048611in}}%
\pgfusepath{stroke,fill}%
}%
\begin{pgfscope}%
\pgfsys@transformshift{2.892841in}{0.499444in}%
\pgfsys@useobject{currentmarker}{}%
\end{pgfscope}%
\end{pgfscope}%
\begin{pgfscope}%
\definecolor{textcolor}{rgb}{0.000000,0.000000,0.000000}%
\pgfsetstrokecolor{textcolor}%
\pgfsetfillcolor{textcolor}%
\pgftext[x=2.892841in,y=0.402222in,,top]{\color{textcolor}\rmfamily\fontsize{10.000000}{12.000000}\selectfont 0.7}%
\end{pgfscope}%
\begin{pgfscope}%
\pgfsetbuttcap%
\pgfsetroundjoin%
\definecolor{currentfill}{rgb}{0.000000,0.000000,0.000000}%
\pgfsetfillcolor{currentfill}%
\pgfsetlinewidth{0.803000pt}%
\definecolor{currentstroke}{rgb}{0.000000,0.000000,0.000000}%
\pgfsetstrokecolor{currentstroke}%
\pgfsetdash{}{0pt}%
\pgfsys@defobject{currentmarker}{\pgfqpoint{0.000000in}{-0.048611in}}{\pgfqpoint{0.000000in}{0.000000in}}{%
\pgfpathmoveto{\pgfqpoint{0.000000in}{0.000000in}}%
\pgfpathlineto{\pgfqpoint{0.000000in}{-0.048611in}}%
\pgfusepath{stroke,fill}%
}%
\begin{pgfscope}%
\pgfsys@transformshift{3.051364in}{0.499444in}%
\pgfsys@useobject{currentmarker}{}%
\end{pgfscope}%
\end{pgfscope}%
\begin{pgfscope}%
\pgfsetbuttcap%
\pgfsetroundjoin%
\definecolor{currentfill}{rgb}{0.000000,0.000000,0.000000}%
\pgfsetfillcolor{currentfill}%
\pgfsetlinewidth{0.803000pt}%
\definecolor{currentstroke}{rgb}{0.000000,0.000000,0.000000}%
\pgfsetstrokecolor{currentstroke}%
\pgfsetdash{}{0pt}%
\pgfsys@defobject{currentmarker}{\pgfqpoint{0.000000in}{-0.048611in}}{\pgfqpoint{0.000000in}{0.000000in}}{%
\pgfpathmoveto{\pgfqpoint{0.000000in}{0.000000in}}%
\pgfpathlineto{\pgfqpoint{0.000000in}{-0.048611in}}%
\pgfusepath{stroke,fill}%
}%
\begin{pgfscope}%
\pgfsys@transformshift{3.209887in}{0.499444in}%
\pgfsys@useobject{currentmarker}{}%
\end{pgfscope}%
\end{pgfscope}%
\begin{pgfscope}%
\definecolor{textcolor}{rgb}{0.000000,0.000000,0.000000}%
\pgfsetstrokecolor{textcolor}%
\pgfsetfillcolor{textcolor}%
\pgftext[x=3.209887in,y=0.402222in,,top]{\color{textcolor}\rmfamily\fontsize{10.000000}{12.000000}\selectfont 0.8}%
\end{pgfscope}%
\begin{pgfscope}%
\pgfsetbuttcap%
\pgfsetroundjoin%
\definecolor{currentfill}{rgb}{0.000000,0.000000,0.000000}%
\pgfsetfillcolor{currentfill}%
\pgfsetlinewidth{0.803000pt}%
\definecolor{currentstroke}{rgb}{0.000000,0.000000,0.000000}%
\pgfsetstrokecolor{currentstroke}%
\pgfsetdash{}{0pt}%
\pgfsys@defobject{currentmarker}{\pgfqpoint{0.000000in}{-0.048611in}}{\pgfqpoint{0.000000in}{0.000000in}}{%
\pgfpathmoveto{\pgfqpoint{0.000000in}{0.000000in}}%
\pgfpathlineto{\pgfqpoint{0.000000in}{-0.048611in}}%
\pgfusepath{stroke,fill}%
}%
\begin{pgfscope}%
\pgfsys@transformshift{3.368409in}{0.499444in}%
\pgfsys@useobject{currentmarker}{}%
\end{pgfscope}%
\end{pgfscope}%
\begin{pgfscope}%
\pgfsetbuttcap%
\pgfsetroundjoin%
\definecolor{currentfill}{rgb}{0.000000,0.000000,0.000000}%
\pgfsetfillcolor{currentfill}%
\pgfsetlinewidth{0.803000pt}%
\definecolor{currentstroke}{rgb}{0.000000,0.000000,0.000000}%
\pgfsetstrokecolor{currentstroke}%
\pgfsetdash{}{0pt}%
\pgfsys@defobject{currentmarker}{\pgfqpoint{0.000000in}{-0.048611in}}{\pgfqpoint{0.000000in}{0.000000in}}{%
\pgfpathmoveto{\pgfqpoint{0.000000in}{0.000000in}}%
\pgfpathlineto{\pgfqpoint{0.000000in}{-0.048611in}}%
\pgfusepath{stroke,fill}%
}%
\begin{pgfscope}%
\pgfsys@transformshift{3.526932in}{0.499444in}%
\pgfsys@useobject{currentmarker}{}%
\end{pgfscope}%
\end{pgfscope}%
\begin{pgfscope}%
\definecolor{textcolor}{rgb}{0.000000,0.000000,0.000000}%
\pgfsetstrokecolor{textcolor}%
\pgfsetfillcolor{textcolor}%
\pgftext[x=3.526932in,y=0.402222in,,top]{\color{textcolor}\rmfamily\fontsize{10.000000}{12.000000}\selectfont 0.9}%
\end{pgfscope}%
\begin{pgfscope}%
\pgfsetbuttcap%
\pgfsetroundjoin%
\definecolor{currentfill}{rgb}{0.000000,0.000000,0.000000}%
\pgfsetfillcolor{currentfill}%
\pgfsetlinewidth{0.803000pt}%
\definecolor{currentstroke}{rgb}{0.000000,0.000000,0.000000}%
\pgfsetstrokecolor{currentstroke}%
\pgfsetdash{}{0pt}%
\pgfsys@defobject{currentmarker}{\pgfqpoint{0.000000in}{-0.048611in}}{\pgfqpoint{0.000000in}{0.000000in}}{%
\pgfpathmoveto{\pgfqpoint{0.000000in}{0.000000in}}%
\pgfpathlineto{\pgfqpoint{0.000000in}{-0.048611in}}%
\pgfusepath{stroke,fill}%
}%
\begin{pgfscope}%
\pgfsys@transformshift{3.685455in}{0.499444in}%
\pgfsys@useobject{currentmarker}{}%
\end{pgfscope}%
\end{pgfscope}%
\begin{pgfscope}%
\pgfsetbuttcap%
\pgfsetroundjoin%
\definecolor{currentfill}{rgb}{0.000000,0.000000,0.000000}%
\pgfsetfillcolor{currentfill}%
\pgfsetlinewidth{0.803000pt}%
\definecolor{currentstroke}{rgb}{0.000000,0.000000,0.000000}%
\pgfsetstrokecolor{currentstroke}%
\pgfsetdash{}{0pt}%
\pgfsys@defobject{currentmarker}{\pgfqpoint{0.000000in}{-0.048611in}}{\pgfqpoint{0.000000in}{0.000000in}}{%
\pgfpathmoveto{\pgfqpoint{0.000000in}{0.000000in}}%
\pgfpathlineto{\pgfqpoint{0.000000in}{-0.048611in}}%
\pgfusepath{stroke,fill}%
}%
\begin{pgfscope}%
\pgfsys@transformshift{3.843978in}{0.499444in}%
\pgfsys@useobject{currentmarker}{}%
\end{pgfscope}%
\end{pgfscope}%
\begin{pgfscope}%
\definecolor{textcolor}{rgb}{0.000000,0.000000,0.000000}%
\pgfsetstrokecolor{textcolor}%
\pgfsetfillcolor{textcolor}%
\pgftext[x=3.843978in,y=0.402222in,,top]{\color{textcolor}\rmfamily\fontsize{10.000000}{12.000000}\selectfont 1.0}%
\end{pgfscope}%
\begin{pgfscope}%
\pgfsetbuttcap%
\pgfsetroundjoin%
\definecolor{currentfill}{rgb}{0.000000,0.000000,0.000000}%
\pgfsetfillcolor{currentfill}%
\pgfsetlinewidth{0.803000pt}%
\definecolor{currentstroke}{rgb}{0.000000,0.000000,0.000000}%
\pgfsetstrokecolor{currentstroke}%
\pgfsetdash{}{0pt}%
\pgfsys@defobject{currentmarker}{\pgfqpoint{0.000000in}{-0.048611in}}{\pgfqpoint{0.000000in}{0.000000in}}{%
\pgfpathmoveto{\pgfqpoint{0.000000in}{0.000000in}}%
\pgfpathlineto{\pgfqpoint{0.000000in}{-0.048611in}}%
\pgfusepath{stroke,fill}%
}%
\begin{pgfscope}%
\pgfsys@transformshift{4.002500in}{0.499444in}%
\pgfsys@useobject{currentmarker}{}%
\end{pgfscope}%
\end{pgfscope}%
\begin{pgfscope}%
\definecolor{textcolor}{rgb}{0.000000,0.000000,0.000000}%
\pgfsetstrokecolor{textcolor}%
\pgfsetfillcolor{textcolor}%
\pgftext[x=2.258750in,y=0.223333in,,top]{\color{textcolor}\rmfamily\fontsize{10.000000}{12.000000}\selectfont \(\displaystyle p\)}%
\end{pgfscope}%
\begin{pgfscope}%
\pgfsetbuttcap%
\pgfsetroundjoin%
\definecolor{currentfill}{rgb}{0.000000,0.000000,0.000000}%
\pgfsetfillcolor{currentfill}%
\pgfsetlinewidth{0.803000pt}%
\definecolor{currentstroke}{rgb}{0.000000,0.000000,0.000000}%
\pgfsetstrokecolor{currentstroke}%
\pgfsetdash{}{0pt}%
\pgfsys@defobject{currentmarker}{\pgfqpoint{-0.048611in}{0.000000in}}{\pgfqpoint{-0.000000in}{0.000000in}}{%
\pgfpathmoveto{\pgfqpoint{-0.000000in}{0.000000in}}%
\pgfpathlineto{\pgfqpoint{-0.048611in}{0.000000in}}%
\pgfusepath{stroke,fill}%
}%
\begin{pgfscope}%
\pgfsys@transformshift{0.515000in}{0.499444in}%
\pgfsys@useobject{currentmarker}{}%
\end{pgfscope}%
\end{pgfscope}%
\begin{pgfscope}%
\definecolor{textcolor}{rgb}{0.000000,0.000000,0.000000}%
\pgfsetstrokecolor{textcolor}%
\pgfsetfillcolor{textcolor}%
\pgftext[x=0.348333in, y=0.451250in, left, base]{\color{textcolor}\rmfamily\fontsize{10.000000}{12.000000}\selectfont \(\displaystyle {0}\)}%
\end{pgfscope}%
\begin{pgfscope}%
\pgfsetbuttcap%
\pgfsetroundjoin%
\definecolor{currentfill}{rgb}{0.000000,0.000000,0.000000}%
\pgfsetfillcolor{currentfill}%
\pgfsetlinewidth{0.803000pt}%
\definecolor{currentstroke}{rgb}{0.000000,0.000000,0.000000}%
\pgfsetstrokecolor{currentstroke}%
\pgfsetdash{}{0pt}%
\pgfsys@defobject{currentmarker}{\pgfqpoint{-0.048611in}{0.000000in}}{\pgfqpoint{-0.000000in}{0.000000in}}{%
\pgfpathmoveto{\pgfqpoint{-0.000000in}{0.000000in}}%
\pgfpathlineto{\pgfqpoint{-0.048611in}{0.000000in}}%
\pgfusepath{stroke,fill}%
}%
\begin{pgfscope}%
\pgfsys@transformshift{0.515000in}{0.894863in}%
\pgfsys@useobject{currentmarker}{}%
\end{pgfscope}%
\end{pgfscope}%
\begin{pgfscope}%
\definecolor{textcolor}{rgb}{0.000000,0.000000,0.000000}%
\pgfsetstrokecolor{textcolor}%
\pgfsetfillcolor{textcolor}%
\pgftext[x=0.348333in, y=0.846669in, left, base]{\color{textcolor}\rmfamily\fontsize{10.000000}{12.000000}\selectfont \(\displaystyle {5}\)}%
\end{pgfscope}%
\begin{pgfscope}%
\pgfsetbuttcap%
\pgfsetroundjoin%
\definecolor{currentfill}{rgb}{0.000000,0.000000,0.000000}%
\pgfsetfillcolor{currentfill}%
\pgfsetlinewidth{0.803000pt}%
\definecolor{currentstroke}{rgb}{0.000000,0.000000,0.000000}%
\pgfsetstrokecolor{currentstroke}%
\pgfsetdash{}{0pt}%
\pgfsys@defobject{currentmarker}{\pgfqpoint{-0.048611in}{0.000000in}}{\pgfqpoint{-0.000000in}{0.000000in}}{%
\pgfpathmoveto{\pgfqpoint{-0.000000in}{0.000000in}}%
\pgfpathlineto{\pgfqpoint{-0.048611in}{0.000000in}}%
\pgfusepath{stroke,fill}%
}%
\begin{pgfscope}%
\pgfsys@transformshift{0.515000in}{1.290282in}%
\pgfsys@useobject{currentmarker}{}%
\end{pgfscope}%
\end{pgfscope}%
\begin{pgfscope}%
\definecolor{textcolor}{rgb}{0.000000,0.000000,0.000000}%
\pgfsetstrokecolor{textcolor}%
\pgfsetfillcolor{textcolor}%
\pgftext[x=0.278889in, y=1.242088in, left, base]{\color{textcolor}\rmfamily\fontsize{10.000000}{12.000000}\selectfont \(\displaystyle {10}\)}%
\end{pgfscope}%
\begin{pgfscope}%
\definecolor{textcolor}{rgb}{0.000000,0.000000,0.000000}%
\pgfsetstrokecolor{textcolor}%
\pgfsetfillcolor{textcolor}%
\pgftext[x=0.223333in,y=1.076944in,,bottom,rotate=90.000000]{\color{textcolor}\rmfamily\fontsize{10.000000}{12.000000}\selectfont Percent of Data Set}%
\end{pgfscope}%
\begin{pgfscope}%
\pgfsetrectcap%
\pgfsetmiterjoin%
\pgfsetlinewidth{0.803000pt}%
\definecolor{currentstroke}{rgb}{0.000000,0.000000,0.000000}%
\pgfsetstrokecolor{currentstroke}%
\pgfsetdash{}{0pt}%
\pgfpathmoveto{\pgfqpoint{0.515000in}{0.499444in}}%
\pgfpathlineto{\pgfqpoint{0.515000in}{1.654444in}}%
\pgfusepath{stroke}%
\end{pgfscope}%
\begin{pgfscope}%
\pgfsetrectcap%
\pgfsetmiterjoin%
\pgfsetlinewidth{0.803000pt}%
\definecolor{currentstroke}{rgb}{0.000000,0.000000,0.000000}%
\pgfsetstrokecolor{currentstroke}%
\pgfsetdash{}{0pt}%
\pgfpathmoveto{\pgfqpoint{4.002500in}{0.499444in}}%
\pgfpathlineto{\pgfqpoint{4.002500in}{1.654444in}}%
\pgfusepath{stroke}%
\end{pgfscope}%
\begin{pgfscope}%
\pgfsetrectcap%
\pgfsetmiterjoin%
\pgfsetlinewidth{0.803000pt}%
\definecolor{currentstroke}{rgb}{0.000000,0.000000,0.000000}%
\pgfsetstrokecolor{currentstroke}%
\pgfsetdash{}{0pt}%
\pgfpathmoveto{\pgfqpoint{0.515000in}{0.499444in}}%
\pgfpathlineto{\pgfqpoint{4.002500in}{0.499444in}}%
\pgfusepath{stroke}%
\end{pgfscope}%
\begin{pgfscope}%
\pgfsetrectcap%
\pgfsetmiterjoin%
\pgfsetlinewidth{0.803000pt}%
\definecolor{currentstroke}{rgb}{0.000000,0.000000,0.000000}%
\pgfsetstrokecolor{currentstroke}%
\pgfsetdash{}{0pt}%
\pgfpathmoveto{\pgfqpoint{0.515000in}{1.654444in}}%
\pgfpathlineto{\pgfqpoint{4.002500in}{1.654444in}}%
\pgfusepath{stroke}%
\end{pgfscope}%
\begin{pgfscope}%
\pgfsetbuttcap%
\pgfsetmiterjoin%
\definecolor{currentfill}{rgb}{1.000000,1.000000,1.000000}%
\pgfsetfillcolor{currentfill}%
\pgfsetfillopacity{0.800000}%
\pgfsetlinewidth{1.003750pt}%
\definecolor{currentstroke}{rgb}{0.800000,0.800000,0.800000}%
\pgfsetstrokecolor{currentstroke}%
\pgfsetstrokeopacity{0.800000}%
\pgfsetdash{}{0pt}%
\pgfpathmoveto{\pgfqpoint{3.225556in}{1.154445in}}%
\pgfpathlineto{\pgfqpoint{3.905278in}{1.154445in}}%
\pgfpathquadraticcurveto{\pgfqpoint{3.933056in}{1.154445in}}{\pgfqpoint{3.933056in}{1.182222in}}%
\pgfpathlineto{\pgfqpoint{3.933056in}{1.557222in}}%
\pgfpathquadraticcurveto{\pgfqpoint{3.933056in}{1.585000in}}{\pgfqpoint{3.905278in}{1.585000in}}%
\pgfpathlineto{\pgfqpoint{3.225556in}{1.585000in}}%
\pgfpathquadraticcurveto{\pgfqpoint{3.197778in}{1.585000in}}{\pgfqpoint{3.197778in}{1.557222in}}%
\pgfpathlineto{\pgfqpoint{3.197778in}{1.182222in}}%
\pgfpathquadraticcurveto{\pgfqpoint{3.197778in}{1.154445in}}{\pgfqpoint{3.225556in}{1.154445in}}%
\pgfpathlineto{\pgfqpoint{3.225556in}{1.154445in}}%
\pgfpathclose%
\pgfusepath{stroke,fill}%
\end{pgfscope}%
\begin{pgfscope}%
\pgfsetbuttcap%
\pgfsetmiterjoin%
\pgfsetlinewidth{1.003750pt}%
\definecolor{currentstroke}{rgb}{0.000000,0.000000,0.000000}%
\pgfsetstrokecolor{currentstroke}%
\pgfsetdash{}{0pt}%
\pgfpathmoveto{\pgfqpoint{3.253334in}{1.432222in}}%
\pgfpathlineto{\pgfqpoint{3.531111in}{1.432222in}}%
\pgfpathlineto{\pgfqpoint{3.531111in}{1.529444in}}%
\pgfpathlineto{\pgfqpoint{3.253334in}{1.529444in}}%
\pgfpathlineto{\pgfqpoint{3.253334in}{1.432222in}}%
\pgfpathclose%
\pgfusepath{stroke}%
\end{pgfscope}%
\begin{pgfscope}%
\definecolor{textcolor}{rgb}{0.000000,0.000000,0.000000}%
\pgfsetstrokecolor{textcolor}%
\pgfsetfillcolor{textcolor}%
\pgftext[x=3.642223in,y=1.432222in,left,base]{\color{textcolor}\rmfamily\fontsize{10.000000}{12.000000}\selectfont Neg}%
\end{pgfscope}%
\begin{pgfscope}%
\pgfsetbuttcap%
\pgfsetmiterjoin%
\definecolor{currentfill}{rgb}{0.000000,0.000000,0.000000}%
\pgfsetfillcolor{currentfill}%
\pgfsetlinewidth{0.000000pt}%
\definecolor{currentstroke}{rgb}{0.000000,0.000000,0.000000}%
\pgfsetstrokecolor{currentstroke}%
\pgfsetstrokeopacity{0.000000}%
\pgfsetdash{}{0pt}%
\pgfpathmoveto{\pgfqpoint{3.253334in}{1.236944in}}%
\pgfpathlineto{\pgfqpoint{3.531111in}{1.236944in}}%
\pgfpathlineto{\pgfqpoint{3.531111in}{1.334167in}}%
\pgfpathlineto{\pgfqpoint{3.253334in}{1.334167in}}%
\pgfpathlineto{\pgfqpoint{3.253334in}{1.236944in}}%
\pgfpathclose%
\pgfusepath{fill}%
\end{pgfscope}%
\begin{pgfscope}%
\definecolor{textcolor}{rgb}{0.000000,0.000000,0.000000}%
\pgfsetstrokecolor{textcolor}%
\pgfsetfillcolor{textcolor}%
\pgftext[x=3.642223in,y=1.236944in,left,base]{\color{textcolor}\rmfamily\fontsize{10.000000}{12.000000}\selectfont Pos}%
\end{pgfscope}%
\end{pgfpicture}%
\makeatother%
\endgroup%
	
&
	\vskip 0pt
	\hfil ROC Curve
	
	%% Creator: Matplotlib, PGF backend
%%
%% To include the figure in your LaTeX document, write
%%   \input{<filename>.pgf}
%%
%% Make sure the required packages are loaded in your preamble
%%   \usepackage{pgf}
%%
%% Also ensure that all the required font packages are loaded; for instance,
%% the lmodern package is sometimes necessary when using math font.
%%   \usepackage{lmodern}
%%
%% Figures using additional raster images can only be included by \input if
%% they are in the same directory as the main LaTeX file. For loading figures
%% from other directories you can use the `import` package
%%   \usepackage{import}
%%
%% and then include the figures with
%%   \import{<path to file>}{<filename>.pgf}
%%
%% Matplotlib used the following preamble
%%   
%%   \usepackage{fontspec}
%%   \makeatletter\@ifpackageloaded{underscore}{}{\usepackage[strings]{underscore}}\makeatother
%%
\begingroup%
\makeatletter%
\begin{pgfpicture}%
\pgfpathrectangle{\pgfpointorigin}{\pgfqpoint{2.221861in}{1.754444in}}%
\pgfusepath{use as bounding box, clip}%
\begin{pgfscope}%
\pgfsetbuttcap%
\pgfsetmiterjoin%
\definecolor{currentfill}{rgb}{1.000000,1.000000,1.000000}%
\pgfsetfillcolor{currentfill}%
\pgfsetlinewidth{0.000000pt}%
\definecolor{currentstroke}{rgb}{1.000000,1.000000,1.000000}%
\pgfsetstrokecolor{currentstroke}%
\pgfsetdash{}{0pt}%
\pgfpathmoveto{\pgfqpoint{0.000000in}{0.000000in}}%
\pgfpathlineto{\pgfqpoint{2.221861in}{0.000000in}}%
\pgfpathlineto{\pgfqpoint{2.221861in}{1.754444in}}%
\pgfpathlineto{\pgfqpoint{0.000000in}{1.754444in}}%
\pgfpathlineto{\pgfqpoint{0.000000in}{0.000000in}}%
\pgfpathclose%
\pgfusepath{fill}%
\end{pgfscope}%
\begin{pgfscope}%
\pgfsetbuttcap%
\pgfsetmiterjoin%
\definecolor{currentfill}{rgb}{1.000000,1.000000,1.000000}%
\pgfsetfillcolor{currentfill}%
\pgfsetlinewidth{0.000000pt}%
\definecolor{currentstroke}{rgb}{0.000000,0.000000,0.000000}%
\pgfsetstrokecolor{currentstroke}%
\pgfsetstrokeopacity{0.000000}%
\pgfsetdash{}{0pt}%
\pgfpathmoveto{\pgfqpoint{0.553581in}{0.499444in}}%
\pgfpathlineto{\pgfqpoint{2.103581in}{0.499444in}}%
\pgfpathlineto{\pgfqpoint{2.103581in}{1.654444in}}%
\pgfpathlineto{\pgfqpoint{0.553581in}{1.654444in}}%
\pgfpathlineto{\pgfqpoint{0.553581in}{0.499444in}}%
\pgfpathclose%
\pgfusepath{fill}%
\end{pgfscope}%
\begin{pgfscope}%
\pgfsetbuttcap%
\pgfsetroundjoin%
\definecolor{currentfill}{rgb}{0.000000,0.000000,0.000000}%
\pgfsetfillcolor{currentfill}%
\pgfsetlinewidth{0.803000pt}%
\definecolor{currentstroke}{rgb}{0.000000,0.000000,0.000000}%
\pgfsetstrokecolor{currentstroke}%
\pgfsetdash{}{0pt}%
\pgfsys@defobject{currentmarker}{\pgfqpoint{0.000000in}{-0.048611in}}{\pgfqpoint{0.000000in}{0.000000in}}{%
\pgfpathmoveto{\pgfqpoint{0.000000in}{0.000000in}}%
\pgfpathlineto{\pgfqpoint{0.000000in}{-0.048611in}}%
\pgfusepath{stroke,fill}%
}%
\begin{pgfscope}%
\pgfsys@transformshift{0.624035in}{0.499444in}%
\pgfsys@useobject{currentmarker}{}%
\end{pgfscope}%
\end{pgfscope}%
\begin{pgfscope}%
\definecolor{textcolor}{rgb}{0.000000,0.000000,0.000000}%
\pgfsetstrokecolor{textcolor}%
\pgfsetfillcolor{textcolor}%
\pgftext[x=0.624035in,y=0.402222in,,top]{\color{textcolor}\rmfamily\fontsize{10.000000}{12.000000}\selectfont \(\displaystyle {0.0}\)}%
\end{pgfscope}%
\begin{pgfscope}%
\pgfsetbuttcap%
\pgfsetroundjoin%
\definecolor{currentfill}{rgb}{0.000000,0.000000,0.000000}%
\pgfsetfillcolor{currentfill}%
\pgfsetlinewidth{0.803000pt}%
\definecolor{currentstroke}{rgb}{0.000000,0.000000,0.000000}%
\pgfsetstrokecolor{currentstroke}%
\pgfsetdash{}{0pt}%
\pgfsys@defobject{currentmarker}{\pgfqpoint{0.000000in}{-0.048611in}}{\pgfqpoint{0.000000in}{0.000000in}}{%
\pgfpathmoveto{\pgfqpoint{0.000000in}{0.000000in}}%
\pgfpathlineto{\pgfqpoint{0.000000in}{-0.048611in}}%
\pgfusepath{stroke,fill}%
}%
\begin{pgfscope}%
\pgfsys@transformshift{1.328581in}{0.499444in}%
\pgfsys@useobject{currentmarker}{}%
\end{pgfscope}%
\end{pgfscope}%
\begin{pgfscope}%
\definecolor{textcolor}{rgb}{0.000000,0.000000,0.000000}%
\pgfsetstrokecolor{textcolor}%
\pgfsetfillcolor{textcolor}%
\pgftext[x=1.328581in,y=0.402222in,,top]{\color{textcolor}\rmfamily\fontsize{10.000000}{12.000000}\selectfont \(\displaystyle {0.5}\)}%
\end{pgfscope}%
\begin{pgfscope}%
\pgfsetbuttcap%
\pgfsetroundjoin%
\definecolor{currentfill}{rgb}{0.000000,0.000000,0.000000}%
\pgfsetfillcolor{currentfill}%
\pgfsetlinewidth{0.803000pt}%
\definecolor{currentstroke}{rgb}{0.000000,0.000000,0.000000}%
\pgfsetstrokecolor{currentstroke}%
\pgfsetdash{}{0pt}%
\pgfsys@defobject{currentmarker}{\pgfqpoint{0.000000in}{-0.048611in}}{\pgfqpoint{0.000000in}{0.000000in}}{%
\pgfpathmoveto{\pgfqpoint{0.000000in}{0.000000in}}%
\pgfpathlineto{\pgfqpoint{0.000000in}{-0.048611in}}%
\pgfusepath{stroke,fill}%
}%
\begin{pgfscope}%
\pgfsys@transformshift{2.033126in}{0.499444in}%
\pgfsys@useobject{currentmarker}{}%
\end{pgfscope}%
\end{pgfscope}%
\begin{pgfscope}%
\definecolor{textcolor}{rgb}{0.000000,0.000000,0.000000}%
\pgfsetstrokecolor{textcolor}%
\pgfsetfillcolor{textcolor}%
\pgftext[x=2.033126in,y=0.402222in,,top]{\color{textcolor}\rmfamily\fontsize{10.000000}{12.000000}\selectfont \(\displaystyle {1.0}\)}%
\end{pgfscope}%
\begin{pgfscope}%
\definecolor{textcolor}{rgb}{0.000000,0.000000,0.000000}%
\pgfsetstrokecolor{textcolor}%
\pgfsetfillcolor{textcolor}%
\pgftext[x=1.328581in,y=0.223333in,,top]{\color{textcolor}\rmfamily\fontsize{10.000000}{12.000000}\selectfont False positive rate}%
\end{pgfscope}%
\begin{pgfscope}%
\pgfsetbuttcap%
\pgfsetroundjoin%
\definecolor{currentfill}{rgb}{0.000000,0.000000,0.000000}%
\pgfsetfillcolor{currentfill}%
\pgfsetlinewidth{0.803000pt}%
\definecolor{currentstroke}{rgb}{0.000000,0.000000,0.000000}%
\pgfsetstrokecolor{currentstroke}%
\pgfsetdash{}{0pt}%
\pgfsys@defobject{currentmarker}{\pgfqpoint{-0.048611in}{0.000000in}}{\pgfqpoint{-0.000000in}{0.000000in}}{%
\pgfpathmoveto{\pgfqpoint{-0.000000in}{0.000000in}}%
\pgfpathlineto{\pgfqpoint{-0.048611in}{0.000000in}}%
\pgfusepath{stroke,fill}%
}%
\begin{pgfscope}%
\pgfsys@transformshift{0.553581in}{0.551944in}%
\pgfsys@useobject{currentmarker}{}%
\end{pgfscope}%
\end{pgfscope}%
\begin{pgfscope}%
\definecolor{textcolor}{rgb}{0.000000,0.000000,0.000000}%
\pgfsetstrokecolor{textcolor}%
\pgfsetfillcolor{textcolor}%
\pgftext[x=0.278889in, y=0.503750in, left, base]{\color{textcolor}\rmfamily\fontsize{10.000000}{12.000000}\selectfont \(\displaystyle {0.0}\)}%
\end{pgfscope}%
\begin{pgfscope}%
\pgfsetbuttcap%
\pgfsetroundjoin%
\definecolor{currentfill}{rgb}{0.000000,0.000000,0.000000}%
\pgfsetfillcolor{currentfill}%
\pgfsetlinewidth{0.803000pt}%
\definecolor{currentstroke}{rgb}{0.000000,0.000000,0.000000}%
\pgfsetstrokecolor{currentstroke}%
\pgfsetdash{}{0pt}%
\pgfsys@defobject{currentmarker}{\pgfqpoint{-0.048611in}{0.000000in}}{\pgfqpoint{-0.000000in}{0.000000in}}{%
\pgfpathmoveto{\pgfqpoint{-0.000000in}{0.000000in}}%
\pgfpathlineto{\pgfqpoint{-0.048611in}{0.000000in}}%
\pgfusepath{stroke,fill}%
}%
\begin{pgfscope}%
\pgfsys@transformshift{0.553581in}{1.076944in}%
\pgfsys@useobject{currentmarker}{}%
\end{pgfscope}%
\end{pgfscope}%
\begin{pgfscope}%
\definecolor{textcolor}{rgb}{0.000000,0.000000,0.000000}%
\pgfsetstrokecolor{textcolor}%
\pgfsetfillcolor{textcolor}%
\pgftext[x=0.278889in, y=1.028750in, left, base]{\color{textcolor}\rmfamily\fontsize{10.000000}{12.000000}\selectfont \(\displaystyle {0.5}\)}%
\end{pgfscope}%
\begin{pgfscope}%
\pgfsetbuttcap%
\pgfsetroundjoin%
\definecolor{currentfill}{rgb}{0.000000,0.000000,0.000000}%
\pgfsetfillcolor{currentfill}%
\pgfsetlinewidth{0.803000pt}%
\definecolor{currentstroke}{rgb}{0.000000,0.000000,0.000000}%
\pgfsetstrokecolor{currentstroke}%
\pgfsetdash{}{0pt}%
\pgfsys@defobject{currentmarker}{\pgfqpoint{-0.048611in}{0.000000in}}{\pgfqpoint{-0.000000in}{0.000000in}}{%
\pgfpathmoveto{\pgfqpoint{-0.000000in}{0.000000in}}%
\pgfpathlineto{\pgfqpoint{-0.048611in}{0.000000in}}%
\pgfusepath{stroke,fill}%
}%
\begin{pgfscope}%
\pgfsys@transformshift{0.553581in}{1.601944in}%
\pgfsys@useobject{currentmarker}{}%
\end{pgfscope}%
\end{pgfscope}%
\begin{pgfscope}%
\definecolor{textcolor}{rgb}{0.000000,0.000000,0.000000}%
\pgfsetstrokecolor{textcolor}%
\pgfsetfillcolor{textcolor}%
\pgftext[x=0.278889in, y=1.553750in, left, base]{\color{textcolor}\rmfamily\fontsize{10.000000}{12.000000}\selectfont \(\displaystyle {1.0}\)}%
\end{pgfscope}%
\begin{pgfscope}%
\definecolor{textcolor}{rgb}{0.000000,0.000000,0.000000}%
\pgfsetstrokecolor{textcolor}%
\pgfsetfillcolor{textcolor}%
\pgftext[x=0.223333in,y=1.076944in,,bottom,rotate=90.000000]{\color{textcolor}\rmfamily\fontsize{10.000000}{12.000000}\selectfont True positive rate}%
\end{pgfscope}%
\begin{pgfscope}%
\pgfpathrectangle{\pgfqpoint{0.553581in}{0.499444in}}{\pgfqpoint{1.550000in}{1.155000in}}%
\pgfusepath{clip}%
\pgfsetbuttcap%
\pgfsetroundjoin%
\pgfsetlinewidth{1.505625pt}%
\definecolor{currentstroke}{rgb}{0.000000,0.000000,0.000000}%
\pgfsetstrokecolor{currentstroke}%
\pgfsetdash{{5.550000pt}{2.400000pt}}{0.000000pt}%
\pgfpathmoveto{\pgfqpoint{0.624035in}{0.551944in}}%
\pgfpathlineto{\pgfqpoint{2.033126in}{1.601944in}}%
\pgfusepath{stroke}%
\end{pgfscope}%
\begin{pgfscope}%
\pgfpathrectangle{\pgfqpoint{0.553581in}{0.499444in}}{\pgfqpoint{1.550000in}{1.155000in}}%
\pgfusepath{clip}%
\pgfsetrectcap%
\pgfsetroundjoin%
\pgfsetlinewidth{1.505625pt}%
\definecolor{currentstroke}{rgb}{0.000000,0.000000,0.000000}%
\pgfsetstrokecolor{currentstroke}%
\pgfsetdash{}{0pt}%
\pgfpathmoveto{\pgfqpoint{0.624035in}{0.551944in}}%
\pgfpathlineto{\pgfqpoint{0.631695in}{0.617022in}}%
\pgfpathlineto{\pgfqpoint{0.652212in}{0.721112in}}%
\pgfpathlineto{\pgfqpoint{0.691254in}{0.845510in}}%
\pgfpathlineto{\pgfqpoint{0.757678in}{0.979925in}}%
\pgfpathlineto{\pgfqpoint{0.858118in}{1.116063in}}%
\pgfpathlineto{\pgfqpoint{1.000461in}{1.248796in}}%
\pgfpathlineto{\pgfqpoint{1.182339in}{1.366600in}}%
\pgfpathlineto{\pgfqpoint{1.397624in}{1.464503in}}%
\pgfpathlineto{\pgfqpoint{1.629562in}{1.536497in}}%
\pgfpathlineto{\pgfqpoint{1.855613in}{1.581354in}}%
\pgfpathlineto{\pgfqpoint{2.033126in}{1.601944in}}%
\pgfpathlineto{\pgfqpoint{2.033126in}{1.601944in}}%
\pgfusepath{stroke}%
\end{pgfscope}%
\begin{pgfscope}%
\pgfsetrectcap%
\pgfsetmiterjoin%
\pgfsetlinewidth{0.803000pt}%
\definecolor{currentstroke}{rgb}{0.000000,0.000000,0.000000}%
\pgfsetstrokecolor{currentstroke}%
\pgfsetdash{}{0pt}%
\pgfpathmoveto{\pgfqpoint{0.553581in}{0.499444in}}%
\pgfpathlineto{\pgfqpoint{0.553581in}{1.654444in}}%
\pgfusepath{stroke}%
\end{pgfscope}%
\begin{pgfscope}%
\pgfsetrectcap%
\pgfsetmiterjoin%
\pgfsetlinewidth{0.803000pt}%
\definecolor{currentstroke}{rgb}{0.000000,0.000000,0.000000}%
\pgfsetstrokecolor{currentstroke}%
\pgfsetdash{}{0pt}%
\pgfpathmoveto{\pgfqpoint{2.103581in}{0.499444in}}%
\pgfpathlineto{\pgfqpoint{2.103581in}{1.654444in}}%
\pgfusepath{stroke}%
\end{pgfscope}%
\begin{pgfscope}%
\pgfsetrectcap%
\pgfsetmiterjoin%
\pgfsetlinewidth{0.803000pt}%
\definecolor{currentstroke}{rgb}{0.000000,0.000000,0.000000}%
\pgfsetstrokecolor{currentstroke}%
\pgfsetdash{}{0pt}%
\pgfpathmoveto{\pgfqpoint{0.553581in}{0.499444in}}%
\pgfpathlineto{\pgfqpoint{2.103581in}{0.499444in}}%
\pgfusepath{stroke}%
\end{pgfscope}%
\begin{pgfscope}%
\pgfsetrectcap%
\pgfsetmiterjoin%
\pgfsetlinewidth{0.803000pt}%
\definecolor{currentstroke}{rgb}{0.000000,0.000000,0.000000}%
\pgfsetstrokecolor{currentstroke}%
\pgfsetdash{}{0pt}%
\pgfpathmoveto{\pgfqpoint{0.553581in}{1.654444in}}%
\pgfpathlineto{\pgfqpoint{2.103581in}{1.654444in}}%
\pgfusepath{stroke}%
\end{pgfscope}%
\begin{pgfscope}%
\pgfsetbuttcap%
\pgfsetmiterjoin%
\definecolor{currentfill}{rgb}{1.000000,1.000000,1.000000}%
\pgfsetfillcolor{currentfill}%
\pgfsetfillopacity{0.800000}%
\pgfsetlinewidth{1.003750pt}%
\definecolor{currentstroke}{rgb}{0.800000,0.800000,0.800000}%
\pgfsetstrokecolor{currentstroke}%
\pgfsetstrokeopacity{0.800000}%
\pgfsetdash{}{0pt}%
\pgfpathmoveto{\pgfqpoint{0.832747in}{0.568889in}}%
\pgfpathlineto{\pgfqpoint{2.006358in}{0.568889in}}%
\pgfpathquadraticcurveto{\pgfqpoint{2.034136in}{0.568889in}}{\pgfqpoint{2.034136in}{0.596666in}}%
\pgfpathlineto{\pgfqpoint{2.034136in}{0.776388in}}%
\pgfpathquadraticcurveto{\pgfqpoint{2.034136in}{0.804166in}}{\pgfqpoint{2.006358in}{0.804166in}}%
\pgfpathlineto{\pgfqpoint{0.832747in}{0.804166in}}%
\pgfpathquadraticcurveto{\pgfqpoint{0.804970in}{0.804166in}}{\pgfqpoint{0.804970in}{0.776388in}}%
\pgfpathlineto{\pgfqpoint{0.804970in}{0.596666in}}%
\pgfpathquadraticcurveto{\pgfqpoint{0.804970in}{0.568889in}}{\pgfqpoint{0.832747in}{0.568889in}}%
\pgfpathlineto{\pgfqpoint{0.832747in}{0.568889in}}%
\pgfpathclose%
\pgfusepath{stroke,fill}%
\end{pgfscope}%
\begin{pgfscope}%
\pgfsetrectcap%
\pgfsetroundjoin%
\pgfsetlinewidth{1.505625pt}%
\definecolor{currentstroke}{rgb}{0.000000,0.000000,0.000000}%
\pgfsetstrokecolor{currentstroke}%
\pgfsetdash{}{0pt}%
\pgfpathmoveto{\pgfqpoint{0.860525in}{0.700000in}}%
\pgfpathlineto{\pgfqpoint{0.999414in}{0.700000in}}%
\pgfpathlineto{\pgfqpoint{1.138303in}{0.700000in}}%
\pgfusepath{stroke}%
\end{pgfscope}%
\begin{pgfscope}%
\definecolor{textcolor}{rgb}{0.000000,0.000000,0.000000}%
\pgfsetstrokecolor{textcolor}%
\pgfsetfillcolor{textcolor}%
\pgftext[x=1.249414in,y=0.651388in,left,base]{\color{textcolor}\rmfamily\fontsize{10.000000}{12.000000}\selectfont AUC=0.764}%
\end{pgfscope}%
\end{pgfpicture}%
\makeatother%
\endgroup%

\end{tabular}

\

\

Other stuff

\


%%%
\begin{comment}
If we set the discrimination threshold about $0.7$, the model would classify almost all of the samples, both positive and negative class, correctly, with about the same number of false positives (sending an ambulance when one is not needed, negative class samples with $p > 0.7$) and false negatives (not sending an ambulance when one is needed, positive class samples with $p < 0.7$).  If we (as a society) were willing to tolerate more false positives, we could set the discrimination threshold lower, and if budgets were tighter we could increase the $p$ threshold.  

The table below gives the number of true negatives (TN), false positives (FP), false negatives (FN), and true positives (TP) for the 499,496 samples in the test set, along with the precision and recall values, for different discrimination thresholds $p$.  The precision is the proportion of ambulances we sent that were needed, and the recall is the proportion of ambulances needed that we sent.  

$$\text{Precision} = \frac{TP}{FP+TP}, \qquad \text{Recall} = \frac{TP}{FN + TP}$$

\begin{center}
\begin{tabular}{rrrrrrrrrrrrrr}
\toprule
$p$ &   TN &       FP &      FN &      TP &  Precision &   Recall \\
\midrule
0.50 &  346,776 &   73,794 &       1 &  78,925 &  0.52 &  1.00       \\
0.60 &  390,335 &   30,235 &      89 &  78,837 &  0.72 &  1.00  \\
0.70 & 411,040 &    9,530 &   2,838 &  76,088 &  0.89 &  0.96 \\
0.80 & 418,739 &    1,831 &  19,174 &  59,752 &  0.97 &  0.76  \\
0.90 & 420,496 &       74 &  53,736 &  25,190 &  1.00 &  0.32 & \\
\bottomrule
\end{tabular}
\end{center}

\end{comment}
%%%



