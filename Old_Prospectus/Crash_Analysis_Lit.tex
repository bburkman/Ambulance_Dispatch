%%%%%
\section{Journals}

\noindent\begin{tabular}{@{}p{4in}p{1in}p{1in}}
	Journal & CiteScore & Impact Factor \cr\hline
	\it Accident Analysis and Prevention & 7.8 & 4.993 \cr
\end{tabular}

\begin{quote}
Accident Analysis \& Prevention provides wide coverage of the general areas relating to accidental injury and damage, including the pre-injury and immediate post-injury phases. Published papers deal with medical, legal, economic, educational, behavioral, theoretical or empirical aspects of transportation accidents, as well as with accidents at other sites. Selected topics within the scope of the Journal may include: studies of human, environmental and vehicular factors influencing the occurrence, type and severity of accidents and injury; the design, implementation and evaluation of countermeasures; biomechanics of impact and human tolerance limits to injury; modelling and statistical analysis of accident data; policy, planning and decision-making in safety. 
\end{quote}

\noindent\begin{tabular}{@{}p{4in}p{1in}p{1in}}
	\it American Journal of Emergency Medicine & 3.2 & 2.469 \cr
\end{tabular}

	\begin{quote}
	A distinctive blend of practicality and scholarliness makes the American Journal of Emergency Medicine a key source for information on emergency medical care. Covering all activities concerned with emergency medicine, it is the journal to turn to for information to help increase the ability to understand, recognize and treat emergency conditions. Issues contain clinical articles, case reports, review articles, editorials, international notes, book reviews and more. The American Journal of Emergency Medicine is recommended for initial purchase in the Brandon-Hill study, Selected List of Books and Journals for the Small Medical Library (2001 Edition).
	\end{quote}

\noindent\begin{tabular}{@{}p{4in}p{1in}p{1in}}
	\it Decision Support Systems & 10.5 & 5.795 \cr
\end{tabular}

	\begin{quote}	
	The common thread of articles published in Decision Support Systems is their relevance to theoretical and technical issues in the support of enhanced decision making. The areas addressed may include foundations, functionality, interfaces, implementation, impacts, and evaluation of decision support systems (DSSs). Manuscripts may draw from diverse methods and methodologies, including those from decision theory, economics, econometrics, statistics, computer supported cooperative work, data base management, linguistics, management science, mathematical modeling, operations management, cognitive science, psychology, user interface management, and others. However, a manuscript focused on direct contributions to any of these related areas should be submitted to an outlet appropriate to the specific area.

Examples of research topics that would be appropriate for Decision Support Systems include the following:

1. DSS Foundations e.g. principles, concepts, and theories of enhanced decision making; formal languages and research methods enabling improvements in decision making. It is important that theory validation be carefully addressed.

2. DSS Functionality e.g. methods, tools, and techniques for developing thefunctional aspects of enhanced decision making; solver, model, and/or data management in DSSs; rule formulation and management in DSSs; DSS development and use in computer supported cooperative work, negotiation, research and product.

3. DSS Interfaces e.g. methods, tools, and techniques for designing and developing DSS interfaces; development, management, and presentation of knowledge in a DSS; coordination of a DSS's interface with its functionality.

4. DSS Implementation - experiences in DSS development and utilization; DSS management and updating; DSS instruction/training. A critical consideration must be how specific experiences provide more general implications.

5. DSS Evaluation and Impact e.g. evaluation metrics and processes; DSS impact on decision makers, organizational processes and performance.
	\end{quote}

\noindent\begin{tabular}{@{}p{4in}p{1in}p{1in}}
	\it Journal of Safety Research & 5.0 & 3.487 \cr
\end{tabular}

	\begin{quote}	
The Journal of Safety Research is a multidisciplinary publication that provides for the exchange of scientific evidence in all areas of safety and health, including traffic, workplace, home, and community. While this research forum invites submissions using rigorous methodologies in all related areas, it focuses on basic and applied research in unintentional injury and illness prevention. Affiliated with the National Safety Council, it seeks to engage the global scientific community including academic researchers, engineers, government agencies, policy makers, corporate decision makers, safety professionals and practitioners, psychologists, social scientists, and public health professionals.
	\end{quote}

\noindent\begin{tabular}{@{}p{4in}p{1in}p{1in}}
	\it Transportation Research Part C:  Emerging Technologies & 14.0 & 8.089 \cr
\end{tabular}

	\begin{quote}	
The focus of Transportation Research: Part C (TR\_C) is high-quality, scholarly research that addresses development, applications, and implications, in the field of transportation systems and emerging technologies . The interest is not in the individual technologies per se, but in their ultimate implications for the planning, design, operation, control, maintenance and rehabilitation of transportation systems, services and components. In other words, the intellectual core of the journal is on the transportation side, not on the technology side. The integration of quantitative methods from fields such as operations research, control systems, complex networks, computer science, artificial intelligence are encouraged.

Of particular interest are the impacts of emerging technologies on transportation system performance, in terms of monitoring, efficiency, safety, reliability, resource consumption and the environment. Submissions in the following areas of transportation are welcome: multimodal and intermodal transportation; on-demand transport; intelligent transportation systems; traffic and demand management; real-time operations; connected and autonomous vehicles; logistics; railways; resource and infrastructure management; aviation; pedestrians and soft modes.

Special emphasis is given in open science initiatives and promoting the opening of large-scale datasets for papers published in TR\_C that can support transferability and benchmarking of different approaches. The realization of data opportunities that arise from emerging technologies and new sensors in transportation can revolutionize how this data reshape our understanding of congestion mechanisms and can contribute in efficient and sustainable mobility management.	
	\end{quote}


%%%%%
\section{Articles using Similar Datasets}

\begin{itemize}
	\item Rahim 2021 \cite{RAHIM2021106090} LSU faculty, similar dataset to what we have.  
	\item Jiang 2020 \cite{JIANG2020105520} used similar data and addressed the challenges we'll have with it.  
	
\end{itemize}

%%%%%
\section{Articles on Imbalanced Crash Data}

\begin{itemize}
	\item  Schlogl 2020 \cite{SCHLOGL2020105398} uses imbalanced data.
\end{itemize}

%%%%%
\section{Ambulances}

%%%
\subsection{Ambulance Response Time}

From 11/29/21 Report.

\begin{itemize}
	\item Found standard for emergency medical service (EMS) response time, from The National Fire Protection Association.
	``1710 NFPA Standard for the Organization and Deployment of Fire Suppression Operations, Emergency Medical Operations,  and Special Operations to the Public by Career Fire Departments, 2020''
	  \S 4.1.2.1
	\begin{itemize}
		\item 60-second turnout time 
		\item 240  seconds  or  less  travel  time  for  the  arrival  of  a  unit with  first  responder  with  automatic  external  defibrillator (AED) or higher-level capability at an emergency medical incident
		\item 480 seconds or less travel time for the arrival of an advanced  life  support  (ALS)  unit  at  an  emergency  medical  incident,  where  this  service  is  provided  by  the  fire department  provided  a  first  responder  with  an  AED  or basic  life  support  (BLS)  unit  arrived  in  240  seconds  or less travel time.
		\item Lots of papers, like Liu (2016)\cite{Liu_2016} cite Rafael Sa'adah (2004), which I think is a response to the NFPA standards, but I can't find it online or in the library database.
	\end{itemize}	
\end{itemize}

%%%%%
\section{iPhone to Automatically Detect Crash and Call Emergency Services}

From 11/29/21 Report.

\begin{itemize}
	\item iPhones and Apple Watches will soon automatically call police when the accelerometer detects a car crash.
	\item Several articles dated 11/1/21, including in the Wall Street Journal.  
	\item Available in 2022
	\item What data would that provide, and what data would the police already have to complement it?  These are just my guesses.
	\item Data from Apple
	\begin{itemize}
		\item Registered owner of the phone (or phones) in the car
		\item Typical users of that phone (Apple knows!)
		\item GPS location
		\item Perhaps a rough idea of how fast the car was going and how suddenly it stopped
		\item If more than one phone sends signal, do these people know each other, or are they likely in different vehicles?
		\item Accelerometer signature of a pedestrian or bicyclist getting hit?
	\end{itemize}
	\item Complementary data from police database
	\begin{itemize}
		\item Type of roadway and speed limit
		\item Was it at an intersection?
		\item Time of day, day of week
		\item Type of vehicle registered to that person
		\item Driving record of user of phone (History of DUI?)
		\item Weather
	\end{itemize}
\end{itemize}

%%%%%
\section{Weather}

From 11/29/21 Report.

\begin{itemize}
	\item Wang et al \cite{WANG2020102619} studied the data of a ride-hailing company, DiDi Chuxing, and looked for how resilient the system was during ``extreme weather events.''  
	\begin{itemize}
	\item They defined such weather to be ``hurricane, flooding, and rainstorm.'' (page 2)   I suspect that ``rainstorm,'' which is really vague in English, is a poor translation of a more specific Chinese word.  
	\item Because these extreme events are rare, they have a sample imbalance problem (page 13).  They solve the problem in an interesting way, by ignoring it and watering down their data set.  ``The characteristics of urban transportation resilience under catastrophic events have generally similar patterns to those under general precipitation events.  Thus we incorporated the rainstorm and usual prediction events data into [sic] data set to strengthen the model training.''  So, as I understand it, they had an imbalanced data problem modeling extreme weather, so they just modeled ordinary weather.  
	\item I like how the authors started their methodology section with a page of definitions.  
	\end{itemize}
	
\end{itemize}


%%%%%
\section{Lagniappe}

\begin{itemize}
	\item Osman 2019 (LSU) \cite{OSMAN2019274} looked much more deeply at the data than other studies, looking for correlations between sets of variables.  
	\item Ziakopoulos 2020 \cite{ZIAKOPOULOS2020105323} is a good overview of the field and its jargon.  
	
	\item Guimmarra 2020 \cite{GIUMMARRA2020105333} is interesting for its text mining of crash reports.  

	\item Park 2019 \cite{Park_2019} has a full-page table categorizing studies of ambulance location, relocation, and dispatching using different optimization methods.  

\end{itemize}


%%%%%
\section{Significant Authors}

From 11/15/21 Notes:

Reviewed all of the 66 articles from 2021 with the word ``crash''  in {\it Transportation Research Part C:  Emerging Technologies}. 

\begin{itemize}
	\item Most of the articles are about autonomous vehicles.
	\item Mohammed Abdel-Aty at the U of Central Florida is a major author in this journal, but not in this year.  In previous years, if there was an article from UCF, his name was on it.  His website does not say that he has retired.  
	\item When I write, I want to include more examples than many authors give.  
\end{itemize}

%%%%%
\section{TR\_C Articles on Machine Learning}

\subsection{Application of articles whose keywords contain {\it machine learning}, {\it deep learning}, or {\it reinforcement learing}}

\begin{itemize}
	\item Autonomous Vehicles
	\begin{itemize}

	\item Control of Autonomous Vehicles 
		\cite{AN2020102777},
		\cite{BAUTISTAMONTESANO2022103662},
		\cite{CAO2022103656},
		\cite{DONG2021103192},
		\cite{DU2022103489}, 
		\cite{GUO2021102980},
		\cite{KALATIAN2021102962},
		\cite{LAZAR2021103258},
		\cite{LI2022103452},
		\cite{SHI2021103421}, 
		\cite{WANG2022103478},
		\cite{WEGENER2021102967},
		\cite{WU2020102649},
		\cite{YE2019155}, 
		\cite{ZHANG2021103140}, 
		\cite{ZHU2020102662},
	\item Preferences for Autonomous Vehicles
		\cite{ZHANG2020102774}
	\end{itemize}
		
% Lagniappe
	\item Lagniappe
	\begin{itemize}		
	\item Anomalous Event Prediction
		\cite{YANG2021102862},
	\item Origin/Destination
		\cite{MA2020102747},
		\cite{SUN2021103114},
		
	\item Variable Speed Limits 
		\cite{WU2020102649}
		
	\item Dynamic Pricing 
		\cite{GENSER2022103485},
		\cite{HAN2022103584},
		\cite{PANDEY2020102715}
	\item Parking
		\cite{MANTOUKA2021103173},
		\cite{YANG2019248},
		\cite{ZHANG2022103624}
	\item Traffic Signal Optimization 
		\cite{LEE2019117},
		\cite{LI2021103059}, 
		\cite{WANG2021103046},
		\cite{WANG2022103670},
		\cite{WU2019246},
		\cite{YOON2021103321},
	\item Perimeter Metering (?) 
		\cite{ZHOU2021102949}
	\item Energy Consumption
		\cite{QI201967},
		\cite{YAO2019276}
	\item Vehicle Idenfification
		\cite{DABIRI2020102644},
		\cite{LI2021102946},
	\item Trip Purpose
		\cite{FAROQI2021103131}
	\end{itemize}
	
% Traffic		
	\item Traffic
	\begin{itemize}
	\item Traffic Prediction
		\cite{BACHIR2019254},
		\cite{BOGAERTS202062},
		\cite{CUI2020102620},
		\cite{DAI2019142},
		\cite{DO201912},
		\cite{DRCHAL2019370},
		\cite{KUMAR2021103432},
		\cite{LI2021102977},
		\cite{LI2021103185},
		\cite{LI2021103389},
		\cite{MA2020352},
		\cite{ROY2021103339},
		\cite{WANG2020102763},
		\cite{WONG2020247},
		\cite{YANG2021103228},
		\cite{ZHANG2021103372},
		\cite{ZHANG2022103659},
	\item Traffic Speed Prediction 
		\cite{NAIR2020269},
		\cite{REMPE2022103448},
		\cite{WANG2019372},
		\cite{ZHANG2019297}
	\item Traffic in Extreme Weather
		\cite{WANG2020102619}
	\item Traffic Signals
		\cite{GUO2021103416},
		\cite{MAHMOUD2021102930},
		\cite{ZHAO2022103522}
	\item Dynamic Traffic Control  
		\cite{SHOU2022103560}
	\end{itemize}

% Individual Driver
	\item Individual Driver
	\begin{itemize}
	\item Vehicle Behavior Modeling
		\cite{CHEN2020102646},
		\cite{MA2020102785},
		\cite{MO2021103240}, 
		\cite{RAHMAN2021103310},
		\cite{YAO2021103415},
		\cite{ZHANG2019287}
	\item Classifying Driving Styles 
		\cite{MOHAMMADNAZAR2021102917}
	\item Driver's Visual Environment
		\cite{CAI2022103541},
		\cite{LI2019132},
		\cite{MA2019317}
	\item Driver Behavior
		\cite{MOHAMMADNAZAR2021102917},
		\cite{XING2021103288},
	\item Driver Distraction 
		\cite{CAI2022103541}
	This article is interesting, perhaps relevant to me, for correlating crashes with something else.

	\end{itemize}
	
% Delivery
	\item Delivery
	\begin{itemize}
	\item Delivery Times
		\cite{HUGHES2019289},
	\item Vehicle Routing Problem
		\cite{XU2022103620},
		\cite{ZHANG2020102861}
	\item Fleet Management
		\cite{TURAN2020102829}
	
	\item Transportation Systems [Survey article] 
		\cite{WANG2019144}
	\end{itemize}
	
% Public Transit
	\item {Public Transit}
	\begin{itemize}
	\item Taxis
		\cite{CHEN2021103272},
		\cite{JIAO2021103289},
		\cite{KE2021102858},
		\cite{MAO2020102626}, 
		\cite{QIN2021103239},
		\cite{SHOU2020102738}, 
		\cite{TANG2020102844},
		\cite{YU2022103640},
	\item Public Transit
		\cite{CHOW2021103264},
		\cite{FENG2022103611},
		\cite{LIU201918},
		\cite{MULLERHANNEMANN2022103566},
		\cite{TANG2021102951},
		\cite{WANG2019387},
		\cite{WANG2020102661},
		\cite{ZHANG2021102928}
	\end{itemize}

% Pedestrians and Passengers
	\item Pedestrians and Passengers
	\begin{itemize}
	\item Pedestrians
		\cite{BUSTOS2021103018},
		\cite{HIDAKA2019115}
	\item Bicycles and Scooters
		\cite{ITO2021103371},
		\cite{LV2021103404},
		\cite{ZUNIGAGARCIA2022103660},
	\item Travel Demand Modeling
		\cite{HAFEZI2021102972},
		\cite{KIM2022103523},
		\cite{LI2021102921},
		\cite{LIU2021103361},
		\cite{PANG2020102706}
	\end{itemize}

% Planes and Trains
	\item {Planes, Trains, and Boats}
	\begin{itemize}
	\item Railway Maintenance
		\cite{ALLAHBUKHSH201935}
	\item Railway Traffic Control
		\cite{GHASEMPOUR202091},
		\cite{TANG2022103679}
	\item Train Delays
		\cite{LI2022103606},
		\cite{NAIR2019196}
	\item Air Traffic Management 
		\cite{BAO2021103323},
		\cite{ALBABA2021103417},
		\cite{CORRADO2021103331},
		\cite{DESHMUKH2021103036},
		\cite{DHIEF2022103704},
		\cite{DU2021103122},
		\cite{KHAN2021103225},
		\cite{OLIVE2020102737},
		\cite{PANG2021103326},
		\cite{PHAM2022103463},
		\cite{SCHULTZ2021103119},
		\cite{VERDONKGALLEGO2019356},
		\cite{WANG2021102892},
		\cite{ZHU2021103179},
	\item Ships
		\cite{GUMUSKAYA2021103383},
		\cite{LU2021102970},
	\end{itemize}
	
% Crashes
	\item Crashes
	\begin{itemize}
	\item Inferring Pre-Crash Impact Data
		\cite{CHEN2021103009},
	\end{itemize}
	
% Back End
	\item Back End (No Application)
	\begin{itemize}
	\item Generative Modeling
		\cite{BORYSOV201973},
		\cite{GARRIDO2020102787}
	\item Preference Learning
		\cite{ZHU2020102849}
	\item Extracting Economic Information (?)
		\cite{WANG2020102701}
	\item Graphs
		\cite{RODRIGUEZDENIZ2022103556}
	\item Discrete Choice Models
		\cite{SFEIR2022103552}
	\item Fairness in Artificial Intelligence
		\cite{ZHENG2021103410}
	\item Discrete Choice Modeling
		\cite{WONG2021103050}
	\end{itemize}

\end{itemize}
 
\subsection{Articles whose abstracts refer to imbalanced data}

\vskip 0pt

Chen \cite{CHEN2020102646} talks about resampling using SMOTE and Tomek.  Used LightGBM classifier.  

Cai \cite{CAI2020102697} used the deep convolutional generative adversarial network (DCGAN).  Compared four models, logistic regression model, support vector machine, artificial neural network, and convolutional neural network.  

Emarani Abou Elassad \cite{ELAMRANIABOUELASSAD2020102708} works with several imbalanced methods.  Use this paper as a model.  

Yu \cite{YU2020102740} used focal loss for real-time crash prediction.  

Shi \cite{SHI2021103414} uses the Grey Wolf Optimizer and SMOTE to balance the data.  

Khan \cite{KHAN2021103225} used SMOTE and ``average balanced recall accuracies,''

Chen \cite{CHEN2022103709} uses bagging.

Anomalous events might also use imbalanced data.  

\subsection{Crashes}

Twenty-one articles in TR\_C have 'crash', 'accident', 'ambulance', 'hospital', 'fatal', or 'injury' in the keywords.  Another forty have them in the abstract.  I'm really only interested in ones that use real data, not simulation.  

Kalatian \cite{KALATIAN2021102962} studies interactions between pedestrians and autonomous vehicles.  

Cai and Abdel-Aty \cite{CAI2020102697} do similar work to ours with machine learning.

Emarani Abou Elassad \cite{ELAMRANIABOUELASSAD2020102708} was mentioned above as a model paper.  Also applied to crashes.  

Yu \cite{YU2020102740} mentioned above.  








