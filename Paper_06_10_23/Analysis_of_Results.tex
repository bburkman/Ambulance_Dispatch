% Analysis of Results

Our ML algorithms assign to each sample (feature vector, crash person) a probability $p \in [0,1]$ that the person needs an ambulance.  The histogram below left shows the percentage of the dataset in each range of $p$, showing the percentages for the negative class (``Does not need an ambulance'') and the positive class (``Needs an ambulance'').  On the right, the Receiver Operating Characteristic (ROC) curve, and particularly the area under the curve (AUC), is a metric for how well the model separates the two classes, with $AUC=1.0$ being perfect and $AUC=0.5$ (the dashed line) being just random assignment with no insight.  

We would love to have results like in the graphs below, where the machine learning (ML) algorithm nearly perfectly separates the two classes.  There is some overlap between $p=0.6$ and $p=0.8$ with some samples the algorithm misclassifies, but the model clearly separates most samples.  Having an AUC of 0.996 would be amazing.  

\noindent\begin{tabular}{@{\hspace{-6pt}}p{4.3in} @{\hspace{-6pt}}p{2.0in}}
	\vskip 0pt
	\hfil Raw Model Output
	
	%% Creator: Matplotlib, PGF backend
%%
%% To include the figure in your LaTeX document, write
%%   \input{<filename>.pgf}
%%
%% Make sure the required packages are loaded in your preamble
%%   \usepackage{pgf}
%%
%% Also ensure that all the required font packages are loaded; for instance,
%% the lmodern package is sometimes necessary when using math font.
%%   \usepackage{lmodern}
%%
%% Figures using additional raster images can only be included by \input if
%% they are in the same directory as the main LaTeX file. For loading figures
%% from other directories you can use the `import` package
%%   \usepackage{import}
%%
%% and then include the figures with
%%   \import{<path to file>}{<filename>.pgf}
%%
%% Matplotlib used the following preamble
%%   
%%   \usepackage{fontspec}
%%   \makeatletter\@ifpackageloaded{underscore}{}{\usepackage[strings]{underscore}}\makeatother
%%
\begingroup%
\makeatletter%
\begin{pgfpicture}%
\pgfpathrectangle{\pgfpointorigin}{\pgfqpoint{4.509306in}{1.754444in}}%
\pgfusepath{use as bounding box, clip}%
\begin{pgfscope}%
\pgfsetbuttcap%
\pgfsetmiterjoin%
\definecolor{currentfill}{rgb}{1.000000,1.000000,1.000000}%
\pgfsetfillcolor{currentfill}%
\pgfsetlinewidth{0.000000pt}%
\definecolor{currentstroke}{rgb}{1.000000,1.000000,1.000000}%
\pgfsetstrokecolor{currentstroke}%
\pgfsetdash{}{0pt}%
\pgfpathmoveto{\pgfqpoint{0.000000in}{0.000000in}}%
\pgfpathlineto{\pgfqpoint{4.509306in}{0.000000in}}%
\pgfpathlineto{\pgfqpoint{4.509306in}{1.754444in}}%
\pgfpathlineto{\pgfqpoint{0.000000in}{1.754444in}}%
\pgfpathlineto{\pgfqpoint{0.000000in}{0.000000in}}%
\pgfpathclose%
\pgfusepath{fill}%
\end{pgfscope}%
\begin{pgfscope}%
\pgfsetbuttcap%
\pgfsetmiterjoin%
\definecolor{currentfill}{rgb}{1.000000,1.000000,1.000000}%
\pgfsetfillcolor{currentfill}%
\pgfsetlinewidth{0.000000pt}%
\definecolor{currentstroke}{rgb}{0.000000,0.000000,0.000000}%
\pgfsetstrokecolor{currentstroke}%
\pgfsetstrokeopacity{0.000000}%
\pgfsetdash{}{0pt}%
\pgfpathmoveto{\pgfqpoint{0.445556in}{0.499444in}}%
\pgfpathlineto{\pgfqpoint{4.320556in}{0.499444in}}%
\pgfpathlineto{\pgfqpoint{4.320556in}{1.654444in}}%
\pgfpathlineto{\pgfqpoint{0.445556in}{1.654444in}}%
\pgfpathlineto{\pgfqpoint{0.445556in}{0.499444in}}%
\pgfpathclose%
\pgfusepath{fill}%
\end{pgfscope}%
\begin{pgfscope}%
\pgfpathrectangle{\pgfqpoint{0.445556in}{0.499444in}}{\pgfqpoint{3.875000in}{1.155000in}}%
\pgfusepath{clip}%
\pgfsetbuttcap%
\pgfsetmiterjoin%
\pgfsetlinewidth{1.003750pt}%
\definecolor{currentstroke}{rgb}{0.000000,0.000000,0.000000}%
\pgfsetstrokecolor{currentstroke}%
\pgfsetdash{}{0pt}%
\pgfpathmoveto{\pgfqpoint{0.435556in}{0.499444in}}%
\pgfpathlineto{\pgfqpoint{0.483922in}{0.499444in}}%
\pgfpathlineto{\pgfqpoint{0.483922in}{0.632510in}}%
\pgfpathlineto{\pgfqpoint{0.435556in}{0.632510in}}%
\pgfusepath{stroke}%
\end{pgfscope}%
\begin{pgfscope}%
\pgfpathrectangle{\pgfqpoint{0.445556in}{0.499444in}}{\pgfqpoint{3.875000in}{1.155000in}}%
\pgfusepath{clip}%
\pgfsetbuttcap%
\pgfsetmiterjoin%
\pgfsetlinewidth{1.003750pt}%
\definecolor{currentstroke}{rgb}{0.000000,0.000000,0.000000}%
\pgfsetstrokecolor{currentstroke}%
\pgfsetdash{}{0pt}%
\pgfpathmoveto{\pgfqpoint{0.576001in}{0.499444in}}%
\pgfpathlineto{\pgfqpoint{0.637387in}{0.499444in}}%
\pgfpathlineto{\pgfqpoint{0.637387in}{0.855505in}}%
\pgfpathlineto{\pgfqpoint{0.576001in}{0.855505in}}%
\pgfpathlineto{\pgfqpoint{0.576001in}{0.499444in}}%
\pgfpathclose%
\pgfusepath{stroke}%
\end{pgfscope}%
\begin{pgfscope}%
\pgfpathrectangle{\pgfqpoint{0.445556in}{0.499444in}}{\pgfqpoint{3.875000in}{1.155000in}}%
\pgfusepath{clip}%
\pgfsetbuttcap%
\pgfsetmiterjoin%
\pgfsetlinewidth{1.003750pt}%
\definecolor{currentstroke}{rgb}{0.000000,0.000000,0.000000}%
\pgfsetstrokecolor{currentstroke}%
\pgfsetdash{}{0pt}%
\pgfpathmoveto{\pgfqpoint{0.729467in}{0.499444in}}%
\pgfpathlineto{\pgfqpoint{0.790853in}{0.499444in}}%
\pgfpathlineto{\pgfqpoint{0.790853in}{1.087595in}}%
\pgfpathlineto{\pgfqpoint{0.729467in}{1.087595in}}%
\pgfpathlineto{\pgfqpoint{0.729467in}{0.499444in}}%
\pgfpathclose%
\pgfusepath{stroke}%
\end{pgfscope}%
\begin{pgfscope}%
\pgfpathrectangle{\pgfqpoint{0.445556in}{0.499444in}}{\pgfqpoint{3.875000in}{1.155000in}}%
\pgfusepath{clip}%
\pgfsetbuttcap%
\pgfsetmiterjoin%
\pgfsetlinewidth{1.003750pt}%
\definecolor{currentstroke}{rgb}{0.000000,0.000000,0.000000}%
\pgfsetstrokecolor{currentstroke}%
\pgfsetdash{}{0pt}%
\pgfpathmoveto{\pgfqpoint{0.882932in}{0.499444in}}%
\pgfpathlineto{\pgfqpoint{0.944318in}{0.499444in}}%
\pgfpathlineto{\pgfqpoint{0.944318in}{1.269324in}}%
\pgfpathlineto{\pgfqpoint{0.882932in}{1.269324in}}%
\pgfpathlineto{\pgfqpoint{0.882932in}{0.499444in}}%
\pgfpathclose%
\pgfusepath{stroke}%
\end{pgfscope}%
\begin{pgfscope}%
\pgfpathrectangle{\pgfqpoint{0.445556in}{0.499444in}}{\pgfqpoint{3.875000in}{1.155000in}}%
\pgfusepath{clip}%
\pgfsetbuttcap%
\pgfsetmiterjoin%
\pgfsetlinewidth{1.003750pt}%
\definecolor{currentstroke}{rgb}{0.000000,0.000000,0.000000}%
\pgfsetstrokecolor{currentstroke}%
\pgfsetdash{}{0pt}%
\pgfpathmoveto{\pgfqpoint{1.036397in}{0.499444in}}%
\pgfpathlineto{\pgfqpoint{1.097783in}{0.499444in}}%
\pgfpathlineto{\pgfqpoint{1.097783in}{1.428301in}}%
\pgfpathlineto{\pgfqpoint{1.036397in}{1.428301in}}%
\pgfpathlineto{\pgfqpoint{1.036397in}{0.499444in}}%
\pgfpathclose%
\pgfusepath{stroke}%
\end{pgfscope}%
\begin{pgfscope}%
\pgfpathrectangle{\pgfqpoint{0.445556in}{0.499444in}}{\pgfqpoint{3.875000in}{1.155000in}}%
\pgfusepath{clip}%
\pgfsetbuttcap%
\pgfsetmiterjoin%
\pgfsetlinewidth{1.003750pt}%
\definecolor{currentstroke}{rgb}{0.000000,0.000000,0.000000}%
\pgfsetstrokecolor{currentstroke}%
\pgfsetdash{}{0pt}%
\pgfpathmoveto{\pgfqpoint{1.189863in}{0.499444in}}%
\pgfpathlineto{\pgfqpoint{1.251249in}{0.499444in}}%
\pgfpathlineto{\pgfqpoint{1.251249in}{1.521593in}}%
\pgfpathlineto{\pgfqpoint{1.189863in}{1.521593in}}%
\pgfpathlineto{\pgfqpoint{1.189863in}{0.499444in}}%
\pgfpathclose%
\pgfusepath{stroke}%
\end{pgfscope}%
\begin{pgfscope}%
\pgfpathrectangle{\pgfqpoint{0.445556in}{0.499444in}}{\pgfqpoint{3.875000in}{1.155000in}}%
\pgfusepath{clip}%
\pgfsetbuttcap%
\pgfsetmiterjoin%
\pgfsetlinewidth{1.003750pt}%
\definecolor{currentstroke}{rgb}{0.000000,0.000000,0.000000}%
\pgfsetstrokecolor{currentstroke}%
\pgfsetdash{}{0pt}%
\pgfpathmoveto{\pgfqpoint{1.343328in}{0.499444in}}%
\pgfpathlineto{\pgfqpoint{1.404714in}{0.499444in}}%
\pgfpathlineto{\pgfqpoint{1.404714in}{1.590875in}}%
\pgfpathlineto{\pgfqpoint{1.343328in}{1.590875in}}%
\pgfpathlineto{\pgfqpoint{1.343328in}{0.499444in}}%
\pgfpathclose%
\pgfusepath{stroke}%
\end{pgfscope}%
\begin{pgfscope}%
\pgfpathrectangle{\pgfqpoint{0.445556in}{0.499444in}}{\pgfqpoint{3.875000in}{1.155000in}}%
\pgfusepath{clip}%
\pgfsetbuttcap%
\pgfsetmiterjoin%
\pgfsetlinewidth{1.003750pt}%
\definecolor{currentstroke}{rgb}{0.000000,0.000000,0.000000}%
\pgfsetstrokecolor{currentstroke}%
\pgfsetdash{}{0pt}%
\pgfpathmoveto{\pgfqpoint{1.496793in}{0.499444in}}%
\pgfpathlineto{\pgfqpoint{1.558179in}{0.499444in}}%
\pgfpathlineto{\pgfqpoint{1.558179in}{1.599444in}}%
\pgfpathlineto{\pgfqpoint{1.496793in}{1.599444in}}%
\pgfpathlineto{\pgfqpoint{1.496793in}{0.499444in}}%
\pgfpathclose%
\pgfusepath{stroke}%
\end{pgfscope}%
\begin{pgfscope}%
\pgfpathrectangle{\pgfqpoint{0.445556in}{0.499444in}}{\pgfqpoint{3.875000in}{1.155000in}}%
\pgfusepath{clip}%
\pgfsetbuttcap%
\pgfsetmiterjoin%
\pgfsetlinewidth{1.003750pt}%
\definecolor{currentstroke}{rgb}{0.000000,0.000000,0.000000}%
\pgfsetstrokecolor{currentstroke}%
\pgfsetdash{}{0pt}%
\pgfpathmoveto{\pgfqpoint{1.650259in}{0.499444in}}%
\pgfpathlineto{\pgfqpoint{1.711645in}{0.499444in}}%
\pgfpathlineto{\pgfqpoint{1.711645in}{1.566953in}}%
\pgfpathlineto{\pgfqpoint{1.650259in}{1.566953in}}%
\pgfpathlineto{\pgfqpoint{1.650259in}{0.499444in}}%
\pgfpathclose%
\pgfusepath{stroke}%
\end{pgfscope}%
\begin{pgfscope}%
\pgfpathrectangle{\pgfqpoint{0.445556in}{0.499444in}}{\pgfqpoint{3.875000in}{1.155000in}}%
\pgfusepath{clip}%
\pgfsetbuttcap%
\pgfsetmiterjoin%
\pgfsetlinewidth{1.003750pt}%
\definecolor{currentstroke}{rgb}{0.000000,0.000000,0.000000}%
\pgfsetstrokecolor{currentstroke}%
\pgfsetdash{}{0pt}%
\pgfpathmoveto{\pgfqpoint{1.803724in}{0.499444in}}%
\pgfpathlineto{\pgfqpoint{1.865110in}{0.499444in}}%
\pgfpathlineto{\pgfqpoint{1.865110in}{1.506942in}}%
\pgfpathlineto{\pgfqpoint{1.803724in}{1.506942in}}%
\pgfpathlineto{\pgfqpoint{1.803724in}{0.499444in}}%
\pgfpathclose%
\pgfusepath{stroke}%
\end{pgfscope}%
\begin{pgfscope}%
\pgfpathrectangle{\pgfqpoint{0.445556in}{0.499444in}}{\pgfqpoint{3.875000in}{1.155000in}}%
\pgfusepath{clip}%
\pgfsetbuttcap%
\pgfsetmiterjoin%
\pgfsetlinewidth{1.003750pt}%
\definecolor{currentstroke}{rgb}{0.000000,0.000000,0.000000}%
\pgfsetstrokecolor{currentstroke}%
\pgfsetdash{}{0pt}%
\pgfpathmoveto{\pgfqpoint{1.957189in}{0.499444in}}%
\pgfpathlineto{\pgfqpoint{2.018575in}{0.499444in}}%
\pgfpathlineto{\pgfqpoint{2.018575in}{1.416545in}}%
\pgfpathlineto{\pgfqpoint{1.957189in}{1.416545in}}%
\pgfpathlineto{\pgfqpoint{1.957189in}{0.499444in}}%
\pgfpathclose%
\pgfusepath{stroke}%
\end{pgfscope}%
\begin{pgfscope}%
\pgfpathrectangle{\pgfqpoint{0.445556in}{0.499444in}}{\pgfqpoint{3.875000in}{1.155000in}}%
\pgfusepath{clip}%
\pgfsetbuttcap%
\pgfsetmiterjoin%
\pgfsetlinewidth{1.003750pt}%
\definecolor{currentstroke}{rgb}{0.000000,0.000000,0.000000}%
\pgfsetstrokecolor{currentstroke}%
\pgfsetdash{}{0pt}%
\pgfpathmoveto{\pgfqpoint{2.110655in}{0.499444in}}%
\pgfpathlineto{\pgfqpoint{2.172041in}{0.499444in}}%
\pgfpathlineto{\pgfqpoint{2.172041in}{1.298482in}}%
\pgfpathlineto{\pgfqpoint{2.110655in}{1.298482in}}%
\pgfpathlineto{\pgfqpoint{2.110655in}{0.499444in}}%
\pgfpathclose%
\pgfusepath{stroke}%
\end{pgfscope}%
\begin{pgfscope}%
\pgfpathrectangle{\pgfqpoint{0.445556in}{0.499444in}}{\pgfqpoint{3.875000in}{1.155000in}}%
\pgfusepath{clip}%
\pgfsetbuttcap%
\pgfsetmiterjoin%
\pgfsetlinewidth{1.003750pt}%
\definecolor{currentstroke}{rgb}{0.000000,0.000000,0.000000}%
\pgfsetstrokecolor{currentstroke}%
\pgfsetdash{}{0pt}%
\pgfpathmoveto{\pgfqpoint{2.264120in}{0.499444in}}%
\pgfpathlineto{\pgfqpoint{2.325506in}{0.499444in}}%
\pgfpathlineto{\pgfqpoint{2.325506in}{1.165884in}}%
\pgfpathlineto{\pgfqpoint{2.264120in}{1.165884in}}%
\pgfpathlineto{\pgfqpoint{2.264120in}{0.499444in}}%
\pgfpathclose%
\pgfusepath{stroke}%
\end{pgfscope}%
\begin{pgfscope}%
\pgfpathrectangle{\pgfqpoint{0.445556in}{0.499444in}}{\pgfqpoint{3.875000in}{1.155000in}}%
\pgfusepath{clip}%
\pgfsetbuttcap%
\pgfsetmiterjoin%
\pgfsetlinewidth{1.003750pt}%
\definecolor{currentstroke}{rgb}{0.000000,0.000000,0.000000}%
\pgfsetstrokecolor{currentstroke}%
\pgfsetdash{}{0pt}%
\pgfpathmoveto{\pgfqpoint{2.417585in}{0.499444in}}%
\pgfpathlineto{\pgfqpoint{2.478972in}{0.499444in}}%
\pgfpathlineto{\pgfqpoint{2.478972in}{1.041650in}}%
\pgfpathlineto{\pgfqpoint{2.417585in}{1.041650in}}%
\pgfpathlineto{\pgfqpoint{2.417585in}{0.499444in}}%
\pgfpathclose%
\pgfusepath{stroke}%
\end{pgfscope}%
\begin{pgfscope}%
\pgfpathrectangle{\pgfqpoint{0.445556in}{0.499444in}}{\pgfqpoint{3.875000in}{1.155000in}}%
\pgfusepath{clip}%
\pgfsetbuttcap%
\pgfsetmiterjoin%
\pgfsetlinewidth{1.003750pt}%
\definecolor{currentstroke}{rgb}{0.000000,0.000000,0.000000}%
\pgfsetstrokecolor{currentstroke}%
\pgfsetdash{}{0pt}%
\pgfpathmoveto{\pgfqpoint{2.571051in}{0.499444in}}%
\pgfpathlineto{\pgfqpoint{2.632437in}{0.499444in}}%
\pgfpathlineto{\pgfqpoint{2.632437in}{0.916218in}}%
\pgfpathlineto{\pgfqpoint{2.571051in}{0.916218in}}%
\pgfpathlineto{\pgfqpoint{2.571051in}{0.499444in}}%
\pgfpathclose%
\pgfusepath{stroke}%
\end{pgfscope}%
\begin{pgfscope}%
\pgfpathrectangle{\pgfqpoint{0.445556in}{0.499444in}}{\pgfqpoint{3.875000in}{1.155000in}}%
\pgfusepath{clip}%
\pgfsetbuttcap%
\pgfsetmiterjoin%
\pgfsetlinewidth{1.003750pt}%
\definecolor{currentstroke}{rgb}{0.000000,0.000000,0.000000}%
\pgfsetstrokecolor{currentstroke}%
\pgfsetdash{}{0pt}%
\pgfpathmoveto{\pgfqpoint{2.724516in}{0.499444in}}%
\pgfpathlineto{\pgfqpoint{2.785902in}{0.499444in}}%
\pgfpathlineto{\pgfqpoint{2.785902in}{0.806958in}}%
\pgfpathlineto{\pgfqpoint{2.724516in}{0.806958in}}%
\pgfpathlineto{\pgfqpoint{2.724516in}{0.499444in}}%
\pgfpathclose%
\pgfusepath{stroke}%
\end{pgfscope}%
\begin{pgfscope}%
\pgfpathrectangle{\pgfqpoint{0.445556in}{0.499444in}}{\pgfqpoint{3.875000in}{1.155000in}}%
\pgfusepath{clip}%
\pgfsetbuttcap%
\pgfsetmiterjoin%
\pgfsetlinewidth{1.003750pt}%
\definecolor{currentstroke}{rgb}{0.000000,0.000000,0.000000}%
\pgfsetstrokecolor{currentstroke}%
\pgfsetdash{}{0pt}%
\pgfpathmoveto{\pgfqpoint{2.877981in}{0.499444in}}%
\pgfpathlineto{\pgfqpoint{2.939368in}{0.499444in}}%
\pgfpathlineto{\pgfqpoint{2.939368in}{0.716063in}}%
\pgfpathlineto{\pgfqpoint{2.877981in}{0.716063in}}%
\pgfpathlineto{\pgfqpoint{2.877981in}{0.499444in}}%
\pgfpathclose%
\pgfusepath{stroke}%
\end{pgfscope}%
\begin{pgfscope}%
\pgfpathrectangle{\pgfqpoint{0.445556in}{0.499444in}}{\pgfqpoint{3.875000in}{1.155000in}}%
\pgfusepath{clip}%
\pgfsetbuttcap%
\pgfsetmiterjoin%
\pgfsetlinewidth{1.003750pt}%
\definecolor{currentstroke}{rgb}{0.000000,0.000000,0.000000}%
\pgfsetstrokecolor{currentstroke}%
\pgfsetdash{}{0pt}%
\pgfpathmoveto{\pgfqpoint{3.031447in}{0.499444in}}%
\pgfpathlineto{\pgfqpoint{3.092833in}{0.499444in}}%
\pgfpathlineto{\pgfqpoint{3.092833in}{0.645904in}}%
\pgfpathlineto{\pgfqpoint{3.031447in}{0.645904in}}%
\pgfpathlineto{\pgfqpoint{3.031447in}{0.499444in}}%
\pgfpathclose%
\pgfusepath{stroke}%
\end{pgfscope}%
\begin{pgfscope}%
\pgfpathrectangle{\pgfqpoint{0.445556in}{0.499444in}}{\pgfqpoint{3.875000in}{1.155000in}}%
\pgfusepath{clip}%
\pgfsetbuttcap%
\pgfsetmiterjoin%
\pgfsetlinewidth{1.003750pt}%
\definecolor{currentstroke}{rgb}{0.000000,0.000000,0.000000}%
\pgfsetstrokecolor{currentstroke}%
\pgfsetdash{}{0pt}%
\pgfpathmoveto{\pgfqpoint{3.184912in}{0.499444in}}%
\pgfpathlineto{\pgfqpoint{3.246298in}{0.499444in}}%
\pgfpathlineto{\pgfqpoint{3.246298in}{0.598936in}}%
\pgfpathlineto{\pgfqpoint{3.184912in}{0.598936in}}%
\pgfpathlineto{\pgfqpoint{3.184912in}{0.499444in}}%
\pgfpathclose%
\pgfusepath{stroke}%
\end{pgfscope}%
\begin{pgfscope}%
\pgfpathrectangle{\pgfqpoint{0.445556in}{0.499444in}}{\pgfqpoint{3.875000in}{1.155000in}}%
\pgfusepath{clip}%
\pgfsetbuttcap%
\pgfsetmiterjoin%
\pgfsetlinewidth{1.003750pt}%
\definecolor{currentstroke}{rgb}{0.000000,0.000000,0.000000}%
\pgfsetstrokecolor{currentstroke}%
\pgfsetdash{}{0pt}%
\pgfpathmoveto{\pgfqpoint{3.338377in}{0.499444in}}%
\pgfpathlineto{\pgfqpoint{3.399764in}{0.499444in}}%
\pgfpathlineto{\pgfqpoint{3.399764in}{0.560040in}}%
\pgfpathlineto{\pgfqpoint{3.338377in}{0.560040in}}%
\pgfpathlineto{\pgfqpoint{3.338377in}{0.499444in}}%
\pgfpathclose%
\pgfusepath{stroke}%
\end{pgfscope}%
\begin{pgfscope}%
\pgfpathrectangle{\pgfqpoint{0.445556in}{0.499444in}}{\pgfqpoint{3.875000in}{1.155000in}}%
\pgfusepath{clip}%
\pgfsetbuttcap%
\pgfsetmiterjoin%
\pgfsetlinewidth{1.003750pt}%
\definecolor{currentstroke}{rgb}{0.000000,0.000000,0.000000}%
\pgfsetstrokecolor{currentstroke}%
\pgfsetdash{}{0pt}%
\pgfpathmoveto{\pgfqpoint{3.491843in}{0.499444in}}%
\pgfpathlineto{\pgfqpoint{3.553229in}{0.499444in}}%
\pgfpathlineto{\pgfqpoint{3.553229in}{0.535591in}}%
\pgfpathlineto{\pgfqpoint{3.491843in}{0.535591in}}%
\pgfpathlineto{\pgfqpoint{3.491843in}{0.499444in}}%
\pgfpathclose%
\pgfusepath{stroke}%
\end{pgfscope}%
\begin{pgfscope}%
\pgfpathrectangle{\pgfqpoint{0.445556in}{0.499444in}}{\pgfqpoint{3.875000in}{1.155000in}}%
\pgfusepath{clip}%
\pgfsetbuttcap%
\pgfsetmiterjoin%
\pgfsetlinewidth{1.003750pt}%
\definecolor{currentstroke}{rgb}{0.000000,0.000000,0.000000}%
\pgfsetstrokecolor{currentstroke}%
\pgfsetdash{}{0pt}%
\pgfpathmoveto{\pgfqpoint{3.645308in}{0.499444in}}%
\pgfpathlineto{\pgfqpoint{3.706694in}{0.499444in}}%
\pgfpathlineto{\pgfqpoint{3.706694in}{0.512137in}}%
\pgfpathlineto{\pgfqpoint{3.645308in}{0.512137in}}%
\pgfpathlineto{\pgfqpoint{3.645308in}{0.499444in}}%
\pgfpathclose%
\pgfusepath{stroke}%
\end{pgfscope}%
\begin{pgfscope}%
\pgfpathrectangle{\pgfqpoint{0.445556in}{0.499444in}}{\pgfqpoint{3.875000in}{1.155000in}}%
\pgfusepath{clip}%
\pgfsetbuttcap%
\pgfsetmiterjoin%
\pgfsetlinewidth{1.003750pt}%
\definecolor{currentstroke}{rgb}{0.000000,0.000000,0.000000}%
\pgfsetstrokecolor{currentstroke}%
\pgfsetdash{}{0pt}%
\pgfpathmoveto{\pgfqpoint{3.798774in}{0.499444in}}%
\pgfpathlineto{\pgfqpoint{3.860160in}{0.499444in}}%
\pgfpathlineto{\pgfqpoint{3.860160in}{0.503305in}}%
\pgfpathlineto{\pgfqpoint{3.798774in}{0.503305in}}%
\pgfpathlineto{\pgfqpoint{3.798774in}{0.499444in}}%
\pgfpathclose%
\pgfusepath{stroke}%
\end{pgfscope}%
\begin{pgfscope}%
\pgfpathrectangle{\pgfqpoint{0.445556in}{0.499444in}}{\pgfqpoint{3.875000in}{1.155000in}}%
\pgfusepath{clip}%
\pgfsetbuttcap%
\pgfsetmiterjoin%
\pgfsetlinewidth{1.003750pt}%
\definecolor{currentstroke}{rgb}{0.000000,0.000000,0.000000}%
\pgfsetstrokecolor{currentstroke}%
\pgfsetdash{}{0pt}%
\pgfpathmoveto{\pgfqpoint{3.952239in}{0.499444in}}%
\pgfpathlineto{\pgfqpoint{4.013625in}{0.499444in}}%
\pgfpathlineto{\pgfqpoint{4.013625in}{0.500234in}}%
\pgfpathlineto{\pgfqpoint{3.952239in}{0.500234in}}%
\pgfpathlineto{\pgfqpoint{3.952239in}{0.499444in}}%
\pgfpathclose%
\pgfusepath{stroke}%
\end{pgfscope}%
\begin{pgfscope}%
\pgfpathrectangle{\pgfqpoint{0.445556in}{0.499444in}}{\pgfqpoint{3.875000in}{1.155000in}}%
\pgfusepath{clip}%
\pgfsetbuttcap%
\pgfsetmiterjoin%
\pgfsetlinewidth{1.003750pt}%
\definecolor{currentstroke}{rgb}{0.000000,0.000000,0.000000}%
\pgfsetstrokecolor{currentstroke}%
\pgfsetdash{}{0pt}%
\pgfpathmoveto{\pgfqpoint{4.105704in}{0.499444in}}%
\pgfpathlineto{\pgfqpoint{4.167090in}{0.499444in}}%
\pgfpathlineto{\pgfqpoint{4.167090in}{0.499503in}}%
\pgfpathlineto{\pgfqpoint{4.105704in}{0.499503in}}%
\pgfpathlineto{\pgfqpoint{4.105704in}{0.499444in}}%
\pgfpathclose%
\pgfusepath{stroke}%
\end{pgfscope}%
\begin{pgfscope}%
\pgfpathrectangle{\pgfqpoint{0.445556in}{0.499444in}}{\pgfqpoint{3.875000in}{1.155000in}}%
\pgfusepath{clip}%
\pgfsetbuttcap%
\pgfsetmiterjoin%
\definecolor{currentfill}{rgb}{0.000000,0.000000,0.000000}%
\pgfsetfillcolor{currentfill}%
\pgfsetlinewidth{0.000000pt}%
\definecolor{currentstroke}{rgb}{0.000000,0.000000,0.000000}%
\pgfsetstrokecolor{currentstroke}%
\pgfsetstrokeopacity{0.000000}%
\pgfsetdash{}{0pt}%
\pgfpathmoveto{\pgfqpoint{0.483922in}{0.499444in}}%
\pgfpathlineto{\pgfqpoint{0.545308in}{0.499444in}}%
\pgfpathlineto{\pgfqpoint{0.545308in}{0.499444in}}%
\pgfpathlineto{\pgfqpoint{0.483922in}{0.499444in}}%
\pgfpathlineto{\pgfqpoint{0.483922in}{0.499444in}}%
\pgfpathclose%
\pgfusepath{fill}%
\end{pgfscope}%
\begin{pgfscope}%
\pgfpathrectangle{\pgfqpoint{0.445556in}{0.499444in}}{\pgfqpoint{3.875000in}{1.155000in}}%
\pgfusepath{clip}%
\pgfsetbuttcap%
\pgfsetmiterjoin%
\definecolor{currentfill}{rgb}{0.000000,0.000000,0.000000}%
\pgfsetfillcolor{currentfill}%
\pgfsetlinewidth{0.000000pt}%
\definecolor{currentstroke}{rgb}{0.000000,0.000000,0.000000}%
\pgfsetstrokecolor{currentstroke}%
\pgfsetstrokeopacity{0.000000}%
\pgfsetdash{}{0pt}%
\pgfpathmoveto{\pgfqpoint{0.637387in}{0.499444in}}%
\pgfpathlineto{\pgfqpoint{0.698774in}{0.499444in}}%
\pgfpathlineto{\pgfqpoint{0.698774in}{0.499444in}}%
\pgfpathlineto{\pgfqpoint{0.637387in}{0.499444in}}%
\pgfpathlineto{\pgfqpoint{0.637387in}{0.499444in}}%
\pgfpathclose%
\pgfusepath{fill}%
\end{pgfscope}%
\begin{pgfscope}%
\pgfpathrectangle{\pgfqpoint{0.445556in}{0.499444in}}{\pgfqpoint{3.875000in}{1.155000in}}%
\pgfusepath{clip}%
\pgfsetbuttcap%
\pgfsetmiterjoin%
\definecolor{currentfill}{rgb}{0.000000,0.000000,0.000000}%
\pgfsetfillcolor{currentfill}%
\pgfsetlinewidth{0.000000pt}%
\definecolor{currentstroke}{rgb}{0.000000,0.000000,0.000000}%
\pgfsetstrokecolor{currentstroke}%
\pgfsetstrokeopacity{0.000000}%
\pgfsetdash{}{0pt}%
\pgfpathmoveto{\pgfqpoint{0.790853in}{0.499444in}}%
\pgfpathlineto{\pgfqpoint{0.852239in}{0.499444in}}%
\pgfpathlineto{\pgfqpoint{0.852239in}{0.499444in}}%
\pgfpathlineto{\pgfqpoint{0.790853in}{0.499444in}}%
\pgfpathlineto{\pgfqpoint{0.790853in}{0.499444in}}%
\pgfpathclose%
\pgfusepath{fill}%
\end{pgfscope}%
\begin{pgfscope}%
\pgfpathrectangle{\pgfqpoint{0.445556in}{0.499444in}}{\pgfqpoint{3.875000in}{1.155000in}}%
\pgfusepath{clip}%
\pgfsetbuttcap%
\pgfsetmiterjoin%
\definecolor{currentfill}{rgb}{0.000000,0.000000,0.000000}%
\pgfsetfillcolor{currentfill}%
\pgfsetlinewidth{0.000000pt}%
\definecolor{currentstroke}{rgb}{0.000000,0.000000,0.000000}%
\pgfsetstrokecolor{currentstroke}%
\pgfsetstrokeopacity{0.000000}%
\pgfsetdash{}{0pt}%
\pgfpathmoveto{\pgfqpoint{0.944318in}{0.499444in}}%
\pgfpathlineto{\pgfqpoint{1.005704in}{0.499444in}}%
\pgfpathlineto{\pgfqpoint{1.005704in}{0.499444in}}%
\pgfpathlineto{\pgfqpoint{0.944318in}{0.499444in}}%
\pgfpathlineto{\pgfqpoint{0.944318in}{0.499444in}}%
\pgfpathclose%
\pgfusepath{fill}%
\end{pgfscope}%
\begin{pgfscope}%
\pgfpathrectangle{\pgfqpoint{0.445556in}{0.499444in}}{\pgfqpoint{3.875000in}{1.155000in}}%
\pgfusepath{clip}%
\pgfsetbuttcap%
\pgfsetmiterjoin%
\definecolor{currentfill}{rgb}{0.000000,0.000000,0.000000}%
\pgfsetfillcolor{currentfill}%
\pgfsetlinewidth{0.000000pt}%
\definecolor{currentstroke}{rgb}{0.000000,0.000000,0.000000}%
\pgfsetstrokecolor{currentstroke}%
\pgfsetstrokeopacity{0.000000}%
\pgfsetdash{}{0pt}%
\pgfpathmoveto{\pgfqpoint{1.097783in}{0.499444in}}%
\pgfpathlineto{\pgfqpoint{1.159170in}{0.499444in}}%
\pgfpathlineto{\pgfqpoint{1.159170in}{0.499444in}}%
\pgfpathlineto{\pgfqpoint{1.097783in}{0.499444in}}%
\pgfpathlineto{\pgfqpoint{1.097783in}{0.499444in}}%
\pgfpathclose%
\pgfusepath{fill}%
\end{pgfscope}%
\begin{pgfscope}%
\pgfpathrectangle{\pgfqpoint{0.445556in}{0.499444in}}{\pgfqpoint{3.875000in}{1.155000in}}%
\pgfusepath{clip}%
\pgfsetbuttcap%
\pgfsetmiterjoin%
\definecolor{currentfill}{rgb}{0.000000,0.000000,0.000000}%
\pgfsetfillcolor{currentfill}%
\pgfsetlinewidth{0.000000pt}%
\definecolor{currentstroke}{rgb}{0.000000,0.000000,0.000000}%
\pgfsetstrokecolor{currentstroke}%
\pgfsetstrokeopacity{0.000000}%
\pgfsetdash{}{0pt}%
\pgfpathmoveto{\pgfqpoint{1.251249in}{0.499444in}}%
\pgfpathlineto{\pgfqpoint{1.312635in}{0.499444in}}%
\pgfpathlineto{\pgfqpoint{1.312635in}{0.499444in}}%
\pgfpathlineto{\pgfqpoint{1.251249in}{0.499444in}}%
\pgfpathlineto{\pgfqpoint{1.251249in}{0.499444in}}%
\pgfpathclose%
\pgfusepath{fill}%
\end{pgfscope}%
\begin{pgfscope}%
\pgfpathrectangle{\pgfqpoint{0.445556in}{0.499444in}}{\pgfqpoint{3.875000in}{1.155000in}}%
\pgfusepath{clip}%
\pgfsetbuttcap%
\pgfsetmiterjoin%
\definecolor{currentfill}{rgb}{0.000000,0.000000,0.000000}%
\pgfsetfillcolor{currentfill}%
\pgfsetlinewidth{0.000000pt}%
\definecolor{currentstroke}{rgb}{0.000000,0.000000,0.000000}%
\pgfsetstrokecolor{currentstroke}%
\pgfsetstrokeopacity{0.000000}%
\pgfsetdash{}{0pt}%
\pgfpathmoveto{\pgfqpoint{1.404714in}{0.499444in}}%
\pgfpathlineto{\pgfqpoint{1.466100in}{0.499444in}}%
\pgfpathlineto{\pgfqpoint{1.466100in}{0.499444in}}%
\pgfpathlineto{\pgfqpoint{1.404714in}{0.499444in}}%
\pgfpathlineto{\pgfqpoint{1.404714in}{0.499444in}}%
\pgfpathclose%
\pgfusepath{fill}%
\end{pgfscope}%
\begin{pgfscope}%
\pgfpathrectangle{\pgfqpoint{0.445556in}{0.499444in}}{\pgfqpoint{3.875000in}{1.155000in}}%
\pgfusepath{clip}%
\pgfsetbuttcap%
\pgfsetmiterjoin%
\definecolor{currentfill}{rgb}{0.000000,0.000000,0.000000}%
\pgfsetfillcolor{currentfill}%
\pgfsetlinewidth{0.000000pt}%
\definecolor{currentstroke}{rgb}{0.000000,0.000000,0.000000}%
\pgfsetstrokecolor{currentstroke}%
\pgfsetstrokeopacity{0.000000}%
\pgfsetdash{}{0pt}%
\pgfpathmoveto{\pgfqpoint{1.558179in}{0.499444in}}%
\pgfpathlineto{\pgfqpoint{1.619566in}{0.499444in}}%
\pgfpathlineto{\pgfqpoint{1.619566in}{0.499444in}}%
\pgfpathlineto{\pgfqpoint{1.558179in}{0.499444in}}%
\pgfpathlineto{\pgfqpoint{1.558179in}{0.499444in}}%
\pgfpathclose%
\pgfusepath{fill}%
\end{pgfscope}%
\begin{pgfscope}%
\pgfpathrectangle{\pgfqpoint{0.445556in}{0.499444in}}{\pgfqpoint{3.875000in}{1.155000in}}%
\pgfusepath{clip}%
\pgfsetbuttcap%
\pgfsetmiterjoin%
\definecolor{currentfill}{rgb}{0.000000,0.000000,0.000000}%
\pgfsetfillcolor{currentfill}%
\pgfsetlinewidth{0.000000pt}%
\definecolor{currentstroke}{rgb}{0.000000,0.000000,0.000000}%
\pgfsetstrokecolor{currentstroke}%
\pgfsetstrokeopacity{0.000000}%
\pgfsetdash{}{0pt}%
\pgfpathmoveto{\pgfqpoint{1.711645in}{0.499444in}}%
\pgfpathlineto{\pgfqpoint{1.773031in}{0.499444in}}%
\pgfpathlineto{\pgfqpoint{1.773031in}{0.499444in}}%
\pgfpathlineto{\pgfqpoint{1.711645in}{0.499444in}}%
\pgfpathlineto{\pgfqpoint{1.711645in}{0.499444in}}%
\pgfpathclose%
\pgfusepath{fill}%
\end{pgfscope}%
\begin{pgfscope}%
\pgfpathrectangle{\pgfqpoint{0.445556in}{0.499444in}}{\pgfqpoint{3.875000in}{1.155000in}}%
\pgfusepath{clip}%
\pgfsetbuttcap%
\pgfsetmiterjoin%
\definecolor{currentfill}{rgb}{0.000000,0.000000,0.000000}%
\pgfsetfillcolor{currentfill}%
\pgfsetlinewidth{0.000000pt}%
\definecolor{currentstroke}{rgb}{0.000000,0.000000,0.000000}%
\pgfsetstrokecolor{currentstroke}%
\pgfsetstrokeopacity{0.000000}%
\pgfsetdash{}{0pt}%
\pgfpathmoveto{\pgfqpoint{1.865110in}{0.499444in}}%
\pgfpathlineto{\pgfqpoint{1.926496in}{0.499444in}}%
\pgfpathlineto{\pgfqpoint{1.926496in}{0.499444in}}%
\pgfpathlineto{\pgfqpoint{1.865110in}{0.499444in}}%
\pgfpathlineto{\pgfqpoint{1.865110in}{0.499444in}}%
\pgfpathclose%
\pgfusepath{fill}%
\end{pgfscope}%
\begin{pgfscope}%
\pgfpathrectangle{\pgfqpoint{0.445556in}{0.499444in}}{\pgfqpoint{3.875000in}{1.155000in}}%
\pgfusepath{clip}%
\pgfsetbuttcap%
\pgfsetmiterjoin%
\definecolor{currentfill}{rgb}{0.000000,0.000000,0.000000}%
\pgfsetfillcolor{currentfill}%
\pgfsetlinewidth{0.000000pt}%
\definecolor{currentstroke}{rgb}{0.000000,0.000000,0.000000}%
\pgfsetstrokecolor{currentstroke}%
\pgfsetstrokeopacity{0.000000}%
\pgfsetdash{}{0pt}%
\pgfpathmoveto{\pgfqpoint{2.018575in}{0.499444in}}%
\pgfpathlineto{\pgfqpoint{2.079962in}{0.499444in}}%
\pgfpathlineto{\pgfqpoint{2.079962in}{0.499444in}}%
\pgfpathlineto{\pgfqpoint{2.018575in}{0.499444in}}%
\pgfpathlineto{\pgfqpoint{2.018575in}{0.499444in}}%
\pgfpathclose%
\pgfusepath{fill}%
\end{pgfscope}%
\begin{pgfscope}%
\pgfpathrectangle{\pgfqpoint{0.445556in}{0.499444in}}{\pgfqpoint{3.875000in}{1.155000in}}%
\pgfusepath{clip}%
\pgfsetbuttcap%
\pgfsetmiterjoin%
\definecolor{currentfill}{rgb}{0.000000,0.000000,0.000000}%
\pgfsetfillcolor{currentfill}%
\pgfsetlinewidth{0.000000pt}%
\definecolor{currentstroke}{rgb}{0.000000,0.000000,0.000000}%
\pgfsetstrokecolor{currentstroke}%
\pgfsetstrokeopacity{0.000000}%
\pgfsetdash{}{0pt}%
\pgfpathmoveto{\pgfqpoint{2.172041in}{0.499444in}}%
\pgfpathlineto{\pgfqpoint{2.233427in}{0.499444in}}%
\pgfpathlineto{\pgfqpoint{2.233427in}{0.499444in}}%
\pgfpathlineto{\pgfqpoint{2.172041in}{0.499444in}}%
\pgfpathlineto{\pgfqpoint{2.172041in}{0.499444in}}%
\pgfpathclose%
\pgfusepath{fill}%
\end{pgfscope}%
\begin{pgfscope}%
\pgfpathrectangle{\pgfqpoint{0.445556in}{0.499444in}}{\pgfqpoint{3.875000in}{1.155000in}}%
\pgfusepath{clip}%
\pgfsetbuttcap%
\pgfsetmiterjoin%
\definecolor{currentfill}{rgb}{0.000000,0.000000,0.000000}%
\pgfsetfillcolor{currentfill}%
\pgfsetlinewidth{0.000000pt}%
\definecolor{currentstroke}{rgb}{0.000000,0.000000,0.000000}%
\pgfsetstrokecolor{currentstroke}%
\pgfsetstrokeopacity{0.000000}%
\pgfsetdash{}{0pt}%
\pgfpathmoveto{\pgfqpoint{2.325506in}{0.499444in}}%
\pgfpathlineto{\pgfqpoint{2.386892in}{0.499444in}}%
\pgfpathlineto{\pgfqpoint{2.386892in}{0.499561in}}%
\pgfpathlineto{\pgfqpoint{2.325506in}{0.499561in}}%
\pgfpathlineto{\pgfqpoint{2.325506in}{0.499444in}}%
\pgfpathclose%
\pgfusepath{fill}%
\end{pgfscope}%
\begin{pgfscope}%
\pgfpathrectangle{\pgfqpoint{0.445556in}{0.499444in}}{\pgfqpoint{3.875000in}{1.155000in}}%
\pgfusepath{clip}%
\pgfsetbuttcap%
\pgfsetmiterjoin%
\definecolor{currentfill}{rgb}{0.000000,0.000000,0.000000}%
\pgfsetfillcolor{currentfill}%
\pgfsetlinewidth{0.000000pt}%
\definecolor{currentstroke}{rgb}{0.000000,0.000000,0.000000}%
\pgfsetstrokecolor{currentstroke}%
\pgfsetstrokeopacity{0.000000}%
\pgfsetdash{}{0pt}%
\pgfpathmoveto{\pgfqpoint{2.478972in}{0.499444in}}%
\pgfpathlineto{\pgfqpoint{2.540358in}{0.499444in}}%
\pgfpathlineto{\pgfqpoint{2.540358in}{0.499854in}}%
\pgfpathlineto{\pgfqpoint{2.478972in}{0.499854in}}%
\pgfpathlineto{\pgfqpoint{2.478972in}{0.499444in}}%
\pgfpathclose%
\pgfusepath{fill}%
\end{pgfscope}%
\begin{pgfscope}%
\pgfpathrectangle{\pgfqpoint{0.445556in}{0.499444in}}{\pgfqpoint{3.875000in}{1.155000in}}%
\pgfusepath{clip}%
\pgfsetbuttcap%
\pgfsetmiterjoin%
\definecolor{currentfill}{rgb}{0.000000,0.000000,0.000000}%
\pgfsetfillcolor{currentfill}%
\pgfsetlinewidth{0.000000pt}%
\definecolor{currentstroke}{rgb}{0.000000,0.000000,0.000000}%
\pgfsetstrokecolor{currentstroke}%
\pgfsetstrokeopacity{0.000000}%
\pgfsetdash{}{0pt}%
\pgfpathmoveto{\pgfqpoint{2.632437in}{0.499444in}}%
\pgfpathlineto{\pgfqpoint{2.693823in}{0.499444in}}%
\pgfpathlineto{\pgfqpoint{2.693823in}{0.501521in}}%
\pgfpathlineto{\pgfqpoint{2.632437in}{0.501521in}}%
\pgfpathlineto{\pgfqpoint{2.632437in}{0.499444in}}%
\pgfpathclose%
\pgfusepath{fill}%
\end{pgfscope}%
\begin{pgfscope}%
\pgfpathrectangle{\pgfqpoint{0.445556in}{0.499444in}}{\pgfqpoint{3.875000in}{1.155000in}}%
\pgfusepath{clip}%
\pgfsetbuttcap%
\pgfsetmiterjoin%
\definecolor{currentfill}{rgb}{0.000000,0.000000,0.000000}%
\pgfsetfillcolor{currentfill}%
\pgfsetlinewidth{0.000000pt}%
\definecolor{currentstroke}{rgb}{0.000000,0.000000,0.000000}%
\pgfsetstrokecolor{currentstroke}%
\pgfsetstrokeopacity{0.000000}%
\pgfsetdash{}{0pt}%
\pgfpathmoveto{\pgfqpoint{2.785902in}{0.499444in}}%
\pgfpathlineto{\pgfqpoint{2.847288in}{0.499444in}}%
\pgfpathlineto{\pgfqpoint{2.847288in}{0.511171in}}%
\pgfpathlineto{\pgfqpoint{2.785902in}{0.511171in}}%
\pgfpathlineto{\pgfqpoint{2.785902in}{0.499444in}}%
\pgfpathclose%
\pgfusepath{fill}%
\end{pgfscope}%
\begin{pgfscope}%
\pgfpathrectangle{\pgfqpoint{0.445556in}{0.499444in}}{\pgfqpoint{3.875000in}{1.155000in}}%
\pgfusepath{clip}%
\pgfsetbuttcap%
\pgfsetmiterjoin%
\definecolor{currentfill}{rgb}{0.000000,0.000000,0.000000}%
\pgfsetfillcolor{currentfill}%
\pgfsetlinewidth{0.000000pt}%
\definecolor{currentstroke}{rgb}{0.000000,0.000000,0.000000}%
\pgfsetstrokecolor{currentstroke}%
\pgfsetstrokeopacity{0.000000}%
\pgfsetdash{}{0pt}%
\pgfpathmoveto{\pgfqpoint{2.939368in}{0.499444in}}%
\pgfpathlineto{\pgfqpoint{3.000754in}{0.499444in}}%
\pgfpathlineto{\pgfqpoint{3.000754in}{0.534889in}}%
\pgfpathlineto{\pgfqpoint{2.939368in}{0.534889in}}%
\pgfpathlineto{\pgfqpoint{2.939368in}{0.499444in}}%
\pgfpathclose%
\pgfusepath{fill}%
\end{pgfscope}%
\begin{pgfscope}%
\pgfpathrectangle{\pgfqpoint{0.445556in}{0.499444in}}{\pgfqpoint{3.875000in}{1.155000in}}%
\pgfusepath{clip}%
\pgfsetbuttcap%
\pgfsetmiterjoin%
\definecolor{currentfill}{rgb}{0.000000,0.000000,0.000000}%
\pgfsetfillcolor{currentfill}%
\pgfsetlinewidth{0.000000pt}%
\definecolor{currentstroke}{rgb}{0.000000,0.000000,0.000000}%
\pgfsetstrokecolor{currentstroke}%
\pgfsetstrokeopacity{0.000000}%
\pgfsetdash{}{0pt}%
\pgfpathmoveto{\pgfqpoint{3.092833in}{0.499444in}}%
\pgfpathlineto{\pgfqpoint{3.154219in}{0.499444in}}%
\pgfpathlineto{\pgfqpoint{3.154219in}{0.582998in}}%
\pgfpathlineto{\pgfqpoint{3.092833in}{0.582998in}}%
\pgfpathlineto{\pgfqpoint{3.092833in}{0.499444in}}%
\pgfpathclose%
\pgfusepath{fill}%
\end{pgfscope}%
\begin{pgfscope}%
\pgfpathrectangle{\pgfqpoint{0.445556in}{0.499444in}}{\pgfqpoint{3.875000in}{1.155000in}}%
\pgfusepath{clip}%
\pgfsetbuttcap%
\pgfsetmiterjoin%
\definecolor{currentfill}{rgb}{0.000000,0.000000,0.000000}%
\pgfsetfillcolor{currentfill}%
\pgfsetlinewidth{0.000000pt}%
\definecolor{currentstroke}{rgb}{0.000000,0.000000,0.000000}%
\pgfsetstrokecolor{currentstroke}%
\pgfsetstrokeopacity{0.000000}%
\pgfsetdash{}{0pt}%
\pgfpathmoveto{\pgfqpoint{3.246298in}{0.499444in}}%
\pgfpathlineto{\pgfqpoint{3.307684in}{0.499444in}}%
\pgfpathlineto{\pgfqpoint{3.307684in}{0.662340in}}%
\pgfpathlineto{\pgfqpoint{3.246298in}{0.662340in}}%
\pgfpathlineto{\pgfqpoint{3.246298in}{0.499444in}}%
\pgfpathclose%
\pgfusepath{fill}%
\end{pgfscope}%
\begin{pgfscope}%
\pgfpathrectangle{\pgfqpoint{0.445556in}{0.499444in}}{\pgfqpoint{3.875000in}{1.155000in}}%
\pgfusepath{clip}%
\pgfsetbuttcap%
\pgfsetmiterjoin%
\definecolor{currentfill}{rgb}{0.000000,0.000000,0.000000}%
\pgfsetfillcolor{currentfill}%
\pgfsetlinewidth{0.000000pt}%
\definecolor{currentstroke}{rgb}{0.000000,0.000000,0.000000}%
\pgfsetstrokecolor{currentstroke}%
\pgfsetstrokeopacity{0.000000}%
\pgfsetdash{}{0pt}%
\pgfpathmoveto{\pgfqpoint{3.399764in}{0.499444in}}%
\pgfpathlineto{\pgfqpoint{3.461150in}{0.499444in}}%
\pgfpathlineto{\pgfqpoint{3.461150in}{0.763967in}}%
\pgfpathlineto{\pgfqpoint{3.399764in}{0.763967in}}%
\pgfpathlineto{\pgfqpoint{3.399764in}{0.499444in}}%
\pgfpathclose%
\pgfusepath{fill}%
\end{pgfscope}%
\begin{pgfscope}%
\pgfpathrectangle{\pgfqpoint{0.445556in}{0.499444in}}{\pgfqpoint{3.875000in}{1.155000in}}%
\pgfusepath{clip}%
\pgfsetbuttcap%
\pgfsetmiterjoin%
\definecolor{currentfill}{rgb}{0.000000,0.000000,0.000000}%
\pgfsetfillcolor{currentfill}%
\pgfsetlinewidth{0.000000pt}%
\definecolor{currentstroke}{rgb}{0.000000,0.000000,0.000000}%
\pgfsetstrokecolor{currentstroke}%
\pgfsetstrokeopacity{0.000000}%
\pgfsetdash{}{0pt}%
\pgfpathmoveto{\pgfqpoint{3.553229in}{0.499444in}}%
\pgfpathlineto{\pgfqpoint{3.614615in}{0.499444in}}%
\pgfpathlineto{\pgfqpoint{3.614615in}{0.871794in}}%
\pgfpathlineto{\pgfqpoint{3.553229in}{0.871794in}}%
\pgfpathlineto{\pgfqpoint{3.553229in}{0.499444in}}%
\pgfpathclose%
\pgfusepath{fill}%
\end{pgfscope}%
\begin{pgfscope}%
\pgfpathrectangle{\pgfqpoint{0.445556in}{0.499444in}}{\pgfqpoint{3.875000in}{1.155000in}}%
\pgfusepath{clip}%
\pgfsetbuttcap%
\pgfsetmiterjoin%
\definecolor{currentfill}{rgb}{0.000000,0.000000,0.000000}%
\pgfsetfillcolor{currentfill}%
\pgfsetlinewidth{0.000000pt}%
\definecolor{currentstroke}{rgb}{0.000000,0.000000,0.000000}%
\pgfsetstrokecolor{currentstroke}%
\pgfsetstrokeopacity{0.000000}%
\pgfsetdash{}{0pt}%
\pgfpathmoveto{\pgfqpoint{3.706694in}{0.499444in}}%
\pgfpathlineto{\pgfqpoint{3.768080in}{0.499444in}}%
\pgfpathlineto{\pgfqpoint{3.768080in}{0.925810in}}%
\pgfpathlineto{\pgfqpoint{3.706694in}{0.925810in}}%
\pgfpathlineto{\pgfqpoint{3.706694in}{0.499444in}}%
\pgfpathclose%
\pgfusepath{fill}%
\end{pgfscope}%
\begin{pgfscope}%
\pgfpathrectangle{\pgfqpoint{0.445556in}{0.499444in}}{\pgfqpoint{3.875000in}{1.155000in}}%
\pgfusepath{clip}%
\pgfsetbuttcap%
\pgfsetmiterjoin%
\definecolor{currentfill}{rgb}{0.000000,0.000000,0.000000}%
\pgfsetfillcolor{currentfill}%
\pgfsetlinewidth{0.000000pt}%
\definecolor{currentstroke}{rgb}{0.000000,0.000000,0.000000}%
\pgfsetstrokecolor{currentstroke}%
\pgfsetstrokeopacity{0.000000}%
\pgfsetdash{}{0pt}%
\pgfpathmoveto{\pgfqpoint{3.860160in}{0.499444in}}%
\pgfpathlineto{\pgfqpoint{3.921546in}{0.499444in}}%
\pgfpathlineto{\pgfqpoint{3.921546in}{0.913586in}}%
\pgfpathlineto{\pgfqpoint{3.860160in}{0.913586in}}%
\pgfpathlineto{\pgfqpoint{3.860160in}{0.499444in}}%
\pgfpathclose%
\pgfusepath{fill}%
\end{pgfscope}%
\begin{pgfscope}%
\pgfpathrectangle{\pgfqpoint{0.445556in}{0.499444in}}{\pgfqpoint{3.875000in}{1.155000in}}%
\pgfusepath{clip}%
\pgfsetbuttcap%
\pgfsetmiterjoin%
\definecolor{currentfill}{rgb}{0.000000,0.000000,0.000000}%
\pgfsetfillcolor{currentfill}%
\pgfsetlinewidth{0.000000pt}%
\definecolor{currentstroke}{rgb}{0.000000,0.000000,0.000000}%
\pgfsetstrokecolor{currentstroke}%
\pgfsetstrokeopacity{0.000000}%
\pgfsetdash{}{0pt}%
\pgfpathmoveto{\pgfqpoint{4.013625in}{0.499444in}}%
\pgfpathlineto{\pgfqpoint{4.075011in}{0.499444in}}%
\pgfpathlineto{\pgfqpoint{4.075011in}{0.851907in}}%
\pgfpathlineto{\pgfqpoint{4.013625in}{0.851907in}}%
\pgfpathlineto{\pgfqpoint{4.013625in}{0.499444in}}%
\pgfpathclose%
\pgfusepath{fill}%
\end{pgfscope}%
\begin{pgfscope}%
\pgfpathrectangle{\pgfqpoint{0.445556in}{0.499444in}}{\pgfqpoint{3.875000in}{1.155000in}}%
\pgfusepath{clip}%
\pgfsetbuttcap%
\pgfsetmiterjoin%
\definecolor{currentfill}{rgb}{0.000000,0.000000,0.000000}%
\pgfsetfillcolor{currentfill}%
\pgfsetlinewidth{0.000000pt}%
\definecolor{currentstroke}{rgb}{0.000000,0.000000,0.000000}%
\pgfsetstrokecolor{currentstroke}%
\pgfsetstrokeopacity{0.000000}%
\pgfsetdash{}{0pt}%
\pgfpathmoveto{\pgfqpoint{4.167090in}{0.499444in}}%
\pgfpathlineto{\pgfqpoint{4.228476in}{0.499444in}}%
\pgfpathlineto{\pgfqpoint{4.228476in}{0.681583in}}%
\pgfpathlineto{\pgfqpoint{4.167090in}{0.681583in}}%
\pgfpathlineto{\pgfqpoint{4.167090in}{0.499444in}}%
\pgfpathclose%
\pgfusepath{fill}%
\end{pgfscope}%
\begin{pgfscope}%
\pgfsetbuttcap%
\pgfsetroundjoin%
\definecolor{currentfill}{rgb}{0.000000,0.000000,0.000000}%
\pgfsetfillcolor{currentfill}%
\pgfsetlinewidth{0.803000pt}%
\definecolor{currentstroke}{rgb}{0.000000,0.000000,0.000000}%
\pgfsetstrokecolor{currentstroke}%
\pgfsetdash{}{0pt}%
\pgfsys@defobject{currentmarker}{\pgfqpoint{0.000000in}{-0.048611in}}{\pgfqpoint{0.000000in}{0.000000in}}{%
\pgfpathmoveto{\pgfqpoint{0.000000in}{0.000000in}}%
\pgfpathlineto{\pgfqpoint{0.000000in}{-0.048611in}}%
\pgfusepath{stroke,fill}%
}%
\begin{pgfscope}%
\pgfsys@transformshift{0.483922in}{0.499444in}%
\pgfsys@useobject{currentmarker}{}%
\end{pgfscope}%
\end{pgfscope}%
\begin{pgfscope}%
\definecolor{textcolor}{rgb}{0.000000,0.000000,0.000000}%
\pgfsetstrokecolor{textcolor}%
\pgfsetfillcolor{textcolor}%
\pgftext[x=0.483922in,y=0.402222in,,top]{\color{textcolor}\rmfamily\fontsize{10.000000}{12.000000}\selectfont 0.0}%
\end{pgfscope}%
\begin{pgfscope}%
\pgfsetbuttcap%
\pgfsetroundjoin%
\definecolor{currentfill}{rgb}{0.000000,0.000000,0.000000}%
\pgfsetfillcolor{currentfill}%
\pgfsetlinewidth{0.803000pt}%
\definecolor{currentstroke}{rgb}{0.000000,0.000000,0.000000}%
\pgfsetstrokecolor{currentstroke}%
\pgfsetdash{}{0pt}%
\pgfsys@defobject{currentmarker}{\pgfqpoint{0.000000in}{-0.048611in}}{\pgfqpoint{0.000000in}{0.000000in}}{%
\pgfpathmoveto{\pgfqpoint{0.000000in}{0.000000in}}%
\pgfpathlineto{\pgfqpoint{0.000000in}{-0.048611in}}%
\pgfusepath{stroke,fill}%
}%
\begin{pgfscope}%
\pgfsys@transformshift{0.867585in}{0.499444in}%
\pgfsys@useobject{currentmarker}{}%
\end{pgfscope}%
\end{pgfscope}%
\begin{pgfscope}%
\definecolor{textcolor}{rgb}{0.000000,0.000000,0.000000}%
\pgfsetstrokecolor{textcolor}%
\pgfsetfillcolor{textcolor}%
\pgftext[x=0.867585in,y=0.402222in,,top]{\color{textcolor}\rmfamily\fontsize{10.000000}{12.000000}\selectfont 0.1}%
\end{pgfscope}%
\begin{pgfscope}%
\pgfsetbuttcap%
\pgfsetroundjoin%
\definecolor{currentfill}{rgb}{0.000000,0.000000,0.000000}%
\pgfsetfillcolor{currentfill}%
\pgfsetlinewidth{0.803000pt}%
\definecolor{currentstroke}{rgb}{0.000000,0.000000,0.000000}%
\pgfsetstrokecolor{currentstroke}%
\pgfsetdash{}{0pt}%
\pgfsys@defobject{currentmarker}{\pgfqpoint{0.000000in}{-0.048611in}}{\pgfqpoint{0.000000in}{0.000000in}}{%
\pgfpathmoveto{\pgfqpoint{0.000000in}{0.000000in}}%
\pgfpathlineto{\pgfqpoint{0.000000in}{-0.048611in}}%
\pgfusepath{stroke,fill}%
}%
\begin{pgfscope}%
\pgfsys@transformshift{1.251249in}{0.499444in}%
\pgfsys@useobject{currentmarker}{}%
\end{pgfscope}%
\end{pgfscope}%
\begin{pgfscope}%
\definecolor{textcolor}{rgb}{0.000000,0.000000,0.000000}%
\pgfsetstrokecolor{textcolor}%
\pgfsetfillcolor{textcolor}%
\pgftext[x=1.251249in,y=0.402222in,,top]{\color{textcolor}\rmfamily\fontsize{10.000000}{12.000000}\selectfont 0.2}%
\end{pgfscope}%
\begin{pgfscope}%
\pgfsetbuttcap%
\pgfsetroundjoin%
\definecolor{currentfill}{rgb}{0.000000,0.000000,0.000000}%
\pgfsetfillcolor{currentfill}%
\pgfsetlinewidth{0.803000pt}%
\definecolor{currentstroke}{rgb}{0.000000,0.000000,0.000000}%
\pgfsetstrokecolor{currentstroke}%
\pgfsetdash{}{0pt}%
\pgfsys@defobject{currentmarker}{\pgfqpoint{0.000000in}{-0.048611in}}{\pgfqpoint{0.000000in}{0.000000in}}{%
\pgfpathmoveto{\pgfqpoint{0.000000in}{0.000000in}}%
\pgfpathlineto{\pgfqpoint{0.000000in}{-0.048611in}}%
\pgfusepath{stroke,fill}%
}%
\begin{pgfscope}%
\pgfsys@transformshift{1.634912in}{0.499444in}%
\pgfsys@useobject{currentmarker}{}%
\end{pgfscope}%
\end{pgfscope}%
\begin{pgfscope}%
\definecolor{textcolor}{rgb}{0.000000,0.000000,0.000000}%
\pgfsetstrokecolor{textcolor}%
\pgfsetfillcolor{textcolor}%
\pgftext[x=1.634912in,y=0.402222in,,top]{\color{textcolor}\rmfamily\fontsize{10.000000}{12.000000}\selectfont 0.3}%
\end{pgfscope}%
\begin{pgfscope}%
\pgfsetbuttcap%
\pgfsetroundjoin%
\definecolor{currentfill}{rgb}{0.000000,0.000000,0.000000}%
\pgfsetfillcolor{currentfill}%
\pgfsetlinewidth{0.803000pt}%
\definecolor{currentstroke}{rgb}{0.000000,0.000000,0.000000}%
\pgfsetstrokecolor{currentstroke}%
\pgfsetdash{}{0pt}%
\pgfsys@defobject{currentmarker}{\pgfqpoint{0.000000in}{-0.048611in}}{\pgfqpoint{0.000000in}{0.000000in}}{%
\pgfpathmoveto{\pgfqpoint{0.000000in}{0.000000in}}%
\pgfpathlineto{\pgfqpoint{0.000000in}{-0.048611in}}%
\pgfusepath{stroke,fill}%
}%
\begin{pgfscope}%
\pgfsys@transformshift{2.018575in}{0.499444in}%
\pgfsys@useobject{currentmarker}{}%
\end{pgfscope}%
\end{pgfscope}%
\begin{pgfscope}%
\definecolor{textcolor}{rgb}{0.000000,0.000000,0.000000}%
\pgfsetstrokecolor{textcolor}%
\pgfsetfillcolor{textcolor}%
\pgftext[x=2.018575in,y=0.402222in,,top]{\color{textcolor}\rmfamily\fontsize{10.000000}{12.000000}\selectfont 0.4}%
\end{pgfscope}%
\begin{pgfscope}%
\pgfsetbuttcap%
\pgfsetroundjoin%
\definecolor{currentfill}{rgb}{0.000000,0.000000,0.000000}%
\pgfsetfillcolor{currentfill}%
\pgfsetlinewidth{0.803000pt}%
\definecolor{currentstroke}{rgb}{0.000000,0.000000,0.000000}%
\pgfsetstrokecolor{currentstroke}%
\pgfsetdash{}{0pt}%
\pgfsys@defobject{currentmarker}{\pgfqpoint{0.000000in}{-0.048611in}}{\pgfqpoint{0.000000in}{0.000000in}}{%
\pgfpathmoveto{\pgfqpoint{0.000000in}{0.000000in}}%
\pgfpathlineto{\pgfqpoint{0.000000in}{-0.048611in}}%
\pgfusepath{stroke,fill}%
}%
\begin{pgfscope}%
\pgfsys@transformshift{2.402239in}{0.499444in}%
\pgfsys@useobject{currentmarker}{}%
\end{pgfscope}%
\end{pgfscope}%
\begin{pgfscope}%
\definecolor{textcolor}{rgb}{0.000000,0.000000,0.000000}%
\pgfsetstrokecolor{textcolor}%
\pgfsetfillcolor{textcolor}%
\pgftext[x=2.402239in,y=0.402222in,,top]{\color{textcolor}\rmfamily\fontsize{10.000000}{12.000000}\selectfont 0.5}%
\end{pgfscope}%
\begin{pgfscope}%
\pgfsetbuttcap%
\pgfsetroundjoin%
\definecolor{currentfill}{rgb}{0.000000,0.000000,0.000000}%
\pgfsetfillcolor{currentfill}%
\pgfsetlinewidth{0.803000pt}%
\definecolor{currentstroke}{rgb}{0.000000,0.000000,0.000000}%
\pgfsetstrokecolor{currentstroke}%
\pgfsetdash{}{0pt}%
\pgfsys@defobject{currentmarker}{\pgfqpoint{0.000000in}{-0.048611in}}{\pgfqpoint{0.000000in}{0.000000in}}{%
\pgfpathmoveto{\pgfqpoint{0.000000in}{0.000000in}}%
\pgfpathlineto{\pgfqpoint{0.000000in}{-0.048611in}}%
\pgfusepath{stroke,fill}%
}%
\begin{pgfscope}%
\pgfsys@transformshift{2.785902in}{0.499444in}%
\pgfsys@useobject{currentmarker}{}%
\end{pgfscope}%
\end{pgfscope}%
\begin{pgfscope}%
\definecolor{textcolor}{rgb}{0.000000,0.000000,0.000000}%
\pgfsetstrokecolor{textcolor}%
\pgfsetfillcolor{textcolor}%
\pgftext[x=2.785902in,y=0.402222in,,top]{\color{textcolor}\rmfamily\fontsize{10.000000}{12.000000}\selectfont 0.6}%
\end{pgfscope}%
\begin{pgfscope}%
\pgfsetbuttcap%
\pgfsetroundjoin%
\definecolor{currentfill}{rgb}{0.000000,0.000000,0.000000}%
\pgfsetfillcolor{currentfill}%
\pgfsetlinewidth{0.803000pt}%
\definecolor{currentstroke}{rgb}{0.000000,0.000000,0.000000}%
\pgfsetstrokecolor{currentstroke}%
\pgfsetdash{}{0pt}%
\pgfsys@defobject{currentmarker}{\pgfqpoint{0.000000in}{-0.048611in}}{\pgfqpoint{0.000000in}{0.000000in}}{%
\pgfpathmoveto{\pgfqpoint{0.000000in}{0.000000in}}%
\pgfpathlineto{\pgfqpoint{0.000000in}{-0.048611in}}%
\pgfusepath{stroke,fill}%
}%
\begin{pgfscope}%
\pgfsys@transformshift{3.169566in}{0.499444in}%
\pgfsys@useobject{currentmarker}{}%
\end{pgfscope}%
\end{pgfscope}%
\begin{pgfscope}%
\definecolor{textcolor}{rgb}{0.000000,0.000000,0.000000}%
\pgfsetstrokecolor{textcolor}%
\pgfsetfillcolor{textcolor}%
\pgftext[x=3.169566in,y=0.402222in,,top]{\color{textcolor}\rmfamily\fontsize{10.000000}{12.000000}\selectfont 0.7}%
\end{pgfscope}%
\begin{pgfscope}%
\pgfsetbuttcap%
\pgfsetroundjoin%
\definecolor{currentfill}{rgb}{0.000000,0.000000,0.000000}%
\pgfsetfillcolor{currentfill}%
\pgfsetlinewidth{0.803000pt}%
\definecolor{currentstroke}{rgb}{0.000000,0.000000,0.000000}%
\pgfsetstrokecolor{currentstroke}%
\pgfsetdash{}{0pt}%
\pgfsys@defobject{currentmarker}{\pgfqpoint{0.000000in}{-0.048611in}}{\pgfqpoint{0.000000in}{0.000000in}}{%
\pgfpathmoveto{\pgfqpoint{0.000000in}{0.000000in}}%
\pgfpathlineto{\pgfqpoint{0.000000in}{-0.048611in}}%
\pgfusepath{stroke,fill}%
}%
\begin{pgfscope}%
\pgfsys@transformshift{3.553229in}{0.499444in}%
\pgfsys@useobject{currentmarker}{}%
\end{pgfscope}%
\end{pgfscope}%
\begin{pgfscope}%
\definecolor{textcolor}{rgb}{0.000000,0.000000,0.000000}%
\pgfsetstrokecolor{textcolor}%
\pgfsetfillcolor{textcolor}%
\pgftext[x=3.553229in,y=0.402222in,,top]{\color{textcolor}\rmfamily\fontsize{10.000000}{12.000000}\selectfont 0.8}%
\end{pgfscope}%
\begin{pgfscope}%
\pgfsetbuttcap%
\pgfsetroundjoin%
\definecolor{currentfill}{rgb}{0.000000,0.000000,0.000000}%
\pgfsetfillcolor{currentfill}%
\pgfsetlinewidth{0.803000pt}%
\definecolor{currentstroke}{rgb}{0.000000,0.000000,0.000000}%
\pgfsetstrokecolor{currentstroke}%
\pgfsetdash{}{0pt}%
\pgfsys@defobject{currentmarker}{\pgfqpoint{0.000000in}{-0.048611in}}{\pgfqpoint{0.000000in}{0.000000in}}{%
\pgfpathmoveto{\pgfqpoint{0.000000in}{0.000000in}}%
\pgfpathlineto{\pgfqpoint{0.000000in}{-0.048611in}}%
\pgfusepath{stroke,fill}%
}%
\begin{pgfscope}%
\pgfsys@transformshift{3.936892in}{0.499444in}%
\pgfsys@useobject{currentmarker}{}%
\end{pgfscope}%
\end{pgfscope}%
\begin{pgfscope}%
\definecolor{textcolor}{rgb}{0.000000,0.000000,0.000000}%
\pgfsetstrokecolor{textcolor}%
\pgfsetfillcolor{textcolor}%
\pgftext[x=3.936892in,y=0.402222in,,top]{\color{textcolor}\rmfamily\fontsize{10.000000}{12.000000}\selectfont 0.9}%
\end{pgfscope}%
\begin{pgfscope}%
\pgfsetbuttcap%
\pgfsetroundjoin%
\definecolor{currentfill}{rgb}{0.000000,0.000000,0.000000}%
\pgfsetfillcolor{currentfill}%
\pgfsetlinewidth{0.803000pt}%
\definecolor{currentstroke}{rgb}{0.000000,0.000000,0.000000}%
\pgfsetstrokecolor{currentstroke}%
\pgfsetdash{}{0pt}%
\pgfsys@defobject{currentmarker}{\pgfqpoint{0.000000in}{-0.048611in}}{\pgfqpoint{0.000000in}{0.000000in}}{%
\pgfpathmoveto{\pgfqpoint{0.000000in}{0.000000in}}%
\pgfpathlineto{\pgfqpoint{0.000000in}{-0.048611in}}%
\pgfusepath{stroke,fill}%
}%
\begin{pgfscope}%
\pgfsys@transformshift{4.320556in}{0.499444in}%
\pgfsys@useobject{currentmarker}{}%
\end{pgfscope}%
\end{pgfscope}%
\begin{pgfscope}%
\definecolor{textcolor}{rgb}{0.000000,0.000000,0.000000}%
\pgfsetstrokecolor{textcolor}%
\pgfsetfillcolor{textcolor}%
\pgftext[x=4.320556in,y=0.402222in,,top]{\color{textcolor}\rmfamily\fontsize{10.000000}{12.000000}\selectfont 1.0}%
\end{pgfscope}%
\begin{pgfscope}%
\definecolor{textcolor}{rgb}{0.000000,0.000000,0.000000}%
\pgfsetstrokecolor{textcolor}%
\pgfsetfillcolor{textcolor}%
\pgftext[x=2.383056in,y=0.223333in,,top]{\color{textcolor}\rmfamily\fontsize{10.000000}{12.000000}\selectfont \(\displaystyle p\)}%
\end{pgfscope}%
\begin{pgfscope}%
\pgfsetbuttcap%
\pgfsetroundjoin%
\definecolor{currentfill}{rgb}{0.000000,0.000000,0.000000}%
\pgfsetfillcolor{currentfill}%
\pgfsetlinewidth{0.803000pt}%
\definecolor{currentstroke}{rgb}{0.000000,0.000000,0.000000}%
\pgfsetstrokecolor{currentstroke}%
\pgfsetdash{}{0pt}%
\pgfsys@defobject{currentmarker}{\pgfqpoint{-0.048611in}{0.000000in}}{\pgfqpoint{-0.000000in}{0.000000in}}{%
\pgfpathmoveto{\pgfqpoint{-0.000000in}{0.000000in}}%
\pgfpathlineto{\pgfqpoint{-0.048611in}{0.000000in}}%
\pgfusepath{stroke,fill}%
}%
\begin{pgfscope}%
\pgfsys@transformshift{0.445556in}{0.499444in}%
\pgfsys@useobject{currentmarker}{}%
\end{pgfscope}%
\end{pgfscope}%
\begin{pgfscope}%
\definecolor{textcolor}{rgb}{0.000000,0.000000,0.000000}%
\pgfsetstrokecolor{textcolor}%
\pgfsetfillcolor{textcolor}%
\pgftext[x=0.278889in, y=0.451250in, left, base]{\color{textcolor}\rmfamily\fontsize{10.000000}{12.000000}\selectfont \(\displaystyle {0}\)}%
\end{pgfscope}%
\begin{pgfscope}%
\pgfsetbuttcap%
\pgfsetroundjoin%
\definecolor{currentfill}{rgb}{0.000000,0.000000,0.000000}%
\pgfsetfillcolor{currentfill}%
\pgfsetlinewidth{0.803000pt}%
\definecolor{currentstroke}{rgb}{0.000000,0.000000,0.000000}%
\pgfsetstrokecolor{currentstroke}%
\pgfsetdash{}{0pt}%
\pgfsys@defobject{currentmarker}{\pgfqpoint{-0.048611in}{0.000000in}}{\pgfqpoint{-0.000000in}{0.000000in}}{%
\pgfpathmoveto{\pgfqpoint{-0.000000in}{0.000000in}}%
\pgfpathlineto{\pgfqpoint{-0.048611in}{0.000000in}}%
\pgfusepath{stroke,fill}%
}%
\begin{pgfscope}%
\pgfsys@transformshift{0.445556in}{0.791601in}%
\pgfsys@useobject{currentmarker}{}%
\end{pgfscope}%
\end{pgfscope}%
\begin{pgfscope}%
\definecolor{textcolor}{rgb}{0.000000,0.000000,0.000000}%
\pgfsetstrokecolor{textcolor}%
\pgfsetfillcolor{textcolor}%
\pgftext[x=0.278889in, y=0.743407in, left, base]{\color{textcolor}\rmfamily\fontsize{10.000000}{12.000000}\selectfont \(\displaystyle {2}\)}%
\end{pgfscope}%
\begin{pgfscope}%
\pgfsetbuttcap%
\pgfsetroundjoin%
\definecolor{currentfill}{rgb}{0.000000,0.000000,0.000000}%
\pgfsetfillcolor{currentfill}%
\pgfsetlinewidth{0.803000pt}%
\definecolor{currentstroke}{rgb}{0.000000,0.000000,0.000000}%
\pgfsetstrokecolor{currentstroke}%
\pgfsetdash{}{0pt}%
\pgfsys@defobject{currentmarker}{\pgfqpoint{-0.048611in}{0.000000in}}{\pgfqpoint{-0.000000in}{0.000000in}}{%
\pgfpathmoveto{\pgfqpoint{-0.000000in}{0.000000in}}%
\pgfpathlineto{\pgfqpoint{-0.048611in}{0.000000in}}%
\pgfusepath{stroke,fill}%
}%
\begin{pgfscope}%
\pgfsys@transformshift{0.445556in}{1.083759in}%
\pgfsys@useobject{currentmarker}{}%
\end{pgfscope}%
\end{pgfscope}%
\begin{pgfscope}%
\definecolor{textcolor}{rgb}{0.000000,0.000000,0.000000}%
\pgfsetstrokecolor{textcolor}%
\pgfsetfillcolor{textcolor}%
\pgftext[x=0.278889in, y=1.035564in, left, base]{\color{textcolor}\rmfamily\fontsize{10.000000}{12.000000}\selectfont \(\displaystyle {4}\)}%
\end{pgfscope}%
\begin{pgfscope}%
\pgfsetbuttcap%
\pgfsetroundjoin%
\definecolor{currentfill}{rgb}{0.000000,0.000000,0.000000}%
\pgfsetfillcolor{currentfill}%
\pgfsetlinewidth{0.803000pt}%
\definecolor{currentstroke}{rgb}{0.000000,0.000000,0.000000}%
\pgfsetstrokecolor{currentstroke}%
\pgfsetdash{}{0pt}%
\pgfsys@defobject{currentmarker}{\pgfqpoint{-0.048611in}{0.000000in}}{\pgfqpoint{-0.000000in}{0.000000in}}{%
\pgfpathmoveto{\pgfqpoint{-0.000000in}{0.000000in}}%
\pgfpathlineto{\pgfqpoint{-0.048611in}{0.000000in}}%
\pgfusepath{stroke,fill}%
}%
\begin{pgfscope}%
\pgfsys@transformshift{0.445556in}{1.375916in}%
\pgfsys@useobject{currentmarker}{}%
\end{pgfscope}%
\end{pgfscope}%
\begin{pgfscope}%
\definecolor{textcolor}{rgb}{0.000000,0.000000,0.000000}%
\pgfsetstrokecolor{textcolor}%
\pgfsetfillcolor{textcolor}%
\pgftext[x=0.278889in, y=1.327722in, left, base]{\color{textcolor}\rmfamily\fontsize{10.000000}{12.000000}\selectfont \(\displaystyle {6}\)}%
\end{pgfscope}%
\begin{pgfscope}%
\definecolor{textcolor}{rgb}{0.000000,0.000000,0.000000}%
\pgfsetstrokecolor{textcolor}%
\pgfsetfillcolor{textcolor}%
\pgftext[x=0.223333in,y=1.076944in,,bottom,rotate=90.000000]{\color{textcolor}\rmfamily\fontsize{10.000000}{12.000000}\selectfont Percent of Data Set}%
\end{pgfscope}%
\begin{pgfscope}%
\pgfsetrectcap%
\pgfsetmiterjoin%
\pgfsetlinewidth{0.803000pt}%
\definecolor{currentstroke}{rgb}{0.000000,0.000000,0.000000}%
\pgfsetstrokecolor{currentstroke}%
\pgfsetdash{}{0pt}%
\pgfpathmoveto{\pgfqpoint{0.445556in}{0.499444in}}%
\pgfpathlineto{\pgfqpoint{0.445556in}{1.654444in}}%
\pgfusepath{stroke}%
\end{pgfscope}%
\begin{pgfscope}%
\pgfsetrectcap%
\pgfsetmiterjoin%
\pgfsetlinewidth{0.803000pt}%
\definecolor{currentstroke}{rgb}{0.000000,0.000000,0.000000}%
\pgfsetstrokecolor{currentstroke}%
\pgfsetdash{}{0pt}%
\pgfpathmoveto{\pgfqpoint{4.320556in}{0.499444in}}%
\pgfpathlineto{\pgfqpoint{4.320556in}{1.654444in}}%
\pgfusepath{stroke}%
\end{pgfscope}%
\begin{pgfscope}%
\pgfsetrectcap%
\pgfsetmiterjoin%
\pgfsetlinewidth{0.803000pt}%
\definecolor{currentstroke}{rgb}{0.000000,0.000000,0.000000}%
\pgfsetstrokecolor{currentstroke}%
\pgfsetdash{}{0pt}%
\pgfpathmoveto{\pgfqpoint{0.445556in}{0.499444in}}%
\pgfpathlineto{\pgfqpoint{4.320556in}{0.499444in}}%
\pgfusepath{stroke}%
\end{pgfscope}%
\begin{pgfscope}%
\pgfsetrectcap%
\pgfsetmiterjoin%
\pgfsetlinewidth{0.803000pt}%
\definecolor{currentstroke}{rgb}{0.000000,0.000000,0.000000}%
\pgfsetstrokecolor{currentstroke}%
\pgfsetdash{}{0pt}%
\pgfpathmoveto{\pgfqpoint{0.445556in}{1.654444in}}%
\pgfpathlineto{\pgfqpoint{4.320556in}{1.654444in}}%
\pgfusepath{stroke}%
\end{pgfscope}%
\begin{pgfscope}%
\pgfsetbuttcap%
\pgfsetmiterjoin%
\definecolor{currentfill}{rgb}{1.000000,1.000000,1.000000}%
\pgfsetfillcolor{currentfill}%
\pgfsetfillopacity{0.800000}%
\pgfsetlinewidth{1.003750pt}%
\definecolor{currentstroke}{rgb}{0.800000,0.800000,0.800000}%
\pgfsetstrokecolor{currentstroke}%
\pgfsetstrokeopacity{0.800000}%
\pgfsetdash{}{0pt}%
\pgfpathmoveto{\pgfqpoint{3.543611in}{1.154445in}}%
\pgfpathlineto{\pgfqpoint{4.223333in}{1.154445in}}%
\pgfpathquadraticcurveto{\pgfqpoint{4.251111in}{1.154445in}}{\pgfqpoint{4.251111in}{1.182222in}}%
\pgfpathlineto{\pgfqpoint{4.251111in}{1.557222in}}%
\pgfpathquadraticcurveto{\pgfqpoint{4.251111in}{1.585000in}}{\pgfqpoint{4.223333in}{1.585000in}}%
\pgfpathlineto{\pgfqpoint{3.543611in}{1.585000in}}%
\pgfpathquadraticcurveto{\pgfqpoint{3.515833in}{1.585000in}}{\pgfqpoint{3.515833in}{1.557222in}}%
\pgfpathlineto{\pgfqpoint{3.515833in}{1.182222in}}%
\pgfpathquadraticcurveto{\pgfqpoint{3.515833in}{1.154445in}}{\pgfqpoint{3.543611in}{1.154445in}}%
\pgfpathlineto{\pgfqpoint{3.543611in}{1.154445in}}%
\pgfpathclose%
\pgfusepath{stroke,fill}%
\end{pgfscope}%
\begin{pgfscope}%
\pgfsetbuttcap%
\pgfsetmiterjoin%
\pgfsetlinewidth{1.003750pt}%
\definecolor{currentstroke}{rgb}{0.000000,0.000000,0.000000}%
\pgfsetstrokecolor{currentstroke}%
\pgfsetdash{}{0pt}%
\pgfpathmoveto{\pgfqpoint{3.571389in}{1.432222in}}%
\pgfpathlineto{\pgfqpoint{3.849167in}{1.432222in}}%
\pgfpathlineto{\pgfqpoint{3.849167in}{1.529444in}}%
\pgfpathlineto{\pgfqpoint{3.571389in}{1.529444in}}%
\pgfpathlineto{\pgfqpoint{3.571389in}{1.432222in}}%
\pgfpathclose%
\pgfusepath{stroke}%
\end{pgfscope}%
\begin{pgfscope}%
\definecolor{textcolor}{rgb}{0.000000,0.000000,0.000000}%
\pgfsetstrokecolor{textcolor}%
\pgfsetfillcolor{textcolor}%
\pgftext[x=3.960278in,y=1.432222in,left,base]{\color{textcolor}\rmfamily\fontsize{10.000000}{12.000000}\selectfont Neg}%
\end{pgfscope}%
\begin{pgfscope}%
\pgfsetbuttcap%
\pgfsetmiterjoin%
\definecolor{currentfill}{rgb}{0.000000,0.000000,0.000000}%
\pgfsetfillcolor{currentfill}%
\pgfsetlinewidth{0.000000pt}%
\definecolor{currentstroke}{rgb}{0.000000,0.000000,0.000000}%
\pgfsetstrokecolor{currentstroke}%
\pgfsetstrokeopacity{0.000000}%
\pgfsetdash{}{0pt}%
\pgfpathmoveto{\pgfqpoint{3.571389in}{1.236944in}}%
\pgfpathlineto{\pgfqpoint{3.849167in}{1.236944in}}%
\pgfpathlineto{\pgfqpoint{3.849167in}{1.334167in}}%
\pgfpathlineto{\pgfqpoint{3.571389in}{1.334167in}}%
\pgfpathlineto{\pgfqpoint{3.571389in}{1.236944in}}%
\pgfpathclose%
\pgfusepath{fill}%
\end{pgfscope}%
\begin{pgfscope}%
\definecolor{textcolor}{rgb}{0.000000,0.000000,0.000000}%
\pgfsetstrokecolor{textcolor}%
\pgfsetfillcolor{textcolor}%
\pgftext[x=3.960278in,y=1.236944in,left,base]{\color{textcolor}\rmfamily\fontsize{10.000000}{12.000000}\selectfont Pos}%
\end{pgfscope}%
\end{pgfpicture}%
\makeatother%
\endgroup%
	
&
	\vskip 0pt
	\hfil ROC Curve
	
	%% Creator: Matplotlib, PGF backend
%%
%% To include the figure in your LaTeX document, write
%%   \input{<filename>.pgf}
%%
%% Make sure the required packages are loaded in your preamble
%%   \usepackage{pgf}
%%
%% Also ensure that all the required font packages are loaded; for instance,
%% the lmodern package is sometimes necessary when using math font.
%%   \usepackage{lmodern}
%%
%% Figures using additional raster images can only be included by \input if
%% they are in the same directory as the main LaTeX file. For loading figures
%% from other directories you can use the `import` package
%%   \usepackage{import}
%%
%% and then include the figures with
%%   \import{<path to file>}{<filename>.pgf}
%%
%% Matplotlib used the following preamble
%%   
%%   \usepackage{fontspec}
%%   \makeatletter\@ifpackageloaded{underscore}{}{\usepackage[strings]{underscore}}\makeatother
%%
\begingroup%
\makeatletter%
\begin{pgfpicture}%
\pgfpathrectangle{\pgfpointorigin}{\pgfqpoint{2.221861in}{1.754444in}}%
\pgfusepath{use as bounding box, clip}%
\begin{pgfscope}%
\pgfsetbuttcap%
\pgfsetmiterjoin%
\definecolor{currentfill}{rgb}{1.000000,1.000000,1.000000}%
\pgfsetfillcolor{currentfill}%
\pgfsetlinewidth{0.000000pt}%
\definecolor{currentstroke}{rgb}{1.000000,1.000000,1.000000}%
\pgfsetstrokecolor{currentstroke}%
\pgfsetdash{}{0pt}%
\pgfpathmoveto{\pgfqpoint{0.000000in}{0.000000in}}%
\pgfpathlineto{\pgfqpoint{2.221861in}{0.000000in}}%
\pgfpathlineto{\pgfqpoint{2.221861in}{1.754444in}}%
\pgfpathlineto{\pgfqpoint{0.000000in}{1.754444in}}%
\pgfpathlineto{\pgfqpoint{0.000000in}{0.000000in}}%
\pgfpathclose%
\pgfusepath{fill}%
\end{pgfscope}%
\begin{pgfscope}%
\pgfsetbuttcap%
\pgfsetmiterjoin%
\definecolor{currentfill}{rgb}{1.000000,1.000000,1.000000}%
\pgfsetfillcolor{currentfill}%
\pgfsetlinewidth{0.000000pt}%
\definecolor{currentstroke}{rgb}{0.000000,0.000000,0.000000}%
\pgfsetstrokecolor{currentstroke}%
\pgfsetstrokeopacity{0.000000}%
\pgfsetdash{}{0pt}%
\pgfpathmoveto{\pgfqpoint{0.553581in}{0.499444in}}%
\pgfpathlineto{\pgfqpoint{2.103581in}{0.499444in}}%
\pgfpathlineto{\pgfqpoint{2.103581in}{1.654444in}}%
\pgfpathlineto{\pgfqpoint{0.553581in}{1.654444in}}%
\pgfpathlineto{\pgfqpoint{0.553581in}{0.499444in}}%
\pgfpathclose%
\pgfusepath{fill}%
\end{pgfscope}%
\begin{pgfscope}%
\pgfsetbuttcap%
\pgfsetroundjoin%
\definecolor{currentfill}{rgb}{0.000000,0.000000,0.000000}%
\pgfsetfillcolor{currentfill}%
\pgfsetlinewidth{0.803000pt}%
\definecolor{currentstroke}{rgb}{0.000000,0.000000,0.000000}%
\pgfsetstrokecolor{currentstroke}%
\pgfsetdash{}{0pt}%
\pgfsys@defobject{currentmarker}{\pgfqpoint{0.000000in}{-0.048611in}}{\pgfqpoint{0.000000in}{0.000000in}}{%
\pgfpathmoveto{\pgfqpoint{0.000000in}{0.000000in}}%
\pgfpathlineto{\pgfqpoint{0.000000in}{-0.048611in}}%
\pgfusepath{stroke,fill}%
}%
\begin{pgfscope}%
\pgfsys@transformshift{0.624035in}{0.499444in}%
\pgfsys@useobject{currentmarker}{}%
\end{pgfscope}%
\end{pgfscope}%
\begin{pgfscope}%
\definecolor{textcolor}{rgb}{0.000000,0.000000,0.000000}%
\pgfsetstrokecolor{textcolor}%
\pgfsetfillcolor{textcolor}%
\pgftext[x=0.624035in,y=0.402222in,,top]{\color{textcolor}\rmfamily\fontsize{10.000000}{12.000000}\selectfont \(\displaystyle {0.0}\)}%
\end{pgfscope}%
\begin{pgfscope}%
\pgfsetbuttcap%
\pgfsetroundjoin%
\definecolor{currentfill}{rgb}{0.000000,0.000000,0.000000}%
\pgfsetfillcolor{currentfill}%
\pgfsetlinewidth{0.803000pt}%
\definecolor{currentstroke}{rgb}{0.000000,0.000000,0.000000}%
\pgfsetstrokecolor{currentstroke}%
\pgfsetdash{}{0pt}%
\pgfsys@defobject{currentmarker}{\pgfqpoint{0.000000in}{-0.048611in}}{\pgfqpoint{0.000000in}{0.000000in}}{%
\pgfpathmoveto{\pgfqpoint{0.000000in}{0.000000in}}%
\pgfpathlineto{\pgfqpoint{0.000000in}{-0.048611in}}%
\pgfusepath{stroke,fill}%
}%
\begin{pgfscope}%
\pgfsys@transformshift{1.328581in}{0.499444in}%
\pgfsys@useobject{currentmarker}{}%
\end{pgfscope}%
\end{pgfscope}%
\begin{pgfscope}%
\definecolor{textcolor}{rgb}{0.000000,0.000000,0.000000}%
\pgfsetstrokecolor{textcolor}%
\pgfsetfillcolor{textcolor}%
\pgftext[x=1.328581in,y=0.402222in,,top]{\color{textcolor}\rmfamily\fontsize{10.000000}{12.000000}\selectfont \(\displaystyle {0.5}\)}%
\end{pgfscope}%
\begin{pgfscope}%
\pgfsetbuttcap%
\pgfsetroundjoin%
\definecolor{currentfill}{rgb}{0.000000,0.000000,0.000000}%
\pgfsetfillcolor{currentfill}%
\pgfsetlinewidth{0.803000pt}%
\definecolor{currentstroke}{rgb}{0.000000,0.000000,0.000000}%
\pgfsetstrokecolor{currentstroke}%
\pgfsetdash{}{0pt}%
\pgfsys@defobject{currentmarker}{\pgfqpoint{0.000000in}{-0.048611in}}{\pgfqpoint{0.000000in}{0.000000in}}{%
\pgfpathmoveto{\pgfqpoint{0.000000in}{0.000000in}}%
\pgfpathlineto{\pgfqpoint{0.000000in}{-0.048611in}}%
\pgfusepath{stroke,fill}%
}%
\begin{pgfscope}%
\pgfsys@transformshift{2.033126in}{0.499444in}%
\pgfsys@useobject{currentmarker}{}%
\end{pgfscope}%
\end{pgfscope}%
\begin{pgfscope}%
\definecolor{textcolor}{rgb}{0.000000,0.000000,0.000000}%
\pgfsetstrokecolor{textcolor}%
\pgfsetfillcolor{textcolor}%
\pgftext[x=2.033126in,y=0.402222in,,top]{\color{textcolor}\rmfamily\fontsize{10.000000}{12.000000}\selectfont \(\displaystyle {1.0}\)}%
\end{pgfscope}%
\begin{pgfscope}%
\definecolor{textcolor}{rgb}{0.000000,0.000000,0.000000}%
\pgfsetstrokecolor{textcolor}%
\pgfsetfillcolor{textcolor}%
\pgftext[x=1.328581in,y=0.223333in,,top]{\color{textcolor}\rmfamily\fontsize{10.000000}{12.000000}\selectfont False positive rate}%
\end{pgfscope}%
\begin{pgfscope}%
\pgfsetbuttcap%
\pgfsetroundjoin%
\definecolor{currentfill}{rgb}{0.000000,0.000000,0.000000}%
\pgfsetfillcolor{currentfill}%
\pgfsetlinewidth{0.803000pt}%
\definecolor{currentstroke}{rgb}{0.000000,0.000000,0.000000}%
\pgfsetstrokecolor{currentstroke}%
\pgfsetdash{}{0pt}%
\pgfsys@defobject{currentmarker}{\pgfqpoint{-0.048611in}{0.000000in}}{\pgfqpoint{-0.000000in}{0.000000in}}{%
\pgfpathmoveto{\pgfqpoint{-0.000000in}{0.000000in}}%
\pgfpathlineto{\pgfqpoint{-0.048611in}{0.000000in}}%
\pgfusepath{stroke,fill}%
}%
\begin{pgfscope}%
\pgfsys@transformshift{0.553581in}{0.551944in}%
\pgfsys@useobject{currentmarker}{}%
\end{pgfscope}%
\end{pgfscope}%
\begin{pgfscope}%
\definecolor{textcolor}{rgb}{0.000000,0.000000,0.000000}%
\pgfsetstrokecolor{textcolor}%
\pgfsetfillcolor{textcolor}%
\pgftext[x=0.278889in, y=0.503750in, left, base]{\color{textcolor}\rmfamily\fontsize{10.000000}{12.000000}\selectfont \(\displaystyle {0.0}\)}%
\end{pgfscope}%
\begin{pgfscope}%
\pgfsetbuttcap%
\pgfsetroundjoin%
\definecolor{currentfill}{rgb}{0.000000,0.000000,0.000000}%
\pgfsetfillcolor{currentfill}%
\pgfsetlinewidth{0.803000pt}%
\definecolor{currentstroke}{rgb}{0.000000,0.000000,0.000000}%
\pgfsetstrokecolor{currentstroke}%
\pgfsetdash{}{0pt}%
\pgfsys@defobject{currentmarker}{\pgfqpoint{-0.048611in}{0.000000in}}{\pgfqpoint{-0.000000in}{0.000000in}}{%
\pgfpathmoveto{\pgfqpoint{-0.000000in}{0.000000in}}%
\pgfpathlineto{\pgfqpoint{-0.048611in}{0.000000in}}%
\pgfusepath{stroke,fill}%
}%
\begin{pgfscope}%
\pgfsys@transformshift{0.553581in}{1.076944in}%
\pgfsys@useobject{currentmarker}{}%
\end{pgfscope}%
\end{pgfscope}%
\begin{pgfscope}%
\definecolor{textcolor}{rgb}{0.000000,0.000000,0.000000}%
\pgfsetstrokecolor{textcolor}%
\pgfsetfillcolor{textcolor}%
\pgftext[x=0.278889in, y=1.028750in, left, base]{\color{textcolor}\rmfamily\fontsize{10.000000}{12.000000}\selectfont \(\displaystyle {0.5}\)}%
\end{pgfscope}%
\begin{pgfscope}%
\pgfsetbuttcap%
\pgfsetroundjoin%
\definecolor{currentfill}{rgb}{0.000000,0.000000,0.000000}%
\pgfsetfillcolor{currentfill}%
\pgfsetlinewidth{0.803000pt}%
\definecolor{currentstroke}{rgb}{0.000000,0.000000,0.000000}%
\pgfsetstrokecolor{currentstroke}%
\pgfsetdash{}{0pt}%
\pgfsys@defobject{currentmarker}{\pgfqpoint{-0.048611in}{0.000000in}}{\pgfqpoint{-0.000000in}{0.000000in}}{%
\pgfpathmoveto{\pgfqpoint{-0.000000in}{0.000000in}}%
\pgfpathlineto{\pgfqpoint{-0.048611in}{0.000000in}}%
\pgfusepath{stroke,fill}%
}%
\begin{pgfscope}%
\pgfsys@transformshift{0.553581in}{1.601944in}%
\pgfsys@useobject{currentmarker}{}%
\end{pgfscope}%
\end{pgfscope}%
\begin{pgfscope}%
\definecolor{textcolor}{rgb}{0.000000,0.000000,0.000000}%
\pgfsetstrokecolor{textcolor}%
\pgfsetfillcolor{textcolor}%
\pgftext[x=0.278889in, y=1.553750in, left, base]{\color{textcolor}\rmfamily\fontsize{10.000000}{12.000000}\selectfont \(\displaystyle {1.0}\)}%
\end{pgfscope}%
\begin{pgfscope}%
\definecolor{textcolor}{rgb}{0.000000,0.000000,0.000000}%
\pgfsetstrokecolor{textcolor}%
\pgfsetfillcolor{textcolor}%
\pgftext[x=0.223333in,y=1.076944in,,bottom,rotate=90.000000]{\color{textcolor}\rmfamily\fontsize{10.000000}{12.000000}\selectfont True positive rate}%
\end{pgfscope}%
\begin{pgfscope}%
\pgfpathrectangle{\pgfqpoint{0.553581in}{0.499444in}}{\pgfqpoint{1.550000in}{1.155000in}}%
\pgfusepath{clip}%
\pgfsetbuttcap%
\pgfsetroundjoin%
\pgfsetlinewidth{1.505625pt}%
\definecolor{currentstroke}{rgb}{0.000000,0.000000,0.000000}%
\pgfsetstrokecolor{currentstroke}%
\pgfsetdash{{5.550000pt}{2.400000pt}}{0.000000pt}%
\pgfpathmoveto{\pgfqpoint{0.624035in}{0.551944in}}%
\pgfpathlineto{\pgfqpoint{2.033126in}{1.601944in}}%
\pgfusepath{stroke}%
\end{pgfscope}%
\begin{pgfscope}%
\pgfpathrectangle{\pgfqpoint{0.553581in}{0.499444in}}{\pgfqpoint{1.550000in}{1.155000in}}%
\pgfusepath{clip}%
\pgfsetrectcap%
\pgfsetroundjoin%
\pgfsetlinewidth{1.505625pt}%
\definecolor{currentstroke}{rgb}{0.000000,0.000000,0.000000}%
\pgfsetstrokecolor{currentstroke}%
\pgfsetdash{}{0pt}%
\pgfpathmoveto{\pgfqpoint{0.624035in}{0.551944in}}%
\pgfpathlineto{\pgfqpoint{0.625087in}{1.081068in}}%
\pgfpathlineto{\pgfqpoint{0.626766in}{1.222951in}}%
\pgfpathlineto{\pgfqpoint{0.630063in}{1.346143in}}%
\pgfpathlineto{\pgfqpoint{0.630170in}{1.346861in}}%
\pgfpathlineto{\pgfqpoint{0.633316in}{1.414483in}}%
\pgfpathlineto{\pgfqpoint{0.637112in}{1.467192in}}%
\pgfpathlineto{\pgfqpoint{0.642134in}{1.509564in}}%
\pgfpathlineto{\pgfqpoint{0.648493in}{1.541240in}}%
\pgfpathlineto{\pgfqpoint{0.648510in}{1.541293in}}%
\pgfpathlineto{\pgfqpoint{0.655596in}{1.563550in}}%
\pgfpathlineto{\pgfqpoint{0.660310in}{1.572583in}}%
\pgfpathlineto{\pgfqpoint{0.670656in}{1.585341in}}%
\pgfpathlineto{\pgfqpoint{0.682708in}{1.593071in}}%
\pgfpathlineto{\pgfqpoint{0.683120in}{1.593230in}}%
\pgfpathlineto{\pgfqpoint{0.697859in}{1.597421in}}%
\pgfpathlineto{\pgfqpoint{0.715458in}{1.600108in}}%
\pgfpathlineto{\pgfqpoint{0.747388in}{1.601439in}}%
\pgfpathlineto{\pgfqpoint{0.871277in}{1.601931in}}%
\pgfpathlineto{\pgfqpoint{2.033126in}{1.601944in}}%
\pgfpathlineto{\pgfqpoint{2.033126in}{1.601944in}}%
\pgfusepath{stroke}%
\end{pgfscope}%
\begin{pgfscope}%
\pgfsetrectcap%
\pgfsetmiterjoin%
\pgfsetlinewidth{0.803000pt}%
\definecolor{currentstroke}{rgb}{0.000000,0.000000,0.000000}%
\pgfsetstrokecolor{currentstroke}%
\pgfsetdash{}{0pt}%
\pgfpathmoveto{\pgfqpoint{0.553581in}{0.499444in}}%
\pgfpathlineto{\pgfqpoint{0.553581in}{1.654444in}}%
\pgfusepath{stroke}%
\end{pgfscope}%
\begin{pgfscope}%
\pgfsetrectcap%
\pgfsetmiterjoin%
\pgfsetlinewidth{0.803000pt}%
\definecolor{currentstroke}{rgb}{0.000000,0.000000,0.000000}%
\pgfsetstrokecolor{currentstroke}%
\pgfsetdash{}{0pt}%
\pgfpathmoveto{\pgfqpoint{2.103581in}{0.499444in}}%
\pgfpathlineto{\pgfqpoint{2.103581in}{1.654444in}}%
\pgfusepath{stroke}%
\end{pgfscope}%
\begin{pgfscope}%
\pgfsetrectcap%
\pgfsetmiterjoin%
\pgfsetlinewidth{0.803000pt}%
\definecolor{currentstroke}{rgb}{0.000000,0.000000,0.000000}%
\pgfsetstrokecolor{currentstroke}%
\pgfsetdash{}{0pt}%
\pgfpathmoveto{\pgfqpoint{0.553581in}{0.499444in}}%
\pgfpathlineto{\pgfqpoint{2.103581in}{0.499444in}}%
\pgfusepath{stroke}%
\end{pgfscope}%
\begin{pgfscope}%
\pgfsetrectcap%
\pgfsetmiterjoin%
\pgfsetlinewidth{0.803000pt}%
\definecolor{currentstroke}{rgb}{0.000000,0.000000,0.000000}%
\pgfsetstrokecolor{currentstroke}%
\pgfsetdash{}{0pt}%
\pgfpathmoveto{\pgfqpoint{0.553581in}{1.654444in}}%
\pgfpathlineto{\pgfqpoint{2.103581in}{1.654444in}}%
\pgfusepath{stroke}%
\end{pgfscope}%
\begin{pgfscope}%
\pgfsetbuttcap%
\pgfsetmiterjoin%
\definecolor{currentfill}{rgb}{1.000000,1.000000,1.000000}%
\pgfsetfillcolor{currentfill}%
\pgfsetfillopacity{0.800000}%
\pgfsetlinewidth{1.003750pt}%
\definecolor{currentstroke}{rgb}{0.800000,0.800000,0.800000}%
\pgfsetstrokecolor{currentstroke}%
\pgfsetstrokeopacity{0.800000}%
\pgfsetdash{}{0pt}%
\pgfpathmoveto{\pgfqpoint{0.832747in}{1.349722in}}%
\pgfpathlineto{\pgfqpoint{2.006358in}{1.349722in}}%
\pgfpathquadraticcurveto{\pgfqpoint{2.034136in}{1.349722in}}{\pgfqpoint{2.034136in}{1.377500in}}%
\pgfpathlineto{\pgfqpoint{2.034136in}{1.557222in}}%
\pgfpathquadraticcurveto{\pgfqpoint{2.034136in}{1.585000in}}{\pgfqpoint{2.006358in}{1.585000in}}%
\pgfpathlineto{\pgfqpoint{0.832747in}{1.585000in}}%
\pgfpathquadraticcurveto{\pgfqpoint{0.804970in}{1.585000in}}{\pgfqpoint{0.804970in}{1.557222in}}%
\pgfpathlineto{\pgfqpoint{0.804970in}{1.377500in}}%
\pgfpathquadraticcurveto{\pgfqpoint{0.804970in}{1.349722in}}{\pgfqpoint{0.832747in}{1.349722in}}%
\pgfpathlineto{\pgfqpoint{0.832747in}{1.349722in}}%
\pgfpathclose%
\pgfusepath{stroke,fill}%
\end{pgfscope}%
\begin{pgfscope}%
\pgfsetrectcap%
\pgfsetroundjoin%
\pgfsetlinewidth{1.505625pt}%
\definecolor{currentstroke}{rgb}{0.000000,0.000000,0.000000}%
\pgfsetstrokecolor{currentstroke}%
\pgfsetdash{}{0pt}%
\pgfpathmoveto{\pgfqpoint{0.860525in}{1.480833in}}%
\pgfpathlineto{\pgfqpoint{0.999414in}{1.480833in}}%
\pgfpathlineto{\pgfqpoint{1.138303in}{1.480833in}}%
\pgfusepath{stroke}%
\end{pgfscope}%
\begin{pgfscope}%
\definecolor{textcolor}{rgb}{0.000000,0.000000,0.000000}%
\pgfsetstrokecolor{textcolor}%
\pgfsetfillcolor{textcolor}%
\pgftext[x=1.249414in,y=1.432222in,left,base]{\color{textcolor}\rmfamily\fontsize{10.000000}{12.000000}\selectfont AUC=0.996}%
\end{pgfscope}%
\end{pgfpicture}%
\makeatother%
\endgroup%

	
\end{tabular}

Unfortunately, our test results do not look quite that nice.  They do not separate the two classes as well.  Some distributions are clustered to one side or in the middle.  Some models give the results in $p \in [0,1]$ rounded to two decimal places so that we cannot hope for a level of detail beyond that, and one algorithm, Bagging, gives $p$ rounded to only one decimal place.  

Let us look at some examples.  In all of them, AUC is in the range $[0.7,0.8]$, so the various models separate the positive and negative classes about equally well overall, with none being dramatically better or worse.  We will later show how we investigated which models do a better job in the ranges of interest.  

\

%
\verb|BRFC_Hard_Tomek_0_alpha_0_5_v1_Test|

\

This model does not separate the negative and positive classes as well as the ideal, giving a much lower AUC (area under the ROC curve).  These results are actually from the same model as the ideal above, but the ideal are the results on the training set and below on the test set, showing overfitting.  

In these results, the 100 most frequent values comprised 93\% of the results, meaning that, while there is some noise making the distribution look continuous, it is mostly discrete to two decimal places, so we cannot hope for fine detail in tuning the decision threshold.  

\noindent\begin{tabular}{@{\hspace{-6pt}}p{4.3in} @{\hspace{-6pt}}p{2.0in}}
	\vskip 0pt
	\hfil Raw Model Output
	
	%% Creator: Matplotlib, PGF backend
%%
%% To include the figure in your LaTeX document, write
%%   \input{<filename>.pgf}
%%
%% Make sure the required packages are loaded in your preamble
%%   \usepackage{pgf}
%%
%% Also ensure that all the required font packages are loaded; for instance,
%% the lmodern package is sometimes necessary when using math font.
%%   \usepackage{lmodern}
%%
%% Figures using additional raster images can only be included by \input if
%% they are in the same directory as the main LaTeX file. For loading figures
%% from other directories you can use the `import` package
%%   \usepackage{import}
%%
%% and then include the figures with
%%   \import{<path to file>}{<filename>.pgf}
%%
%% Matplotlib used the following preamble
%%   
%%   \usepackage{fontspec}
%%   \makeatletter\@ifpackageloaded{underscore}{}{\usepackage[strings]{underscore}}\makeatother
%%
\begingroup%
\makeatletter%
\begin{pgfpicture}%
\pgfpathrectangle{\pgfpointorigin}{\pgfqpoint{4.509306in}{1.754444in}}%
\pgfusepath{use as bounding box, clip}%
\begin{pgfscope}%
\pgfsetbuttcap%
\pgfsetmiterjoin%
\definecolor{currentfill}{rgb}{1.000000,1.000000,1.000000}%
\pgfsetfillcolor{currentfill}%
\pgfsetlinewidth{0.000000pt}%
\definecolor{currentstroke}{rgb}{1.000000,1.000000,1.000000}%
\pgfsetstrokecolor{currentstroke}%
\pgfsetdash{}{0pt}%
\pgfpathmoveto{\pgfqpoint{0.000000in}{0.000000in}}%
\pgfpathlineto{\pgfqpoint{4.509306in}{0.000000in}}%
\pgfpathlineto{\pgfqpoint{4.509306in}{1.754444in}}%
\pgfpathlineto{\pgfqpoint{0.000000in}{1.754444in}}%
\pgfpathlineto{\pgfqpoint{0.000000in}{0.000000in}}%
\pgfpathclose%
\pgfusepath{fill}%
\end{pgfscope}%
\begin{pgfscope}%
\pgfsetbuttcap%
\pgfsetmiterjoin%
\definecolor{currentfill}{rgb}{1.000000,1.000000,1.000000}%
\pgfsetfillcolor{currentfill}%
\pgfsetlinewidth{0.000000pt}%
\definecolor{currentstroke}{rgb}{0.000000,0.000000,0.000000}%
\pgfsetstrokecolor{currentstroke}%
\pgfsetstrokeopacity{0.000000}%
\pgfsetdash{}{0pt}%
\pgfpathmoveto{\pgfqpoint{0.445556in}{0.499444in}}%
\pgfpathlineto{\pgfqpoint{4.320556in}{0.499444in}}%
\pgfpathlineto{\pgfqpoint{4.320556in}{1.654444in}}%
\pgfpathlineto{\pgfqpoint{0.445556in}{1.654444in}}%
\pgfpathlineto{\pgfqpoint{0.445556in}{0.499444in}}%
\pgfpathclose%
\pgfusepath{fill}%
\end{pgfscope}%
\begin{pgfscope}%
\pgfpathrectangle{\pgfqpoint{0.445556in}{0.499444in}}{\pgfqpoint{3.875000in}{1.155000in}}%
\pgfusepath{clip}%
\pgfsetbuttcap%
\pgfsetmiterjoin%
\pgfsetlinewidth{1.003750pt}%
\definecolor{currentstroke}{rgb}{0.000000,0.000000,0.000000}%
\pgfsetstrokecolor{currentstroke}%
\pgfsetdash{}{0pt}%
\pgfpathmoveto{\pgfqpoint{0.435556in}{0.499444in}}%
\pgfpathlineto{\pgfqpoint{0.483922in}{0.499444in}}%
\pgfpathlineto{\pgfqpoint{0.483922in}{0.618288in}}%
\pgfpathlineto{\pgfqpoint{0.435556in}{0.618288in}}%
\pgfusepath{stroke}%
\end{pgfscope}%
\begin{pgfscope}%
\pgfpathrectangle{\pgfqpoint{0.445556in}{0.499444in}}{\pgfqpoint{3.875000in}{1.155000in}}%
\pgfusepath{clip}%
\pgfsetbuttcap%
\pgfsetmiterjoin%
\pgfsetlinewidth{1.003750pt}%
\definecolor{currentstroke}{rgb}{0.000000,0.000000,0.000000}%
\pgfsetstrokecolor{currentstroke}%
\pgfsetdash{}{0pt}%
\pgfpathmoveto{\pgfqpoint{0.576001in}{0.499444in}}%
\pgfpathlineto{\pgfqpoint{0.637387in}{0.499444in}}%
\pgfpathlineto{\pgfqpoint{0.637387in}{0.801196in}}%
\pgfpathlineto{\pgfqpoint{0.576001in}{0.801196in}}%
\pgfpathlineto{\pgfqpoint{0.576001in}{0.499444in}}%
\pgfpathclose%
\pgfusepath{stroke}%
\end{pgfscope}%
\begin{pgfscope}%
\pgfpathrectangle{\pgfqpoint{0.445556in}{0.499444in}}{\pgfqpoint{3.875000in}{1.155000in}}%
\pgfusepath{clip}%
\pgfsetbuttcap%
\pgfsetmiterjoin%
\pgfsetlinewidth{1.003750pt}%
\definecolor{currentstroke}{rgb}{0.000000,0.000000,0.000000}%
\pgfsetstrokecolor{currentstroke}%
\pgfsetdash{}{0pt}%
\pgfpathmoveto{\pgfqpoint{0.729467in}{0.499444in}}%
\pgfpathlineto{\pgfqpoint{0.790853in}{0.499444in}}%
\pgfpathlineto{\pgfqpoint{0.790853in}{1.026194in}}%
\pgfpathlineto{\pgfqpoint{0.729467in}{1.026194in}}%
\pgfpathlineto{\pgfqpoint{0.729467in}{0.499444in}}%
\pgfpathclose%
\pgfusepath{stroke}%
\end{pgfscope}%
\begin{pgfscope}%
\pgfpathrectangle{\pgfqpoint{0.445556in}{0.499444in}}{\pgfqpoint{3.875000in}{1.155000in}}%
\pgfusepath{clip}%
\pgfsetbuttcap%
\pgfsetmiterjoin%
\pgfsetlinewidth{1.003750pt}%
\definecolor{currentstroke}{rgb}{0.000000,0.000000,0.000000}%
\pgfsetstrokecolor{currentstroke}%
\pgfsetdash{}{0pt}%
\pgfpathmoveto{\pgfqpoint{0.882932in}{0.499444in}}%
\pgfpathlineto{\pgfqpoint{0.944318in}{0.499444in}}%
\pgfpathlineto{\pgfqpoint{0.944318in}{1.203686in}}%
\pgfpathlineto{\pgfqpoint{0.882932in}{1.203686in}}%
\pgfpathlineto{\pgfqpoint{0.882932in}{0.499444in}}%
\pgfpathclose%
\pgfusepath{stroke}%
\end{pgfscope}%
\begin{pgfscope}%
\pgfpathrectangle{\pgfqpoint{0.445556in}{0.499444in}}{\pgfqpoint{3.875000in}{1.155000in}}%
\pgfusepath{clip}%
\pgfsetbuttcap%
\pgfsetmiterjoin%
\pgfsetlinewidth{1.003750pt}%
\definecolor{currentstroke}{rgb}{0.000000,0.000000,0.000000}%
\pgfsetstrokecolor{currentstroke}%
\pgfsetdash{}{0pt}%
\pgfpathmoveto{\pgfqpoint{1.036397in}{0.499444in}}%
\pgfpathlineto{\pgfqpoint{1.097783in}{0.499444in}}%
\pgfpathlineto{\pgfqpoint{1.097783in}{1.348758in}}%
\pgfpathlineto{\pgfqpoint{1.036397in}{1.348758in}}%
\pgfpathlineto{\pgfqpoint{1.036397in}{0.499444in}}%
\pgfpathclose%
\pgfusepath{stroke}%
\end{pgfscope}%
\begin{pgfscope}%
\pgfpathrectangle{\pgfqpoint{0.445556in}{0.499444in}}{\pgfqpoint{3.875000in}{1.155000in}}%
\pgfusepath{clip}%
\pgfsetbuttcap%
\pgfsetmiterjoin%
\pgfsetlinewidth{1.003750pt}%
\definecolor{currentstroke}{rgb}{0.000000,0.000000,0.000000}%
\pgfsetstrokecolor{currentstroke}%
\pgfsetdash{}{0pt}%
\pgfpathmoveto{\pgfqpoint{1.189863in}{0.499444in}}%
\pgfpathlineto{\pgfqpoint{1.251249in}{0.499444in}}%
\pgfpathlineto{\pgfqpoint{1.251249in}{1.455609in}}%
\pgfpathlineto{\pgfqpoint{1.189863in}{1.455609in}}%
\pgfpathlineto{\pgfqpoint{1.189863in}{0.499444in}}%
\pgfpathclose%
\pgfusepath{stroke}%
\end{pgfscope}%
\begin{pgfscope}%
\pgfpathrectangle{\pgfqpoint{0.445556in}{0.499444in}}{\pgfqpoint{3.875000in}{1.155000in}}%
\pgfusepath{clip}%
\pgfsetbuttcap%
\pgfsetmiterjoin%
\pgfsetlinewidth{1.003750pt}%
\definecolor{currentstroke}{rgb}{0.000000,0.000000,0.000000}%
\pgfsetstrokecolor{currentstroke}%
\pgfsetdash{}{0pt}%
\pgfpathmoveto{\pgfqpoint{1.343328in}{0.499444in}}%
\pgfpathlineto{\pgfqpoint{1.404714in}{0.499444in}}%
\pgfpathlineto{\pgfqpoint{1.404714in}{1.549075in}}%
\pgfpathlineto{\pgfqpoint{1.343328in}{1.549075in}}%
\pgfpathlineto{\pgfqpoint{1.343328in}{0.499444in}}%
\pgfpathclose%
\pgfusepath{stroke}%
\end{pgfscope}%
\begin{pgfscope}%
\pgfpathrectangle{\pgfqpoint{0.445556in}{0.499444in}}{\pgfqpoint{3.875000in}{1.155000in}}%
\pgfusepath{clip}%
\pgfsetbuttcap%
\pgfsetmiterjoin%
\pgfsetlinewidth{1.003750pt}%
\definecolor{currentstroke}{rgb}{0.000000,0.000000,0.000000}%
\pgfsetstrokecolor{currentstroke}%
\pgfsetdash{}{0pt}%
\pgfpathmoveto{\pgfqpoint{1.496793in}{0.499444in}}%
\pgfpathlineto{\pgfqpoint{1.558179in}{0.499444in}}%
\pgfpathlineto{\pgfqpoint{1.558179in}{1.599444in}}%
\pgfpathlineto{\pgfqpoint{1.496793in}{1.599444in}}%
\pgfpathlineto{\pgfqpoint{1.496793in}{0.499444in}}%
\pgfpathclose%
\pgfusepath{stroke}%
\end{pgfscope}%
\begin{pgfscope}%
\pgfpathrectangle{\pgfqpoint{0.445556in}{0.499444in}}{\pgfqpoint{3.875000in}{1.155000in}}%
\pgfusepath{clip}%
\pgfsetbuttcap%
\pgfsetmiterjoin%
\pgfsetlinewidth{1.003750pt}%
\definecolor{currentstroke}{rgb}{0.000000,0.000000,0.000000}%
\pgfsetstrokecolor{currentstroke}%
\pgfsetdash{}{0pt}%
\pgfpathmoveto{\pgfqpoint{1.650259in}{0.499444in}}%
\pgfpathlineto{\pgfqpoint{1.711645in}{0.499444in}}%
\pgfpathlineto{\pgfqpoint{1.711645in}{1.597742in}}%
\pgfpathlineto{\pgfqpoint{1.650259in}{1.597742in}}%
\pgfpathlineto{\pgfqpoint{1.650259in}{0.499444in}}%
\pgfpathclose%
\pgfusepath{stroke}%
\end{pgfscope}%
\begin{pgfscope}%
\pgfpathrectangle{\pgfqpoint{0.445556in}{0.499444in}}{\pgfqpoint{3.875000in}{1.155000in}}%
\pgfusepath{clip}%
\pgfsetbuttcap%
\pgfsetmiterjoin%
\pgfsetlinewidth{1.003750pt}%
\definecolor{currentstroke}{rgb}{0.000000,0.000000,0.000000}%
\pgfsetstrokecolor{currentstroke}%
\pgfsetdash{}{0pt}%
\pgfpathmoveto{\pgfqpoint{1.803724in}{0.499444in}}%
\pgfpathlineto{\pgfqpoint{1.865110in}{0.499444in}}%
\pgfpathlineto{\pgfqpoint{1.865110in}{1.571977in}}%
\pgfpathlineto{\pgfqpoint{1.803724in}{1.571977in}}%
\pgfpathlineto{\pgfqpoint{1.803724in}{0.499444in}}%
\pgfpathclose%
\pgfusepath{stroke}%
\end{pgfscope}%
\begin{pgfscope}%
\pgfpathrectangle{\pgfqpoint{0.445556in}{0.499444in}}{\pgfqpoint{3.875000in}{1.155000in}}%
\pgfusepath{clip}%
\pgfsetbuttcap%
\pgfsetmiterjoin%
\pgfsetlinewidth{1.003750pt}%
\definecolor{currentstroke}{rgb}{0.000000,0.000000,0.000000}%
\pgfsetstrokecolor{currentstroke}%
\pgfsetdash{}{0pt}%
\pgfpathmoveto{\pgfqpoint{1.957189in}{0.499444in}}%
\pgfpathlineto{\pgfqpoint{2.018575in}{0.499444in}}%
\pgfpathlineto{\pgfqpoint{2.018575in}{1.535535in}}%
\pgfpathlineto{\pgfqpoint{1.957189in}{1.535535in}}%
\pgfpathlineto{\pgfqpoint{1.957189in}{0.499444in}}%
\pgfpathclose%
\pgfusepath{stroke}%
\end{pgfscope}%
\begin{pgfscope}%
\pgfpathrectangle{\pgfqpoint{0.445556in}{0.499444in}}{\pgfqpoint{3.875000in}{1.155000in}}%
\pgfusepath{clip}%
\pgfsetbuttcap%
\pgfsetmiterjoin%
\pgfsetlinewidth{1.003750pt}%
\definecolor{currentstroke}{rgb}{0.000000,0.000000,0.000000}%
\pgfsetstrokecolor{currentstroke}%
\pgfsetdash{}{0pt}%
\pgfpathmoveto{\pgfqpoint{2.110655in}{0.499444in}}%
\pgfpathlineto{\pgfqpoint{2.172041in}{0.499444in}}%
\pgfpathlineto{\pgfqpoint{2.172041in}{1.450812in}}%
\pgfpathlineto{\pgfqpoint{2.110655in}{1.450812in}}%
\pgfpathlineto{\pgfqpoint{2.110655in}{0.499444in}}%
\pgfpathclose%
\pgfusepath{stroke}%
\end{pgfscope}%
\begin{pgfscope}%
\pgfpathrectangle{\pgfqpoint{0.445556in}{0.499444in}}{\pgfqpoint{3.875000in}{1.155000in}}%
\pgfusepath{clip}%
\pgfsetbuttcap%
\pgfsetmiterjoin%
\pgfsetlinewidth{1.003750pt}%
\definecolor{currentstroke}{rgb}{0.000000,0.000000,0.000000}%
\pgfsetstrokecolor{currentstroke}%
\pgfsetdash{}{0pt}%
\pgfpathmoveto{\pgfqpoint{2.264120in}{0.499444in}}%
\pgfpathlineto{\pgfqpoint{2.325506in}{0.499444in}}%
\pgfpathlineto{\pgfqpoint{2.325506in}{1.379398in}}%
\pgfpathlineto{\pgfqpoint{2.264120in}{1.379398in}}%
\pgfpathlineto{\pgfqpoint{2.264120in}{0.499444in}}%
\pgfpathclose%
\pgfusepath{stroke}%
\end{pgfscope}%
\begin{pgfscope}%
\pgfpathrectangle{\pgfqpoint{0.445556in}{0.499444in}}{\pgfqpoint{3.875000in}{1.155000in}}%
\pgfusepath{clip}%
\pgfsetbuttcap%
\pgfsetmiterjoin%
\pgfsetlinewidth{1.003750pt}%
\definecolor{currentstroke}{rgb}{0.000000,0.000000,0.000000}%
\pgfsetstrokecolor{currentstroke}%
\pgfsetdash{}{0pt}%
\pgfpathmoveto{\pgfqpoint{2.417585in}{0.499444in}}%
\pgfpathlineto{\pgfqpoint{2.478972in}{0.499444in}}%
\pgfpathlineto{\pgfqpoint{2.478972in}{1.248329in}}%
\pgfpathlineto{\pgfqpoint{2.417585in}{1.248329in}}%
\pgfpathlineto{\pgfqpoint{2.417585in}{0.499444in}}%
\pgfpathclose%
\pgfusepath{stroke}%
\end{pgfscope}%
\begin{pgfscope}%
\pgfpathrectangle{\pgfqpoint{0.445556in}{0.499444in}}{\pgfqpoint{3.875000in}{1.155000in}}%
\pgfusepath{clip}%
\pgfsetbuttcap%
\pgfsetmiterjoin%
\pgfsetlinewidth{1.003750pt}%
\definecolor{currentstroke}{rgb}{0.000000,0.000000,0.000000}%
\pgfsetstrokecolor{currentstroke}%
\pgfsetdash{}{0pt}%
\pgfpathmoveto{\pgfqpoint{2.571051in}{0.499444in}}%
\pgfpathlineto{\pgfqpoint{2.632437in}{0.499444in}}%
\pgfpathlineto{\pgfqpoint{2.632437in}{1.129640in}}%
\pgfpathlineto{\pgfqpoint{2.571051in}{1.129640in}}%
\pgfpathlineto{\pgfqpoint{2.571051in}{0.499444in}}%
\pgfpathclose%
\pgfusepath{stroke}%
\end{pgfscope}%
\begin{pgfscope}%
\pgfpathrectangle{\pgfqpoint{0.445556in}{0.499444in}}{\pgfqpoint{3.875000in}{1.155000in}}%
\pgfusepath{clip}%
\pgfsetbuttcap%
\pgfsetmiterjoin%
\pgfsetlinewidth{1.003750pt}%
\definecolor{currentstroke}{rgb}{0.000000,0.000000,0.000000}%
\pgfsetstrokecolor{currentstroke}%
\pgfsetdash{}{0pt}%
\pgfpathmoveto{\pgfqpoint{2.724516in}{0.499444in}}%
\pgfpathlineto{\pgfqpoint{2.785902in}{0.499444in}}%
\pgfpathlineto{\pgfqpoint{2.785902in}{1.011880in}}%
\pgfpathlineto{\pgfqpoint{2.724516in}{1.011880in}}%
\pgfpathlineto{\pgfqpoint{2.724516in}{0.499444in}}%
\pgfpathclose%
\pgfusepath{stroke}%
\end{pgfscope}%
\begin{pgfscope}%
\pgfpathrectangle{\pgfqpoint{0.445556in}{0.499444in}}{\pgfqpoint{3.875000in}{1.155000in}}%
\pgfusepath{clip}%
\pgfsetbuttcap%
\pgfsetmiterjoin%
\pgfsetlinewidth{1.003750pt}%
\definecolor{currentstroke}{rgb}{0.000000,0.000000,0.000000}%
\pgfsetstrokecolor{currentstroke}%
\pgfsetdash{}{0pt}%
\pgfpathmoveto{\pgfqpoint{2.877981in}{0.499444in}}%
\pgfpathlineto{\pgfqpoint{2.939368in}{0.499444in}}%
\pgfpathlineto{\pgfqpoint{2.939368in}{0.912921in}}%
\pgfpathlineto{\pgfqpoint{2.877981in}{0.912921in}}%
\pgfpathlineto{\pgfqpoint{2.877981in}{0.499444in}}%
\pgfpathclose%
\pgfusepath{stroke}%
\end{pgfscope}%
\begin{pgfscope}%
\pgfpathrectangle{\pgfqpoint{0.445556in}{0.499444in}}{\pgfqpoint{3.875000in}{1.155000in}}%
\pgfusepath{clip}%
\pgfsetbuttcap%
\pgfsetmiterjoin%
\pgfsetlinewidth{1.003750pt}%
\definecolor{currentstroke}{rgb}{0.000000,0.000000,0.000000}%
\pgfsetstrokecolor{currentstroke}%
\pgfsetdash{}{0pt}%
\pgfpathmoveto{\pgfqpoint{3.031447in}{0.499444in}}%
\pgfpathlineto{\pgfqpoint{3.092833in}{0.499444in}}%
\pgfpathlineto{\pgfqpoint{3.092833in}{0.805374in}}%
\pgfpathlineto{\pgfqpoint{3.031447in}{0.805374in}}%
\pgfpathlineto{\pgfqpoint{3.031447in}{0.499444in}}%
\pgfpathclose%
\pgfusepath{stroke}%
\end{pgfscope}%
\begin{pgfscope}%
\pgfpathrectangle{\pgfqpoint{0.445556in}{0.499444in}}{\pgfqpoint{3.875000in}{1.155000in}}%
\pgfusepath{clip}%
\pgfsetbuttcap%
\pgfsetmiterjoin%
\pgfsetlinewidth{1.003750pt}%
\definecolor{currentstroke}{rgb}{0.000000,0.000000,0.000000}%
\pgfsetstrokecolor{currentstroke}%
\pgfsetdash{}{0pt}%
\pgfpathmoveto{\pgfqpoint{3.184912in}{0.499444in}}%
\pgfpathlineto{\pgfqpoint{3.246298in}{0.499444in}}%
\pgfpathlineto{\pgfqpoint{3.246298in}{0.730710in}}%
\pgfpathlineto{\pgfqpoint{3.184912in}{0.730710in}}%
\pgfpathlineto{\pgfqpoint{3.184912in}{0.499444in}}%
\pgfpathclose%
\pgfusepath{stroke}%
\end{pgfscope}%
\begin{pgfscope}%
\pgfpathrectangle{\pgfqpoint{0.445556in}{0.499444in}}{\pgfqpoint{3.875000in}{1.155000in}}%
\pgfusepath{clip}%
\pgfsetbuttcap%
\pgfsetmiterjoin%
\pgfsetlinewidth{1.003750pt}%
\definecolor{currentstroke}{rgb}{0.000000,0.000000,0.000000}%
\pgfsetstrokecolor{currentstroke}%
\pgfsetdash{}{0pt}%
\pgfpathmoveto{\pgfqpoint{3.338377in}{0.499444in}}%
\pgfpathlineto{\pgfqpoint{3.399764in}{0.499444in}}%
\pgfpathlineto{\pgfqpoint{3.399764in}{0.672603in}}%
\pgfpathlineto{\pgfqpoint{3.338377in}{0.672603in}}%
\pgfpathlineto{\pgfqpoint{3.338377in}{0.499444in}}%
\pgfpathclose%
\pgfusepath{stroke}%
\end{pgfscope}%
\begin{pgfscope}%
\pgfpathrectangle{\pgfqpoint{0.445556in}{0.499444in}}{\pgfqpoint{3.875000in}{1.155000in}}%
\pgfusepath{clip}%
\pgfsetbuttcap%
\pgfsetmiterjoin%
\pgfsetlinewidth{1.003750pt}%
\definecolor{currentstroke}{rgb}{0.000000,0.000000,0.000000}%
\pgfsetstrokecolor{currentstroke}%
\pgfsetdash{}{0pt}%
\pgfpathmoveto{\pgfqpoint{3.491843in}{0.499444in}}%
\pgfpathlineto{\pgfqpoint{3.553229in}{0.499444in}}%
\pgfpathlineto{\pgfqpoint{3.553229in}{0.616663in}}%
\pgfpathlineto{\pgfqpoint{3.491843in}{0.616663in}}%
\pgfpathlineto{\pgfqpoint{3.491843in}{0.499444in}}%
\pgfpathclose%
\pgfusepath{stroke}%
\end{pgfscope}%
\begin{pgfscope}%
\pgfpathrectangle{\pgfqpoint{0.445556in}{0.499444in}}{\pgfqpoint{3.875000in}{1.155000in}}%
\pgfusepath{clip}%
\pgfsetbuttcap%
\pgfsetmiterjoin%
\pgfsetlinewidth{1.003750pt}%
\definecolor{currentstroke}{rgb}{0.000000,0.000000,0.000000}%
\pgfsetstrokecolor{currentstroke}%
\pgfsetdash{}{0pt}%
\pgfpathmoveto{\pgfqpoint{3.645308in}{0.499444in}}%
\pgfpathlineto{\pgfqpoint{3.706694in}{0.499444in}}%
\pgfpathlineto{\pgfqpoint{3.706694in}{0.574031in}}%
\pgfpathlineto{\pgfqpoint{3.645308in}{0.574031in}}%
\pgfpathlineto{\pgfqpoint{3.645308in}{0.499444in}}%
\pgfpathclose%
\pgfusepath{stroke}%
\end{pgfscope}%
\begin{pgfscope}%
\pgfpathrectangle{\pgfqpoint{0.445556in}{0.499444in}}{\pgfqpoint{3.875000in}{1.155000in}}%
\pgfusepath{clip}%
\pgfsetbuttcap%
\pgfsetmiterjoin%
\pgfsetlinewidth{1.003750pt}%
\definecolor{currentstroke}{rgb}{0.000000,0.000000,0.000000}%
\pgfsetstrokecolor{currentstroke}%
\pgfsetdash{}{0pt}%
\pgfpathmoveto{\pgfqpoint{3.798774in}{0.499444in}}%
\pgfpathlineto{\pgfqpoint{3.860160in}{0.499444in}}%
\pgfpathlineto{\pgfqpoint{3.860160in}{0.550587in}}%
\pgfpathlineto{\pgfqpoint{3.798774in}{0.550587in}}%
\pgfpathlineto{\pgfqpoint{3.798774in}{0.499444in}}%
\pgfpathclose%
\pgfusepath{stroke}%
\end{pgfscope}%
\begin{pgfscope}%
\pgfpathrectangle{\pgfqpoint{0.445556in}{0.499444in}}{\pgfqpoint{3.875000in}{1.155000in}}%
\pgfusepath{clip}%
\pgfsetbuttcap%
\pgfsetmiterjoin%
\pgfsetlinewidth{1.003750pt}%
\definecolor{currentstroke}{rgb}{0.000000,0.000000,0.000000}%
\pgfsetstrokecolor{currentstroke}%
\pgfsetdash{}{0pt}%
\pgfpathmoveto{\pgfqpoint{3.952239in}{0.499444in}}%
\pgfpathlineto{\pgfqpoint{4.013625in}{0.499444in}}%
\pgfpathlineto{\pgfqpoint{4.013625in}{0.526679in}}%
\pgfpathlineto{\pgfqpoint{3.952239in}{0.526679in}}%
\pgfpathlineto{\pgfqpoint{3.952239in}{0.499444in}}%
\pgfpathclose%
\pgfusepath{stroke}%
\end{pgfscope}%
\begin{pgfscope}%
\pgfpathrectangle{\pgfqpoint{0.445556in}{0.499444in}}{\pgfqpoint{3.875000in}{1.155000in}}%
\pgfusepath{clip}%
\pgfsetbuttcap%
\pgfsetmiterjoin%
\pgfsetlinewidth{1.003750pt}%
\definecolor{currentstroke}{rgb}{0.000000,0.000000,0.000000}%
\pgfsetstrokecolor{currentstroke}%
\pgfsetdash{}{0pt}%
\pgfpathmoveto{\pgfqpoint{4.105704in}{0.499444in}}%
\pgfpathlineto{\pgfqpoint{4.167090in}{0.499444in}}%
\pgfpathlineto{\pgfqpoint{4.167090in}{0.506640in}}%
\pgfpathlineto{\pgfqpoint{4.105704in}{0.506640in}}%
\pgfpathlineto{\pgfqpoint{4.105704in}{0.499444in}}%
\pgfpathclose%
\pgfusepath{stroke}%
\end{pgfscope}%
\begin{pgfscope}%
\pgfpathrectangle{\pgfqpoint{0.445556in}{0.499444in}}{\pgfqpoint{3.875000in}{1.155000in}}%
\pgfusepath{clip}%
\pgfsetbuttcap%
\pgfsetmiterjoin%
\definecolor{currentfill}{rgb}{0.000000,0.000000,0.000000}%
\pgfsetfillcolor{currentfill}%
\pgfsetlinewidth{0.000000pt}%
\definecolor{currentstroke}{rgb}{0.000000,0.000000,0.000000}%
\pgfsetstrokecolor{currentstroke}%
\pgfsetstrokeopacity{0.000000}%
\pgfsetdash{}{0pt}%
\pgfpathmoveto{\pgfqpoint{0.483922in}{0.499444in}}%
\pgfpathlineto{\pgfqpoint{0.545308in}{0.499444in}}%
\pgfpathlineto{\pgfqpoint{0.545308in}{0.500527in}}%
\pgfpathlineto{\pgfqpoint{0.483922in}{0.500527in}}%
\pgfpathlineto{\pgfqpoint{0.483922in}{0.499444in}}%
\pgfpathclose%
\pgfusepath{fill}%
\end{pgfscope}%
\begin{pgfscope}%
\pgfpathrectangle{\pgfqpoint{0.445556in}{0.499444in}}{\pgfqpoint{3.875000in}{1.155000in}}%
\pgfusepath{clip}%
\pgfsetbuttcap%
\pgfsetmiterjoin%
\definecolor{currentfill}{rgb}{0.000000,0.000000,0.000000}%
\pgfsetfillcolor{currentfill}%
\pgfsetlinewidth{0.000000pt}%
\definecolor{currentstroke}{rgb}{0.000000,0.000000,0.000000}%
\pgfsetstrokecolor{currentstroke}%
\pgfsetstrokeopacity{0.000000}%
\pgfsetdash{}{0pt}%
\pgfpathmoveto{\pgfqpoint{0.637387in}{0.499444in}}%
\pgfpathlineto{\pgfqpoint{0.698774in}{0.499444in}}%
\pgfpathlineto{\pgfqpoint{0.698774in}{0.502230in}}%
\pgfpathlineto{\pgfqpoint{0.637387in}{0.502230in}}%
\pgfpathlineto{\pgfqpoint{0.637387in}{0.499444in}}%
\pgfpathclose%
\pgfusepath{fill}%
\end{pgfscope}%
\begin{pgfscope}%
\pgfpathrectangle{\pgfqpoint{0.445556in}{0.499444in}}{\pgfqpoint{3.875000in}{1.155000in}}%
\pgfusepath{clip}%
\pgfsetbuttcap%
\pgfsetmiterjoin%
\definecolor{currentfill}{rgb}{0.000000,0.000000,0.000000}%
\pgfsetfillcolor{currentfill}%
\pgfsetlinewidth{0.000000pt}%
\definecolor{currentstroke}{rgb}{0.000000,0.000000,0.000000}%
\pgfsetstrokecolor{currentstroke}%
\pgfsetstrokeopacity{0.000000}%
\pgfsetdash{}{0pt}%
\pgfpathmoveto{\pgfqpoint{0.790853in}{0.499444in}}%
\pgfpathlineto{\pgfqpoint{0.852239in}{0.499444in}}%
\pgfpathlineto{\pgfqpoint{0.852239in}{0.506330in}}%
\pgfpathlineto{\pgfqpoint{0.790853in}{0.506330in}}%
\pgfpathlineto{\pgfqpoint{0.790853in}{0.499444in}}%
\pgfpathclose%
\pgfusepath{fill}%
\end{pgfscope}%
\begin{pgfscope}%
\pgfpathrectangle{\pgfqpoint{0.445556in}{0.499444in}}{\pgfqpoint{3.875000in}{1.155000in}}%
\pgfusepath{clip}%
\pgfsetbuttcap%
\pgfsetmiterjoin%
\definecolor{currentfill}{rgb}{0.000000,0.000000,0.000000}%
\pgfsetfillcolor{currentfill}%
\pgfsetlinewidth{0.000000pt}%
\definecolor{currentstroke}{rgb}{0.000000,0.000000,0.000000}%
\pgfsetstrokecolor{currentstroke}%
\pgfsetstrokeopacity{0.000000}%
\pgfsetdash{}{0pt}%
\pgfpathmoveto{\pgfqpoint{0.944318in}{0.499444in}}%
\pgfpathlineto{\pgfqpoint{1.005704in}{0.499444in}}%
\pgfpathlineto{\pgfqpoint{1.005704in}{0.514068in}}%
\pgfpathlineto{\pgfqpoint{0.944318in}{0.514068in}}%
\pgfpathlineto{\pgfqpoint{0.944318in}{0.499444in}}%
\pgfpathclose%
\pgfusepath{fill}%
\end{pgfscope}%
\begin{pgfscope}%
\pgfpathrectangle{\pgfqpoint{0.445556in}{0.499444in}}{\pgfqpoint{3.875000in}{1.155000in}}%
\pgfusepath{clip}%
\pgfsetbuttcap%
\pgfsetmiterjoin%
\definecolor{currentfill}{rgb}{0.000000,0.000000,0.000000}%
\pgfsetfillcolor{currentfill}%
\pgfsetlinewidth{0.000000pt}%
\definecolor{currentstroke}{rgb}{0.000000,0.000000,0.000000}%
\pgfsetstrokecolor{currentstroke}%
\pgfsetstrokeopacity{0.000000}%
\pgfsetdash{}{0pt}%
\pgfpathmoveto{\pgfqpoint{1.097783in}{0.499444in}}%
\pgfpathlineto{\pgfqpoint{1.159170in}{0.499444in}}%
\pgfpathlineto{\pgfqpoint{1.159170in}{0.524977in}}%
\pgfpathlineto{\pgfqpoint{1.097783in}{0.524977in}}%
\pgfpathlineto{\pgfqpoint{1.097783in}{0.499444in}}%
\pgfpathclose%
\pgfusepath{fill}%
\end{pgfscope}%
\begin{pgfscope}%
\pgfpathrectangle{\pgfqpoint{0.445556in}{0.499444in}}{\pgfqpoint{3.875000in}{1.155000in}}%
\pgfusepath{clip}%
\pgfsetbuttcap%
\pgfsetmiterjoin%
\definecolor{currentfill}{rgb}{0.000000,0.000000,0.000000}%
\pgfsetfillcolor{currentfill}%
\pgfsetlinewidth{0.000000pt}%
\definecolor{currentstroke}{rgb}{0.000000,0.000000,0.000000}%
\pgfsetstrokecolor{currentstroke}%
\pgfsetstrokeopacity{0.000000}%
\pgfsetdash{}{0pt}%
\pgfpathmoveto{\pgfqpoint{1.251249in}{0.499444in}}%
\pgfpathlineto{\pgfqpoint{1.312635in}{0.499444in}}%
\pgfpathlineto{\pgfqpoint{1.312635in}{0.535267in}}%
\pgfpathlineto{\pgfqpoint{1.251249in}{0.535267in}}%
\pgfpathlineto{\pgfqpoint{1.251249in}{0.499444in}}%
\pgfpathclose%
\pgfusepath{fill}%
\end{pgfscope}%
\begin{pgfscope}%
\pgfpathrectangle{\pgfqpoint{0.445556in}{0.499444in}}{\pgfqpoint{3.875000in}{1.155000in}}%
\pgfusepath{clip}%
\pgfsetbuttcap%
\pgfsetmiterjoin%
\definecolor{currentfill}{rgb}{0.000000,0.000000,0.000000}%
\pgfsetfillcolor{currentfill}%
\pgfsetlinewidth{0.000000pt}%
\definecolor{currentstroke}{rgb}{0.000000,0.000000,0.000000}%
\pgfsetstrokecolor{currentstroke}%
\pgfsetstrokeopacity{0.000000}%
\pgfsetdash{}{0pt}%
\pgfpathmoveto{\pgfqpoint{1.404714in}{0.499444in}}%
\pgfpathlineto{\pgfqpoint{1.466100in}{0.499444in}}%
\pgfpathlineto{\pgfqpoint{1.466100in}{0.549659in}}%
\pgfpathlineto{\pgfqpoint{1.404714in}{0.549659in}}%
\pgfpathlineto{\pgfqpoint{1.404714in}{0.499444in}}%
\pgfpathclose%
\pgfusepath{fill}%
\end{pgfscope}%
\begin{pgfscope}%
\pgfpathrectangle{\pgfqpoint{0.445556in}{0.499444in}}{\pgfqpoint{3.875000in}{1.155000in}}%
\pgfusepath{clip}%
\pgfsetbuttcap%
\pgfsetmiterjoin%
\definecolor{currentfill}{rgb}{0.000000,0.000000,0.000000}%
\pgfsetfillcolor{currentfill}%
\pgfsetlinewidth{0.000000pt}%
\definecolor{currentstroke}{rgb}{0.000000,0.000000,0.000000}%
\pgfsetstrokecolor{currentstroke}%
\pgfsetstrokeopacity{0.000000}%
\pgfsetdash{}{0pt}%
\pgfpathmoveto{\pgfqpoint{1.558179in}{0.499444in}}%
\pgfpathlineto{\pgfqpoint{1.619566in}{0.499444in}}%
\pgfpathlineto{\pgfqpoint{1.619566in}{0.568228in}}%
\pgfpathlineto{\pgfqpoint{1.558179in}{0.568228in}}%
\pgfpathlineto{\pgfqpoint{1.558179in}{0.499444in}}%
\pgfpathclose%
\pgfusepath{fill}%
\end{pgfscope}%
\begin{pgfscope}%
\pgfpathrectangle{\pgfqpoint{0.445556in}{0.499444in}}{\pgfqpoint{3.875000in}{1.155000in}}%
\pgfusepath{clip}%
\pgfsetbuttcap%
\pgfsetmiterjoin%
\definecolor{currentfill}{rgb}{0.000000,0.000000,0.000000}%
\pgfsetfillcolor{currentfill}%
\pgfsetlinewidth{0.000000pt}%
\definecolor{currentstroke}{rgb}{0.000000,0.000000,0.000000}%
\pgfsetstrokecolor{currentstroke}%
\pgfsetstrokeopacity{0.000000}%
\pgfsetdash{}{0pt}%
\pgfpathmoveto{\pgfqpoint{1.711645in}{0.499444in}}%
\pgfpathlineto{\pgfqpoint{1.773031in}{0.499444in}}%
\pgfpathlineto{\pgfqpoint{1.773031in}{0.590743in}}%
\pgfpathlineto{\pgfqpoint{1.711645in}{0.590743in}}%
\pgfpathlineto{\pgfqpoint{1.711645in}{0.499444in}}%
\pgfpathclose%
\pgfusepath{fill}%
\end{pgfscope}%
\begin{pgfscope}%
\pgfpathrectangle{\pgfqpoint{0.445556in}{0.499444in}}{\pgfqpoint{3.875000in}{1.155000in}}%
\pgfusepath{clip}%
\pgfsetbuttcap%
\pgfsetmiterjoin%
\definecolor{currentfill}{rgb}{0.000000,0.000000,0.000000}%
\pgfsetfillcolor{currentfill}%
\pgfsetlinewidth{0.000000pt}%
\definecolor{currentstroke}{rgb}{0.000000,0.000000,0.000000}%
\pgfsetstrokecolor{currentstroke}%
\pgfsetstrokeopacity{0.000000}%
\pgfsetdash{}{0pt}%
\pgfpathmoveto{\pgfqpoint{1.865110in}{0.499444in}}%
\pgfpathlineto{\pgfqpoint{1.926496in}{0.499444in}}%
\pgfpathlineto{\pgfqpoint{1.926496in}{0.605521in}}%
\pgfpathlineto{\pgfqpoint{1.865110in}{0.605521in}}%
\pgfpathlineto{\pgfqpoint{1.865110in}{0.499444in}}%
\pgfpathclose%
\pgfusepath{fill}%
\end{pgfscope}%
\begin{pgfscope}%
\pgfpathrectangle{\pgfqpoint{0.445556in}{0.499444in}}{\pgfqpoint{3.875000in}{1.155000in}}%
\pgfusepath{clip}%
\pgfsetbuttcap%
\pgfsetmiterjoin%
\definecolor{currentfill}{rgb}{0.000000,0.000000,0.000000}%
\pgfsetfillcolor{currentfill}%
\pgfsetlinewidth{0.000000pt}%
\definecolor{currentstroke}{rgb}{0.000000,0.000000,0.000000}%
\pgfsetstrokecolor{currentstroke}%
\pgfsetstrokeopacity{0.000000}%
\pgfsetdash{}{0pt}%
\pgfpathmoveto{\pgfqpoint{2.018575in}{0.499444in}}%
\pgfpathlineto{\pgfqpoint{2.079962in}{0.499444in}}%
\pgfpathlineto{\pgfqpoint{2.079962in}{0.624632in}}%
\pgfpathlineto{\pgfqpoint{2.018575in}{0.624632in}}%
\pgfpathlineto{\pgfqpoint{2.018575in}{0.499444in}}%
\pgfpathclose%
\pgfusepath{fill}%
\end{pgfscope}%
\begin{pgfscope}%
\pgfpathrectangle{\pgfqpoint{0.445556in}{0.499444in}}{\pgfqpoint{3.875000in}{1.155000in}}%
\pgfusepath{clip}%
\pgfsetbuttcap%
\pgfsetmiterjoin%
\definecolor{currentfill}{rgb}{0.000000,0.000000,0.000000}%
\pgfsetfillcolor{currentfill}%
\pgfsetlinewidth{0.000000pt}%
\definecolor{currentstroke}{rgb}{0.000000,0.000000,0.000000}%
\pgfsetstrokecolor{currentstroke}%
\pgfsetstrokeopacity{0.000000}%
\pgfsetdash{}{0pt}%
\pgfpathmoveto{\pgfqpoint{2.172041in}{0.499444in}}%
\pgfpathlineto{\pgfqpoint{2.233427in}{0.499444in}}%
\pgfpathlineto{\pgfqpoint{2.233427in}{0.643356in}}%
\pgfpathlineto{\pgfqpoint{2.172041in}{0.643356in}}%
\pgfpathlineto{\pgfqpoint{2.172041in}{0.499444in}}%
\pgfpathclose%
\pgfusepath{fill}%
\end{pgfscope}%
\begin{pgfscope}%
\pgfpathrectangle{\pgfqpoint{0.445556in}{0.499444in}}{\pgfqpoint{3.875000in}{1.155000in}}%
\pgfusepath{clip}%
\pgfsetbuttcap%
\pgfsetmiterjoin%
\definecolor{currentfill}{rgb}{0.000000,0.000000,0.000000}%
\pgfsetfillcolor{currentfill}%
\pgfsetlinewidth{0.000000pt}%
\definecolor{currentstroke}{rgb}{0.000000,0.000000,0.000000}%
\pgfsetstrokecolor{currentstroke}%
\pgfsetstrokeopacity{0.000000}%
\pgfsetdash{}{0pt}%
\pgfpathmoveto{\pgfqpoint{2.325506in}{0.499444in}}%
\pgfpathlineto{\pgfqpoint{2.386892in}{0.499444in}}%
\pgfpathlineto{\pgfqpoint{2.386892in}{0.662622in}}%
\pgfpathlineto{\pgfqpoint{2.325506in}{0.662622in}}%
\pgfpathlineto{\pgfqpoint{2.325506in}{0.499444in}}%
\pgfpathclose%
\pgfusepath{fill}%
\end{pgfscope}%
\begin{pgfscope}%
\pgfpathrectangle{\pgfqpoint{0.445556in}{0.499444in}}{\pgfqpoint{3.875000in}{1.155000in}}%
\pgfusepath{clip}%
\pgfsetbuttcap%
\pgfsetmiterjoin%
\definecolor{currentfill}{rgb}{0.000000,0.000000,0.000000}%
\pgfsetfillcolor{currentfill}%
\pgfsetlinewidth{0.000000pt}%
\definecolor{currentstroke}{rgb}{0.000000,0.000000,0.000000}%
\pgfsetstrokecolor{currentstroke}%
\pgfsetstrokeopacity{0.000000}%
\pgfsetdash{}{0pt}%
\pgfpathmoveto{\pgfqpoint{2.478972in}{0.499444in}}%
\pgfpathlineto{\pgfqpoint{2.540358in}{0.499444in}}%
\pgfpathlineto{\pgfqpoint{2.540358in}{0.672216in}}%
\pgfpathlineto{\pgfqpoint{2.478972in}{0.672216in}}%
\pgfpathlineto{\pgfqpoint{2.478972in}{0.499444in}}%
\pgfpathclose%
\pgfusepath{fill}%
\end{pgfscope}%
\begin{pgfscope}%
\pgfpathrectangle{\pgfqpoint{0.445556in}{0.499444in}}{\pgfqpoint{3.875000in}{1.155000in}}%
\pgfusepath{clip}%
\pgfsetbuttcap%
\pgfsetmiterjoin%
\definecolor{currentfill}{rgb}{0.000000,0.000000,0.000000}%
\pgfsetfillcolor{currentfill}%
\pgfsetlinewidth{0.000000pt}%
\definecolor{currentstroke}{rgb}{0.000000,0.000000,0.000000}%
\pgfsetstrokecolor{currentstroke}%
\pgfsetstrokeopacity{0.000000}%
\pgfsetdash{}{0pt}%
\pgfpathmoveto{\pgfqpoint{2.632437in}{0.499444in}}%
\pgfpathlineto{\pgfqpoint{2.693823in}{0.499444in}}%
\pgfpathlineto{\pgfqpoint{2.693823in}{0.684286in}}%
\pgfpathlineto{\pgfqpoint{2.632437in}{0.684286in}}%
\pgfpathlineto{\pgfqpoint{2.632437in}{0.499444in}}%
\pgfpathclose%
\pgfusepath{fill}%
\end{pgfscope}%
\begin{pgfscope}%
\pgfpathrectangle{\pgfqpoint{0.445556in}{0.499444in}}{\pgfqpoint{3.875000in}{1.155000in}}%
\pgfusepath{clip}%
\pgfsetbuttcap%
\pgfsetmiterjoin%
\definecolor{currentfill}{rgb}{0.000000,0.000000,0.000000}%
\pgfsetfillcolor{currentfill}%
\pgfsetlinewidth{0.000000pt}%
\definecolor{currentstroke}{rgb}{0.000000,0.000000,0.000000}%
\pgfsetstrokecolor{currentstroke}%
\pgfsetstrokeopacity{0.000000}%
\pgfsetdash{}{0pt}%
\pgfpathmoveto{\pgfqpoint{2.785902in}{0.499444in}}%
\pgfpathlineto{\pgfqpoint{2.847288in}{0.499444in}}%
\pgfpathlineto{\pgfqpoint{2.847288in}{0.686607in}}%
\pgfpathlineto{\pgfqpoint{2.785902in}{0.686607in}}%
\pgfpathlineto{\pgfqpoint{2.785902in}{0.499444in}}%
\pgfpathclose%
\pgfusepath{fill}%
\end{pgfscope}%
\begin{pgfscope}%
\pgfpathrectangle{\pgfqpoint{0.445556in}{0.499444in}}{\pgfqpoint{3.875000in}{1.155000in}}%
\pgfusepath{clip}%
\pgfsetbuttcap%
\pgfsetmiterjoin%
\definecolor{currentfill}{rgb}{0.000000,0.000000,0.000000}%
\pgfsetfillcolor{currentfill}%
\pgfsetlinewidth{0.000000pt}%
\definecolor{currentstroke}{rgb}{0.000000,0.000000,0.000000}%
\pgfsetstrokecolor{currentstroke}%
\pgfsetstrokeopacity{0.000000}%
\pgfsetdash{}{0pt}%
\pgfpathmoveto{\pgfqpoint{2.939368in}{0.499444in}}%
\pgfpathlineto{\pgfqpoint{3.000754in}{0.499444in}}%
\pgfpathlineto{\pgfqpoint{3.000754in}{0.682816in}}%
\pgfpathlineto{\pgfqpoint{2.939368in}{0.682816in}}%
\pgfpathlineto{\pgfqpoint{2.939368in}{0.499444in}}%
\pgfpathclose%
\pgfusepath{fill}%
\end{pgfscope}%
\begin{pgfscope}%
\pgfpathrectangle{\pgfqpoint{0.445556in}{0.499444in}}{\pgfqpoint{3.875000in}{1.155000in}}%
\pgfusepath{clip}%
\pgfsetbuttcap%
\pgfsetmiterjoin%
\definecolor{currentfill}{rgb}{0.000000,0.000000,0.000000}%
\pgfsetfillcolor{currentfill}%
\pgfsetlinewidth{0.000000pt}%
\definecolor{currentstroke}{rgb}{0.000000,0.000000,0.000000}%
\pgfsetstrokecolor{currentstroke}%
\pgfsetstrokeopacity{0.000000}%
\pgfsetdash{}{0pt}%
\pgfpathmoveto{\pgfqpoint{3.092833in}{0.499444in}}%
\pgfpathlineto{\pgfqpoint{3.154219in}{0.499444in}}%
\pgfpathlineto{\pgfqpoint{3.154219in}{0.681037in}}%
\pgfpathlineto{\pgfqpoint{3.092833in}{0.681037in}}%
\pgfpathlineto{\pgfqpoint{3.092833in}{0.499444in}}%
\pgfpathclose%
\pgfusepath{fill}%
\end{pgfscope}%
\begin{pgfscope}%
\pgfpathrectangle{\pgfqpoint{0.445556in}{0.499444in}}{\pgfqpoint{3.875000in}{1.155000in}}%
\pgfusepath{clip}%
\pgfsetbuttcap%
\pgfsetmiterjoin%
\definecolor{currentfill}{rgb}{0.000000,0.000000,0.000000}%
\pgfsetfillcolor{currentfill}%
\pgfsetlinewidth{0.000000pt}%
\definecolor{currentstroke}{rgb}{0.000000,0.000000,0.000000}%
\pgfsetstrokecolor{currentstroke}%
\pgfsetstrokeopacity{0.000000}%
\pgfsetdash{}{0pt}%
\pgfpathmoveto{\pgfqpoint{3.246298in}{0.499444in}}%
\pgfpathlineto{\pgfqpoint{3.307684in}{0.499444in}}%
\pgfpathlineto{\pgfqpoint{3.307684in}{0.677787in}}%
\pgfpathlineto{\pgfqpoint{3.246298in}{0.677787in}}%
\pgfpathlineto{\pgfqpoint{3.246298in}{0.499444in}}%
\pgfpathclose%
\pgfusepath{fill}%
\end{pgfscope}%
\begin{pgfscope}%
\pgfpathrectangle{\pgfqpoint{0.445556in}{0.499444in}}{\pgfqpoint{3.875000in}{1.155000in}}%
\pgfusepath{clip}%
\pgfsetbuttcap%
\pgfsetmiterjoin%
\definecolor{currentfill}{rgb}{0.000000,0.000000,0.000000}%
\pgfsetfillcolor{currentfill}%
\pgfsetlinewidth{0.000000pt}%
\definecolor{currentstroke}{rgb}{0.000000,0.000000,0.000000}%
\pgfsetstrokecolor{currentstroke}%
\pgfsetstrokeopacity{0.000000}%
\pgfsetdash{}{0pt}%
\pgfpathmoveto{\pgfqpoint{3.399764in}{0.499444in}}%
\pgfpathlineto{\pgfqpoint{3.461150in}{0.499444in}}%
\pgfpathlineto{\pgfqpoint{3.461150in}{0.659527in}}%
\pgfpathlineto{\pgfqpoint{3.399764in}{0.659527in}}%
\pgfpathlineto{\pgfqpoint{3.399764in}{0.499444in}}%
\pgfpathclose%
\pgfusepath{fill}%
\end{pgfscope}%
\begin{pgfscope}%
\pgfpathrectangle{\pgfqpoint{0.445556in}{0.499444in}}{\pgfqpoint{3.875000in}{1.155000in}}%
\pgfusepath{clip}%
\pgfsetbuttcap%
\pgfsetmiterjoin%
\definecolor{currentfill}{rgb}{0.000000,0.000000,0.000000}%
\pgfsetfillcolor{currentfill}%
\pgfsetlinewidth{0.000000pt}%
\definecolor{currentstroke}{rgb}{0.000000,0.000000,0.000000}%
\pgfsetstrokecolor{currentstroke}%
\pgfsetstrokeopacity{0.000000}%
\pgfsetdash{}{0pt}%
\pgfpathmoveto{\pgfqpoint{3.553229in}{0.499444in}}%
\pgfpathlineto{\pgfqpoint{3.614615in}{0.499444in}}%
\pgfpathlineto{\pgfqpoint{3.614615in}{0.657206in}}%
\pgfpathlineto{\pgfqpoint{3.553229in}{0.657206in}}%
\pgfpathlineto{\pgfqpoint{3.553229in}{0.499444in}}%
\pgfpathclose%
\pgfusepath{fill}%
\end{pgfscope}%
\begin{pgfscope}%
\pgfpathrectangle{\pgfqpoint{0.445556in}{0.499444in}}{\pgfqpoint{3.875000in}{1.155000in}}%
\pgfusepath{clip}%
\pgfsetbuttcap%
\pgfsetmiterjoin%
\definecolor{currentfill}{rgb}{0.000000,0.000000,0.000000}%
\pgfsetfillcolor{currentfill}%
\pgfsetlinewidth{0.000000pt}%
\definecolor{currentstroke}{rgb}{0.000000,0.000000,0.000000}%
\pgfsetstrokecolor{currentstroke}%
\pgfsetstrokeopacity{0.000000}%
\pgfsetdash{}{0pt}%
\pgfpathmoveto{\pgfqpoint{3.706694in}{0.499444in}}%
\pgfpathlineto{\pgfqpoint{3.768080in}{0.499444in}}%
\pgfpathlineto{\pgfqpoint{3.768080in}{0.639333in}}%
\pgfpathlineto{\pgfqpoint{3.706694in}{0.639333in}}%
\pgfpathlineto{\pgfqpoint{3.706694in}{0.499444in}}%
\pgfpathclose%
\pgfusepath{fill}%
\end{pgfscope}%
\begin{pgfscope}%
\pgfpathrectangle{\pgfqpoint{0.445556in}{0.499444in}}{\pgfqpoint{3.875000in}{1.155000in}}%
\pgfusepath{clip}%
\pgfsetbuttcap%
\pgfsetmiterjoin%
\definecolor{currentfill}{rgb}{0.000000,0.000000,0.000000}%
\pgfsetfillcolor{currentfill}%
\pgfsetlinewidth{0.000000pt}%
\definecolor{currentstroke}{rgb}{0.000000,0.000000,0.000000}%
\pgfsetstrokecolor{currentstroke}%
\pgfsetstrokeopacity{0.000000}%
\pgfsetdash{}{0pt}%
\pgfpathmoveto{\pgfqpoint{3.860160in}{0.499444in}}%
\pgfpathlineto{\pgfqpoint{3.921546in}{0.499444in}}%
\pgfpathlineto{\pgfqpoint{3.921546in}{0.623626in}}%
\pgfpathlineto{\pgfqpoint{3.860160in}{0.623626in}}%
\pgfpathlineto{\pgfqpoint{3.860160in}{0.499444in}}%
\pgfpathclose%
\pgfusepath{fill}%
\end{pgfscope}%
\begin{pgfscope}%
\pgfpathrectangle{\pgfqpoint{0.445556in}{0.499444in}}{\pgfqpoint{3.875000in}{1.155000in}}%
\pgfusepath{clip}%
\pgfsetbuttcap%
\pgfsetmiterjoin%
\definecolor{currentfill}{rgb}{0.000000,0.000000,0.000000}%
\pgfsetfillcolor{currentfill}%
\pgfsetlinewidth{0.000000pt}%
\definecolor{currentstroke}{rgb}{0.000000,0.000000,0.000000}%
\pgfsetstrokecolor{currentstroke}%
\pgfsetstrokeopacity{0.000000}%
\pgfsetdash{}{0pt}%
\pgfpathmoveto{\pgfqpoint{4.013625in}{0.499444in}}%
\pgfpathlineto{\pgfqpoint{4.075011in}{0.499444in}}%
\pgfpathlineto{\pgfqpoint{4.075011in}{0.580840in}}%
\pgfpathlineto{\pgfqpoint{4.013625in}{0.580840in}}%
\pgfpathlineto{\pgfqpoint{4.013625in}{0.499444in}}%
\pgfpathclose%
\pgfusepath{fill}%
\end{pgfscope}%
\begin{pgfscope}%
\pgfpathrectangle{\pgfqpoint{0.445556in}{0.499444in}}{\pgfqpoint{3.875000in}{1.155000in}}%
\pgfusepath{clip}%
\pgfsetbuttcap%
\pgfsetmiterjoin%
\definecolor{currentfill}{rgb}{0.000000,0.000000,0.000000}%
\pgfsetfillcolor{currentfill}%
\pgfsetlinewidth{0.000000pt}%
\definecolor{currentstroke}{rgb}{0.000000,0.000000,0.000000}%
\pgfsetstrokecolor{currentstroke}%
\pgfsetstrokeopacity{0.000000}%
\pgfsetdash{}{0pt}%
\pgfpathmoveto{\pgfqpoint{4.167090in}{0.499444in}}%
\pgfpathlineto{\pgfqpoint{4.228476in}{0.499444in}}%
\pgfpathlineto{\pgfqpoint{4.228476in}{0.529774in}}%
\pgfpathlineto{\pgfqpoint{4.167090in}{0.529774in}}%
\pgfpathlineto{\pgfqpoint{4.167090in}{0.499444in}}%
\pgfpathclose%
\pgfusepath{fill}%
\end{pgfscope}%
\begin{pgfscope}%
\pgfsetbuttcap%
\pgfsetroundjoin%
\definecolor{currentfill}{rgb}{0.000000,0.000000,0.000000}%
\pgfsetfillcolor{currentfill}%
\pgfsetlinewidth{0.803000pt}%
\definecolor{currentstroke}{rgb}{0.000000,0.000000,0.000000}%
\pgfsetstrokecolor{currentstroke}%
\pgfsetdash{}{0pt}%
\pgfsys@defobject{currentmarker}{\pgfqpoint{0.000000in}{-0.048611in}}{\pgfqpoint{0.000000in}{0.000000in}}{%
\pgfpathmoveto{\pgfqpoint{0.000000in}{0.000000in}}%
\pgfpathlineto{\pgfqpoint{0.000000in}{-0.048611in}}%
\pgfusepath{stroke,fill}%
}%
\begin{pgfscope}%
\pgfsys@transformshift{0.483922in}{0.499444in}%
\pgfsys@useobject{currentmarker}{}%
\end{pgfscope}%
\end{pgfscope}%
\begin{pgfscope}%
\definecolor{textcolor}{rgb}{0.000000,0.000000,0.000000}%
\pgfsetstrokecolor{textcolor}%
\pgfsetfillcolor{textcolor}%
\pgftext[x=0.483922in,y=0.402222in,,top]{\color{textcolor}\rmfamily\fontsize{10.000000}{12.000000}\selectfont 0.0}%
\end{pgfscope}%
\begin{pgfscope}%
\pgfsetbuttcap%
\pgfsetroundjoin%
\definecolor{currentfill}{rgb}{0.000000,0.000000,0.000000}%
\pgfsetfillcolor{currentfill}%
\pgfsetlinewidth{0.803000pt}%
\definecolor{currentstroke}{rgb}{0.000000,0.000000,0.000000}%
\pgfsetstrokecolor{currentstroke}%
\pgfsetdash{}{0pt}%
\pgfsys@defobject{currentmarker}{\pgfqpoint{0.000000in}{-0.048611in}}{\pgfqpoint{0.000000in}{0.000000in}}{%
\pgfpathmoveto{\pgfqpoint{0.000000in}{0.000000in}}%
\pgfpathlineto{\pgfqpoint{0.000000in}{-0.048611in}}%
\pgfusepath{stroke,fill}%
}%
\begin{pgfscope}%
\pgfsys@transformshift{0.867585in}{0.499444in}%
\pgfsys@useobject{currentmarker}{}%
\end{pgfscope}%
\end{pgfscope}%
\begin{pgfscope}%
\definecolor{textcolor}{rgb}{0.000000,0.000000,0.000000}%
\pgfsetstrokecolor{textcolor}%
\pgfsetfillcolor{textcolor}%
\pgftext[x=0.867585in,y=0.402222in,,top]{\color{textcolor}\rmfamily\fontsize{10.000000}{12.000000}\selectfont 0.1}%
\end{pgfscope}%
\begin{pgfscope}%
\pgfsetbuttcap%
\pgfsetroundjoin%
\definecolor{currentfill}{rgb}{0.000000,0.000000,0.000000}%
\pgfsetfillcolor{currentfill}%
\pgfsetlinewidth{0.803000pt}%
\definecolor{currentstroke}{rgb}{0.000000,0.000000,0.000000}%
\pgfsetstrokecolor{currentstroke}%
\pgfsetdash{}{0pt}%
\pgfsys@defobject{currentmarker}{\pgfqpoint{0.000000in}{-0.048611in}}{\pgfqpoint{0.000000in}{0.000000in}}{%
\pgfpathmoveto{\pgfqpoint{0.000000in}{0.000000in}}%
\pgfpathlineto{\pgfqpoint{0.000000in}{-0.048611in}}%
\pgfusepath{stroke,fill}%
}%
\begin{pgfscope}%
\pgfsys@transformshift{1.251249in}{0.499444in}%
\pgfsys@useobject{currentmarker}{}%
\end{pgfscope}%
\end{pgfscope}%
\begin{pgfscope}%
\definecolor{textcolor}{rgb}{0.000000,0.000000,0.000000}%
\pgfsetstrokecolor{textcolor}%
\pgfsetfillcolor{textcolor}%
\pgftext[x=1.251249in,y=0.402222in,,top]{\color{textcolor}\rmfamily\fontsize{10.000000}{12.000000}\selectfont 0.2}%
\end{pgfscope}%
\begin{pgfscope}%
\pgfsetbuttcap%
\pgfsetroundjoin%
\definecolor{currentfill}{rgb}{0.000000,0.000000,0.000000}%
\pgfsetfillcolor{currentfill}%
\pgfsetlinewidth{0.803000pt}%
\definecolor{currentstroke}{rgb}{0.000000,0.000000,0.000000}%
\pgfsetstrokecolor{currentstroke}%
\pgfsetdash{}{0pt}%
\pgfsys@defobject{currentmarker}{\pgfqpoint{0.000000in}{-0.048611in}}{\pgfqpoint{0.000000in}{0.000000in}}{%
\pgfpathmoveto{\pgfqpoint{0.000000in}{0.000000in}}%
\pgfpathlineto{\pgfqpoint{0.000000in}{-0.048611in}}%
\pgfusepath{stroke,fill}%
}%
\begin{pgfscope}%
\pgfsys@transformshift{1.634912in}{0.499444in}%
\pgfsys@useobject{currentmarker}{}%
\end{pgfscope}%
\end{pgfscope}%
\begin{pgfscope}%
\definecolor{textcolor}{rgb}{0.000000,0.000000,0.000000}%
\pgfsetstrokecolor{textcolor}%
\pgfsetfillcolor{textcolor}%
\pgftext[x=1.634912in,y=0.402222in,,top]{\color{textcolor}\rmfamily\fontsize{10.000000}{12.000000}\selectfont 0.3}%
\end{pgfscope}%
\begin{pgfscope}%
\pgfsetbuttcap%
\pgfsetroundjoin%
\definecolor{currentfill}{rgb}{0.000000,0.000000,0.000000}%
\pgfsetfillcolor{currentfill}%
\pgfsetlinewidth{0.803000pt}%
\definecolor{currentstroke}{rgb}{0.000000,0.000000,0.000000}%
\pgfsetstrokecolor{currentstroke}%
\pgfsetdash{}{0pt}%
\pgfsys@defobject{currentmarker}{\pgfqpoint{0.000000in}{-0.048611in}}{\pgfqpoint{0.000000in}{0.000000in}}{%
\pgfpathmoveto{\pgfqpoint{0.000000in}{0.000000in}}%
\pgfpathlineto{\pgfqpoint{0.000000in}{-0.048611in}}%
\pgfusepath{stroke,fill}%
}%
\begin{pgfscope}%
\pgfsys@transformshift{2.018575in}{0.499444in}%
\pgfsys@useobject{currentmarker}{}%
\end{pgfscope}%
\end{pgfscope}%
\begin{pgfscope}%
\definecolor{textcolor}{rgb}{0.000000,0.000000,0.000000}%
\pgfsetstrokecolor{textcolor}%
\pgfsetfillcolor{textcolor}%
\pgftext[x=2.018575in,y=0.402222in,,top]{\color{textcolor}\rmfamily\fontsize{10.000000}{12.000000}\selectfont 0.4}%
\end{pgfscope}%
\begin{pgfscope}%
\pgfsetbuttcap%
\pgfsetroundjoin%
\definecolor{currentfill}{rgb}{0.000000,0.000000,0.000000}%
\pgfsetfillcolor{currentfill}%
\pgfsetlinewidth{0.803000pt}%
\definecolor{currentstroke}{rgb}{0.000000,0.000000,0.000000}%
\pgfsetstrokecolor{currentstroke}%
\pgfsetdash{}{0pt}%
\pgfsys@defobject{currentmarker}{\pgfqpoint{0.000000in}{-0.048611in}}{\pgfqpoint{0.000000in}{0.000000in}}{%
\pgfpathmoveto{\pgfqpoint{0.000000in}{0.000000in}}%
\pgfpathlineto{\pgfqpoint{0.000000in}{-0.048611in}}%
\pgfusepath{stroke,fill}%
}%
\begin{pgfscope}%
\pgfsys@transformshift{2.402239in}{0.499444in}%
\pgfsys@useobject{currentmarker}{}%
\end{pgfscope}%
\end{pgfscope}%
\begin{pgfscope}%
\definecolor{textcolor}{rgb}{0.000000,0.000000,0.000000}%
\pgfsetstrokecolor{textcolor}%
\pgfsetfillcolor{textcolor}%
\pgftext[x=2.402239in,y=0.402222in,,top]{\color{textcolor}\rmfamily\fontsize{10.000000}{12.000000}\selectfont 0.5}%
\end{pgfscope}%
\begin{pgfscope}%
\pgfsetbuttcap%
\pgfsetroundjoin%
\definecolor{currentfill}{rgb}{0.000000,0.000000,0.000000}%
\pgfsetfillcolor{currentfill}%
\pgfsetlinewidth{0.803000pt}%
\definecolor{currentstroke}{rgb}{0.000000,0.000000,0.000000}%
\pgfsetstrokecolor{currentstroke}%
\pgfsetdash{}{0pt}%
\pgfsys@defobject{currentmarker}{\pgfqpoint{0.000000in}{-0.048611in}}{\pgfqpoint{0.000000in}{0.000000in}}{%
\pgfpathmoveto{\pgfqpoint{0.000000in}{0.000000in}}%
\pgfpathlineto{\pgfqpoint{0.000000in}{-0.048611in}}%
\pgfusepath{stroke,fill}%
}%
\begin{pgfscope}%
\pgfsys@transformshift{2.785902in}{0.499444in}%
\pgfsys@useobject{currentmarker}{}%
\end{pgfscope}%
\end{pgfscope}%
\begin{pgfscope}%
\definecolor{textcolor}{rgb}{0.000000,0.000000,0.000000}%
\pgfsetstrokecolor{textcolor}%
\pgfsetfillcolor{textcolor}%
\pgftext[x=2.785902in,y=0.402222in,,top]{\color{textcolor}\rmfamily\fontsize{10.000000}{12.000000}\selectfont 0.6}%
\end{pgfscope}%
\begin{pgfscope}%
\pgfsetbuttcap%
\pgfsetroundjoin%
\definecolor{currentfill}{rgb}{0.000000,0.000000,0.000000}%
\pgfsetfillcolor{currentfill}%
\pgfsetlinewidth{0.803000pt}%
\definecolor{currentstroke}{rgb}{0.000000,0.000000,0.000000}%
\pgfsetstrokecolor{currentstroke}%
\pgfsetdash{}{0pt}%
\pgfsys@defobject{currentmarker}{\pgfqpoint{0.000000in}{-0.048611in}}{\pgfqpoint{0.000000in}{0.000000in}}{%
\pgfpathmoveto{\pgfqpoint{0.000000in}{0.000000in}}%
\pgfpathlineto{\pgfqpoint{0.000000in}{-0.048611in}}%
\pgfusepath{stroke,fill}%
}%
\begin{pgfscope}%
\pgfsys@transformshift{3.169566in}{0.499444in}%
\pgfsys@useobject{currentmarker}{}%
\end{pgfscope}%
\end{pgfscope}%
\begin{pgfscope}%
\definecolor{textcolor}{rgb}{0.000000,0.000000,0.000000}%
\pgfsetstrokecolor{textcolor}%
\pgfsetfillcolor{textcolor}%
\pgftext[x=3.169566in,y=0.402222in,,top]{\color{textcolor}\rmfamily\fontsize{10.000000}{12.000000}\selectfont 0.7}%
\end{pgfscope}%
\begin{pgfscope}%
\pgfsetbuttcap%
\pgfsetroundjoin%
\definecolor{currentfill}{rgb}{0.000000,0.000000,0.000000}%
\pgfsetfillcolor{currentfill}%
\pgfsetlinewidth{0.803000pt}%
\definecolor{currentstroke}{rgb}{0.000000,0.000000,0.000000}%
\pgfsetstrokecolor{currentstroke}%
\pgfsetdash{}{0pt}%
\pgfsys@defobject{currentmarker}{\pgfqpoint{0.000000in}{-0.048611in}}{\pgfqpoint{0.000000in}{0.000000in}}{%
\pgfpathmoveto{\pgfqpoint{0.000000in}{0.000000in}}%
\pgfpathlineto{\pgfqpoint{0.000000in}{-0.048611in}}%
\pgfusepath{stroke,fill}%
}%
\begin{pgfscope}%
\pgfsys@transformshift{3.553229in}{0.499444in}%
\pgfsys@useobject{currentmarker}{}%
\end{pgfscope}%
\end{pgfscope}%
\begin{pgfscope}%
\definecolor{textcolor}{rgb}{0.000000,0.000000,0.000000}%
\pgfsetstrokecolor{textcolor}%
\pgfsetfillcolor{textcolor}%
\pgftext[x=3.553229in,y=0.402222in,,top]{\color{textcolor}\rmfamily\fontsize{10.000000}{12.000000}\selectfont 0.8}%
\end{pgfscope}%
\begin{pgfscope}%
\pgfsetbuttcap%
\pgfsetroundjoin%
\definecolor{currentfill}{rgb}{0.000000,0.000000,0.000000}%
\pgfsetfillcolor{currentfill}%
\pgfsetlinewidth{0.803000pt}%
\definecolor{currentstroke}{rgb}{0.000000,0.000000,0.000000}%
\pgfsetstrokecolor{currentstroke}%
\pgfsetdash{}{0pt}%
\pgfsys@defobject{currentmarker}{\pgfqpoint{0.000000in}{-0.048611in}}{\pgfqpoint{0.000000in}{0.000000in}}{%
\pgfpathmoveto{\pgfqpoint{0.000000in}{0.000000in}}%
\pgfpathlineto{\pgfqpoint{0.000000in}{-0.048611in}}%
\pgfusepath{stroke,fill}%
}%
\begin{pgfscope}%
\pgfsys@transformshift{3.936892in}{0.499444in}%
\pgfsys@useobject{currentmarker}{}%
\end{pgfscope}%
\end{pgfscope}%
\begin{pgfscope}%
\definecolor{textcolor}{rgb}{0.000000,0.000000,0.000000}%
\pgfsetstrokecolor{textcolor}%
\pgfsetfillcolor{textcolor}%
\pgftext[x=3.936892in,y=0.402222in,,top]{\color{textcolor}\rmfamily\fontsize{10.000000}{12.000000}\selectfont 0.9}%
\end{pgfscope}%
\begin{pgfscope}%
\pgfsetbuttcap%
\pgfsetroundjoin%
\definecolor{currentfill}{rgb}{0.000000,0.000000,0.000000}%
\pgfsetfillcolor{currentfill}%
\pgfsetlinewidth{0.803000pt}%
\definecolor{currentstroke}{rgb}{0.000000,0.000000,0.000000}%
\pgfsetstrokecolor{currentstroke}%
\pgfsetdash{}{0pt}%
\pgfsys@defobject{currentmarker}{\pgfqpoint{0.000000in}{-0.048611in}}{\pgfqpoint{0.000000in}{0.000000in}}{%
\pgfpathmoveto{\pgfqpoint{0.000000in}{0.000000in}}%
\pgfpathlineto{\pgfqpoint{0.000000in}{-0.048611in}}%
\pgfusepath{stroke,fill}%
}%
\begin{pgfscope}%
\pgfsys@transformshift{4.320556in}{0.499444in}%
\pgfsys@useobject{currentmarker}{}%
\end{pgfscope}%
\end{pgfscope}%
\begin{pgfscope}%
\definecolor{textcolor}{rgb}{0.000000,0.000000,0.000000}%
\pgfsetstrokecolor{textcolor}%
\pgfsetfillcolor{textcolor}%
\pgftext[x=4.320556in,y=0.402222in,,top]{\color{textcolor}\rmfamily\fontsize{10.000000}{12.000000}\selectfont 1.0}%
\end{pgfscope}%
\begin{pgfscope}%
\definecolor{textcolor}{rgb}{0.000000,0.000000,0.000000}%
\pgfsetstrokecolor{textcolor}%
\pgfsetfillcolor{textcolor}%
\pgftext[x=2.383056in,y=0.223333in,,top]{\color{textcolor}\rmfamily\fontsize{10.000000}{12.000000}\selectfont \(\displaystyle p\)}%
\end{pgfscope}%
\begin{pgfscope}%
\pgfsetbuttcap%
\pgfsetroundjoin%
\definecolor{currentfill}{rgb}{0.000000,0.000000,0.000000}%
\pgfsetfillcolor{currentfill}%
\pgfsetlinewidth{0.803000pt}%
\definecolor{currentstroke}{rgb}{0.000000,0.000000,0.000000}%
\pgfsetstrokecolor{currentstroke}%
\pgfsetdash{}{0pt}%
\pgfsys@defobject{currentmarker}{\pgfqpoint{-0.048611in}{0.000000in}}{\pgfqpoint{-0.000000in}{0.000000in}}{%
\pgfpathmoveto{\pgfqpoint{-0.000000in}{0.000000in}}%
\pgfpathlineto{\pgfqpoint{-0.048611in}{0.000000in}}%
\pgfusepath{stroke,fill}%
}%
\begin{pgfscope}%
\pgfsys@transformshift{0.445556in}{0.499444in}%
\pgfsys@useobject{currentmarker}{}%
\end{pgfscope}%
\end{pgfscope}%
\begin{pgfscope}%
\definecolor{textcolor}{rgb}{0.000000,0.000000,0.000000}%
\pgfsetstrokecolor{textcolor}%
\pgfsetfillcolor{textcolor}%
\pgftext[x=0.278889in, y=0.451250in, left, base]{\color{textcolor}\rmfamily\fontsize{10.000000}{12.000000}\selectfont \(\displaystyle {0}\)}%
\end{pgfscope}%
\begin{pgfscope}%
\pgfsetbuttcap%
\pgfsetroundjoin%
\definecolor{currentfill}{rgb}{0.000000,0.000000,0.000000}%
\pgfsetfillcolor{currentfill}%
\pgfsetlinewidth{0.803000pt}%
\definecolor{currentstroke}{rgb}{0.000000,0.000000,0.000000}%
\pgfsetstrokecolor{currentstroke}%
\pgfsetdash{}{0pt}%
\pgfsys@defobject{currentmarker}{\pgfqpoint{-0.048611in}{0.000000in}}{\pgfqpoint{-0.000000in}{0.000000in}}{%
\pgfpathmoveto{\pgfqpoint{-0.000000in}{0.000000in}}%
\pgfpathlineto{\pgfqpoint{-0.048611in}{0.000000in}}%
\pgfusepath{stroke,fill}%
}%
\begin{pgfscope}%
\pgfsys@transformshift{0.445556in}{0.830705in}%
\pgfsys@useobject{currentmarker}{}%
\end{pgfscope}%
\end{pgfscope}%
\begin{pgfscope}%
\definecolor{textcolor}{rgb}{0.000000,0.000000,0.000000}%
\pgfsetstrokecolor{textcolor}%
\pgfsetfillcolor{textcolor}%
\pgftext[x=0.278889in, y=0.782511in, left, base]{\color{textcolor}\rmfamily\fontsize{10.000000}{12.000000}\selectfont \(\displaystyle {2}\)}%
\end{pgfscope}%
\begin{pgfscope}%
\pgfsetbuttcap%
\pgfsetroundjoin%
\definecolor{currentfill}{rgb}{0.000000,0.000000,0.000000}%
\pgfsetfillcolor{currentfill}%
\pgfsetlinewidth{0.803000pt}%
\definecolor{currentstroke}{rgb}{0.000000,0.000000,0.000000}%
\pgfsetstrokecolor{currentstroke}%
\pgfsetdash{}{0pt}%
\pgfsys@defobject{currentmarker}{\pgfqpoint{-0.048611in}{0.000000in}}{\pgfqpoint{-0.000000in}{0.000000in}}{%
\pgfpathmoveto{\pgfqpoint{-0.000000in}{0.000000in}}%
\pgfpathlineto{\pgfqpoint{-0.048611in}{0.000000in}}%
\pgfusepath{stroke,fill}%
}%
\begin{pgfscope}%
\pgfsys@transformshift{0.445556in}{1.161966in}%
\pgfsys@useobject{currentmarker}{}%
\end{pgfscope}%
\end{pgfscope}%
\begin{pgfscope}%
\definecolor{textcolor}{rgb}{0.000000,0.000000,0.000000}%
\pgfsetstrokecolor{textcolor}%
\pgfsetfillcolor{textcolor}%
\pgftext[x=0.278889in, y=1.113772in, left, base]{\color{textcolor}\rmfamily\fontsize{10.000000}{12.000000}\selectfont \(\displaystyle {4}\)}%
\end{pgfscope}%
\begin{pgfscope}%
\pgfsetbuttcap%
\pgfsetroundjoin%
\definecolor{currentfill}{rgb}{0.000000,0.000000,0.000000}%
\pgfsetfillcolor{currentfill}%
\pgfsetlinewidth{0.803000pt}%
\definecolor{currentstroke}{rgb}{0.000000,0.000000,0.000000}%
\pgfsetstrokecolor{currentstroke}%
\pgfsetdash{}{0pt}%
\pgfsys@defobject{currentmarker}{\pgfqpoint{-0.048611in}{0.000000in}}{\pgfqpoint{-0.000000in}{0.000000in}}{%
\pgfpathmoveto{\pgfqpoint{-0.000000in}{0.000000in}}%
\pgfpathlineto{\pgfqpoint{-0.048611in}{0.000000in}}%
\pgfusepath{stroke,fill}%
}%
\begin{pgfscope}%
\pgfsys@transformshift{0.445556in}{1.493228in}%
\pgfsys@useobject{currentmarker}{}%
\end{pgfscope}%
\end{pgfscope}%
\begin{pgfscope}%
\definecolor{textcolor}{rgb}{0.000000,0.000000,0.000000}%
\pgfsetstrokecolor{textcolor}%
\pgfsetfillcolor{textcolor}%
\pgftext[x=0.278889in, y=1.445033in, left, base]{\color{textcolor}\rmfamily\fontsize{10.000000}{12.000000}\selectfont \(\displaystyle {6}\)}%
\end{pgfscope}%
\begin{pgfscope}%
\definecolor{textcolor}{rgb}{0.000000,0.000000,0.000000}%
\pgfsetstrokecolor{textcolor}%
\pgfsetfillcolor{textcolor}%
\pgftext[x=0.223333in,y=1.076944in,,bottom,rotate=90.000000]{\color{textcolor}\rmfamily\fontsize{10.000000}{12.000000}\selectfont Percent of Data Set}%
\end{pgfscope}%
\begin{pgfscope}%
\pgfsetrectcap%
\pgfsetmiterjoin%
\pgfsetlinewidth{0.803000pt}%
\definecolor{currentstroke}{rgb}{0.000000,0.000000,0.000000}%
\pgfsetstrokecolor{currentstroke}%
\pgfsetdash{}{0pt}%
\pgfpathmoveto{\pgfqpoint{0.445556in}{0.499444in}}%
\pgfpathlineto{\pgfqpoint{0.445556in}{1.654444in}}%
\pgfusepath{stroke}%
\end{pgfscope}%
\begin{pgfscope}%
\pgfsetrectcap%
\pgfsetmiterjoin%
\pgfsetlinewidth{0.803000pt}%
\definecolor{currentstroke}{rgb}{0.000000,0.000000,0.000000}%
\pgfsetstrokecolor{currentstroke}%
\pgfsetdash{}{0pt}%
\pgfpathmoveto{\pgfqpoint{4.320556in}{0.499444in}}%
\pgfpathlineto{\pgfqpoint{4.320556in}{1.654444in}}%
\pgfusepath{stroke}%
\end{pgfscope}%
\begin{pgfscope}%
\pgfsetrectcap%
\pgfsetmiterjoin%
\pgfsetlinewidth{0.803000pt}%
\definecolor{currentstroke}{rgb}{0.000000,0.000000,0.000000}%
\pgfsetstrokecolor{currentstroke}%
\pgfsetdash{}{0pt}%
\pgfpathmoveto{\pgfqpoint{0.445556in}{0.499444in}}%
\pgfpathlineto{\pgfqpoint{4.320556in}{0.499444in}}%
\pgfusepath{stroke}%
\end{pgfscope}%
\begin{pgfscope}%
\pgfsetrectcap%
\pgfsetmiterjoin%
\pgfsetlinewidth{0.803000pt}%
\definecolor{currentstroke}{rgb}{0.000000,0.000000,0.000000}%
\pgfsetstrokecolor{currentstroke}%
\pgfsetdash{}{0pt}%
\pgfpathmoveto{\pgfqpoint{0.445556in}{1.654444in}}%
\pgfpathlineto{\pgfqpoint{4.320556in}{1.654444in}}%
\pgfusepath{stroke}%
\end{pgfscope}%
\begin{pgfscope}%
\pgfsetbuttcap%
\pgfsetmiterjoin%
\definecolor{currentfill}{rgb}{1.000000,1.000000,1.000000}%
\pgfsetfillcolor{currentfill}%
\pgfsetfillopacity{0.800000}%
\pgfsetlinewidth{1.003750pt}%
\definecolor{currentstroke}{rgb}{0.800000,0.800000,0.800000}%
\pgfsetstrokecolor{currentstroke}%
\pgfsetstrokeopacity{0.800000}%
\pgfsetdash{}{0pt}%
\pgfpathmoveto{\pgfqpoint{3.543611in}{1.154445in}}%
\pgfpathlineto{\pgfqpoint{4.223333in}{1.154445in}}%
\pgfpathquadraticcurveto{\pgfqpoint{4.251111in}{1.154445in}}{\pgfqpoint{4.251111in}{1.182222in}}%
\pgfpathlineto{\pgfqpoint{4.251111in}{1.557222in}}%
\pgfpathquadraticcurveto{\pgfqpoint{4.251111in}{1.585000in}}{\pgfqpoint{4.223333in}{1.585000in}}%
\pgfpathlineto{\pgfqpoint{3.543611in}{1.585000in}}%
\pgfpathquadraticcurveto{\pgfqpoint{3.515833in}{1.585000in}}{\pgfqpoint{3.515833in}{1.557222in}}%
\pgfpathlineto{\pgfqpoint{3.515833in}{1.182222in}}%
\pgfpathquadraticcurveto{\pgfqpoint{3.515833in}{1.154445in}}{\pgfqpoint{3.543611in}{1.154445in}}%
\pgfpathlineto{\pgfqpoint{3.543611in}{1.154445in}}%
\pgfpathclose%
\pgfusepath{stroke,fill}%
\end{pgfscope}%
\begin{pgfscope}%
\pgfsetbuttcap%
\pgfsetmiterjoin%
\pgfsetlinewidth{1.003750pt}%
\definecolor{currentstroke}{rgb}{0.000000,0.000000,0.000000}%
\pgfsetstrokecolor{currentstroke}%
\pgfsetdash{}{0pt}%
\pgfpathmoveto{\pgfqpoint{3.571389in}{1.432222in}}%
\pgfpathlineto{\pgfqpoint{3.849167in}{1.432222in}}%
\pgfpathlineto{\pgfqpoint{3.849167in}{1.529444in}}%
\pgfpathlineto{\pgfqpoint{3.571389in}{1.529444in}}%
\pgfpathlineto{\pgfqpoint{3.571389in}{1.432222in}}%
\pgfpathclose%
\pgfusepath{stroke}%
\end{pgfscope}%
\begin{pgfscope}%
\definecolor{textcolor}{rgb}{0.000000,0.000000,0.000000}%
\pgfsetstrokecolor{textcolor}%
\pgfsetfillcolor{textcolor}%
\pgftext[x=3.960278in,y=1.432222in,left,base]{\color{textcolor}\rmfamily\fontsize{10.000000}{12.000000}\selectfont Neg}%
\end{pgfscope}%
\begin{pgfscope}%
\pgfsetbuttcap%
\pgfsetmiterjoin%
\definecolor{currentfill}{rgb}{0.000000,0.000000,0.000000}%
\pgfsetfillcolor{currentfill}%
\pgfsetlinewidth{0.000000pt}%
\definecolor{currentstroke}{rgb}{0.000000,0.000000,0.000000}%
\pgfsetstrokecolor{currentstroke}%
\pgfsetstrokeopacity{0.000000}%
\pgfsetdash{}{0pt}%
\pgfpathmoveto{\pgfqpoint{3.571389in}{1.236944in}}%
\pgfpathlineto{\pgfqpoint{3.849167in}{1.236944in}}%
\pgfpathlineto{\pgfqpoint{3.849167in}{1.334167in}}%
\pgfpathlineto{\pgfqpoint{3.571389in}{1.334167in}}%
\pgfpathlineto{\pgfqpoint{3.571389in}{1.236944in}}%
\pgfpathclose%
\pgfusepath{fill}%
\end{pgfscope}%
\begin{pgfscope}%
\definecolor{textcolor}{rgb}{0.000000,0.000000,0.000000}%
\pgfsetstrokecolor{textcolor}%
\pgfsetfillcolor{textcolor}%
\pgftext[x=3.960278in,y=1.236944in,left,base]{\color{textcolor}\rmfamily\fontsize{10.000000}{12.000000}\selectfont Pos}%
\end{pgfscope}%
\end{pgfpicture}%
\makeatother%
\endgroup%
	
&
	\vskip 0pt
	\hfil ROC Curve
	
	%% Creator: Matplotlib, PGF backend
%%
%% To include the figure in your LaTeX document, write
%%   \input{<filename>.pgf}
%%
%% Make sure the required packages are loaded in your preamble
%%   \usepackage{pgf}
%%
%% Also ensure that all the required font packages are loaded; for instance,
%% the lmodern package is sometimes necessary when using math font.
%%   \usepackage{lmodern}
%%
%% Figures using additional raster images can only be included by \input if
%% they are in the same directory as the main LaTeX file. For loading figures
%% from other directories you can use the `import` package
%%   \usepackage{import}
%%
%% and then include the figures with
%%   \import{<path to file>}{<filename>.pgf}
%%
%% Matplotlib used the following preamble
%%   
%%   \usepackage{fontspec}
%%   \makeatletter\@ifpackageloaded{underscore}{}{\usepackage[strings]{underscore}}\makeatother
%%
\begingroup%
\makeatletter%
\begin{pgfpicture}%
\pgfpathrectangle{\pgfpointorigin}{\pgfqpoint{2.221861in}{1.754444in}}%
\pgfusepath{use as bounding box, clip}%
\begin{pgfscope}%
\pgfsetbuttcap%
\pgfsetmiterjoin%
\definecolor{currentfill}{rgb}{1.000000,1.000000,1.000000}%
\pgfsetfillcolor{currentfill}%
\pgfsetlinewidth{0.000000pt}%
\definecolor{currentstroke}{rgb}{1.000000,1.000000,1.000000}%
\pgfsetstrokecolor{currentstroke}%
\pgfsetdash{}{0pt}%
\pgfpathmoveto{\pgfqpoint{0.000000in}{0.000000in}}%
\pgfpathlineto{\pgfqpoint{2.221861in}{0.000000in}}%
\pgfpathlineto{\pgfqpoint{2.221861in}{1.754444in}}%
\pgfpathlineto{\pgfqpoint{0.000000in}{1.754444in}}%
\pgfpathlineto{\pgfqpoint{0.000000in}{0.000000in}}%
\pgfpathclose%
\pgfusepath{fill}%
\end{pgfscope}%
\begin{pgfscope}%
\pgfsetbuttcap%
\pgfsetmiterjoin%
\definecolor{currentfill}{rgb}{1.000000,1.000000,1.000000}%
\pgfsetfillcolor{currentfill}%
\pgfsetlinewidth{0.000000pt}%
\definecolor{currentstroke}{rgb}{0.000000,0.000000,0.000000}%
\pgfsetstrokecolor{currentstroke}%
\pgfsetstrokeopacity{0.000000}%
\pgfsetdash{}{0pt}%
\pgfpathmoveto{\pgfqpoint{0.553581in}{0.499444in}}%
\pgfpathlineto{\pgfqpoint{2.103581in}{0.499444in}}%
\pgfpathlineto{\pgfqpoint{2.103581in}{1.654444in}}%
\pgfpathlineto{\pgfqpoint{0.553581in}{1.654444in}}%
\pgfpathlineto{\pgfqpoint{0.553581in}{0.499444in}}%
\pgfpathclose%
\pgfusepath{fill}%
\end{pgfscope}%
\begin{pgfscope}%
\pgfsetbuttcap%
\pgfsetroundjoin%
\definecolor{currentfill}{rgb}{0.000000,0.000000,0.000000}%
\pgfsetfillcolor{currentfill}%
\pgfsetlinewidth{0.803000pt}%
\definecolor{currentstroke}{rgb}{0.000000,0.000000,0.000000}%
\pgfsetstrokecolor{currentstroke}%
\pgfsetdash{}{0pt}%
\pgfsys@defobject{currentmarker}{\pgfqpoint{0.000000in}{-0.048611in}}{\pgfqpoint{0.000000in}{0.000000in}}{%
\pgfpathmoveto{\pgfqpoint{0.000000in}{0.000000in}}%
\pgfpathlineto{\pgfqpoint{0.000000in}{-0.048611in}}%
\pgfusepath{stroke,fill}%
}%
\begin{pgfscope}%
\pgfsys@transformshift{0.624035in}{0.499444in}%
\pgfsys@useobject{currentmarker}{}%
\end{pgfscope}%
\end{pgfscope}%
\begin{pgfscope}%
\definecolor{textcolor}{rgb}{0.000000,0.000000,0.000000}%
\pgfsetstrokecolor{textcolor}%
\pgfsetfillcolor{textcolor}%
\pgftext[x=0.624035in,y=0.402222in,,top]{\color{textcolor}\rmfamily\fontsize{10.000000}{12.000000}\selectfont \(\displaystyle {0.0}\)}%
\end{pgfscope}%
\begin{pgfscope}%
\pgfsetbuttcap%
\pgfsetroundjoin%
\definecolor{currentfill}{rgb}{0.000000,0.000000,0.000000}%
\pgfsetfillcolor{currentfill}%
\pgfsetlinewidth{0.803000pt}%
\definecolor{currentstroke}{rgb}{0.000000,0.000000,0.000000}%
\pgfsetstrokecolor{currentstroke}%
\pgfsetdash{}{0pt}%
\pgfsys@defobject{currentmarker}{\pgfqpoint{0.000000in}{-0.048611in}}{\pgfqpoint{0.000000in}{0.000000in}}{%
\pgfpathmoveto{\pgfqpoint{0.000000in}{0.000000in}}%
\pgfpathlineto{\pgfqpoint{0.000000in}{-0.048611in}}%
\pgfusepath{stroke,fill}%
}%
\begin{pgfscope}%
\pgfsys@transformshift{1.328581in}{0.499444in}%
\pgfsys@useobject{currentmarker}{}%
\end{pgfscope}%
\end{pgfscope}%
\begin{pgfscope}%
\definecolor{textcolor}{rgb}{0.000000,0.000000,0.000000}%
\pgfsetstrokecolor{textcolor}%
\pgfsetfillcolor{textcolor}%
\pgftext[x=1.328581in,y=0.402222in,,top]{\color{textcolor}\rmfamily\fontsize{10.000000}{12.000000}\selectfont \(\displaystyle {0.5}\)}%
\end{pgfscope}%
\begin{pgfscope}%
\pgfsetbuttcap%
\pgfsetroundjoin%
\definecolor{currentfill}{rgb}{0.000000,0.000000,0.000000}%
\pgfsetfillcolor{currentfill}%
\pgfsetlinewidth{0.803000pt}%
\definecolor{currentstroke}{rgb}{0.000000,0.000000,0.000000}%
\pgfsetstrokecolor{currentstroke}%
\pgfsetdash{}{0pt}%
\pgfsys@defobject{currentmarker}{\pgfqpoint{0.000000in}{-0.048611in}}{\pgfqpoint{0.000000in}{0.000000in}}{%
\pgfpathmoveto{\pgfqpoint{0.000000in}{0.000000in}}%
\pgfpathlineto{\pgfqpoint{0.000000in}{-0.048611in}}%
\pgfusepath{stroke,fill}%
}%
\begin{pgfscope}%
\pgfsys@transformshift{2.033126in}{0.499444in}%
\pgfsys@useobject{currentmarker}{}%
\end{pgfscope}%
\end{pgfscope}%
\begin{pgfscope}%
\definecolor{textcolor}{rgb}{0.000000,0.000000,0.000000}%
\pgfsetstrokecolor{textcolor}%
\pgfsetfillcolor{textcolor}%
\pgftext[x=2.033126in,y=0.402222in,,top]{\color{textcolor}\rmfamily\fontsize{10.000000}{12.000000}\selectfont \(\displaystyle {1.0}\)}%
\end{pgfscope}%
\begin{pgfscope}%
\definecolor{textcolor}{rgb}{0.000000,0.000000,0.000000}%
\pgfsetstrokecolor{textcolor}%
\pgfsetfillcolor{textcolor}%
\pgftext[x=1.328581in,y=0.223333in,,top]{\color{textcolor}\rmfamily\fontsize{10.000000}{12.000000}\selectfont False positive rate}%
\end{pgfscope}%
\begin{pgfscope}%
\pgfsetbuttcap%
\pgfsetroundjoin%
\definecolor{currentfill}{rgb}{0.000000,0.000000,0.000000}%
\pgfsetfillcolor{currentfill}%
\pgfsetlinewidth{0.803000pt}%
\definecolor{currentstroke}{rgb}{0.000000,0.000000,0.000000}%
\pgfsetstrokecolor{currentstroke}%
\pgfsetdash{}{0pt}%
\pgfsys@defobject{currentmarker}{\pgfqpoint{-0.048611in}{0.000000in}}{\pgfqpoint{-0.000000in}{0.000000in}}{%
\pgfpathmoveto{\pgfqpoint{-0.000000in}{0.000000in}}%
\pgfpathlineto{\pgfqpoint{-0.048611in}{0.000000in}}%
\pgfusepath{stroke,fill}%
}%
\begin{pgfscope}%
\pgfsys@transformshift{0.553581in}{0.551944in}%
\pgfsys@useobject{currentmarker}{}%
\end{pgfscope}%
\end{pgfscope}%
\begin{pgfscope}%
\definecolor{textcolor}{rgb}{0.000000,0.000000,0.000000}%
\pgfsetstrokecolor{textcolor}%
\pgfsetfillcolor{textcolor}%
\pgftext[x=0.278889in, y=0.503750in, left, base]{\color{textcolor}\rmfamily\fontsize{10.000000}{12.000000}\selectfont \(\displaystyle {0.0}\)}%
\end{pgfscope}%
\begin{pgfscope}%
\pgfsetbuttcap%
\pgfsetroundjoin%
\definecolor{currentfill}{rgb}{0.000000,0.000000,0.000000}%
\pgfsetfillcolor{currentfill}%
\pgfsetlinewidth{0.803000pt}%
\definecolor{currentstroke}{rgb}{0.000000,0.000000,0.000000}%
\pgfsetstrokecolor{currentstroke}%
\pgfsetdash{}{0pt}%
\pgfsys@defobject{currentmarker}{\pgfqpoint{-0.048611in}{0.000000in}}{\pgfqpoint{-0.000000in}{0.000000in}}{%
\pgfpathmoveto{\pgfqpoint{-0.000000in}{0.000000in}}%
\pgfpathlineto{\pgfqpoint{-0.048611in}{0.000000in}}%
\pgfusepath{stroke,fill}%
}%
\begin{pgfscope}%
\pgfsys@transformshift{0.553581in}{1.076944in}%
\pgfsys@useobject{currentmarker}{}%
\end{pgfscope}%
\end{pgfscope}%
\begin{pgfscope}%
\definecolor{textcolor}{rgb}{0.000000,0.000000,0.000000}%
\pgfsetstrokecolor{textcolor}%
\pgfsetfillcolor{textcolor}%
\pgftext[x=0.278889in, y=1.028750in, left, base]{\color{textcolor}\rmfamily\fontsize{10.000000}{12.000000}\selectfont \(\displaystyle {0.5}\)}%
\end{pgfscope}%
\begin{pgfscope}%
\pgfsetbuttcap%
\pgfsetroundjoin%
\definecolor{currentfill}{rgb}{0.000000,0.000000,0.000000}%
\pgfsetfillcolor{currentfill}%
\pgfsetlinewidth{0.803000pt}%
\definecolor{currentstroke}{rgb}{0.000000,0.000000,0.000000}%
\pgfsetstrokecolor{currentstroke}%
\pgfsetdash{}{0pt}%
\pgfsys@defobject{currentmarker}{\pgfqpoint{-0.048611in}{0.000000in}}{\pgfqpoint{-0.000000in}{0.000000in}}{%
\pgfpathmoveto{\pgfqpoint{-0.000000in}{0.000000in}}%
\pgfpathlineto{\pgfqpoint{-0.048611in}{0.000000in}}%
\pgfusepath{stroke,fill}%
}%
\begin{pgfscope}%
\pgfsys@transformshift{0.553581in}{1.601944in}%
\pgfsys@useobject{currentmarker}{}%
\end{pgfscope}%
\end{pgfscope}%
\begin{pgfscope}%
\definecolor{textcolor}{rgb}{0.000000,0.000000,0.000000}%
\pgfsetstrokecolor{textcolor}%
\pgfsetfillcolor{textcolor}%
\pgftext[x=0.278889in, y=1.553750in, left, base]{\color{textcolor}\rmfamily\fontsize{10.000000}{12.000000}\selectfont \(\displaystyle {1.0}\)}%
\end{pgfscope}%
\begin{pgfscope}%
\definecolor{textcolor}{rgb}{0.000000,0.000000,0.000000}%
\pgfsetstrokecolor{textcolor}%
\pgfsetfillcolor{textcolor}%
\pgftext[x=0.223333in,y=1.076944in,,bottom,rotate=90.000000]{\color{textcolor}\rmfamily\fontsize{10.000000}{12.000000}\selectfont True positive rate}%
\end{pgfscope}%
\begin{pgfscope}%
\pgfpathrectangle{\pgfqpoint{0.553581in}{0.499444in}}{\pgfqpoint{1.550000in}{1.155000in}}%
\pgfusepath{clip}%
\pgfsetbuttcap%
\pgfsetroundjoin%
\pgfsetlinewidth{1.505625pt}%
\definecolor{currentstroke}{rgb}{0.000000,0.000000,0.000000}%
\pgfsetstrokecolor{currentstroke}%
\pgfsetdash{{5.550000pt}{2.400000pt}}{0.000000pt}%
\pgfpathmoveto{\pgfqpoint{0.624035in}{0.551944in}}%
\pgfpathlineto{\pgfqpoint{2.033126in}{1.601944in}}%
\pgfusepath{stroke}%
\end{pgfscope}%
\begin{pgfscope}%
\pgfpathrectangle{\pgfqpoint{0.553581in}{0.499444in}}{\pgfqpoint{1.550000in}{1.155000in}}%
\pgfusepath{clip}%
\pgfsetrectcap%
\pgfsetroundjoin%
\pgfsetlinewidth{1.505625pt}%
\definecolor{currentstroke}{rgb}{0.000000,0.000000,0.000000}%
\pgfsetstrokecolor{currentstroke}%
\pgfsetdash{}{0pt}%
\pgfpathmoveto{\pgfqpoint{0.624035in}{0.551944in}}%
\pgfpathlineto{\pgfqpoint{0.625818in}{0.578454in}}%
\pgfpathlineto{\pgfqpoint{0.635863in}{0.672977in}}%
\pgfpathlineto{\pgfqpoint{0.642954in}{0.717523in}}%
\pgfpathlineto{\pgfqpoint{0.652061in}{0.766011in}}%
\pgfpathlineto{\pgfqpoint{0.660082in}{0.797456in}}%
\pgfpathlineto{\pgfqpoint{0.680307in}{0.866246in}}%
\pgfpathlineto{\pgfqpoint{0.692854in}{0.901510in}}%
\pgfpathlineto{\pgfqpoint{0.692924in}{0.901789in}}%
\pgfpathlineto{\pgfqpoint{0.715377in}{0.955057in}}%
\pgfpathlineto{\pgfqpoint{0.723835in}{0.974645in}}%
\pgfpathlineto{\pgfqpoint{0.765613in}{1.048215in}}%
\pgfpathlineto{\pgfqpoint{0.817389in}{1.123306in}}%
\pgfpathlineto{\pgfqpoint{0.847119in}{1.160680in}}%
\pgfpathlineto{\pgfqpoint{0.879485in}{1.196068in}}%
\pgfpathlineto{\pgfqpoint{0.880837in}{1.197248in}}%
\pgfpathlineto{\pgfqpoint{0.881064in}{1.197465in}}%
\pgfpathlineto{\pgfqpoint{0.936850in}{1.250454in}}%
\pgfpathlineto{\pgfqpoint{0.977643in}{1.283607in}}%
\pgfpathlineto{\pgfqpoint{0.999563in}{1.300153in}}%
\pgfpathlineto{\pgfqpoint{1.092633in}{1.363292in}}%
\pgfpathlineto{\pgfqpoint{1.116703in}{1.376827in}}%
\pgfpathlineto{\pgfqpoint{1.141766in}{1.389989in}}%
\pgfpathlineto{\pgfqpoint{1.167416in}{1.404609in}}%
\pgfpathlineto{\pgfqpoint{1.246452in}{1.440215in}}%
\pgfpathlineto{\pgfqpoint{1.273478in}{1.452507in}}%
\pgfpathlineto{\pgfqpoint{1.299753in}{1.462689in}}%
\pgfpathlineto{\pgfqpoint{1.438524in}{1.510618in}}%
\pgfpathlineto{\pgfqpoint{1.465792in}{1.519403in}}%
\pgfpathlineto{\pgfqpoint{1.521602in}{1.533651in}}%
\pgfpathlineto{\pgfqpoint{1.576935in}{1.547000in}}%
\pgfpathlineto{\pgfqpoint{1.707928in}{1.571678in}}%
\pgfpathlineto{\pgfqpoint{1.802459in}{1.584933in}}%
\pgfpathlineto{\pgfqpoint{1.865414in}{1.591762in}}%
\pgfpathlineto{\pgfqpoint{1.936570in}{1.597629in}}%
\pgfpathlineto{\pgfqpoint{1.989793in}{1.600392in}}%
\pgfpathlineto{\pgfqpoint{2.033126in}{1.601944in}}%
\pgfpathlineto{\pgfqpoint{2.033126in}{1.601944in}}%
\pgfusepath{stroke}%
\end{pgfscope}%
\begin{pgfscope}%
\pgfsetrectcap%
\pgfsetmiterjoin%
\pgfsetlinewidth{0.803000pt}%
\definecolor{currentstroke}{rgb}{0.000000,0.000000,0.000000}%
\pgfsetstrokecolor{currentstroke}%
\pgfsetdash{}{0pt}%
\pgfpathmoveto{\pgfqpoint{0.553581in}{0.499444in}}%
\pgfpathlineto{\pgfqpoint{0.553581in}{1.654444in}}%
\pgfusepath{stroke}%
\end{pgfscope}%
\begin{pgfscope}%
\pgfsetrectcap%
\pgfsetmiterjoin%
\pgfsetlinewidth{0.803000pt}%
\definecolor{currentstroke}{rgb}{0.000000,0.000000,0.000000}%
\pgfsetstrokecolor{currentstroke}%
\pgfsetdash{}{0pt}%
\pgfpathmoveto{\pgfqpoint{2.103581in}{0.499444in}}%
\pgfpathlineto{\pgfqpoint{2.103581in}{1.654444in}}%
\pgfusepath{stroke}%
\end{pgfscope}%
\begin{pgfscope}%
\pgfsetrectcap%
\pgfsetmiterjoin%
\pgfsetlinewidth{0.803000pt}%
\definecolor{currentstroke}{rgb}{0.000000,0.000000,0.000000}%
\pgfsetstrokecolor{currentstroke}%
\pgfsetdash{}{0pt}%
\pgfpathmoveto{\pgfqpoint{0.553581in}{0.499444in}}%
\pgfpathlineto{\pgfqpoint{2.103581in}{0.499444in}}%
\pgfusepath{stroke}%
\end{pgfscope}%
\begin{pgfscope}%
\pgfsetrectcap%
\pgfsetmiterjoin%
\pgfsetlinewidth{0.803000pt}%
\definecolor{currentstroke}{rgb}{0.000000,0.000000,0.000000}%
\pgfsetstrokecolor{currentstroke}%
\pgfsetdash{}{0pt}%
\pgfpathmoveto{\pgfqpoint{0.553581in}{1.654444in}}%
\pgfpathlineto{\pgfqpoint{2.103581in}{1.654444in}}%
\pgfusepath{stroke}%
\end{pgfscope}%
\begin{pgfscope}%
\pgfsetbuttcap%
\pgfsetmiterjoin%
\definecolor{currentfill}{rgb}{1.000000,1.000000,1.000000}%
\pgfsetfillcolor{currentfill}%
\pgfsetfillopacity{0.800000}%
\pgfsetlinewidth{1.003750pt}%
\definecolor{currentstroke}{rgb}{0.800000,0.800000,0.800000}%
\pgfsetstrokecolor{currentstroke}%
\pgfsetstrokeopacity{0.800000}%
\pgfsetdash{}{0pt}%
\pgfpathmoveto{\pgfqpoint{0.832747in}{0.568889in}}%
\pgfpathlineto{\pgfqpoint{2.006358in}{0.568889in}}%
\pgfpathquadraticcurveto{\pgfqpoint{2.034136in}{0.568889in}}{\pgfqpoint{2.034136in}{0.596666in}}%
\pgfpathlineto{\pgfqpoint{2.034136in}{0.776388in}}%
\pgfpathquadraticcurveto{\pgfqpoint{2.034136in}{0.804166in}}{\pgfqpoint{2.006358in}{0.804166in}}%
\pgfpathlineto{\pgfqpoint{0.832747in}{0.804166in}}%
\pgfpathquadraticcurveto{\pgfqpoint{0.804970in}{0.804166in}}{\pgfqpoint{0.804970in}{0.776388in}}%
\pgfpathlineto{\pgfqpoint{0.804970in}{0.596666in}}%
\pgfpathquadraticcurveto{\pgfqpoint{0.804970in}{0.568889in}}{\pgfqpoint{0.832747in}{0.568889in}}%
\pgfpathlineto{\pgfqpoint{0.832747in}{0.568889in}}%
\pgfpathclose%
\pgfusepath{stroke,fill}%
\end{pgfscope}%
\begin{pgfscope}%
\pgfsetrectcap%
\pgfsetroundjoin%
\pgfsetlinewidth{1.505625pt}%
\definecolor{currentstroke}{rgb}{0.000000,0.000000,0.000000}%
\pgfsetstrokecolor{currentstroke}%
\pgfsetdash{}{0pt}%
\pgfpathmoveto{\pgfqpoint{0.860525in}{0.700000in}}%
\pgfpathlineto{\pgfqpoint{0.999414in}{0.700000in}}%
\pgfpathlineto{\pgfqpoint{1.138303in}{0.700000in}}%
\pgfusepath{stroke}%
\end{pgfscope}%
\begin{pgfscope}%
\definecolor{textcolor}{rgb}{0.000000,0.000000,0.000000}%
\pgfsetstrokecolor{textcolor}%
\pgfsetfillcolor{textcolor}%
\pgftext[x=1.249414in,y=0.651388in,left,base]{\color{textcolor}\rmfamily\fontsize{10.000000}{12.000000}\selectfont AUC=0.799}%
\end{pgfscope}%
\end{pgfpicture}%
\makeatother%
\endgroup%

\end{tabular}

\

%
\verb|KBFC_Hard_Tomek_0_alpha_0_5_gamma_0_0_v1_Test|

\

This model is almost as effective at separating the two classes ($\text{ROC}=0.778$), but the distribution is skewed to the left.  Its results were nearly continuous, with the 214,070 samples returning 210,157 unique values of $p$, so we can fine tune the decision threshold.  

\noindent\begin{tabular}{@{\hspace{-6pt}}p{4.3in} @{\hspace{-6pt}}p{2.0in}}
	\vskip 0pt
	\hfil Raw Model Output
	
	%% Creator: Matplotlib, PGF backend
%%
%% To include the figure in your LaTeX document, write
%%   \input{<filename>.pgf}
%%
%% Make sure the required packages are loaded in your preamble
%%   \usepackage{pgf}
%%
%% Also ensure that all the required font packages are loaded; for instance,
%% the lmodern package is sometimes necessary when using math font.
%%   \usepackage{lmodern}
%%
%% Figures using additional raster images can only be included by \input if
%% they are in the same directory as the main LaTeX file. For loading figures
%% from other directories you can use the `import` package
%%   \usepackage{import}
%%
%% and then include the figures with
%%   \import{<path to file>}{<filename>.pgf}
%%
%% Matplotlib used the following preamble
%%   
%%   \usepackage{fontspec}
%%   \makeatletter\@ifpackageloaded{underscore}{}{\usepackage[strings]{underscore}}\makeatother
%%
\begingroup%
\makeatletter%
\begin{pgfpicture}%
\pgfpathrectangle{\pgfpointorigin}{\pgfqpoint{4.578750in}{1.754444in}}%
\pgfusepath{use as bounding box, clip}%
\begin{pgfscope}%
\pgfsetbuttcap%
\pgfsetmiterjoin%
\definecolor{currentfill}{rgb}{1.000000,1.000000,1.000000}%
\pgfsetfillcolor{currentfill}%
\pgfsetlinewidth{0.000000pt}%
\definecolor{currentstroke}{rgb}{1.000000,1.000000,1.000000}%
\pgfsetstrokecolor{currentstroke}%
\pgfsetdash{}{0pt}%
\pgfpathmoveto{\pgfqpoint{0.000000in}{0.000000in}}%
\pgfpathlineto{\pgfqpoint{4.578750in}{0.000000in}}%
\pgfpathlineto{\pgfqpoint{4.578750in}{1.754444in}}%
\pgfpathlineto{\pgfqpoint{0.000000in}{1.754444in}}%
\pgfpathlineto{\pgfqpoint{0.000000in}{0.000000in}}%
\pgfpathclose%
\pgfusepath{fill}%
\end{pgfscope}%
\begin{pgfscope}%
\pgfsetbuttcap%
\pgfsetmiterjoin%
\definecolor{currentfill}{rgb}{1.000000,1.000000,1.000000}%
\pgfsetfillcolor{currentfill}%
\pgfsetlinewidth{0.000000pt}%
\definecolor{currentstroke}{rgb}{0.000000,0.000000,0.000000}%
\pgfsetstrokecolor{currentstroke}%
\pgfsetstrokeopacity{0.000000}%
\pgfsetdash{}{0pt}%
\pgfpathmoveto{\pgfqpoint{0.515000in}{0.499444in}}%
\pgfpathlineto{\pgfqpoint{4.390000in}{0.499444in}}%
\pgfpathlineto{\pgfqpoint{4.390000in}{1.654444in}}%
\pgfpathlineto{\pgfqpoint{0.515000in}{1.654444in}}%
\pgfpathlineto{\pgfqpoint{0.515000in}{0.499444in}}%
\pgfpathclose%
\pgfusepath{fill}%
\end{pgfscope}%
\begin{pgfscope}%
\pgfpathrectangle{\pgfqpoint{0.515000in}{0.499444in}}{\pgfqpoint{3.875000in}{1.155000in}}%
\pgfusepath{clip}%
\pgfsetbuttcap%
\pgfsetmiterjoin%
\pgfsetlinewidth{1.003750pt}%
\definecolor{currentstroke}{rgb}{0.000000,0.000000,0.000000}%
\pgfsetstrokecolor{currentstroke}%
\pgfsetdash{}{0pt}%
\pgfpathmoveto{\pgfqpoint{0.505000in}{0.499444in}}%
\pgfpathlineto{\pgfqpoint{0.553367in}{0.499444in}}%
\pgfpathlineto{\pgfqpoint{0.553367in}{1.364137in}}%
\pgfpathlineto{\pgfqpoint{0.505000in}{1.364137in}}%
\pgfusepath{stroke}%
\end{pgfscope}%
\begin{pgfscope}%
\pgfpathrectangle{\pgfqpoint{0.515000in}{0.499444in}}{\pgfqpoint{3.875000in}{1.155000in}}%
\pgfusepath{clip}%
\pgfsetbuttcap%
\pgfsetmiterjoin%
\pgfsetlinewidth{1.003750pt}%
\definecolor{currentstroke}{rgb}{0.000000,0.000000,0.000000}%
\pgfsetstrokecolor{currentstroke}%
\pgfsetdash{}{0pt}%
\pgfpathmoveto{\pgfqpoint{0.645446in}{0.499444in}}%
\pgfpathlineto{\pgfqpoint{0.706832in}{0.499444in}}%
\pgfpathlineto{\pgfqpoint{0.706832in}{1.599444in}}%
\pgfpathlineto{\pgfqpoint{0.645446in}{1.599444in}}%
\pgfpathlineto{\pgfqpoint{0.645446in}{0.499444in}}%
\pgfpathclose%
\pgfusepath{stroke}%
\end{pgfscope}%
\begin{pgfscope}%
\pgfpathrectangle{\pgfqpoint{0.515000in}{0.499444in}}{\pgfqpoint{3.875000in}{1.155000in}}%
\pgfusepath{clip}%
\pgfsetbuttcap%
\pgfsetmiterjoin%
\pgfsetlinewidth{1.003750pt}%
\definecolor{currentstroke}{rgb}{0.000000,0.000000,0.000000}%
\pgfsetstrokecolor{currentstroke}%
\pgfsetdash{}{0pt}%
\pgfpathmoveto{\pgfqpoint{0.798911in}{0.499444in}}%
\pgfpathlineto{\pgfqpoint{0.860297in}{0.499444in}}%
\pgfpathlineto{\pgfqpoint{0.860297in}{1.444405in}}%
\pgfpathlineto{\pgfqpoint{0.798911in}{1.444405in}}%
\pgfpathlineto{\pgfqpoint{0.798911in}{0.499444in}}%
\pgfpathclose%
\pgfusepath{stroke}%
\end{pgfscope}%
\begin{pgfscope}%
\pgfpathrectangle{\pgfqpoint{0.515000in}{0.499444in}}{\pgfqpoint{3.875000in}{1.155000in}}%
\pgfusepath{clip}%
\pgfsetbuttcap%
\pgfsetmiterjoin%
\pgfsetlinewidth{1.003750pt}%
\definecolor{currentstroke}{rgb}{0.000000,0.000000,0.000000}%
\pgfsetstrokecolor{currentstroke}%
\pgfsetdash{}{0pt}%
\pgfpathmoveto{\pgfqpoint{0.952377in}{0.499444in}}%
\pgfpathlineto{\pgfqpoint{1.013763in}{0.499444in}}%
\pgfpathlineto{\pgfqpoint{1.013763in}{1.297632in}}%
\pgfpathlineto{\pgfqpoint{0.952377in}{1.297632in}}%
\pgfpathlineto{\pgfqpoint{0.952377in}{0.499444in}}%
\pgfpathclose%
\pgfusepath{stroke}%
\end{pgfscope}%
\begin{pgfscope}%
\pgfpathrectangle{\pgfqpoint{0.515000in}{0.499444in}}{\pgfqpoint{3.875000in}{1.155000in}}%
\pgfusepath{clip}%
\pgfsetbuttcap%
\pgfsetmiterjoin%
\pgfsetlinewidth{1.003750pt}%
\definecolor{currentstroke}{rgb}{0.000000,0.000000,0.000000}%
\pgfsetstrokecolor{currentstroke}%
\pgfsetdash{}{0pt}%
\pgfpathmoveto{\pgfqpoint{1.105842in}{0.499444in}}%
\pgfpathlineto{\pgfqpoint{1.167228in}{0.499444in}}%
\pgfpathlineto{\pgfqpoint{1.167228in}{1.163531in}}%
\pgfpathlineto{\pgfqpoint{1.105842in}{1.163531in}}%
\pgfpathlineto{\pgfqpoint{1.105842in}{0.499444in}}%
\pgfpathclose%
\pgfusepath{stroke}%
\end{pgfscope}%
\begin{pgfscope}%
\pgfpathrectangle{\pgfqpoint{0.515000in}{0.499444in}}{\pgfqpoint{3.875000in}{1.155000in}}%
\pgfusepath{clip}%
\pgfsetbuttcap%
\pgfsetmiterjoin%
\pgfsetlinewidth{1.003750pt}%
\definecolor{currentstroke}{rgb}{0.000000,0.000000,0.000000}%
\pgfsetstrokecolor{currentstroke}%
\pgfsetdash{}{0pt}%
\pgfpathmoveto{\pgfqpoint{1.259307in}{0.499444in}}%
\pgfpathlineto{\pgfqpoint{1.320693in}{0.499444in}}%
\pgfpathlineto{\pgfqpoint{1.320693in}{1.056577in}}%
\pgfpathlineto{\pgfqpoint{1.259307in}{1.056577in}}%
\pgfpathlineto{\pgfqpoint{1.259307in}{0.499444in}}%
\pgfpathclose%
\pgfusepath{stroke}%
\end{pgfscope}%
\begin{pgfscope}%
\pgfpathrectangle{\pgfqpoint{0.515000in}{0.499444in}}{\pgfqpoint{3.875000in}{1.155000in}}%
\pgfusepath{clip}%
\pgfsetbuttcap%
\pgfsetmiterjoin%
\pgfsetlinewidth{1.003750pt}%
\definecolor{currentstroke}{rgb}{0.000000,0.000000,0.000000}%
\pgfsetstrokecolor{currentstroke}%
\pgfsetdash{}{0pt}%
\pgfpathmoveto{\pgfqpoint{1.412773in}{0.499444in}}%
\pgfpathlineto{\pgfqpoint{1.474159in}{0.499444in}}%
\pgfpathlineto{\pgfqpoint{1.474159in}{0.962841in}}%
\pgfpathlineto{\pgfqpoint{1.412773in}{0.962841in}}%
\pgfpathlineto{\pgfqpoint{1.412773in}{0.499444in}}%
\pgfpathclose%
\pgfusepath{stroke}%
\end{pgfscope}%
\begin{pgfscope}%
\pgfpathrectangle{\pgfqpoint{0.515000in}{0.499444in}}{\pgfqpoint{3.875000in}{1.155000in}}%
\pgfusepath{clip}%
\pgfsetbuttcap%
\pgfsetmiterjoin%
\pgfsetlinewidth{1.003750pt}%
\definecolor{currentstroke}{rgb}{0.000000,0.000000,0.000000}%
\pgfsetstrokecolor{currentstroke}%
\pgfsetdash{}{0pt}%
\pgfpathmoveto{\pgfqpoint{1.566238in}{0.499444in}}%
\pgfpathlineto{\pgfqpoint{1.627624in}{0.499444in}}%
\pgfpathlineto{\pgfqpoint{1.627624in}{0.885300in}}%
\pgfpathlineto{\pgfqpoint{1.566238in}{0.885300in}}%
\pgfpathlineto{\pgfqpoint{1.566238in}{0.499444in}}%
\pgfpathclose%
\pgfusepath{stroke}%
\end{pgfscope}%
\begin{pgfscope}%
\pgfpathrectangle{\pgfqpoint{0.515000in}{0.499444in}}{\pgfqpoint{3.875000in}{1.155000in}}%
\pgfusepath{clip}%
\pgfsetbuttcap%
\pgfsetmiterjoin%
\pgfsetlinewidth{1.003750pt}%
\definecolor{currentstroke}{rgb}{0.000000,0.000000,0.000000}%
\pgfsetstrokecolor{currentstroke}%
\pgfsetdash{}{0pt}%
\pgfpathmoveto{\pgfqpoint{1.719703in}{0.499444in}}%
\pgfpathlineto{\pgfqpoint{1.781089in}{0.499444in}}%
\pgfpathlineto{\pgfqpoint{1.781089in}{0.822571in}}%
\pgfpathlineto{\pgfqpoint{1.719703in}{0.822571in}}%
\pgfpathlineto{\pgfqpoint{1.719703in}{0.499444in}}%
\pgfpathclose%
\pgfusepath{stroke}%
\end{pgfscope}%
\begin{pgfscope}%
\pgfpathrectangle{\pgfqpoint{0.515000in}{0.499444in}}{\pgfqpoint{3.875000in}{1.155000in}}%
\pgfusepath{clip}%
\pgfsetbuttcap%
\pgfsetmiterjoin%
\pgfsetlinewidth{1.003750pt}%
\definecolor{currentstroke}{rgb}{0.000000,0.000000,0.000000}%
\pgfsetstrokecolor{currentstroke}%
\pgfsetdash{}{0pt}%
\pgfpathmoveto{\pgfqpoint{1.873169in}{0.499444in}}%
\pgfpathlineto{\pgfqpoint{1.934555in}{0.499444in}}%
\pgfpathlineto{\pgfqpoint{1.934555in}{0.777759in}}%
\pgfpathlineto{\pgfqpoint{1.873169in}{0.777759in}}%
\pgfpathlineto{\pgfqpoint{1.873169in}{0.499444in}}%
\pgfpathclose%
\pgfusepath{stroke}%
\end{pgfscope}%
\begin{pgfscope}%
\pgfpathrectangle{\pgfqpoint{0.515000in}{0.499444in}}{\pgfqpoint{3.875000in}{1.155000in}}%
\pgfusepath{clip}%
\pgfsetbuttcap%
\pgfsetmiterjoin%
\pgfsetlinewidth{1.003750pt}%
\definecolor{currentstroke}{rgb}{0.000000,0.000000,0.000000}%
\pgfsetstrokecolor{currentstroke}%
\pgfsetdash{}{0pt}%
\pgfpathmoveto{\pgfqpoint{2.026634in}{0.499444in}}%
\pgfpathlineto{\pgfqpoint{2.088020in}{0.499444in}}%
\pgfpathlineto{\pgfqpoint{2.088020in}{0.737017in}}%
\pgfpathlineto{\pgfqpoint{2.026634in}{0.737017in}}%
\pgfpathlineto{\pgfqpoint{2.026634in}{0.499444in}}%
\pgfpathclose%
\pgfusepath{stroke}%
\end{pgfscope}%
\begin{pgfscope}%
\pgfpathrectangle{\pgfqpoint{0.515000in}{0.499444in}}{\pgfqpoint{3.875000in}{1.155000in}}%
\pgfusepath{clip}%
\pgfsetbuttcap%
\pgfsetmiterjoin%
\pgfsetlinewidth{1.003750pt}%
\definecolor{currentstroke}{rgb}{0.000000,0.000000,0.000000}%
\pgfsetstrokecolor{currentstroke}%
\pgfsetdash{}{0pt}%
\pgfpathmoveto{\pgfqpoint{2.180099in}{0.499444in}}%
\pgfpathlineto{\pgfqpoint{2.241485in}{0.499444in}}%
\pgfpathlineto{\pgfqpoint{2.241485in}{0.699841in}}%
\pgfpathlineto{\pgfqpoint{2.180099in}{0.699841in}}%
\pgfpathlineto{\pgfqpoint{2.180099in}{0.499444in}}%
\pgfpathclose%
\pgfusepath{stroke}%
\end{pgfscope}%
\begin{pgfscope}%
\pgfpathrectangle{\pgfqpoint{0.515000in}{0.499444in}}{\pgfqpoint{3.875000in}{1.155000in}}%
\pgfusepath{clip}%
\pgfsetbuttcap%
\pgfsetmiterjoin%
\pgfsetlinewidth{1.003750pt}%
\definecolor{currentstroke}{rgb}{0.000000,0.000000,0.000000}%
\pgfsetstrokecolor{currentstroke}%
\pgfsetdash{}{0pt}%
\pgfpathmoveto{\pgfqpoint{2.333565in}{0.499444in}}%
\pgfpathlineto{\pgfqpoint{2.394951in}{0.499444in}}%
\pgfpathlineto{\pgfqpoint{2.394951in}{0.662329in}}%
\pgfpathlineto{\pgfqpoint{2.333565in}{0.662329in}}%
\pgfpathlineto{\pgfqpoint{2.333565in}{0.499444in}}%
\pgfpathclose%
\pgfusepath{stroke}%
\end{pgfscope}%
\begin{pgfscope}%
\pgfpathrectangle{\pgfqpoint{0.515000in}{0.499444in}}{\pgfqpoint{3.875000in}{1.155000in}}%
\pgfusepath{clip}%
\pgfsetbuttcap%
\pgfsetmiterjoin%
\pgfsetlinewidth{1.003750pt}%
\definecolor{currentstroke}{rgb}{0.000000,0.000000,0.000000}%
\pgfsetstrokecolor{currentstroke}%
\pgfsetdash{}{0pt}%
\pgfpathmoveto{\pgfqpoint{2.487030in}{0.499444in}}%
\pgfpathlineto{\pgfqpoint{2.548416in}{0.499444in}}%
\pgfpathlineto{\pgfqpoint{2.548416in}{0.633420in}}%
\pgfpathlineto{\pgfqpoint{2.487030in}{0.633420in}}%
\pgfpathlineto{\pgfqpoint{2.487030in}{0.499444in}}%
\pgfpathclose%
\pgfusepath{stroke}%
\end{pgfscope}%
\begin{pgfscope}%
\pgfpathrectangle{\pgfqpoint{0.515000in}{0.499444in}}{\pgfqpoint{3.875000in}{1.155000in}}%
\pgfusepath{clip}%
\pgfsetbuttcap%
\pgfsetmiterjoin%
\pgfsetlinewidth{1.003750pt}%
\definecolor{currentstroke}{rgb}{0.000000,0.000000,0.000000}%
\pgfsetstrokecolor{currentstroke}%
\pgfsetdash{}{0pt}%
\pgfpathmoveto{\pgfqpoint{2.640495in}{0.499444in}}%
\pgfpathlineto{\pgfqpoint{2.701881in}{0.499444in}}%
\pgfpathlineto{\pgfqpoint{2.701881in}{0.610510in}}%
\pgfpathlineto{\pgfqpoint{2.640495in}{0.610510in}}%
\pgfpathlineto{\pgfqpoint{2.640495in}{0.499444in}}%
\pgfpathclose%
\pgfusepath{stroke}%
\end{pgfscope}%
\begin{pgfscope}%
\pgfpathrectangle{\pgfqpoint{0.515000in}{0.499444in}}{\pgfqpoint{3.875000in}{1.155000in}}%
\pgfusepath{clip}%
\pgfsetbuttcap%
\pgfsetmiterjoin%
\pgfsetlinewidth{1.003750pt}%
\definecolor{currentstroke}{rgb}{0.000000,0.000000,0.000000}%
\pgfsetstrokecolor{currentstroke}%
\pgfsetdash{}{0pt}%
\pgfpathmoveto{\pgfqpoint{2.793961in}{0.499444in}}%
\pgfpathlineto{\pgfqpoint{2.855347in}{0.499444in}}%
\pgfpathlineto{\pgfqpoint{2.855347in}{0.586929in}}%
\pgfpathlineto{\pgfqpoint{2.793961in}{0.586929in}}%
\pgfpathlineto{\pgfqpoint{2.793961in}{0.499444in}}%
\pgfpathclose%
\pgfusepath{stroke}%
\end{pgfscope}%
\begin{pgfscope}%
\pgfpathrectangle{\pgfqpoint{0.515000in}{0.499444in}}{\pgfqpoint{3.875000in}{1.155000in}}%
\pgfusepath{clip}%
\pgfsetbuttcap%
\pgfsetmiterjoin%
\pgfsetlinewidth{1.003750pt}%
\definecolor{currentstroke}{rgb}{0.000000,0.000000,0.000000}%
\pgfsetstrokecolor{currentstroke}%
\pgfsetdash{}{0pt}%
\pgfpathmoveto{\pgfqpoint{2.947426in}{0.499444in}}%
\pgfpathlineto{\pgfqpoint{3.008812in}{0.499444in}}%
\pgfpathlineto{\pgfqpoint{3.008812in}{0.566453in}}%
\pgfpathlineto{\pgfqpoint{2.947426in}{0.566453in}}%
\pgfpathlineto{\pgfqpoint{2.947426in}{0.499444in}}%
\pgfpathclose%
\pgfusepath{stroke}%
\end{pgfscope}%
\begin{pgfscope}%
\pgfpathrectangle{\pgfqpoint{0.515000in}{0.499444in}}{\pgfqpoint{3.875000in}{1.155000in}}%
\pgfusepath{clip}%
\pgfsetbuttcap%
\pgfsetmiterjoin%
\pgfsetlinewidth{1.003750pt}%
\definecolor{currentstroke}{rgb}{0.000000,0.000000,0.000000}%
\pgfsetstrokecolor{currentstroke}%
\pgfsetdash{}{0pt}%
\pgfpathmoveto{\pgfqpoint{3.100891in}{0.499444in}}%
\pgfpathlineto{\pgfqpoint{3.162278in}{0.499444in}}%
\pgfpathlineto{\pgfqpoint{3.162278in}{0.551809in}}%
\pgfpathlineto{\pgfqpoint{3.100891in}{0.551809in}}%
\pgfpathlineto{\pgfqpoint{3.100891in}{0.499444in}}%
\pgfpathclose%
\pgfusepath{stroke}%
\end{pgfscope}%
\begin{pgfscope}%
\pgfpathrectangle{\pgfqpoint{0.515000in}{0.499444in}}{\pgfqpoint{3.875000in}{1.155000in}}%
\pgfusepath{clip}%
\pgfsetbuttcap%
\pgfsetmiterjoin%
\pgfsetlinewidth{1.003750pt}%
\definecolor{currentstroke}{rgb}{0.000000,0.000000,0.000000}%
\pgfsetstrokecolor{currentstroke}%
\pgfsetdash{}{0pt}%
\pgfpathmoveto{\pgfqpoint{3.254357in}{0.499444in}}%
\pgfpathlineto{\pgfqpoint{3.315743in}{0.499444in}}%
\pgfpathlineto{\pgfqpoint{3.315743in}{0.540019in}}%
\pgfpathlineto{\pgfqpoint{3.254357in}{0.540019in}}%
\pgfpathlineto{\pgfqpoint{3.254357in}{0.499444in}}%
\pgfpathclose%
\pgfusepath{stroke}%
\end{pgfscope}%
\begin{pgfscope}%
\pgfpathrectangle{\pgfqpoint{0.515000in}{0.499444in}}{\pgfqpoint{3.875000in}{1.155000in}}%
\pgfusepath{clip}%
\pgfsetbuttcap%
\pgfsetmiterjoin%
\pgfsetlinewidth{1.003750pt}%
\definecolor{currentstroke}{rgb}{0.000000,0.000000,0.000000}%
\pgfsetstrokecolor{currentstroke}%
\pgfsetdash{}{0pt}%
\pgfpathmoveto{\pgfqpoint{3.407822in}{0.499444in}}%
\pgfpathlineto{\pgfqpoint{3.469208in}{0.499444in}}%
\pgfpathlineto{\pgfqpoint{3.469208in}{0.528312in}}%
\pgfpathlineto{\pgfqpoint{3.407822in}{0.528312in}}%
\pgfpathlineto{\pgfqpoint{3.407822in}{0.499444in}}%
\pgfpathclose%
\pgfusepath{stroke}%
\end{pgfscope}%
\begin{pgfscope}%
\pgfpathrectangle{\pgfqpoint{0.515000in}{0.499444in}}{\pgfqpoint{3.875000in}{1.155000in}}%
\pgfusepath{clip}%
\pgfsetbuttcap%
\pgfsetmiterjoin%
\pgfsetlinewidth{1.003750pt}%
\definecolor{currentstroke}{rgb}{0.000000,0.000000,0.000000}%
\pgfsetstrokecolor{currentstroke}%
\pgfsetdash{}{0pt}%
\pgfpathmoveto{\pgfqpoint{3.561287in}{0.499444in}}%
\pgfpathlineto{\pgfqpoint{3.622674in}{0.499444in}}%
\pgfpathlineto{\pgfqpoint{3.622674in}{0.522522in}}%
\pgfpathlineto{\pgfqpoint{3.561287in}{0.522522in}}%
\pgfpathlineto{\pgfqpoint{3.561287in}{0.499444in}}%
\pgfpathclose%
\pgfusepath{stroke}%
\end{pgfscope}%
\begin{pgfscope}%
\pgfpathrectangle{\pgfqpoint{0.515000in}{0.499444in}}{\pgfqpoint{3.875000in}{1.155000in}}%
\pgfusepath{clip}%
\pgfsetbuttcap%
\pgfsetmiterjoin%
\pgfsetlinewidth{1.003750pt}%
\definecolor{currentstroke}{rgb}{0.000000,0.000000,0.000000}%
\pgfsetstrokecolor{currentstroke}%
\pgfsetdash{}{0pt}%
\pgfpathmoveto{\pgfqpoint{3.714753in}{0.499444in}}%
\pgfpathlineto{\pgfqpoint{3.776139in}{0.499444in}}%
\pgfpathlineto{\pgfqpoint{3.776139in}{0.517990in}}%
\pgfpathlineto{\pgfqpoint{3.714753in}{0.517990in}}%
\pgfpathlineto{\pgfqpoint{3.714753in}{0.499444in}}%
\pgfpathclose%
\pgfusepath{stroke}%
\end{pgfscope}%
\begin{pgfscope}%
\pgfpathrectangle{\pgfqpoint{0.515000in}{0.499444in}}{\pgfqpoint{3.875000in}{1.155000in}}%
\pgfusepath{clip}%
\pgfsetbuttcap%
\pgfsetmiterjoin%
\pgfsetlinewidth{1.003750pt}%
\definecolor{currentstroke}{rgb}{0.000000,0.000000,0.000000}%
\pgfsetstrokecolor{currentstroke}%
\pgfsetdash{}{0pt}%
\pgfpathmoveto{\pgfqpoint{3.868218in}{0.499444in}}%
\pgfpathlineto{\pgfqpoint{3.929604in}{0.499444in}}%
\pgfpathlineto{\pgfqpoint{3.929604in}{0.513542in}}%
\pgfpathlineto{\pgfqpoint{3.868218in}{0.513542in}}%
\pgfpathlineto{\pgfqpoint{3.868218in}{0.499444in}}%
\pgfpathclose%
\pgfusepath{stroke}%
\end{pgfscope}%
\begin{pgfscope}%
\pgfpathrectangle{\pgfqpoint{0.515000in}{0.499444in}}{\pgfqpoint{3.875000in}{1.155000in}}%
\pgfusepath{clip}%
\pgfsetbuttcap%
\pgfsetmiterjoin%
\pgfsetlinewidth{1.003750pt}%
\definecolor{currentstroke}{rgb}{0.000000,0.000000,0.000000}%
\pgfsetstrokecolor{currentstroke}%
\pgfsetdash{}{0pt}%
\pgfpathmoveto{\pgfqpoint{4.021683in}{0.499444in}}%
\pgfpathlineto{\pgfqpoint{4.083070in}{0.499444in}}%
\pgfpathlineto{\pgfqpoint{4.083070in}{0.504269in}}%
\pgfpathlineto{\pgfqpoint{4.021683in}{0.504269in}}%
\pgfpathlineto{\pgfqpoint{4.021683in}{0.499444in}}%
\pgfpathclose%
\pgfusepath{stroke}%
\end{pgfscope}%
\begin{pgfscope}%
\pgfpathrectangle{\pgfqpoint{0.515000in}{0.499444in}}{\pgfqpoint{3.875000in}{1.155000in}}%
\pgfusepath{clip}%
\pgfsetbuttcap%
\pgfsetmiterjoin%
\pgfsetlinewidth{1.003750pt}%
\definecolor{currentstroke}{rgb}{0.000000,0.000000,0.000000}%
\pgfsetstrokecolor{currentstroke}%
\pgfsetdash{}{0pt}%
\pgfpathmoveto{\pgfqpoint{4.175149in}{0.499444in}}%
\pgfpathlineto{\pgfqpoint{4.236535in}{0.499444in}}%
\pgfpathlineto{\pgfqpoint{4.236535in}{0.499864in}}%
\pgfpathlineto{\pgfqpoint{4.175149in}{0.499864in}}%
\pgfpathlineto{\pgfqpoint{4.175149in}{0.499444in}}%
\pgfpathclose%
\pgfusepath{stroke}%
\end{pgfscope}%
\begin{pgfscope}%
\pgfpathrectangle{\pgfqpoint{0.515000in}{0.499444in}}{\pgfqpoint{3.875000in}{1.155000in}}%
\pgfusepath{clip}%
\pgfsetbuttcap%
\pgfsetmiterjoin%
\definecolor{currentfill}{rgb}{0.000000,0.000000,0.000000}%
\pgfsetfillcolor{currentfill}%
\pgfsetlinewidth{0.000000pt}%
\definecolor{currentstroke}{rgb}{0.000000,0.000000,0.000000}%
\pgfsetstrokecolor{currentstroke}%
\pgfsetstrokeopacity{0.000000}%
\pgfsetdash{}{0pt}%
\pgfpathmoveto{\pgfqpoint{0.553367in}{0.499444in}}%
\pgfpathlineto{\pgfqpoint{0.614753in}{0.499444in}}%
\pgfpathlineto{\pgfqpoint{0.614753in}{0.511696in}}%
\pgfpathlineto{\pgfqpoint{0.553367in}{0.511696in}}%
\pgfpathlineto{\pgfqpoint{0.553367in}{0.499444in}}%
\pgfpathclose%
\pgfusepath{fill}%
\end{pgfscope}%
\begin{pgfscope}%
\pgfpathrectangle{\pgfqpoint{0.515000in}{0.499444in}}{\pgfqpoint{3.875000in}{1.155000in}}%
\pgfusepath{clip}%
\pgfsetbuttcap%
\pgfsetmiterjoin%
\definecolor{currentfill}{rgb}{0.000000,0.000000,0.000000}%
\pgfsetfillcolor{currentfill}%
\pgfsetlinewidth{0.000000pt}%
\definecolor{currentstroke}{rgb}{0.000000,0.000000,0.000000}%
\pgfsetstrokecolor{currentstroke}%
\pgfsetstrokeopacity{0.000000}%
\pgfsetdash{}{0pt}%
\pgfpathmoveto{\pgfqpoint{0.706832in}{0.499444in}}%
\pgfpathlineto{\pgfqpoint{0.768218in}{0.499444in}}%
\pgfpathlineto{\pgfqpoint{0.768218in}{0.543124in}}%
\pgfpathlineto{\pgfqpoint{0.706832in}{0.543124in}}%
\pgfpathlineto{\pgfqpoint{0.706832in}{0.499444in}}%
\pgfpathclose%
\pgfusepath{fill}%
\end{pgfscope}%
\begin{pgfscope}%
\pgfpathrectangle{\pgfqpoint{0.515000in}{0.499444in}}{\pgfqpoint{3.875000in}{1.155000in}}%
\pgfusepath{clip}%
\pgfsetbuttcap%
\pgfsetmiterjoin%
\definecolor{currentfill}{rgb}{0.000000,0.000000,0.000000}%
\pgfsetfillcolor{currentfill}%
\pgfsetlinewidth{0.000000pt}%
\definecolor{currentstroke}{rgb}{0.000000,0.000000,0.000000}%
\pgfsetstrokecolor{currentstroke}%
\pgfsetstrokeopacity{0.000000}%
\pgfsetdash{}{0pt}%
\pgfpathmoveto{\pgfqpoint{0.860297in}{0.499444in}}%
\pgfpathlineto{\pgfqpoint{0.921683in}{0.499444in}}%
\pgfpathlineto{\pgfqpoint{0.921683in}{0.562215in}}%
\pgfpathlineto{\pgfqpoint{0.860297in}{0.562215in}}%
\pgfpathlineto{\pgfqpoint{0.860297in}{0.499444in}}%
\pgfpathclose%
\pgfusepath{fill}%
\end{pgfscope}%
\begin{pgfscope}%
\pgfpathrectangle{\pgfqpoint{0.515000in}{0.499444in}}{\pgfqpoint{3.875000in}{1.155000in}}%
\pgfusepath{clip}%
\pgfsetbuttcap%
\pgfsetmiterjoin%
\definecolor{currentfill}{rgb}{0.000000,0.000000,0.000000}%
\pgfsetfillcolor{currentfill}%
\pgfsetlinewidth{0.000000pt}%
\definecolor{currentstroke}{rgb}{0.000000,0.000000,0.000000}%
\pgfsetstrokecolor{currentstroke}%
\pgfsetstrokeopacity{0.000000}%
\pgfsetdash{}{0pt}%
\pgfpathmoveto{\pgfqpoint{1.013763in}{0.499444in}}%
\pgfpathlineto{\pgfqpoint{1.075149in}{0.499444in}}%
\pgfpathlineto{\pgfqpoint{1.075149in}{0.570775in}}%
\pgfpathlineto{\pgfqpoint{1.013763in}{0.570775in}}%
\pgfpathlineto{\pgfqpoint{1.013763in}{0.499444in}}%
\pgfpathclose%
\pgfusepath{fill}%
\end{pgfscope}%
\begin{pgfscope}%
\pgfpathrectangle{\pgfqpoint{0.515000in}{0.499444in}}{\pgfqpoint{3.875000in}{1.155000in}}%
\pgfusepath{clip}%
\pgfsetbuttcap%
\pgfsetmiterjoin%
\definecolor{currentfill}{rgb}{0.000000,0.000000,0.000000}%
\pgfsetfillcolor{currentfill}%
\pgfsetlinewidth{0.000000pt}%
\definecolor{currentstroke}{rgb}{0.000000,0.000000,0.000000}%
\pgfsetstrokecolor{currentstroke}%
\pgfsetstrokeopacity{0.000000}%
\pgfsetdash{}{0pt}%
\pgfpathmoveto{\pgfqpoint{1.167228in}{0.499444in}}%
\pgfpathlineto{\pgfqpoint{1.228614in}{0.499444in}}%
\pgfpathlineto{\pgfqpoint{1.228614in}{0.578117in}}%
\pgfpathlineto{\pgfqpoint{1.167228in}{0.578117in}}%
\pgfpathlineto{\pgfqpoint{1.167228in}{0.499444in}}%
\pgfpathclose%
\pgfusepath{fill}%
\end{pgfscope}%
\begin{pgfscope}%
\pgfpathrectangle{\pgfqpoint{0.515000in}{0.499444in}}{\pgfqpoint{3.875000in}{1.155000in}}%
\pgfusepath{clip}%
\pgfsetbuttcap%
\pgfsetmiterjoin%
\definecolor{currentfill}{rgb}{0.000000,0.000000,0.000000}%
\pgfsetfillcolor{currentfill}%
\pgfsetlinewidth{0.000000pt}%
\definecolor{currentstroke}{rgb}{0.000000,0.000000,0.000000}%
\pgfsetstrokecolor{currentstroke}%
\pgfsetstrokeopacity{0.000000}%
\pgfsetdash{}{0pt}%
\pgfpathmoveto{\pgfqpoint{1.320693in}{0.499444in}}%
\pgfpathlineto{\pgfqpoint{1.382079in}{0.499444in}}%
\pgfpathlineto{\pgfqpoint{1.382079in}{0.578999in}}%
\pgfpathlineto{\pgfqpoint{1.320693in}{0.578999in}}%
\pgfpathlineto{\pgfqpoint{1.320693in}{0.499444in}}%
\pgfpathclose%
\pgfusepath{fill}%
\end{pgfscope}%
\begin{pgfscope}%
\pgfpathrectangle{\pgfqpoint{0.515000in}{0.499444in}}{\pgfqpoint{3.875000in}{1.155000in}}%
\pgfusepath{clip}%
\pgfsetbuttcap%
\pgfsetmiterjoin%
\definecolor{currentfill}{rgb}{0.000000,0.000000,0.000000}%
\pgfsetfillcolor{currentfill}%
\pgfsetlinewidth{0.000000pt}%
\definecolor{currentstroke}{rgb}{0.000000,0.000000,0.000000}%
\pgfsetstrokecolor{currentstroke}%
\pgfsetstrokeopacity{0.000000}%
\pgfsetdash{}{0pt}%
\pgfpathmoveto{\pgfqpoint{1.474159in}{0.499444in}}%
\pgfpathlineto{\pgfqpoint{1.535545in}{0.499444in}}%
\pgfpathlineto{\pgfqpoint{1.535545in}{0.585922in}}%
\pgfpathlineto{\pgfqpoint{1.474159in}{0.585922in}}%
\pgfpathlineto{\pgfqpoint{1.474159in}{0.499444in}}%
\pgfpathclose%
\pgfusepath{fill}%
\end{pgfscope}%
\begin{pgfscope}%
\pgfpathrectangle{\pgfqpoint{0.515000in}{0.499444in}}{\pgfqpoint{3.875000in}{1.155000in}}%
\pgfusepath{clip}%
\pgfsetbuttcap%
\pgfsetmiterjoin%
\definecolor{currentfill}{rgb}{0.000000,0.000000,0.000000}%
\pgfsetfillcolor{currentfill}%
\pgfsetlinewidth{0.000000pt}%
\definecolor{currentstroke}{rgb}{0.000000,0.000000,0.000000}%
\pgfsetstrokecolor{currentstroke}%
\pgfsetstrokeopacity{0.000000}%
\pgfsetdash{}{0pt}%
\pgfpathmoveto{\pgfqpoint{1.627624in}{0.499444in}}%
\pgfpathlineto{\pgfqpoint{1.689010in}{0.499444in}}%
\pgfpathlineto{\pgfqpoint{1.689010in}{0.580425in}}%
\pgfpathlineto{\pgfqpoint{1.627624in}{0.580425in}}%
\pgfpathlineto{\pgfqpoint{1.627624in}{0.499444in}}%
\pgfpathclose%
\pgfusepath{fill}%
\end{pgfscope}%
\begin{pgfscope}%
\pgfpathrectangle{\pgfqpoint{0.515000in}{0.499444in}}{\pgfqpoint{3.875000in}{1.155000in}}%
\pgfusepath{clip}%
\pgfsetbuttcap%
\pgfsetmiterjoin%
\definecolor{currentfill}{rgb}{0.000000,0.000000,0.000000}%
\pgfsetfillcolor{currentfill}%
\pgfsetlinewidth{0.000000pt}%
\definecolor{currentstroke}{rgb}{0.000000,0.000000,0.000000}%
\pgfsetstrokecolor{currentstroke}%
\pgfsetstrokeopacity{0.000000}%
\pgfsetdash{}{0pt}%
\pgfpathmoveto{\pgfqpoint{1.781089in}{0.499444in}}%
\pgfpathlineto{\pgfqpoint{1.842476in}{0.499444in}}%
\pgfpathlineto{\pgfqpoint{1.842476in}{0.580257in}}%
\pgfpathlineto{\pgfqpoint{1.781089in}{0.580257in}}%
\pgfpathlineto{\pgfqpoint{1.781089in}{0.499444in}}%
\pgfpathclose%
\pgfusepath{fill}%
\end{pgfscope}%
\begin{pgfscope}%
\pgfpathrectangle{\pgfqpoint{0.515000in}{0.499444in}}{\pgfqpoint{3.875000in}{1.155000in}}%
\pgfusepath{clip}%
\pgfsetbuttcap%
\pgfsetmiterjoin%
\definecolor{currentfill}{rgb}{0.000000,0.000000,0.000000}%
\pgfsetfillcolor{currentfill}%
\pgfsetlinewidth{0.000000pt}%
\definecolor{currentstroke}{rgb}{0.000000,0.000000,0.000000}%
\pgfsetstrokecolor{currentstroke}%
\pgfsetstrokeopacity{0.000000}%
\pgfsetdash{}{0pt}%
\pgfpathmoveto{\pgfqpoint{1.934555in}{0.499444in}}%
\pgfpathlineto{\pgfqpoint{1.995941in}{0.499444in}}%
\pgfpathlineto{\pgfqpoint{1.995941in}{0.579250in}}%
\pgfpathlineto{\pgfqpoint{1.934555in}{0.579250in}}%
\pgfpathlineto{\pgfqpoint{1.934555in}{0.499444in}}%
\pgfpathclose%
\pgfusepath{fill}%
\end{pgfscope}%
\begin{pgfscope}%
\pgfpathrectangle{\pgfqpoint{0.515000in}{0.499444in}}{\pgfqpoint{3.875000in}{1.155000in}}%
\pgfusepath{clip}%
\pgfsetbuttcap%
\pgfsetmiterjoin%
\definecolor{currentfill}{rgb}{0.000000,0.000000,0.000000}%
\pgfsetfillcolor{currentfill}%
\pgfsetlinewidth{0.000000pt}%
\definecolor{currentstroke}{rgb}{0.000000,0.000000,0.000000}%
\pgfsetstrokecolor{currentstroke}%
\pgfsetstrokeopacity{0.000000}%
\pgfsetdash{}{0pt}%
\pgfpathmoveto{\pgfqpoint{2.088020in}{0.499444in}}%
\pgfpathlineto{\pgfqpoint{2.149406in}{0.499444in}}%
\pgfpathlineto{\pgfqpoint{2.149406in}{0.576943in}}%
\pgfpathlineto{\pgfqpoint{2.088020in}{0.576943in}}%
\pgfpathlineto{\pgfqpoint{2.088020in}{0.499444in}}%
\pgfpathclose%
\pgfusepath{fill}%
\end{pgfscope}%
\begin{pgfscope}%
\pgfpathrectangle{\pgfqpoint{0.515000in}{0.499444in}}{\pgfqpoint{3.875000in}{1.155000in}}%
\pgfusepath{clip}%
\pgfsetbuttcap%
\pgfsetmiterjoin%
\definecolor{currentfill}{rgb}{0.000000,0.000000,0.000000}%
\pgfsetfillcolor{currentfill}%
\pgfsetlinewidth{0.000000pt}%
\definecolor{currentstroke}{rgb}{0.000000,0.000000,0.000000}%
\pgfsetstrokecolor{currentstroke}%
\pgfsetstrokeopacity{0.000000}%
\pgfsetdash{}{0pt}%
\pgfpathmoveto{\pgfqpoint{2.241485in}{0.499444in}}%
\pgfpathlineto{\pgfqpoint{2.302872in}{0.499444in}}%
\pgfpathlineto{\pgfqpoint{2.302872in}{0.574341in}}%
\pgfpathlineto{\pgfqpoint{2.241485in}{0.574341in}}%
\pgfpathlineto{\pgfqpoint{2.241485in}{0.499444in}}%
\pgfpathclose%
\pgfusepath{fill}%
\end{pgfscope}%
\begin{pgfscope}%
\pgfpathrectangle{\pgfqpoint{0.515000in}{0.499444in}}{\pgfqpoint{3.875000in}{1.155000in}}%
\pgfusepath{clip}%
\pgfsetbuttcap%
\pgfsetmiterjoin%
\definecolor{currentfill}{rgb}{0.000000,0.000000,0.000000}%
\pgfsetfillcolor{currentfill}%
\pgfsetlinewidth{0.000000pt}%
\definecolor{currentstroke}{rgb}{0.000000,0.000000,0.000000}%
\pgfsetstrokecolor{currentstroke}%
\pgfsetstrokeopacity{0.000000}%
\pgfsetdash{}{0pt}%
\pgfpathmoveto{\pgfqpoint{2.394951in}{0.499444in}}%
\pgfpathlineto{\pgfqpoint{2.456337in}{0.499444in}}%
\pgfpathlineto{\pgfqpoint{2.456337in}{0.569642in}}%
\pgfpathlineto{\pgfqpoint{2.394951in}{0.569642in}}%
\pgfpathlineto{\pgfqpoint{2.394951in}{0.499444in}}%
\pgfpathclose%
\pgfusepath{fill}%
\end{pgfscope}%
\begin{pgfscope}%
\pgfpathrectangle{\pgfqpoint{0.515000in}{0.499444in}}{\pgfqpoint{3.875000in}{1.155000in}}%
\pgfusepath{clip}%
\pgfsetbuttcap%
\pgfsetmiterjoin%
\definecolor{currentfill}{rgb}{0.000000,0.000000,0.000000}%
\pgfsetfillcolor{currentfill}%
\pgfsetlinewidth{0.000000pt}%
\definecolor{currentstroke}{rgb}{0.000000,0.000000,0.000000}%
\pgfsetstrokecolor{currentstroke}%
\pgfsetstrokeopacity{0.000000}%
\pgfsetdash{}{0pt}%
\pgfpathmoveto{\pgfqpoint{2.548416in}{0.499444in}}%
\pgfpathlineto{\pgfqpoint{2.609802in}{0.499444in}}%
\pgfpathlineto{\pgfqpoint{2.609802in}{0.567670in}}%
\pgfpathlineto{\pgfqpoint{2.548416in}{0.567670in}}%
\pgfpathlineto{\pgfqpoint{2.548416in}{0.499444in}}%
\pgfpathclose%
\pgfusepath{fill}%
\end{pgfscope}%
\begin{pgfscope}%
\pgfpathrectangle{\pgfqpoint{0.515000in}{0.499444in}}{\pgfqpoint{3.875000in}{1.155000in}}%
\pgfusepath{clip}%
\pgfsetbuttcap%
\pgfsetmiterjoin%
\definecolor{currentfill}{rgb}{0.000000,0.000000,0.000000}%
\pgfsetfillcolor{currentfill}%
\pgfsetlinewidth{0.000000pt}%
\definecolor{currentstroke}{rgb}{0.000000,0.000000,0.000000}%
\pgfsetstrokecolor{currentstroke}%
\pgfsetstrokeopacity{0.000000}%
\pgfsetdash{}{0pt}%
\pgfpathmoveto{\pgfqpoint{2.701881in}{0.499444in}}%
\pgfpathlineto{\pgfqpoint{2.763268in}{0.499444in}}%
\pgfpathlineto{\pgfqpoint{2.763268in}{0.561628in}}%
\pgfpathlineto{\pgfqpoint{2.701881in}{0.561628in}}%
\pgfpathlineto{\pgfqpoint{2.701881in}{0.499444in}}%
\pgfpathclose%
\pgfusepath{fill}%
\end{pgfscope}%
\begin{pgfscope}%
\pgfpathrectangle{\pgfqpoint{0.515000in}{0.499444in}}{\pgfqpoint{3.875000in}{1.155000in}}%
\pgfusepath{clip}%
\pgfsetbuttcap%
\pgfsetmiterjoin%
\definecolor{currentfill}{rgb}{0.000000,0.000000,0.000000}%
\pgfsetfillcolor{currentfill}%
\pgfsetlinewidth{0.000000pt}%
\definecolor{currentstroke}{rgb}{0.000000,0.000000,0.000000}%
\pgfsetstrokecolor{currentstroke}%
\pgfsetstrokeopacity{0.000000}%
\pgfsetdash{}{0pt}%
\pgfpathmoveto{\pgfqpoint{2.855347in}{0.499444in}}%
\pgfpathlineto{\pgfqpoint{2.916733in}{0.499444in}}%
\pgfpathlineto{\pgfqpoint{2.916733in}{0.561124in}}%
\pgfpathlineto{\pgfqpoint{2.855347in}{0.561124in}}%
\pgfpathlineto{\pgfqpoint{2.855347in}{0.499444in}}%
\pgfpathclose%
\pgfusepath{fill}%
\end{pgfscope}%
\begin{pgfscope}%
\pgfpathrectangle{\pgfqpoint{0.515000in}{0.499444in}}{\pgfqpoint{3.875000in}{1.155000in}}%
\pgfusepath{clip}%
\pgfsetbuttcap%
\pgfsetmiterjoin%
\definecolor{currentfill}{rgb}{0.000000,0.000000,0.000000}%
\pgfsetfillcolor{currentfill}%
\pgfsetlinewidth{0.000000pt}%
\definecolor{currentstroke}{rgb}{0.000000,0.000000,0.000000}%
\pgfsetstrokecolor{currentstroke}%
\pgfsetstrokeopacity{0.000000}%
\pgfsetdash{}{0pt}%
\pgfpathmoveto{\pgfqpoint{3.008812in}{0.499444in}}%
\pgfpathlineto{\pgfqpoint{3.070198in}{0.499444in}}%
\pgfpathlineto{\pgfqpoint{3.070198in}{0.554243in}}%
\pgfpathlineto{\pgfqpoint{3.008812in}{0.554243in}}%
\pgfpathlineto{\pgfqpoint{3.008812in}{0.499444in}}%
\pgfpathclose%
\pgfusepath{fill}%
\end{pgfscope}%
\begin{pgfscope}%
\pgfpathrectangle{\pgfqpoint{0.515000in}{0.499444in}}{\pgfqpoint{3.875000in}{1.155000in}}%
\pgfusepath{clip}%
\pgfsetbuttcap%
\pgfsetmiterjoin%
\definecolor{currentfill}{rgb}{0.000000,0.000000,0.000000}%
\pgfsetfillcolor{currentfill}%
\pgfsetlinewidth{0.000000pt}%
\definecolor{currentstroke}{rgb}{0.000000,0.000000,0.000000}%
\pgfsetstrokecolor{currentstroke}%
\pgfsetstrokeopacity{0.000000}%
\pgfsetdash{}{0pt}%
\pgfpathmoveto{\pgfqpoint{3.162278in}{0.499444in}}%
\pgfpathlineto{\pgfqpoint{3.223664in}{0.499444in}}%
\pgfpathlineto{\pgfqpoint{3.223664in}{0.550676in}}%
\pgfpathlineto{\pgfqpoint{3.162278in}{0.550676in}}%
\pgfpathlineto{\pgfqpoint{3.162278in}{0.499444in}}%
\pgfpathclose%
\pgfusepath{fill}%
\end{pgfscope}%
\begin{pgfscope}%
\pgfpathrectangle{\pgfqpoint{0.515000in}{0.499444in}}{\pgfqpoint{3.875000in}{1.155000in}}%
\pgfusepath{clip}%
\pgfsetbuttcap%
\pgfsetmiterjoin%
\definecolor{currentfill}{rgb}{0.000000,0.000000,0.000000}%
\pgfsetfillcolor{currentfill}%
\pgfsetlinewidth{0.000000pt}%
\definecolor{currentstroke}{rgb}{0.000000,0.000000,0.000000}%
\pgfsetstrokecolor{currentstroke}%
\pgfsetstrokeopacity{0.000000}%
\pgfsetdash{}{0pt}%
\pgfpathmoveto{\pgfqpoint{3.315743in}{0.499444in}}%
\pgfpathlineto{\pgfqpoint{3.377129in}{0.499444in}}%
\pgfpathlineto{\pgfqpoint{3.377129in}{0.545515in}}%
\pgfpathlineto{\pgfqpoint{3.315743in}{0.545515in}}%
\pgfpathlineto{\pgfqpoint{3.315743in}{0.499444in}}%
\pgfpathclose%
\pgfusepath{fill}%
\end{pgfscope}%
\begin{pgfscope}%
\pgfpathrectangle{\pgfqpoint{0.515000in}{0.499444in}}{\pgfqpoint{3.875000in}{1.155000in}}%
\pgfusepath{clip}%
\pgfsetbuttcap%
\pgfsetmiterjoin%
\definecolor{currentfill}{rgb}{0.000000,0.000000,0.000000}%
\pgfsetfillcolor{currentfill}%
\pgfsetlinewidth{0.000000pt}%
\definecolor{currentstroke}{rgb}{0.000000,0.000000,0.000000}%
\pgfsetstrokecolor{currentstroke}%
\pgfsetstrokeopacity{0.000000}%
\pgfsetdash{}{0pt}%
\pgfpathmoveto{\pgfqpoint{3.469208in}{0.499444in}}%
\pgfpathlineto{\pgfqpoint{3.530594in}{0.499444in}}%
\pgfpathlineto{\pgfqpoint{3.530594in}{0.539431in}}%
\pgfpathlineto{\pgfqpoint{3.469208in}{0.539431in}}%
\pgfpathlineto{\pgfqpoint{3.469208in}{0.499444in}}%
\pgfpathclose%
\pgfusepath{fill}%
\end{pgfscope}%
\begin{pgfscope}%
\pgfpathrectangle{\pgfqpoint{0.515000in}{0.499444in}}{\pgfqpoint{3.875000in}{1.155000in}}%
\pgfusepath{clip}%
\pgfsetbuttcap%
\pgfsetmiterjoin%
\definecolor{currentfill}{rgb}{0.000000,0.000000,0.000000}%
\pgfsetfillcolor{currentfill}%
\pgfsetlinewidth{0.000000pt}%
\definecolor{currentstroke}{rgb}{0.000000,0.000000,0.000000}%
\pgfsetstrokecolor{currentstroke}%
\pgfsetstrokeopacity{0.000000}%
\pgfsetdash{}{0pt}%
\pgfpathmoveto{\pgfqpoint{3.622674in}{0.499444in}}%
\pgfpathlineto{\pgfqpoint{3.684060in}{0.499444in}}%
\pgfpathlineto{\pgfqpoint{3.684060in}{0.536956in}}%
\pgfpathlineto{\pgfqpoint{3.622674in}{0.536956in}}%
\pgfpathlineto{\pgfqpoint{3.622674in}{0.499444in}}%
\pgfpathclose%
\pgfusepath{fill}%
\end{pgfscope}%
\begin{pgfscope}%
\pgfpathrectangle{\pgfqpoint{0.515000in}{0.499444in}}{\pgfqpoint{3.875000in}{1.155000in}}%
\pgfusepath{clip}%
\pgfsetbuttcap%
\pgfsetmiterjoin%
\definecolor{currentfill}{rgb}{0.000000,0.000000,0.000000}%
\pgfsetfillcolor{currentfill}%
\pgfsetlinewidth{0.000000pt}%
\definecolor{currentstroke}{rgb}{0.000000,0.000000,0.000000}%
\pgfsetstrokecolor{currentstroke}%
\pgfsetstrokeopacity{0.000000}%
\pgfsetdash{}{0pt}%
\pgfpathmoveto{\pgfqpoint{3.776139in}{0.499444in}}%
\pgfpathlineto{\pgfqpoint{3.837525in}{0.499444in}}%
\pgfpathlineto{\pgfqpoint{3.837525in}{0.538802in}}%
\pgfpathlineto{\pgfqpoint{3.776139in}{0.538802in}}%
\pgfpathlineto{\pgfqpoint{3.776139in}{0.499444in}}%
\pgfpathclose%
\pgfusepath{fill}%
\end{pgfscope}%
\begin{pgfscope}%
\pgfpathrectangle{\pgfqpoint{0.515000in}{0.499444in}}{\pgfqpoint{3.875000in}{1.155000in}}%
\pgfusepath{clip}%
\pgfsetbuttcap%
\pgfsetmiterjoin%
\definecolor{currentfill}{rgb}{0.000000,0.000000,0.000000}%
\pgfsetfillcolor{currentfill}%
\pgfsetlinewidth{0.000000pt}%
\definecolor{currentstroke}{rgb}{0.000000,0.000000,0.000000}%
\pgfsetstrokecolor{currentstroke}%
\pgfsetstrokeopacity{0.000000}%
\pgfsetdash{}{0pt}%
\pgfpathmoveto{\pgfqpoint{3.929604in}{0.499444in}}%
\pgfpathlineto{\pgfqpoint{3.990990in}{0.499444in}}%
\pgfpathlineto{\pgfqpoint{3.990990in}{0.537585in}}%
\pgfpathlineto{\pgfqpoint{3.929604in}{0.537585in}}%
\pgfpathlineto{\pgfqpoint{3.929604in}{0.499444in}}%
\pgfpathclose%
\pgfusepath{fill}%
\end{pgfscope}%
\begin{pgfscope}%
\pgfpathrectangle{\pgfqpoint{0.515000in}{0.499444in}}{\pgfqpoint{3.875000in}{1.155000in}}%
\pgfusepath{clip}%
\pgfsetbuttcap%
\pgfsetmiterjoin%
\definecolor{currentfill}{rgb}{0.000000,0.000000,0.000000}%
\pgfsetfillcolor{currentfill}%
\pgfsetlinewidth{0.000000pt}%
\definecolor{currentstroke}{rgb}{0.000000,0.000000,0.000000}%
\pgfsetstrokecolor{currentstroke}%
\pgfsetstrokeopacity{0.000000}%
\pgfsetdash{}{0pt}%
\pgfpathmoveto{\pgfqpoint{4.083070in}{0.499444in}}%
\pgfpathlineto{\pgfqpoint{4.144456in}{0.499444in}}%
\pgfpathlineto{\pgfqpoint{4.144456in}{0.518661in}}%
\pgfpathlineto{\pgfqpoint{4.083070in}{0.518661in}}%
\pgfpathlineto{\pgfqpoint{4.083070in}{0.499444in}}%
\pgfpathclose%
\pgfusepath{fill}%
\end{pgfscope}%
\begin{pgfscope}%
\pgfpathrectangle{\pgfqpoint{0.515000in}{0.499444in}}{\pgfqpoint{3.875000in}{1.155000in}}%
\pgfusepath{clip}%
\pgfsetbuttcap%
\pgfsetmiterjoin%
\definecolor{currentfill}{rgb}{0.000000,0.000000,0.000000}%
\pgfsetfillcolor{currentfill}%
\pgfsetlinewidth{0.000000pt}%
\definecolor{currentstroke}{rgb}{0.000000,0.000000,0.000000}%
\pgfsetstrokecolor{currentstroke}%
\pgfsetstrokeopacity{0.000000}%
\pgfsetdash{}{0pt}%
\pgfpathmoveto{\pgfqpoint{4.236535in}{0.499444in}}%
\pgfpathlineto{\pgfqpoint{4.297921in}{0.499444in}}%
\pgfpathlineto{\pgfqpoint{4.297921in}{0.501374in}}%
\pgfpathlineto{\pgfqpoint{4.236535in}{0.501374in}}%
\pgfpathlineto{\pgfqpoint{4.236535in}{0.499444in}}%
\pgfpathclose%
\pgfusepath{fill}%
\end{pgfscope}%
\begin{pgfscope}%
\pgfsetbuttcap%
\pgfsetroundjoin%
\definecolor{currentfill}{rgb}{0.000000,0.000000,0.000000}%
\pgfsetfillcolor{currentfill}%
\pgfsetlinewidth{0.803000pt}%
\definecolor{currentstroke}{rgb}{0.000000,0.000000,0.000000}%
\pgfsetstrokecolor{currentstroke}%
\pgfsetdash{}{0pt}%
\pgfsys@defobject{currentmarker}{\pgfqpoint{0.000000in}{-0.048611in}}{\pgfqpoint{0.000000in}{0.000000in}}{%
\pgfpathmoveto{\pgfqpoint{0.000000in}{0.000000in}}%
\pgfpathlineto{\pgfqpoint{0.000000in}{-0.048611in}}%
\pgfusepath{stroke,fill}%
}%
\begin{pgfscope}%
\pgfsys@transformshift{0.553367in}{0.499444in}%
\pgfsys@useobject{currentmarker}{}%
\end{pgfscope}%
\end{pgfscope}%
\begin{pgfscope}%
\definecolor{textcolor}{rgb}{0.000000,0.000000,0.000000}%
\pgfsetstrokecolor{textcolor}%
\pgfsetfillcolor{textcolor}%
\pgftext[x=0.553367in,y=0.402222in,,top]{\color{textcolor}\rmfamily\fontsize{10.000000}{12.000000}\selectfont 0.0}%
\end{pgfscope}%
\begin{pgfscope}%
\pgfsetbuttcap%
\pgfsetroundjoin%
\definecolor{currentfill}{rgb}{0.000000,0.000000,0.000000}%
\pgfsetfillcolor{currentfill}%
\pgfsetlinewidth{0.803000pt}%
\definecolor{currentstroke}{rgb}{0.000000,0.000000,0.000000}%
\pgfsetstrokecolor{currentstroke}%
\pgfsetdash{}{0pt}%
\pgfsys@defobject{currentmarker}{\pgfqpoint{0.000000in}{-0.048611in}}{\pgfqpoint{0.000000in}{0.000000in}}{%
\pgfpathmoveto{\pgfqpoint{0.000000in}{0.000000in}}%
\pgfpathlineto{\pgfqpoint{0.000000in}{-0.048611in}}%
\pgfusepath{stroke,fill}%
}%
\begin{pgfscope}%
\pgfsys@transformshift{0.937030in}{0.499444in}%
\pgfsys@useobject{currentmarker}{}%
\end{pgfscope}%
\end{pgfscope}%
\begin{pgfscope}%
\definecolor{textcolor}{rgb}{0.000000,0.000000,0.000000}%
\pgfsetstrokecolor{textcolor}%
\pgfsetfillcolor{textcolor}%
\pgftext[x=0.937030in,y=0.402222in,,top]{\color{textcolor}\rmfamily\fontsize{10.000000}{12.000000}\selectfont 0.1}%
\end{pgfscope}%
\begin{pgfscope}%
\pgfsetbuttcap%
\pgfsetroundjoin%
\definecolor{currentfill}{rgb}{0.000000,0.000000,0.000000}%
\pgfsetfillcolor{currentfill}%
\pgfsetlinewidth{0.803000pt}%
\definecolor{currentstroke}{rgb}{0.000000,0.000000,0.000000}%
\pgfsetstrokecolor{currentstroke}%
\pgfsetdash{}{0pt}%
\pgfsys@defobject{currentmarker}{\pgfqpoint{0.000000in}{-0.048611in}}{\pgfqpoint{0.000000in}{0.000000in}}{%
\pgfpathmoveto{\pgfqpoint{0.000000in}{0.000000in}}%
\pgfpathlineto{\pgfqpoint{0.000000in}{-0.048611in}}%
\pgfusepath{stroke,fill}%
}%
\begin{pgfscope}%
\pgfsys@transformshift{1.320693in}{0.499444in}%
\pgfsys@useobject{currentmarker}{}%
\end{pgfscope}%
\end{pgfscope}%
\begin{pgfscope}%
\definecolor{textcolor}{rgb}{0.000000,0.000000,0.000000}%
\pgfsetstrokecolor{textcolor}%
\pgfsetfillcolor{textcolor}%
\pgftext[x=1.320693in,y=0.402222in,,top]{\color{textcolor}\rmfamily\fontsize{10.000000}{12.000000}\selectfont 0.2}%
\end{pgfscope}%
\begin{pgfscope}%
\pgfsetbuttcap%
\pgfsetroundjoin%
\definecolor{currentfill}{rgb}{0.000000,0.000000,0.000000}%
\pgfsetfillcolor{currentfill}%
\pgfsetlinewidth{0.803000pt}%
\definecolor{currentstroke}{rgb}{0.000000,0.000000,0.000000}%
\pgfsetstrokecolor{currentstroke}%
\pgfsetdash{}{0pt}%
\pgfsys@defobject{currentmarker}{\pgfqpoint{0.000000in}{-0.048611in}}{\pgfqpoint{0.000000in}{0.000000in}}{%
\pgfpathmoveto{\pgfqpoint{0.000000in}{0.000000in}}%
\pgfpathlineto{\pgfqpoint{0.000000in}{-0.048611in}}%
\pgfusepath{stroke,fill}%
}%
\begin{pgfscope}%
\pgfsys@transformshift{1.704357in}{0.499444in}%
\pgfsys@useobject{currentmarker}{}%
\end{pgfscope}%
\end{pgfscope}%
\begin{pgfscope}%
\definecolor{textcolor}{rgb}{0.000000,0.000000,0.000000}%
\pgfsetstrokecolor{textcolor}%
\pgfsetfillcolor{textcolor}%
\pgftext[x=1.704357in,y=0.402222in,,top]{\color{textcolor}\rmfamily\fontsize{10.000000}{12.000000}\selectfont 0.3}%
\end{pgfscope}%
\begin{pgfscope}%
\pgfsetbuttcap%
\pgfsetroundjoin%
\definecolor{currentfill}{rgb}{0.000000,0.000000,0.000000}%
\pgfsetfillcolor{currentfill}%
\pgfsetlinewidth{0.803000pt}%
\definecolor{currentstroke}{rgb}{0.000000,0.000000,0.000000}%
\pgfsetstrokecolor{currentstroke}%
\pgfsetdash{}{0pt}%
\pgfsys@defobject{currentmarker}{\pgfqpoint{0.000000in}{-0.048611in}}{\pgfqpoint{0.000000in}{0.000000in}}{%
\pgfpathmoveto{\pgfqpoint{0.000000in}{0.000000in}}%
\pgfpathlineto{\pgfqpoint{0.000000in}{-0.048611in}}%
\pgfusepath{stroke,fill}%
}%
\begin{pgfscope}%
\pgfsys@transformshift{2.088020in}{0.499444in}%
\pgfsys@useobject{currentmarker}{}%
\end{pgfscope}%
\end{pgfscope}%
\begin{pgfscope}%
\definecolor{textcolor}{rgb}{0.000000,0.000000,0.000000}%
\pgfsetstrokecolor{textcolor}%
\pgfsetfillcolor{textcolor}%
\pgftext[x=2.088020in,y=0.402222in,,top]{\color{textcolor}\rmfamily\fontsize{10.000000}{12.000000}\selectfont 0.4}%
\end{pgfscope}%
\begin{pgfscope}%
\pgfsetbuttcap%
\pgfsetroundjoin%
\definecolor{currentfill}{rgb}{0.000000,0.000000,0.000000}%
\pgfsetfillcolor{currentfill}%
\pgfsetlinewidth{0.803000pt}%
\definecolor{currentstroke}{rgb}{0.000000,0.000000,0.000000}%
\pgfsetstrokecolor{currentstroke}%
\pgfsetdash{}{0pt}%
\pgfsys@defobject{currentmarker}{\pgfqpoint{0.000000in}{-0.048611in}}{\pgfqpoint{0.000000in}{0.000000in}}{%
\pgfpathmoveto{\pgfqpoint{0.000000in}{0.000000in}}%
\pgfpathlineto{\pgfqpoint{0.000000in}{-0.048611in}}%
\pgfusepath{stroke,fill}%
}%
\begin{pgfscope}%
\pgfsys@transformshift{2.471683in}{0.499444in}%
\pgfsys@useobject{currentmarker}{}%
\end{pgfscope}%
\end{pgfscope}%
\begin{pgfscope}%
\definecolor{textcolor}{rgb}{0.000000,0.000000,0.000000}%
\pgfsetstrokecolor{textcolor}%
\pgfsetfillcolor{textcolor}%
\pgftext[x=2.471683in,y=0.402222in,,top]{\color{textcolor}\rmfamily\fontsize{10.000000}{12.000000}\selectfont 0.5}%
\end{pgfscope}%
\begin{pgfscope}%
\pgfsetbuttcap%
\pgfsetroundjoin%
\definecolor{currentfill}{rgb}{0.000000,0.000000,0.000000}%
\pgfsetfillcolor{currentfill}%
\pgfsetlinewidth{0.803000pt}%
\definecolor{currentstroke}{rgb}{0.000000,0.000000,0.000000}%
\pgfsetstrokecolor{currentstroke}%
\pgfsetdash{}{0pt}%
\pgfsys@defobject{currentmarker}{\pgfqpoint{0.000000in}{-0.048611in}}{\pgfqpoint{0.000000in}{0.000000in}}{%
\pgfpathmoveto{\pgfqpoint{0.000000in}{0.000000in}}%
\pgfpathlineto{\pgfqpoint{0.000000in}{-0.048611in}}%
\pgfusepath{stroke,fill}%
}%
\begin{pgfscope}%
\pgfsys@transformshift{2.855347in}{0.499444in}%
\pgfsys@useobject{currentmarker}{}%
\end{pgfscope}%
\end{pgfscope}%
\begin{pgfscope}%
\definecolor{textcolor}{rgb}{0.000000,0.000000,0.000000}%
\pgfsetstrokecolor{textcolor}%
\pgfsetfillcolor{textcolor}%
\pgftext[x=2.855347in,y=0.402222in,,top]{\color{textcolor}\rmfamily\fontsize{10.000000}{12.000000}\selectfont 0.6}%
\end{pgfscope}%
\begin{pgfscope}%
\pgfsetbuttcap%
\pgfsetroundjoin%
\definecolor{currentfill}{rgb}{0.000000,0.000000,0.000000}%
\pgfsetfillcolor{currentfill}%
\pgfsetlinewidth{0.803000pt}%
\definecolor{currentstroke}{rgb}{0.000000,0.000000,0.000000}%
\pgfsetstrokecolor{currentstroke}%
\pgfsetdash{}{0pt}%
\pgfsys@defobject{currentmarker}{\pgfqpoint{0.000000in}{-0.048611in}}{\pgfqpoint{0.000000in}{0.000000in}}{%
\pgfpathmoveto{\pgfqpoint{0.000000in}{0.000000in}}%
\pgfpathlineto{\pgfqpoint{0.000000in}{-0.048611in}}%
\pgfusepath{stroke,fill}%
}%
\begin{pgfscope}%
\pgfsys@transformshift{3.239010in}{0.499444in}%
\pgfsys@useobject{currentmarker}{}%
\end{pgfscope}%
\end{pgfscope}%
\begin{pgfscope}%
\definecolor{textcolor}{rgb}{0.000000,0.000000,0.000000}%
\pgfsetstrokecolor{textcolor}%
\pgfsetfillcolor{textcolor}%
\pgftext[x=3.239010in,y=0.402222in,,top]{\color{textcolor}\rmfamily\fontsize{10.000000}{12.000000}\selectfont 0.7}%
\end{pgfscope}%
\begin{pgfscope}%
\pgfsetbuttcap%
\pgfsetroundjoin%
\definecolor{currentfill}{rgb}{0.000000,0.000000,0.000000}%
\pgfsetfillcolor{currentfill}%
\pgfsetlinewidth{0.803000pt}%
\definecolor{currentstroke}{rgb}{0.000000,0.000000,0.000000}%
\pgfsetstrokecolor{currentstroke}%
\pgfsetdash{}{0pt}%
\pgfsys@defobject{currentmarker}{\pgfqpoint{0.000000in}{-0.048611in}}{\pgfqpoint{0.000000in}{0.000000in}}{%
\pgfpathmoveto{\pgfqpoint{0.000000in}{0.000000in}}%
\pgfpathlineto{\pgfqpoint{0.000000in}{-0.048611in}}%
\pgfusepath{stroke,fill}%
}%
\begin{pgfscope}%
\pgfsys@transformshift{3.622674in}{0.499444in}%
\pgfsys@useobject{currentmarker}{}%
\end{pgfscope}%
\end{pgfscope}%
\begin{pgfscope}%
\definecolor{textcolor}{rgb}{0.000000,0.000000,0.000000}%
\pgfsetstrokecolor{textcolor}%
\pgfsetfillcolor{textcolor}%
\pgftext[x=3.622674in,y=0.402222in,,top]{\color{textcolor}\rmfamily\fontsize{10.000000}{12.000000}\selectfont 0.8}%
\end{pgfscope}%
\begin{pgfscope}%
\pgfsetbuttcap%
\pgfsetroundjoin%
\definecolor{currentfill}{rgb}{0.000000,0.000000,0.000000}%
\pgfsetfillcolor{currentfill}%
\pgfsetlinewidth{0.803000pt}%
\definecolor{currentstroke}{rgb}{0.000000,0.000000,0.000000}%
\pgfsetstrokecolor{currentstroke}%
\pgfsetdash{}{0pt}%
\pgfsys@defobject{currentmarker}{\pgfqpoint{0.000000in}{-0.048611in}}{\pgfqpoint{0.000000in}{0.000000in}}{%
\pgfpathmoveto{\pgfqpoint{0.000000in}{0.000000in}}%
\pgfpathlineto{\pgfqpoint{0.000000in}{-0.048611in}}%
\pgfusepath{stroke,fill}%
}%
\begin{pgfscope}%
\pgfsys@transformshift{4.006337in}{0.499444in}%
\pgfsys@useobject{currentmarker}{}%
\end{pgfscope}%
\end{pgfscope}%
\begin{pgfscope}%
\definecolor{textcolor}{rgb}{0.000000,0.000000,0.000000}%
\pgfsetstrokecolor{textcolor}%
\pgfsetfillcolor{textcolor}%
\pgftext[x=4.006337in,y=0.402222in,,top]{\color{textcolor}\rmfamily\fontsize{10.000000}{12.000000}\selectfont 0.9}%
\end{pgfscope}%
\begin{pgfscope}%
\pgfsetbuttcap%
\pgfsetroundjoin%
\definecolor{currentfill}{rgb}{0.000000,0.000000,0.000000}%
\pgfsetfillcolor{currentfill}%
\pgfsetlinewidth{0.803000pt}%
\definecolor{currentstroke}{rgb}{0.000000,0.000000,0.000000}%
\pgfsetstrokecolor{currentstroke}%
\pgfsetdash{}{0pt}%
\pgfsys@defobject{currentmarker}{\pgfqpoint{0.000000in}{-0.048611in}}{\pgfqpoint{0.000000in}{0.000000in}}{%
\pgfpathmoveto{\pgfqpoint{0.000000in}{0.000000in}}%
\pgfpathlineto{\pgfqpoint{0.000000in}{-0.048611in}}%
\pgfusepath{stroke,fill}%
}%
\begin{pgfscope}%
\pgfsys@transformshift{4.390000in}{0.499444in}%
\pgfsys@useobject{currentmarker}{}%
\end{pgfscope}%
\end{pgfscope}%
\begin{pgfscope}%
\definecolor{textcolor}{rgb}{0.000000,0.000000,0.000000}%
\pgfsetstrokecolor{textcolor}%
\pgfsetfillcolor{textcolor}%
\pgftext[x=4.390000in,y=0.402222in,,top]{\color{textcolor}\rmfamily\fontsize{10.000000}{12.000000}\selectfont 1.0}%
\end{pgfscope}%
\begin{pgfscope}%
\definecolor{textcolor}{rgb}{0.000000,0.000000,0.000000}%
\pgfsetstrokecolor{textcolor}%
\pgfsetfillcolor{textcolor}%
\pgftext[x=2.452500in,y=0.223333in,,top]{\color{textcolor}\rmfamily\fontsize{10.000000}{12.000000}\selectfont \(\displaystyle p\)}%
\end{pgfscope}%
\begin{pgfscope}%
\pgfsetbuttcap%
\pgfsetroundjoin%
\definecolor{currentfill}{rgb}{0.000000,0.000000,0.000000}%
\pgfsetfillcolor{currentfill}%
\pgfsetlinewidth{0.803000pt}%
\definecolor{currentstroke}{rgb}{0.000000,0.000000,0.000000}%
\pgfsetstrokecolor{currentstroke}%
\pgfsetdash{}{0pt}%
\pgfsys@defobject{currentmarker}{\pgfqpoint{-0.048611in}{0.000000in}}{\pgfqpoint{-0.000000in}{0.000000in}}{%
\pgfpathmoveto{\pgfqpoint{-0.000000in}{0.000000in}}%
\pgfpathlineto{\pgfqpoint{-0.048611in}{0.000000in}}%
\pgfusepath{stroke,fill}%
}%
\begin{pgfscope}%
\pgfsys@transformshift{0.515000in}{0.499444in}%
\pgfsys@useobject{currentmarker}{}%
\end{pgfscope}%
\end{pgfscope}%
\begin{pgfscope}%
\definecolor{textcolor}{rgb}{0.000000,0.000000,0.000000}%
\pgfsetstrokecolor{textcolor}%
\pgfsetfillcolor{textcolor}%
\pgftext[x=0.348333in, y=0.451250in, left, base]{\color{textcolor}\rmfamily\fontsize{10.000000}{12.000000}\selectfont \(\displaystyle {0}\)}%
\end{pgfscope}%
\begin{pgfscope}%
\pgfsetbuttcap%
\pgfsetroundjoin%
\definecolor{currentfill}{rgb}{0.000000,0.000000,0.000000}%
\pgfsetfillcolor{currentfill}%
\pgfsetlinewidth{0.803000pt}%
\definecolor{currentstroke}{rgb}{0.000000,0.000000,0.000000}%
\pgfsetstrokecolor{currentstroke}%
\pgfsetdash{}{0pt}%
\pgfsys@defobject{currentmarker}{\pgfqpoint{-0.048611in}{0.000000in}}{\pgfqpoint{-0.000000in}{0.000000in}}{%
\pgfpathmoveto{\pgfqpoint{-0.000000in}{0.000000in}}%
\pgfpathlineto{\pgfqpoint{-0.048611in}{0.000000in}}%
\pgfusepath{stroke,fill}%
}%
\begin{pgfscope}%
\pgfsys@transformshift{0.515000in}{0.948553in}%
\pgfsys@useobject{currentmarker}{}%
\end{pgfscope}%
\end{pgfscope}%
\begin{pgfscope}%
\definecolor{textcolor}{rgb}{0.000000,0.000000,0.000000}%
\pgfsetstrokecolor{textcolor}%
\pgfsetfillcolor{textcolor}%
\pgftext[x=0.348333in, y=0.900359in, left, base]{\color{textcolor}\rmfamily\fontsize{10.000000}{12.000000}\selectfont \(\displaystyle {5}\)}%
\end{pgfscope}%
\begin{pgfscope}%
\pgfsetbuttcap%
\pgfsetroundjoin%
\definecolor{currentfill}{rgb}{0.000000,0.000000,0.000000}%
\pgfsetfillcolor{currentfill}%
\pgfsetlinewidth{0.803000pt}%
\definecolor{currentstroke}{rgb}{0.000000,0.000000,0.000000}%
\pgfsetstrokecolor{currentstroke}%
\pgfsetdash{}{0pt}%
\pgfsys@defobject{currentmarker}{\pgfqpoint{-0.048611in}{0.000000in}}{\pgfqpoint{-0.000000in}{0.000000in}}{%
\pgfpathmoveto{\pgfqpoint{-0.000000in}{0.000000in}}%
\pgfpathlineto{\pgfqpoint{-0.048611in}{0.000000in}}%
\pgfusepath{stroke,fill}%
}%
\begin{pgfscope}%
\pgfsys@transformshift{0.515000in}{1.397663in}%
\pgfsys@useobject{currentmarker}{}%
\end{pgfscope}%
\end{pgfscope}%
\begin{pgfscope}%
\definecolor{textcolor}{rgb}{0.000000,0.000000,0.000000}%
\pgfsetstrokecolor{textcolor}%
\pgfsetfillcolor{textcolor}%
\pgftext[x=0.278889in, y=1.349468in, left, base]{\color{textcolor}\rmfamily\fontsize{10.000000}{12.000000}\selectfont \(\displaystyle {10}\)}%
\end{pgfscope}%
\begin{pgfscope}%
\definecolor{textcolor}{rgb}{0.000000,0.000000,0.000000}%
\pgfsetstrokecolor{textcolor}%
\pgfsetfillcolor{textcolor}%
\pgftext[x=0.223333in,y=1.076944in,,bottom,rotate=90.000000]{\color{textcolor}\rmfamily\fontsize{10.000000}{12.000000}\selectfont Percent of Data Set}%
\end{pgfscope}%
\begin{pgfscope}%
\pgfsetrectcap%
\pgfsetmiterjoin%
\pgfsetlinewidth{0.803000pt}%
\definecolor{currentstroke}{rgb}{0.000000,0.000000,0.000000}%
\pgfsetstrokecolor{currentstroke}%
\pgfsetdash{}{0pt}%
\pgfpathmoveto{\pgfqpoint{0.515000in}{0.499444in}}%
\pgfpathlineto{\pgfqpoint{0.515000in}{1.654444in}}%
\pgfusepath{stroke}%
\end{pgfscope}%
\begin{pgfscope}%
\pgfsetrectcap%
\pgfsetmiterjoin%
\pgfsetlinewidth{0.803000pt}%
\definecolor{currentstroke}{rgb}{0.000000,0.000000,0.000000}%
\pgfsetstrokecolor{currentstroke}%
\pgfsetdash{}{0pt}%
\pgfpathmoveto{\pgfqpoint{4.390000in}{0.499444in}}%
\pgfpathlineto{\pgfqpoint{4.390000in}{1.654444in}}%
\pgfusepath{stroke}%
\end{pgfscope}%
\begin{pgfscope}%
\pgfsetrectcap%
\pgfsetmiterjoin%
\pgfsetlinewidth{0.803000pt}%
\definecolor{currentstroke}{rgb}{0.000000,0.000000,0.000000}%
\pgfsetstrokecolor{currentstroke}%
\pgfsetdash{}{0pt}%
\pgfpathmoveto{\pgfqpoint{0.515000in}{0.499444in}}%
\pgfpathlineto{\pgfqpoint{4.390000in}{0.499444in}}%
\pgfusepath{stroke}%
\end{pgfscope}%
\begin{pgfscope}%
\pgfsetrectcap%
\pgfsetmiterjoin%
\pgfsetlinewidth{0.803000pt}%
\definecolor{currentstroke}{rgb}{0.000000,0.000000,0.000000}%
\pgfsetstrokecolor{currentstroke}%
\pgfsetdash{}{0pt}%
\pgfpathmoveto{\pgfqpoint{0.515000in}{1.654444in}}%
\pgfpathlineto{\pgfqpoint{4.390000in}{1.654444in}}%
\pgfusepath{stroke}%
\end{pgfscope}%
\begin{pgfscope}%
\pgfsetbuttcap%
\pgfsetmiterjoin%
\definecolor{currentfill}{rgb}{1.000000,1.000000,1.000000}%
\pgfsetfillcolor{currentfill}%
\pgfsetfillopacity{0.800000}%
\pgfsetlinewidth{1.003750pt}%
\definecolor{currentstroke}{rgb}{0.800000,0.800000,0.800000}%
\pgfsetstrokecolor{currentstroke}%
\pgfsetstrokeopacity{0.800000}%
\pgfsetdash{}{0pt}%
\pgfpathmoveto{\pgfqpoint{3.613056in}{1.154445in}}%
\pgfpathlineto{\pgfqpoint{4.292778in}{1.154445in}}%
\pgfpathquadraticcurveto{\pgfqpoint{4.320556in}{1.154445in}}{\pgfqpoint{4.320556in}{1.182222in}}%
\pgfpathlineto{\pgfqpoint{4.320556in}{1.557222in}}%
\pgfpathquadraticcurveto{\pgfqpoint{4.320556in}{1.585000in}}{\pgfqpoint{4.292778in}{1.585000in}}%
\pgfpathlineto{\pgfqpoint{3.613056in}{1.585000in}}%
\pgfpathquadraticcurveto{\pgfqpoint{3.585278in}{1.585000in}}{\pgfqpoint{3.585278in}{1.557222in}}%
\pgfpathlineto{\pgfqpoint{3.585278in}{1.182222in}}%
\pgfpathquadraticcurveto{\pgfqpoint{3.585278in}{1.154445in}}{\pgfqpoint{3.613056in}{1.154445in}}%
\pgfpathlineto{\pgfqpoint{3.613056in}{1.154445in}}%
\pgfpathclose%
\pgfusepath{stroke,fill}%
\end{pgfscope}%
\begin{pgfscope}%
\pgfsetbuttcap%
\pgfsetmiterjoin%
\pgfsetlinewidth{1.003750pt}%
\definecolor{currentstroke}{rgb}{0.000000,0.000000,0.000000}%
\pgfsetstrokecolor{currentstroke}%
\pgfsetdash{}{0pt}%
\pgfpathmoveto{\pgfqpoint{3.640834in}{1.432222in}}%
\pgfpathlineto{\pgfqpoint{3.918611in}{1.432222in}}%
\pgfpathlineto{\pgfqpoint{3.918611in}{1.529444in}}%
\pgfpathlineto{\pgfqpoint{3.640834in}{1.529444in}}%
\pgfpathlineto{\pgfqpoint{3.640834in}{1.432222in}}%
\pgfpathclose%
\pgfusepath{stroke}%
\end{pgfscope}%
\begin{pgfscope}%
\definecolor{textcolor}{rgb}{0.000000,0.000000,0.000000}%
\pgfsetstrokecolor{textcolor}%
\pgfsetfillcolor{textcolor}%
\pgftext[x=4.029723in,y=1.432222in,left,base]{\color{textcolor}\rmfamily\fontsize{10.000000}{12.000000}\selectfont Neg}%
\end{pgfscope}%
\begin{pgfscope}%
\pgfsetbuttcap%
\pgfsetmiterjoin%
\definecolor{currentfill}{rgb}{0.000000,0.000000,0.000000}%
\pgfsetfillcolor{currentfill}%
\pgfsetlinewidth{0.000000pt}%
\definecolor{currentstroke}{rgb}{0.000000,0.000000,0.000000}%
\pgfsetstrokecolor{currentstroke}%
\pgfsetstrokeopacity{0.000000}%
\pgfsetdash{}{0pt}%
\pgfpathmoveto{\pgfqpoint{3.640834in}{1.236944in}}%
\pgfpathlineto{\pgfqpoint{3.918611in}{1.236944in}}%
\pgfpathlineto{\pgfqpoint{3.918611in}{1.334167in}}%
\pgfpathlineto{\pgfqpoint{3.640834in}{1.334167in}}%
\pgfpathlineto{\pgfqpoint{3.640834in}{1.236944in}}%
\pgfpathclose%
\pgfusepath{fill}%
\end{pgfscope}%
\begin{pgfscope}%
\definecolor{textcolor}{rgb}{0.000000,0.000000,0.000000}%
\pgfsetstrokecolor{textcolor}%
\pgfsetfillcolor{textcolor}%
\pgftext[x=4.029723in,y=1.236944in,left,base]{\color{textcolor}\rmfamily\fontsize{10.000000}{12.000000}\selectfont Pos}%
\end{pgfscope}%
\end{pgfpicture}%
\makeatother%
\endgroup%
	
&
	\vskip 0pt
	\hfil ROC Curve
	
	%% Creator: Matplotlib, PGF backend
%%
%% To include the figure in your LaTeX document, write
%%   \input{<filename>.pgf}
%%
%% Make sure the required packages are loaded in your preamble
%%   \usepackage{pgf}
%%
%% Also ensure that all the required font packages are loaded; for instance,
%% the lmodern package is sometimes necessary when using math font.
%%   \usepackage{lmodern}
%%
%% Figures using additional raster images can only be included by \input if
%% they are in the same directory as the main LaTeX file. For loading figures
%% from other directories you can use the `import` package
%%   \usepackage{import}
%%
%% and then include the figures with
%%   \import{<path to file>}{<filename>.pgf}
%%
%% Matplotlib used the following preamble
%%   
%%   \usepackage{fontspec}
%%   \makeatletter\@ifpackageloaded{underscore}{}{\usepackage[strings]{underscore}}\makeatother
%%
\begingroup%
\makeatletter%
\begin{pgfpicture}%
\pgfpathrectangle{\pgfpointorigin}{\pgfqpoint{2.221861in}{1.754444in}}%
\pgfusepath{use as bounding box, clip}%
\begin{pgfscope}%
\pgfsetbuttcap%
\pgfsetmiterjoin%
\definecolor{currentfill}{rgb}{1.000000,1.000000,1.000000}%
\pgfsetfillcolor{currentfill}%
\pgfsetlinewidth{0.000000pt}%
\definecolor{currentstroke}{rgb}{1.000000,1.000000,1.000000}%
\pgfsetstrokecolor{currentstroke}%
\pgfsetdash{}{0pt}%
\pgfpathmoveto{\pgfqpoint{0.000000in}{0.000000in}}%
\pgfpathlineto{\pgfqpoint{2.221861in}{0.000000in}}%
\pgfpathlineto{\pgfqpoint{2.221861in}{1.754444in}}%
\pgfpathlineto{\pgfqpoint{0.000000in}{1.754444in}}%
\pgfpathlineto{\pgfqpoint{0.000000in}{0.000000in}}%
\pgfpathclose%
\pgfusepath{fill}%
\end{pgfscope}%
\begin{pgfscope}%
\pgfsetbuttcap%
\pgfsetmiterjoin%
\definecolor{currentfill}{rgb}{1.000000,1.000000,1.000000}%
\pgfsetfillcolor{currentfill}%
\pgfsetlinewidth{0.000000pt}%
\definecolor{currentstroke}{rgb}{0.000000,0.000000,0.000000}%
\pgfsetstrokecolor{currentstroke}%
\pgfsetstrokeopacity{0.000000}%
\pgfsetdash{}{0pt}%
\pgfpathmoveto{\pgfqpoint{0.553581in}{0.499444in}}%
\pgfpathlineto{\pgfqpoint{2.103581in}{0.499444in}}%
\pgfpathlineto{\pgfqpoint{2.103581in}{1.654444in}}%
\pgfpathlineto{\pgfqpoint{0.553581in}{1.654444in}}%
\pgfpathlineto{\pgfqpoint{0.553581in}{0.499444in}}%
\pgfpathclose%
\pgfusepath{fill}%
\end{pgfscope}%
\begin{pgfscope}%
\pgfsetbuttcap%
\pgfsetroundjoin%
\definecolor{currentfill}{rgb}{0.000000,0.000000,0.000000}%
\pgfsetfillcolor{currentfill}%
\pgfsetlinewidth{0.803000pt}%
\definecolor{currentstroke}{rgb}{0.000000,0.000000,0.000000}%
\pgfsetstrokecolor{currentstroke}%
\pgfsetdash{}{0pt}%
\pgfsys@defobject{currentmarker}{\pgfqpoint{0.000000in}{-0.048611in}}{\pgfqpoint{0.000000in}{0.000000in}}{%
\pgfpathmoveto{\pgfqpoint{0.000000in}{0.000000in}}%
\pgfpathlineto{\pgfqpoint{0.000000in}{-0.048611in}}%
\pgfusepath{stroke,fill}%
}%
\begin{pgfscope}%
\pgfsys@transformshift{0.624035in}{0.499444in}%
\pgfsys@useobject{currentmarker}{}%
\end{pgfscope}%
\end{pgfscope}%
\begin{pgfscope}%
\definecolor{textcolor}{rgb}{0.000000,0.000000,0.000000}%
\pgfsetstrokecolor{textcolor}%
\pgfsetfillcolor{textcolor}%
\pgftext[x=0.624035in,y=0.402222in,,top]{\color{textcolor}\rmfamily\fontsize{10.000000}{12.000000}\selectfont \(\displaystyle {0.0}\)}%
\end{pgfscope}%
\begin{pgfscope}%
\pgfsetbuttcap%
\pgfsetroundjoin%
\definecolor{currentfill}{rgb}{0.000000,0.000000,0.000000}%
\pgfsetfillcolor{currentfill}%
\pgfsetlinewidth{0.803000pt}%
\definecolor{currentstroke}{rgb}{0.000000,0.000000,0.000000}%
\pgfsetstrokecolor{currentstroke}%
\pgfsetdash{}{0pt}%
\pgfsys@defobject{currentmarker}{\pgfqpoint{0.000000in}{-0.048611in}}{\pgfqpoint{0.000000in}{0.000000in}}{%
\pgfpathmoveto{\pgfqpoint{0.000000in}{0.000000in}}%
\pgfpathlineto{\pgfqpoint{0.000000in}{-0.048611in}}%
\pgfusepath{stroke,fill}%
}%
\begin{pgfscope}%
\pgfsys@transformshift{1.328581in}{0.499444in}%
\pgfsys@useobject{currentmarker}{}%
\end{pgfscope}%
\end{pgfscope}%
\begin{pgfscope}%
\definecolor{textcolor}{rgb}{0.000000,0.000000,0.000000}%
\pgfsetstrokecolor{textcolor}%
\pgfsetfillcolor{textcolor}%
\pgftext[x=1.328581in,y=0.402222in,,top]{\color{textcolor}\rmfamily\fontsize{10.000000}{12.000000}\selectfont \(\displaystyle {0.5}\)}%
\end{pgfscope}%
\begin{pgfscope}%
\pgfsetbuttcap%
\pgfsetroundjoin%
\definecolor{currentfill}{rgb}{0.000000,0.000000,0.000000}%
\pgfsetfillcolor{currentfill}%
\pgfsetlinewidth{0.803000pt}%
\definecolor{currentstroke}{rgb}{0.000000,0.000000,0.000000}%
\pgfsetstrokecolor{currentstroke}%
\pgfsetdash{}{0pt}%
\pgfsys@defobject{currentmarker}{\pgfqpoint{0.000000in}{-0.048611in}}{\pgfqpoint{0.000000in}{0.000000in}}{%
\pgfpathmoveto{\pgfqpoint{0.000000in}{0.000000in}}%
\pgfpathlineto{\pgfqpoint{0.000000in}{-0.048611in}}%
\pgfusepath{stroke,fill}%
}%
\begin{pgfscope}%
\pgfsys@transformshift{2.033126in}{0.499444in}%
\pgfsys@useobject{currentmarker}{}%
\end{pgfscope}%
\end{pgfscope}%
\begin{pgfscope}%
\definecolor{textcolor}{rgb}{0.000000,0.000000,0.000000}%
\pgfsetstrokecolor{textcolor}%
\pgfsetfillcolor{textcolor}%
\pgftext[x=2.033126in,y=0.402222in,,top]{\color{textcolor}\rmfamily\fontsize{10.000000}{12.000000}\selectfont \(\displaystyle {1.0}\)}%
\end{pgfscope}%
\begin{pgfscope}%
\definecolor{textcolor}{rgb}{0.000000,0.000000,0.000000}%
\pgfsetstrokecolor{textcolor}%
\pgfsetfillcolor{textcolor}%
\pgftext[x=1.328581in,y=0.223333in,,top]{\color{textcolor}\rmfamily\fontsize{10.000000}{12.000000}\selectfont False positive rate}%
\end{pgfscope}%
\begin{pgfscope}%
\pgfsetbuttcap%
\pgfsetroundjoin%
\definecolor{currentfill}{rgb}{0.000000,0.000000,0.000000}%
\pgfsetfillcolor{currentfill}%
\pgfsetlinewidth{0.803000pt}%
\definecolor{currentstroke}{rgb}{0.000000,0.000000,0.000000}%
\pgfsetstrokecolor{currentstroke}%
\pgfsetdash{}{0pt}%
\pgfsys@defobject{currentmarker}{\pgfqpoint{-0.048611in}{0.000000in}}{\pgfqpoint{-0.000000in}{0.000000in}}{%
\pgfpathmoveto{\pgfqpoint{-0.000000in}{0.000000in}}%
\pgfpathlineto{\pgfqpoint{-0.048611in}{0.000000in}}%
\pgfusepath{stroke,fill}%
}%
\begin{pgfscope}%
\pgfsys@transformshift{0.553581in}{0.551944in}%
\pgfsys@useobject{currentmarker}{}%
\end{pgfscope}%
\end{pgfscope}%
\begin{pgfscope}%
\definecolor{textcolor}{rgb}{0.000000,0.000000,0.000000}%
\pgfsetstrokecolor{textcolor}%
\pgfsetfillcolor{textcolor}%
\pgftext[x=0.278889in, y=0.503750in, left, base]{\color{textcolor}\rmfamily\fontsize{10.000000}{12.000000}\selectfont \(\displaystyle {0.0}\)}%
\end{pgfscope}%
\begin{pgfscope}%
\pgfsetbuttcap%
\pgfsetroundjoin%
\definecolor{currentfill}{rgb}{0.000000,0.000000,0.000000}%
\pgfsetfillcolor{currentfill}%
\pgfsetlinewidth{0.803000pt}%
\definecolor{currentstroke}{rgb}{0.000000,0.000000,0.000000}%
\pgfsetstrokecolor{currentstroke}%
\pgfsetdash{}{0pt}%
\pgfsys@defobject{currentmarker}{\pgfqpoint{-0.048611in}{0.000000in}}{\pgfqpoint{-0.000000in}{0.000000in}}{%
\pgfpathmoveto{\pgfqpoint{-0.000000in}{0.000000in}}%
\pgfpathlineto{\pgfqpoint{-0.048611in}{0.000000in}}%
\pgfusepath{stroke,fill}%
}%
\begin{pgfscope}%
\pgfsys@transformshift{0.553581in}{1.076944in}%
\pgfsys@useobject{currentmarker}{}%
\end{pgfscope}%
\end{pgfscope}%
\begin{pgfscope}%
\definecolor{textcolor}{rgb}{0.000000,0.000000,0.000000}%
\pgfsetstrokecolor{textcolor}%
\pgfsetfillcolor{textcolor}%
\pgftext[x=0.278889in, y=1.028750in, left, base]{\color{textcolor}\rmfamily\fontsize{10.000000}{12.000000}\selectfont \(\displaystyle {0.5}\)}%
\end{pgfscope}%
\begin{pgfscope}%
\pgfsetbuttcap%
\pgfsetroundjoin%
\definecolor{currentfill}{rgb}{0.000000,0.000000,0.000000}%
\pgfsetfillcolor{currentfill}%
\pgfsetlinewidth{0.803000pt}%
\definecolor{currentstroke}{rgb}{0.000000,0.000000,0.000000}%
\pgfsetstrokecolor{currentstroke}%
\pgfsetdash{}{0pt}%
\pgfsys@defobject{currentmarker}{\pgfqpoint{-0.048611in}{0.000000in}}{\pgfqpoint{-0.000000in}{0.000000in}}{%
\pgfpathmoveto{\pgfqpoint{-0.000000in}{0.000000in}}%
\pgfpathlineto{\pgfqpoint{-0.048611in}{0.000000in}}%
\pgfusepath{stroke,fill}%
}%
\begin{pgfscope}%
\pgfsys@transformshift{0.553581in}{1.601944in}%
\pgfsys@useobject{currentmarker}{}%
\end{pgfscope}%
\end{pgfscope}%
\begin{pgfscope}%
\definecolor{textcolor}{rgb}{0.000000,0.000000,0.000000}%
\pgfsetstrokecolor{textcolor}%
\pgfsetfillcolor{textcolor}%
\pgftext[x=0.278889in, y=1.553750in, left, base]{\color{textcolor}\rmfamily\fontsize{10.000000}{12.000000}\selectfont \(\displaystyle {1.0}\)}%
\end{pgfscope}%
\begin{pgfscope}%
\definecolor{textcolor}{rgb}{0.000000,0.000000,0.000000}%
\pgfsetstrokecolor{textcolor}%
\pgfsetfillcolor{textcolor}%
\pgftext[x=0.223333in,y=1.076944in,,bottom,rotate=90.000000]{\color{textcolor}\rmfamily\fontsize{10.000000}{12.000000}\selectfont True positive rate}%
\end{pgfscope}%
\begin{pgfscope}%
\pgfpathrectangle{\pgfqpoint{0.553581in}{0.499444in}}{\pgfqpoint{1.550000in}{1.155000in}}%
\pgfusepath{clip}%
\pgfsetbuttcap%
\pgfsetroundjoin%
\pgfsetlinewidth{1.505625pt}%
\definecolor{currentstroke}{rgb}{0.000000,0.000000,0.000000}%
\pgfsetstrokecolor{currentstroke}%
\pgfsetdash{{5.550000pt}{2.400000pt}}{0.000000pt}%
\pgfpathmoveto{\pgfqpoint{0.624035in}{0.551944in}}%
\pgfpathlineto{\pgfqpoint{2.033126in}{1.601944in}}%
\pgfusepath{stroke}%
\end{pgfscope}%
\begin{pgfscope}%
\pgfpathrectangle{\pgfqpoint{0.553581in}{0.499444in}}{\pgfqpoint{1.550000in}{1.155000in}}%
\pgfusepath{clip}%
\pgfsetrectcap%
\pgfsetroundjoin%
\pgfsetlinewidth{1.505625pt}%
\definecolor{currentstroke}{rgb}{0.000000,0.000000,0.000000}%
\pgfsetstrokecolor{currentstroke}%
\pgfsetdash{}{0pt}%
\pgfpathmoveto{\pgfqpoint{0.624035in}{0.551944in}}%
\pgfpathlineto{\pgfqpoint{0.625138in}{0.569980in}}%
\pgfpathlineto{\pgfqpoint{0.625200in}{0.570694in}}%
\pgfpathlineto{\pgfqpoint{0.626310in}{0.584880in}}%
\pgfpathlineto{\pgfqpoint{0.626427in}{0.585656in}}%
\pgfpathlineto{\pgfqpoint{0.627538in}{0.595155in}}%
\pgfpathlineto{\pgfqpoint{0.627686in}{0.596117in}}%
\pgfpathlineto{\pgfqpoint{0.628796in}{0.606423in}}%
\pgfpathlineto{\pgfqpoint{0.628968in}{0.607479in}}%
\pgfpathlineto{\pgfqpoint{0.630070in}{0.616853in}}%
\pgfpathlineto{\pgfqpoint{0.630235in}{0.617940in}}%
\pgfpathlineto{\pgfqpoint{0.631345in}{0.626911in}}%
\pgfpathlineto{\pgfqpoint{0.631485in}{0.627935in}}%
\pgfpathlineto{\pgfqpoint{0.632572in}{0.635572in}}%
\pgfpathlineto{\pgfqpoint{0.632736in}{0.636658in}}%
\pgfpathlineto{\pgfqpoint{0.633846in}{0.644574in}}%
\pgfpathlineto{\pgfqpoint{0.634097in}{0.645660in}}%
\pgfpathlineto{\pgfqpoint{0.635191in}{0.651807in}}%
\pgfpathlineto{\pgfqpoint{0.635426in}{0.652831in}}%
\pgfpathlineto{\pgfqpoint{0.636536in}{0.660002in}}%
\pgfpathlineto{\pgfqpoint{0.636755in}{0.661088in}}%
\pgfpathlineto{\pgfqpoint{0.637857in}{0.667855in}}%
\pgfpathlineto{\pgfqpoint{0.638005in}{0.668911in}}%
\pgfpathlineto{\pgfqpoint{0.639115in}{0.674312in}}%
\pgfpathlineto{\pgfqpoint{0.639420in}{0.675275in}}%
\pgfpathlineto{\pgfqpoint{0.640523in}{0.680769in}}%
\pgfpathlineto{\pgfqpoint{0.640679in}{0.681855in}}%
\pgfpathlineto{\pgfqpoint{0.641781in}{0.687785in}}%
\pgfpathlineto{\pgfqpoint{0.642055in}{0.688871in}}%
\pgfpathlineto{\pgfqpoint{0.643165in}{0.694459in}}%
\pgfpathlineto{\pgfqpoint{0.643493in}{0.695514in}}%
\pgfpathlineto{\pgfqpoint{0.644588in}{0.700574in}}%
\pgfpathlineto{\pgfqpoint{0.644963in}{0.701598in}}%
\pgfpathlineto{\pgfqpoint{0.646073in}{0.707031in}}%
\pgfpathlineto{\pgfqpoint{0.646386in}{0.708086in}}%
\pgfpathlineto{\pgfqpoint{0.647496in}{0.713456in}}%
\pgfpathlineto{\pgfqpoint{0.647949in}{0.714543in}}%
\pgfpathlineto{\pgfqpoint{0.649060in}{0.719944in}}%
\pgfpathlineto{\pgfqpoint{0.649247in}{0.720813in}}%
\pgfpathlineto{\pgfqpoint{0.650357in}{0.725470in}}%
\pgfpathlineto{\pgfqpoint{0.650686in}{0.726556in}}%
\pgfpathlineto{\pgfqpoint{0.651780in}{0.730064in}}%
\pgfpathlineto{\pgfqpoint{0.652015in}{0.731150in}}%
\pgfpathlineto{\pgfqpoint{0.653086in}{0.735683in}}%
\pgfpathlineto{\pgfqpoint{0.653367in}{0.736676in}}%
\pgfpathlineto{\pgfqpoint{0.653367in}{0.736738in}}%
\pgfpathlineto{\pgfqpoint{0.654477in}{0.741022in}}%
\pgfpathlineto{\pgfqpoint{0.654727in}{0.742046in}}%
\pgfpathlineto{\pgfqpoint{0.655837in}{0.746051in}}%
\pgfpathlineto{\pgfqpoint{0.656048in}{0.747106in}}%
\pgfpathlineto{\pgfqpoint{0.657151in}{0.751079in}}%
\pgfpathlineto{\pgfqpoint{0.657518in}{0.752135in}}%
\pgfpathlineto{\pgfqpoint{0.658605in}{0.755922in}}%
\pgfpathlineto{\pgfqpoint{0.658910in}{0.757008in}}%
\pgfpathlineto{\pgfqpoint{0.660004in}{0.760734in}}%
\pgfpathlineto{\pgfqpoint{0.660223in}{0.761789in}}%
\pgfpathlineto{\pgfqpoint{0.661325in}{0.765918in}}%
\pgfpathlineto{\pgfqpoint{0.661716in}{0.767004in}}%
\pgfpathlineto{\pgfqpoint{0.662826in}{0.771133in}}%
\pgfpathlineto{\pgfqpoint{0.663233in}{0.772219in}}%
\pgfpathlineto{\pgfqpoint{0.664343in}{0.775603in}}%
\pgfpathlineto{\pgfqpoint{0.664765in}{0.776627in}}%
\pgfpathlineto{\pgfqpoint{0.665867in}{0.780321in}}%
\pgfpathlineto{\pgfqpoint{0.666219in}{0.781345in}}%
\pgfpathlineto{\pgfqpoint{0.667322in}{0.785195in}}%
\pgfpathlineto{\pgfqpoint{0.667666in}{0.786219in}}%
\pgfpathlineto{\pgfqpoint{0.668776in}{0.789199in}}%
\pgfpathlineto{\pgfqpoint{0.669065in}{0.790255in}}%
\pgfpathlineto{\pgfqpoint{0.670175in}{0.793576in}}%
\pgfpathlineto{\pgfqpoint{0.670558in}{0.794663in}}%
\pgfpathlineto{\pgfqpoint{0.671668in}{0.797922in}}%
\pgfpathlineto{\pgfqpoint{0.671942in}{0.798977in}}%
\pgfpathlineto{\pgfqpoint{0.673052in}{0.802423in}}%
\pgfpathlineto{\pgfqpoint{0.673357in}{0.803447in}}%
\pgfpathlineto{\pgfqpoint{0.674443in}{0.806055in}}%
\pgfpathlineto{\pgfqpoint{0.674780in}{0.807142in}}%
\pgfpathlineto{\pgfqpoint{0.675882in}{0.811084in}}%
\pgfpathlineto{\pgfqpoint{0.676359in}{0.812139in}}%
\pgfpathlineto{\pgfqpoint{0.677445in}{0.814995in}}%
\pgfpathlineto{\pgfqpoint{0.677961in}{0.816051in}}%
\pgfpathlineto{\pgfqpoint{0.679056in}{0.819527in}}%
\pgfpathlineto{\pgfqpoint{0.679486in}{0.820614in}}%
\pgfpathlineto{\pgfqpoint{0.680580in}{0.823904in}}%
\pgfpathlineto{\pgfqpoint{0.680987in}{0.824991in}}%
\pgfpathlineto{\pgfqpoint{0.682097in}{0.827722in}}%
\pgfpathlineto{\pgfqpoint{0.682417in}{0.828809in}}%
\pgfpathlineto{\pgfqpoint{0.683442in}{0.831572in}}%
\pgfpathlineto{\pgfqpoint{0.683825in}{0.832658in}}%
\pgfpathlineto{\pgfqpoint{0.684888in}{0.835980in}}%
\pgfpathlineto{\pgfqpoint{0.685435in}{0.837066in}}%
\pgfpathlineto{\pgfqpoint{0.686545in}{0.839829in}}%
\pgfpathlineto{\pgfqpoint{0.687053in}{0.840853in}}%
\pgfpathlineto{\pgfqpoint{0.688140in}{0.844516in}}%
\pgfpathlineto{\pgfqpoint{0.688617in}{0.845603in}}%
\pgfpathlineto{\pgfqpoint{0.689703in}{0.848490in}}%
\pgfpathlineto{\pgfqpoint{0.690227in}{0.849545in}}%
\pgfpathlineto{\pgfqpoint{0.691290in}{0.852215in}}%
\pgfpathlineto{\pgfqpoint{0.691814in}{0.853239in}}%
\pgfpathlineto{\pgfqpoint{0.692901in}{0.855071in}}%
\pgfpathlineto{\pgfqpoint{0.693229in}{0.856126in}}%
\pgfpathlineto{\pgfqpoint{0.694339in}{0.858827in}}%
\pgfpathlineto{\pgfqpoint{0.694722in}{0.859913in}}%
\pgfpathlineto{\pgfqpoint{0.695817in}{0.862365in}}%
\pgfpathlineto{\pgfqpoint{0.696442in}{0.863452in}}%
\pgfpathlineto{\pgfqpoint{0.697490in}{0.866867in}}%
\pgfpathlineto{\pgfqpoint{0.697951in}{0.867953in}}%
\pgfpathlineto{\pgfqpoint{0.699053in}{0.869909in}}%
\pgfpathlineto{\pgfqpoint{0.699499in}{0.870995in}}%
\pgfpathlineto{\pgfqpoint{0.700609in}{0.873510in}}%
\pgfpathlineto{\pgfqpoint{0.701039in}{0.874596in}}%
\pgfpathlineto{\pgfqpoint{0.702149in}{0.876459in}}%
\pgfpathlineto{\pgfqpoint{0.702556in}{0.877545in}}%
\pgfpathlineto{\pgfqpoint{0.703642in}{0.879842in}}%
\pgfpathlineto{\pgfqpoint{0.704448in}{0.880898in}}%
\pgfpathlineto{\pgfqpoint{0.705558in}{0.882915in}}%
\pgfpathlineto{\pgfqpoint{0.706136in}{0.884002in}}%
\pgfpathlineto{\pgfqpoint{0.707246in}{0.886020in}}%
\pgfpathlineto{\pgfqpoint{0.707919in}{0.887106in}}%
\pgfpathlineto{\pgfqpoint{0.709005in}{0.889062in}}%
\pgfpathlineto{\pgfqpoint{0.709623in}{0.890148in}}%
\pgfpathlineto{\pgfqpoint{0.710733in}{0.892787in}}%
\pgfpathlineto{\pgfqpoint{0.711452in}{0.893873in}}%
\pgfpathlineto{\pgfqpoint{0.712547in}{0.896077in}}%
\pgfpathlineto{\pgfqpoint{0.713141in}{0.897133in}}%
\pgfpathlineto{\pgfqpoint{0.714220in}{0.899181in}}%
\pgfpathlineto{\pgfqpoint{0.714845in}{0.900237in}}%
\pgfpathlineto{\pgfqpoint{0.715955in}{0.903000in}}%
\pgfpathlineto{\pgfqpoint{0.716338in}{0.904086in}}%
\pgfpathlineto{\pgfqpoint{0.717440in}{0.906476in}}%
\pgfpathlineto{\pgfqpoint{0.717949in}{0.907563in}}%
\pgfpathlineto{\pgfqpoint{0.719051in}{0.910046in}}%
\pgfpathlineto{\pgfqpoint{0.719731in}{0.911133in}}%
\pgfpathlineto{\pgfqpoint{0.720825in}{0.913430in}}%
\pgfpathlineto{\pgfqpoint{0.721357in}{0.914454in}}%
\pgfpathlineto{\pgfqpoint{0.722467in}{0.917248in}}%
\pgfpathlineto{\pgfqpoint{0.722968in}{0.918334in}}%
\pgfpathlineto{\pgfqpoint{0.724078in}{0.920507in}}%
\pgfpathlineto{\pgfqpoint{0.724836in}{0.921594in}}%
\pgfpathlineto{\pgfqpoint{0.725946in}{0.923953in}}%
\pgfpathlineto{\pgfqpoint{0.726439in}{0.925040in}}%
\pgfpathlineto{\pgfqpoint{0.727549in}{0.927585in}}%
\pgfpathlineto{\pgfqpoint{0.728002in}{0.928671in}}%
\pgfpathlineto{\pgfqpoint{0.729081in}{0.930534in}}%
\pgfpathlineto{\pgfqpoint{0.729761in}{0.931589in}}%
\pgfpathlineto{\pgfqpoint{0.730785in}{0.933514in}}%
\pgfpathlineto{\pgfqpoint{0.731403in}{0.934600in}}%
\pgfpathlineto{\pgfqpoint{0.732513in}{0.936867in}}%
\pgfpathlineto{\pgfqpoint{0.733013in}{0.937953in}}%
\pgfpathlineto{\pgfqpoint{0.734092in}{0.940498in}}%
\pgfpathlineto{\pgfqpoint{0.734968in}{0.941554in}}%
\pgfpathlineto{\pgfqpoint{0.736000in}{0.943199in}}%
\pgfpathlineto{\pgfqpoint{0.736820in}{0.944286in}}%
\pgfpathlineto{\pgfqpoint{0.737891in}{0.946303in}}%
\pgfpathlineto{\pgfqpoint{0.738587in}{0.947359in}}%
\pgfpathlineto{\pgfqpoint{0.739697in}{0.948973in}}%
\pgfpathlineto{\pgfqpoint{0.740268in}{0.950059in}}%
\pgfpathlineto{\pgfqpoint{0.741378in}{0.952512in}}%
\pgfpathlineto{\pgfqpoint{0.741949in}{0.953567in}}%
\pgfpathlineto{\pgfqpoint{0.743035in}{0.955088in}}%
\pgfpathlineto{\pgfqpoint{0.743669in}{0.956144in}}%
\pgfpathlineto{\pgfqpoint{0.744763in}{0.958565in}}%
\pgfpathlineto{\pgfqpoint{0.744779in}{0.958565in}}%
\pgfpathlineto{\pgfqpoint{0.745498in}{0.959651in}}%
\pgfpathlineto{\pgfqpoint{0.746553in}{0.962259in}}%
\pgfpathlineto{\pgfqpoint{0.746991in}{0.963221in}}%
\pgfpathlineto{\pgfqpoint{0.748101in}{0.965674in}}%
\pgfpathlineto{\pgfqpoint{0.748946in}{0.966760in}}%
\pgfpathlineto{\pgfqpoint{0.750056in}{0.968498in}}%
\pgfpathlineto{\pgfqpoint{0.750853in}{0.969585in}}%
\pgfpathlineto{\pgfqpoint{0.751916in}{0.971416in}}%
\pgfpathlineto{\pgfqpoint{0.752620in}{0.972503in}}%
\pgfpathlineto{\pgfqpoint{0.753722in}{0.974272in}}%
\pgfpathlineto{\pgfqpoint{0.754301in}{0.975359in}}%
\pgfpathlineto{\pgfqpoint{0.755364in}{0.977221in}}%
\pgfpathlineto{\pgfqpoint{0.755403in}{0.977221in}}%
\pgfpathlineto{\pgfqpoint{0.756192in}{0.978246in}}%
\pgfpathlineto{\pgfqpoint{0.757303in}{0.980263in}}%
\pgfpathlineto{\pgfqpoint{0.757881in}{0.981319in}}%
\pgfpathlineto{\pgfqpoint{0.758936in}{0.982964in}}%
\pgfpathlineto{\pgfqpoint{0.759601in}{0.984020in}}%
\pgfpathlineto{\pgfqpoint{0.760703in}{0.985727in}}%
\pgfpathlineto{\pgfqpoint{0.761657in}{0.986751in}}%
\pgfpathlineto{\pgfqpoint{0.762751in}{0.988676in}}%
\pgfpathlineto{\pgfqpoint{0.763369in}{0.989762in}}%
\pgfpathlineto{\pgfqpoint{0.764471in}{0.991594in}}%
\pgfpathlineto{\pgfqpoint{0.765073in}{0.992680in}}%
\pgfpathlineto{\pgfqpoint{0.766183in}{0.994232in}}%
\pgfpathlineto{\pgfqpoint{0.766973in}{0.995319in}}%
\pgfpathlineto{\pgfqpoint{0.768028in}{0.997057in}}%
\pgfpathlineto{\pgfqpoint{0.769013in}{0.998144in}}%
\pgfpathlineto{\pgfqpoint{0.770124in}{1.000006in}}%
\pgfpathlineto{\pgfqpoint{0.770538in}{1.001093in}}%
\pgfpathlineto{\pgfqpoint{0.771648in}{1.003576in}}%
\pgfpathlineto{\pgfqpoint{0.772211in}{1.004600in}}%
\pgfpathlineto{\pgfqpoint{0.773313in}{1.006370in}}%
\pgfpathlineto{\pgfqpoint{0.773923in}{1.007456in}}%
\pgfpathlineto{\pgfqpoint{0.775025in}{1.009288in}}%
\pgfpathlineto{\pgfqpoint{0.775807in}{1.010374in}}%
\pgfpathlineto{\pgfqpoint{0.776862in}{1.012268in}}%
\pgfpathlineto{\pgfqpoint{0.777808in}{1.013354in}}%
\pgfpathlineto{\pgfqpoint{0.778903in}{1.015124in}}%
\pgfpathlineto{\pgfqpoint{0.779786in}{1.016210in}}%
\pgfpathlineto{\pgfqpoint{0.780896in}{1.017887in}}%
\pgfpathlineto{\pgfqpoint{0.781701in}{1.018973in}}%
\pgfpathlineto{\pgfqpoint{0.782796in}{1.020525in}}%
\pgfpathlineto{\pgfqpoint{0.783875in}{1.021612in}}%
\pgfpathlineto{\pgfqpoint{0.784985in}{1.023226in}}%
\pgfpathlineto{\pgfqpoint{0.786572in}{1.024995in}}%
\pgfpathlineto{\pgfqpoint{0.787455in}{1.026082in}}%
\pgfpathlineto{\pgfqpoint{0.788565in}{1.027944in}}%
\pgfpathlineto{\pgfqpoint{0.789073in}{1.029031in}}%
\pgfpathlineto{\pgfqpoint{0.790176in}{1.030490in}}%
\pgfpathlineto{\pgfqpoint{0.791075in}{1.031576in}}%
\pgfpathlineto{\pgfqpoint{0.792146in}{1.032973in}}%
\pgfpathlineto{\pgfqpoint{0.792826in}{1.034059in}}%
\pgfpathlineto{\pgfqpoint{0.793920in}{1.035487in}}%
\pgfpathlineto{\pgfqpoint{0.794647in}{1.036574in}}%
\pgfpathlineto{\pgfqpoint{0.795750in}{1.038343in}}%
\pgfpathlineto{\pgfqpoint{0.796524in}{1.039368in}}%
\pgfpathlineto{\pgfqpoint{0.797618in}{1.041044in}}%
\pgfpathlineto{\pgfqpoint{0.798314in}{1.042099in}}%
\pgfpathlineto{\pgfqpoint{0.799424in}{1.043496in}}%
\pgfpathlineto{\pgfqpoint{0.800386in}{1.044552in}}%
\pgfpathlineto{\pgfqpoint{0.801488in}{1.046507in}}%
\pgfpathlineto{\pgfqpoint{0.802465in}{1.047594in}}%
\pgfpathlineto{\pgfqpoint{0.803560in}{1.049425in}}%
\pgfpathlineto{\pgfqpoint{0.804318in}{1.050450in}}%
\pgfpathlineto{\pgfqpoint{0.805397in}{1.051940in}}%
\pgfpathlineto{\pgfqpoint{0.806257in}{1.053026in}}%
\pgfpathlineto{\pgfqpoint{0.807312in}{1.054547in}}%
\pgfpathlineto{\pgfqpoint{0.808070in}{1.055634in}}%
\pgfpathlineto{\pgfqpoint{0.809673in}{1.057651in}}%
\pgfpathlineto{\pgfqpoint{0.810556in}{1.058614in}}%
\pgfpathlineto{\pgfqpoint{0.811659in}{1.059887in}}%
\pgfpathlineto{\pgfqpoint{0.812558in}{1.060973in}}%
\pgfpathlineto{\pgfqpoint{0.813652in}{1.063239in}}%
\pgfpathlineto{\pgfqpoint{0.814457in}{1.064326in}}%
\pgfpathlineto{\pgfqpoint{0.815528in}{1.065412in}}%
\pgfpathlineto{\pgfqpoint{0.816248in}{1.066498in}}%
\pgfpathlineto{\pgfqpoint{0.817326in}{1.068144in}}%
\pgfpathlineto{\pgfqpoint{0.818194in}{1.069199in}}%
\pgfpathlineto{\pgfqpoint{0.819281in}{1.070844in}}%
\pgfpathlineto{\pgfqpoint{0.820031in}{1.071900in}}%
\pgfpathlineto{\pgfqpoint{0.821095in}{1.073359in}}%
\pgfpathlineto{\pgfqpoint{0.822392in}{1.074445in}}%
\pgfpathlineto{\pgfqpoint{0.823487in}{1.076184in}}%
\pgfpathlineto{\pgfqpoint{0.824503in}{1.077270in}}%
\pgfpathlineto{\pgfqpoint{0.825590in}{1.078481in}}%
\pgfpathlineto{\pgfqpoint{0.826239in}{1.079567in}}%
\pgfpathlineto{\pgfqpoint{0.827341in}{1.080747in}}%
\pgfpathlineto{\pgfqpoint{0.827974in}{1.081833in}}%
\pgfpathlineto{\pgfqpoint{0.829084in}{1.083416in}}%
\pgfpathlineto{\pgfqpoint{0.830007in}{1.084503in}}%
\pgfpathlineto{\pgfqpoint{0.831109in}{1.086055in}}%
\pgfpathlineto{\pgfqpoint{0.832102in}{1.087142in}}%
\pgfpathlineto{\pgfqpoint{0.833173in}{1.088259in}}%
\pgfpathlineto{\pgfqpoint{0.834166in}{1.089345in}}%
\pgfpathlineto{\pgfqpoint{0.835260in}{1.090494in}}%
\pgfpathlineto{\pgfqpoint{0.836120in}{1.091581in}}%
\pgfpathlineto{\pgfqpoint{0.837222in}{1.093040in}}%
\pgfpathlineto{\pgfqpoint{0.838270in}{1.094126in}}%
\pgfpathlineto{\pgfqpoint{0.839364in}{1.095430in}}%
\pgfpathlineto{\pgfqpoint{0.840318in}{1.096516in}}%
\pgfpathlineto{\pgfqpoint{0.841405in}{1.098130in}}%
\pgfpathlineto{\pgfqpoint{0.842296in}{1.099217in}}%
\pgfpathlineto{\pgfqpoint{0.843398in}{1.100334in}}%
\pgfpathlineto{\pgfqpoint{0.844188in}{1.101421in}}%
\pgfpathlineto{\pgfqpoint{0.845267in}{1.102538in}}%
\pgfpathlineto{\pgfqpoint{0.846236in}{1.103594in}}%
\pgfpathlineto{\pgfqpoint{0.847338in}{1.105115in}}%
\pgfpathlineto{\pgfqpoint{0.848276in}{1.106201in}}%
\pgfpathlineto{\pgfqpoint{0.849379in}{1.107629in}}%
\pgfpathlineto{\pgfqpoint{0.850450in}{1.108685in}}%
\pgfpathlineto{\pgfqpoint{0.851505in}{1.109926in}}%
\pgfpathlineto{\pgfqpoint{0.852396in}{1.111013in}}%
\pgfpathlineto{\pgfqpoint{0.853475in}{1.112224in}}%
\pgfpathlineto{\pgfqpoint{0.854327in}{1.113310in}}%
\pgfpathlineto{\pgfqpoint{0.855430in}{1.114396in}}%
\pgfpathlineto{\pgfqpoint{0.856266in}{1.115483in}}%
\pgfpathlineto{\pgfqpoint{0.857368in}{1.116569in}}%
\pgfpathlineto{\pgfqpoint{0.858260in}{1.117656in}}%
\pgfpathlineto{\pgfqpoint{0.859346in}{1.119208in}}%
\pgfpathlineto{\pgfqpoint{0.859362in}{1.119208in}}%
\pgfpathlineto{\pgfqpoint{0.860183in}{1.120294in}}%
\pgfpathlineto{\pgfqpoint{0.861277in}{1.121598in}}%
\pgfpathlineto{\pgfqpoint{0.862497in}{1.122685in}}%
\pgfpathlineto{\pgfqpoint{0.863607in}{1.123740in}}%
\pgfpathlineto{\pgfqpoint{0.864475in}{1.124827in}}%
\pgfpathlineto{\pgfqpoint{0.865546in}{1.126130in}}%
\pgfpathlineto{\pgfqpoint{0.866507in}{1.127217in}}%
\pgfpathlineto{\pgfqpoint{0.867617in}{1.128272in}}%
\pgfpathlineto{\pgfqpoint{0.868704in}{1.129359in}}%
\pgfpathlineto{\pgfqpoint{0.869814in}{1.131004in}}%
\pgfpathlineto{\pgfqpoint{0.870572in}{1.132091in}}%
\pgfpathlineto{\pgfqpoint{0.872230in}{1.134170in}}%
\pgfpathlineto{\pgfqpoint{0.872988in}{1.135257in}}%
\pgfpathlineto{\pgfqpoint{0.874043in}{1.136654in}}%
\pgfpathlineto{\pgfqpoint{0.874083in}{1.136654in}}%
\pgfpathlineto{\pgfqpoint{0.875028in}{1.137740in}}%
\pgfpathlineto{\pgfqpoint{0.876084in}{1.139044in}}%
\pgfpathlineto{\pgfqpoint{0.877468in}{1.140130in}}%
\pgfpathlineto{\pgfqpoint{0.878578in}{1.141527in}}%
\pgfpathlineto{\pgfqpoint{0.879688in}{1.142614in}}%
\pgfpathlineto{\pgfqpoint{0.880798in}{1.144166in}}%
\pgfpathlineto{\pgfqpoint{0.881744in}{1.145252in}}%
\pgfpathlineto{\pgfqpoint{0.882768in}{1.146401in}}%
\pgfpathlineto{\pgfqpoint{0.884144in}{1.147456in}}%
\pgfpathlineto{\pgfqpoint{0.885176in}{1.148884in}}%
\pgfpathlineto{\pgfqpoint{0.886012in}{1.149940in}}%
\pgfpathlineto{\pgfqpoint{0.887068in}{1.151368in}}%
\pgfpathlineto{\pgfqpoint{0.887083in}{1.151368in}}%
\pgfpathlineto{\pgfqpoint{0.888131in}{1.152454in}}%
\pgfpathlineto{\pgfqpoint{0.889210in}{1.153727in}}%
\pgfpathlineto{\pgfqpoint{0.890195in}{1.154751in}}%
\pgfpathlineto{\pgfqpoint{0.891242in}{1.156148in}}%
\pgfpathlineto{\pgfqpoint{0.892368in}{1.157235in}}%
\pgfpathlineto{\pgfqpoint{0.893447in}{1.158445in}}%
\pgfpathlineto{\pgfqpoint{0.894924in}{1.159532in}}%
\pgfpathlineto{\pgfqpoint{0.895925in}{1.160246in}}%
\pgfpathlineto{\pgfqpoint{0.896910in}{1.161301in}}%
\pgfpathlineto{\pgfqpoint{0.897958in}{1.162201in}}%
\pgfpathlineto{\pgfqpoint{0.898794in}{1.163288in}}%
\pgfpathlineto{\pgfqpoint{0.899834in}{1.164623in}}%
\pgfpathlineto{\pgfqpoint{0.900850in}{1.165709in}}%
\pgfpathlineto{\pgfqpoint{0.902429in}{1.167541in}}%
\pgfpathlineto{\pgfqpoint{0.903203in}{1.168565in}}%
\pgfpathlineto{\pgfqpoint{0.904266in}{1.169745in}}%
\pgfpathlineto{\pgfqpoint{0.904923in}{1.170800in}}%
\pgfpathlineto{\pgfqpoint{0.906018in}{1.172197in}}%
\pgfpathlineto{\pgfqpoint{0.907253in}{1.173283in}}%
\pgfpathlineto{\pgfqpoint{0.908347in}{1.174184in}}%
\pgfpathlineto{\pgfqpoint{0.909301in}{1.175270in}}%
\pgfpathlineto{\pgfqpoint{0.910341in}{1.176108in}}%
\pgfpathlineto{\pgfqpoint{0.911850in}{1.177195in}}%
\pgfpathlineto{\pgfqpoint{0.912928in}{1.178467in}}%
\pgfpathlineto{\pgfqpoint{0.912944in}{1.178467in}}%
\pgfpathlineto{\pgfqpoint{0.913820in}{1.179523in}}%
\pgfpathlineto{\pgfqpoint{0.914930in}{1.180299in}}%
\pgfpathlineto{\pgfqpoint{0.916079in}{1.181385in}}%
\pgfpathlineto{\pgfqpoint{0.917142in}{1.182875in}}%
\pgfpathlineto{\pgfqpoint{0.918541in}{1.183931in}}%
\pgfpathlineto{\pgfqpoint{0.919620in}{1.185266in}}%
\pgfpathlineto{\pgfqpoint{0.919652in}{1.185266in}}%
\pgfpathlineto{\pgfqpoint{0.920519in}{1.186352in}}%
\pgfpathlineto{\pgfqpoint{0.921583in}{1.187532in}}%
\pgfpathlineto{\pgfqpoint{0.922630in}{1.188618in}}%
\pgfpathlineto{\pgfqpoint{0.923732in}{1.190046in}}%
\pgfpathlineto{\pgfqpoint{0.924655in}{1.191071in}}%
\pgfpathlineto{\pgfqpoint{0.925757in}{1.192157in}}%
\pgfpathlineto{\pgfqpoint{0.927266in}{1.193244in}}%
\pgfpathlineto{\pgfqpoint{0.928368in}{1.194299in}}%
\pgfpathlineto{\pgfqpoint{0.929322in}{1.195385in}}%
\pgfpathlineto{\pgfqpoint{0.930432in}{1.196441in}}%
\pgfpathlineto{\pgfqpoint{0.931206in}{1.197496in}}%
\pgfpathlineto{\pgfqpoint{0.932246in}{1.199017in}}%
\pgfpathlineto{\pgfqpoint{0.933418in}{1.200104in}}%
\pgfpathlineto{\pgfqpoint{0.934505in}{1.201408in}}%
\pgfpathlineto{\pgfqpoint{0.935709in}{1.202494in}}%
\pgfpathlineto{\pgfqpoint{0.936780in}{1.203425in}}%
\pgfpathlineto{\pgfqpoint{0.937640in}{1.204512in}}%
\pgfpathlineto{\pgfqpoint{0.938727in}{1.205629in}}%
\pgfpathlineto{\pgfqpoint{0.940001in}{1.206716in}}%
\pgfpathlineto{\pgfqpoint{0.941088in}{1.207864in}}%
\pgfpathlineto{\pgfqpoint{0.942190in}{1.208951in}}%
\pgfpathlineto{\pgfqpoint{0.943206in}{1.209634in}}%
\pgfpathlineto{\pgfqpoint{0.945239in}{1.210720in}}%
\pgfpathlineto{\pgfqpoint{0.946349in}{1.211931in}}%
\pgfpathlineto{\pgfqpoint{0.947873in}{1.212986in}}%
\pgfpathlineto{\pgfqpoint{0.948968in}{1.214197in}}%
\pgfpathlineto{\pgfqpoint{0.950187in}{1.215252in}}%
\pgfpathlineto{\pgfqpoint{0.951290in}{1.216370in}}%
\pgfpathlineto{\pgfqpoint{0.952736in}{1.217425in}}%
\pgfpathlineto{\pgfqpoint{0.953729in}{1.218263in}}%
\pgfpathlineto{\pgfqpoint{0.953838in}{1.218263in}}%
\pgfpathlineto{\pgfqpoint{0.954987in}{1.219350in}}%
\pgfpathlineto{\pgfqpoint{0.956074in}{1.220157in}}%
\pgfpathlineto{\pgfqpoint{0.956090in}{1.220157in}}%
\pgfpathlineto{\pgfqpoint{0.957137in}{1.221244in}}%
\pgfpathlineto{\pgfqpoint{0.958247in}{1.222268in}}%
\pgfpathlineto{\pgfqpoint{0.959506in}{1.223354in}}%
\pgfpathlineto{\pgfqpoint{0.960616in}{1.224224in}}%
\pgfpathlineto{\pgfqpoint{0.962187in}{1.225310in}}%
\pgfpathlineto{\pgfqpoint{0.963297in}{1.226428in}}%
\pgfpathlineto{\pgfqpoint{0.964869in}{1.227514in}}%
\pgfpathlineto{\pgfqpoint{0.965963in}{1.228787in}}%
\pgfpathlineto{\pgfqpoint{0.967253in}{1.229873in}}%
\pgfpathlineto{\pgfqpoint{0.968355in}{1.230742in}}%
\pgfpathlineto{\pgfqpoint{0.969559in}{1.231829in}}%
\pgfpathlineto{\pgfqpoint{0.970669in}{1.233133in}}%
\pgfpathlineto{\pgfqpoint{0.971975in}{1.234219in}}%
\pgfpathlineto{\pgfqpoint{0.972991in}{1.235244in}}%
\pgfpathlineto{\pgfqpoint{0.974485in}{1.236330in}}%
\pgfpathlineto{\pgfqpoint{0.975595in}{1.237479in}}%
\pgfpathlineto{\pgfqpoint{0.977322in}{1.238472in}}%
\pgfpathlineto{\pgfqpoint{0.978425in}{1.239589in}}%
\pgfpathlineto{\pgfqpoint{0.979894in}{1.240645in}}%
\pgfpathlineto{\pgfqpoint{0.980942in}{1.241545in}}%
\pgfpathlineto{\pgfqpoint{0.982287in}{1.242632in}}%
\pgfpathlineto{\pgfqpoint{0.983397in}{1.243687in}}%
\pgfpathlineto{\pgfqpoint{0.984874in}{1.244773in}}%
\pgfpathlineto{\pgfqpoint{0.985969in}{1.245767in}}%
\pgfpathlineto{\pgfqpoint{0.987415in}{1.246760in}}%
\pgfpathlineto{\pgfqpoint{0.987415in}{1.246822in}}%
\pgfpathlineto{\pgfqpoint{0.988502in}{1.247909in}}%
\pgfpathlineto{\pgfqpoint{0.989565in}{1.248995in}}%
\pgfpathlineto{\pgfqpoint{0.990636in}{1.250051in}}%
\pgfpathlineto{\pgfqpoint{0.992027in}{1.251137in}}%
\pgfpathlineto{\pgfqpoint{0.993098in}{1.252099in}}%
\pgfpathlineto{\pgfqpoint{0.994740in}{1.253186in}}%
\pgfpathlineto{\pgfqpoint{0.995795in}{1.254024in}}%
\pgfpathlineto{\pgfqpoint{0.997296in}{1.255110in}}%
\pgfpathlineto{\pgfqpoint{0.998790in}{1.256600in}}%
\pgfpathlineto{\pgfqpoint{0.999845in}{1.257687in}}%
\pgfpathlineto{\pgfqpoint{1.000939in}{1.258711in}}%
\pgfpathlineto{\pgfqpoint{1.002737in}{1.259798in}}%
\pgfpathlineto{\pgfqpoint{1.003816in}{1.260698in}}%
\pgfpathlineto{\pgfqpoint{1.005091in}{1.261785in}}%
\pgfpathlineto{\pgfqpoint{1.006177in}{1.262778in}}%
\pgfpathlineto{\pgfqpoint{1.007233in}{1.263833in}}%
\pgfpathlineto{\pgfqpoint{1.008343in}{1.264734in}}%
\pgfpathlineto{\pgfqpoint{1.009875in}{1.265820in}}%
\pgfpathlineto{\pgfqpoint{1.010923in}{1.266534in}}%
\pgfpathlineto{\pgfqpoint{1.012431in}{1.267620in}}%
\pgfpathlineto{\pgfqpoint{1.013534in}{1.268521in}}%
\pgfpathlineto{\pgfqpoint{1.014894in}{1.269607in}}%
\pgfpathlineto{\pgfqpoint{1.016004in}{1.270756in}}%
\pgfpathlineto{\pgfqpoint{1.017224in}{1.271842in}}%
\pgfpathlineto{\pgfqpoint{1.018334in}{1.272804in}}%
\pgfpathlineto{\pgfqpoint{1.019960in}{1.273891in}}%
\pgfpathlineto{\pgfqpoint{1.020898in}{1.274853in}}%
\pgfpathlineto{\pgfqpoint{1.022735in}{1.275940in}}%
\pgfpathlineto{\pgfqpoint{1.023790in}{1.276716in}}%
\pgfpathlineto{\pgfqpoint{1.023845in}{1.276716in}}%
\pgfpathlineto{\pgfqpoint{1.025674in}{1.277802in}}%
\pgfpathlineto{\pgfqpoint{1.026785in}{1.279168in}}%
\pgfpathlineto{\pgfqpoint{1.028387in}{1.280255in}}%
\pgfpathlineto{\pgfqpoint{1.029396in}{1.281248in}}%
\pgfpathlineto{\pgfqpoint{1.030709in}{1.282334in}}%
\pgfpathlineto{\pgfqpoint{1.031811in}{1.283576in}}%
\pgfpathlineto{\pgfqpoint{1.033328in}{1.284663in}}%
\pgfpathlineto{\pgfqpoint{1.034399in}{1.285532in}}%
\pgfpathlineto{\pgfqpoint{1.035884in}{1.286618in}}%
\pgfpathlineto{\pgfqpoint{1.036979in}{1.287394in}}%
\pgfpathlineto{\pgfqpoint{1.038824in}{1.288481in}}%
\pgfpathlineto{\pgfqpoint{1.039801in}{1.289443in}}%
\pgfpathlineto{\pgfqpoint{1.041060in}{1.290530in}}%
\pgfpathlineto{\pgfqpoint{1.042146in}{1.291926in}}%
\pgfpathlineto{\pgfqpoint{1.043178in}{1.293013in}}%
\pgfpathlineto{\pgfqpoint{1.044210in}{1.294130in}}%
\pgfpathlineto{\pgfqpoint{1.045398in}{1.295217in}}%
\pgfpathlineto{\pgfqpoint{1.046438in}{1.296024in}}%
\pgfpathlineto{\pgfqpoint{1.046501in}{1.296024in}}%
\pgfpathlineto{\pgfqpoint{1.048041in}{1.297110in}}%
\pgfpathlineto{\pgfqpoint{1.049143in}{1.297824in}}%
\pgfpathlineto{\pgfqpoint{1.050441in}{1.298849in}}%
\pgfpathlineto{\pgfqpoint{1.051473in}{1.299749in}}%
\pgfpathlineto{\pgfqpoint{1.053060in}{1.300836in}}%
\pgfpathlineto{\pgfqpoint{1.054037in}{1.301860in}}%
\pgfpathlineto{\pgfqpoint{1.056163in}{1.302946in}}%
\pgfpathlineto{\pgfqpoint{1.057226in}{1.303909in}}%
\pgfpathlineto{\pgfqpoint{1.058665in}{1.304995in}}%
\pgfpathlineto{\pgfqpoint{1.059736in}{1.305492in}}%
\pgfpathlineto{\pgfqpoint{1.061448in}{1.306578in}}%
\pgfpathlineto{\pgfqpoint{1.062496in}{1.307199in}}%
\pgfpathlineto{\pgfqpoint{1.063981in}{1.308286in}}%
\pgfpathlineto{\pgfqpoint{1.065052in}{1.308938in}}%
\pgfpathlineto{\pgfqpoint{1.066146in}{1.310024in}}%
\pgfpathlineto{\pgfqpoint{1.067233in}{1.310893in}}%
\pgfpathlineto{\pgfqpoint{1.069328in}{1.311980in}}%
\pgfpathlineto{\pgfqpoint{1.070360in}{1.312663in}}%
\pgfpathlineto{\pgfqpoint{1.070384in}{1.312663in}}%
\pgfpathlineto{\pgfqpoint{1.072385in}{1.313749in}}%
\pgfpathlineto{\pgfqpoint{1.073393in}{1.314649in}}%
\pgfpathlineto{\pgfqpoint{1.074988in}{1.315736in}}%
\pgfpathlineto{\pgfqpoint{1.076043in}{1.316357in}}%
\pgfpathlineto{\pgfqpoint{1.076067in}{1.316357in}}%
\pgfpathlineto{\pgfqpoint{1.077584in}{1.317443in}}%
\pgfpathlineto{\pgfqpoint{1.078694in}{1.318312in}}%
\pgfpathlineto{\pgfqpoint{1.080038in}{1.319399in}}%
\pgfpathlineto{\pgfqpoint{1.081133in}{1.320268in}}%
\pgfpathlineto{\pgfqpoint{1.082423in}{1.321354in}}%
\pgfpathlineto{\pgfqpoint{1.083517in}{1.322255in}}%
\pgfpathlineto{\pgfqpoint{1.084932in}{1.323279in}}%
\pgfpathlineto{\pgfqpoint{1.085980in}{1.323993in}}%
\pgfpathlineto{\pgfqpoint{1.086042in}{1.323993in}}%
\pgfpathlineto{\pgfqpoint{1.087637in}{1.325048in}}%
\pgfpathlineto{\pgfqpoint{1.088739in}{1.325855in}}%
\pgfpathlineto{\pgfqpoint{1.090592in}{1.326942in}}%
\pgfpathlineto{\pgfqpoint{1.091687in}{1.328122in}}%
\pgfpathlineto{\pgfqpoint{1.093281in}{1.329208in}}%
\pgfpathlineto{\pgfqpoint{1.094227in}{1.330108in}}%
\pgfpathlineto{\pgfqpoint{1.094282in}{1.330108in}}%
\pgfpathlineto{\pgfqpoint{1.096096in}{1.331164in}}%
\pgfpathlineto{\pgfqpoint{1.097112in}{1.331909in}}%
\pgfpathlineto{\pgfqpoint{1.097206in}{1.331909in}}%
\pgfpathlineto{\pgfqpoint{1.099442in}{1.332995in}}%
\pgfpathlineto{\pgfqpoint{1.100505in}{1.333864in}}%
\pgfpathlineto{\pgfqpoint{1.102076in}{1.334951in}}%
\pgfpathlineto{\pgfqpoint{1.103155in}{1.335665in}}%
\pgfpathlineto{\pgfqpoint{1.103179in}{1.335665in}}%
\pgfpathlineto{\pgfqpoint{1.104601in}{1.336751in}}%
\pgfpathlineto{\pgfqpoint{1.105704in}{1.337558in}}%
\pgfpathlineto{\pgfqpoint{1.107353in}{1.338645in}}%
\pgfpathlineto{\pgfqpoint{1.108455in}{1.339359in}}%
\pgfpathlineto{\pgfqpoint{1.109956in}{1.340414in}}%
\pgfpathlineto{\pgfqpoint{1.111059in}{1.341501in}}%
\pgfpathlineto{\pgfqpoint{1.112224in}{1.342587in}}%
\pgfpathlineto{\pgfqpoint{1.113310in}{1.343332in}}%
\pgfpathlineto{\pgfqpoint{1.115186in}{1.344419in}}%
\pgfpathlineto{\pgfqpoint{1.116164in}{1.345350in}}%
\pgfpathlineto{\pgfqpoint{1.116281in}{1.345350in}}%
\pgfpathlineto{\pgfqpoint{1.118001in}{1.346436in}}%
\pgfpathlineto{\pgfqpoint{1.119111in}{1.347181in}}%
\pgfpathlineto{\pgfqpoint{1.120323in}{1.348268in}}%
\pgfpathlineto{\pgfqpoint{1.121245in}{1.349013in}}%
\pgfpathlineto{\pgfqpoint{1.122637in}{1.350068in}}%
\pgfpathlineto{\pgfqpoint{1.123747in}{1.350503in}}%
\pgfpathlineto{\pgfqpoint{1.125302in}{1.351527in}}%
\pgfpathlineto{\pgfqpoint{1.126303in}{1.352428in}}%
\pgfpathlineto{\pgfqpoint{1.128093in}{1.353514in}}%
\pgfpathlineto{\pgfqpoint{1.129203in}{1.353855in}}%
\pgfpathlineto{\pgfqpoint{1.131017in}{1.354942in}}%
\pgfpathlineto{\pgfqpoint{1.132119in}{1.355625in}}%
\pgfpathlineto{\pgfqpoint{1.134441in}{1.356711in}}%
\pgfpathlineto{\pgfqpoint{1.135356in}{1.357363in}}%
\pgfpathlineto{\pgfqpoint{1.137514in}{1.358450in}}%
\pgfpathlineto{\pgfqpoint{1.138561in}{1.359195in}}%
\pgfpathlineto{\pgfqpoint{1.140062in}{1.360219in}}%
\pgfpathlineto{\pgfqpoint{1.141141in}{1.360871in}}%
\pgfpathlineto{\pgfqpoint{1.142751in}{1.361957in}}%
\pgfpathlineto{\pgfqpoint{1.143854in}{1.362609in}}%
\pgfpathlineto{\pgfqpoint{1.145441in}{1.363634in}}%
\pgfpathlineto{\pgfqpoint{1.146527in}{1.364627in}}%
\pgfpathlineto{\pgfqpoint{1.148779in}{1.365714in}}%
\pgfpathlineto{\pgfqpoint{1.149842in}{1.366334in}}%
\pgfpathlineto{\pgfqpoint{1.151648in}{1.367421in}}%
\pgfpathlineto{\pgfqpoint{1.152672in}{1.368042in}}%
\pgfpathlineto{\pgfqpoint{1.154759in}{1.369128in}}%
\pgfpathlineto{\pgfqpoint{1.155846in}{1.369935in}}%
\pgfpathlineto{\pgfqpoint{1.157222in}{1.370991in}}%
\pgfpathlineto{\pgfqpoint{1.158215in}{1.371705in}}%
\pgfpathlineto{\pgfqpoint{1.160247in}{1.372791in}}%
\pgfpathlineto{\pgfqpoint{1.161326in}{1.373226in}}%
\pgfpathlineto{\pgfqpoint{1.163312in}{1.374312in}}%
\pgfpathlineto{\pgfqpoint{1.164406in}{1.374933in}}%
\pgfpathlineto{\pgfqpoint{1.166720in}{1.376020in}}%
\pgfpathlineto{\pgfqpoint{1.167823in}{1.376796in}}%
\pgfpathlineto{\pgfqpoint{1.169926in}{1.377882in}}%
\pgfpathlineto{\pgfqpoint{1.170965in}{1.378441in}}%
\pgfpathlineto{\pgfqpoint{1.173545in}{1.379527in}}%
\pgfpathlineto{\pgfqpoint{1.174647in}{1.380117in}}%
\pgfpathlineto{\pgfqpoint{1.176328in}{1.381204in}}%
\pgfpathlineto{\pgfqpoint{1.177430in}{1.381793in}}%
\pgfpathlineto{\pgfqpoint{1.179275in}{1.382849in}}%
\pgfpathlineto{\pgfqpoint{1.180346in}{1.383594in}}%
\pgfpathlineto{\pgfqpoint{1.182660in}{1.384680in}}%
\pgfpathlineto{\pgfqpoint{1.183677in}{1.385301in}}%
\pgfpathlineto{\pgfqpoint{1.185717in}{1.386388in}}%
\pgfpathlineto{\pgfqpoint{1.186796in}{1.387102in}}%
\pgfpathlineto{\pgfqpoint{1.188977in}{1.388188in}}%
\pgfpathlineto{\pgfqpoint{1.190087in}{1.388809in}}%
\pgfpathlineto{\pgfqpoint{1.192268in}{1.389895in}}%
\pgfpathlineto{\pgfqpoint{1.193371in}{1.390858in}}%
\pgfpathlineto{\pgfqpoint{1.195216in}{1.391944in}}%
\pgfpathlineto{\pgfqpoint{1.196169in}{1.392410in}}%
\pgfpathlineto{\pgfqpoint{1.198319in}{1.393496in}}%
\pgfpathlineto{\pgfqpoint{1.199367in}{1.394055in}}%
\pgfpathlineto{\pgfqpoint{1.199382in}{1.394055in}}%
\pgfpathlineto{\pgfqpoint{1.202486in}{1.395142in}}%
\pgfpathlineto{\pgfqpoint{1.203526in}{1.395918in}}%
\pgfpathlineto{\pgfqpoint{1.205465in}{1.397004in}}%
\pgfpathlineto{\pgfqpoint{1.206504in}{1.397656in}}%
\pgfpathlineto{\pgfqpoint{1.206528in}{1.397656in}}%
\pgfpathlineto{\pgfqpoint{1.209545in}{1.398742in}}%
\pgfpathlineto{\pgfqpoint{1.210421in}{1.399208in}}%
\pgfpathlineto{\pgfqpoint{1.210616in}{1.399208in}}%
\pgfpathlineto{\pgfqpoint{1.212524in}{1.400294in}}%
\pgfpathlineto{\pgfqpoint{1.213493in}{1.400822in}}%
\pgfpathlineto{\pgfqpoint{1.215307in}{1.401909in}}%
\pgfpathlineto{\pgfqpoint{1.216378in}{1.402343in}}%
\pgfpathlineto{\pgfqpoint{1.219489in}{1.403430in}}%
\pgfpathlineto{\pgfqpoint{1.220600in}{1.404144in}}%
\pgfpathlineto{\pgfqpoint{1.222398in}{1.405230in}}%
\pgfpathlineto{\pgfqpoint{1.223508in}{1.405665in}}%
\pgfpathlineto{\pgfqpoint{1.225392in}{1.406751in}}%
\pgfpathlineto{\pgfqpoint{1.226463in}{1.407496in}}%
\pgfpathlineto{\pgfqpoint{1.226486in}{1.407496in}}%
\pgfpathlineto{\pgfqpoint{1.228706in}{1.408583in}}%
\pgfpathlineto{\pgfqpoint{1.229723in}{1.408924in}}%
\pgfpathlineto{\pgfqpoint{1.232256in}{1.410011in}}%
\pgfpathlineto{\pgfqpoint{1.233225in}{1.410538in}}%
\pgfpathlineto{\pgfqpoint{1.236000in}{1.411625in}}%
\pgfpathlineto{\pgfqpoint{1.237110in}{1.412184in}}%
\pgfpathlineto{\pgfqpoint{1.239596in}{1.413239in}}%
\pgfpathlineto{\pgfqpoint{1.240660in}{1.413922in}}%
\pgfpathlineto{\pgfqpoint{1.243615in}{1.415008in}}%
\pgfpathlineto{\pgfqpoint{1.244717in}{1.415536in}}%
\pgfpathlineto{\pgfqpoint{1.247226in}{1.416623in}}%
\pgfpathlineto{\pgfqpoint{1.248337in}{1.417492in}}%
\pgfpathlineto{\pgfqpoint{1.250283in}{1.418578in}}%
\pgfpathlineto{\pgfqpoint{1.251292in}{1.419013in}}%
\pgfpathlineto{\pgfqpoint{1.253785in}{1.420099in}}%
\pgfpathlineto{\pgfqpoint{1.254770in}{1.420472in}}%
\pgfpathlineto{\pgfqpoint{1.256928in}{1.421558in}}%
\pgfpathlineto{\pgfqpoint{1.257999in}{1.422365in}}%
\pgfpathlineto{\pgfqpoint{1.259500in}{1.423452in}}%
\pgfpathlineto{\pgfqpoint{1.260540in}{1.423949in}}%
\pgfpathlineto{\pgfqpoint{1.262526in}{1.425035in}}%
\pgfpathlineto{\pgfqpoint{1.263597in}{1.425501in}}%
\pgfpathlineto{\pgfqpoint{1.265668in}{1.426587in}}%
\pgfpathlineto{\pgfqpoint{1.266778in}{1.427208in}}%
\pgfpathlineto{\pgfqpoint{1.268787in}{1.428294in}}%
\pgfpathlineto{\pgfqpoint{1.269812in}{1.428915in}}%
\pgfpathlineto{\pgfqpoint{1.271664in}{1.430002in}}%
\pgfpathlineto{\pgfqpoint{1.272774in}{1.430530in}}%
\pgfpathlineto{\pgfqpoint{1.275026in}{1.431616in}}%
\pgfpathlineto{\pgfqpoint{1.275995in}{1.432175in}}%
\pgfpathlineto{\pgfqpoint{1.279177in}{1.433261in}}%
\pgfpathlineto{\pgfqpoint{1.280264in}{1.433727in}}%
\pgfpathlineto{\pgfqpoint{1.282625in}{1.434813in}}%
\pgfpathlineto{\pgfqpoint{1.283555in}{1.435341in}}%
\pgfpathlineto{\pgfqpoint{1.286526in}{1.436428in}}%
\pgfpathlineto{\pgfqpoint{1.287550in}{1.436986in}}%
\pgfpathlineto{\pgfqpoint{1.290231in}{1.438073in}}%
\pgfpathlineto{\pgfqpoint{1.291240in}{1.438632in}}%
\pgfpathlineto{\pgfqpoint{1.291255in}{1.438632in}}%
\pgfpathlineto{\pgfqpoint{1.293718in}{1.439718in}}%
\pgfpathlineto{\pgfqpoint{1.294805in}{1.440122in}}%
\pgfpathlineto{\pgfqpoint{1.294828in}{1.440122in}}%
\pgfpathlineto{\pgfqpoint{1.297244in}{1.441208in}}%
\pgfpathlineto{\pgfqpoint{1.298260in}{1.441643in}}%
\pgfpathlineto{\pgfqpoint{1.300840in}{1.442729in}}%
\pgfpathlineto{\pgfqpoint{1.301950in}{1.443319in}}%
\pgfpathlineto{\pgfqpoint{1.303818in}{1.444374in}}%
\pgfpathlineto{\pgfqpoint{1.304874in}{1.444840in}}%
\pgfpathlineto{\pgfqpoint{1.308415in}{1.445926in}}%
\pgfpathlineto{\pgfqpoint{1.309525in}{1.446392in}}%
\pgfpathlineto{\pgfqpoint{1.312269in}{1.447479in}}%
\pgfpathlineto{\pgfqpoint{1.313239in}{1.448068in}}%
\pgfpathlineto{\pgfqpoint{1.313301in}{1.448068in}}%
\pgfpathlineto{\pgfqpoint{1.315271in}{1.449124in}}%
\pgfpathlineto{\pgfqpoint{1.316217in}{1.449558in}}%
\pgfpathlineto{\pgfqpoint{1.316287in}{1.449558in}}%
\pgfpathlineto{\pgfqpoint{1.318211in}{1.450583in}}%
\pgfpathlineto{\pgfqpoint{1.319297in}{1.451328in}}%
\pgfpathlineto{\pgfqpoint{1.321017in}{1.452414in}}%
\pgfpathlineto{\pgfqpoint{1.322119in}{1.453159in}}%
\pgfpathlineto{\pgfqpoint{1.324629in}{1.454246in}}%
\pgfpathlineto{\pgfqpoint{1.325661in}{1.454463in}}%
\pgfpathlineto{\pgfqpoint{1.328076in}{1.455549in}}%
\pgfpathlineto{\pgfqpoint{1.329163in}{1.455984in}}%
\pgfpathlineto{\pgfqpoint{1.331876in}{1.457071in}}%
\pgfpathlineto{\pgfqpoint{1.332869in}{1.457722in}}%
\pgfpathlineto{\pgfqpoint{1.335331in}{1.458809in}}%
\pgfpathlineto{\pgfqpoint{1.336308in}{1.459057in}}%
\pgfpathlineto{\pgfqpoint{1.336434in}{1.459057in}}%
\pgfpathlineto{\pgfqpoint{1.339748in}{1.460144in}}%
\pgfpathlineto{\pgfqpoint{1.340999in}{1.460858in}}%
\pgfpathlineto{\pgfqpoint{1.343399in}{1.461944in}}%
\pgfpathlineto{\pgfqpoint{1.344611in}{1.462410in}}%
\pgfpathlineto{\pgfqpoint{1.347378in}{1.463496in}}%
\pgfpathlineto{\pgfqpoint{1.348387in}{1.463900in}}%
\pgfpathlineto{\pgfqpoint{1.348410in}{1.463900in}}%
\pgfpathlineto{\pgfqpoint{1.350920in}{1.464986in}}%
\pgfpathlineto{\pgfqpoint{1.351764in}{1.465514in}}%
\pgfpathlineto{\pgfqpoint{1.354148in}{1.466600in}}%
\pgfpathlineto{\pgfqpoint{1.355180in}{1.467097in}}%
\pgfpathlineto{\pgfqpoint{1.358292in}{1.468153in}}%
\pgfpathlineto{\pgfqpoint{1.359386in}{1.468836in}}%
\pgfpathlineto{\pgfqpoint{1.361927in}{1.469922in}}%
\pgfpathlineto{\pgfqpoint{1.362982in}{1.470357in}}%
\pgfpathlineto{\pgfqpoint{1.366680in}{1.471443in}}%
\pgfpathlineto{\pgfqpoint{1.367688in}{1.471753in}}%
\pgfpathlineto{\pgfqpoint{1.371472in}{1.472840in}}%
\pgfpathlineto{\pgfqpoint{1.372504in}{1.473026in}}%
\pgfpathlineto{\pgfqpoint{1.376171in}{1.474113in}}%
\pgfpathlineto{\pgfqpoint{1.377257in}{1.474702in}}%
\pgfpathlineto{\pgfqpoint{1.380306in}{1.475789in}}%
\pgfpathlineto{\pgfqpoint{1.381291in}{1.476317in}}%
\pgfpathlineto{\pgfqpoint{1.384965in}{1.477403in}}%
\pgfpathlineto{\pgfqpoint{1.385935in}{1.477807in}}%
\pgfpathlineto{\pgfqpoint{1.386076in}{1.477807in}}%
\pgfpathlineto{\pgfqpoint{1.388851in}{1.478893in}}%
\pgfpathlineto{\pgfqpoint{1.389734in}{1.479359in}}%
\pgfpathlineto{\pgfqpoint{1.393354in}{1.480445in}}%
\pgfpathlineto{\pgfqpoint{1.394323in}{1.480942in}}%
\pgfpathlineto{\pgfqpoint{1.396606in}{1.482028in}}%
\pgfpathlineto{\pgfqpoint{1.397638in}{1.482525in}}%
\pgfpathlineto{\pgfqpoint{1.401750in}{1.483612in}}%
\pgfpathlineto{\pgfqpoint{1.402837in}{1.484170in}}%
\pgfpathlineto{\pgfqpoint{1.406534in}{1.485257in}}%
\pgfpathlineto{\pgfqpoint{1.407582in}{1.485567in}}%
\pgfpathlineto{\pgfqpoint{1.410287in}{1.486654in}}%
\pgfpathlineto{\pgfqpoint{1.411256in}{1.487088in}}%
\pgfpathlineto{\pgfqpoint{1.413687in}{1.488175in}}%
\pgfpathlineto{\pgfqpoint{1.414758in}{1.488734in}}%
\pgfpathlineto{\pgfqpoint{1.419027in}{1.489820in}}%
\pgfpathlineto{\pgfqpoint{1.419957in}{1.490193in}}%
\pgfpathlineto{\pgfqpoint{1.420137in}{1.490193in}}%
\pgfpathlineto{\pgfqpoint{1.423358in}{1.491279in}}%
\pgfpathlineto{\pgfqpoint{1.424445in}{1.491683in}}%
\pgfpathlineto{\pgfqpoint{1.427728in}{1.492769in}}%
\pgfpathlineto{\pgfqpoint{1.428838in}{1.493266in}}%
\pgfpathlineto{\pgfqpoint{1.432247in}{1.494352in}}%
\pgfpathlineto{\pgfqpoint{1.433216in}{1.494818in}}%
\pgfpathlineto{\pgfqpoint{1.437039in}{1.495904in}}%
\pgfpathlineto{\pgfqpoint{1.437735in}{1.496091in}}%
\pgfpathlineto{\pgfqpoint{1.438047in}{1.496091in}}%
\pgfpathlineto{\pgfqpoint{1.441831in}{1.497177in}}%
\pgfpathlineto{\pgfqpoint{1.442941in}{1.497456in}}%
\pgfpathlineto{\pgfqpoint{1.444739in}{1.498543in}}%
\pgfpathlineto{\pgfqpoint{1.445763in}{1.499164in}}%
\pgfpathlineto{\pgfqpoint{1.449273in}{1.500250in}}%
\pgfpathlineto{\pgfqpoint{1.450313in}{1.500498in}}%
\pgfpathlineto{\pgfqpoint{1.454011in}{1.501585in}}%
\pgfpathlineto{\pgfqpoint{1.454824in}{1.502051in}}%
\pgfpathlineto{\pgfqpoint{1.455051in}{1.502051in}}%
\pgfpathlineto{\pgfqpoint{1.459108in}{1.503137in}}%
\pgfpathlineto{\pgfqpoint{1.460030in}{1.503634in}}%
\pgfpathlineto{\pgfqpoint{1.460109in}{1.503634in}}%
\pgfpathlineto{\pgfqpoint{1.461875in}{1.504720in}}%
\pgfpathlineto{\pgfqpoint{1.462688in}{1.504969in}}%
\pgfpathlineto{\pgfqpoint{1.462829in}{1.504969in}}%
\pgfpathlineto{\pgfqpoint{1.465831in}{1.506055in}}%
\pgfpathlineto{\pgfqpoint{1.466887in}{1.506365in}}%
\pgfpathlineto{\pgfqpoint{1.469756in}{1.507452in}}%
\pgfpathlineto{\pgfqpoint{1.470819in}{1.507700in}}%
\pgfpathlineto{\pgfqpoint{1.473680in}{1.508787in}}%
\pgfpathlineto{\pgfqpoint{1.474657in}{1.509128in}}%
\pgfpathlineto{\pgfqpoint{1.478754in}{1.510215in}}%
\pgfpathlineto{\pgfqpoint{1.479637in}{1.510525in}}%
\pgfpathlineto{\pgfqpoint{1.483233in}{1.511612in}}%
\pgfpathlineto{\pgfqpoint{1.484281in}{1.512046in}}%
\pgfpathlineto{\pgfqpoint{1.487431in}{1.513133in}}%
\pgfpathlineto{\pgfqpoint{1.488534in}{1.513691in}}%
\pgfpathlineto{\pgfqpoint{1.492732in}{1.514778in}}%
\pgfpathlineto{\pgfqpoint{1.493717in}{1.515306in}}%
\pgfpathlineto{\pgfqpoint{1.493779in}{1.515306in}}%
\pgfpathlineto{\pgfqpoint{1.497196in}{1.516392in}}%
\pgfpathlineto{\pgfqpoint{1.498243in}{1.516734in}}%
\pgfpathlineto{\pgfqpoint{1.503059in}{1.517820in}}%
\pgfpathlineto{\pgfqpoint{1.504036in}{1.518224in}}%
\pgfpathlineto{\pgfqpoint{1.504052in}{1.518224in}}%
\pgfpathlineto{\pgfqpoint{1.507788in}{1.519310in}}%
\pgfpathlineto{\pgfqpoint{1.508797in}{1.519745in}}%
\pgfpathlineto{\pgfqpoint{1.508813in}{1.519745in}}%
\pgfpathlineto{\pgfqpoint{1.513183in}{1.520800in}}%
\pgfpathlineto{\pgfqpoint{1.514160in}{1.521142in}}%
\pgfpathlineto{\pgfqpoint{1.517420in}{1.522228in}}%
\pgfpathlineto{\pgfqpoint{1.518491in}{1.522538in}}%
\pgfpathlineto{\pgfqpoint{1.521860in}{1.523625in}}%
\pgfpathlineto{\pgfqpoint{1.522845in}{1.523935in}}%
\pgfpathlineto{\pgfqpoint{1.526301in}{1.525022in}}%
\pgfpathlineto{\pgfqpoint{1.527333in}{1.525425in}}%
\pgfpathlineto{\pgfqpoint{1.527395in}{1.525425in}}%
\pgfpathlineto{\pgfqpoint{1.531656in}{1.526512in}}%
\pgfpathlineto{\pgfqpoint{1.532735in}{1.526946in}}%
\pgfpathlineto{\pgfqpoint{1.535650in}{1.528033in}}%
\pgfpathlineto{\pgfqpoint{1.536737in}{1.528592in}}%
\pgfpathlineto{\pgfqpoint{1.536753in}{1.528592in}}%
\pgfpathlineto{\pgfqpoint{1.540529in}{1.529647in}}%
\pgfpathlineto{\pgfqpoint{1.541592in}{1.530020in}}%
\pgfpathlineto{\pgfqpoint{1.546619in}{1.531106in}}%
\pgfpathlineto{\pgfqpoint{1.547674in}{1.531479in}}%
\pgfpathlineto{\pgfqpoint{1.551849in}{1.532565in}}%
\pgfpathlineto{\pgfqpoint{1.552951in}{1.533000in}}%
\pgfpathlineto{\pgfqpoint{1.556633in}{1.534086in}}%
\pgfpathlineto{\pgfqpoint{1.557696in}{1.534396in}}%
\pgfpathlineto{\pgfqpoint{1.561300in}{1.535483in}}%
\pgfpathlineto{\pgfqpoint{1.562090in}{1.535918in}}%
\pgfpathlineto{\pgfqpoint{1.562246in}{1.535918in}}%
\pgfpathlineto{\pgfqpoint{1.567195in}{1.537004in}}%
\pgfpathlineto{\pgfqpoint{1.568203in}{1.537439in}}%
\pgfpathlineto{\pgfqpoint{1.572128in}{1.538463in}}%
\pgfpathlineto{\pgfqpoint{1.573230in}{1.538804in}}%
\pgfpathlineto{\pgfqpoint{1.577514in}{1.539891in}}%
\pgfpathlineto{\pgfqpoint{1.578624in}{1.540232in}}%
\pgfpathlineto{\pgfqpoint{1.583424in}{1.541319in}}%
\pgfpathlineto{\pgfqpoint{1.584362in}{1.541722in}}%
\pgfpathlineto{\pgfqpoint{1.588693in}{1.542809in}}%
\pgfpathlineto{\pgfqpoint{1.589631in}{1.543306in}}%
\pgfpathlineto{\pgfqpoint{1.595080in}{1.544392in}}%
\pgfpathlineto{\pgfqpoint{1.596018in}{1.544702in}}%
\pgfpathlineto{\pgfqpoint{1.601014in}{1.545758in}}%
\pgfpathlineto{\pgfqpoint{1.602100in}{1.546130in}}%
\pgfpathlineto{\pgfqpoint{1.607448in}{1.547217in}}%
\pgfpathlineto{\pgfqpoint{1.608441in}{1.547558in}}%
\pgfpathlineto{\pgfqpoint{1.608487in}{1.547558in}}%
\pgfpathlineto{\pgfqpoint{1.612279in}{1.548645in}}%
\pgfpathlineto{\pgfqpoint{1.613045in}{1.548893in}}%
\pgfpathlineto{\pgfqpoint{1.613287in}{1.548893in}}%
\pgfpathlineto{\pgfqpoint{1.618150in}{1.549980in}}%
\pgfpathlineto{\pgfqpoint{1.619190in}{1.550507in}}%
\pgfpathlineto{\pgfqpoint{1.619252in}{1.550507in}}%
\pgfpathlineto{\pgfqpoint{1.627125in}{1.551594in}}%
\pgfpathlineto{\pgfqpoint{1.628227in}{1.551935in}}%
\pgfpathlineto{\pgfqpoint{1.633543in}{1.553022in}}%
\pgfpathlineto{\pgfqpoint{1.634387in}{1.553239in}}%
\pgfpathlineto{\pgfqpoint{1.639946in}{1.554326in}}%
\pgfpathlineto{\pgfqpoint{1.640970in}{1.554698in}}%
\pgfpathlineto{\pgfqpoint{1.647388in}{1.555785in}}%
\pgfpathlineto{\pgfqpoint{1.648264in}{1.556002in}}%
\pgfpathlineto{\pgfqpoint{1.652110in}{1.557088in}}%
\pgfpathlineto{\pgfqpoint{1.652931in}{1.557337in}}%
\pgfpathlineto{\pgfqpoint{1.653173in}{1.557337in}}%
\pgfpathlineto{\pgfqpoint{1.657660in}{1.558423in}}%
\pgfpathlineto{\pgfqpoint{1.658771in}{1.558951in}}%
\pgfpathlineto{\pgfqpoint{1.665423in}{1.560037in}}%
\pgfpathlineto{\pgfqpoint{1.666518in}{1.560410in}}%
\pgfpathlineto{\pgfqpoint{1.672217in}{1.561496in}}%
\pgfpathlineto{\pgfqpoint{1.673272in}{1.561745in}}%
\pgfpathlineto{\pgfqpoint{1.680449in}{1.562831in}}%
\pgfpathlineto{\pgfqpoint{1.681528in}{1.562986in}}%
\pgfpathlineto{\pgfqpoint{1.685601in}{1.564073in}}%
\pgfpathlineto{\pgfqpoint{1.686703in}{1.564445in}}%
\pgfpathlineto{\pgfqpoint{1.693027in}{1.565532in}}%
\pgfpathlineto{\pgfqpoint{1.694083in}{1.565842in}}%
\pgfpathlineto{\pgfqpoint{1.699313in}{1.566929in}}%
\pgfpathlineto{\pgfqpoint{1.700313in}{1.567208in}}%
\pgfpathlineto{\pgfqpoint{1.705848in}{1.568294in}}%
\pgfpathlineto{\pgfqpoint{1.706927in}{1.568512in}}%
\pgfpathlineto{\pgfqpoint{1.711907in}{1.569598in}}%
\pgfpathlineto{\pgfqpoint{1.712908in}{1.569909in}}%
\pgfpathlineto{\pgfqpoint{1.718349in}{1.570995in}}%
\pgfpathlineto{\pgfqpoint{1.719146in}{1.571337in}}%
\pgfpathlineto{\pgfqpoint{1.725580in}{1.572423in}}%
\pgfpathlineto{\pgfqpoint{1.726440in}{1.572609in}}%
\pgfpathlineto{\pgfqpoint{1.732514in}{1.573696in}}%
\pgfpathlineto{\pgfqpoint{1.733531in}{1.574006in}}%
\pgfpathlineto{\pgfqpoint{1.743217in}{1.575093in}}%
\pgfpathlineto{\pgfqpoint{1.744327in}{1.575310in}}%
\pgfpathlineto{\pgfqpoint{1.749666in}{1.576396in}}%
\pgfpathlineto{\pgfqpoint{1.750448in}{1.576645in}}%
\pgfpathlineto{\pgfqpoint{1.757718in}{1.577731in}}%
\pgfpathlineto{\pgfqpoint{1.758805in}{1.577980in}}%
\pgfpathlineto{\pgfqpoint{1.767334in}{1.579066in}}%
\pgfpathlineto{\pgfqpoint{1.768171in}{1.579377in}}%
\pgfpathlineto{\pgfqpoint{1.775738in}{1.580463in}}%
\pgfpathlineto{\pgfqpoint{1.776481in}{1.580618in}}%
\pgfpathlineto{\pgfqpoint{1.776637in}{1.580618in}}%
\pgfpathlineto{\pgfqpoint{1.784150in}{1.581705in}}%
\pgfpathlineto{\pgfqpoint{1.785197in}{1.581953in}}%
\pgfpathlineto{\pgfqpoint{1.794766in}{1.583040in}}%
\pgfpathlineto{\pgfqpoint{1.795853in}{1.583164in}}%
\pgfpathlineto{\pgfqpoint{1.805625in}{1.584250in}}%
\pgfpathlineto{\pgfqpoint{1.806657in}{1.584405in}}%
\pgfpathlineto{\pgfqpoint{1.813091in}{1.585492in}}%
\pgfpathlineto{\pgfqpoint{1.813654in}{1.585647in}}%
\pgfpathlineto{\pgfqpoint{1.813912in}{1.585647in}}%
\pgfpathlineto{\pgfqpoint{1.821034in}{1.586734in}}%
\pgfpathlineto{\pgfqpoint{1.822003in}{1.586889in}}%
\pgfpathlineto{\pgfqpoint{1.831142in}{1.587975in}}%
\pgfpathlineto{\pgfqpoint{1.831142in}{1.588006in}}%
\pgfpathlineto{\pgfqpoint{1.831423in}{1.588006in}}%
\pgfpathlineto{\pgfqpoint{1.839999in}{1.589093in}}%
\pgfpathlineto{\pgfqpoint{1.840859in}{1.589248in}}%
\pgfpathlineto{\pgfqpoint{1.850866in}{1.590334in}}%
\pgfpathlineto{\pgfqpoint{1.851561in}{1.590428in}}%
\pgfpathlineto{\pgfqpoint{1.851827in}{1.590428in}}%
\pgfpathlineto{\pgfqpoint{1.860591in}{1.591514in}}%
\pgfpathlineto{\pgfqpoint{1.861161in}{1.591638in}}%
\pgfpathlineto{\pgfqpoint{1.861662in}{1.591638in}}%
\pgfpathlineto{\pgfqpoint{1.871231in}{1.592725in}}%
\pgfpathlineto{\pgfqpoint{1.872333in}{1.592911in}}%
\pgfpathlineto{\pgfqpoint{1.884677in}{1.593997in}}%
\pgfpathlineto{\pgfqpoint{1.885498in}{1.594153in}}%
\pgfpathlineto{\pgfqpoint{1.903064in}{1.595239in}}%
\pgfpathlineto{\pgfqpoint{1.903822in}{1.595301in}}%
\pgfpathlineto{\pgfqpoint{1.904159in}{1.595301in}}%
\pgfpathlineto{\pgfqpoint{1.917167in}{1.596388in}}%
\pgfpathlineto{\pgfqpoint{1.917323in}{1.596450in}}%
\pgfpathlineto{\pgfqpoint{1.917996in}{1.596450in}}%
\pgfpathlineto{\pgfqpoint{1.939463in}{1.597536in}}%
\pgfpathlineto{\pgfqpoint{1.940151in}{1.597629in}}%
\pgfpathlineto{\pgfqpoint{1.940526in}{1.597629in}}%
\pgfpathlineto{\pgfqpoint{1.958241in}{1.598716in}}%
\pgfpathlineto{\pgfqpoint{1.958241in}{1.598747in}}%
\pgfpathlineto{\pgfqpoint{1.959093in}{1.598747in}}%
\pgfpathlineto{\pgfqpoint{1.977058in}{1.599833in}}%
\pgfpathlineto{\pgfqpoint{1.978059in}{1.599926in}}%
\pgfpathlineto{\pgfqpoint{2.002254in}{1.601013in}}%
\pgfpathlineto{\pgfqpoint{2.002332in}{1.601075in}}%
\pgfpathlineto{\pgfqpoint{2.003364in}{1.601075in}}%
\pgfpathlineto{\pgfqpoint{2.033126in}{1.601944in}}%
\pgfpathlineto{\pgfqpoint{2.033126in}{1.601944in}}%
\pgfusepath{stroke}%
\end{pgfscope}%
\begin{pgfscope}%
\pgfsetrectcap%
\pgfsetmiterjoin%
\pgfsetlinewidth{0.803000pt}%
\definecolor{currentstroke}{rgb}{0.000000,0.000000,0.000000}%
\pgfsetstrokecolor{currentstroke}%
\pgfsetdash{}{0pt}%
\pgfpathmoveto{\pgfqpoint{0.553581in}{0.499444in}}%
\pgfpathlineto{\pgfqpoint{0.553581in}{1.654444in}}%
\pgfusepath{stroke}%
\end{pgfscope}%
\begin{pgfscope}%
\pgfsetrectcap%
\pgfsetmiterjoin%
\pgfsetlinewidth{0.803000pt}%
\definecolor{currentstroke}{rgb}{0.000000,0.000000,0.000000}%
\pgfsetstrokecolor{currentstroke}%
\pgfsetdash{}{0pt}%
\pgfpathmoveto{\pgfqpoint{2.103581in}{0.499444in}}%
\pgfpathlineto{\pgfqpoint{2.103581in}{1.654444in}}%
\pgfusepath{stroke}%
\end{pgfscope}%
\begin{pgfscope}%
\pgfsetrectcap%
\pgfsetmiterjoin%
\pgfsetlinewidth{0.803000pt}%
\definecolor{currentstroke}{rgb}{0.000000,0.000000,0.000000}%
\pgfsetstrokecolor{currentstroke}%
\pgfsetdash{}{0pt}%
\pgfpathmoveto{\pgfqpoint{0.553581in}{0.499444in}}%
\pgfpathlineto{\pgfqpoint{2.103581in}{0.499444in}}%
\pgfusepath{stroke}%
\end{pgfscope}%
\begin{pgfscope}%
\pgfsetrectcap%
\pgfsetmiterjoin%
\pgfsetlinewidth{0.803000pt}%
\definecolor{currentstroke}{rgb}{0.000000,0.000000,0.000000}%
\pgfsetstrokecolor{currentstroke}%
\pgfsetdash{}{0pt}%
\pgfpathmoveto{\pgfqpoint{0.553581in}{1.654444in}}%
\pgfpathlineto{\pgfqpoint{2.103581in}{1.654444in}}%
\pgfusepath{stroke}%
\end{pgfscope}%
\begin{pgfscope}%
\pgfsetbuttcap%
\pgfsetmiterjoin%
\definecolor{currentfill}{rgb}{1.000000,1.000000,1.000000}%
\pgfsetfillcolor{currentfill}%
\pgfsetfillopacity{0.800000}%
\pgfsetlinewidth{1.003750pt}%
\definecolor{currentstroke}{rgb}{0.800000,0.800000,0.800000}%
\pgfsetstrokecolor{currentstroke}%
\pgfsetstrokeopacity{0.800000}%
\pgfsetdash{}{0pt}%
\pgfpathmoveto{\pgfqpoint{0.832747in}{0.568889in}}%
\pgfpathlineto{\pgfqpoint{2.006358in}{0.568889in}}%
\pgfpathquadraticcurveto{\pgfqpoint{2.034136in}{0.568889in}}{\pgfqpoint{2.034136in}{0.596666in}}%
\pgfpathlineto{\pgfqpoint{2.034136in}{0.776388in}}%
\pgfpathquadraticcurveto{\pgfqpoint{2.034136in}{0.804166in}}{\pgfqpoint{2.006358in}{0.804166in}}%
\pgfpathlineto{\pgfqpoint{0.832747in}{0.804166in}}%
\pgfpathquadraticcurveto{\pgfqpoint{0.804970in}{0.804166in}}{\pgfqpoint{0.804970in}{0.776388in}}%
\pgfpathlineto{\pgfqpoint{0.804970in}{0.596666in}}%
\pgfpathquadraticcurveto{\pgfqpoint{0.804970in}{0.568889in}}{\pgfqpoint{0.832747in}{0.568889in}}%
\pgfpathlineto{\pgfqpoint{0.832747in}{0.568889in}}%
\pgfpathclose%
\pgfusepath{stroke,fill}%
\end{pgfscope}%
\begin{pgfscope}%
\pgfsetrectcap%
\pgfsetroundjoin%
\pgfsetlinewidth{1.505625pt}%
\definecolor{currentstroke}{rgb}{0.000000,0.000000,0.000000}%
\pgfsetstrokecolor{currentstroke}%
\pgfsetdash{}{0pt}%
\pgfpathmoveto{\pgfqpoint{0.860525in}{0.700000in}}%
\pgfpathlineto{\pgfqpoint{0.999414in}{0.700000in}}%
\pgfpathlineto{\pgfqpoint{1.138303in}{0.700000in}}%
\pgfusepath{stroke}%
\end{pgfscope}%
\begin{pgfscope}%
\definecolor{textcolor}{rgb}{0.000000,0.000000,0.000000}%
\pgfsetstrokecolor{textcolor}%
\pgfsetfillcolor{textcolor}%
\pgftext[x=1.249414in,y=0.651388in,left,base]{\color{textcolor}\rmfamily\fontsize{10.000000}{12.000000}\selectfont AUC=0.778}%
\end{pgfscope}%
\end{pgfpicture}%
\makeatother%
\endgroup%

\end{tabular}

\


%
\verb|AdaBoost_Hard_Tomek_0_v1_Test|

\

In this model the values are clustered very tightly, but in that small range the 214,070 samples return 210,442 different values of $p$, so there is much diversity that we can't see in this representation.  

\noindent\begin{tabular}{@{\hspace{-6pt}}p{4.3in} @{\hspace{-6pt}}p{2.0in}}
	\vskip 0pt
	\hfil Raw Model Output
	
	%% Creator: Matplotlib, PGF backend
%%
%% To include the figure in your LaTeX document, write
%%   \input{<filename>.pgf}
%%
%% Make sure the required packages are loaded in your preamble
%%   \usepackage{pgf}
%%
%% Also ensure that all the required font packages are loaded; for instance,
%% the lmodern package is sometimes necessary when using math font.
%%   \usepackage{lmodern}
%%
%% Figures using additional raster images can only be included by \input if
%% they are in the same directory as the main LaTeX file. For loading figures
%% from other directories you can use the `import` package
%%   \usepackage{import}
%%
%% and then include the figures with
%%   \import{<path to file>}{<filename>.pgf}
%%
%% Matplotlib used the following preamble
%%   
%%   \usepackage{fontspec}
%%   \makeatletter\@ifpackageloaded{underscore}{}{\usepackage[strings]{underscore}}\makeatother
%%
\begingroup%
\makeatletter%
\begin{pgfpicture}%
\pgfpathrectangle{\pgfpointorigin}{\pgfqpoint{4.578750in}{1.754444in}}%
\pgfusepath{use as bounding box, clip}%
\begin{pgfscope}%
\pgfsetbuttcap%
\pgfsetmiterjoin%
\definecolor{currentfill}{rgb}{1.000000,1.000000,1.000000}%
\pgfsetfillcolor{currentfill}%
\pgfsetlinewidth{0.000000pt}%
\definecolor{currentstroke}{rgb}{1.000000,1.000000,1.000000}%
\pgfsetstrokecolor{currentstroke}%
\pgfsetdash{}{0pt}%
\pgfpathmoveto{\pgfqpoint{0.000000in}{0.000000in}}%
\pgfpathlineto{\pgfqpoint{4.578750in}{0.000000in}}%
\pgfpathlineto{\pgfqpoint{4.578750in}{1.754444in}}%
\pgfpathlineto{\pgfqpoint{0.000000in}{1.754444in}}%
\pgfpathlineto{\pgfqpoint{0.000000in}{0.000000in}}%
\pgfpathclose%
\pgfusepath{fill}%
\end{pgfscope}%
\begin{pgfscope}%
\pgfsetbuttcap%
\pgfsetmiterjoin%
\definecolor{currentfill}{rgb}{1.000000,1.000000,1.000000}%
\pgfsetfillcolor{currentfill}%
\pgfsetlinewidth{0.000000pt}%
\definecolor{currentstroke}{rgb}{0.000000,0.000000,0.000000}%
\pgfsetstrokecolor{currentstroke}%
\pgfsetstrokeopacity{0.000000}%
\pgfsetdash{}{0pt}%
\pgfpathmoveto{\pgfqpoint{0.515000in}{0.499444in}}%
\pgfpathlineto{\pgfqpoint{4.390000in}{0.499444in}}%
\pgfpathlineto{\pgfqpoint{4.390000in}{1.654444in}}%
\pgfpathlineto{\pgfqpoint{0.515000in}{1.654444in}}%
\pgfpathlineto{\pgfqpoint{0.515000in}{0.499444in}}%
\pgfpathclose%
\pgfusepath{fill}%
\end{pgfscope}%
\begin{pgfscope}%
\pgfpathrectangle{\pgfqpoint{0.515000in}{0.499444in}}{\pgfqpoint{3.875000in}{1.155000in}}%
\pgfusepath{clip}%
\pgfsetbuttcap%
\pgfsetmiterjoin%
\pgfsetlinewidth{1.003750pt}%
\definecolor{currentstroke}{rgb}{0.000000,0.000000,0.000000}%
\pgfsetstrokecolor{currentstroke}%
\pgfsetdash{}{0pt}%
\pgfpathmoveto{\pgfqpoint{0.505000in}{0.499444in}}%
\pgfpathlineto{\pgfqpoint{0.553367in}{0.499444in}}%
\pgfpathlineto{\pgfqpoint{0.553367in}{0.499444in}}%
\pgfpathlineto{\pgfqpoint{0.505000in}{0.499444in}}%
\pgfusepath{stroke}%
\end{pgfscope}%
\begin{pgfscope}%
\pgfpathrectangle{\pgfqpoint{0.515000in}{0.499444in}}{\pgfqpoint{3.875000in}{1.155000in}}%
\pgfusepath{clip}%
\pgfsetbuttcap%
\pgfsetmiterjoin%
\pgfsetlinewidth{1.003750pt}%
\definecolor{currentstroke}{rgb}{0.000000,0.000000,0.000000}%
\pgfsetstrokecolor{currentstroke}%
\pgfsetdash{}{0pt}%
\pgfpathmoveto{\pgfqpoint{0.645446in}{0.499444in}}%
\pgfpathlineto{\pgfqpoint{0.706832in}{0.499444in}}%
\pgfpathlineto{\pgfqpoint{0.706832in}{0.499444in}}%
\pgfpathlineto{\pgfqpoint{0.645446in}{0.499444in}}%
\pgfpathlineto{\pgfqpoint{0.645446in}{0.499444in}}%
\pgfpathclose%
\pgfusepath{stroke}%
\end{pgfscope}%
\begin{pgfscope}%
\pgfpathrectangle{\pgfqpoint{0.515000in}{0.499444in}}{\pgfqpoint{3.875000in}{1.155000in}}%
\pgfusepath{clip}%
\pgfsetbuttcap%
\pgfsetmiterjoin%
\pgfsetlinewidth{1.003750pt}%
\definecolor{currentstroke}{rgb}{0.000000,0.000000,0.000000}%
\pgfsetstrokecolor{currentstroke}%
\pgfsetdash{}{0pt}%
\pgfpathmoveto{\pgfqpoint{0.798911in}{0.499444in}}%
\pgfpathlineto{\pgfqpoint{0.860297in}{0.499444in}}%
\pgfpathlineto{\pgfqpoint{0.860297in}{0.499444in}}%
\pgfpathlineto{\pgfqpoint{0.798911in}{0.499444in}}%
\pgfpathlineto{\pgfqpoint{0.798911in}{0.499444in}}%
\pgfpathclose%
\pgfusepath{stroke}%
\end{pgfscope}%
\begin{pgfscope}%
\pgfpathrectangle{\pgfqpoint{0.515000in}{0.499444in}}{\pgfqpoint{3.875000in}{1.155000in}}%
\pgfusepath{clip}%
\pgfsetbuttcap%
\pgfsetmiterjoin%
\pgfsetlinewidth{1.003750pt}%
\definecolor{currentstroke}{rgb}{0.000000,0.000000,0.000000}%
\pgfsetstrokecolor{currentstroke}%
\pgfsetdash{}{0pt}%
\pgfpathmoveto{\pgfqpoint{0.952377in}{0.499444in}}%
\pgfpathlineto{\pgfqpoint{1.013763in}{0.499444in}}%
\pgfpathlineto{\pgfqpoint{1.013763in}{0.499444in}}%
\pgfpathlineto{\pgfqpoint{0.952377in}{0.499444in}}%
\pgfpathlineto{\pgfqpoint{0.952377in}{0.499444in}}%
\pgfpathclose%
\pgfusepath{stroke}%
\end{pgfscope}%
\begin{pgfscope}%
\pgfpathrectangle{\pgfqpoint{0.515000in}{0.499444in}}{\pgfqpoint{3.875000in}{1.155000in}}%
\pgfusepath{clip}%
\pgfsetbuttcap%
\pgfsetmiterjoin%
\pgfsetlinewidth{1.003750pt}%
\definecolor{currentstroke}{rgb}{0.000000,0.000000,0.000000}%
\pgfsetstrokecolor{currentstroke}%
\pgfsetdash{}{0pt}%
\pgfpathmoveto{\pgfqpoint{1.105842in}{0.499444in}}%
\pgfpathlineto{\pgfqpoint{1.167228in}{0.499444in}}%
\pgfpathlineto{\pgfqpoint{1.167228in}{0.499444in}}%
\pgfpathlineto{\pgfqpoint{1.105842in}{0.499444in}}%
\pgfpathlineto{\pgfqpoint{1.105842in}{0.499444in}}%
\pgfpathclose%
\pgfusepath{stroke}%
\end{pgfscope}%
\begin{pgfscope}%
\pgfpathrectangle{\pgfqpoint{0.515000in}{0.499444in}}{\pgfqpoint{3.875000in}{1.155000in}}%
\pgfusepath{clip}%
\pgfsetbuttcap%
\pgfsetmiterjoin%
\pgfsetlinewidth{1.003750pt}%
\definecolor{currentstroke}{rgb}{0.000000,0.000000,0.000000}%
\pgfsetstrokecolor{currentstroke}%
\pgfsetdash{}{0pt}%
\pgfpathmoveto{\pgfqpoint{1.259307in}{0.499444in}}%
\pgfpathlineto{\pgfqpoint{1.320693in}{0.499444in}}%
\pgfpathlineto{\pgfqpoint{1.320693in}{0.499444in}}%
\pgfpathlineto{\pgfqpoint{1.259307in}{0.499444in}}%
\pgfpathlineto{\pgfqpoint{1.259307in}{0.499444in}}%
\pgfpathclose%
\pgfusepath{stroke}%
\end{pgfscope}%
\begin{pgfscope}%
\pgfpathrectangle{\pgfqpoint{0.515000in}{0.499444in}}{\pgfqpoint{3.875000in}{1.155000in}}%
\pgfusepath{clip}%
\pgfsetbuttcap%
\pgfsetmiterjoin%
\pgfsetlinewidth{1.003750pt}%
\definecolor{currentstroke}{rgb}{0.000000,0.000000,0.000000}%
\pgfsetstrokecolor{currentstroke}%
\pgfsetdash{}{0pt}%
\pgfpathmoveto{\pgfqpoint{1.412773in}{0.499444in}}%
\pgfpathlineto{\pgfqpoint{1.474159in}{0.499444in}}%
\pgfpathlineto{\pgfqpoint{1.474159in}{0.499444in}}%
\pgfpathlineto{\pgfqpoint{1.412773in}{0.499444in}}%
\pgfpathlineto{\pgfqpoint{1.412773in}{0.499444in}}%
\pgfpathclose%
\pgfusepath{stroke}%
\end{pgfscope}%
\begin{pgfscope}%
\pgfpathrectangle{\pgfqpoint{0.515000in}{0.499444in}}{\pgfqpoint{3.875000in}{1.155000in}}%
\pgfusepath{clip}%
\pgfsetbuttcap%
\pgfsetmiterjoin%
\pgfsetlinewidth{1.003750pt}%
\definecolor{currentstroke}{rgb}{0.000000,0.000000,0.000000}%
\pgfsetstrokecolor{currentstroke}%
\pgfsetdash{}{0pt}%
\pgfpathmoveto{\pgfqpoint{1.566238in}{0.499444in}}%
\pgfpathlineto{\pgfqpoint{1.627624in}{0.499444in}}%
\pgfpathlineto{\pgfqpoint{1.627624in}{0.499444in}}%
\pgfpathlineto{\pgfqpoint{1.566238in}{0.499444in}}%
\pgfpathlineto{\pgfqpoint{1.566238in}{0.499444in}}%
\pgfpathclose%
\pgfusepath{stroke}%
\end{pgfscope}%
\begin{pgfscope}%
\pgfpathrectangle{\pgfqpoint{0.515000in}{0.499444in}}{\pgfqpoint{3.875000in}{1.155000in}}%
\pgfusepath{clip}%
\pgfsetbuttcap%
\pgfsetmiterjoin%
\pgfsetlinewidth{1.003750pt}%
\definecolor{currentstroke}{rgb}{0.000000,0.000000,0.000000}%
\pgfsetstrokecolor{currentstroke}%
\pgfsetdash{}{0pt}%
\pgfpathmoveto{\pgfqpoint{1.719703in}{0.499444in}}%
\pgfpathlineto{\pgfqpoint{1.781089in}{0.499444in}}%
\pgfpathlineto{\pgfqpoint{1.781089in}{0.499444in}}%
\pgfpathlineto{\pgfqpoint{1.719703in}{0.499444in}}%
\pgfpathlineto{\pgfqpoint{1.719703in}{0.499444in}}%
\pgfpathclose%
\pgfusepath{stroke}%
\end{pgfscope}%
\begin{pgfscope}%
\pgfpathrectangle{\pgfqpoint{0.515000in}{0.499444in}}{\pgfqpoint{3.875000in}{1.155000in}}%
\pgfusepath{clip}%
\pgfsetbuttcap%
\pgfsetmiterjoin%
\pgfsetlinewidth{1.003750pt}%
\definecolor{currentstroke}{rgb}{0.000000,0.000000,0.000000}%
\pgfsetstrokecolor{currentstroke}%
\pgfsetdash{}{0pt}%
\pgfpathmoveto{\pgfqpoint{1.873169in}{0.499444in}}%
\pgfpathlineto{\pgfqpoint{1.934555in}{0.499444in}}%
\pgfpathlineto{\pgfqpoint{1.934555in}{0.499444in}}%
\pgfpathlineto{\pgfqpoint{1.873169in}{0.499444in}}%
\pgfpathlineto{\pgfqpoint{1.873169in}{0.499444in}}%
\pgfpathclose%
\pgfusepath{stroke}%
\end{pgfscope}%
\begin{pgfscope}%
\pgfpathrectangle{\pgfqpoint{0.515000in}{0.499444in}}{\pgfqpoint{3.875000in}{1.155000in}}%
\pgfusepath{clip}%
\pgfsetbuttcap%
\pgfsetmiterjoin%
\pgfsetlinewidth{1.003750pt}%
\definecolor{currentstroke}{rgb}{0.000000,0.000000,0.000000}%
\pgfsetstrokecolor{currentstroke}%
\pgfsetdash{}{0pt}%
\pgfpathmoveto{\pgfqpoint{2.026634in}{0.499444in}}%
\pgfpathlineto{\pgfqpoint{2.088020in}{0.499444in}}%
\pgfpathlineto{\pgfqpoint{2.088020in}{0.499444in}}%
\pgfpathlineto{\pgfqpoint{2.026634in}{0.499444in}}%
\pgfpathlineto{\pgfqpoint{2.026634in}{0.499444in}}%
\pgfpathclose%
\pgfusepath{stroke}%
\end{pgfscope}%
\begin{pgfscope}%
\pgfpathrectangle{\pgfqpoint{0.515000in}{0.499444in}}{\pgfqpoint{3.875000in}{1.155000in}}%
\pgfusepath{clip}%
\pgfsetbuttcap%
\pgfsetmiterjoin%
\pgfsetlinewidth{1.003750pt}%
\definecolor{currentstroke}{rgb}{0.000000,0.000000,0.000000}%
\pgfsetstrokecolor{currentstroke}%
\pgfsetdash{}{0pt}%
\pgfpathmoveto{\pgfqpoint{2.180099in}{0.499444in}}%
\pgfpathlineto{\pgfqpoint{2.241485in}{0.499444in}}%
\pgfpathlineto{\pgfqpoint{2.241485in}{0.499444in}}%
\pgfpathlineto{\pgfqpoint{2.180099in}{0.499444in}}%
\pgfpathlineto{\pgfqpoint{2.180099in}{0.499444in}}%
\pgfpathclose%
\pgfusepath{stroke}%
\end{pgfscope}%
\begin{pgfscope}%
\pgfpathrectangle{\pgfqpoint{0.515000in}{0.499444in}}{\pgfqpoint{3.875000in}{1.155000in}}%
\pgfusepath{clip}%
\pgfsetbuttcap%
\pgfsetmiterjoin%
\pgfsetlinewidth{1.003750pt}%
\definecolor{currentstroke}{rgb}{0.000000,0.000000,0.000000}%
\pgfsetstrokecolor{currentstroke}%
\pgfsetdash{}{0pt}%
\pgfpathmoveto{\pgfqpoint{2.333565in}{0.499444in}}%
\pgfpathlineto{\pgfqpoint{2.394951in}{0.499444in}}%
\pgfpathlineto{\pgfqpoint{2.394951in}{1.599444in}}%
\pgfpathlineto{\pgfqpoint{2.333565in}{1.599444in}}%
\pgfpathlineto{\pgfqpoint{2.333565in}{0.499444in}}%
\pgfpathclose%
\pgfusepath{stroke}%
\end{pgfscope}%
\begin{pgfscope}%
\pgfpathrectangle{\pgfqpoint{0.515000in}{0.499444in}}{\pgfqpoint{3.875000in}{1.155000in}}%
\pgfusepath{clip}%
\pgfsetbuttcap%
\pgfsetmiterjoin%
\pgfsetlinewidth{1.003750pt}%
\definecolor{currentstroke}{rgb}{0.000000,0.000000,0.000000}%
\pgfsetstrokecolor{currentstroke}%
\pgfsetdash{}{0pt}%
\pgfpathmoveto{\pgfqpoint{2.487030in}{0.499444in}}%
\pgfpathlineto{\pgfqpoint{2.548416in}{0.499444in}}%
\pgfpathlineto{\pgfqpoint{2.548416in}{0.499444in}}%
\pgfpathlineto{\pgfqpoint{2.487030in}{0.499444in}}%
\pgfpathlineto{\pgfqpoint{2.487030in}{0.499444in}}%
\pgfpathclose%
\pgfusepath{stroke}%
\end{pgfscope}%
\begin{pgfscope}%
\pgfpathrectangle{\pgfqpoint{0.515000in}{0.499444in}}{\pgfqpoint{3.875000in}{1.155000in}}%
\pgfusepath{clip}%
\pgfsetbuttcap%
\pgfsetmiterjoin%
\pgfsetlinewidth{1.003750pt}%
\definecolor{currentstroke}{rgb}{0.000000,0.000000,0.000000}%
\pgfsetstrokecolor{currentstroke}%
\pgfsetdash{}{0pt}%
\pgfpathmoveto{\pgfqpoint{2.640495in}{0.499444in}}%
\pgfpathlineto{\pgfqpoint{2.701881in}{0.499444in}}%
\pgfpathlineto{\pgfqpoint{2.701881in}{0.499444in}}%
\pgfpathlineto{\pgfqpoint{2.640495in}{0.499444in}}%
\pgfpathlineto{\pgfqpoint{2.640495in}{0.499444in}}%
\pgfpathclose%
\pgfusepath{stroke}%
\end{pgfscope}%
\begin{pgfscope}%
\pgfpathrectangle{\pgfqpoint{0.515000in}{0.499444in}}{\pgfqpoint{3.875000in}{1.155000in}}%
\pgfusepath{clip}%
\pgfsetbuttcap%
\pgfsetmiterjoin%
\pgfsetlinewidth{1.003750pt}%
\definecolor{currentstroke}{rgb}{0.000000,0.000000,0.000000}%
\pgfsetstrokecolor{currentstroke}%
\pgfsetdash{}{0pt}%
\pgfpathmoveto{\pgfqpoint{2.793961in}{0.499444in}}%
\pgfpathlineto{\pgfqpoint{2.855347in}{0.499444in}}%
\pgfpathlineto{\pgfqpoint{2.855347in}{0.499444in}}%
\pgfpathlineto{\pgfqpoint{2.793961in}{0.499444in}}%
\pgfpathlineto{\pgfqpoint{2.793961in}{0.499444in}}%
\pgfpathclose%
\pgfusepath{stroke}%
\end{pgfscope}%
\begin{pgfscope}%
\pgfpathrectangle{\pgfqpoint{0.515000in}{0.499444in}}{\pgfqpoint{3.875000in}{1.155000in}}%
\pgfusepath{clip}%
\pgfsetbuttcap%
\pgfsetmiterjoin%
\pgfsetlinewidth{1.003750pt}%
\definecolor{currentstroke}{rgb}{0.000000,0.000000,0.000000}%
\pgfsetstrokecolor{currentstroke}%
\pgfsetdash{}{0pt}%
\pgfpathmoveto{\pgfqpoint{2.947426in}{0.499444in}}%
\pgfpathlineto{\pgfqpoint{3.008812in}{0.499444in}}%
\pgfpathlineto{\pgfqpoint{3.008812in}{0.499444in}}%
\pgfpathlineto{\pgfqpoint{2.947426in}{0.499444in}}%
\pgfpathlineto{\pgfqpoint{2.947426in}{0.499444in}}%
\pgfpathclose%
\pgfusepath{stroke}%
\end{pgfscope}%
\begin{pgfscope}%
\pgfpathrectangle{\pgfqpoint{0.515000in}{0.499444in}}{\pgfqpoint{3.875000in}{1.155000in}}%
\pgfusepath{clip}%
\pgfsetbuttcap%
\pgfsetmiterjoin%
\pgfsetlinewidth{1.003750pt}%
\definecolor{currentstroke}{rgb}{0.000000,0.000000,0.000000}%
\pgfsetstrokecolor{currentstroke}%
\pgfsetdash{}{0pt}%
\pgfpathmoveto{\pgfqpoint{3.100891in}{0.499444in}}%
\pgfpathlineto{\pgfqpoint{3.162278in}{0.499444in}}%
\pgfpathlineto{\pgfqpoint{3.162278in}{0.499444in}}%
\pgfpathlineto{\pgfqpoint{3.100891in}{0.499444in}}%
\pgfpathlineto{\pgfqpoint{3.100891in}{0.499444in}}%
\pgfpathclose%
\pgfusepath{stroke}%
\end{pgfscope}%
\begin{pgfscope}%
\pgfpathrectangle{\pgfqpoint{0.515000in}{0.499444in}}{\pgfqpoint{3.875000in}{1.155000in}}%
\pgfusepath{clip}%
\pgfsetbuttcap%
\pgfsetmiterjoin%
\pgfsetlinewidth{1.003750pt}%
\definecolor{currentstroke}{rgb}{0.000000,0.000000,0.000000}%
\pgfsetstrokecolor{currentstroke}%
\pgfsetdash{}{0pt}%
\pgfpathmoveto{\pgfqpoint{3.254357in}{0.499444in}}%
\pgfpathlineto{\pgfqpoint{3.315743in}{0.499444in}}%
\pgfpathlineto{\pgfqpoint{3.315743in}{0.499444in}}%
\pgfpathlineto{\pgfqpoint{3.254357in}{0.499444in}}%
\pgfpathlineto{\pgfqpoint{3.254357in}{0.499444in}}%
\pgfpathclose%
\pgfusepath{stroke}%
\end{pgfscope}%
\begin{pgfscope}%
\pgfpathrectangle{\pgfqpoint{0.515000in}{0.499444in}}{\pgfqpoint{3.875000in}{1.155000in}}%
\pgfusepath{clip}%
\pgfsetbuttcap%
\pgfsetmiterjoin%
\pgfsetlinewidth{1.003750pt}%
\definecolor{currentstroke}{rgb}{0.000000,0.000000,0.000000}%
\pgfsetstrokecolor{currentstroke}%
\pgfsetdash{}{0pt}%
\pgfpathmoveto{\pgfqpoint{3.407822in}{0.499444in}}%
\pgfpathlineto{\pgfqpoint{3.469208in}{0.499444in}}%
\pgfpathlineto{\pgfqpoint{3.469208in}{0.499444in}}%
\pgfpathlineto{\pgfqpoint{3.407822in}{0.499444in}}%
\pgfpathlineto{\pgfqpoint{3.407822in}{0.499444in}}%
\pgfpathclose%
\pgfusepath{stroke}%
\end{pgfscope}%
\begin{pgfscope}%
\pgfpathrectangle{\pgfqpoint{0.515000in}{0.499444in}}{\pgfqpoint{3.875000in}{1.155000in}}%
\pgfusepath{clip}%
\pgfsetbuttcap%
\pgfsetmiterjoin%
\pgfsetlinewidth{1.003750pt}%
\definecolor{currentstroke}{rgb}{0.000000,0.000000,0.000000}%
\pgfsetstrokecolor{currentstroke}%
\pgfsetdash{}{0pt}%
\pgfpathmoveto{\pgfqpoint{3.561287in}{0.499444in}}%
\pgfpathlineto{\pgfqpoint{3.622674in}{0.499444in}}%
\pgfpathlineto{\pgfqpoint{3.622674in}{0.499444in}}%
\pgfpathlineto{\pgfqpoint{3.561287in}{0.499444in}}%
\pgfpathlineto{\pgfqpoint{3.561287in}{0.499444in}}%
\pgfpathclose%
\pgfusepath{stroke}%
\end{pgfscope}%
\begin{pgfscope}%
\pgfpathrectangle{\pgfqpoint{0.515000in}{0.499444in}}{\pgfqpoint{3.875000in}{1.155000in}}%
\pgfusepath{clip}%
\pgfsetbuttcap%
\pgfsetmiterjoin%
\pgfsetlinewidth{1.003750pt}%
\definecolor{currentstroke}{rgb}{0.000000,0.000000,0.000000}%
\pgfsetstrokecolor{currentstroke}%
\pgfsetdash{}{0pt}%
\pgfpathmoveto{\pgfqpoint{3.714753in}{0.499444in}}%
\pgfpathlineto{\pgfqpoint{3.776139in}{0.499444in}}%
\pgfpathlineto{\pgfqpoint{3.776139in}{0.499444in}}%
\pgfpathlineto{\pgfqpoint{3.714753in}{0.499444in}}%
\pgfpathlineto{\pgfqpoint{3.714753in}{0.499444in}}%
\pgfpathclose%
\pgfusepath{stroke}%
\end{pgfscope}%
\begin{pgfscope}%
\pgfpathrectangle{\pgfqpoint{0.515000in}{0.499444in}}{\pgfqpoint{3.875000in}{1.155000in}}%
\pgfusepath{clip}%
\pgfsetbuttcap%
\pgfsetmiterjoin%
\pgfsetlinewidth{1.003750pt}%
\definecolor{currentstroke}{rgb}{0.000000,0.000000,0.000000}%
\pgfsetstrokecolor{currentstroke}%
\pgfsetdash{}{0pt}%
\pgfpathmoveto{\pgfqpoint{3.868218in}{0.499444in}}%
\pgfpathlineto{\pgfqpoint{3.929604in}{0.499444in}}%
\pgfpathlineto{\pgfqpoint{3.929604in}{0.499444in}}%
\pgfpathlineto{\pgfqpoint{3.868218in}{0.499444in}}%
\pgfpathlineto{\pgfqpoint{3.868218in}{0.499444in}}%
\pgfpathclose%
\pgfusepath{stroke}%
\end{pgfscope}%
\begin{pgfscope}%
\pgfpathrectangle{\pgfqpoint{0.515000in}{0.499444in}}{\pgfqpoint{3.875000in}{1.155000in}}%
\pgfusepath{clip}%
\pgfsetbuttcap%
\pgfsetmiterjoin%
\pgfsetlinewidth{1.003750pt}%
\definecolor{currentstroke}{rgb}{0.000000,0.000000,0.000000}%
\pgfsetstrokecolor{currentstroke}%
\pgfsetdash{}{0pt}%
\pgfpathmoveto{\pgfqpoint{4.021683in}{0.499444in}}%
\pgfpathlineto{\pgfqpoint{4.083070in}{0.499444in}}%
\pgfpathlineto{\pgfqpoint{4.083070in}{0.499444in}}%
\pgfpathlineto{\pgfqpoint{4.021683in}{0.499444in}}%
\pgfpathlineto{\pgfqpoint{4.021683in}{0.499444in}}%
\pgfpathclose%
\pgfusepath{stroke}%
\end{pgfscope}%
\begin{pgfscope}%
\pgfpathrectangle{\pgfqpoint{0.515000in}{0.499444in}}{\pgfqpoint{3.875000in}{1.155000in}}%
\pgfusepath{clip}%
\pgfsetbuttcap%
\pgfsetmiterjoin%
\pgfsetlinewidth{1.003750pt}%
\definecolor{currentstroke}{rgb}{0.000000,0.000000,0.000000}%
\pgfsetstrokecolor{currentstroke}%
\pgfsetdash{}{0pt}%
\pgfpathmoveto{\pgfqpoint{4.175149in}{0.499444in}}%
\pgfpathlineto{\pgfqpoint{4.236535in}{0.499444in}}%
\pgfpathlineto{\pgfqpoint{4.236535in}{0.499444in}}%
\pgfpathlineto{\pgfqpoint{4.175149in}{0.499444in}}%
\pgfpathlineto{\pgfqpoint{4.175149in}{0.499444in}}%
\pgfpathclose%
\pgfusepath{stroke}%
\end{pgfscope}%
\begin{pgfscope}%
\pgfpathrectangle{\pgfqpoint{0.515000in}{0.499444in}}{\pgfqpoint{3.875000in}{1.155000in}}%
\pgfusepath{clip}%
\pgfsetbuttcap%
\pgfsetmiterjoin%
\definecolor{currentfill}{rgb}{0.000000,0.000000,0.000000}%
\pgfsetfillcolor{currentfill}%
\pgfsetlinewidth{0.000000pt}%
\definecolor{currentstroke}{rgb}{0.000000,0.000000,0.000000}%
\pgfsetstrokecolor{currentstroke}%
\pgfsetstrokeopacity{0.000000}%
\pgfsetdash{}{0pt}%
\pgfpathmoveto{\pgfqpoint{0.553367in}{0.499444in}}%
\pgfpathlineto{\pgfqpoint{0.614753in}{0.499444in}}%
\pgfpathlineto{\pgfqpoint{0.614753in}{0.499444in}}%
\pgfpathlineto{\pgfqpoint{0.553367in}{0.499444in}}%
\pgfpathlineto{\pgfqpoint{0.553367in}{0.499444in}}%
\pgfpathclose%
\pgfusepath{fill}%
\end{pgfscope}%
\begin{pgfscope}%
\pgfpathrectangle{\pgfqpoint{0.515000in}{0.499444in}}{\pgfqpoint{3.875000in}{1.155000in}}%
\pgfusepath{clip}%
\pgfsetbuttcap%
\pgfsetmiterjoin%
\definecolor{currentfill}{rgb}{0.000000,0.000000,0.000000}%
\pgfsetfillcolor{currentfill}%
\pgfsetlinewidth{0.000000pt}%
\definecolor{currentstroke}{rgb}{0.000000,0.000000,0.000000}%
\pgfsetstrokecolor{currentstroke}%
\pgfsetstrokeopacity{0.000000}%
\pgfsetdash{}{0pt}%
\pgfpathmoveto{\pgfqpoint{0.706832in}{0.499444in}}%
\pgfpathlineto{\pgfqpoint{0.768218in}{0.499444in}}%
\pgfpathlineto{\pgfqpoint{0.768218in}{0.499444in}}%
\pgfpathlineto{\pgfqpoint{0.706832in}{0.499444in}}%
\pgfpathlineto{\pgfqpoint{0.706832in}{0.499444in}}%
\pgfpathclose%
\pgfusepath{fill}%
\end{pgfscope}%
\begin{pgfscope}%
\pgfpathrectangle{\pgfqpoint{0.515000in}{0.499444in}}{\pgfqpoint{3.875000in}{1.155000in}}%
\pgfusepath{clip}%
\pgfsetbuttcap%
\pgfsetmiterjoin%
\definecolor{currentfill}{rgb}{0.000000,0.000000,0.000000}%
\pgfsetfillcolor{currentfill}%
\pgfsetlinewidth{0.000000pt}%
\definecolor{currentstroke}{rgb}{0.000000,0.000000,0.000000}%
\pgfsetstrokecolor{currentstroke}%
\pgfsetstrokeopacity{0.000000}%
\pgfsetdash{}{0pt}%
\pgfpathmoveto{\pgfqpoint{0.860297in}{0.499444in}}%
\pgfpathlineto{\pgfqpoint{0.921683in}{0.499444in}}%
\pgfpathlineto{\pgfqpoint{0.921683in}{0.499444in}}%
\pgfpathlineto{\pgfqpoint{0.860297in}{0.499444in}}%
\pgfpathlineto{\pgfqpoint{0.860297in}{0.499444in}}%
\pgfpathclose%
\pgfusepath{fill}%
\end{pgfscope}%
\begin{pgfscope}%
\pgfpathrectangle{\pgfqpoint{0.515000in}{0.499444in}}{\pgfqpoint{3.875000in}{1.155000in}}%
\pgfusepath{clip}%
\pgfsetbuttcap%
\pgfsetmiterjoin%
\definecolor{currentfill}{rgb}{0.000000,0.000000,0.000000}%
\pgfsetfillcolor{currentfill}%
\pgfsetlinewidth{0.000000pt}%
\definecolor{currentstroke}{rgb}{0.000000,0.000000,0.000000}%
\pgfsetstrokecolor{currentstroke}%
\pgfsetstrokeopacity{0.000000}%
\pgfsetdash{}{0pt}%
\pgfpathmoveto{\pgfqpoint{1.013763in}{0.499444in}}%
\pgfpathlineto{\pgfqpoint{1.075149in}{0.499444in}}%
\pgfpathlineto{\pgfqpoint{1.075149in}{0.499444in}}%
\pgfpathlineto{\pgfqpoint{1.013763in}{0.499444in}}%
\pgfpathlineto{\pgfqpoint{1.013763in}{0.499444in}}%
\pgfpathclose%
\pgfusepath{fill}%
\end{pgfscope}%
\begin{pgfscope}%
\pgfpathrectangle{\pgfqpoint{0.515000in}{0.499444in}}{\pgfqpoint{3.875000in}{1.155000in}}%
\pgfusepath{clip}%
\pgfsetbuttcap%
\pgfsetmiterjoin%
\definecolor{currentfill}{rgb}{0.000000,0.000000,0.000000}%
\pgfsetfillcolor{currentfill}%
\pgfsetlinewidth{0.000000pt}%
\definecolor{currentstroke}{rgb}{0.000000,0.000000,0.000000}%
\pgfsetstrokecolor{currentstroke}%
\pgfsetstrokeopacity{0.000000}%
\pgfsetdash{}{0pt}%
\pgfpathmoveto{\pgfqpoint{1.167228in}{0.499444in}}%
\pgfpathlineto{\pgfqpoint{1.228614in}{0.499444in}}%
\pgfpathlineto{\pgfqpoint{1.228614in}{0.499444in}}%
\pgfpathlineto{\pgfqpoint{1.167228in}{0.499444in}}%
\pgfpathlineto{\pgfqpoint{1.167228in}{0.499444in}}%
\pgfpathclose%
\pgfusepath{fill}%
\end{pgfscope}%
\begin{pgfscope}%
\pgfpathrectangle{\pgfqpoint{0.515000in}{0.499444in}}{\pgfqpoint{3.875000in}{1.155000in}}%
\pgfusepath{clip}%
\pgfsetbuttcap%
\pgfsetmiterjoin%
\definecolor{currentfill}{rgb}{0.000000,0.000000,0.000000}%
\pgfsetfillcolor{currentfill}%
\pgfsetlinewidth{0.000000pt}%
\definecolor{currentstroke}{rgb}{0.000000,0.000000,0.000000}%
\pgfsetstrokecolor{currentstroke}%
\pgfsetstrokeopacity{0.000000}%
\pgfsetdash{}{0pt}%
\pgfpathmoveto{\pgfqpoint{1.320693in}{0.499444in}}%
\pgfpathlineto{\pgfqpoint{1.382079in}{0.499444in}}%
\pgfpathlineto{\pgfqpoint{1.382079in}{0.499444in}}%
\pgfpathlineto{\pgfqpoint{1.320693in}{0.499444in}}%
\pgfpathlineto{\pgfqpoint{1.320693in}{0.499444in}}%
\pgfpathclose%
\pgfusepath{fill}%
\end{pgfscope}%
\begin{pgfscope}%
\pgfpathrectangle{\pgfqpoint{0.515000in}{0.499444in}}{\pgfqpoint{3.875000in}{1.155000in}}%
\pgfusepath{clip}%
\pgfsetbuttcap%
\pgfsetmiterjoin%
\definecolor{currentfill}{rgb}{0.000000,0.000000,0.000000}%
\pgfsetfillcolor{currentfill}%
\pgfsetlinewidth{0.000000pt}%
\definecolor{currentstroke}{rgb}{0.000000,0.000000,0.000000}%
\pgfsetstrokecolor{currentstroke}%
\pgfsetstrokeopacity{0.000000}%
\pgfsetdash{}{0pt}%
\pgfpathmoveto{\pgfqpoint{1.474159in}{0.499444in}}%
\pgfpathlineto{\pgfqpoint{1.535545in}{0.499444in}}%
\pgfpathlineto{\pgfqpoint{1.535545in}{0.499444in}}%
\pgfpathlineto{\pgfqpoint{1.474159in}{0.499444in}}%
\pgfpathlineto{\pgfqpoint{1.474159in}{0.499444in}}%
\pgfpathclose%
\pgfusepath{fill}%
\end{pgfscope}%
\begin{pgfscope}%
\pgfpathrectangle{\pgfqpoint{0.515000in}{0.499444in}}{\pgfqpoint{3.875000in}{1.155000in}}%
\pgfusepath{clip}%
\pgfsetbuttcap%
\pgfsetmiterjoin%
\definecolor{currentfill}{rgb}{0.000000,0.000000,0.000000}%
\pgfsetfillcolor{currentfill}%
\pgfsetlinewidth{0.000000pt}%
\definecolor{currentstroke}{rgb}{0.000000,0.000000,0.000000}%
\pgfsetstrokecolor{currentstroke}%
\pgfsetstrokeopacity{0.000000}%
\pgfsetdash{}{0pt}%
\pgfpathmoveto{\pgfqpoint{1.627624in}{0.499444in}}%
\pgfpathlineto{\pgfqpoint{1.689010in}{0.499444in}}%
\pgfpathlineto{\pgfqpoint{1.689010in}{0.499444in}}%
\pgfpathlineto{\pgfqpoint{1.627624in}{0.499444in}}%
\pgfpathlineto{\pgfqpoint{1.627624in}{0.499444in}}%
\pgfpathclose%
\pgfusepath{fill}%
\end{pgfscope}%
\begin{pgfscope}%
\pgfpathrectangle{\pgfqpoint{0.515000in}{0.499444in}}{\pgfqpoint{3.875000in}{1.155000in}}%
\pgfusepath{clip}%
\pgfsetbuttcap%
\pgfsetmiterjoin%
\definecolor{currentfill}{rgb}{0.000000,0.000000,0.000000}%
\pgfsetfillcolor{currentfill}%
\pgfsetlinewidth{0.000000pt}%
\definecolor{currentstroke}{rgb}{0.000000,0.000000,0.000000}%
\pgfsetstrokecolor{currentstroke}%
\pgfsetstrokeopacity{0.000000}%
\pgfsetdash{}{0pt}%
\pgfpathmoveto{\pgfqpoint{1.781089in}{0.499444in}}%
\pgfpathlineto{\pgfqpoint{1.842476in}{0.499444in}}%
\pgfpathlineto{\pgfqpoint{1.842476in}{0.499444in}}%
\pgfpathlineto{\pgfqpoint{1.781089in}{0.499444in}}%
\pgfpathlineto{\pgfqpoint{1.781089in}{0.499444in}}%
\pgfpathclose%
\pgfusepath{fill}%
\end{pgfscope}%
\begin{pgfscope}%
\pgfpathrectangle{\pgfqpoint{0.515000in}{0.499444in}}{\pgfqpoint{3.875000in}{1.155000in}}%
\pgfusepath{clip}%
\pgfsetbuttcap%
\pgfsetmiterjoin%
\definecolor{currentfill}{rgb}{0.000000,0.000000,0.000000}%
\pgfsetfillcolor{currentfill}%
\pgfsetlinewidth{0.000000pt}%
\definecolor{currentstroke}{rgb}{0.000000,0.000000,0.000000}%
\pgfsetstrokecolor{currentstroke}%
\pgfsetstrokeopacity{0.000000}%
\pgfsetdash{}{0pt}%
\pgfpathmoveto{\pgfqpoint{1.934555in}{0.499444in}}%
\pgfpathlineto{\pgfqpoint{1.995941in}{0.499444in}}%
\pgfpathlineto{\pgfqpoint{1.995941in}{0.499444in}}%
\pgfpathlineto{\pgfqpoint{1.934555in}{0.499444in}}%
\pgfpathlineto{\pgfqpoint{1.934555in}{0.499444in}}%
\pgfpathclose%
\pgfusepath{fill}%
\end{pgfscope}%
\begin{pgfscope}%
\pgfpathrectangle{\pgfqpoint{0.515000in}{0.499444in}}{\pgfqpoint{3.875000in}{1.155000in}}%
\pgfusepath{clip}%
\pgfsetbuttcap%
\pgfsetmiterjoin%
\definecolor{currentfill}{rgb}{0.000000,0.000000,0.000000}%
\pgfsetfillcolor{currentfill}%
\pgfsetlinewidth{0.000000pt}%
\definecolor{currentstroke}{rgb}{0.000000,0.000000,0.000000}%
\pgfsetstrokecolor{currentstroke}%
\pgfsetstrokeopacity{0.000000}%
\pgfsetdash{}{0pt}%
\pgfpathmoveto{\pgfqpoint{2.088020in}{0.499444in}}%
\pgfpathlineto{\pgfqpoint{2.149406in}{0.499444in}}%
\pgfpathlineto{\pgfqpoint{2.149406in}{0.499444in}}%
\pgfpathlineto{\pgfqpoint{2.088020in}{0.499444in}}%
\pgfpathlineto{\pgfqpoint{2.088020in}{0.499444in}}%
\pgfpathclose%
\pgfusepath{fill}%
\end{pgfscope}%
\begin{pgfscope}%
\pgfpathrectangle{\pgfqpoint{0.515000in}{0.499444in}}{\pgfqpoint{3.875000in}{1.155000in}}%
\pgfusepath{clip}%
\pgfsetbuttcap%
\pgfsetmiterjoin%
\definecolor{currentfill}{rgb}{0.000000,0.000000,0.000000}%
\pgfsetfillcolor{currentfill}%
\pgfsetlinewidth{0.000000pt}%
\definecolor{currentstroke}{rgb}{0.000000,0.000000,0.000000}%
\pgfsetstrokecolor{currentstroke}%
\pgfsetstrokeopacity{0.000000}%
\pgfsetdash{}{0pt}%
\pgfpathmoveto{\pgfqpoint{2.241485in}{0.499444in}}%
\pgfpathlineto{\pgfqpoint{2.302872in}{0.499444in}}%
\pgfpathlineto{\pgfqpoint{2.302872in}{0.499444in}}%
\pgfpathlineto{\pgfqpoint{2.241485in}{0.499444in}}%
\pgfpathlineto{\pgfqpoint{2.241485in}{0.499444in}}%
\pgfpathclose%
\pgfusepath{fill}%
\end{pgfscope}%
\begin{pgfscope}%
\pgfpathrectangle{\pgfqpoint{0.515000in}{0.499444in}}{\pgfqpoint{3.875000in}{1.155000in}}%
\pgfusepath{clip}%
\pgfsetbuttcap%
\pgfsetmiterjoin%
\definecolor{currentfill}{rgb}{0.000000,0.000000,0.000000}%
\pgfsetfillcolor{currentfill}%
\pgfsetlinewidth{0.000000pt}%
\definecolor{currentstroke}{rgb}{0.000000,0.000000,0.000000}%
\pgfsetstrokecolor{currentstroke}%
\pgfsetstrokeopacity{0.000000}%
\pgfsetdash{}{0pt}%
\pgfpathmoveto{\pgfqpoint{2.394951in}{0.499444in}}%
\pgfpathlineto{\pgfqpoint{2.456337in}{0.499444in}}%
\pgfpathlineto{\pgfqpoint{2.456337in}{0.705872in}}%
\pgfpathlineto{\pgfqpoint{2.394951in}{0.705872in}}%
\pgfpathlineto{\pgfqpoint{2.394951in}{0.499444in}}%
\pgfpathclose%
\pgfusepath{fill}%
\end{pgfscope}%
\begin{pgfscope}%
\pgfpathrectangle{\pgfqpoint{0.515000in}{0.499444in}}{\pgfqpoint{3.875000in}{1.155000in}}%
\pgfusepath{clip}%
\pgfsetbuttcap%
\pgfsetmiterjoin%
\definecolor{currentfill}{rgb}{0.000000,0.000000,0.000000}%
\pgfsetfillcolor{currentfill}%
\pgfsetlinewidth{0.000000pt}%
\definecolor{currentstroke}{rgb}{0.000000,0.000000,0.000000}%
\pgfsetstrokecolor{currentstroke}%
\pgfsetstrokeopacity{0.000000}%
\pgfsetdash{}{0pt}%
\pgfpathmoveto{\pgfqpoint{2.548416in}{0.499444in}}%
\pgfpathlineto{\pgfqpoint{2.609802in}{0.499444in}}%
\pgfpathlineto{\pgfqpoint{2.609802in}{0.499444in}}%
\pgfpathlineto{\pgfqpoint{2.548416in}{0.499444in}}%
\pgfpathlineto{\pgfqpoint{2.548416in}{0.499444in}}%
\pgfpathclose%
\pgfusepath{fill}%
\end{pgfscope}%
\begin{pgfscope}%
\pgfpathrectangle{\pgfqpoint{0.515000in}{0.499444in}}{\pgfqpoint{3.875000in}{1.155000in}}%
\pgfusepath{clip}%
\pgfsetbuttcap%
\pgfsetmiterjoin%
\definecolor{currentfill}{rgb}{0.000000,0.000000,0.000000}%
\pgfsetfillcolor{currentfill}%
\pgfsetlinewidth{0.000000pt}%
\definecolor{currentstroke}{rgb}{0.000000,0.000000,0.000000}%
\pgfsetstrokecolor{currentstroke}%
\pgfsetstrokeopacity{0.000000}%
\pgfsetdash{}{0pt}%
\pgfpathmoveto{\pgfqpoint{2.701881in}{0.499444in}}%
\pgfpathlineto{\pgfqpoint{2.763268in}{0.499444in}}%
\pgfpathlineto{\pgfqpoint{2.763268in}{0.499444in}}%
\pgfpathlineto{\pgfqpoint{2.701881in}{0.499444in}}%
\pgfpathlineto{\pgfqpoint{2.701881in}{0.499444in}}%
\pgfpathclose%
\pgfusepath{fill}%
\end{pgfscope}%
\begin{pgfscope}%
\pgfpathrectangle{\pgfqpoint{0.515000in}{0.499444in}}{\pgfqpoint{3.875000in}{1.155000in}}%
\pgfusepath{clip}%
\pgfsetbuttcap%
\pgfsetmiterjoin%
\definecolor{currentfill}{rgb}{0.000000,0.000000,0.000000}%
\pgfsetfillcolor{currentfill}%
\pgfsetlinewidth{0.000000pt}%
\definecolor{currentstroke}{rgb}{0.000000,0.000000,0.000000}%
\pgfsetstrokecolor{currentstroke}%
\pgfsetstrokeopacity{0.000000}%
\pgfsetdash{}{0pt}%
\pgfpathmoveto{\pgfqpoint{2.855347in}{0.499444in}}%
\pgfpathlineto{\pgfqpoint{2.916733in}{0.499444in}}%
\pgfpathlineto{\pgfqpoint{2.916733in}{0.499444in}}%
\pgfpathlineto{\pgfqpoint{2.855347in}{0.499444in}}%
\pgfpathlineto{\pgfqpoint{2.855347in}{0.499444in}}%
\pgfpathclose%
\pgfusepath{fill}%
\end{pgfscope}%
\begin{pgfscope}%
\pgfpathrectangle{\pgfqpoint{0.515000in}{0.499444in}}{\pgfqpoint{3.875000in}{1.155000in}}%
\pgfusepath{clip}%
\pgfsetbuttcap%
\pgfsetmiterjoin%
\definecolor{currentfill}{rgb}{0.000000,0.000000,0.000000}%
\pgfsetfillcolor{currentfill}%
\pgfsetlinewidth{0.000000pt}%
\definecolor{currentstroke}{rgb}{0.000000,0.000000,0.000000}%
\pgfsetstrokecolor{currentstroke}%
\pgfsetstrokeopacity{0.000000}%
\pgfsetdash{}{0pt}%
\pgfpathmoveto{\pgfqpoint{3.008812in}{0.499444in}}%
\pgfpathlineto{\pgfqpoint{3.070198in}{0.499444in}}%
\pgfpathlineto{\pgfqpoint{3.070198in}{0.499444in}}%
\pgfpathlineto{\pgfqpoint{3.008812in}{0.499444in}}%
\pgfpathlineto{\pgfqpoint{3.008812in}{0.499444in}}%
\pgfpathclose%
\pgfusepath{fill}%
\end{pgfscope}%
\begin{pgfscope}%
\pgfpathrectangle{\pgfqpoint{0.515000in}{0.499444in}}{\pgfqpoint{3.875000in}{1.155000in}}%
\pgfusepath{clip}%
\pgfsetbuttcap%
\pgfsetmiterjoin%
\definecolor{currentfill}{rgb}{0.000000,0.000000,0.000000}%
\pgfsetfillcolor{currentfill}%
\pgfsetlinewidth{0.000000pt}%
\definecolor{currentstroke}{rgb}{0.000000,0.000000,0.000000}%
\pgfsetstrokecolor{currentstroke}%
\pgfsetstrokeopacity{0.000000}%
\pgfsetdash{}{0pt}%
\pgfpathmoveto{\pgfqpoint{3.162278in}{0.499444in}}%
\pgfpathlineto{\pgfqpoint{3.223664in}{0.499444in}}%
\pgfpathlineto{\pgfqpoint{3.223664in}{0.499444in}}%
\pgfpathlineto{\pgfqpoint{3.162278in}{0.499444in}}%
\pgfpathlineto{\pgfqpoint{3.162278in}{0.499444in}}%
\pgfpathclose%
\pgfusepath{fill}%
\end{pgfscope}%
\begin{pgfscope}%
\pgfpathrectangle{\pgfqpoint{0.515000in}{0.499444in}}{\pgfqpoint{3.875000in}{1.155000in}}%
\pgfusepath{clip}%
\pgfsetbuttcap%
\pgfsetmiterjoin%
\definecolor{currentfill}{rgb}{0.000000,0.000000,0.000000}%
\pgfsetfillcolor{currentfill}%
\pgfsetlinewidth{0.000000pt}%
\definecolor{currentstroke}{rgb}{0.000000,0.000000,0.000000}%
\pgfsetstrokecolor{currentstroke}%
\pgfsetstrokeopacity{0.000000}%
\pgfsetdash{}{0pt}%
\pgfpathmoveto{\pgfqpoint{3.315743in}{0.499444in}}%
\pgfpathlineto{\pgfqpoint{3.377129in}{0.499444in}}%
\pgfpathlineto{\pgfqpoint{3.377129in}{0.499444in}}%
\pgfpathlineto{\pgfqpoint{3.315743in}{0.499444in}}%
\pgfpathlineto{\pgfqpoint{3.315743in}{0.499444in}}%
\pgfpathclose%
\pgfusepath{fill}%
\end{pgfscope}%
\begin{pgfscope}%
\pgfpathrectangle{\pgfqpoint{0.515000in}{0.499444in}}{\pgfqpoint{3.875000in}{1.155000in}}%
\pgfusepath{clip}%
\pgfsetbuttcap%
\pgfsetmiterjoin%
\definecolor{currentfill}{rgb}{0.000000,0.000000,0.000000}%
\pgfsetfillcolor{currentfill}%
\pgfsetlinewidth{0.000000pt}%
\definecolor{currentstroke}{rgb}{0.000000,0.000000,0.000000}%
\pgfsetstrokecolor{currentstroke}%
\pgfsetstrokeopacity{0.000000}%
\pgfsetdash{}{0pt}%
\pgfpathmoveto{\pgfqpoint{3.469208in}{0.499444in}}%
\pgfpathlineto{\pgfqpoint{3.530594in}{0.499444in}}%
\pgfpathlineto{\pgfqpoint{3.530594in}{0.499444in}}%
\pgfpathlineto{\pgfqpoint{3.469208in}{0.499444in}}%
\pgfpathlineto{\pgfqpoint{3.469208in}{0.499444in}}%
\pgfpathclose%
\pgfusepath{fill}%
\end{pgfscope}%
\begin{pgfscope}%
\pgfpathrectangle{\pgfqpoint{0.515000in}{0.499444in}}{\pgfqpoint{3.875000in}{1.155000in}}%
\pgfusepath{clip}%
\pgfsetbuttcap%
\pgfsetmiterjoin%
\definecolor{currentfill}{rgb}{0.000000,0.000000,0.000000}%
\pgfsetfillcolor{currentfill}%
\pgfsetlinewidth{0.000000pt}%
\definecolor{currentstroke}{rgb}{0.000000,0.000000,0.000000}%
\pgfsetstrokecolor{currentstroke}%
\pgfsetstrokeopacity{0.000000}%
\pgfsetdash{}{0pt}%
\pgfpathmoveto{\pgfqpoint{3.622674in}{0.499444in}}%
\pgfpathlineto{\pgfqpoint{3.684060in}{0.499444in}}%
\pgfpathlineto{\pgfqpoint{3.684060in}{0.499444in}}%
\pgfpathlineto{\pgfqpoint{3.622674in}{0.499444in}}%
\pgfpathlineto{\pgfqpoint{3.622674in}{0.499444in}}%
\pgfpathclose%
\pgfusepath{fill}%
\end{pgfscope}%
\begin{pgfscope}%
\pgfpathrectangle{\pgfqpoint{0.515000in}{0.499444in}}{\pgfqpoint{3.875000in}{1.155000in}}%
\pgfusepath{clip}%
\pgfsetbuttcap%
\pgfsetmiterjoin%
\definecolor{currentfill}{rgb}{0.000000,0.000000,0.000000}%
\pgfsetfillcolor{currentfill}%
\pgfsetlinewidth{0.000000pt}%
\definecolor{currentstroke}{rgb}{0.000000,0.000000,0.000000}%
\pgfsetstrokecolor{currentstroke}%
\pgfsetstrokeopacity{0.000000}%
\pgfsetdash{}{0pt}%
\pgfpathmoveto{\pgfqpoint{3.776139in}{0.499444in}}%
\pgfpathlineto{\pgfqpoint{3.837525in}{0.499444in}}%
\pgfpathlineto{\pgfqpoint{3.837525in}{0.499444in}}%
\pgfpathlineto{\pgfqpoint{3.776139in}{0.499444in}}%
\pgfpathlineto{\pgfqpoint{3.776139in}{0.499444in}}%
\pgfpathclose%
\pgfusepath{fill}%
\end{pgfscope}%
\begin{pgfscope}%
\pgfpathrectangle{\pgfqpoint{0.515000in}{0.499444in}}{\pgfqpoint{3.875000in}{1.155000in}}%
\pgfusepath{clip}%
\pgfsetbuttcap%
\pgfsetmiterjoin%
\definecolor{currentfill}{rgb}{0.000000,0.000000,0.000000}%
\pgfsetfillcolor{currentfill}%
\pgfsetlinewidth{0.000000pt}%
\definecolor{currentstroke}{rgb}{0.000000,0.000000,0.000000}%
\pgfsetstrokecolor{currentstroke}%
\pgfsetstrokeopacity{0.000000}%
\pgfsetdash{}{0pt}%
\pgfpathmoveto{\pgfqpoint{3.929604in}{0.499444in}}%
\pgfpathlineto{\pgfqpoint{3.990990in}{0.499444in}}%
\pgfpathlineto{\pgfqpoint{3.990990in}{0.499444in}}%
\pgfpathlineto{\pgfqpoint{3.929604in}{0.499444in}}%
\pgfpathlineto{\pgfqpoint{3.929604in}{0.499444in}}%
\pgfpathclose%
\pgfusepath{fill}%
\end{pgfscope}%
\begin{pgfscope}%
\pgfpathrectangle{\pgfqpoint{0.515000in}{0.499444in}}{\pgfqpoint{3.875000in}{1.155000in}}%
\pgfusepath{clip}%
\pgfsetbuttcap%
\pgfsetmiterjoin%
\definecolor{currentfill}{rgb}{0.000000,0.000000,0.000000}%
\pgfsetfillcolor{currentfill}%
\pgfsetlinewidth{0.000000pt}%
\definecolor{currentstroke}{rgb}{0.000000,0.000000,0.000000}%
\pgfsetstrokecolor{currentstroke}%
\pgfsetstrokeopacity{0.000000}%
\pgfsetdash{}{0pt}%
\pgfpathmoveto{\pgfqpoint{4.083070in}{0.499444in}}%
\pgfpathlineto{\pgfqpoint{4.144456in}{0.499444in}}%
\pgfpathlineto{\pgfqpoint{4.144456in}{0.499444in}}%
\pgfpathlineto{\pgfqpoint{4.083070in}{0.499444in}}%
\pgfpathlineto{\pgfqpoint{4.083070in}{0.499444in}}%
\pgfpathclose%
\pgfusepath{fill}%
\end{pgfscope}%
\begin{pgfscope}%
\pgfpathrectangle{\pgfqpoint{0.515000in}{0.499444in}}{\pgfqpoint{3.875000in}{1.155000in}}%
\pgfusepath{clip}%
\pgfsetbuttcap%
\pgfsetmiterjoin%
\definecolor{currentfill}{rgb}{0.000000,0.000000,0.000000}%
\pgfsetfillcolor{currentfill}%
\pgfsetlinewidth{0.000000pt}%
\definecolor{currentstroke}{rgb}{0.000000,0.000000,0.000000}%
\pgfsetstrokecolor{currentstroke}%
\pgfsetstrokeopacity{0.000000}%
\pgfsetdash{}{0pt}%
\pgfpathmoveto{\pgfqpoint{4.236535in}{0.499444in}}%
\pgfpathlineto{\pgfqpoint{4.297921in}{0.499444in}}%
\pgfpathlineto{\pgfqpoint{4.297921in}{0.499444in}}%
\pgfpathlineto{\pgfqpoint{4.236535in}{0.499444in}}%
\pgfpathlineto{\pgfqpoint{4.236535in}{0.499444in}}%
\pgfpathclose%
\pgfusepath{fill}%
\end{pgfscope}%
\begin{pgfscope}%
\pgfsetbuttcap%
\pgfsetroundjoin%
\definecolor{currentfill}{rgb}{0.000000,0.000000,0.000000}%
\pgfsetfillcolor{currentfill}%
\pgfsetlinewidth{0.803000pt}%
\definecolor{currentstroke}{rgb}{0.000000,0.000000,0.000000}%
\pgfsetstrokecolor{currentstroke}%
\pgfsetdash{}{0pt}%
\pgfsys@defobject{currentmarker}{\pgfqpoint{0.000000in}{-0.048611in}}{\pgfqpoint{0.000000in}{0.000000in}}{%
\pgfpathmoveto{\pgfqpoint{0.000000in}{0.000000in}}%
\pgfpathlineto{\pgfqpoint{0.000000in}{-0.048611in}}%
\pgfusepath{stroke,fill}%
}%
\begin{pgfscope}%
\pgfsys@transformshift{0.553367in}{0.499444in}%
\pgfsys@useobject{currentmarker}{}%
\end{pgfscope}%
\end{pgfscope}%
\begin{pgfscope}%
\definecolor{textcolor}{rgb}{0.000000,0.000000,0.000000}%
\pgfsetstrokecolor{textcolor}%
\pgfsetfillcolor{textcolor}%
\pgftext[x=0.553367in,y=0.402222in,,top]{\color{textcolor}\rmfamily\fontsize{10.000000}{12.000000}\selectfont 0.0}%
\end{pgfscope}%
\begin{pgfscope}%
\pgfsetbuttcap%
\pgfsetroundjoin%
\definecolor{currentfill}{rgb}{0.000000,0.000000,0.000000}%
\pgfsetfillcolor{currentfill}%
\pgfsetlinewidth{0.803000pt}%
\definecolor{currentstroke}{rgb}{0.000000,0.000000,0.000000}%
\pgfsetstrokecolor{currentstroke}%
\pgfsetdash{}{0pt}%
\pgfsys@defobject{currentmarker}{\pgfqpoint{0.000000in}{-0.048611in}}{\pgfqpoint{0.000000in}{0.000000in}}{%
\pgfpathmoveto{\pgfqpoint{0.000000in}{0.000000in}}%
\pgfpathlineto{\pgfqpoint{0.000000in}{-0.048611in}}%
\pgfusepath{stroke,fill}%
}%
\begin{pgfscope}%
\pgfsys@transformshift{0.937030in}{0.499444in}%
\pgfsys@useobject{currentmarker}{}%
\end{pgfscope}%
\end{pgfscope}%
\begin{pgfscope}%
\definecolor{textcolor}{rgb}{0.000000,0.000000,0.000000}%
\pgfsetstrokecolor{textcolor}%
\pgfsetfillcolor{textcolor}%
\pgftext[x=0.937030in,y=0.402222in,,top]{\color{textcolor}\rmfamily\fontsize{10.000000}{12.000000}\selectfont 0.1}%
\end{pgfscope}%
\begin{pgfscope}%
\pgfsetbuttcap%
\pgfsetroundjoin%
\definecolor{currentfill}{rgb}{0.000000,0.000000,0.000000}%
\pgfsetfillcolor{currentfill}%
\pgfsetlinewidth{0.803000pt}%
\definecolor{currentstroke}{rgb}{0.000000,0.000000,0.000000}%
\pgfsetstrokecolor{currentstroke}%
\pgfsetdash{}{0pt}%
\pgfsys@defobject{currentmarker}{\pgfqpoint{0.000000in}{-0.048611in}}{\pgfqpoint{0.000000in}{0.000000in}}{%
\pgfpathmoveto{\pgfqpoint{0.000000in}{0.000000in}}%
\pgfpathlineto{\pgfqpoint{0.000000in}{-0.048611in}}%
\pgfusepath{stroke,fill}%
}%
\begin{pgfscope}%
\pgfsys@transformshift{1.320693in}{0.499444in}%
\pgfsys@useobject{currentmarker}{}%
\end{pgfscope}%
\end{pgfscope}%
\begin{pgfscope}%
\definecolor{textcolor}{rgb}{0.000000,0.000000,0.000000}%
\pgfsetstrokecolor{textcolor}%
\pgfsetfillcolor{textcolor}%
\pgftext[x=1.320693in,y=0.402222in,,top]{\color{textcolor}\rmfamily\fontsize{10.000000}{12.000000}\selectfont 0.2}%
\end{pgfscope}%
\begin{pgfscope}%
\pgfsetbuttcap%
\pgfsetroundjoin%
\definecolor{currentfill}{rgb}{0.000000,0.000000,0.000000}%
\pgfsetfillcolor{currentfill}%
\pgfsetlinewidth{0.803000pt}%
\definecolor{currentstroke}{rgb}{0.000000,0.000000,0.000000}%
\pgfsetstrokecolor{currentstroke}%
\pgfsetdash{}{0pt}%
\pgfsys@defobject{currentmarker}{\pgfqpoint{0.000000in}{-0.048611in}}{\pgfqpoint{0.000000in}{0.000000in}}{%
\pgfpathmoveto{\pgfqpoint{0.000000in}{0.000000in}}%
\pgfpathlineto{\pgfqpoint{0.000000in}{-0.048611in}}%
\pgfusepath{stroke,fill}%
}%
\begin{pgfscope}%
\pgfsys@transformshift{1.704357in}{0.499444in}%
\pgfsys@useobject{currentmarker}{}%
\end{pgfscope}%
\end{pgfscope}%
\begin{pgfscope}%
\definecolor{textcolor}{rgb}{0.000000,0.000000,0.000000}%
\pgfsetstrokecolor{textcolor}%
\pgfsetfillcolor{textcolor}%
\pgftext[x=1.704357in,y=0.402222in,,top]{\color{textcolor}\rmfamily\fontsize{10.000000}{12.000000}\selectfont 0.3}%
\end{pgfscope}%
\begin{pgfscope}%
\pgfsetbuttcap%
\pgfsetroundjoin%
\definecolor{currentfill}{rgb}{0.000000,0.000000,0.000000}%
\pgfsetfillcolor{currentfill}%
\pgfsetlinewidth{0.803000pt}%
\definecolor{currentstroke}{rgb}{0.000000,0.000000,0.000000}%
\pgfsetstrokecolor{currentstroke}%
\pgfsetdash{}{0pt}%
\pgfsys@defobject{currentmarker}{\pgfqpoint{0.000000in}{-0.048611in}}{\pgfqpoint{0.000000in}{0.000000in}}{%
\pgfpathmoveto{\pgfqpoint{0.000000in}{0.000000in}}%
\pgfpathlineto{\pgfqpoint{0.000000in}{-0.048611in}}%
\pgfusepath{stroke,fill}%
}%
\begin{pgfscope}%
\pgfsys@transformshift{2.088020in}{0.499444in}%
\pgfsys@useobject{currentmarker}{}%
\end{pgfscope}%
\end{pgfscope}%
\begin{pgfscope}%
\definecolor{textcolor}{rgb}{0.000000,0.000000,0.000000}%
\pgfsetstrokecolor{textcolor}%
\pgfsetfillcolor{textcolor}%
\pgftext[x=2.088020in,y=0.402222in,,top]{\color{textcolor}\rmfamily\fontsize{10.000000}{12.000000}\selectfont 0.4}%
\end{pgfscope}%
\begin{pgfscope}%
\pgfsetbuttcap%
\pgfsetroundjoin%
\definecolor{currentfill}{rgb}{0.000000,0.000000,0.000000}%
\pgfsetfillcolor{currentfill}%
\pgfsetlinewidth{0.803000pt}%
\definecolor{currentstroke}{rgb}{0.000000,0.000000,0.000000}%
\pgfsetstrokecolor{currentstroke}%
\pgfsetdash{}{0pt}%
\pgfsys@defobject{currentmarker}{\pgfqpoint{0.000000in}{-0.048611in}}{\pgfqpoint{0.000000in}{0.000000in}}{%
\pgfpathmoveto{\pgfqpoint{0.000000in}{0.000000in}}%
\pgfpathlineto{\pgfqpoint{0.000000in}{-0.048611in}}%
\pgfusepath{stroke,fill}%
}%
\begin{pgfscope}%
\pgfsys@transformshift{2.471683in}{0.499444in}%
\pgfsys@useobject{currentmarker}{}%
\end{pgfscope}%
\end{pgfscope}%
\begin{pgfscope}%
\definecolor{textcolor}{rgb}{0.000000,0.000000,0.000000}%
\pgfsetstrokecolor{textcolor}%
\pgfsetfillcolor{textcolor}%
\pgftext[x=2.471683in,y=0.402222in,,top]{\color{textcolor}\rmfamily\fontsize{10.000000}{12.000000}\selectfont 0.5}%
\end{pgfscope}%
\begin{pgfscope}%
\pgfsetbuttcap%
\pgfsetroundjoin%
\definecolor{currentfill}{rgb}{0.000000,0.000000,0.000000}%
\pgfsetfillcolor{currentfill}%
\pgfsetlinewidth{0.803000pt}%
\definecolor{currentstroke}{rgb}{0.000000,0.000000,0.000000}%
\pgfsetstrokecolor{currentstroke}%
\pgfsetdash{}{0pt}%
\pgfsys@defobject{currentmarker}{\pgfqpoint{0.000000in}{-0.048611in}}{\pgfqpoint{0.000000in}{0.000000in}}{%
\pgfpathmoveto{\pgfqpoint{0.000000in}{0.000000in}}%
\pgfpathlineto{\pgfqpoint{0.000000in}{-0.048611in}}%
\pgfusepath{stroke,fill}%
}%
\begin{pgfscope}%
\pgfsys@transformshift{2.855347in}{0.499444in}%
\pgfsys@useobject{currentmarker}{}%
\end{pgfscope}%
\end{pgfscope}%
\begin{pgfscope}%
\definecolor{textcolor}{rgb}{0.000000,0.000000,0.000000}%
\pgfsetstrokecolor{textcolor}%
\pgfsetfillcolor{textcolor}%
\pgftext[x=2.855347in,y=0.402222in,,top]{\color{textcolor}\rmfamily\fontsize{10.000000}{12.000000}\selectfont 0.6}%
\end{pgfscope}%
\begin{pgfscope}%
\pgfsetbuttcap%
\pgfsetroundjoin%
\definecolor{currentfill}{rgb}{0.000000,0.000000,0.000000}%
\pgfsetfillcolor{currentfill}%
\pgfsetlinewidth{0.803000pt}%
\definecolor{currentstroke}{rgb}{0.000000,0.000000,0.000000}%
\pgfsetstrokecolor{currentstroke}%
\pgfsetdash{}{0pt}%
\pgfsys@defobject{currentmarker}{\pgfqpoint{0.000000in}{-0.048611in}}{\pgfqpoint{0.000000in}{0.000000in}}{%
\pgfpathmoveto{\pgfqpoint{0.000000in}{0.000000in}}%
\pgfpathlineto{\pgfqpoint{0.000000in}{-0.048611in}}%
\pgfusepath{stroke,fill}%
}%
\begin{pgfscope}%
\pgfsys@transformshift{3.239010in}{0.499444in}%
\pgfsys@useobject{currentmarker}{}%
\end{pgfscope}%
\end{pgfscope}%
\begin{pgfscope}%
\definecolor{textcolor}{rgb}{0.000000,0.000000,0.000000}%
\pgfsetstrokecolor{textcolor}%
\pgfsetfillcolor{textcolor}%
\pgftext[x=3.239010in,y=0.402222in,,top]{\color{textcolor}\rmfamily\fontsize{10.000000}{12.000000}\selectfont 0.7}%
\end{pgfscope}%
\begin{pgfscope}%
\pgfsetbuttcap%
\pgfsetroundjoin%
\definecolor{currentfill}{rgb}{0.000000,0.000000,0.000000}%
\pgfsetfillcolor{currentfill}%
\pgfsetlinewidth{0.803000pt}%
\definecolor{currentstroke}{rgb}{0.000000,0.000000,0.000000}%
\pgfsetstrokecolor{currentstroke}%
\pgfsetdash{}{0pt}%
\pgfsys@defobject{currentmarker}{\pgfqpoint{0.000000in}{-0.048611in}}{\pgfqpoint{0.000000in}{0.000000in}}{%
\pgfpathmoveto{\pgfqpoint{0.000000in}{0.000000in}}%
\pgfpathlineto{\pgfqpoint{0.000000in}{-0.048611in}}%
\pgfusepath{stroke,fill}%
}%
\begin{pgfscope}%
\pgfsys@transformshift{3.622674in}{0.499444in}%
\pgfsys@useobject{currentmarker}{}%
\end{pgfscope}%
\end{pgfscope}%
\begin{pgfscope}%
\definecolor{textcolor}{rgb}{0.000000,0.000000,0.000000}%
\pgfsetstrokecolor{textcolor}%
\pgfsetfillcolor{textcolor}%
\pgftext[x=3.622674in,y=0.402222in,,top]{\color{textcolor}\rmfamily\fontsize{10.000000}{12.000000}\selectfont 0.8}%
\end{pgfscope}%
\begin{pgfscope}%
\pgfsetbuttcap%
\pgfsetroundjoin%
\definecolor{currentfill}{rgb}{0.000000,0.000000,0.000000}%
\pgfsetfillcolor{currentfill}%
\pgfsetlinewidth{0.803000pt}%
\definecolor{currentstroke}{rgb}{0.000000,0.000000,0.000000}%
\pgfsetstrokecolor{currentstroke}%
\pgfsetdash{}{0pt}%
\pgfsys@defobject{currentmarker}{\pgfqpoint{0.000000in}{-0.048611in}}{\pgfqpoint{0.000000in}{0.000000in}}{%
\pgfpathmoveto{\pgfqpoint{0.000000in}{0.000000in}}%
\pgfpathlineto{\pgfqpoint{0.000000in}{-0.048611in}}%
\pgfusepath{stroke,fill}%
}%
\begin{pgfscope}%
\pgfsys@transformshift{4.006337in}{0.499444in}%
\pgfsys@useobject{currentmarker}{}%
\end{pgfscope}%
\end{pgfscope}%
\begin{pgfscope}%
\definecolor{textcolor}{rgb}{0.000000,0.000000,0.000000}%
\pgfsetstrokecolor{textcolor}%
\pgfsetfillcolor{textcolor}%
\pgftext[x=4.006337in,y=0.402222in,,top]{\color{textcolor}\rmfamily\fontsize{10.000000}{12.000000}\selectfont 0.9}%
\end{pgfscope}%
\begin{pgfscope}%
\pgfsetbuttcap%
\pgfsetroundjoin%
\definecolor{currentfill}{rgb}{0.000000,0.000000,0.000000}%
\pgfsetfillcolor{currentfill}%
\pgfsetlinewidth{0.803000pt}%
\definecolor{currentstroke}{rgb}{0.000000,0.000000,0.000000}%
\pgfsetstrokecolor{currentstroke}%
\pgfsetdash{}{0pt}%
\pgfsys@defobject{currentmarker}{\pgfqpoint{0.000000in}{-0.048611in}}{\pgfqpoint{0.000000in}{0.000000in}}{%
\pgfpathmoveto{\pgfqpoint{0.000000in}{0.000000in}}%
\pgfpathlineto{\pgfqpoint{0.000000in}{-0.048611in}}%
\pgfusepath{stroke,fill}%
}%
\begin{pgfscope}%
\pgfsys@transformshift{4.390000in}{0.499444in}%
\pgfsys@useobject{currentmarker}{}%
\end{pgfscope}%
\end{pgfscope}%
\begin{pgfscope}%
\definecolor{textcolor}{rgb}{0.000000,0.000000,0.000000}%
\pgfsetstrokecolor{textcolor}%
\pgfsetfillcolor{textcolor}%
\pgftext[x=4.390000in,y=0.402222in,,top]{\color{textcolor}\rmfamily\fontsize{10.000000}{12.000000}\selectfont 1.0}%
\end{pgfscope}%
\begin{pgfscope}%
\definecolor{textcolor}{rgb}{0.000000,0.000000,0.000000}%
\pgfsetstrokecolor{textcolor}%
\pgfsetfillcolor{textcolor}%
\pgftext[x=2.452500in,y=0.223333in,,top]{\color{textcolor}\rmfamily\fontsize{10.000000}{12.000000}\selectfont \(\displaystyle p\)}%
\end{pgfscope}%
\begin{pgfscope}%
\pgfsetbuttcap%
\pgfsetroundjoin%
\definecolor{currentfill}{rgb}{0.000000,0.000000,0.000000}%
\pgfsetfillcolor{currentfill}%
\pgfsetlinewidth{0.803000pt}%
\definecolor{currentstroke}{rgb}{0.000000,0.000000,0.000000}%
\pgfsetstrokecolor{currentstroke}%
\pgfsetdash{}{0pt}%
\pgfsys@defobject{currentmarker}{\pgfqpoint{-0.048611in}{0.000000in}}{\pgfqpoint{-0.000000in}{0.000000in}}{%
\pgfpathmoveto{\pgfqpoint{-0.000000in}{0.000000in}}%
\pgfpathlineto{\pgfqpoint{-0.048611in}{0.000000in}}%
\pgfusepath{stroke,fill}%
}%
\begin{pgfscope}%
\pgfsys@transformshift{0.515000in}{0.499444in}%
\pgfsys@useobject{currentmarker}{}%
\end{pgfscope}%
\end{pgfscope}%
\begin{pgfscope}%
\definecolor{textcolor}{rgb}{0.000000,0.000000,0.000000}%
\pgfsetstrokecolor{textcolor}%
\pgfsetfillcolor{textcolor}%
\pgftext[x=0.348333in, y=0.451250in, left, base]{\color{textcolor}\rmfamily\fontsize{10.000000}{12.000000}\selectfont \(\displaystyle {0}\)}%
\end{pgfscope}%
\begin{pgfscope}%
\pgfsetbuttcap%
\pgfsetroundjoin%
\definecolor{currentfill}{rgb}{0.000000,0.000000,0.000000}%
\pgfsetfillcolor{currentfill}%
\pgfsetlinewidth{0.803000pt}%
\definecolor{currentstroke}{rgb}{0.000000,0.000000,0.000000}%
\pgfsetstrokecolor{currentstroke}%
\pgfsetdash{}{0pt}%
\pgfsys@defobject{currentmarker}{\pgfqpoint{-0.048611in}{0.000000in}}{\pgfqpoint{-0.000000in}{0.000000in}}{%
\pgfpathmoveto{\pgfqpoint{-0.000000in}{0.000000in}}%
\pgfpathlineto{\pgfqpoint{-0.048611in}{0.000000in}}%
\pgfusepath{stroke,fill}%
}%
\begin{pgfscope}%
\pgfsys@transformshift{0.515000in}{0.826051in}%
\pgfsys@useobject{currentmarker}{}%
\end{pgfscope}%
\end{pgfscope}%
\begin{pgfscope}%
\definecolor{textcolor}{rgb}{0.000000,0.000000,0.000000}%
\pgfsetstrokecolor{textcolor}%
\pgfsetfillcolor{textcolor}%
\pgftext[x=0.278889in, y=0.777857in, left, base]{\color{textcolor}\rmfamily\fontsize{10.000000}{12.000000}\selectfont \(\displaystyle {25}\)}%
\end{pgfscope}%
\begin{pgfscope}%
\pgfsetbuttcap%
\pgfsetroundjoin%
\definecolor{currentfill}{rgb}{0.000000,0.000000,0.000000}%
\pgfsetfillcolor{currentfill}%
\pgfsetlinewidth{0.803000pt}%
\definecolor{currentstroke}{rgb}{0.000000,0.000000,0.000000}%
\pgfsetstrokecolor{currentstroke}%
\pgfsetdash{}{0pt}%
\pgfsys@defobject{currentmarker}{\pgfqpoint{-0.048611in}{0.000000in}}{\pgfqpoint{-0.000000in}{0.000000in}}{%
\pgfpathmoveto{\pgfqpoint{-0.000000in}{0.000000in}}%
\pgfpathlineto{\pgfqpoint{-0.048611in}{0.000000in}}%
\pgfusepath{stroke,fill}%
}%
\begin{pgfscope}%
\pgfsys@transformshift{0.515000in}{1.152658in}%
\pgfsys@useobject{currentmarker}{}%
\end{pgfscope}%
\end{pgfscope}%
\begin{pgfscope}%
\definecolor{textcolor}{rgb}{0.000000,0.000000,0.000000}%
\pgfsetstrokecolor{textcolor}%
\pgfsetfillcolor{textcolor}%
\pgftext[x=0.278889in, y=1.104463in, left, base]{\color{textcolor}\rmfamily\fontsize{10.000000}{12.000000}\selectfont \(\displaystyle {50}\)}%
\end{pgfscope}%
\begin{pgfscope}%
\pgfsetbuttcap%
\pgfsetroundjoin%
\definecolor{currentfill}{rgb}{0.000000,0.000000,0.000000}%
\pgfsetfillcolor{currentfill}%
\pgfsetlinewidth{0.803000pt}%
\definecolor{currentstroke}{rgb}{0.000000,0.000000,0.000000}%
\pgfsetstrokecolor{currentstroke}%
\pgfsetdash{}{0pt}%
\pgfsys@defobject{currentmarker}{\pgfqpoint{-0.048611in}{0.000000in}}{\pgfqpoint{-0.000000in}{0.000000in}}{%
\pgfpathmoveto{\pgfqpoint{-0.000000in}{0.000000in}}%
\pgfpathlineto{\pgfqpoint{-0.048611in}{0.000000in}}%
\pgfusepath{stroke,fill}%
}%
\begin{pgfscope}%
\pgfsys@transformshift{0.515000in}{1.479265in}%
\pgfsys@useobject{currentmarker}{}%
\end{pgfscope}%
\end{pgfscope}%
\begin{pgfscope}%
\definecolor{textcolor}{rgb}{0.000000,0.000000,0.000000}%
\pgfsetstrokecolor{textcolor}%
\pgfsetfillcolor{textcolor}%
\pgftext[x=0.278889in, y=1.431070in, left, base]{\color{textcolor}\rmfamily\fontsize{10.000000}{12.000000}\selectfont \(\displaystyle {75}\)}%
\end{pgfscope}%
\begin{pgfscope}%
\definecolor{textcolor}{rgb}{0.000000,0.000000,0.000000}%
\pgfsetstrokecolor{textcolor}%
\pgfsetfillcolor{textcolor}%
\pgftext[x=0.223333in,y=1.076944in,,bottom,rotate=90.000000]{\color{textcolor}\rmfamily\fontsize{10.000000}{12.000000}\selectfont Percent of Data Set}%
\end{pgfscope}%
\begin{pgfscope}%
\pgfsetrectcap%
\pgfsetmiterjoin%
\pgfsetlinewidth{0.803000pt}%
\definecolor{currentstroke}{rgb}{0.000000,0.000000,0.000000}%
\pgfsetstrokecolor{currentstroke}%
\pgfsetdash{}{0pt}%
\pgfpathmoveto{\pgfqpoint{0.515000in}{0.499444in}}%
\pgfpathlineto{\pgfqpoint{0.515000in}{1.654444in}}%
\pgfusepath{stroke}%
\end{pgfscope}%
\begin{pgfscope}%
\pgfsetrectcap%
\pgfsetmiterjoin%
\pgfsetlinewidth{0.803000pt}%
\definecolor{currentstroke}{rgb}{0.000000,0.000000,0.000000}%
\pgfsetstrokecolor{currentstroke}%
\pgfsetdash{}{0pt}%
\pgfpathmoveto{\pgfqpoint{4.390000in}{0.499444in}}%
\pgfpathlineto{\pgfqpoint{4.390000in}{1.654444in}}%
\pgfusepath{stroke}%
\end{pgfscope}%
\begin{pgfscope}%
\pgfsetrectcap%
\pgfsetmiterjoin%
\pgfsetlinewidth{0.803000pt}%
\definecolor{currentstroke}{rgb}{0.000000,0.000000,0.000000}%
\pgfsetstrokecolor{currentstroke}%
\pgfsetdash{}{0pt}%
\pgfpathmoveto{\pgfqpoint{0.515000in}{0.499444in}}%
\pgfpathlineto{\pgfqpoint{4.390000in}{0.499444in}}%
\pgfusepath{stroke}%
\end{pgfscope}%
\begin{pgfscope}%
\pgfsetrectcap%
\pgfsetmiterjoin%
\pgfsetlinewidth{0.803000pt}%
\definecolor{currentstroke}{rgb}{0.000000,0.000000,0.000000}%
\pgfsetstrokecolor{currentstroke}%
\pgfsetdash{}{0pt}%
\pgfpathmoveto{\pgfqpoint{0.515000in}{1.654444in}}%
\pgfpathlineto{\pgfqpoint{4.390000in}{1.654444in}}%
\pgfusepath{stroke}%
\end{pgfscope}%
\begin{pgfscope}%
\pgfsetbuttcap%
\pgfsetmiterjoin%
\definecolor{currentfill}{rgb}{1.000000,1.000000,1.000000}%
\pgfsetfillcolor{currentfill}%
\pgfsetfillopacity{0.800000}%
\pgfsetlinewidth{1.003750pt}%
\definecolor{currentstroke}{rgb}{0.800000,0.800000,0.800000}%
\pgfsetstrokecolor{currentstroke}%
\pgfsetstrokeopacity{0.800000}%
\pgfsetdash{}{0pt}%
\pgfpathmoveto{\pgfqpoint{3.613056in}{1.154445in}}%
\pgfpathlineto{\pgfqpoint{4.292778in}{1.154445in}}%
\pgfpathquadraticcurveto{\pgfqpoint{4.320556in}{1.154445in}}{\pgfqpoint{4.320556in}{1.182222in}}%
\pgfpathlineto{\pgfqpoint{4.320556in}{1.557222in}}%
\pgfpathquadraticcurveto{\pgfqpoint{4.320556in}{1.585000in}}{\pgfqpoint{4.292778in}{1.585000in}}%
\pgfpathlineto{\pgfqpoint{3.613056in}{1.585000in}}%
\pgfpathquadraticcurveto{\pgfqpoint{3.585278in}{1.585000in}}{\pgfqpoint{3.585278in}{1.557222in}}%
\pgfpathlineto{\pgfqpoint{3.585278in}{1.182222in}}%
\pgfpathquadraticcurveto{\pgfqpoint{3.585278in}{1.154445in}}{\pgfqpoint{3.613056in}{1.154445in}}%
\pgfpathlineto{\pgfqpoint{3.613056in}{1.154445in}}%
\pgfpathclose%
\pgfusepath{stroke,fill}%
\end{pgfscope}%
\begin{pgfscope}%
\pgfsetbuttcap%
\pgfsetmiterjoin%
\pgfsetlinewidth{1.003750pt}%
\definecolor{currentstroke}{rgb}{0.000000,0.000000,0.000000}%
\pgfsetstrokecolor{currentstroke}%
\pgfsetdash{}{0pt}%
\pgfpathmoveto{\pgfqpoint{3.640834in}{1.432222in}}%
\pgfpathlineto{\pgfqpoint{3.918611in}{1.432222in}}%
\pgfpathlineto{\pgfqpoint{3.918611in}{1.529444in}}%
\pgfpathlineto{\pgfqpoint{3.640834in}{1.529444in}}%
\pgfpathlineto{\pgfqpoint{3.640834in}{1.432222in}}%
\pgfpathclose%
\pgfusepath{stroke}%
\end{pgfscope}%
\begin{pgfscope}%
\definecolor{textcolor}{rgb}{0.000000,0.000000,0.000000}%
\pgfsetstrokecolor{textcolor}%
\pgfsetfillcolor{textcolor}%
\pgftext[x=4.029723in,y=1.432222in,left,base]{\color{textcolor}\rmfamily\fontsize{10.000000}{12.000000}\selectfont Neg}%
\end{pgfscope}%
\begin{pgfscope}%
\pgfsetbuttcap%
\pgfsetmiterjoin%
\definecolor{currentfill}{rgb}{0.000000,0.000000,0.000000}%
\pgfsetfillcolor{currentfill}%
\pgfsetlinewidth{0.000000pt}%
\definecolor{currentstroke}{rgb}{0.000000,0.000000,0.000000}%
\pgfsetstrokecolor{currentstroke}%
\pgfsetstrokeopacity{0.000000}%
\pgfsetdash{}{0pt}%
\pgfpathmoveto{\pgfqpoint{3.640834in}{1.236944in}}%
\pgfpathlineto{\pgfqpoint{3.918611in}{1.236944in}}%
\pgfpathlineto{\pgfqpoint{3.918611in}{1.334167in}}%
\pgfpathlineto{\pgfqpoint{3.640834in}{1.334167in}}%
\pgfpathlineto{\pgfqpoint{3.640834in}{1.236944in}}%
\pgfpathclose%
\pgfusepath{fill}%
\end{pgfscope}%
\begin{pgfscope}%
\definecolor{textcolor}{rgb}{0.000000,0.000000,0.000000}%
\pgfsetstrokecolor{textcolor}%
\pgfsetfillcolor{textcolor}%
\pgftext[x=4.029723in,y=1.236944in,left,base]{\color{textcolor}\rmfamily\fontsize{10.000000}{12.000000}\selectfont Pos}%
\end{pgfscope}%
\end{pgfpicture}%
\makeatother%
\endgroup%
	
&
	\vskip 0pt
	\hfil ROC Curve
	
	%% Creator: Matplotlib, PGF backend
%%
%% To include the figure in your LaTeX document, write
%%   \input{<filename>.pgf}
%%
%% Make sure the required packages are loaded in your preamble
%%   \usepackage{pgf}
%%
%% Also ensure that all the required font packages are loaded; for instance,
%% the lmodern package is sometimes necessary when using math font.
%%   \usepackage{lmodern}
%%
%% Figures using additional raster images can only be included by \input if
%% they are in the same directory as the main LaTeX file. For loading figures
%% from other directories you can use the `import` package
%%   \usepackage{import}
%%
%% and then include the figures with
%%   \import{<path to file>}{<filename>.pgf}
%%
%% Matplotlib used the following preamble
%%   
%%   \usepackage{fontspec}
%%   \makeatletter\@ifpackageloaded{underscore}{}{\usepackage[strings]{underscore}}\makeatother
%%
\begingroup%
\makeatletter%
\begin{pgfpicture}%
\pgfpathrectangle{\pgfpointorigin}{\pgfqpoint{2.221861in}{1.754444in}}%
\pgfusepath{use as bounding box, clip}%
\begin{pgfscope}%
\pgfsetbuttcap%
\pgfsetmiterjoin%
\definecolor{currentfill}{rgb}{1.000000,1.000000,1.000000}%
\pgfsetfillcolor{currentfill}%
\pgfsetlinewidth{0.000000pt}%
\definecolor{currentstroke}{rgb}{1.000000,1.000000,1.000000}%
\pgfsetstrokecolor{currentstroke}%
\pgfsetdash{}{0pt}%
\pgfpathmoveto{\pgfqpoint{0.000000in}{0.000000in}}%
\pgfpathlineto{\pgfqpoint{2.221861in}{0.000000in}}%
\pgfpathlineto{\pgfqpoint{2.221861in}{1.754444in}}%
\pgfpathlineto{\pgfqpoint{0.000000in}{1.754444in}}%
\pgfpathlineto{\pgfqpoint{0.000000in}{0.000000in}}%
\pgfpathclose%
\pgfusepath{fill}%
\end{pgfscope}%
\begin{pgfscope}%
\pgfsetbuttcap%
\pgfsetmiterjoin%
\definecolor{currentfill}{rgb}{1.000000,1.000000,1.000000}%
\pgfsetfillcolor{currentfill}%
\pgfsetlinewidth{0.000000pt}%
\definecolor{currentstroke}{rgb}{0.000000,0.000000,0.000000}%
\pgfsetstrokecolor{currentstroke}%
\pgfsetstrokeopacity{0.000000}%
\pgfsetdash{}{0pt}%
\pgfpathmoveto{\pgfqpoint{0.553581in}{0.499444in}}%
\pgfpathlineto{\pgfqpoint{2.103581in}{0.499444in}}%
\pgfpathlineto{\pgfqpoint{2.103581in}{1.654444in}}%
\pgfpathlineto{\pgfqpoint{0.553581in}{1.654444in}}%
\pgfpathlineto{\pgfqpoint{0.553581in}{0.499444in}}%
\pgfpathclose%
\pgfusepath{fill}%
\end{pgfscope}%
\begin{pgfscope}%
\pgfsetbuttcap%
\pgfsetroundjoin%
\definecolor{currentfill}{rgb}{0.000000,0.000000,0.000000}%
\pgfsetfillcolor{currentfill}%
\pgfsetlinewidth{0.803000pt}%
\definecolor{currentstroke}{rgb}{0.000000,0.000000,0.000000}%
\pgfsetstrokecolor{currentstroke}%
\pgfsetdash{}{0pt}%
\pgfsys@defobject{currentmarker}{\pgfqpoint{0.000000in}{-0.048611in}}{\pgfqpoint{0.000000in}{0.000000in}}{%
\pgfpathmoveto{\pgfqpoint{0.000000in}{0.000000in}}%
\pgfpathlineto{\pgfqpoint{0.000000in}{-0.048611in}}%
\pgfusepath{stroke,fill}%
}%
\begin{pgfscope}%
\pgfsys@transformshift{0.624035in}{0.499444in}%
\pgfsys@useobject{currentmarker}{}%
\end{pgfscope}%
\end{pgfscope}%
\begin{pgfscope}%
\definecolor{textcolor}{rgb}{0.000000,0.000000,0.000000}%
\pgfsetstrokecolor{textcolor}%
\pgfsetfillcolor{textcolor}%
\pgftext[x=0.624035in,y=0.402222in,,top]{\color{textcolor}\rmfamily\fontsize{10.000000}{12.000000}\selectfont \(\displaystyle {0.0}\)}%
\end{pgfscope}%
\begin{pgfscope}%
\pgfsetbuttcap%
\pgfsetroundjoin%
\definecolor{currentfill}{rgb}{0.000000,0.000000,0.000000}%
\pgfsetfillcolor{currentfill}%
\pgfsetlinewidth{0.803000pt}%
\definecolor{currentstroke}{rgb}{0.000000,0.000000,0.000000}%
\pgfsetstrokecolor{currentstroke}%
\pgfsetdash{}{0pt}%
\pgfsys@defobject{currentmarker}{\pgfqpoint{0.000000in}{-0.048611in}}{\pgfqpoint{0.000000in}{0.000000in}}{%
\pgfpathmoveto{\pgfqpoint{0.000000in}{0.000000in}}%
\pgfpathlineto{\pgfqpoint{0.000000in}{-0.048611in}}%
\pgfusepath{stroke,fill}%
}%
\begin{pgfscope}%
\pgfsys@transformshift{1.328581in}{0.499444in}%
\pgfsys@useobject{currentmarker}{}%
\end{pgfscope}%
\end{pgfscope}%
\begin{pgfscope}%
\definecolor{textcolor}{rgb}{0.000000,0.000000,0.000000}%
\pgfsetstrokecolor{textcolor}%
\pgfsetfillcolor{textcolor}%
\pgftext[x=1.328581in,y=0.402222in,,top]{\color{textcolor}\rmfamily\fontsize{10.000000}{12.000000}\selectfont \(\displaystyle {0.5}\)}%
\end{pgfscope}%
\begin{pgfscope}%
\pgfsetbuttcap%
\pgfsetroundjoin%
\definecolor{currentfill}{rgb}{0.000000,0.000000,0.000000}%
\pgfsetfillcolor{currentfill}%
\pgfsetlinewidth{0.803000pt}%
\definecolor{currentstroke}{rgb}{0.000000,0.000000,0.000000}%
\pgfsetstrokecolor{currentstroke}%
\pgfsetdash{}{0pt}%
\pgfsys@defobject{currentmarker}{\pgfqpoint{0.000000in}{-0.048611in}}{\pgfqpoint{0.000000in}{0.000000in}}{%
\pgfpathmoveto{\pgfqpoint{0.000000in}{0.000000in}}%
\pgfpathlineto{\pgfqpoint{0.000000in}{-0.048611in}}%
\pgfusepath{stroke,fill}%
}%
\begin{pgfscope}%
\pgfsys@transformshift{2.033126in}{0.499444in}%
\pgfsys@useobject{currentmarker}{}%
\end{pgfscope}%
\end{pgfscope}%
\begin{pgfscope}%
\definecolor{textcolor}{rgb}{0.000000,0.000000,0.000000}%
\pgfsetstrokecolor{textcolor}%
\pgfsetfillcolor{textcolor}%
\pgftext[x=2.033126in,y=0.402222in,,top]{\color{textcolor}\rmfamily\fontsize{10.000000}{12.000000}\selectfont \(\displaystyle {1.0}\)}%
\end{pgfscope}%
\begin{pgfscope}%
\definecolor{textcolor}{rgb}{0.000000,0.000000,0.000000}%
\pgfsetstrokecolor{textcolor}%
\pgfsetfillcolor{textcolor}%
\pgftext[x=1.328581in,y=0.223333in,,top]{\color{textcolor}\rmfamily\fontsize{10.000000}{12.000000}\selectfont False positive rate}%
\end{pgfscope}%
\begin{pgfscope}%
\pgfsetbuttcap%
\pgfsetroundjoin%
\definecolor{currentfill}{rgb}{0.000000,0.000000,0.000000}%
\pgfsetfillcolor{currentfill}%
\pgfsetlinewidth{0.803000pt}%
\definecolor{currentstroke}{rgb}{0.000000,0.000000,0.000000}%
\pgfsetstrokecolor{currentstroke}%
\pgfsetdash{}{0pt}%
\pgfsys@defobject{currentmarker}{\pgfqpoint{-0.048611in}{0.000000in}}{\pgfqpoint{-0.000000in}{0.000000in}}{%
\pgfpathmoveto{\pgfqpoint{-0.000000in}{0.000000in}}%
\pgfpathlineto{\pgfqpoint{-0.048611in}{0.000000in}}%
\pgfusepath{stroke,fill}%
}%
\begin{pgfscope}%
\pgfsys@transformshift{0.553581in}{0.551944in}%
\pgfsys@useobject{currentmarker}{}%
\end{pgfscope}%
\end{pgfscope}%
\begin{pgfscope}%
\definecolor{textcolor}{rgb}{0.000000,0.000000,0.000000}%
\pgfsetstrokecolor{textcolor}%
\pgfsetfillcolor{textcolor}%
\pgftext[x=0.278889in, y=0.503750in, left, base]{\color{textcolor}\rmfamily\fontsize{10.000000}{12.000000}\selectfont \(\displaystyle {0.0}\)}%
\end{pgfscope}%
\begin{pgfscope}%
\pgfsetbuttcap%
\pgfsetroundjoin%
\definecolor{currentfill}{rgb}{0.000000,0.000000,0.000000}%
\pgfsetfillcolor{currentfill}%
\pgfsetlinewidth{0.803000pt}%
\definecolor{currentstroke}{rgb}{0.000000,0.000000,0.000000}%
\pgfsetstrokecolor{currentstroke}%
\pgfsetdash{}{0pt}%
\pgfsys@defobject{currentmarker}{\pgfqpoint{-0.048611in}{0.000000in}}{\pgfqpoint{-0.000000in}{0.000000in}}{%
\pgfpathmoveto{\pgfqpoint{-0.000000in}{0.000000in}}%
\pgfpathlineto{\pgfqpoint{-0.048611in}{0.000000in}}%
\pgfusepath{stroke,fill}%
}%
\begin{pgfscope}%
\pgfsys@transformshift{0.553581in}{1.076944in}%
\pgfsys@useobject{currentmarker}{}%
\end{pgfscope}%
\end{pgfscope}%
\begin{pgfscope}%
\definecolor{textcolor}{rgb}{0.000000,0.000000,0.000000}%
\pgfsetstrokecolor{textcolor}%
\pgfsetfillcolor{textcolor}%
\pgftext[x=0.278889in, y=1.028750in, left, base]{\color{textcolor}\rmfamily\fontsize{10.000000}{12.000000}\selectfont \(\displaystyle {0.5}\)}%
\end{pgfscope}%
\begin{pgfscope}%
\pgfsetbuttcap%
\pgfsetroundjoin%
\definecolor{currentfill}{rgb}{0.000000,0.000000,0.000000}%
\pgfsetfillcolor{currentfill}%
\pgfsetlinewidth{0.803000pt}%
\definecolor{currentstroke}{rgb}{0.000000,0.000000,0.000000}%
\pgfsetstrokecolor{currentstroke}%
\pgfsetdash{}{0pt}%
\pgfsys@defobject{currentmarker}{\pgfqpoint{-0.048611in}{0.000000in}}{\pgfqpoint{-0.000000in}{0.000000in}}{%
\pgfpathmoveto{\pgfqpoint{-0.000000in}{0.000000in}}%
\pgfpathlineto{\pgfqpoint{-0.048611in}{0.000000in}}%
\pgfusepath{stroke,fill}%
}%
\begin{pgfscope}%
\pgfsys@transformshift{0.553581in}{1.601944in}%
\pgfsys@useobject{currentmarker}{}%
\end{pgfscope}%
\end{pgfscope}%
\begin{pgfscope}%
\definecolor{textcolor}{rgb}{0.000000,0.000000,0.000000}%
\pgfsetstrokecolor{textcolor}%
\pgfsetfillcolor{textcolor}%
\pgftext[x=0.278889in, y=1.553750in, left, base]{\color{textcolor}\rmfamily\fontsize{10.000000}{12.000000}\selectfont \(\displaystyle {1.0}\)}%
\end{pgfscope}%
\begin{pgfscope}%
\definecolor{textcolor}{rgb}{0.000000,0.000000,0.000000}%
\pgfsetstrokecolor{textcolor}%
\pgfsetfillcolor{textcolor}%
\pgftext[x=0.223333in,y=1.076944in,,bottom,rotate=90.000000]{\color{textcolor}\rmfamily\fontsize{10.000000}{12.000000}\selectfont True positive rate}%
\end{pgfscope}%
\begin{pgfscope}%
\pgfpathrectangle{\pgfqpoint{0.553581in}{0.499444in}}{\pgfqpoint{1.550000in}{1.155000in}}%
\pgfusepath{clip}%
\pgfsetbuttcap%
\pgfsetroundjoin%
\pgfsetlinewidth{1.505625pt}%
\definecolor{currentstroke}{rgb}{0.000000,0.000000,0.000000}%
\pgfsetstrokecolor{currentstroke}%
\pgfsetdash{{5.550000pt}{2.400000pt}}{0.000000pt}%
\pgfpathmoveto{\pgfqpoint{0.624035in}{0.551944in}}%
\pgfpathlineto{\pgfqpoint{2.033126in}{1.601944in}}%
\pgfusepath{stroke}%
\end{pgfscope}%
\begin{pgfscope}%
\pgfpathrectangle{\pgfqpoint{0.553581in}{0.499444in}}{\pgfqpoint{1.550000in}{1.155000in}}%
\pgfusepath{clip}%
\pgfsetrectcap%
\pgfsetroundjoin%
\pgfsetlinewidth{1.505625pt}%
\definecolor{currentstroke}{rgb}{0.000000,0.000000,0.000000}%
\pgfsetstrokecolor{currentstroke}%
\pgfsetdash{}{0pt}%
\pgfpathmoveto{\pgfqpoint{0.624035in}{0.551944in}}%
\pgfpathlineto{\pgfqpoint{0.625130in}{0.562374in}}%
\pgfpathlineto{\pgfqpoint{0.625403in}{0.563430in}}%
\pgfpathlineto{\pgfqpoint{0.626513in}{0.572028in}}%
\pgfpathlineto{\pgfqpoint{0.626701in}{0.573084in}}%
\pgfpathlineto{\pgfqpoint{0.627796in}{0.578547in}}%
\pgfpathlineto{\pgfqpoint{0.627983in}{0.579603in}}%
\pgfpathlineto{\pgfqpoint{0.629093in}{0.584725in}}%
\pgfpathlineto{\pgfqpoint{0.629312in}{0.585780in}}%
\pgfpathlineto{\pgfqpoint{0.630407in}{0.592609in}}%
\pgfpathlineto{\pgfqpoint{0.630672in}{0.593634in}}%
\pgfpathlineto{\pgfqpoint{0.631775in}{0.598973in}}%
\pgfpathlineto{\pgfqpoint{0.631986in}{0.600059in}}%
\pgfpathlineto{\pgfqpoint{0.633088in}{0.605554in}}%
\pgfpathlineto{\pgfqpoint{0.633323in}{0.606640in}}%
\pgfpathlineto{\pgfqpoint{0.634433in}{0.611421in}}%
\pgfpathlineto{\pgfqpoint{0.634691in}{0.612507in}}%
\pgfpathlineto{\pgfqpoint{0.635793in}{0.617040in}}%
\pgfpathlineto{\pgfqpoint{0.636121in}{0.618126in}}%
\pgfpathlineto{\pgfqpoint{0.637231in}{0.622503in}}%
\pgfpathlineto{\pgfqpoint{0.637521in}{0.623527in}}%
\pgfpathlineto{\pgfqpoint{0.638623in}{0.628711in}}%
\pgfpathlineto{\pgfqpoint{0.638983in}{0.629767in}}%
\pgfpathlineto{\pgfqpoint{0.640077in}{0.633771in}}%
\pgfpathlineto{\pgfqpoint{0.640257in}{0.634858in}}%
\pgfpathlineto{\pgfqpoint{0.641351in}{0.639017in}}%
\pgfpathlineto{\pgfqpoint{0.641625in}{0.640042in}}%
\pgfpathlineto{\pgfqpoint{0.642735in}{0.644326in}}%
\pgfpathlineto{\pgfqpoint{0.642884in}{0.645226in}}%
\pgfpathlineto{\pgfqpoint{0.643986in}{0.649851in}}%
\pgfpathlineto{\pgfqpoint{0.644252in}{0.650844in}}%
\pgfpathlineto{\pgfqpoint{0.645330in}{0.655345in}}%
\pgfpathlineto{\pgfqpoint{0.645659in}{0.656339in}}%
\pgfpathlineto{\pgfqpoint{0.646761in}{0.660188in}}%
\pgfpathlineto{\pgfqpoint{0.647066in}{0.661275in}}%
\pgfpathlineto{\pgfqpoint{0.648160in}{0.665651in}}%
\pgfpathlineto{\pgfqpoint{0.648481in}{0.666645in}}%
\pgfpathlineto{\pgfqpoint{0.649591in}{0.671363in}}%
\pgfpathlineto{\pgfqpoint{0.649982in}{0.672419in}}%
\pgfpathlineto{\pgfqpoint{0.651053in}{0.677696in}}%
\pgfpathlineto{\pgfqpoint{0.651381in}{0.678751in}}%
\pgfpathlineto{\pgfqpoint{0.652484in}{0.682787in}}%
\pgfpathlineto{\pgfqpoint{0.652828in}{0.683842in}}%
\pgfpathlineto{\pgfqpoint{0.653930in}{0.688343in}}%
\pgfpathlineto{\pgfqpoint{0.654204in}{0.689399in}}%
\pgfpathlineto{\pgfqpoint{0.655314in}{0.693651in}}%
\pgfpathlineto{\pgfqpoint{0.655619in}{0.694738in}}%
\pgfpathlineto{\pgfqpoint{0.656721in}{0.698091in}}%
\pgfpathlineto{\pgfqpoint{0.657080in}{0.699177in}}%
\pgfpathlineto{\pgfqpoint{0.658191in}{0.703275in}}%
\pgfpathlineto{\pgfqpoint{0.658644in}{0.704361in}}%
\pgfpathlineto{\pgfqpoint{0.659746in}{0.707093in}}%
\pgfpathlineto{\pgfqpoint{0.660176in}{0.708117in}}%
\pgfpathlineto{\pgfqpoint{0.660176in}{0.708179in}}%
\pgfpathlineto{\pgfqpoint{0.661278in}{0.711377in}}%
\pgfpathlineto{\pgfqpoint{0.661599in}{0.712370in}}%
\pgfpathlineto{\pgfqpoint{0.662709in}{0.716343in}}%
\pgfpathlineto{\pgfqpoint{0.662998in}{0.717399in}}%
\pgfpathlineto{\pgfqpoint{0.664077in}{0.720751in}}%
\pgfpathlineto{\pgfqpoint{0.664546in}{0.721838in}}%
\pgfpathlineto{\pgfqpoint{0.665656in}{0.724694in}}%
\pgfpathlineto{\pgfqpoint{0.665977in}{0.725780in}}%
\pgfpathlineto{\pgfqpoint{0.667079in}{0.729226in}}%
\pgfpathlineto{\pgfqpoint{0.667556in}{0.730312in}}%
\pgfpathlineto{\pgfqpoint{0.668651in}{0.733385in}}%
\pgfpathlineto{\pgfqpoint{0.669088in}{0.734410in}}%
\pgfpathlineto{\pgfqpoint{0.669088in}{0.734441in}}%
\pgfpathlineto{\pgfqpoint{0.670198in}{0.737887in}}%
\pgfpathlineto{\pgfqpoint{0.670652in}{0.738973in}}%
\pgfpathlineto{\pgfqpoint{0.671754in}{0.741767in}}%
\pgfpathlineto{\pgfqpoint{0.672270in}{0.742822in}}%
\pgfpathlineto{\pgfqpoint{0.673372in}{0.746578in}}%
\pgfpathlineto{\pgfqpoint{0.673865in}{0.747603in}}%
\pgfpathlineto{\pgfqpoint{0.674975in}{0.751452in}}%
\pgfpathlineto{\pgfqpoint{0.675350in}{0.752538in}}%
\pgfpathlineto{\pgfqpoint{0.676453in}{0.756108in}}%
\pgfpathlineto{\pgfqpoint{0.676734in}{0.757040in}}%
\pgfpathlineto{\pgfqpoint{0.677844in}{0.760237in}}%
\pgfpathlineto{\pgfqpoint{0.678344in}{0.761261in}}%
\pgfpathlineto{\pgfqpoint{0.679415in}{0.764241in}}%
\pgfpathlineto{\pgfqpoint{0.679822in}{0.765328in}}%
\pgfpathlineto{\pgfqpoint{0.680916in}{0.768091in}}%
\pgfpathlineto{\pgfqpoint{0.681417in}{0.769177in}}%
\pgfpathlineto{\pgfqpoint{0.682488in}{0.772250in}}%
\pgfpathlineto{\pgfqpoint{0.683004in}{0.773337in}}%
\pgfpathlineto{\pgfqpoint{0.684114in}{0.776410in}}%
\pgfpathlineto{\pgfqpoint{0.684552in}{0.777496in}}%
\pgfpathlineto{\pgfqpoint{0.685662in}{0.779855in}}%
\pgfpathlineto{\pgfqpoint{0.686185in}{0.780942in}}%
\pgfpathlineto{\pgfqpoint{0.687288in}{0.784015in}}%
\pgfpathlineto{\pgfqpoint{0.687757in}{0.785071in}}%
\pgfpathlineto{\pgfqpoint{0.688867in}{0.787368in}}%
\pgfpathlineto{\pgfqpoint{0.689125in}{0.788423in}}%
\pgfpathlineto{\pgfqpoint{0.690212in}{0.791093in}}%
\pgfpathlineto{\pgfqpoint{0.690728in}{0.792179in}}%
\pgfpathlineto{\pgfqpoint{0.691830in}{0.794942in}}%
\pgfpathlineto{\pgfqpoint{0.692377in}{0.796028in}}%
\pgfpathlineto{\pgfqpoint{0.693448in}{0.799102in}}%
\pgfpathlineto{\pgfqpoint{0.694034in}{0.800188in}}%
\pgfpathlineto{\pgfqpoint{0.695137in}{0.803292in}}%
\pgfpathlineto{\pgfqpoint{0.695668in}{0.804379in}}%
\pgfpathlineto{\pgfqpoint{0.696731in}{0.806459in}}%
\pgfpathlineto{\pgfqpoint{0.697271in}{0.807545in}}%
\pgfpathlineto{\pgfqpoint{0.698365in}{0.810618in}}%
\pgfpathlineto{\pgfqpoint{0.698881in}{0.811705in}}%
\pgfpathlineto{\pgfqpoint{0.699976in}{0.814343in}}%
\pgfpathlineto{\pgfqpoint{0.700531in}{0.815430in}}%
\pgfpathlineto{\pgfqpoint{0.701618in}{0.817323in}}%
\pgfpathlineto{\pgfqpoint{0.702157in}{0.818317in}}%
\pgfpathlineto{\pgfqpoint{0.703251in}{0.820490in}}%
\pgfpathlineto{\pgfqpoint{0.703853in}{0.821545in}}%
\pgfpathlineto{\pgfqpoint{0.704963in}{0.824463in}}%
\pgfpathlineto{\pgfqpoint{0.705362in}{0.825549in}}%
\pgfpathlineto{\pgfqpoint{0.706464in}{0.828033in}}%
\pgfpathlineto{\pgfqpoint{0.707254in}{0.829057in}}%
\pgfpathlineto{\pgfqpoint{0.707254in}{0.829119in}}%
\pgfpathlineto{\pgfqpoint{0.708349in}{0.831758in}}%
\pgfpathlineto{\pgfqpoint{0.708943in}{0.832844in}}%
\pgfpathlineto{\pgfqpoint{0.710045in}{0.835483in}}%
\pgfpathlineto{\pgfqpoint{0.710569in}{0.836569in}}%
\pgfpathlineto{\pgfqpoint{0.711679in}{0.839115in}}%
\pgfpathlineto{\pgfqpoint{0.712218in}{0.840201in}}%
\pgfpathlineto{\pgfqpoint{0.713328in}{0.843306in}}%
\pgfpathlineto{\pgfqpoint{0.713930in}{0.844361in}}%
\pgfpathlineto{\pgfqpoint{0.715025in}{0.847403in}}%
\pgfpathlineto{\pgfqpoint{0.715431in}{0.848490in}}%
\pgfpathlineto{\pgfqpoint{0.716471in}{0.850880in}}%
\pgfpathlineto{\pgfqpoint{0.717034in}{0.851904in}}%
\pgfpathlineto{\pgfqpoint{0.718136in}{0.854108in}}%
\pgfpathlineto{\pgfqpoint{0.718769in}{0.855195in}}%
\pgfpathlineto{\pgfqpoint{0.719880in}{0.857740in}}%
\pgfpathlineto{\pgfqpoint{0.720333in}{0.858827in}}%
\pgfpathlineto{\pgfqpoint{0.721435in}{0.861217in}}%
\pgfpathlineto{\pgfqpoint{0.721990in}{0.862241in}}%
\pgfpathlineto{\pgfqpoint{0.723038in}{0.864135in}}%
\pgfpathlineto{\pgfqpoint{0.723749in}{0.865159in}}%
\pgfpathlineto{\pgfqpoint{0.724859in}{0.867984in}}%
\pgfpathlineto{\pgfqpoint{0.725375in}{0.869071in}}%
\pgfpathlineto{\pgfqpoint{0.726478in}{0.871181in}}%
\pgfpathlineto{\pgfqpoint{0.727080in}{0.872268in}}%
\pgfpathlineto{\pgfqpoint{0.728151in}{0.874844in}}%
\pgfpathlineto{\pgfqpoint{0.728885in}{0.875931in}}%
\pgfpathlineto{\pgfqpoint{0.729910in}{0.877980in}}%
\pgfpathlineto{\pgfqpoint{0.729956in}{0.877980in}}%
\pgfpathlineto{\pgfqpoint{0.730347in}{0.879066in}}%
\pgfpathlineto{\pgfqpoint{0.731364in}{0.881363in}}%
\pgfpathlineto{\pgfqpoint{0.731950in}{0.882450in}}%
\pgfpathlineto{\pgfqpoint{0.733060in}{0.884126in}}%
\pgfpathlineto{\pgfqpoint{0.733412in}{0.885150in}}%
\pgfpathlineto{\pgfqpoint{0.734499in}{0.887292in}}%
\pgfpathlineto{\pgfqpoint{0.735210in}{0.888379in}}%
\pgfpathlineto{\pgfqpoint{0.736312in}{0.890490in}}%
\pgfpathlineto{\pgfqpoint{0.736844in}{0.891514in}}%
\pgfpathlineto{\pgfqpoint{0.737954in}{0.893408in}}%
\pgfpathlineto{\pgfqpoint{0.738556in}{0.894494in}}%
\pgfpathlineto{\pgfqpoint{0.739643in}{0.896543in}}%
\pgfpathlineto{\pgfqpoint{0.740479in}{0.897629in}}%
\pgfpathlineto{\pgfqpoint{0.741589in}{0.900051in}}%
\pgfpathlineto{\pgfqpoint{0.742121in}{0.901137in}}%
\pgfpathlineto{\pgfqpoint{0.743215in}{0.903279in}}%
\pgfpathlineto{\pgfqpoint{0.743809in}{0.904272in}}%
\pgfpathlineto{\pgfqpoint{0.744919in}{0.906632in}}%
\pgfpathlineto{\pgfqpoint{0.745553in}{0.907718in}}%
\pgfpathlineto{\pgfqpoint{0.746647in}{0.909581in}}%
\pgfpathlineto{\pgfqpoint{0.747398in}{0.910667in}}%
\pgfpathlineto{\pgfqpoint{0.748476in}{0.912933in}}%
\pgfpathlineto{\pgfqpoint{0.749180in}{0.913989in}}%
\pgfpathlineto{\pgfqpoint{0.750282in}{0.916348in}}%
\pgfpathlineto{\pgfqpoint{0.750900in}{0.917434in}}%
\pgfpathlineto{\pgfqpoint{0.752010in}{0.919607in}}%
\pgfpathlineto{\pgfqpoint{0.752534in}{0.920694in}}%
\pgfpathlineto{\pgfqpoint{0.753628in}{0.922960in}}%
\pgfpathlineto{\pgfqpoint{0.754574in}{0.924046in}}%
\pgfpathlineto{\pgfqpoint{0.755645in}{0.926436in}}%
\pgfpathlineto{\pgfqpoint{0.756208in}{0.927523in}}%
\pgfpathlineto{\pgfqpoint{0.757279in}{0.929230in}}%
\pgfpathlineto{\pgfqpoint{0.757303in}{0.929230in}}%
\pgfpathlineto{\pgfqpoint{0.757779in}{0.930255in}}%
\pgfpathlineto{\pgfqpoint{0.758882in}{0.931651in}}%
\pgfpathlineto{\pgfqpoint{0.759476in}{0.932645in}}%
\pgfpathlineto{\pgfqpoint{0.760578in}{0.934197in}}%
\pgfpathlineto{\pgfqpoint{0.761329in}{0.935283in}}%
\pgfpathlineto{\pgfqpoint{0.762439in}{0.937146in}}%
\pgfpathlineto{\pgfqpoint{0.763009in}{0.938232in}}%
\pgfpathlineto{\pgfqpoint{0.764096in}{0.939878in}}%
\pgfpathlineto{\pgfqpoint{0.764854in}{0.940964in}}%
\pgfpathlineto{\pgfqpoint{0.765933in}{0.943044in}}%
\pgfpathlineto{\pgfqpoint{0.766512in}{0.944099in}}%
\pgfpathlineto{\pgfqpoint{0.767606in}{0.945589in}}%
\pgfpathlineto{\pgfqpoint{0.768396in}{0.946645in}}%
\pgfpathlineto{\pgfqpoint{0.769490in}{0.948663in}}%
\pgfpathlineto{\pgfqpoint{0.770092in}{0.949749in}}%
\pgfpathlineto{\pgfqpoint{0.771202in}{0.951891in}}%
\pgfpathlineto{\pgfqpoint{0.772000in}{0.952977in}}%
\pgfpathlineto{\pgfqpoint{0.773086in}{0.954654in}}%
\pgfpathlineto{\pgfqpoint{0.773939in}{0.955740in}}%
\pgfpathlineto{\pgfqpoint{0.775049in}{0.957634in}}%
\pgfpathlineto{\pgfqpoint{0.775783in}{0.958720in}}%
\pgfpathlineto{\pgfqpoint{0.776894in}{0.960148in}}%
\pgfpathlineto{\pgfqpoint{0.777636in}{0.961235in}}%
\pgfpathlineto{\pgfqpoint{0.778731in}{0.963377in}}%
\pgfpathlineto{\pgfqpoint{0.779450in}{0.964432in}}%
\pgfpathlineto{\pgfqpoint{0.780552in}{0.966201in}}%
\pgfpathlineto{\pgfqpoint{0.781217in}{0.967288in}}%
\pgfpathlineto{\pgfqpoint{0.782327in}{0.969740in}}%
\pgfpathlineto{\pgfqpoint{0.783202in}{0.970765in}}%
\pgfpathlineto{\pgfqpoint{0.784266in}{0.972813in}}%
\pgfpathlineto{\pgfqpoint{0.784930in}{0.973838in}}%
\pgfpathlineto{\pgfqpoint{0.786040in}{0.975173in}}%
\pgfpathlineto{\pgfqpoint{0.786759in}{0.976259in}}%
\pgfpathlineto{\pgfqpoint{0.787870in}{0.978153in}}%
\pgfpathlineto{\pgfqpoint{0.788612in}{0.979239in}}%
\pgfpathlineto{\pgfqpoint{0.789629in}{0.981040in}}%
\pgfpathlineto{\pgfqpoint{0.790301in}{0.982126in}}%
\pgfpathlineto{\pgfqpoint{0.791411in}{0.983989in}}%
\pgfpathlineto{\pgfqpoint{0.791896in}{0.985075in}}%
\pgfpathlineto{\pgfqpoint{0.792998in}{0.986472in}}%
\pgfpathlineto{\pgfqpoint{0.793670in}{0.987558in}}%
\pgfpathlineto{\pgfqpoint{0.794694in}{0.989297in}}%
\pgfpathlineto{\pgfqpoint{0.794741in}{0.989297in}}%
\pgfpathlineto{\pgfqpoint{0.795726in}{0.990383in}}%
\pgfpathlineto{\pgfqpoint{0.796829in}{0.992525in}}%
\pgfpathlineto{\pgfqpoint{0.797767in}{0.993581in}}%
\pgfpathlineto{\pgfqpoint{0.798861in}{0.995102in}}%
\pgfpathlineto{\pgfqpoint{0.799580in}{0.996188in}}%
\pgfpathlineto{\pgfqpoint{0.800636in}{0.997554in}}%
\pgfpathlineto{\pgfqpoint{0.801371in}{0.998640in}}%
\pgfpathlineto{\pgfqpoint{0.802481in}{1.000379in}}%
\pgfpathlineto{\pgfqpoint{0.803216in}{1.001465in}}%
\pgfpathlineto{\pgfqpoint{0.804326in}{1.002552in}}%
\pgfpathlineto{\pgfqpoint{0.805084in}{1.003607in}}%
\pgfpathlineto{\pgfqpoint{0.806186in}{1.005035in}}%
\pgfpathlineto{\pgfqpoint{0.807015in}{1.006122in}}%
\pgfpathlineto{\pgfqpoint{0.808109in}{1.007767in}}%
\pgfpathlineto{\pgfqpoint{0.808938in}{1.008853in}}%
\pgfpathlineto{\pgfqpoint{0.810040in}{1.010561in}}%
\pgfpathlineto{\pgfqpoint{0.811072in}{1.011647in}}%
\pgfpathlineto{\pgfqpoint{0.812151in}{1.013541in}}%
\pgfpathlineto{\pgfqpoint{0.813089in}{1.014627in}}%
\pgfpathlineto{\pgfqpoint{0.814176in}{1.015869in}}%
\pgfpathlineto{\pgfqpoint{0.815145in}{1.016955in}}%
\pgfpathlineto{\pgfqpoint{0.816248in}{1.018569in}}%
\pgfpathlineto{\pgfqpoint{0.816936in}{1.019625in}}%
\pgfpathlineto{\pgfqpoint{0.818030in}{1.021177in}}%
\pgfpathlineto{\pgfqpoint{0.818781in}{1.022263in}}%
\pgfpathlineto{\pgfqpoint{0.819891in}{1.023722in}}%
\pgfpathlineto{\pgfqpoint{0.820868in}{1.024809in}}%
\pgfpathlineto{\pgfqpoint{0.821962in}{1.026020in}}%
\pgfpathlineto{\pgfqpoint{0.822736in}{1.027075in}}%
\pgfpathlineto{\pgfqpoint{0.823823in}{1.028441in}}%
\pgfpathlineto{\pgfqpoint{0.824902in}{1.029496in}}%
\pgfpathlineto{\pgfqpoint{0.825957in}{1.031110in}}%
\pgfpathlineto{\pgfqpoint{0.826692in}{1.032197in}}%
\pgfpathlineto{\pgfqpoint{0.827763in}{1.033252in}}%
\pgfpathlineto{\pgfqpoint{0.828599in}{1.034308in}}%
\pgfpathlineto{\pgfqpoint{0.829710in}{1.035643in}}%
\pgfpathlineto{\pgfqpoint{0.830601in}{1.036698in}}%
\pgfpathlineto{\pgfqpoint{0.831695in}{1.038561in}}%
\pgfpathlineto{\pgfqpoint{0.832837in}{1.039647in}}%
\pgfpathlineto{\pgfqpoint{0.833947in}{1.040982in}}%
\pgfpathlineto{\pgfqpoint{0.834721in}{1.042037in}}%
\pgfpathlineto{\pgfqpoint{0.835792in}{1.043589in}}%
\pgfpathlineto{\pgfqpoint{0.836605in}{1.044645in}}%
\pgfpathlineto{\pgfqpoint{0.837676in}{1.045918in}}%
\pgfpathlineto{\pgfqpoint{0.838786in}{1.047004in}}%
\pgfpathlineto{\pgfqpoint{0.839880in}{1.048370in}}%
\pgfpathlineto{\pgfqpoint{0.840623in}{1.049456in}}%
\pgfpathlineto{\pgfqpoint{0.841733in}{1.050915in}}%
\pgfpathlineto{\pgfqpoint{0.842742in}{1.052002in}}%
\pgfpathlineto{\pgfqpoint{0.843836in}{1.053212in}}%
\pgfpathlineto{\pgfqpoint{0.844837in}{1.054268in}}%
\pgfpathlineto{\pgfqpoint{0.845900in}{1.055199in}}%
\pgfpathlineto{\pgfqpoint{0.847104in}{1.056286in}}%
\pgfpathlineto{\pgfqpoint{0.848206in}{1.057372in}}%
\pgfpathlineto{\pgfqpoint{0.849504in}{1.058459in}}%
\pgfpathlineto{\pgfqpoint{0.850614in}{1.060228in}}%
\pgfpathlineto{\pgfqpoint{0.851787in}{1.061190in}}%
\pgfpathlineto{\pgfqpoint{0.851787in}{1.061283in}}%
\pgfpathlineto{\pgfqpoint{0.852881in}{1.062680in}}%
\pgfpathlineto{\pgfqpoint{0.854007in}{1.063767in}}%
\pgfpathlineto{\pgfqpoint{0.855101in}{1.065195in}}%
\pgfpathlineto{\pgfqpoint{0.855805in}{1.066281in}}%
\pgfpathlineto{\pgfqpoint{0.856907in}{1.067523in}}%
\pgfpathlineto{\pgfqpoint{0.856915in}{1.067523in}}%
\pgfpathlineto{\pgfqpoint{0.857642in}{1.068609in}}%
\pgfpathlineto{\pgfqpoint{0.858713in}{1.069851in}}%
\pgfpathlineto{\pgfqpoint{0.859745in}{1.070938in}}%
\pgfpathlineto{\pgfqpoint{0.860855in}{1.072179in}}%
\pgfpathlineto{\pgfqpoint{0.861410in}{1.073204in}}%
\pgfpathlineto{\pgfqpoint{0.862505in}{1.074600in}}%
\pgfpathlineto{\pgfqpoint{0.863662in}{1.075687in}}%
\pgfpathlineto{\pgfqpoint{0.864756in}{1.077115in}}%
\pgfpathlineto{\pgfqpoint{0.865522in}{1.078201in}}%
\pgfpathlineto{\pgfqpoint{0.866624in}{1.079536in}}%
\pgfpathlineto{\pgfqpoint{0.867907in}{1.080592in}}%
\pgfpathlineto{\pgfqpoint{0.869017in}{1.082268in}}%
\pgfpathlineto{\pgfqpoint{0.870002in}{1.083354in}}%
\pgfpathlineto{\pgfqpoint{0.871096in}{1.084689in}}%
\pgfpathlineto{\pgfqpoint{0.872425in}{1.085776in}}%
\pgfpathlineto{\pgfqpoint{0.873535in}{1.086955in}}%
\pgfpathlineto{\pgfqpoint{0.874098in}{1.088042in}}%
\pgfpathlineto{\pgfqpoint{0.875193in}{1.089345in}}%
\pgfpathlineto{\pgfqpoint{0.876193in}{1.090432in}}%
\pgfpathlineto{\pgfqpoint{0.877288in}{1.091953in}}%
\pgfpathlineto{\pgfqpoint{0.878085in}{1.093040in}}%
\pgfpathlineto{\pgfqpoint{0.879172in}{1.094654in}}%
\pgfpathlineto{\pgfqpoint{0.879946in}{1.095740in}}%
\pgfpathlineto{\pgfqpoint{0.880993in}{1.097075in}}%
\pgfpathlineto{\pgfqpoint{0.882307in}{1.098130in}}%
\pgfpathlineto{\pgfqpoint{0.883417in}{1.099403in}}%
\pgfpathlineto{\pgfqpoint{0.884300in}{1.100490in}}%
\pgfpathlineto{\pgfqpoint{0.885356in}{1.101731in}}%
\pgfpathlineto{\pgfqpoint{0.886630in}{1.102818in}}%
\pgfpathlineto{\pgfqpoint{0.887740in}{1.103904in}}%
\pgfpathlineto{\pgfqpoint{0.888748in}{1.104991in}}%
\pgfpathlineto{\pgfqpoint{0.889812in}{1.106326in}}%
\pgfpathlineto{\pgfqpoint{0.889843in}{1.106326in}}%
\pgfpathlineto{\pgfqpoint{0.891086in}{1.107412in}}%
\pgfpathlineto{\pgfqpoint{0.892165in}{1.108871in}}%
\pgfpathlineto{\pgfqpoint{0.893158in}{1.109957in}}%
\pgfpathlineto{\pgfqpoint{0.894260in}{1.111447in}}%
\pgfpathlineto{\pgfqpoint{0.895253in}{1.112503in}}%
\pgfpathlineto{\pgfqpoint{0.896324in}{1.113651in}}%
\pgfpathlineto{\pgfqpoint{0.897207in}{1.114707in}}%
\pgfpathlineto{\pgfqpoint{0.898302in}{1.116321in}}%
\pgfpathlineto{\pgfqpoint{0.899560in}{1.117377in}}%
\pgfpathlineto{\pgfqpoint{0.900670in}{1.118370in}}%
\pgfpathlineto{\pgfqpoint{0.901702in}{1.119456in}}%
\pgfpathlineto{\pgfqpoint{0.902797in}{1.120481in}}%
\pgfpathlineto{\pgfqpoint{0.903774in}{1.121567in}}%
\pgfpathlineto{\pgfqpoint{0.904829in}{1.122561in}}%
\pgfpathlineto{\pgfqpoint{0.905744in}{1.123616in}}%
\pgfpathlineto{\pgfqpoint{0.906776in}{1.124889in}}%
\pgfpathlineto{\pgfqpoint{0.908089in}{1.125975in}}%
\pgfpathlineto{\pgfqpoint{0.909199in}{1.127341in}}%
\pgfpathlineto{\pgfqpoint{0.910349in}{1.128428in}}%
\pgfpathlineto{\pgfqpoint{0.911451in}{1.129700in}}%
\pgfpathlineto{\pgfqpoint{0.912467in}{1.130787in}}%
\pgfpathlineto{\pgfqpoint{0.913507in}{1.131687in}}%
\pgfpathlineto{\pgfqpoint{0.913538in}{1.131687in}}%
\pgfpathlineto{\pgfqpoint{0.914468in}{1.132773in}}%
\pgfpathlineto{\pgfqpoint{0.915579in}{1.134170in}}%
\pgfpathlineto{\pgfqpoint{0.916861in}{1.135257in}}%
\pgfpathlineto{\pgfqpoint{0.917971in}{1.136623in}}%
\pgfpathlineto{\pgfqpoint{0.918995in}{1.137709in}}%
\pgfpathlineto{\pgfqpoint{0.920058in}{1.139168in}}%
\pgfpathlineto{\pgfqpoint{0.921622in}{1.140255in}}%
\pgfpathlineto{\pgfqpoint{0.922716in}{1.141434in}}%
\pgfpathlineto{\pgfqpoint{0.924108in}{1.142521in}}%
\pgfpathlineto{\pgfqpoint{0.925179in}{1.143855in}}%
\pgfpathlineto{\pgfqpoint{0.926218in}{1.144911in}}%
\pgfpathlineto{\pgfqpoint{0.927305in}{1.146122in}}%
\pgfpathlineto{\pgfqpoint{0.928548in}{1.147208in}}%
\pgfpathlineto{\pgfqpoint{0.929650in}{1.148294in}}%
\pgfpathlineto{\pgfqpoint{0.930925in}{1.149319in}}%
\pgfpathlineto{\pgfqpoint{0.932003in}{1.150467in}}%
\pgfpathlineto{\pgfqpoint{0.933880in}{1.151554in}}%
\pgfpathlineto{\pgfqpoint{0.934959in}{1.152796in}}%
\pgfpathlineto{\pgfqpoint{0.935842in}{1.153882in}}%
\pgfpathlineto{\pgfqpoint{0.936803in}{1.155217in}}%
\pgfpathlineto{\pgfqpoint{0.938273in}{1.156303in}}%
\pgfpathlineto{\pgfqpoint{0.939368in}{1.157390in}}%
\pgfpathlineto{\pgfqpoint{0.940572in}{1.158414in}}%
\pgfpathlineto{\pgfqpoint{0.941682in}{1.159780in}}%
\pgfpathlineto{\pgfqpoint{0.942893in}{1.160867in}}%
\pgfpathlineto{\pgfqpoint{0.943980in}{1.162046in}}%
\pgfpathlineto{\pgfqpoint{0.945387in}{1.163133in}}%
\pgfpathlineto{\pgfqpoint{0.946497in}{1.164281in}}%
\pgfpathlineto{\pgfqpoint{0.948170in}{1.165368in}}%
\pgfpathlineto{\pgfqpoint{0.949249in}{1.165957in}}%
\pgfpathlineto{\pgfqpoint{0.949280in}{1.165957in}}%
\pgfpathlineto{\pgfqpoint{0.950015in}{1.167044in}}%
\pgfpathlineto{\pgfqpoint{0.951008in}{1.167882in}}%
\pgfpathlineto{\pgfqpoint{0.952368in}{1.168969in}}%
\pgfpathlineto{\pgfqpoint{0.953416in}{1.170024in}}%
\pgfpathlineto{\pgfqpoint{0.954651in}{1.171110in}}%
\pgfpathlineto{\pgfqpoint{0.955753in}{1.172725in}}%
\pgfpathlineto{\pgfqpoint{0.957247in}{1.173811in}}%
\pgfpathlineto{\pgfqpoint{0.958310in}{1.175146in}}%
\pgfpathlineto{\pgfqpoint{0.959811in}{1.176201in}}%
\pgfpathlineto{\pgfqpoint{0.960897in}{1.177567in}}%
\pgfpathlineto{\pgfqpoint{0.961976in}{1.178654in}}%
\pgfpathlineto{\pgfqpoint{0.963063in}{1.179740in}}%
\pgfpathlineto{\pgfqpoint{0.964408in}{1.180827in}}%
\pgfpathlineto{\pgfqpoint{0.965486in}{1.181913in}}%
\pgfpathlineto{\pgfqpoint{0.966385in}{1.183000in}}%
\pgfpathlineto{\pgfqpoint{0.967464in}{1.184241in}}%
\pgfpathlineto{\pgfqpoint{0.968879in}{1.185297in}}%
\pgfpathlineto{\pgfqpoint{0.969989in}{1.186476in}}%
\pgfpathlineto{\pgfqpoint{0.971600in}{1.187563in}}%
\pgfpathlineto{\pgfqpoint{0.972632in}{1.188463in}}%
\pgfpathlineto{\pgfqpoint{0.972655in}{1.188463in}}%
\pgfpathlineto{\pgfqpoint{0.973812in}{1.189549in}}%
\pgfpathlineto{\pgfqpoint{0.974914in}{1.190853in}}%
\pgfpathlineto{\pgfqpoint{0.975884in}{1.191909in}}%
\pgfpathlineto{\pgfqpoint{0.976971in}{1.192778in}}%
\pgfpathlineto{\pgfqpoint{0.978386in}{1.193833in}}%
\pgfpathlineto{\pgfqpoint{0.979433in}{1.194827in}}%
\pgfpathlineto{\pgfqpoint{0.980457in}{1.195913in}}%
\pgfpathlineto{\pgfqpoint{0.981442in}{1.196813in}}%
\pgfpathlineto{\pgfqpoint{0.982888in}{1.197900in}}%
\pgfpathlineto{\pgfqpoint{0.983952in}{1.198893in}}%
\pgfpathlineto{\pgfqpoint{0.985750in}{1.199949in}}%
\pgfpathlineto{\pgfqpoint{0.986836in}{1.201190in}}%
\pgfpathlineto{\pgfqpoint{0.988017in}{1.202277in}}%
\pgfpathlineto{\pgfqpoint{0.989127in}{1.203612in}}%
\pgfpathlineto{\pgfqpoint{0.990268in}{1.204636in}}%
\pgfpathlineto{\pgfqpoint{0.991371in}{1.205785in}}%
\pgfpathlineto{\pgfqpoint{0.992614in}{1.206871in}}%
\pgfpathlineto{\pgfqpoint{0.993661in}{1.207802in}}%
\pgfpathlineto{\pgfqpoint{0.994482in}{1.208889in}}%
\pgfpathlineto{\pgfqpoint{0.995530in}{1.209727in}}%
\pgfpathlineto{\pgfqpoint{0.996851in}{1.210813in}}%
\pgfpathlineto{\pgfqpoint{0.997961in}{1.211838in}}%
\pgfpathlineto{\pgfqpoint{0.999079in}{1.212924in}}%
\pgfpathlineto{\pgfqpoint{1.000189in}{1.214445in}}%
\pgfpathlineto{\pgfqpoint{1.001463in}{1.215501in}}%
\pgfpathlineto{\pgfqpoint{1.002573in}{1.216494in}}%
\pgfpathlineto{\pgfqpoint{1.004082in}{1.217581in}}%
\pgfpathlineto{\pgfqpoint{1.005153in}{1.218481in}}%
\pgfpathlineto{\pgfqpoint{1.006474in}{1.219567in}}%
\pgfpathlineto{\pgfqpoint{1.007545in}{1.220902in}}%
\pgfpathlineto{\pgfqpoint{1.008890in}{1.221989in}}%
\pgfpathlineto{\pgfqpoint{1.009953in}{1.223044in}}%
\pgfpathlineto{\pgfqpoint{1.009992in}{1.223044in}}%
\pgfpathlineto{\pgfqpoint{1.011095in}{1.224130in}}%
\pgfpathlineto{\pgfqpoint{1.012205in}{1.224782in}}%
\pgfpathlineto{\pgfqpoint{1.013463in}{1.225869in}}%
\pgfpathlineto{\pgfqpoint{1.014472in}{1.226893in}}%
\pgfpathlineto{\pgfqpoint{1.016137in}{1.227980in}}%
\pgfpathlineto{\pgfqpoint{1.017177in}{1.228911in}}%
\pgfpathlineto{\pgfqpoint{1.018678in}{1.229997in}}%
\pgfpathlineto{\pgfqpoint{1.019756in}{1.230742in}}%
\pgfpathlineto{\pgfqpoint{1.020953in}{1.231829in}}%
\pgfpathlineto{\pgfqpoint{1.022039in}{1.232977in}}%
\pgfpathlineto{\pgfqpoint{1.023173in}{1.234064in}}%
\pgfpathlineto{\pgfqpoint{1.024283in}{1.235150in}}%
\pgfpathlineto{\pgfqpoint{1.025518in}{1.236237in}}%
\pgfpathlineto{\pgfqpoint{1.026613in}{1.237075in}}%
\pgfpathlineto{\pgfqpoint{1.028043in}{1.238161in}}%
\pgfpathlineto{\pgfqpoint{1.029106in}{1.239403in}}%
\pgfpathlineto{\pgfqpoint{1.030490in}{1.240490in}}%
\pgfpathlineto{\pgfqpoint{1.031428in}{1.241204in}}%
\pgfpathlineto{\pgfqpoint{1.031452in}{1.241204in}}%
\pgfpathlineto{\pgfqpoint{1.032718in}{1.242290in}}%
\pgfpathlineto{\pgfqpoint{1.033789in}{1.243159in}}%
\pgfpathlineto{\pgfqpoint{1.035384in}{1.244246in}}%
\pgfpathlineto{\pgfqpoint{1.036494in}{1.245177in}}%
\pgfpathlineto{\pgfqpoint{1.037682in}{1.246263in}}%
\pgfpathlineto{\pgfqpoint{1.038769in}{1.247474in}}%
\pgfpathlineto{\pgfqpoint{1.039973in}{1.248561in}}%
\pgfpathlineto{\pgfqpoint{1.041044in}{1.249244in}}%
\pgfpathlineto{\pgfqpoint{1.042482in}{1.250330in}}%
\pgfpathlineto{\pgfqpoint{1.043592in}{1.251354in}}%
\pgfpathlineto{\pgfqpoint{1.045133in}{1.252348in}}%
\pgfpathlineto{\pgfqpoint{1.045133in}{1.252410in}}%
\pgfpathlineto{\pgfqpoint{1.046227in}{1.253279in}}%
\pgfpathlineto{\pgfqpoint{1.048009in}{1.254365in}}%
\pgfpathlineto{\pgfqpoint{1.049104in}{1.255328in}}%
\pgfpathlineto{\pgfqpoint{1.050128in}{1.256414in}}%
\pgfpathlineto{\pgfqpoint{1.051222in}{1.257470in}}%
\pgfpathlineto{\pgfqpoint{1.052903in}{1.258556in}}%
\pgfpathlineto{\pgfqpoint{1.053990in}{1.259643in}}%
\pgfpathlineto{\pgfqpoint{1.055295in}{1.260729in}}%
\pgfpathlineto{\pgfqpoint{1.056398in}{1.261660in}}%
\pgfpathlineto{\pgfqpoint{1.057946in}{1.262747in}}%
\pgfpathlineto{\pgfqpoint{1.058970in}{1.263740in}}%
\pgfpathlineto{\pgfqpoint{1.060869in}{1.264827in}}%
\pgfpathlineto{\pgfqpoint{1.061901in}{1.265416in}}%
\pgfpathlineto{\pgfqpoint{1.063395in}{1.266503in}}%
\pgfpathlineto{\pgfqpoint{1.064473in}{1.267403in}}%
\pgfpathlineto{\pgfqpoint{1.065818in}{1.268459in}}%
\pgfpathlineto{\pgfqpoint{1.066811in}{1.269110in}}%
\pgfpathlineto{\pgfqpoint{1.068077in}{1.270197in}}%
\pgfpathlineto{\pgfqpoint{1.069117in}{1.271035in}}%
\pgfpathlineto{\pgfqpoint{1.069187in}{1.271035in}}%
\pgfpathlineto{\pgfqpoint{1.070595in}{1.272122in}}%
\pgfpathlineto{\pgfqpoint{1.071681in}{1.273084in}}%
\pgfpathlineto{\pgfqpoint{1.073456in}{1.274170in}}%
\pgfpathlineto{\pgfqpoint{1.074527in}{1.274915in}}%
\pgfpathlineto{\pgfqpoint{1.075887in}{1.275971in}}%
\pgfpathlineto{\pgfqpoint{1.076794in}{1.276467in}}%
\pgfpathlineto{\pgfqpoint{1.078623in}{1.277554in}}%
\pgfpathlineto{\pgfqpoint{1.079710in}{1.278423in}}%
\pgfpathlineto{\pgfqpoint{1.081625in}{1.279510in}}%
\pgfpathlineto{\pgfqpoint{1.082665in}{1.280534in}}%
\pgfpathlineto{\pgfqpoint{1.084690in}{1.281620in}}%
\pgfpathlineto{\pgfqpoint{1.085745in}{1.282272in}}%
\pgfpathlineto{\pgfqpoint{1.087285in}{1.283359in}}%
\pgfpathlineto{\pgfqpoint{1.088270in}{1.284011in}}%
\pgfpathlineto{\pgfqpoint{1.089849in}{1.285066in}}%
\pgfpathlineto{\pgfqpoint{1.090897in}{1.286059in}}%
\pgfpathlineto{\pgfqpoint{1.092101in}{1.287146in}}%
\pgfpathlineto{\pgfqpoint{1.093195in}{1.288077in}}%
\pgfpathlineto{\pgfqpoint{1.094814in}{1.289164in}}%
\pgfpathlineto{\pgfqpoint{1.095916in}{1.290188in}}%
\pgfpathlineto{\pgfqpoint{1.097362in}{1.291275in}}%
\pgfpathlineto{\pgfqpoint{1.098371in}{1.291895in}}%
\pgfpathlineto{\pgfqpoint{1.099809in}{1.292951in}}%
\pgfpathlineto{\pgfqpoint{1.100919in}{1.293727in}}%
\pgfpathlineto{\pgfqpoint{1.102154in}{1.294813in}}%
\pgfpathlineto{\pgfqpoint{1.103155in}{1.295527in}}%
\pgfpathlineto{\pgfqpoint{1.104977in}{1.296614in}}%
\pgfpathlineto{\pgfqpoint{1.106055in}{1.297390in}}%
\pgfpathlineto{\pgfqpoint{1.107291in}{1.298476in}}%
\pgfpathlineto{\pgfqpoint{1.108401in}{1.299252in}}%
\pgfpathlineto{\pgfqpoint{1.110050in}{1.300339in}}%
\pgfpathlineto{\pgfqpoint{1.111090in}{1.301146in}}%
\pgfpathlineto{\pgfqpoint{1.112661in}{1.302201in}}%
\pgfpathlineto{\pgfqpoint{1.113771in}{1.303288in}}%
\pgfpathlineto{\pgfqpoint{1.115038in}{1.304374in}}%
\pgfpathlineto{\pgfqpoint{1.116148in}{1.304995in}}%
\pgfpathlineto{\pgfqpoint{1.117876in}{1.306082in}}%
\pgfpathlineto{\pgfqpoint{1.118947in}{1.306547in}}%
\pgfpathlineto{\pgfqpoint{1.120682in}{1.307634in}}%
\pgfpathlineto{\pgfqpoint{1.121769in}{1.308379in}}%
\pgfpathlineto{\pgfqpoint{1.123536in}{1.309465in}}%
\pgfpathlineto{\pgfqpoint{1.124591in}{1.310210in}}%
\pgfpathlineto{\pgfqpoint{1.125998in}{1.311297in}}%
\pgfpathlineto{\pgfqpoint{1.127108in}{1.312352in}}%
\pgfpathlineto{\pgfqpoint{1.128547in}{1.313439in}}%
\pgfpathlineto{\pgfqpoint{1.129602in}{1.314028in}}%
\pgfpathlineto{\pgfqpoint{1.131197in}{1.315084in}}%
\pgfpathlineto{\pgfqpoint{1.131197in}{1.315115in}}%
\pgfpathlineto{\pgfqpoint{1.132291in}{1.316046in}}%
\pgfpathlineto{\pgfqpoint{1.134074in}{1.317133in}}%
\pgfpathlineto{\pgfqpoint{1.135161in}{1.317567in}}%
\pgfpathlineto{\pgfqpoint{1.137005in}{1.318654in}}%
\pgfpathlineto{\pgfqpoint{1.137990in}{1.319461in}}%
\pgfpathlineto{\pgfqpoint{1.140258in}{1.320547in}}%
\pgfpathlineto{\pgfqpoint{1.141250in}{1.321106in}}%
\pgfpathlineto{\pgfqpoint{1.143127in}{1.322193in}}%
\pgfpathlineto{\pgfqpoint{1.144221in}{1.322938in}}%
\pgfpathlineto{\pgfqpoint{1.146207in}{1.324024in}}%
\pgfpathlineto{\pgfqpoint{1.147254in}{1.324986in}}%
\pgfpathlineto{\pgfqpoint{1.149076in}{1.326073in}}%
\pgfpathlineto{\pgfqpoint{1.150178in}{1.326694in}}%
\pgfpathlineto{\pgfqpoint{1.151593in}{1.327780in}}%
\pgfpathlineto{\pgfqpoint{1.152594in}{1.328711in}}%
\pgfpathlineto{\pgfqpoint{1.154650in}{1.329798in}}%
\pgfpathlineto{\pgfqpoint{1.155760in}{1.330481in}}%
\pgfpathlineto{\pgfqpoint{1.157542in}{1.331567in}}%
\pgfpathlineto{\pgfqpoint{1.158621in}{1.332281in}}%
\pgfpathlineto{\pgfqpoint{1.160607in}{1.333368in}}%
\pgfpathlineto{\pgfqpoint{1.161670in}{1.334051in}}%
\pgfpathlineto{\pgfqpoint{1.163140in}{1.335137in}}%
\pgfpathlineto{\pgfqpoint{1.164234in}{1.335882in}}%
\pgfpathlineto{\pgfqpoint{1.165524in}{1.336969in}}%
\pgfpathlineto{\pgfqpoint{1.166634in}{1.337962in}}%
\pgfpathlineto{\pgfqpoint{1.168417in}{1.339017in}}%
\pgfpathlineto{\pgfqpoint{1.169511in}{1.339824in}}%
\pgfpathlineto{\pgfqpoint{1.172193in}{1.340911in}}%
\pgfpathlineto{\pgfqpoint{1.173154in}{1.341718in}}%
\pgfpathlineto{\pgfqpoint{1.175070in}{1.342773in}}%
\pgfpathlineto{\pgfqpoint{1.176062in}{1.343705in}}%
\pgfpathlineto{\pgfqpoint{1.177665in}{1.344791in}}%
\pgfpathlineto{\pgfqpoint{1.178736in}{1.345474in}}%
\pgfpathlineto{\pgfqpoint{1.180354in}{1.346530in}}%
\pgfpathlineto{\pgfqpoint{1.181308in}{1.347492in}}%
\pgfpathlineto{\pgfqpoint{1.182801in}{1.348578in}}%
\pgfpathlineto{\pgfqpoint{1.183708in}{1.349044in}}%
\pgfpathlineto{\pgfqpoint{1.185365in}{1.350130in}}%
\pgfpathlineto{\pgfqpoint{1.186421in}{1.351000in}}%
\pgfpathlineto{\pgfqpoint{1.187851in}{1.352086in}}%
\pgfpathlineto{\pgfqpoint{1.188860in}{1.352893in}}%
\pgfpathlineto{\pgfqpoint{1.188899in}{1.352893in}}%
\pgfpathlineto{\pgfqpoint{1.190416in}{1.353980in}}%
\pgfpathlineto{\pgfqpoint{1.191315in}{1.354507in}}%
\pgfpathlineto{\pgfqpoint{1.191494in}{1.354507in}}%
\pgfpathlineto{\pgfqpoint{1.192917in}{1.355594in}}%
\pgfpathlineto{\pgfqpoint{1.194020in}{1.356339in}}%
\pgfpathlineto{\pgfqpoint{1.195646in}{1.357425in}}%
\pgfpathlineto{\pgfqpoint{1.196724in}{1.358046in}}%
\pgfpathlineto{\pgfqpoint{1.198710in}{1.359133in}}%
\pgfpathlineto{\pgfqpoint{1.199765in}{1.359847in}}%
\pgfpathlineto{\pgfqpoint{1.201822in}{1.360933in}}%
\pgfpathlineto{\pgfqpoint{1.202861in}{1.361647in}}%
\pgfpathlineto{\pgfqpoint{1.204222in}{1.362734in}}%
\pgfpathlineto{\pgfqpoint{1.205277in}{1.363354in}}%
\pgfpathlineto{\pgfqpoint{1.206934in}{1.364441in}}%
\pgfpathlineto{\pgfqpoint{1.208005in}{1.365000in}}%
\pgfpathlineto{\pgfqpoint{1.209960in}{1.366086in}}%
\pgfpathlineto{\pgfqpoint{1.211039in}{1.366831in}}%
\pgfpathlineto{\pgfqpoint{1.213423in}{1.367918in}}%
\pgfpathlineto{\pgfqpoint{1.214510in}{1.368663in}}%
\pgfpathlineto{\pgfqpoint{1.215885in}{1.369749in}}%
\pgfpathlineto{\pgfqpoint{1.216988in}{1.370494in}}%
\pgfpathlineto{\pgfqpoint{1.218856in}{1.371581in}}%
\pgfpathlineto{\pgfqpoint{1.219833in}{1.372263in}}%
\pgfpathlineto{\pgfqpoint{1.219919in}{1.372263in}}%
\pgfpathlineto{\pgfqpoint{1.221577in}{1.373319in}}%
\pgfpathlineto{\pgfqpoint{1.222640in}{1.373816in}}%
\pgfpathlineto{\pgfqpoint{1.224633in}{1.374902in}}%
\pgfpathlineto{\pgfqpoint{1.225728in}{1.375740in}}%
\pgfpathlineto{\pgfqpoint{1.227159in}{1.376827in}}%
\pgfpathlineto{\pgfqpoint{1.228245in}{1.377727in}}%
\pgfpathlineto{\pgfqpoint{1.230372in}{1.378813in}}%
\pgfpathlineto{\pgfqpoint{1.231482in}{1.379496in}}%
\pgfpathlineto{\pgfqpoint{1.232865in}{1.380583in}}%
\pgfpathlineto{\pgfqpoint{1.233952in}{1.381079in}}%
\pgfpathlineto{\pgfqpoint{1.236759in}{1.382104in}}%
\pgfpathlineto{\pgfqpoint{1.237845in}{1.382756in}}%
\pgfpathlineto{\pgfqpoint{1.239745in}{1.383842in}}%
\pgfpathlineto{\pgfqpoint{1.240808in}{1.384432in}}%
\pgfpathlineto{\pgfqpoint{1.240839in}{1.384432in}}%
\pgfpathlineto{\pgfqpoint{1.242411in}{1.385518in}}%
\pgfpathlineto{\pgfqpoint{1.243388in}{1.385953in}}%
\pgfpathlineto{\pgfqpoint{1.245741in}{1.387008in}}%
\pgfpathlineto{\pgfqpoint{1.246851in}{1.387816in}}%
\pgfpathlineto{\pgfqpoint{1.248923in}{1.388902in}}%
\pgfpathlineto{\pgfqpoint{1.249931in}{1.389368in}}%
\pgfpathlineto{\pgfqpoint{1.251909in}{1.390454in}}%
\pgfpathlineto{\pgfqpoint{1.253011in}{1.390951in}}%
\pgfpathlineto{\pgfqpoint{1.255239in}{1.392037in}}%
\pgfpathlineto{\pgfqpoint{1.256303in}{1.392596in}}%
\pgfpathlineto{\pgfqpoint{1.258515in}{1.393683in}}%
\pgfpathlineto{\pgfqpoint{1.259492in}{1.394490in}}%
\pgfpathlineto{\pgfqpoint{1.261705in}{1.395576in}}%
\pgfpathlineto{\pgfqpoint{1.262713in}{1.396135in}}%
\pgfpathlineto{\pgfqpoint{1.264191in}{1.397221in}}%
\pgfpathlineto{\pgfqpoint{1.265230in}{1.397780in}}%
\pgfpathlineto{\pgfqpoint{1.267685in}{1.398867in}}%
\pgfpathlineto{\pgfqpoint{1.268764in}{1.399518in}}%
\pgfpathlineto{\pgfqpoint{1.270328in}{1.400605in}}%
\pgfpathlineto{\pgfqpoint{1.271422in}{1.401319in}}%
\pgfpathlineto{\pgfqpoint{1.274119in}{1.402405in}}%
\pgfpathlineto{\pgfqpoint{1.275190in}{1.403150in}}%
\pgfpathlineto{\pgfqpoint{1.278106in}{1.404237in}}%
\pgfpathlineto{\pgfqpoint{1.279216in}{1.404889in}}%
\pgfpathlineto{\pgfqpoint{1.281397in}{1.405975in}}%
\pgfpathlineto{\pgfqpoint{1.282507in}{1.406534in}}%
\pgfpathlineto{\pgfqpoint{1.284947in}{1.407620in}}%
\pgfpathlineto{\pgfqpoint{1.285994in}{1.408396in}}%
\pgfpathlineto{\pgfqpoint{1.286041in}{1.408396in}}%
\pgfpathlineto{\pgfqpoint{1.288629in}{1.409483in}}%
\pgfpathlineto{\pgfqpoint{1.289614in}{1.410011in}}%
\pgfpathlineto{\pgfqpoint{1.289645in}{1.410011in}}%
\pgfpathlineto{\pgfqpoint{1.292076in}{1.411097in}}%
\pgfpathlineto{\pgfqpoint{1.293147in}{1.411718in}}%
\pgfpathlineto{\pgfqpoint{1.293186in}{1.411718in}}%
\pgfpathlineto{\pgfqpoint{1.295195in}{1.412804in}}%
\pgfpathlineto{\pgfqpoint{1.296462in}{1.413612in}}%
\pgfpathlineto{\pgfqpoint{1.298526in}{1.414667in}}%
\pgfpathlineto{\pgfqpoint{1.299636in}{1.415443in}}%
\pgfpathlineto{\pgfqpoint{1.300871in}{1.416530in}}%
\pgfpathlineto{\pgfqpoint{1.301958in}{1.416840in}}%
\pgfpathlineto{\pgfqpoint{1.303896in}{1.417864in}}%
\pgfpathlineto{\pgfqpoint{1.304968in}{1.418485in}}%
\pgfpathlineto{\pgfqpoint{1.306859in}{1.419572in}}%
\pgfpathlineto{\pgfqpoint{1.307907in}{1.420099in}}%
\pgfpathlineto{\pgfqpoint{1.310096in}{1.421186in}}%
\pgfpathlineto{\pgfqpoint{1.311143in}{1.421838in}}%
\pgfpathlineto{\pgfqpoint{1.314286in}{1.422924in}}%
\pgfpathlineto{\pgfqpoint{1.315381in}{1.423545in}}%
\pgfpathlineto{\pgfqpoint{1.318000in}{1.424600in}}%
\pgfpathlineto{\pgfqpoint{1.319071in}{1.425097in}}%
\pgfpathlineto{\pgfqpoint{1.320650in}{1.426153in}}%
\pgfpathlineto{\pgfqpoint{1.321643in}{1.426898in}}%
\pgfpathlineto{\pgfqpoint{1.324418in}{1.427984in}}%
\pgfpathlineto{\pgfqpoint{1.325528in}{1.428450in}}%
\pgfpathlineto{\pgfqpoint{1.327482in}{1.429536in}}%
\pgfpathlineto{\pgfqpoint{1.328452in}{1.430250in}}%
\pgfpathlineto{\pgfqpoint{1.331039in}{1.431337in}}%
\pgfpathlineto{\pgfqpoint{1.332134in}{1.431926in}}%
\pgfpathlineto{\pgfqpoint{1.334033in}{1.433013in}}%
\pgfpathlineto{\pgfqpoint{1.335026in}{1.433603in}}%
\pgfpathlineto{\pgfqpoint{1.337317in}{1.434689in}}%
\pgfpathlineto{\pgfqpoint{1.338052in}{1.435279in}}%
\pgfpathlineto{\pgfqpoint{1.338364in}{1.435279in}}%
\pgfpathlineto{\pgfqpoint{1.340507in}{1.436365in}}%
\pgfpathlineto{\pgfqpoint{1.341593in}{1.436831in}}%
\pgfpathlineto{\pgfqpoint{1.344126in}{1.437918in}}%
\pgfpathlineto{\pgfqpoint{1.345228in}{1.438445in}}%
\pgfpathlineto{\pgfqpoint{1.347793in}{1.439532in}}%
\pgfpathlineto{\pgfqpoint{1.348809in}{1.439935in}}%
\pgfpathlineto{\pgfqpoint{1.351967in}{1.441022in}}%
\pgfpathlineto{\pgfqpoint{1.352929in}{1.441674in}}%
\pgfpathlineto{\pgfqpoint{1.355837in}{1.442760in}}%
\pgfpathlineto{\pgfqpoint{1.356924in}{1.443257in}}%
\pgfpathlineto{\pgfqpoint{1.359339in}{1.444343in}}%
\pgfpathlineto{\pgfqpoint{1.360223in}{1.444685in}}%
\pgfpathlineto{\pgfqpoint{1.362849in}{1.445771in}}%
\pgfpathlineto{\pgfqpoint{1.363959in}{1.446392in}}%
\pgfpathlineto{\pgfqpoint{1.366289in}{1.447447in}}%
\pgfpathlineto{\pgfqpoint{1.367337in}{1.447913in}}%
\pgfpathlineto{\pgfqpoint{1.370370in}{1.449000in}}%
\pgfpathlineto{\pgfqpoint{1.371425in}{1.449620in}}%
\pgfpathlineto{\pgfqpoint{1.374654in}{1.450707in}}%
\pgfpathlineto{\pgfqpoint{1.375764in}{1.451266in}}%
\pgfpathlineto{\pgfqpoint{1.378563in}{1.452352in}}%
\pgfpathlineto{\pgfqpoint{1.379610in}{1.452725in}}%
\pgfpathlineto{\pgfqpoint{1.381956in}{1.453811in}}%
\pgfpathlineto{\pgfqpoint{1.382691in}{1.454184in}}%
\pgfpathlineto{\pgfqpoint{1.382948in}{1.454184in}}%
\pgfpathlineto{\pgfqpoint{1.385028in}{1.455270in}}%
\pgfpathlineto{\pgfqpoint{1.386083in}{1.455798in}}%
\pgfpathlineto{\pgfqpoint{1.386099in}{1.455798in}}%
\pgfpathlineto{\pgfqpoint{1.388296in}{1.456884in}}%
\pgfpathlineto{\pgfqpoint{1.389382in}{1.457381in}}%
\pgfpathlineto{\pgfqpoint{1.392635in}{1.458467in}}%
\pgfpathlineto{\pgfqpoint{1.393745in}{1.458995in}}%
\pgfpathlineto{\pgfqpoint{1.395941in}{1.460082in}}%
\pgfpathlineto{\pgfqpoint{1.396919in}{1.460640in}}%
\pgfpathlineto{\pgfqpoint{1.399655in}{1.461696in}}%
\pgfpathlineto{\pgfqpoint{1.400718in}{1.462224in}}%
\pgfpathlineto{\pgfqpoint{1.400765in}{1.462224in}}%
\pgfpathlineto{\pgfqpoint{1.403571in}{1.463310in}}%
\pgfpathlineto{\pgfqpoint{1.404541in}{1.463838in}}%
\pgfpathlineto{\pgfqpoint{1.406996in}{1.464924in}}%
\pgfpathlineto{\pgfqpoint{1.408090in}{1.465483in}}%
\pgfpathlineto{\pgfqpoint{1.411467in}{1.466569in}}%
\pgfpathlineto{\pgfqpoint{1.412382in}{1.466942in}}%
\pgfpathlineto{\pgfqpoint{1.416650in}{1.468028in}}%
\pgfpathlineto{\pgfqpoint{1.417542in}{1.468308in}}%
\pgfpathlineto{\pgfqpoint{1.417674in}{1.468308in}}%
\pgfpathlineto{\pgfqpoint{1.420286in}{1.469394in}}%
\pgfpathlineto{\pgfqpoint{1.421357in}{1.469829in}}%
\pgfpathlineto{\pgfqpoint{1.425187in}{1.470915in}}%
\pgfpathlineto{\pgfqpoint{1.426227in}{1.471536in}}%
\pgfpathlineto{\pgfqpoint{1.429205in}{1.472623in}}%
\pgfpathlineto{\pgfqpoint{1.430214in}{1.473212in}}%
\pgfpathlineto{\pgfqpoint{1.432778in}{1.474299in}}%
\pgfpathlineto{\pgfqpoint{1.433794in}{1.475013in}}%
\pgfpathlineto{\pgfqpoint{1.433865in}{1.475013in}}%
\pgfpathlineto{\pgfqpoint{1.436820in}{1.476099in}}%
\pgfpathlineto{\pgfqpoint{1.437922in}{1.476658in}}%
\pgfpathlineto{\pgfqpoint{1.440791in}{1.477745in}}%
\pgfpathlineto{\pgfqpoint{1.441886in}{1.478303in}}%
\pgfpathlineto{\pgfqpoint{1.444552in}{1.479390in}}%
\pgfpathlineto{\pgfqpoint{1.445466in}{1.479949in}}%
\pgfpathlineto{\pgfqpoint{1.448890in}{1.481035in}}%
\pgfpathlineto{\pgfqpoint{1.449914in}{1.481532in}}%
\pgfpathlineto{\pgfqpoint{1.449993in}{1.481532in}}%
\pgfpathlineto{\pgfqpoint{1.453315in}{1.482587in}}%
\pgfpathlineto{\pgfqpoint{1.454355in}{1.483053in}}%
\pgfpathlineto{\pgfqpoint{1.457412in}{1.484139in}}%
\pgfpathlineto{\pgfqpoint{1.458631in}{1.484791in}}%
\pgfpathlineto{\pgfqpoint{1.461750in}{1.485878in}}%
\pgfpathlineto{\pgfqpoint{1.462782in}{1.486281in}}%
\pgfpathlineto{\pgfqpoint{1.466144in}{1.487368in}}%
\pgfpathlineto{\pgfqpoint{1.467113in}{1.488082in}}%
\pgfpathlineto{\pgfqpoint{1.467238in}{1.488082in}}%
\pgfpathlineto{\pgfqpoint{1.469834in}{1.489168in}}%
\pgfpathlineto{\pgfqpoint{1.470811in}{1.489820in}}%
\pgfpathlineto{\pgfqpoint{1.470842in}{1.489820in}}%
\pgfpathlineto{\pgfqpoint{1.474438in}{1.490906in}}%
\pgfpathlineto{\pgfqpoint{1.475525in}{1.491403in}}%
\pgfpathlineto{\pgfqpoint{1.478543in}{1.492490in}}%
\pgfpathlineto{\pgfqpoint{1.479489in}{1.493173in}}%
\pgfpathlineto{\pgfqpoint{1.479528in}{1.493173in}}%
\pgfpathlineto{\pgfqpoint{1.482920in}{1.494259in}}%
\pgfpathlineto{\pgfqpoint{1.484023in}{1.494663in}}%
\pgfpathlineto{\pgfqpoint{1.486579in}{1.495749in}}%
\pgfpathlineto{\pgfqpoint{1.487439in}{1.496028in}}%
\pgfpathlineto{\pgfqpoint{1.490808in}{1.497115in}}%
\pgfpathlineto{\pgfqpoint{1.491770in}{1.497332in}}%
\pgfpathlineto{\pgfqpoint{1.491911in}{1.497332in}}%
\pgfpathlineto{\pgfqpoint{1.496101in}{1.498419in}}%
\pgfpathlineto{\pgfqpoint{1.497188in}{1.498977in}}%
\pgfpathlineto{\pgfqpoint{1.499893in}{1.500064in}}%
\pgfpathlineto{\pgfqpoint{1.500940in}{1.500530in}}%
\pgfpathlineto{\pgfqpoint{1.503864in}{1.501616in}}%
\pgfpathlineto{\pgfqpoint{1.504919in}{1.502237in}}%
\pgfpathlineto{\pgfqpoint{1.507679in}{1.503292in}}%
\pgfpathlineto{\pgfqpoint{1.508758in}{1.503820in}}%
\pgfpathlineto{\pgfqpoint{1.513042in}{1.504906in}}%
\pgfpathlineto{\pgfqpoint{1.514097in}{1.505217in}}%
\pgfpathlineto{\pgfqpoint{1.514144in}{1.505217in}}%
\pgfpathlineto{\pgfqpoint{1.516802in}{1.506303in}}%
\pgfpathlineto{\pgfqpoint{1.517865in}{1.506924in}}%
\pgfpathlineto{\pgfqpoint{1.517912in}{1.506924in}}%
\pgfpathlineto{\pgfqpoint{1.520985in}{1.508011in}}%
\pgfpathlineto{\pgfqpoint{1.521931in}{1.508383in}}%
\pgfpathlineto{\pgfqpoint{1.525183in}{1.509470in}}%
\pgfpathlineto{\pgfqpoint{1.526285in}{1.509935in}}%
\pgfpathlineto{\pgfqpoint{1.530233in}{1.511022in}}%
\pgfpathlineto{\pgfqpoint{1.530819in}{1.511332in}}%
\pgfpathlineto{\pgfqpoint{1.531304in}{1.511332in}}%
\pgfpathlineto{\pgfqpoint{1.536104in}{1.512419in}}%
\pgfpathlineto{\pgfqpoint{1.537105in}{1.512636in}}%
\pgfpathlineto{\pgfqpoint{1.540458in}{1.513722in}}%
\pgfpathlineto{\pgfqpoint{1.541529in}{1.514312in}}%
\pgfpathlineto{\pgfqpoint{1.545086in}{1.515368in}}%
\pgfpathlineto{\pgfqpoint{1.546079in}{1.515895in}}%
\pgfpathlineto{\pgfqpoint{1.551724in}{1.516982in}}%
\pgfpathlineto{\pgfqpoint{1.552709in}{1.517354in}}%
\pgfpathlineto{\pgfqpoint{1.552771in}{1.517354in}}%
\pgfpathlineto{\pgfqpoint{1.556219in}{1.518441in}}%
\pgfpathlineto{\pgfqpoint{1.557212in}{1.518751in}}%
\pgfpathlineto{\pgfqpoint{1.557329in}{1.518751in}}%
\pgfpathlineto{\pgfqpoint{1.560753in}{1.519838in}}%
\pgfpathlineto{\pgfqpoint{1.561808in}{1.520241in}}%
\pgfpathlineto{\pgfqpoint{1.565694in}{1.521328in}}%
\pgfpathlineto{\pgfqpoint{1.566632in}{1.521700in}}%
\pgfpathlineto{\pgfqpoint{1.570572in}{1.522787in}}%
\pgfpathlineto{\pgfqpoint{1.571510in}{1.523252in}}%
\pgfpathlineto{\pgfqpoint{1.574778in}{1.524339in}}%
\pgfpathlineto{\pgfqpoint{1.575849in}{1.524773in}}%
\pgfpathlineto{\pgfqpoint{1.578436in}{1.525860in}}%
\pgfpathlineto{\pgfqpoint{1.579515in}{1.526232in}}%
\pgfpathlineto{\pgfqpoint{1.579539in}{1.526232in}}%
\pgfpathlineto{\pgfqpoint{1.585035in}{1.527319in}}%
\pgfpathlineto{\pgfqpoint{1.586051in}{1.527816in}}%
\pgfpathlineto{\pgfqpoint{1.589334in}{1.528902in}}%
\pgfpathlineto{\pgfqpoint{1.590366in}{1.529275in}}%
\pgfpathlineto{\pgfqpoint{1.590397in}{1.529275in}}%
\pgfpathlineto{\pgfqpoint{1.595393in}{1.530361in}}%
\pgfpathlineto{\pgfqpoint{1.596503in}{1.530734in}}%
\pgfpathlineto{\pgfqpoint{1.601303in}{1.531820in}}%
\pgfpathlineto{\pgfqpoint{1.602358in}{1.532193in}}%
\pgfpathlineto{\pgfqpoint{1.605611in}{1.533279in}}%
\pgfpathlineto{\pgfqpoint{1.606322in}{1.533496in}}%
\pgfpathlineto{\pgfqpoint{1.610903in}{1.534583in}}%
\pgfpathlineto{\pgfqpoint{1.611896in}{1.534893in}}%
\pgfpathlineto{\pgfqpoint{1.615523in}{1.535980in}}%
\pgfpathlineto{\pgfqpoint{1.616407in}{1.536166in}}%
\pgfpathlineto{\pgfqpoint{1.619862in}{1.537252in}}%
\pgfpathlineto{\pgfqpoint{1.620957in}{1.537718in}}%
\pgfpathlineto{\pgfqpoint{1.624897in}{1.538804in}}%
\pgfpathlineto{\pgfqpoint{1.625905in}{1.539084in}}%
\pgfpathlineto{\pgfqpoint{1.629791in}{1.540170in}}%
\pgfpathlineto{\pgfqpoint{1.630830in}{1.540667in}}%
\pgfpathlineto{\pgfqpoint{1.634809in}{1.541753in}}%
\pgfpathlineto{\pgfqpoint{1.635904in}{1.541971in}}%
\pgfpathlineto{\pgfqpoint{1.639438in}{1.543026in}}%
\pgfpathlineto{\pgfqpoint{1.640430in}{1.543554in}}%
\pgfpathlineto{\pgfqpoint{1.644574in}{1.544640in}}%
\pgfpathlineto{\pgfqpoint{1.645684in}{1.544920in}}%
\pgfpathlineto{\pgfqpoint{1.650695in}{1.546006in}}%
\pgfpathlineto{\pgfqpoint{1.651719in}{1.546348in}}%
\pgfpathlineto{\pgfqpoint{1.651766in}{1.546348in}}%
\pgfpathlineto{\pgfqpoint{1.656605in}{1.547434in}}%
\pgfpathlineto{\pgfqpoint{1.657707in}{1.547838in}}%
\pgfpathlineto{\pgfqpoint{1.661624in}{1.548924in}}%
\pgfpathlineto{\pgfqpoint{1.662460in}{1.549235in}}%
\pgfpathlineto{\pgfqpoint{1.666682in}{1.550321in}}%
\pgfpathlineto{\pgfqpoint{1.667683in}{1.550507in}}%
\pgfpathlineto{\pgfqpoint{1.671435in}{1.551594in}}%
\pgfpathlineto{\pgfqpoint{1.672389in}{1.551749in}}%
\pgfpathlineto{\pgfqpoint{1.677588in}{1.552836in}}%
\pgfpathlineto{\pgfqpoint{1.678690in}{1.553115in}}%
\pgfpathlineto{\pgfqpoint{1.683951in}{1.554201in}}%
\pgfpathlineto{\pgfqpoint{1.684913in}{1.554450in}}%
\pgfpathlineto{\pgfqpoint{1.689009in}{1.555536in}}%
\pgfpathlineto{\pgfqpoint{1.690018in}{1.555785in}}%
\pgfpathlineto{\pgfqpoint{1.694372in}{1.556871in}}%
\pgfpathlineto{\pgfqpoint{1.695459in}{1.557088in}}%
\pgfpathlineto{\pgfqpoint{1.700556in}{1.558175in}}%
\pgfpathlineto{\pgfqpoint{1.701611in}{1.558485in}}%
\pgfpathlineto{\pgfqpoint{1.707834in}{1.559572in}}%
\pgfpathlineto{\pgfqpoint{1.708866in}{1.559820in}}%
\pgfpathlineto{\pgfqpoint{1.708897in}{1.559820in}}%
\pgfpathlineto{\pgfqpoint{1.713869in}{1.560906in}}%
\pgfpathlineto{\pgfqpoint{1.714893in}{1.561186in}}%
\pgfpathlineto{\pgfqpoint{1.719256in}{1.562272in}}%
\pgfpathlineto{\pgfqpoint{1.720327in}{1.562552in}}%
\pgfpathlineto{\pgfqpoint{1.724681in}{1.563638in}}%
\pgfpathlineto{\pgfqpoint{1.725580in}{1.563731in}}%
\pgfpathlineto{\pgfqpoint{1.732092in}{1.564818in}}%
\pgfpathlineto{\pgfqpoint{1.733030in}{1.565190in}}%
\pgfpathlineto{\pgfqpoint{1.739073in}{1.566277in}}%
\pgfpathlineto{\pgfqpoint{1.739793in}{1.566556in}}%
\pgfpathlineto{\pgfqpoint{1.744702in}{1.567643in}}%
\pgfpathlineto{\pgfqpoint{1.745711in}{1.567829in}}%
\pgfpathlineto{\pgfqpoint{1.750964in}{1.568915in}}%
\pgfpathlineto{\pgfqpoint{1.752066in}{1.569195in}}%
\pgfpathlineto{\pgfqpoint{1.758117in}{1.570281in}}%
\pgfpathlineto{\pgfqpoint{1.759157in}{1.570685in}}%
\pgfpathlineto{\pgfqpoint{1.764723in}{1.571771in}}%
\pgfpathlineto{\pgfqpoint{1.765732in}{1.571957in}}%
\pgfpathlineto{\pgfqpoint{1.765825in}{1.571957in}}%
\pgfpathlineto{\pgfqpoint{1.772736in}{1.573044in}}%
\pgfpathlineto{\pgfqpoint{1.773846in}{1.573199in}}%
\pgfpathlineto{\pgfqpoint{1.780992in}{1.574286in}}%
\pgfpathlineto{\pgfqpoint{1.782102in}{1.574534in}}%
\pgfpathlineto{\pgfqpoint{1.789208in}{1.575620in}}%
\pgfpathlineto{\pgfqpoint{1.789990in}{1.575807in}}%
\pgfpathlineto{\pgfqpoint{1.790318in}{1.575807in}}%
\pgfpathlineto{\pgfqpoint{1.796009in}{1.576862in}}%
\pgfpathlineto{\pgfqpoint{1.796744in}{1.577079in}}%
\pgfpathlineto{\pgfqpoint{1.804476in}{1.578166in}}%
\pgfpathlineto{\pgfqpoint{1.805547in}{1.578290in}}%
\pgfpathlineto{\pgfqpoint{1.811731in}{1.579377in}}%
\pgfpathlineto{\pgfqpoint{1.812168in}{1.579501in}}%
\pgfpathlineto{\pgfqpoint{1.812770in}{1.579501in}}%
\pgfpathlineto{\pgfqpoint{1.819533in}{1.580587in}}%
\pgfpathlineto{\pgfqpoint{1.820181in}{1.580804in}}%
\pgfpathlineto{\pgfqpoint{1.820385in}{1.580804in}}%
\pgfpathlineto{\pgfqpoint{1.830610in}{1.581891in}}%
\pgfpathlineto{\pgfqpoint{1.831666in}{1.582046in}}%
\pgfpathlineto{\pgfqpoint{1.838959in}{1.583133in}}%
\pgfpathlineto{\pgfqpoint{1.839835in}{1.583381in}}%
\pgfpathlineto{\pgfqpoint{1.847348in}{1.584467in}}%
\pgfpathlineto{\pgfqpoint{1.848247in}{1.584623in}}%
\pgfpathlineto{\pgfqpoint{1.855025in}{1.585709in}}%
\pgfpathlineto{\pgfqpoint{1.855861in}{1.585957in}}%
\pgfpathlineto{\pgfqpoint{1.862311in}{1.587013in}}%
\pgfpathlineto{\pgfqpoint{1.863366in}{1.587354in}}%
\pgfpathlineto{\pgfqpoint{1.863405in}{1.587354in}}%
\pgfpathlineto{\pgfqpoint{1.873685in}{1.588441in}}%
\pgfpathlineto{\pgfqpoint{1.874577in}{1.588658in}}%
\pgfpathlineto{\pgfqpoint{1.880682in}{1.589714in}}%
\pgfpathlineto{\pgfqpoint{1.881151in}{1.589869in}}%
\pgfpathlineto{\pgfqpoint{1.881730in}{1.589869in}}%
\pgfpathlineto{\pgfqpoint{1.893081in}{1.590955in}}%
\pgfpathlineto{\pgfqpoint{1.893464in}{1.591048in}}%
\pgfpathlineto{\pgfqpoint{1.893597in}{1.591048in}}%
\pgfpathlineto{\pgfqpoint{1.903205in}{1.592135in}}%
\pgfpathlineto{\pgfqpoint{1.904245in}{1.592321in}}%
\pgfpathlineto{\pgfqpoint{1.912414in}{1.593408in}}%
\pgfpathlineto{\pgfqpoint{1.913368in}{1.593501in}}%
\pgfpathlineto{\pgfqpoint{1.922264in}{1.594587in}}%
\pgfpathlineto{\pgfqpoint{1.922827in}{1.594742in}}%
\pgfpathlineto{\pgfqpoint{1.923108in}{1.594742in}}%
\pgfpathlineto{\pgfqpoint{1.935820in}{1.595829in}}%
\pgfpathlineto{\pgfqpoint{1.936641in}{1.595953in}}%
\pgfpathlineto{\pgfqpoint{1.936836in}{1.595953in}}%
\pgfpathlineto{\pgfqpoint{1.950845in}{1.597040in}}%
\pgfpathlineto{\pgfqpoint{1.951346in}{1.597226in}}%
\pgfpathlineto{\pgfqpoint{1.951666in}{1.597226in}}%
\pgfpathlineto{\pgfqpoint{1.968521in}{1.598312in}}%
\pgfpathlineto{\pgfqpoint{1.968990in}{1.598374in}}%
\pgfpathlineto{\pgfqpoint{1.969272in}{1.598374in}}%
\pgfpathlineto{\pgfqpoint{1.982585in}{1.599461in}}%
\pgfpathlineto{\pgfqpoint{1.982585in}{1.599492in}}%
\pgfpathlineto{\pgfqpoint{1.983203in}{1.599492in}}%
\pgfpathlineto{\pgfqpoint{2.001293in}{1.600578in}}%
\pgfpathlineto{\pgfqpoint{2.002168in}{1.600671in}}%
\pgfpathlineto{\pgfqpoint{2.002278in}{1.600671in}}%
\pgfpathlineto{\pgfqpoint{2.028592in}{1.601758in}}%
\pgfpathlineto{\pgfqpoint{2.028592in}{1.601789in}}%
\pgfpathlineto{\pgfqpoint{2.029514in}{1.601789in}}%
\pgfpathlineto{\pgfqpoint{2.033126in}{1.601944in}}%
\pgfpathlineto{\pgfqpoint{2.033126in}{1.601944in}}%
\pgfusepath{stroke}%
\end{pgfscope}%
\begin{pgfscope}%
\pgfsetrectcap%
\pgfsetmiterjoin%
\pgfsetlinewidth{0.803000pt}%
\definecolor{currentstroke}{rgb}{0.000000,0.000000,0.000000}%
\pgfsetstrokecolor{currentstroke}%
\pgfsetdash{}{0pt}%
\pgfpathmoveto{\pgfqpoint{0.553581in}{0.499444in}}%
\pgfpathlineto{\pgfqpoint{0.553581in}{1.654444in}}%
\pgfusepath{stroke}%
\end{pgfscope}%
\begin{pgfscope}%
\pgfsetrectcap%
\pgfsetmiterjoin%
\pgfsetlinewidth{0.803000pt}%
\definecolor{currentstroke}{rgb}{0.000000,0.000000,0.000000}%
\pgfsetstrokecolor{currentstroke}%
\pgfsetdash{}{0pt}%
\pgfpathmoveto{\pgfqpoint{2.103581in}{0.499444in}}%
\pgfpathlineto{\pgfqpoint{2.103581in}{1.654444in}}%
\pgfusepath{stroke}%
\end{pgfscope}%
\begin{pgfscope}%
\pgfsetrectcap%
\pgfsetmiterjoin%
\pgfsetlinewidth{0.803000pt}%
\definecolor{currentstroke}{rgb}{0.000000,0.000000,0.000000}%
\pgfsetstrokecolor{currentstroke}%
\pgfsetdash{}{0pt}%
\pgfpathmoveto{\pgfqpoint{0.553581in}{0.499444in}}%
\pgfpathlineto{\pgfqpoint{2.103581in}{0.499444in}}%
\pgfusepath{stroke}%
\end{pgfscope}%
\begin{pgfscope}%
\pgfsetrectcap%
\pgfsetmiterjoin%
\pgfsetlinewidth{0.803000pt}%
\definecolor{currentstroke}{rgb}{0.000000,0.000000,0.000000}%
\pgfsetstrokecolor{currentstroke}%
\pgfsetdash{}{0pt}%
\pgfpathmoveto{\pgfqpoint{0.553581in}{1.654444in}}%
\pgfpathlineto{\pgfqpoint{2.103581in}{1.654444in}}%
\pgfusepath{stroke}%
\end{pgfscope}%
\begin{pgfscope}%
\pgfsetbuttcap%
\pgfsetmiterjoin%
\definecolor{currentfill}{rgb}{1.000000,1.000000,1.000000}%
\pgfsetfillcolor{currentfill}%
\pgfsetfillopacity{0.800000}%
\pgfsetlinewidth{1.003750pt}%
\definecolor{currentstroke}{rgb}{0.800000,0.800000,0.800000}%
\pgfsetstrokecolor{currentstroke}%
\pgfsetstrokeopacity{0.800000}%
\pgfsetdash{}{0pt}%
\pgfpathmoveto{\pgfqpoint{0.832747in}{0.568889in}}%
\pgfpathlineto{\pgfqpoint{2.006358in}{0.568889in}}%
\pgfpathquadraticcurveto{\pgfqpoint{2.034136in}{0.568889in}}{\pgfqpoint{2.034136in}{0.596666in}}%
\pgfpathlineto{\pgfqpoint{2.034136in}{0.776388in}}%
\pgfpathquadraticcurveto{\pgfqpoint{2.034136in}{0.804166in}}{\pgfqpoint{2.006358in}{0.804166in}}%
\pgfpathlineto{\pgfqpoint{0.832747in}{0.804166in}}%
\pgfpathquadraticcurveto{\pgfqpoint{0.804970in}{0.804166in}}{\pgfqpoint{0.804970in}{0.776388in}}%
\pgfpathlineto{\pgfqpoint{0.804970in}{0.596666in}}%
\pgfpathquadraticcurveto{\pgfqpoint{0.804970in}{0.568889in}}{\pgfqpoint{0.832747in}{0.568889in}}%
\pgfpathlineto{\pgfqpoint{0.832747in}{0.568889in}}%
\pgfpathclose%
\pgfusepath{stroke,fill}%
\end{pgfscope}%
\begin{pgfscope}%
\pgfsetrectcap%
\pgfsetroundjoin%
\pgfsetlinewidth{1.505625pt}%
\definecolor{currentstroke}{rgb}{0.000000,0.000000,0.000000}%
\pgfsetstrokecolor{currentstroke}%
\pgfsetdash{}{0pt}%
\pgfpathmoveto{\pgfqpoint{0.860525in}{0.700000in}}%
\pgfpathlineto{\pgfqpoint{0.999414in}{0.700000in}}%
\pgfpathlineto{\pgfqpoint{1.138303in}{0.700000in}}%
\pgfusepath{stroke}%
\end{pgfscope}%
\begin{pgfscope}%
\definecolor{textcolor}{rgb}{0.000000,0.000000,0.000000}%
\pgfsetstrokecolor{textcolor}%
\pgfsetfillcolor{textcolor}%
\pgftext[x=1.249414in,y=0.651388in,left,base]{\color{textcolor}\rmfamily\fontsize{10.000000}{12.000000}\selectfont AUC=0.752}%
\end{pgfscope}%
\end{pgfpicture}%
\makeatother%
\endgroup%

\end{tabular}


\

To make a useful visualization of the results where we can see the interplay between the negative and positive classes, we can transform the data.  A transformation that preserves rank will have no effect on the ROC curve.  [Cite]  For the graph below, we mapped the smallest value in the set to 0 and the largest to 1.  

%
\noindent\begin{tabular}{@{\hspace{-6pt}}p{4.3in} @{\hspace{-6pt}}p{2.0in}}
	\vskip 0pt
	\hfil Raw Model Output
	
	\input{../Keras/Images/AdaBoost_Hard_Tomek_0_v1_Test_Transformed_100_Pred_Wide.pgf}	
&
	\vskip 0pt
	\hfil ROC Curve
	
	\input{../Keras/Images/AdaBoost_Hard_Tomek_0_v1_Test_Transformed_100_ROC.pgf}
\end{tabular}

The distribution has long tails, so we can make a more useful visualization by truncating the ends.  For this graph we mapped the 0.01 quantile to 0 and the 0.99 quantile to 1 leaving the center 98\% of the distribution and truncated the ends.  Our goal in clipping the tails is to make all of the models' results have approximately the same granularity when we choose the decision thresholds that give us the (politically) desired results.  

%
\noindent\begin{tabular}{@{\hspace{-6pt}}p{4.3in} @{\hspace{-6pt}}p{2.0in}}
	\vskip 0pt
	\hfil Raw Model Output
	
	\input{../Keras/Images/AdaBoost_Hard_Tomek_0_v1_Test_Transformed_98_Pred_Wide.pgf}	
&
	\vskip 0pt
	\hfil ROC Curve
	
	\input{../Keras/Images/AdaBoost_Hard_Tomek_0_v1_Test_Transformed_98_ROC.pgf}
\end{tabular}

\

%
\verb|Bagging_Hard_Tomek_0_v1_Test|

\

This model returned 217 different values, but most of them were rare.  Taking out the 5\% of the data set with the least frequent values, 95\% of the samples had only 10 values of $p$.  It may be a useful model, but we will not be able to fine tune the decision threshold.  

\noindent\begin{tabular}{@{\hspace{-6pt}}p{4.3in} @{\hspace{-6pt}}p{2.0in}}
	\vskip 0pt
	\hfil Raw Model Output
	
	%% Creator: Matplotlib, PGF backend
%%
%% To include the figure in your LaTeX document, write
%%   \input{<filename>.pgf}
%%
%% Make sure the required packages are loaded in your preamble
%%   \usepackage{pgf}
%%
%% Also ensure that all the required font packages are loaded; for instance,
%% the lmodern package is sometimes necessary when using math font.
%%   \usepackage{lmodern}
%%
%% Figures using additional raster images can only be included by \input if
%% they are in the same directory as the main LaTeX file. For loading figures
%% from other directories you can use the `import` package
%%   \usepackage{import}
%%
%% and then include the figures with
%%   \import{<path to file>}{<filename>.pgf}
%%
%% Matplotlib used the following preamble
%%   
%%   \usepackage{fontspec}
%%   \makeatletter\@ifpackageloaded{underscore}{}{\usepackage[strings]{underscore}}\makeatother
%%
\begingroup%
\makeatletter%
\begin{pgfpicture}%
\pgfpathrectangle{\pgfpointorigin}{\pgfqpoint{4.578750in}{1.754444in}}%
\pgfusepath{use as bounding box, clip}%
\begin{pgfscope}%
\pgfsetbuttcap%
\pgfsetmiterjoin%
\definecolor{currentfill}{rgb}{1.000000,1.000000,1.000000}%
\pgfsetfillcolor{currentfill}%
\pgfsetlinewidth{0.000000pt}%
\definecolor{currentstroke}{rgb}{1.000000,1.000000,1.000000}%
\pgfsetstrokecolor{currentstroke}%
\pgfsetdash{}{0pt}%
\pgfpathmoveto{\pgfqpoint{0.000000in}{0.000000in}}%
\pgfpathlineto{\pgfqpoint{4.578750in}{0.000000in}}%
\pgfpathlineto{\pgfqpoint{4.578750in}{1.754444in}}%
\pgfpathlineto{\pgfqpoint{0.000000in}{1.754444in}}%
\pgfpathlineto{\pgfqpoint{0.000000in}{0.000000in}}%
\pgfpathclose%
\pgfusepath{fill}%
\end{pgfscope}%
\begin{pgfscope}%
\pgfsetbuttcap%
\pgfsetmiterjoin%
\definecolor{currentfill}{rgb}{1.000000,1.000000,1.000000}%
\pgfsetfillcolor{currentfill}%
\pgfsetlinewidth{0.000000pt}%
\definecolor{currentstroke}{rgb}{0.000000,0.000000,0.000000}%
\pgfsetstrokecolor{currentstroke}%
\pgfsetstrokeopacity{0.000000}%
\pgfsetdash{}{0pt}%
\pgfpathmoveto{\pgfqpoint{0.515000in}{0.499444in}}%
\pgfpathlineto{\pgfqpoint{4.390000in}{0.499444in}}%
\pgfpathlineto{\pgfqpoint{4.390000in}{1.654444in}}%
\pgfpathlineto{\pgfqpoint{0.515000in}{1.654444in}}%
\pgfpathlineto{\pgfqpoint{0.515000in}{0.499444in}}%
\pgfpathclose%
\pgfusepath{fill}%
\end{pgfscope}%
\begin{pgfscope}%
\pgfpathrectangle{\pgfqpoint{0.515000in}{0.499444in}}{\pgfqpoint{3.875000in}{1.155000in}}%
\pgfusepath{clip}%
\pgfsetbuttcap%
\pgfsetmiterjoin%
\pgfsetlinewidth{1.003750pt}%
\definecolor{currentstroke}{rgb}{0.000000,0.000000,0.000000}%
\pgfsetstrokecolor{currentstroke}%
\pgfsetdash{}{0pt}%
\pgfpathmoveto{\pgfqpoint{0.505000in}{0.499444in}}%
\pgfpathlineto{\pgfqpoint{0.553367in}{0.499444in}}%
\pgfpathlineto{\pgfqpoint{0.553367in}{0.499779in}}%
\pgfpathlineto{\pgfqpoint{0.505000in}{0.499779in}}%
\pgfusepath{stroke}%
\end{pgfscope}%
\begin{pgfscope}%
\pgfpathrectangle{\pgfqpoint{0.515000in}{0.499444in}}{\pgfqpoint{3.875000in}{1.155000in}}%
\pgfusepath{clip}%
\pgfsetbuttcap%
\pgfsetmiterjoin%
\pgfsetlinewidth{1.003750pt}%
\definecolor{currentstroke}{rgb}{0.000000,0.000000,0.000000}%
\pgfsetstrokecolor{currentstroke}%
\pgfsetdash{}{0pt}%
\pgfpathmoveto{\pgfqpoint{0.645446in}{0.499444in}}%
\pgfpathlineto{\pgfqpoint{0.706832in}{0.499444in}}%
\pgfpathlineto{\pgfqpoint{0.706832in}{0.502938in}}%
\pgfpathlineto{\pgfqpoint{0.645446in}{0.502938in}}%
\pgfpathlineto{\pgfqpoint{0.645446in}{0.499444in}}%
\pgfpathclose%
\pgfusepath{stroke}%
\end{pgfscope}%
\begin{pgfscope}%
\pgfpathrectangle{\pgfqpoint{0.515000in}{0.499444in}}{\pgfqpoint{3.875000in}{1.155000in}}%
\pgfusepath{clip}%
\pgfsetbuttcap%
\pgfsetmiterjoin%
\pgfsetlinewidth{1.003750pt}%
\definecolor{currentstroke}{rgb}{0.000000,0.000000,0.000000}%
\pgfsetstrokecolor{currentstroke}%
\pgfsetdash{}{0pt}%
\pgfpathmoveto{\pgfqpoint{0.798911in}{0.499444in}}%
\pgfpathlineto{\pgfqpoint{0.860297in}{0.499444in}}%
\pgfpathlineto{\pgfqpoint{0.860297in}{1.573461in}}%
\pgfpathlineto{\pgfqpoint{0.798911in}{1.573461in}}%
\pgfpathlineto{\pgfqpoint{0.798911in}{0.499444in}}%
\pgfpathclose%
\pgfusepath{stroke}%
\end{pgfscope}%
\begin{pgfscope}%
\pgfpathrectangle{\pgfqpoint{0.515000in}{0.499444in}}{\pgfqpoint{3.875000in}{1.155000in}}%
\pgfusepath{clip}%
\pgfsetbuttcap%
\pgfsetmiterjoin%
\pgfsetlinewidth{1.003750pt}%
\definecolor{currentstroke}{rgb}{0.000000,0.000000,0.000000}%
\pgfsetstrokecolor{currentstroke}%
\pgfsetdash{}{0pt}%
\pgfpathmoveto{\pgfqpoint{0.952377in}{0.499444in}}%
\pgfpathlineto{\pgfqpoint{1.013763in}{0.499444in}}%
\pgfpathlineto{\pgfqpoint{1.013763in}{0.503905in}}%
\pgfpathlineto{\pgfqpoint{0.952377in}{0.503905in}}%
\pgfpathlineto{\pgfqpoint{0.952377in}{0.499444in}}%
\pgfpathclose%
\pgfusepath{stroke}%
\end{pgfscope}%
\begin{pgfscope}%
\pgfpathrectangle{\pgfqpoint{0.515000in}{0.499444in}}{\pgfqpoint{3.875000in}{1.155000in}}%
\pgfusepath{clip}%
\pgfsetbuttcap%
\pgfsetmiterjoin%
\pgfsetlinewidth{1.003750pt}%
\definecolor{currentstroke}{rgb}{0.000000,0.000000,0.000000}%
\pgfsetstrokecolor{currentstroke}%
\pgfsetdash{}{0pt}%
\pgfpathmoveto{\pgfqpoint{1.105842in}{0.499444in}}%
\pgfpathlineto{\pgfqpoint{1.167228in}{0.499444in}}%
\pgfpathlineto{\pgfqpoint{1.167228in}{1.599444in}}%
\pgfpathlineto{\pgfqpoint{1.105842in}{1.599444in}}%
\pgfpathlineto{\pgfqpoint{1.105842in}{0.499444in}}%
\pgfpathclose%
\pgfusepath{stroke}%
\end{pgfscope}%
\begin{pgfscope}%
\pgfpathrectangle{\pgfqpoint{0.515000in}{0.499444in}}{\pgfqpoint{3.875000in}{1.155000in}}%
\pgfusepath{clip}%
\pgfsetbuttcap%
\pgfsetmiterjoin%
\pgfsetlinewidth{1.003750pt}%
\definecolor{currentstroke}{rgb}{0.000000,0.000000,0.000000}%
\pgfsetstrokecolor{currentstroke}%
\pgfsetdash{}{0pt}%
\pgfpathmoveto{\pgfqpoint{1.259307in}{0.499444in}}%
\pgfpathlineto{\pgfqpoint{1.320693in}{0.499444in}}%
\pgfpathlineto{\pgfqpoint{1.320693in}{0.500299in}}%
\pgfpathlineto{\pgfqpoint{1.259307in}{0.500299in}}%
\pgfpathlineto{\pgfqpoint{1.259307in}{0.499444in}}%
\pgfpathclose%
\pgfusepath{stroke}%
\end{pgfscope}%
\begin{pgfscope}%
\pgfpathrectangle{\pgfqpoint{0.515000in}{0.499444in}}{\pgfqpoint{3.875000in}{1.155000in}}%
\pgfusepath{clip}%
\pgfsetbuttcap%
\pgfsetmiterjoin%
\pgfsetlinewidth{1.003750pt}%
\definecolor{currentstroke}{rgb}{0.000000,0.000000,0.000000}%
\pgfsetstrokecolor{currentstroke}%
\pgfsetdash{}{0pt}%
\pgfpathmoveto{\pgfqpoint{1.412773in}{0.499444in}}%
\pgfpathlineto{\pgfqpoint{1.474159in}{0.499444in}}%
\pgfpathlineto{\pgfqpoint{1.474159in}{0.507734in}}%
\pgfpathlineto{\pgfqpoint{1.412773in}{0.507734in}}%
\pgfpathlineto{\pgfqpoint{1.412773in}{0.499444in}}%
\pgfpathclose%
\pgfusepath{stroke}%
\end{pgfscope}%
\begin{pgfscope}%
\pgfpathrectangle{\pgfqpoint{0.515000in}{0.499444in}}{\pgfqpoint{3.875000in}{1.155000in}}%
\pgfusepath{clip}%
\pgfsetbuttcap%
\pgfsetmiterjoin%
\pgfsetlinewidth{1.003750pt}%
\definecolor{currentstroke}{rgb}{0.000000,0.000000,0.000000}%
\pgfsetstrokecolor{currentstroke}%
\pgfsetdash{}{0pt}%
\pgfpathmoveto{\pgfqpoint{1.566238in}{0.499444in}}%
\pgfpathlineto{\pgfqpoint{1.627624in}{0.499444in}}%
\pgfpathlineto{\pgfqpoint{1.627624in}{1.517554in}}%
\pgfpathlineto{\pgfqpoint{1.566238in}{1.517554in}}%
\pgfpathlineto{\pgfqpoint{1.566238in}{0.499444in}}%
\pgfpathclose%
\pgfusepath{stroke}%
\end{pgfscope}%
\begin{pgfscope}%
\pgfpathrectangle{\pgfqpoint{0.515000in}{0.499444in}}{\pgfqpoint{3.875000in}{1.155000in}}%
\pgfusepath{clip}%
\pgfsetbuttcap%
\pgfsetmiterjoin%
\pgfsetlinewidth{1.003750pt}%
\definecolor{currentstroke}{rgb}{0.000000,0.000000,0.000000}%
\pgfsetstrokecolor{currentstroke}%
\pgfsetdash{}{0pt}%
\pgfpathmoveto{\pgfqpoint{1.719703in}{0.499444in}}%
\pgfpathlineto{\pgfqpoint{1.781089in}{0.499444in}}%
\pgfpathlineto{\pgfqpoint{1.781089in}{0.506024in}}%
\pgfpathlineto{\pgfqpoint{1.719703in}{0.506024in}}%
\pgfpathlineto{\pgfqpoint{1.719703in}{0.499444in}}%
\pgfpathclose%
\pgfusepath{stroke}%
\end{pgfscope}%
\begin{pgfscope}%
\pgfpathrectangle{\pgfqpoint{0.515000in}{0.499444in}}{\pgfqpoint{3.875000in}{1.155000in}}%
\pgfusepath{clip}%
\pgfsetbuttcap%
\pgfsetmiterjoin%
\pgfsetlinewidth{1.003750pt}%
\definecolor{currentstroke}{rgb}{0.000000,0.000000,0.000000}%
\pgfsetstrokecolor{currentstroke}%
\pgfsetdash{}{0pt}%
\pgfpathmoveto{\pgfqpoint{1.873169in}{0.499444in}}%
\pgfpathlineto{\pgfqpoint{1.934555in}{0.499444in}}%
\pgfpathlineto{\pgfqpoint{1.934555in}{1.355371in}}%
\pgfpathlineto{\pgfqpoint{1.873169in}{1.355371in}}%
\pgfpathlineto{\pgfqpoint{1.873169in}{0.499444in}}%
\pgfpathclose%
\pgfusepath{stroke}%
\end{pgfscope}%
\begin{pgfscope}%
\pgfpathrectangle{\pgfqpoint{0.515000in}{0.499444in}}{\pgfqpoint{3.875000in}{1.155000in}}%
\pgfusepath{clip}%
\pgfsetbuttcap%
\pgfsetmiterjoin%
\pgfsetlinewidth{1.003750pt}%
\definecolor{currentstroke}{rgb}{0.000000,0.000000,0.000000}%
\pgfsetstrokecolor{currentstroke}%
\pgfsetdash{}{0pt}%
\pgfpathmoveto{\pgfqpoint{2.026634in}{0.499444in}}%
\pgfpathlineto{\pgfqpoint{2.088020in}{0.499444in}}%
\pgfpathlineto{\pgfqpoint{2.088020in}{0.500262in}}%
\pgfpathlineto{\pgfqpoint{2.026634in}{0.500262in}}%
\pgfpathlineto{\pgfqpoint{2.026634in}{0.499444in}}%
\pgfpathclose%
\pgfusepath{stroke}%
\end{pgfscope}%
\begin{pgfscope}%
\pgfpathrectangle{\pgfqpoint{0.515000in}{0.499444in}}{\pgfqpoint{3.875000in}{1.155000in}}%
\pgfusepath{clip}%
\pgfsetbuttcap%
\pgfsetmiterjoin%
\pgfsetlinewidth{1.003750pt}%
\definecolor{currentstroke}{rgb}{0.000000,0.000000,0.000000}%
\pgfsetstrokecolor{currentstroke}%
\pgfsetdash{}{0pt}%
\pgfpathmoveto{\pgfqpoint{2.180099in}{0.499444in}}%
\pgfpathlineto{\pgfqpoint{2.241485in}{0.499444in}}%
\pgfpathlineto{\pgfqpoint{2.241485in}{0.508365in}}%
\pgfpathlineto{\pgfqpoint{2.180099in}{0.508365in}}%
\pgfpathlineto{\pgfqpoint{2.180099in}{0.499444in}}%
\pgfpathclose%
\pgfusepath{stroke}%
\end{pgfscope}%
\begin{pgfscope}%
\pgfpathrectangle{\pgfqpoint{0.515000in}{0.499444in}}{\pgfqpoint{3.875000in}{1.155000in}}%
\pgfusepath{clip}%
\pgfsetbuttcap%
\pgfsetmiterjoin%
\pgfsetlinewidth{1.003750pt}%
\definecolor{currentstroke}{rgb}{0.000000,0.000000,0.000000}%
\pgfsetstrokecolor{currentstroke}%
\pgfsetdash{}{0pt}%
\pgfpathmoveto{\pgfqpoint{2.333565in}{0.499444in}}%
\pgfpathlineto{\pgfqpoint{2.394951in}{0.499444in}}%
\pgfpathlineto{\pgfqpoint{2.394951in}{1.172596in}}%
\pgfpathlineto{\pgfqpoint{2.333565in}{1.172596in}}%
\pgfpathlineto{\pgfqpoint{2.333565in}{0.499444in}}%
\pgfpathclose%
\pgfusepath{stroke}%
\end{pgfscope}%
\begin{pgfscope}%
\pgfpathrectangle{\pgfqpoint{0.515000in}{0.499444in}}{\pgfqpoint{3.875000in}{1.155000in}}%
\pgfusepath{clip}%
\pgfsetbuttcap%
\pgfsetmiterjoin%
\pgfsetlinewidth{1.003750pt}%
\definecolor{currentstroke}{rgb}{0.000000,0.000000,0.000000}%
\pgfsetstrokecolor{currentstroke}%
\pgfsetdash{}{0pt}%
\pgfpathmoveto{\pgfqpoint{2.487030in}{0.499444in}}%
\pgfpathlineto{\pgfqpoint{2.548416in}{0.499444in}}%
\pgfpathlineto{\pgfqpoint{2.548416in}{0.504165in}}%
\pgfpathlineto{\pgfqpoint{2.487030in}{0.504165in}}%
\pgfpathlineto{\pgfqpoint{2.487030in}{0.499444in}}%
\pgfpathclose%
\pgfusepath{stroke}%
\end{pgfscope}%
\begin{pgfscope}%
\pgfpathrectangle{\pgfqpoint{0.515000in}{0.499444in}}{\pgfqpoint{3.875000in}{1.155000in}}%
\pgfusepath{clip}%
\pgfsetbuttcap%
\pgfsetmiterjoin%
\pgfsetlinewidth{1.003750pt}%
\definecolor{currentstroke}{rgb}{0.000000,0.000000,0.000000}%
\pgfsetstrokecolor{currentstroke}%
\pgfsetdash{}{0pt}%
\pgfpathmoveto{\pgfqpoint{2.640495in}{0.499444in}}%
\pgfpathlineto{\pgfqpoint{2.701881in}{0.499444in}}%
\pgfpathlineto{\pgfqpoint{2.701881in}{0.973055in}}%
\pgfpathlineto{\pgfqpoint{2.640495in}{0.973055in}}%
\pgfpathlineto{\pgfqpoint{2.640495in}{0.499444in}}%
\pgfpathclose%
\pgfusepath{stroke}%
\end{pgfscope}%
\begin{pgfscope}%
\pgfpathrectangle{\pgfqpoint{0.515000in}{0.499444in}}{\pgfqpoint{3.875000in}{1.155000in}}%
\pgfusepath{clip}%
\pgfsetbuttcap%
\pgfsetmiterjoin%
\pgfsetlinewidth{1.003750pt}%
\definecolor{currentstroke}{rgb}{0.000000,0.000000,0.000000}%
\pgfsetstrokecolor{currentstroke}%
\pgfsetdash{}{0pt}%
\pgfpathmoveto{\pgfqpoint{2.793961in}{0.499444in}}%
\pgfpathlineto{\pgfqpoint{2.855347in}{0.499444in}}%
\pgfpathlineto{\pgfqpoint{2.855347in}{0.500559in}}%
\pgfpathlineto{\pgfqpoint{2.793961in}{0.500559in}}%
\pgfpathlineto{\pgfqpoint{2.793961in}{0.499444in}}%
\pgfpathclose%
\pgfusepath{stroke}%
\end{pgfscope}%
\begin{pgfscope}%
\pgfpathrectangle{\pgfqpoint{0.515000in}{0.499444in}}{\pgfqpoint{3.875000in}{1.155000in}}%
\pgfusepath{clip}%
\pgfsetbuttcap%
\pgfsetmiterjoin%
\pgfsetlinewidth{1.003750pt}%
\definecolor{currentstroke}{rgb}{0.000000,0.000000,0.000000}%
\pgfsetstrokecolor{currentstroke}%
\pgfsetdash{}{0pt}%
\pgfpathmoveto{\pgfqpoint{2.947426in}{0.499444in}}%
\pgfpathlineto{\pgfqpoint{3.008812in}{0.499444in}}%
\pgfpathlineto{\pgfqpoint{3.008812in}{0.504202in}}%
\pgfpathlineto{\pgfqpoint{2.947426in}{0.504202in}}%
\pgfpathlineto{\pgfqpoint{2.947426in}{0.499444in}}%
\pgfpathclose%
\pgfusepath{stroke}%
\end{pgfscope}%
\begin{pgfscope}%
\pgfpathrectangle{\pgfqpoint{0.515000in}{0.499444in}}{\pgfqpoint{3.875000in}{1.155000in}}%
\pgfusepath{clip}%
\pgfsetbuttcap%
\pgfsetmiterjoin%
\pgfsetlinewidth{1.003750pt}%
\definecolor{currentstroke}{rgb}{0.000000,0.000000,0.000000}%
\pgfsetstrokecolor{currentstroke}%
\pgfsetdash{}{0pt}%
\pgfpathmoveto{\pgfqpoint{3.100891in}{0.499444in}}%
\pgfpathlineto{\pgfqpoint{3.162278in}{0.499444in}}%
\pgfpathlineto{\pgfqpoint{3.162278in}{0.805557in}}%
\pgfpathlineto{\pgfqpoint{3.100891in}{0.805557in}}%
\pgfpathlineto{\pgfqpoint{3.100891in}{0.499444in}}%
\pgfpathclose%
\pgfusepath{stroke}%
\end{pgfscope}%
\begin{pgfscope}%
\pgfpathrectangle{\pgfqpoint{0.515000in}{0.499444in}}{\pgfqpoint{3.875000in}{1.155000in}}%
\pgfusepath{clip}%
\pgfsetbuttcap%
\pgfsetmiterjoin%
\pgfsetlinewidth{1.003750pt}%
\definecolor{currentstroke}{rgb}{0.000000,0.000000,0.000000}%
\pgfsetstrokecolor{currentstroke}%
\pgfsetdash{}{0pt}%
\pgfpathmoveto{\pgfqpoint{3.254357in}{0.499444in}}%
\pgfpathlineto{\pgfqpoint{3.315743in}{0.499444in}}%
\pgfpathlineto{\pgfqpoint{3.315743in}{0.501637in}}%
\pgfpathlineto{\pgfqpoint{3.254357in}{0.501637in}}%
\pgfpathlineto{\pgfqpoint{3.254357in}{0.499444in}}%
\pgfpathclose%
\pgfusepath{stroke}%
\end{pgfscope}%
\begin{pgfscope}%
\pgfpathrectangle{\pgfqpoint{0.515000in}{0.499444in}}{\pgfqpoint{3.875000in}{1.155000in}}%
\pgfusepath{clip}%
\pgfsetbuttcap%
\pgfsetmiterjoin%
\pgfsetlinewidth{1.003750pt}%
\definecolor{currentstroke}{rgb}{0.000000,0.000000,0.000000}%
\pgfsetstrokecolor{currentstroke}%
\pgfsetdash{}{0pt}%
\pgfpathmoveto{\pgfqpoint{3.407822in}{0.499444in}}%
\pgfpathlineto{\pgfqpoint{3.469208in}{0.499444in}}%
\pgfpathlineto{\pgfqpoint{3.469208in}{0.690546in}}%
\pgfpathlineto{\pgfqpoint{3.407822in}{0.690546in}}%
\pgfpathlineto{\pgfqpoint{3.407822in}{0.499444in}}%
\pgfpathclose%
\pgfusepath{stroke}%
\end{pgfscope}%
\begin{pgfscope}%
\pgfpathrectangle{\pgfqpoint{0.515000in}{0.499444in}}{\pgfqpoint{3.875000in}{1.155000in}}%
\pgfusepath{clip}%
\pgfsetbuttcap%
\pgfsetmiterjoin%
\pgfsetlinewidth{1.003750pt}%
\definecolor{currentstroke}{rgb}{0.000000,0.000000,0.000000}%
\pgfsetstrokecolor{currentstroke}%
\pgfsetdash{}{0pt}%
\pgfpathmoveto{\pgfqpoint{3.561287in}{0.499444in}}%
\pgfpathlineto{\pgfqpoint{3.622674in}{0.499444in}}%
\pgfpathlineto{\pgfqpoint{3.622674in}{0.499742in}}%
\pgfpathlineto{\pgfqpoint{3.561287in}{0.499742in}}%
\pgfpathlineto{\pgfqpoint{3.561287in}{0.499444in}}%
\pgfpathclose%
\pgfusepath{stroke}%
\end{pgfscope}%
\begin{pgfscope}%
\pgfpathrectangle{\pgfqpoint{0.515000in}{0.499444in}}{\pgfqpoint{3.875000in}{1.155000in}}%
\pgfusepath{clip}%
\pgfsetbuttcap%
\pgfsetmiterjoin%
\pgfsetlinewidth{1.003750pt}%
\definecolor{currentstroke}{rgb}{0.000000,0.000000,0.000000}%
\pgfsetstrokecolor{currentstroke}%
\pgfsetdash{}{0pt}%
\pgfpathmoveto{\pgfqpoint{3.714753in}{0.499444in}}%
\pgfpathlineto{\pgfqpoint{3.776139in}{0.499444in}}%
\pgfpathlineto{\pgfqpoint{3.776139in}{0.501080in}}%
\pgfpathlineto{\pgfqpoint{3.714753in}{0.501080in}}%
\pgfpathlineto{\pgfqpoint{3.714753in}{0.499444in}}%
\pgfpathclose%
\pgfusepath{stroke}%
\end{pgfscope}%
\begin{pgfscope}%
\pgfpathrectangle{\pgfqpoint{0.515000in}{0.499444in}}{\pgfqpoint{3.875000in}{1.155000in}}%
\pgfusepath{clip}%
\pgfsetbuttcap%
\pgfsetmiterjoin%
\pgfsetlinewidth{1.003750pt}%
\definecolor{currentstroke}{rgb}{0.000000,0.000000,0.000000}%
\pgfsetstrokecolor{currentstroke}%
\pgfsetdash{}{0pt}%
\pgfpathmoveto{\pgfqpoint{3.868218in}{0.499444in}}%
\pgfpathlineto{\pgfqpoint{3.929604in}{0.499444in}}%
\pgfpathlineto{\pgfqpoint{3.929604in}{0.598508in}}%
\pgfpathlineto{\pgfqpoint{3.868218in}{0.598508in}}%
\pgfpathlineto{\pgfqpoint{3.868218in}{0.499444in}}%
\pgfpathclose%
\pgfusepath{stroke}%
\end{pgfscope}%
\begin{pgfscope}%
\pgfpathrectangle{\pgfqpoint{0.515000in}{0.499444in}}{\pgfqpoint{3.875000in}{1.155000in}}%
\pgfusepath{clip}%
\pgfsetbuttcap%
\pgfsetmiterjoin%
\pgfsetlinewidth{1.003750pt}%
\definecolor{currentstroke}{rgb}{0.000000,0.000000,0.000000}%
\pgfsetstrokecolor{currentstroke}%
\pgfsetdash{}{0pt}%
\pgfpathmoveto{\pgfqpoint{4.021683in}{0.499444in}}%
\pgfpathlineto{\pgfqpoint{4.083070in}{0.499444in}}%
\pgfpathlineto{\pgfqpoint{4.083070in}{0.500002in}}%
\pgfpathlineto{\pgfqpoint{4.021683in}{0.500002in}}%
\pgfpathlineto{\pgfqpoint{4.021683in}{0.499444in}}%
\pgfpathclose%
\pgfusepath{stroke}%
\end{pgfscope}%
\begin{pgfscope}%
\pgfpathrectangle{\pgfqpoint{0.515000in}{0.499444in}}{\pgfqpoint{3.875000in}{1.155000in}}%
\pgfusepath{clip}%
\pgfsetbuttcap%
\pgfsetmiterjoin%
\pgfsetlinewidth{1.003750pt}%
\definecolor{currentstroke}{rgb}{0.000000,0.000000,0.000000}%
\pgfsetstrokecolor{currentstroke}%
\pgfsetdash{}{0pt}%
\pgfpathmoveto{\pgfqpoint{4.175149in}{0.499444in}}%
\pgfpathlineto{\pgfqpoint{4.236535in}{0.499444in}}%
\pgfpathlineto{\pgfqpoint{4.236535in}{0.538103in}}%
\pgfpathlineto{\pgfqpoint{4.175149in}{0.538103in}}%
\pgfpathlineto{\pgfqpoint{4.175149in}{0.499444in}}%
\pgfpathclose%
\pgfusepath{stroke}%
\end{pgfscope}%
\begin{pgfscope}%
\pgfpathrectangle{\pgfqpoint{0.515000in}{0.499444in}}{\pgfqpoint{3.875000in}{1.155000in}}%
\pgfusepath{clip}%
\pgfsetbuttcap%
\pgfsetmiterjoin%
\definecolor{currentfill}{rgb}{0.000000,0.000000,0.000000}%
\pgfsetfillcolor{currentfill}%
\pgfsetlinewidth{0.000000pt}%
\definecolor{currentstroke}{rgb}{0.000000,0.000000,0.000000}%
\pgfsetstrokecolor{currentstroke}%
\pgfsetstrokeopacity{0.000000}%
\pgfsetdash{}{0pt}%
\pgfpathmoveto{\pgfqpoint{0.553367in}{0.499444in}}%
\pgfpathlineto{\pgfqpoint{0.614753in}{0.499444in}}%
\pgfpathlineto{\pgfqpoint{0.614753in}{0.499444in}}%
\pgfpathlineto{\pgfqpoint{0.553367in}{0.499444in}}%
\pgfpathlineto{\pgfqpoint{0.553367in}{0.499444in}}%
\pgfpathclose%
\pgfusepath{fill}%
\end{pgfscope}%
\begin{pgfscope}%
\pgfpathrectangle{\pgfqpoint{0.515000in}{0.499444in}}{\pgfqpoint{3.875000in}{1.155000in}}%
\pgfusepath{clip}%
\pgfsetbuttcap%
\pgfsetmiterjoin%
\definecolor{currentfill}{rgb}{0.000000,0.000000,0.000000}%
\pgfsetfillcolor{currentfill}%
\pgfsetlinewidth{0.000000pt}%
\definecolor{currentstroke}{rgb}{0.000000,0.000000,0.000000}%
\pgfsetstrokecolor{currentstroke}%
\pgfsetstrokeopacity{0.000000}%
\pgfsetdash{}{0pt}%
\pgfpathmoveto{\pgfqpoint{0.706832in}{0.499444in}}%
\pgfpathlineto{\pgfqpoint{0.768218in}{0.499444in}}%
\pgfpathlineto{\pgfqpoint{0.768218in}{0.499630in}}%
\pgfpathlineto{\pgfqpoint{0.706832in}{0.499630in}}%
\pgfpathlineto{\pgfqpoint{0.706832in}{0.499444in}}%
\pgfpathclose%
\pgfusepath{fill}%
\end{pgfscope}%
\begin{pgfscope}%
\pgfpathrectangle{\pgfqpoint{0.515000in}{0.499444in}}{\pgfqpoint{3.875000in}{1.155000in}}%
\pgfusepath{clip}%
\pgfsetbuttcap%
\pgfsetmiterjoin%
\definecolor{currentfill}{rgb}{0.000000,0.000000,0.000000}%
\pgfsetfillcolor{currentfill}%
\pgfsetlinewidth{0.000000pt}%
\definecolor{currentstroke}{rgb}{0.000000,0.000000,0.000000}%
\pgfsetstrokecolor{currentstroke}%
\pgfsetstrokeopacity{0.000000}%
\pgfsetdash{}{0pt}%
\pgfpathmoveto{\pgfqpoint{0.860297in}{0.499444in}}%
\pgfpathlineto{\pgfqpoint{0.921683in}{0.499444in}}%
\pgfpathlineto{\pgfqpoint{0.921683in}{0.557173in}}%
\pgfpathlineto{\pgfqpoint{0.860297in}{0.557173in}}%
\pgfpathlineto{\pgfqpoint{0.860297in}{0.499444in}}%
\pgfpathclose%
\pgfusepath{fill}%
\end{pgfscope}%
\begin{pgfscope}%
\pgfpathrectangle{\pgfqpoint{0.515000in}{0.499444in}}{\pgfqpoint{3.875000in}{1.155000in}}%
\pgfusepath{clip}%
\pgfsetbuttcap%
\pgfsetmiterjoin%
\definecolor{currentfill}{rgb}{0.000000,0.000000,0.000000}%
\pgfsetfillcolor{currentfill}%
\pgfsetlinewidth{0.000000pt}%
\definecolor{currentstroke}{rgb}{0.000000,0.000000,0.000000}%
\pgfsetstrokecolor{currentstroke}%
\pgfsetstrokeopacity{0.000000}%
\pgfsetdash{}{0pt}%
\pgfpathmoveto{\pgfqpoint{1.013763in}{0.499444in}}%
\pgfpathlineto{\pgfqpoint{1.075149in}{0.499444in}}%
\pgfpathlineto{\pgfqpoint{1.075149in}{0.499853in}}%
\pgfpathlineto{\pgfqpoint{1.013763in}{0.499853in}}%
\pgfpathlineto{\pgfqpoint{1.013763in}{0.499444in}}%
\pgfpathclose%
\pgfusepath{fill}%
\end{pgfscope}%
\begin{pgfscope}%
\pgfpathrectangle{\pgfqpoint{0.515000in}{0.499444in}}{\pgfqpoint{3.875000in}{1.155000in}}%
\pgfusepath{clip}%
\pgfsetbuttcap%
\pgfsetmiterjoin%
\definecolor{currentfill}{rgb}{0.000000,0.000000,0.000000}%
\pgfsetfillcolor{currentfill}%
\pgfsetlinewidth{0.000000pt}%
\definecolor{currentstroke}{rgb}{0.000000,0.000000,0.000000}%
\pgfsetstrokecolor{currentstroke}%
\pgfsetstrokeopacity{0.000000}%
\pgfsetdash{}{0pt}%
\pgfpathmoveto{\pgfqpoint{1.167228in}{0.499444in}}%
\pgfpathlineto{\pgfqpoint{1.228614in}{0.499444in}}%
\pgfpathlineto{\pgfqpoint{1.228614in}{0.589141in}}%
\pgfpathlineto{\pgfqpoint{1.167228in}{0.589141in}}%
\pgfpathlineto{\pgfqpoint{1.167228in}{0.499444in}}%
\pgfpathclose%
\pgfusepath{fill}%
\end{pgfscope}%
\begin{pgfscope}%
\pgfpathrectangle{\pgfqpoint{0.515000in}{0.499444in}}{\pgfqpoint{3.875000in}{1.155000in}}%
\pgfusepath{clip}%
\pgfsetbuttcap%
\pgfsetmiterjoin%
\definecolor{currentfill}{rgb}{0.000000,0.000000,0.000000}%
\pgfsetfillcolor{currentfill}%
\pgfsetlinewidth{0.000000pt}%
\definecolor{currentstroke}{rgb}{0.000000,0.000000,0.000000}%
\pgfsetstrokecolor{currentstroke}%
\pgfsetstrokeopacity{0.000000}%
\pgfsetdash{}{0pt}%
\pgfpathmoveto{\pgfqpoint{1.320693in}{0.499444in}}%
\pgfpathlineto{\pgfqpoint{1.382079in}{0.499444in}}%
\pgfpathlineto{\pgfqpoint{1.382079in}{0.499481in}}%
\pgfpathlineto{\pgfqpoint{1.320693in}{0.499481in}}%
\pgfpathlineto{\pgfqpoint{1.320693in}{0.499444in}}%
\pgfpathclose%
\pgfusepath{fill}%
\end{pgfscope}%
\begin{pgfscope}%
\pgfpathrectangle{\pgfqpoint{0.515000in}{0.499444in}}{\pgfqpoint{3.875000in}{1.155000in}}%
\pgfusepath{clip}%
\pgfsetbuttcap%
\pgfsetmiterjoin%
\definecolor{currentfill}{rgb}{0.000000,0.000000,0.000000}%
\pgfsetfillcolor{currentfill}%
\pgfsetlinewidth{0.000000pt}%
\definecolor{currentstroke}{rgb}{0.000000,0.000000,0.000000}%
\pgfsetstrokecolor{currentstroke}%
\pgfsetstrokeopacity{0.000000}%
\pgfsetdash{}{0pt}%
\pgfpathmoveto{\pgfqpoint{1.474159in}{0.499444in}}%
\pgfpathlineto{\pgfqpoint{1.535545in}{0.499444in}}%
\pgfpathlineto{\pgfqpoint{1.535545in}{0.500336in}}%
\pgfpathlineto{\pgfqpoint{1.474159in}{0.500336in}}%
\pgfpathlineto{\pgfqpoint{1.474159in}{0.499444in}}%
\pgfpathclose%
\pgfusepath{fill}%
\end{pgfscope}%
\begin{pgfscope}%
\pgfpathrectangle{\pgfqpoint{0.515000in}{0.499444in}}{\pgfqpoint{3.875000in}{1.155000in}}%
\pgfusepath{clip}%
\pgfsetbuttcap%
\pgfsetmiterjoin%
\definecolor{currentfill}{rgb}{0.000000,0.000000,0.000000}%
\pgfsetfillcolor{currentfill}%
\pgfsetlinewidth{0.000000pt}%
\definecolor{currentstroke}{rgb}{0.000000,0.000000,0.000000}%
\pgfsetstrokecolor{currentstroke}%
\pgfsetstrokeopacity{0.000000}%
\pgfsetdash{}{0pt}%
\pgfpathmoveto{\pgfqpoint{1.627624in}{0.499444in}}%
\pgfpathlineto{\pgfqpoint{1.689010in}{0.499444in}}%
\pgfpathlineto{\pgfqpoint{1.689010in}{0.616165in}}%
\pgfpathlineto{\pgfqpoint{1.627624in}{0.616165in}}%
\pgfpathlineto{\pgfqpoint{1.627624in}{0.499444in}}%
\pgfpathclose%
\pgfusepath{fill}%
\end{pgfscope}%
\begin{pgfscope}%
\pgfpathrectangle{\pgfqpoint{0.515000in}{0.499444in}}{\pgfqpoint{3.875000in}{1.155000in}}%
\pgfusepath{clip}%
\pgfsetbuttcap%
\pgfsetmiterjoin%
\definecolor{currentfill}{rgb}{0.000000,0.000000,0.000000}%
\pgfsetfillcolor{currentfill}%
\pgfsetlinewidth{0.000000pt}%
\definecolor{currentstroke}{rgb}{0.000000,0.000000,0.000000}%
\pgfsetstrokecolor{currentstroke}%
\pgfsetstrokeopacity{0.000000}%
\pgfsetdash{}{0pt}%
\pgfpathmoveto{\pgfqpoint{1.781089in}{0.499444in}}%
\pgfpathlineto{\pgfqpoint{1.842476in}{0.499444in}}%
\pgfpathlineto{\pgfqpoint{1.842476in}{0.500448in}}%
\pgfpathlineto{\pgfqpoint{1.781089in}{0.500448in}}%
\pgfpathlineto{\pgfqpoint{1.781089in}{0.499444in}}%
\pgfpathclose%
\pgfusepath{fill}%
\end{pgfscope}%
\begin{pgfscope}%
\pgfpathrectangle{\pgfqpoint{0.515000in}{0.499444in}}{\pgfqpoint{3.875000in}{1.155000in}}%
\pgfusepath{clip}%
\pgfsetbuttcap%
\pgfsetmiterjoin%
\definecolor{currentfill}{rgb}{0.000000,0.000000,0.000000}%
\pgfsetfillcolor{currentfill}%
\pgfsetlinewidth{0.000000pt}%
\definecolor{currentstroke}{rgb}{0.000000,0.000000,0.000000}%
\pgfsetstrokecolor{currentstroke}%
\pgfsetstrokeopacity{0.000000}%
\pgfsetdash{}{0pt}%
\pgfpathmoveto{\pgfqpoint{1.934555in}{0.499444in}}%
\pgfpathlineto{\pgfqpoint{1.995941in}{0.499444in}}%
\pgfpathlineto{\pgfqpoint{1.995941in}{0.640550in}}%
\pgfpathlineto{\pgfqpoint{1.934555in}{0.640550in}}%
\pgfpathlineto{\pgfqpoint{1.934555in}{0.499444in}}%
\pgfpathclose%
\pgfusepath{fill}%
\end{pgfscope}%
\begin{pgfscope}%
\pgfpathrectangle{\pgfqpoint{0.515000in}{0.499444in}}{\pgfqpoint{3.875000in}{1.155000in}}%
\pgfusepath{clip}%
\pgfsetbuttcap%
\pgfsetmiterjoin%
\definecolor{currentfill}{rgb}{0.000000,0.000000,0.000000}%
\pgfsetfillcolor{currentfill}%
\pgfsetlinewidth{0.000000pt}%
\definecolor{currentstroke}{rgb}{0.000000,0.000000,0.000000}%
\pgfsetstrokecolor{currentstroke}%
\pgfsetstrokeopacity{0.000000}%
\pgfsetdash{}{0pt}%
\pgfpathmoveto{\pgfqpoint{2.088020in}{0.499444in}}%
\pgfpathlineto{\pgfqpoint{2.149406in}{0.499444in}}%
\pgfpathlineto{\pgfqpoint{2.149406in}{0.499630in}}%
\pgfpathlineto{\pgfqpoint{2.088020in}{0.499630in}}%
\pgfpathlineto{\pgfqpoint{2.088020in}{0.499444in}}%
\pgfpathclose%
\pgfusepath{fill}%
\end{pgfscope}%
\begin{pgfscope}%
\pgfpathrectangle{\pgfqpoint{0.515000in}{0.499444in}}{\pgfqpoint{3.875000in}{1.155000in}}%
\pgfusepath{clip}%
\pgfsetbuttcap%
\pgfsetmiterjoin%
\definecolor{currentfill}{rgb}{0.000000,0.000000,0.000000}%
\pgfsetfillcolor{currentfill}%
\pgfsetlinewidth{0.000000pt}%
\definecolor{currentstroke}{rgb}{0.000000,0.000000,0.000000}%
\pgfsetstrokecolor{currentstroke}%
\pgfsetstrokeopacity{0.000000}%
\pgfsetdash{}{0pt}%
\pgfpathmoveto{\pgfqpoint{2.241485in}{0.499444in}}%
\pgfpathlineto{\pgfqpoint{2.302872in}{0.499444in}}%
\pgfpathlineto{\pgfqpoint{2.302872in}{0.501340in}}%
\pgfpathlineto{\pgfqpoint{2.241485in}{0.501340in}}%
\pgfpathlineto{\pgfqpoint{2.241485in}{0.499444in}}%
\pgfpathclose%
\pgfusepath{fill}%
\end{pgfscope}%
\begin{pgfscope}%
\pgfpathrectangle{\pgfqpoint{0.515000in}{0.499444in}}{\pgfqpoint{3.875000in}{1.155000in}}%
\pgfusepath{clip}%
\pgfsetbuttcap%
\pgfsetmiterjoin%
\definecolor{currentfill}{rgb}{0.000000,0.000000,0.000000}%
\pgfsetfillcolor{currentfill}%
\pgfsetlinewidth{0.000000pt}%
\definecolor{currentstroke}{rgb}{0.000000,0.000000,0.000000}%
\pgfsetstrokecolor{currentstroke}%
\pgfsetstrokeopacity{0.000000}%
\pgfsetdash{}{0pt}%
\pgfpathmoveto{\pgfqpoint{2.394951in}{0.499444in}}%
\pgfpathlineto{\pgfqpoint{2.456337in}{0.499444in}}%
\pgfpathlineto{\pgfqpoint{2.456337in}{0.652928in}}%
\pgfpathlineto{\pgfqpoint{2.394951in}{0.652928in}}%
\pgfpathlineto{\pgfqpoint{2.394951in}{0.499444in}}%
\pgfpathclose%
\pgfusepath{fill}%
\end{pgfscope}%
\begin{pgfscope}%
\pgfpathrectangle{\pgfqpoint{0.515000in}{0.499444in}}{\pgfqpoint{3.875000in}{1.155000in}}%
\pgfusepath{clip}%
\pgfsetbuttcap%
\pgfsetmiterjoin%
\definecolor{currentfill}{rgb}{0.000000,0.000000,0.000000}%
\pgfsetfillcolor{currentfill}%
\pgfsetlinewidth{0.000000pt}%
\definecolor{currentstroke}{rgb}{0.000000,0.000000,0.000000}%
\pgfsetstrokecolor{currentstroke}%
\pgfsetstrokeopacity{0.000000}%
\pgfsetdash{}{0pt}%
\pgfpathmoveto{\pgfqpoint{2.548416in}{0.499444in}}%
\pgfpathlineto{\pgfqpoint{2.609802in}{0.499444in}}%
\pgfpathlineto{\pgfqpoint{2.609802in}{0.500782in}}%
\pgfpathlineto{\pgfqpoint{2.548416in}{0.500782in}}%
\pgfpathlineto{\pgfqpoint{2.548416in}{0.499444in}}%
\pgfpathclose%
\pgfusepath{fill}%
\end{pgfscope}%
\begin{pgfscope}%
\pgfpathrectangle{\pgfqpoint{0.515000in}{0.499444in}}{\pgfqpoint{3.875000in}{1.155000in}}%
\pgfusepath{clip}%
\pgfsetbuttcap%
\pgfsetmiterjoin%
\definecolor{currentfill}{rgb}{0.000000,0.000000,0.000000}%
\pgfsetfillcolor{currentfill}%
\pgfsetlinewidth{0.000000pt}%
\definecolor{currentstroke}{rgb}{0.000000,0.000000,0.000000}%
\pgfsetstrokecolor{currentstroke}%
\pgfsetstrokeopacity{0.000000}%
\pgfsetdash{}{0pt}%
\pgfpathmoveto{\pgfqpoint{2.701881in}{0.499444in}}%
\pgfpathlineto{\pgfqpoint{2.763268in}{0.499444in}}%
\pgfpathlineto{\pgfqpoint{2.763268in}{0.656943in}}%
\pgfpathlineto{\pgfqpoint{2.701881in}{0.656943in}}%
\pgfpathlineto{\pgfqpoint{2.701881in}{0.499444in}}%
\pgfpathclose%
\pgfusepath{fill}%
\end{pgfscope}%
\begin{pgfscope}%
\pgfpathrectangle{\pgfqpoint{0.515000in}{0.499444in}}{\pgfqpoint{3.875000in}{1.155000in}}%
\pgfusepath{clip}%
\pgfsetbuttcap%
\pgfsetmiterjoin%
\definecolor{currentfill}{rgb}{0.000000,0.000000,0.000000}%
\pgfsetfillcolor{currentfill}%
\pgfsetlinewidth{0.000000pt}%
\definecolor{currentstroke}{rgb}{0.000000,0.000000,0.000000}%
\pgfsetstrokecolor{currentstroke}%
\pgfsetstrokeopacity{0.000000}%
\pgfsetdash{}{0pt}%
\pgfpathmoveto{\pgfqpoint{2.855347in}{0.499444in}}%
\pgfpathlineto{\pgfqpoint{2.916733in}{0.499444in}}%
\pgfpathlineto{\pgfqpoint{2.916733in}{0.499890in}}%
\pgfpathlineto{\pgfqpoint{2.855347in}{0.499890in}}%
\pgfpathlineto{\pgfqpoint{2.855347in}{0.499444in}}%
\pgfpathclose%
\pgfusepath{fill}%
\end{pgfscope}%
\begin{pgfscope}%
\pgfpathrectangle{\pgfqpoint{0.515000in}{0.499444in}}{\pgfqpoint{3.875000in}{1.155000in}}%
\pgfusepath{clip}%
\pgfsetbuttcap%
\pgfsetmiterjoin%
\definecolor{currentfill}{rgb}{0.000000,0.000000,0.000000}%
\pgfsetfillcolor{currentfill}%
\pgfsetlinewidth{0.000000pt}%
\definecolor{currentstroke}{rgb}{0.000000,0.000000,0.000000}%
\pgfsetstrokecolor{currentstroke}%
\pgfsetstrokeopacity{0.000000}%
\pgfsetdash{}{0pt}%
\pgfpathmoveto{\pgfqpoint{3.008812in}{0.499444in}}%
\pgfpathlineto{\pgfqpoint{3.070198in}{0.499444in}}%
\pgfpathlineto{\pgfqpoint{3.070198in}{0.502046in}}%
\pgfpathlineto{\pgfqpoint{3.008812in}{0.502046in}}%
\pgfpathlineto{\pgfqpoint{3.008812in}{0.499444in}}%
\pgfpathclose%
\pgfusepath{fill}%
\end{pgfscope}%
\begin{pgfscope}%
\pgfpathrectangle{\pgfqpoint{0.515000in}{0.499444in}}{\pgfqpoint{3.875000in}{1.155000in}}%
\pgfusepath{clip}%
\pgfsetbuttcap%
\pgfsetmiterjoin%
\definecolor{currentfill}{rgb}{0.000000,0.000000,0.000000}%
\pgfsetfillcolor{currentfill}%
\pgfsetlinewidth{0.000000pt}%
\definecolor{currentstroke}{rgb}{0.000000,0.000000,0.000000}%
\pgfsetstrokecolor{currentstroke}%
\pgfsetstrokeopacity{0.000000}%
\pgfsetdash{}{0pt}%
\pgfpathmoveto{\pgfqpoint{3.162278in}{0.499444in}}%
\pgfpathlineto{\pgfqpoint{3.223664in}{0.499444in}}%
\pgfpathlineto{\pgfqpoint{3.223664in}{0.658578in}}%
\pgfpathlineto{\pgfqpoint{3.162278in}{0.658578in}}%
\pgfpathlineto{\pgfqpoint{3.162278in}{0.499444in}}%
\pgfpathclose%
\pgfusepath{fill}%
\end{pgfscope}%
\begin{pgfscope}%
\pgfpathrectangle{\pgfqpoint{0.515000in}{0.499444in}}{\pgfqpoint{3.875000in}{1.155000in}}%
\pgfusepath{clip}%
\pgfsetbuttcap%
\pgfsetmiterjoin%
\definecolor{currentfill}{rgb}{0.000000,0.000000,0.000000}%
\pgfsetfillcolor{currentfill}%
\pgfsetlinewidth{0.000000pt}%
\definecolor{currentstroke}{rgb}{0.000000,0.000000,0.000000}%
\pgfsetstrokecolor{currentstroke}%
\pgfsetstrokeopacity{0.000000}%
\pgfsetdash{}{0pt}%
\pgfpathmoveto{\pgfqpoint{3.315743in}{0.499444in}}%
\pgfpathlineto{\pgfqpoint{3.377129in}{0.499444in}}%
\pgfpathlineto{\pgfqpoint{3.377129in}{0.500782in}}%
\pgfpathlineto{\pgfqpoint{3.315743in}{0.500782in}}%
\pgfpathlineto{\pgfqpoint{3.315743in}{0.499444in}}%
\pgfpathclose%
\pgfusepath{fill}%
\end{pgfscope}%
\begin{pgfscope}%
\pgfpathrectangle{\pgfqpoint{0.515000in}{0.499444in}}{\pgfqpoint{3.875000in}{1.155000in}}%
\pgfusepath{clip}%
\pgfsetbuttcap%
\pgfsetmiterjoin%
\definecolor{currentfill}{rgb}{0.000000,0.000000,0.000000}%
\pgfsetfillcolor{currentfill}%
\pgfsetlinewidth{0.000000pt}%
\definecolor{currentstroke}{rgb}{0.000000,0.000000,0.000000}%
\pgfsetstrokecolor{currentstroke}%
\pgfsetstrokeopacity{0.000000}%
\pgfsetdash{}{0pt}%
\pgfpathmoveto{\pgfqpoint{3.469208in}{0.499444in}}%
\pgfpathlineto{\pgfqpoint{3.530594in}{0.499444in}}%
\pgfpathlineto{\pgfqpoint{3.530594in}{0.644156in}}%
\pgfpathlineto{\pgfqpoint{3.469208in}{0.644156in}}%
\pgfpathlineto{\pgfqpoint{3.469208in}{0.499444in}}%
\pgfpathclose%
\pgfusepath{fill}%
\end{pgfscope}%
\begin{pgfscope}%
\pgfpathrectangle{\pgfqpoint{0.515000in}{0.499444in}}{\pgfqpoint{3.875000in}{1.155000in}}%
\pgfusepath{clip}%
\pgfsetbuttcap%
\pgfsetmiterjoin%
\definecolor{currentfill}{rgb}{0.000000,0.000000,0.000000}%
\pgfsetfillcolor{currentfill}%
\pgfsetlinewidth{0.000000pt}%
\definecolor{currentstroke}{rgb}{0.000000,0.000000,0.000000}%
\pgfsetstrokecolor{currentstroke}%
\pgfsetstrokeopacity{0.000000}%
\pgfsetdash{}{0pt}%
\pgfpathmoveto{\pgfqpoint{3.622674in}{0.499444in}}%
\pgfpathlineto{\pgfqpoint{3.684060in}{0.499444in}}%
\pgfpathlineto{\pgfqpoint{3.684060in}{0.500002in}}%
\pgfpathlineto{\pgfqpoint{3.622674in}{0.500002in}}%
\pgfpathlineto{\pgfqpoint{3.622674in}{0.499444in}}%
\pgfpathclose%
\pgfusepath{fill}%
\end{pgfscope}%
\begin{pgfscope}%
\pgfpathrectangle{\pgfqpoint{0.515000in}{0.499444in}}{\pgfqpoint{3.875000in}{1.155000in}}%
\pgfusepath{clip}%
\pgfsetbuttcap%
\pgfsetmiterjoin%
\definecolor{currentfill}{rgb}{0.000000,0.000000,0.000000}%
\pgfsetfillcolor{currentfill}%
\pgfsetlinewidth{0.000000pt}%
\definecolor{currentstroke}{rgb}{0.000000,0.000000,0.000000}%
\pgfsetstrokecolor{currentstroke}%
\pgfsetstrokeopacity{0.000000}%
\pgfsetdash{}{0pt}%
\pgfpathmoveto{\pgfqpoint{3.776139in}{0.499444in}}%
\pgfpathlineto{\pgfqpoint{3.837525in}{0.499444in}}%
\pgfpathlineto{\pgfqpoint{3.837525in}{0.500968in}}%
\pgfpathlineto{\pgfqpoint{3.776139in}{0.500968in}}%
\pgfpathlineto{\pgfqpoint{3.776139in}{0.499444in}}%
\pgfpathclose%
\pgfusepath{fill}%
\end{pgfscope}%
\begin{pgfscope}%
\pgfpathrectangle{\pgfqpoint{0.515000in}{0.499444in}}{\pgfqpoint{3.875000in}{1.155000in}}%
\pgfusepath{clip}%
\pgfsetbuttcap%
\pgfsetmiterjoin%
\definecolor{currentfill}{rgb}{0.000000,0.000000,0.000000}%
\pgfsetfillcolor{currentfill}%
\pgfsetlinewidth{0.000000pt}%
\definecolor{currentstroke}{rgb}{0.000000,0.000000,0.000000}%
\pgfsetstrokecolor{currentstroke}%
\pgfsetstrokeopacity{0.000000}%
\pgfsetdash{}{0pt}%
\pgfpathmoveto{\pgfqpoint{3.929604in}{0.499444in}}%
\pgfpathlineto{\pgfqpoint{3.990990in}{0.499444in}}%
\pgfpathlineto{\pgfqpoint{3.990990in}{0.621704in}}%
\pgfpathlineto{\pgfqpoint{3.929604in}{0.621704in}}%
\pgfpathlineto{\pgfqpoint{3.929604in}{0.499444in}}%
\pgfpathclose%
\pgfusepath{fill}%
\end{pgfscope}%
\begin{pgfscope}%
\pgfpathrectangle{\pgfqpoint{0.515000in}{0.499444in}}{\pgfqpoint{3.875000in}{1.155000in}}%
\pgfusepath{clip}%
\pgfsetbuttcap%
\pgfsetmiterjoin%
\definecolor{currentfill}{rgb}{0.000000,0.000000,0.000000}%
\pgfsetfillcolor{currentfill}%
\pgfsetlinewidth{0.000000pt}%
\definecolor{currentstroke}{rgb}{0.000000,0.000000,0.000000}%
\pgfsetstrokecolor{currentstroke}%
\pgfsetstrokeopacity{0.000000}%
\pgfsetdash{}{0pt}%
\pgfpathmoveto{\pgfqpoint{4.083070in}{0.499444in}}%
\pgfpathlineto{\pgfqpoint{4.144456in}{0.499444in}}%
\pgfpathlineto{\pgfqpoint{4.144456in}{0.499965in}}%
\pgfpathlineto{\pgfqpoint{4.083070in}{0.499965in}}%
\pgfpathlineto{\pgfqpoint{4.083070in}{0.499444in}}%
\pgfpathclose%
\pgfusepath{fill}%
\end{pgfscope}%
\begin{pgfscope}%
\pgfpathrectangle{\pgfqpoint{0.515000in}{0.499444in}}{\pgfqpoint{3.875000in}{1.155000in}}%
\pgfusepath{clip}%
\pgfsetbuttcap%
\pgfsetmiterjoin%
\definecolor{currentfill}{rgb}{0.000000,0.000000,0.000000}%
\pgfsetfillcolor{currentfill}%
\pgfsetlinewidth{0.000000pt}%
\definecolor{currentstroke}{rgb}{0.000000,0.000000,0.000000}%
\pgfsetstrokecolor{currentstroke}%
\pgfsetstrokeopacity{0.000000}%
\pgfsetdash{}{0pt}%
\pgfpathmoveto{\pgfqpoint{4.236535in}{0.499444in}}%
\pgfpathlineto{\pgfqpoint{4.297921in}{0.499444in}}%
\pgfpathlineto{\pgfqpoint{4.297921in}{0.576316in}}%
\pgfpathlineto{\pgfqpoint{4.236535in}{0.576316in}}%
\pgfpathlineto{\pgfqpoint{4.236535in}{0.499444in}}%
\pgfpathclose%
\pgfusepath{fill}%
\end{pgfscope}%
\begin{pgfscope}%
\pgfsetbuttcap%
\pgfsetroundjoin%
\definecolor{currentfill}{rgb}{0.000000,0.000000,0.000000}%
\pgfsetfillcolor{currentfill}%
\pgfsetlinewidth{0.803000pt}%
\definecolor{currentstroke}{rgb}{0.000000,0.000000,0.000000}%
\pgfsetstrokecolor{currentstroke}%
\pgfsetdash{}{0pt}%
\pgfsys@defobject{currentmarker}{\pgfqpoint{0.000000in}{-0.048611in}}{\pgfqpoint{0.000000in}{0.000000in}}{%
\pgfpathmoveto{\pgfqpoint{0.000000in}{0.000000in}}%
\pgfpathlineto{\pgfqpoint{0.000000in}{-0.048611in}}%
\pgfusepath{stroke,fill}%
}%
\begin{pgfscope}%
\pgfsys@transformshift{0.553367in}{0.499444in}%
\pgfsys@useobject{currentmarker}{}%
\end{pgfscope}%
\end{pgfscope}%
\begin{pgfscope}%
\definecolor{textcolor}{rgb}{0.000000,0.000000,0.000000}%
\pgfsetstrokecolor{textcolor}%
\pgfsetfillcolor{textcolor}%
\pgftext[x=0.553367in,y=0.402222in,,top]{\color{textcolor}\rmfamily\fontsize{10.000000}{12.000000}\selectfont 0.0}%
\end{pgfscope}%
\begin{pgfscope}%
\pgfsetbuttcap%
\pgfsetroundjoin%
\definecolor{currentfill}{rgb}{0.000000,0.000000,0.000000}%
\pgfsetfillcolor{currentfill}%
\pgfsetlinewidth{0.803000pt}%
\definecolor{currentstroke}{rgb}{0.000000,0.000000,0.000000}%
\pgfsetstrokecolor{currentstroke}%
\pgfsetdash{}{0pt}%
\pgfsys@defobject{currentmarker}{\pgfqpoint{0.000000in}{-0.048611in}}{\pgfqpoint{0.000000in}{0.000000in}}{%
\pgfpathmoveto{\pgfqpoint{0.000000in}{0.000000in}}%
\pgfpathlineto{\pgfqpoint{0.000000in}{-0.048611in}}%
\pgfusepath{stroke,fill}%
}%
\begin{pgfscope}%
\pgfsys@transformshift{0.937030in}{0.499444in}%
\pgfsys@useobject{currentmarker}{}%
\end{pgfscope}%
\end{pgfscope}%
\begin{pgfscope}%
\definecolor{textcolor}{rgb}{0.000000,0.000000,0.000000}%
\pgfsetstrokecolor{textcolor}%
\pgfsetfillcolor{textcolor}%
\pgftext[x=0.937030in,y=0.402222in,,top]{\color{textcolor}\rmfamily\fontsize{10.000000}{12.000000}\selectfont 0.1}%
\end{pgfscope}%
\begin{pgfscope}%
\pgfsetbuttcap%
\pgfsetroundjoin%
\definecolor{currentfill}{rgb}{0.000000,0.000000,0.000000}%
\pgfsetfillcolor{currentfill}%
\pgfsetlinewidth{0.803000pt}%
\definecolor{currentstroke}{rgb}{0.000000,0.000000,0.000000}%
\pgfsetstrokecolor{currentstroke}%
\pgfsetdash{}{0pt}%
\pgfsys@defobject{currentmarker}{\pgfqpoint{0.000000in}{-0.048611in}}{\pgfqpoint{0.000000in}{0.000000in}}{%
\pgfpathmoveto{\pgfqpoint{0.000000in}{0.000000in}}%
\pgfpathlineto{\pgfqpoint{0.000000in}{-0.048611in}}%
\pgfusepath{stroke,fill}%
}%
\begin{pgfscope}%
\pgfsys@transformshift{1.320693in}{0.499444in}%
\pgfsys@useobject{currentmarker}{}%
\end{pgfscope}%
\end{pgfscope}%
\begin{pgfscope}%
\definecolor{textcolor}{rgb}{0.000000,0.000000,0.000000}%
\pgfsetstrokecolor{textcolor}%
\pgfsetfillcolor{textcolor}%
\pgftext[x=1.320693in,y=0.402222in,,top]{\color{textcolor}\rmfamily\fontsize{10.000000}{12.000000}\selectfont 0.2}%
\end{pgfscope}%
\begin{pgfscope}%
\pgfsetbuttcap%
\pgfsetroundjoin%
\definecolor{currentfill}{rgb}{0.000000,0.000000,0.000000}%
\pgfsetfillcolor{currentfill}%
\pgfsetlinewidth{0.803000pt}%
\definecolor{currentstroke}{rgb}{0.000000,0.000000,0.000000}%
\pgfsetstrokecolor{currentstroke}%
\pgfsetdash{}{0pt}%
\pgfsys@defobject{currentmarker}{\pgfqpoint{0.000000in}{-0.048611in}}{\pgfqpoint{0.000000in}{0.000000in}}{%
\pgfpathmoveto{\pgfqpoint{0.000000in}{0.000000in}}%
\pgfpathlineto{\pgfqpoint{0.000000in}{-0.048611in}}%
\pgfusepath{stroke,fill}%
}%
\begin{pgfscope}%
\pgfsys@transformshift{1.704357in}{0.499444in}%
\pgfsys@useobject{currentmarker}{}%
\end{pgfscope}%
\end{pgfscope}%
\begin{pgfscope}%
\definecolor{textcolor}{rgb}{0.000000,0.000000,0.000000}%
\pgfsetstrokecolor{textcolor}%
\pgfsetfillcolor{textcolor}%
\pgftext[x=1.704357in,y=0.402222in,,top]{\color{textcolor}\rmfamily\fontsize{10.000000}{12.000000}\selectfont 0.3}%
\end{pgfscope}%
\begin{pgfscope}%
\pgfsetbuttcap%
\pgfsetroundjoin%
\definecolor{currentfill}{rgb}{0.000000,0.000000,0.000000}%
\pgfsetfillcolor{currentfill}%
\pgfsetlinewidth{0.803000pt}%
\definecolor{currentstroke}{rgb}{0.000000,0.000000,0.000000}%
\pgfsetstrokecolor{currentstroke}%
\pgfsetdash{}{0pt}%
\pgfsys@defobject{currentmarker}{\pgfqpoint{0.000000in}{-0.048611in}}{\pgfqpoint{0.000000in}{0.000000in}}{%
\pgfpathmoveto{\pgfqpoint{0.000000in}{0.000000in}}%
\pgfpathlineto{\pgfqpoint{0.000000in}{-0.048611in}}%
\pgfusepath{stroke,fill}%
}%
\begin{pgfscope}%
\pgfsys@transformshift{2.088020in}{0.499444in}%
\pgfsys@useobject{currentmarker}{}%
\end{pgfscope}%
\end{pgfscope}%
\begin{pgfscope}%
\definecolor{textcolor}{rgb}{0.000000,0.000000,0.000000}%
\pgfsetstrokecolor{textcolor}%
\pgfsetfillcolor{textcolor}%
\pgftext[x=2.088020in,y=0.402222in,,top]{\color{textcolor}\rmfamily\fontsize{10.000000}{12.000000}\selectfont 0.4}%
\end{pgfscope}%
\begin{pgfscope}%
\pgfsetbuttcap%
\pgfsetroundjoin%
\definecolor{currentfill}{rgb}{0.000000,0.000000,0.000000}%
\pgfsetfillcolor{currentfill}%
\pgfsetlinewidth{0.803000pt}%
\definecolor{currentstroke}{rgb}{0.000000,0.000000,0.000000}%
\pgfsetstrokecolor{currentstroke}%
\pgfsetdash{}{0pt}%
\pgfsys@defobject{currentmarker}{\pgfqpoint{0.000000in}{-0.048611in}}{\pgfqpoint{0.000000in}{0.000000in}}{%
\pgfpathmoveto{\pgfqpoint{0.000000in}{0.000000in}}%
\pgfpathlineto{\pgfqpoint{0.000000in}{-0.048611in}}%
\pgfusepath{stroke,fill}%
}%
\begin{pgfscope}%
\pgfsys@transformshift{2.471683in}{0.499444in}%
\pgfsys@useobject{currentmarker}{}%
\end{pgfscope}%
\end{pgfscope}%
\begin{pgfscope}%
\definecolor{textcolor}{rgb}{0.000000,0.000000,0.000000}%
\pgfsetstrokecolor{textcolor}%
\pgfsetfillcolor{textcolor}%
\pgftext[x=2.471683in,y=0.402222in,,top]{\color{textcolor}\rmfamily\fontsize{10.000000}{12.000000}\selectfont 0.5}%
\end{pgfscope}%
\begin{pgfscope}%
\pgfsetbuttcap%
\pgfsetroundjoin%
\definecolor{currentfill}{rgb}{0.000000,0.000000,0.000000}%
\pgfsetfillcolor{currentfill}%
\pgfsetlinewidth{0.803000pt}%
\definecolor{currentstroke}{rgb}{0.000000,0.000000,0.000000}%
\pgfsetstrokecolor{currentstroke}%
\pgfsetdash{}{0pt}%
\pgfsys@defobject{currentmarker}{\pgfqpoint{0.000000in}{-0.048611in}}{\pgfqpoint{0.000000in}{0.000000in}}{%
\pgfpathmoveto{\pgfqpoint{0.000000in}{0.000000in}}%
\pgfpathlineto{\pgfqpoint{0.000000in}{-0.048611in}}%
\pgfusepath{stroke,fill}%
}%
\begin{pgfscope}%
\pgfsys@transformshift{2.855347in}{0.499444in}%
\pgfsys@useobject{currentmarker}{}%
\end{pgfscope}%
\end{pgfscope}%
\begin{pgfscope}%
\definecolor{textcolor}{rgb}{0.000000,0.000000,0.000000}%
\pgfsetstrokecolor{textcolor}%
\pgfsetfillcolor{textcolor}%
\pgftext[x=2.855347in,y=0.402222in,,top]{\color{textcolor}\rmfamily\fontsize{10.000000}{12.000000}\selectfont 0.6}%
\end{pgfscope}%
\begin{pgfscope}%
\pgfsetbuttcap%
\pgfsetroundjoin%
\definecolor{currentfill}{rgb}{0.000000,0.000000,0.000000}%
\pgfsetfillcolor{currentfill}%
\pgfsetlinewidth{0.803000pt}%
\definecolor{currentstroke}{rgb}{0.000000,0.000000,0.000000}%
\pgfsetstrokecolor{currentstroke}%
\pgfsetdash{}{0pt}%
\pgfsys@defobject{currentmarker}{\pgfqpoint{0.000000in}{-0.048611in}}{\pgfqpoint{0.000000in}{0.000000in}}{%
\pgfpathmoveto{\pgfqpoint{0.000000in}{0.000000in}}%
\pgfpathlineto{\pgfqpoint{0.000000in}{-0.048611in}}%
\pgfusepath{stroke,fill}%
}%
\begin{pgfscope}%
\pgfsys@transformshift{3.239010in}{0.499444in}%
\pgfsys@useobject{currentmarker}{}%
\end{pgfscope}%
\end{pgfscope}%
\begin{pgfscope}%
\definecolor{textcolor}{rgb}{0.000000,0.000000,0.000000}%
\pgfsetstrokecolor{textcolor}%
\pgfsetfillcolor{textcolor}%
\pgftext[x=3.239010in,y=0.402222in,,top]{\color{textcolor}\rmfamily\fontsize{10.000000}{12.000000}\selectfont 0.7}%
\end{pgfscope}%
\begin{pgfscope}%
\pgfsetbuttcap%
\pgfsetroundjoin%
\definecolor{currentfill}{rgb}{0.000000,0.000000,0.000000}%
\pgfsetfillcolor{currentfill}%
\pgfsetlinewidth{0.803000pt}%
\definecolor{currentstroke}{rgb}{0.000000,0.000000,0.000000}%
\pgfsetstrokecolor{currentstroke}%
\pgfsetdash{}{0pt}%
\pgfsys@defobject{currentmarker}{\pgfqpoint{0.000000in}{-0.048611in}}{\pgfqpoint{0.000000in}{0.000000in}}{%
\pgfpathmoveto{\pgfqpoint{0.000000in}{0.000000in}}%
\pgfpathlineto{\pgfqpoint{0.000000in}{-0.048611in}}%
\pgfusepath{stroke,fill}%
}%
\begin{pgfscope}%
\pgfsys@transformshift{3.622674in}{0.499444in}%
\pgfsys@useobject{currentmarker}{}%
\end{pgfscope}%
\end{pgfscope}%
\begin{pgfscope}%
\definecolor{textcolor}{rgb}{0.000000,0.000000,0.000000}%
\pgfsetstrokecolor{textcolor}%
\pgfsetfillcolor{textcolor}%
\pgftext[x=3.622674in,y=0.402222in,,top]{\color{textcolor}\rmfamily\fontsize{10.000000}{12.000000}\selectfont 0.8}%
\end{pgfscope}%
\begin{pgfscope}%
\pgfsetbuttcap%
\pgfsetroundjoin%
\definecolor{currentfill}{rgb}{0.000000,0.000000,0.000000}%
\pgfsetfillcolor{currentfill}%
\pgfsetlinewidth{0.803000pt}%
\definecolor{currentstroke}{rgb}{0.000000,0.000000,0.000000}%
\pgfsetstrokecolor{currentstroke}%
\pgfsetdash{}{0pt}%
\pgfsys@defobject{currentmarker}{\pgfqpoint{0.000000in}{-0.048611in}}{\pgfqpoint{0.000000in}{0.000000in}}{%
\pgfpathmoveto{\pgfqpoint{0.000000in}{0.000000in}}%
\pgfpathlineto{\pgfqpoint{0.000000in}{-0.048611in}}%
\pgfusepath{stroke,fill}%
}%
\begin{pgfscope}%
\pgfsys@transformshift{4.006337in}{0.499444in}%
\pgfsys@useobject{currentmarker}{}%
\end{pgfscope}%
\end{pgfscope}%
\begin{pgfscope}%
\definecolor{textcolor}{rgb}{0.000000,0.000000,0.000000}%
\pgfsetstrokecolor{textcolor}%
\pgfsetfillcolor{textcolor}%
\pgftext[x=4.006337in,y=0.402222in,,top]{\color{textcolor}\rmfamily\fontsize{10.000000}{12.000000}\selectfont 0.9}%
\end{pgfscope}%
\begin{pgfscope}%
\pgfsetbuttcap%
\pgfsetroundjoin%
\definecolor{currentfill}{rgb}{0.000000,0.000000,0.000000}%
\pgfsetfillcolor{currentfill}%
\pgfsetlinewidth{0.803000pt}%
\definecolor{currentstroke}{rgb}{0.000000,0.000000,0.000000}%
\pgfsetstrokecolor{currentstroke}%
\pgfsetdash{}{0pt}%
\pgfsys@defobject{currentmarker}{\pgfqpoint{0.000000in}{-0.048611in}}{\pgfqpoint{0.000000in}{0.000000in}}{%
\pgfpathmoveto{\pgfqpoint{0.000000in}{0.000000in}}%
\pgfpathlineto{\pgfqpoint{0.000000in}{-0.048611in}}%
\pgfusepath{stroke,fill}%
}%
\begin{pgfscope}%
\pgfsys@transformshift{4.390000in}{0.499444in}%
\pgfsys@useobject{currentmarker}{}%
\end{pgfscope}%
\end{pgfscope}%
\begin{pgfscope}%
\definecolor{textcolor}{rgb}{0.000000,0.000000,0.000000}%
\pgfsetstrokecolor{textcolor}%
\pgfsetfillcolor{textcolor}%
\pgftext[x=4.390000in,y=0.402222in,,top]{\color{textcolor}\rmfamily\fontsize{10.000000}{12.000000}\selectfont 1.0}%
\end{pgfscope}%
\begin{pgfscope}%
\definecolor{textcolor}{rgb}{0.000000,0.000000,0.000000}%
\pgfsetstrokecolor{textcolor}%
\pgfsetfillcolor{textcolor}%
\pgftext[x=2.452500in,y=0.223333in,,top]{\color{textcolor}\rmfamily\fontsize{10.000000}{12.000000}\selectfont \(\displaystyle p\)}%
\end{pgfscope}%
\begin{pgfscope}%
\pgfsetbuttcap%
\pgfsetroundjoin%
\definecolor{currentfill}{rgb}{0.000000,0.000000,0.000000}%
\pgfsetfillcolor{currentfill}%
\pgfsetlinewidth{0.803000pt}%
\definecolor{currentstroke}{rgb}{0.000000,0.000000,0.000000}%
\pgfsetstrokecolor{currentstroke}%
\pgfsetdash{}{0pt}%
\pgfsys@defobject{currentmarker}{\pgfqpoint{-0.048611in}{0.000000in}}{\pgfqpoint{-0.000000in}{0.000000in}}{%
\pgfpathmoveto{\pgfqpoint{-0.000000in}{0.000000in}}%
\pgfpathlineto{\pgfqpoint{-0.048611in}{0.000000in}}%
\pgfusepath{stroke,fill}%
}%
\begin{pgfscope}%
\pgfsys@transformshift{0.515000in}{0.499444in}%
\pgfsys@useobject{currentmarker}{}%
\end{pgfscope}%
\end{pgfscope}%
\begin{pgfscope}%
\definecolor{textcolor}{rgb}{0.000000,0.000000,0.000000}%
\pgfsetstrokecolor{textcolor}%
\pgfsetfillcolor{textcolor}%
\pgftext[x=0.348333in, y=0.451250in, left, base]{\color{textcolor}\rmfamily\fontsize{10.000000}{12.000000}\selectfont \(\displaystyle {0}\)}%
\end{pgfscope}%
\begin{pgfscope}%
\pgfsetbuttcap%
\pgfsetroundjoin%
\definecolor{currentfill}{rgb}{0.000000,0.000000,0.000000}%
\pgfsetfillcolor{currentfill}%
\pgfsetlinewidth{0.803000pt}%
\definecolor{currentstroke}{rgb}{0.000000,0.000000,0.000000}%
\pgfsetstrokecolor{currentstroke}%
\pgfsetdash{}{0pt}%
\pgfsys@defobject{currentmarker}{\pgfqpoint{-0.048611in}{0.000000in}}{\pgfqpoint{-0.000000in}{0.000000in}}{%
\pgfpathmoveto{\pgfqpoint{-0.000000in}{0.000000in}}%
\pgfpathlineto{\pgfqpoint{-0.048611in}{0.000000in}}%
\pgfusepath{stroke,fill}%
}%
\begin{pgfscope}%
\pgfsys@transformshift{0.515000in}{0.897317in}%
\pgfsys@useobject{currentmarker}{}%
\end{pgfscope}%
\end{pgfscope}%
\begin{pgfscope}%
\definecolor{textcolor}{rgb}{0.000000,0.000000,0.000000}%
\pgfsetstrokecolor{textcolor}%
\pgfsetfillcolor{textcolor}%
\pgftext[x=0.348333in, y=0.849122in, left, base]{\color{textcolor}\rmfamily\fontsize{10.000000}{12.000000}\selectfont \(\displaystyle {5}\)}%
\end{pgfscope}%
\begin{pgfscope}%
\pgfsetbuttcap%
\pgfsetroundjoin%
\definecolor{currentfill}{rgb}{0.000000,0.000000,0.000000}%
\pgfsetfillcolor{currentfill}%
\pgfsetlinewidth{0.803000pt}%
\definecolor{currentstroke}{rgb}{0.000000,0.000000,0.000000}%
\pgfsetstrokecolor{currentstroke}%
\pgfsetdash{}{0pt}%
\pgfsys@defobject{currentmarker}{\pgfqpoint{-0.048611in}{0.000000in}}{\pgfqpoint{-0.000000in}{0.000000in}}{%
\pgfpathmoveto{\pgfqpoint{-0.000000in}{0.000000in}}%
\pgfpathlineto{\pgfqpoint{-0.048611in}{0.000000in}}%
\pgfusepath{stroke,fill}%
}%
\begin{pgfscope}%
\pgfsys@transformshift{0.515000in}{1.295190in}%
\pgfsys@useobject{currentmarker}{}%
\end{pgfscope}%
\end{pgfscope}%
\begin{pgfscope}%
\definecolor{textcolor}{rgb}{0.000000,0.000000,0.000000}%
\pgfsetstrokecolor{textcolor}%
\pgfsetfillcolor{textcolor}%
\pgftext[x=0.278889in, y=1.246995in, left, base]{\color{textcolor}\rmfamily\fontsize{10.000000}{12.000000}\selectfont \(\displaystyle {10}\)}%
\end{pgfscope}%
\begin{pgfscope}%
\definecolor{textcolor}{rgb}{0.000000,0.000000,0.000000}%
\pgfsetstrokecolor{textcolor}%
\pgfsetfillcolor{textcolor}%
\pgftext[x=0.223333in,y=1.076944in,,bottom,rotate=90.000000]{\color{textcolor}\rmfamily\fontsize{10.000000}{12.000000}\selectfont Percent of Data Set}%
\end{pgfscope}%
\begin{pgfscope}%
\pgfsetrectcap%
\pgfsetmiterjoin%
\pgfsetlinewidth{0.803000pt}%
\definecolor{currentstroke}{rgb}{0.000000,0.000000,0.000000}%
\pgfsetstrokecolor{currentstroke}%
\pgfsetdash{}{0pt}%
\pgfpathmoveto{\pgfqpoint{0.515000in}{0.499444in}}%
\pgfpathlineto{\pgfqpoint{0.515000in}{1.654444in}}%
\pgfusepath{stroke}%
\end{pgfscope}%
\begin{pgfscope}%
\pgfsetrectcap%
\pgfsetmiterjoin%
\pgfsetlinewidth{0.803000pt}%
\definecolor{currentstroke}{rgb}{0.000000,0.000000,0.000000}%
\pgfsetstrokecolor{currentstroke}%
\pgfsetdash{}{0pt}%
\pgfpathmoveto{\pgfqpoint{4.390000in}{0.499444in}}%
\pgfpathlineto{\pgfqpoint{4.390000in}{1.654444in}}%
\pgfusepath{stroke}%
\end{pgfscope}%
\begin{pgfscope}%
\pgfsetrectcap%
\pgfsetmiterjoin%
\pgfsetlinewidth{0.803000pt}%
\definecolor{currentstroke}{rgb}{0.000000,0.000000,0.000000}%
\pgfsetstrokecolor{currentstroke}%
\pgfsetdash{}{0pt}%
\pgfpathmoveto{\pgfqpoint{0.515000in}{0.499444in}}%
\pgfpathlineto{\pgfqpoint{4.390000in}{0.499444in}}%
\pgfusepath{stroke}%
\end{pgfscope}%
\begin{pgfscope}%
\pgfsetrectcap%
\pgfsetmiterjoin%
\pgfsetlinewidth{0.803000pt}%
\definecolor{currentstroke}{rgb}{0.000000,0.000000,0.000000}%
\pgfsetstrokecolor{currentstroke}%
\pgfsetdash{}{0pt}%
\pgfpathmoveto{\pgfqpoint{0.515000in}{1.654444in}}%
\pgfpathlineto{\pgfqpoint{4.390000in}{1.654444in}}%
\pgfusepath{stroke}%
\end{pgfscope}%
\begin{pgfscope}%
\pgfsetbuttcap%
\pgfsetmiterjoin%
\definecolor{currentfill}{rgb}{1.000000,1.000000,1.000000}%
\pgfsetfillcolor{currentfill}%
\pgfsetfillopacity{0.800000}%
\pgfsetlinewidth{1.003750pt}%
\definecolor{currentstroke}{rgb}{0.800000,0.800000,0.800000}%
\pgfsetstrokecolor{currentstroke}%
\pgfsetstrokeopacity{0.800000}%
\pgfsetdash{}{0pt}%
\pgfpathmoveto{\pgfqpoint{3.613056in}{1.154445in}}%
\pgfpathlineto{\pgfqpoint{4.292778in}{1.154445in}}%
\pgfpathquadraticcurveto{\pgfqpoint{4.320556in}{1.154445in}}{\pgfqpoint{4.320556in}{1.182222in}}%
\pgfpathlineto{\pgfqpoint{4.320556in}{1.557222in}}%
\pgfpathquadraticcurveto{\pgfqpoint{4.320556in}{1.585000in}}{\pgfqpoint{4.292778in}{1.585000in}}%
\pgfpathlineto{\pgfqpoint{3.613056in}{1.585000in}}%
\pgfpathquadraticcurveto{\pgfqpoint{3.585278in}{1.585000in}}{\pgfqpoint{3.585278in}{1.557222in}}%
\pgfpathlineto{\pgfqpoint{3.585278in}{1.182222in}}%
\pgfpathquadraticcurveto{\pgfqpoint{3.585278in}{1.154445in}}{\pgfqpoint{3.613056in}{1.154445in}}%
\pgfpathlineto{\pgfqpoint{3.613056in}{1.154445in}}%
\pgfpathclose%
\pgfusepath{stroke,fill}%
\end{pgfscope}%
\begin{pgfscope}%
\pgfsetbuttcap%
\pgfsetmiterjoin%
\pgfsetlinewidth{1.003750pt}%
\definecolor{currentstroke}{rgb}{0.000000,0.000000,0.000000}%
\pgfsetstrokecolor{currentstroke}%
\pgfsetdash{}{0pt}%
\pgfpathmoveto{\pgfqpoint{3.640834in}{1.432222in}}%
\pgfpathlineto{\pgfqpoint{3.918611in}{1.432222in}}%
\pgfpathlineto{\pgfqpoint{3.918611in}{1.529444in}}%
\pgfpathlineto{\pgfqpoint{3.640834in}{1.529444in}}%
\pgfpathlineto{\pgfqpoint{3.640834in}{1.432222in}}%
\pgfpathclose%
\pgfusepath{stroke}%
\end{pgfscope}%
\begin{pgfscope}%
\definecolor{textcolor}{rgb}{0.000000,0.000000,0.000000}%
\pgfsetstrokecolor{textcolor}%
\pgfsetfillcolor{textcolor}%
\pgftext[x=4.029723in,y=1.432222in,left,base]{\color{textcolor}\rmfamily\fontsize{10.000000}{12.000000}\selectfont Neg}%
\end{pgfscope}%
\begin{pgfscope}%
\pgfsetbuttcap%
\pgfsetmiterjoin%
\definecolor{currentfill}{rgb}{0.000000,0.000000,0.000000}%
\pgfsetfillcolor{currentfill}%
\pgfsetlinewidth{0.000000pt}%
\definecolor{currentstroke}{rgb}{0.000000,0.000000,0.000000}%
\pgfsetstrokecolor{currentstroke}%
\pgfsetstrokeopacity{0.000000}%
\pgfsetdash{}{0pt}%
\pgfpathmoveto{\pgfqpoint{3.640834in}{1.236944in}}%
\pgfpathlineto{\pgfqpoint{3.918611in}{1.236944in}}%
\pgfpathlineto{\pgfqpoint{3.918611in}{1.334167in}}%
\pgfpathlineto{\pgfqpoint{3.640834in}{1.334167in}}%
\pgfpathlineto{\pgfqpoint{3.640834in}{1.236944in}}%
\pgfpathclose%
\pgfusepath{fill}%
\end{pgfscope}%
\begin{pgfscope}%
\definecolor{textcolor}{rgb}{0.000000,0.000000,0.000000}%
\pgfsetstrokecolor{textcolor}%
\pgfsetfillcolor{textcolor}%
\pgftext[x=4.029723in,y=1.236944in,left,base]{\color{textcolor}\rmfamily\fontsize{10.000000}{12.000000}\selectfont Pos}%
\end{pgfscope}%
\end{pgfpicture}%
\makeatother%
\endgroup%
	
&
	\vskip 0pt
	\hfil ROC Curve
	
	%% Creator: Matplotlib, PGF backend
%%
%% To include the figure in your LaTeX document, write
%%   \input{<filename>.pgf}
%%
%% Make sure the required packages are loaded in your preamble
%%   \usepackage{pgf}
%%
%% Also ensure that all the required font packages are loaded; for instance,
%% the lmodern package is sometimes necessary when using math font.
%%   \usepackage{lmodern}
%%
%% Figures using additional raster images can only be included by \input if
%% they are in the same directory as the main LaTeX file. For loading figures
%% from other directories you can use the `import` package
%%   \usepackage{import}
%%
%% and then include the figures with
%%   \import{<path to file>}{<filename>.pgf}
%%
%% Matplotlib used the following preamble
%%   
%%   \usepackage{fontspec}
%%   \makeatletter\@ifpackageloaded{underscore}{}{\usepackage[strings]{underscore}}\makeatother
%%
\begingroup%
\makeatletter%
\begin{pgfpicture}%
\pgfpathrectangle{\pgfpointorigin}{\pgfqpoint{2.221861in}{1.754444in}}%
\pgfusepath{use as bounding box, clip}%
\begin{pgfscope}%
\pgfsetbuttcap%
\pgfsetmiterjoin%
\definecolor{currentfill}{rgb}{1.000000,1.000000,1.000000}%
\pgfsetfillcolor{currentfill}%
\pgfsetlinewidth{0.000000pt}%
\definecolor{currentstroke}{rgb}{1.000000,1.000000,1.000000}%
\pgfsetstrokecolor{currentstroke}%
\pgfsetdash{}{0pt}%
\pgfpathmoveto{\pgfqpoint{0.000000in}{0.000000in}}%
\pgfpathlineto{\pgfqpoint{2.221861in}{0.000000in}}%
\pgfpathlineto{\pgfqpoint{2.221861in}{1.754444in}}%
\pgfpathlineto{\pgfqpoint{0.000000in}{1.754444in}}%
\pgfpathlineto{\pgfqpoint{0.000000in}{0.000000in}}%
\pgfpathclose%
\pgfusepath{fill}%
\end{pgfscope}%
\begin{pgfscope}%
\pgfsetbuttcap%
\pgfsetmiterjoin%
\definecolor{currentfill}{rgb}{1.000000,1.000000,1.000000}%
\pgfsetfillcolor{currentfill}%
\pgfsetlinewidth{0.000000pt}%
\definecolor{currentstroke}{rgb}{0.000000,0.000000,0.000000}%
\pgfsetstrokecolor{currentstroke}%
\pgfsetstrokeopacity{0.000000}%
\pgfsetdash{}{0pt}%
\pgfpathmoveto{\pgfqpoint{0.553581in}{0.499444in}}%
\pgfpathlineto{\pgfqpoint{2.103581in}{0.499444in}}%
\pgfpathlineto{\pgfqpoint{2.103581in}{1.654444in}}%
\pgfpathlineto{\pgfqpoint{0.553581in}{1.654444in}}%
\pgfpathlineto{\pgfqpoint{0.553581in}{0.499444in}}%
\pgfpathclose%
\pgfusepath{fill}%
\end{pgfscope}%
\begin{pgfscope}%
\pgfsetbuttcap%
\pgfsetroundjoin%
\definecolor{currentfill}{rgb}{0.000000,0.000000,0.000000}%
\pgfsetfillcolor{currentfill}%
\pgfsetlinewidth{0.803000pt}%
\definecolor{currentstroke}{rgb}{0.000000,0.000000,0.000000}%
\pgfsetstrokecolor{currentstroke}%
\pgfsetdash{}{0pt}%
\pgfsys@defobject{currentmarker}{\pgfqpoint{0.000000in}{-0.048611in}}{\pgfqpoint{0.000000in}{0.000000in}}{%
\pgfpathmoveto{\pgfqpoint{0.000000in}{0.000000in}}%
\pgfpathlineto{\pgfqpoint{0.000000in}{-0.048611in}}%
\pgfusepath{stroke,fill}%
}%
\begin{pgfscope}%
\pgfsys@transformshift{0.624035in}{0.499444in}%
\pgfsys@useobject{currentmarker}{}%
\end{pgfscope}%
\end{pgfscope}%
\begin{pgfscope}%
\definecolor{textcolor}{rgb}{0.000000,0.000000,0.000000}%
\pgfsetstrokecolor{textcolor}%
\pgfsetfillcolor{textcolor}%
\pgftext[x=0.624035in,y=0.402222in,,top]{\color{textcolor}\rmfamily\fontsize{10.000000}{12.000000}\selectfont \(\displaystyle {0.0}\)}%
\end{pgfscope}%
\begin{pgfscope}%
\pgfsetbuttcap%
\pgfsetroundjoin%
\definecolor{currentfill}{rgb}{0.000000,0.000000,0.000000}%
\pgfsetfillcolor{currentfill}%
\pgfsetlinewidth{0.803000pt}%
\definecolor{currentstroke}{rgb}{0.000000,0.000000,0.000000}%
\pgfsetstrokecolor{currentstroke}%
\pgfsetdash{}{0pt}%
\pgfsys@defobject{currentmarker}{\pgfqpoint{0.000000in}{-0.048611in}}{\pgfqpoint{0.000000in}{0.000000in}}{%
\pgfpathmoveto{\pgfqpoint{0.000000in}{0.000000in}}%
\pgfpathlineto{\pgfqpoint{0.000000in}{-0.048611in}}%
\pgfusepath{stroke,fill}%
}%
\begin{pgfscope}%
\pgfsys@transformshift{1.328581in}{0.499444in}%
\pgfsys@useobject{currentmarker}{}%
\end{pgfscope}%
\end{pgfscope}%
\begin{pgfscope}%
\definecolor{textcolor}{rgb}{0.000000,0.000000,0.000000}%
\pgfsetstrokecolor{textcolor}%
\pgfsetfillcolor{textcolor}%
\pgftext[x=1.328581in,y=0.402222in,,top]{\color{textcolor}\rmfamily\fontsize{10.000000}{12.000000}\selectfont \(\displaystyle {0.5}\)}%
\end{pgfscope}%
\begin{pgfscope}%
\pgfsetbuttcap%
\pgfsetroundjoin%
\definecolor{currentfill}{rgb}{0.000000,0.000000,0.000000}%
\pgfsetfillcolor{currentfill}%
\pgfsetlinewidth{0.803000pt}%
\definecolor{currentstroke}{rgb}{0.000000,0.000000,0.000000}%
\pgfsetstrokecolor{currentstroke}%
\pgfsetdash{}{0pt}%
\pgfsys@defobject{currentmarker}{\pgfqpoint{0.000000in}{-0.048611in}}{\pgfqpoint{0.000000in}{0.000000in}}{%
\pgfpathmoveto{\pgfqpoint{0.000000in}{0.000000in}}%
\pgfpathlineto{\pgfqpoint{0.000000in}{-0.048611in}}%
\pgfusepath{stroke,fill}%
}%
\begin{pgfscope}%
\pgfsys@transformshift{2.033126in}{0.499444in}%
\pgfsys@useobject{currentmarker}{}%
\end{pgfscope}%
\end{pgfscope}%
\begin{pgfscope}%
\definecolor{textcolor}{rgb}{0.000000,0.000000,0.000000}%
\pgfsetstrokecolor{textcolor}%
\pgfsetfillcolor{textcolor}%
\pgftext[x=2.033126in,y=0.402222in,,top]{\color{textcolor}\rmfamily\fontsize{10.000000}{12.000000}\selectfont \(\displaystyle {1.0}\)}%
\end{pgfscope}%
\begin{pgfscope}%
\definecolor{textcolor}{rgb}{0.000000,0.000000,0.000000}%
\pgfsetstrokecolor{textcolor}%
\pgfsetfillcolor{textcolor}%
\pgftext[x=1.328581in,y=0.223333in,,top]{\color{textcolor}\rmfamily\fontsize{10.000000}{12.000000}\selectfont False positive rate}%
\end{pgfscope}%
\begin{pgfscope}%
\pgfsetbuttcap%
\pgfsetroundjoin%
\definecolor{currentfill}{rgb}{0.000000,0.000000,0.000000}%
\pgfsetfillcolor{currentfill}%
\pgfsetlinewidth{0.803000pt}%
\definecolor{currentstroke}{rgb}{0.000000,0.000000,0.000000}%
\pgfsetstrokecolor{currentstroke}%
\pgfsetdash{}{0pt}%
\pgfsys@defobject{currentmarker}{\pgfqpoint{-0.048611in}{0.000000in}}{\pgfqpoint{-0.000000in}{0.000000in}}{%
\pgfpathmoveto{\pgfqpoint{-0.000000in}{0.000000in}}%
\pgfpathlineto{\pgfqpoint{-0.048611in}{0.000000in}}%
\pgfusepath{stroke,fill}%
}%
\begin{pgfscope}%
\pgfsys@transformshift{0.553581in}{0.551944in}%
\pgfsys@useobject{currentmarker}{}%
\end{pgfscope}%
\end{pgfscope}%
\begin{pgfscope}%
\definecolor{textcolor}{rgb}{0.000000,0.000000,0.000000}%
\pgfsetstrokecolor{textcolor}%
\pgfsetfillcolor{textcolor}%
\pgftext[x=0.278889in, y=0.503750in, left, base]{\color{textcolor}\rmfamily\fontsize{10.000000}{12.000000}\selectfont \(\displaystyle {0.0}\)}%
\end{pgfscope}%
\begin{pgfscope}%
\pgfsetbuttcap%
\pgfsetroundjoin%
\definecolor{currentfill}{rgb}{0.000000,0.000000,0.000000}%
\pgfsetfillcolor{currentfill}%
\pgfsetlinewidth{0.803000pt}%
\definecolor{currentstroke}{rgb}{0.000000,0.000000,0.000000}%
\pgfsetstrokecolor{currentstroke}%
\pgfsetdash{}{0pt}%
\pgfsys@defobject{currentmarker}{\pgfqpoint{-0.048611in}{0.000000in}}{\pgfqpoint{-0.000000in}{0.000000in}}{%
\pgfpathmoveto{\pgfqpoint{-0.000000in}{0.000000in}}%
\pgfpathlineto{\pgfqpoint{-0.048611in}{0.000000in}}%
\pgfusepath{stroke,fill}%
}%
\begin{pgfscope}%
\pgfsys@transformshift{0.553581in}{1.076944in}%
\pgfsys@useobject{currentmarker}{}%
\end{pgfscope}%
\end{pgfscope}%
\begin{pgfscope}%
\definecolor{textcolor}{rgb}{0.000000,0.000000,0.000000}%
\pgfsetstrokecolor{textcolor}%
\pgfsetfillcolor{textcolor}%
\pgftext[x=0.278889in, y=1.028750in, left, base]{\color{textcolor}\rmfamily\fontsize{10.000000}{12.000000}\selectfont \(\displaystyle {0.5}\)}%
\end{pgfscope}%
\begin{pgfscope}%
\pgfsetbuttcap%
\pgfsetroundjoin%
\definecolor{currentfill}{rgb}{0.000000,0.000000,0.000000}%
\pgfsetfillcolor{currentfill}%
\pgfsetlinewidth{0.803000pt}%
\definecolor{currentstroke}{rgb}{0.000000,0.000000,0.000000}%
\pgfsetstrokecolor{currentstroke}%
\pgfsetdash{}{0pt}%
\pgfsys@defobject{currentmarker}{\pgfqpoint{-0.048611in}{0.000000in}}{\pgfqpoint{-0.000000in}{0.000000in}}{%
\pgfpathmoveto{\pgfqpoint{-0.000000in}{0.000000in}}%
\pgfpathlineto{\pgfqpoint{-0.048611in}{0.000000in}}%
\pgfusepath{stroke,fill}%
}%
\begin{pgfscope}%
\pgfsys@transformshift{0.553581in}{1.601944in}%
\pgfsys@useobject{currentmarker}{}%
\end{pgfscope}%
\end{pgfscope}%
\begin{pgfscope}%
\definecolor{textcolor}{rgb}{0.000000,0.000000,0.000000}%
\pgfsetstrokecolor{textcolor}%
\pgfsetfillcolor{textcolor}%
\pgftext[x=0.278889in, y=1.553750in, left, base]{\color{textcolor}\rmfamily\fontsize{10.000000}{12.000000}\selectfont \(\displaystyle {1.0}\)}%
\end{pgfscope}%
\begin{pgfscope}%
\definecolor{textcolor}{rgb}{0.000000,0.000000,0.000000}%
\pgfsetstrokecolor{textcolor}%
\pgfsetfillcolor{textcolor}%
\pgftext[x=0.223333in,y=1.076944in,,bottom,rotate=90.000000]{\color{textcolor}\rmfamily\fontsize{10.000000}{12.000000}\selectfont True positive rate}%
\end{pgfscope}%
\begin{pgfscope}%
\pgfpathrectangle{\pgfqpoint{0.553581in}{0.499444in}}{\pgfqpoint{1.550000in}{1.155000in}}%
\pgfusepath{clip}%
\pgfsetbuttcap%
\pgfsetroundjoin%
\pgfsetlinewidth{1.505625pt}%
\definecolor{currentstroke}{rgb}{0.000000,0.000000,0.000000}%
\pgfsetstrokecolor{currentstroke}%
\pgfsetdash{{5.550000pt}{2.400000pt}}{0.000000pt}%
\pgfpathmoveto{\pgfqpoint{0.624035in}{0.551944in}}%
\pgfpathlineto{\pgfqpoint{2.033126in}{1.601944in}}%
\pgfusepath{stroke}%
\end{pgfscope}%
\begin{pgfscope}%
\pgfpathrectangle{\pgfqpoint{0.553581in}{0.499444in}}{\pgfqpoint{1.550000in}{1.155000in}}%
\pgfusepath{clip}%
\pgfsetrectcap%
\pgfsetroundjoin%
\pgfsetlinewidth{1.505625pt}%
\definecolor{currentstroke}{rgb}{0.000000,0.000000,0.000000}%
\pgfsetstrokecolor{currentstroke}%
\pgfsetdash{}{0pt}%
\pgfpathmoveto{\pgfqpoint{0.624035in}{0.551944in}}%
\pgfpathlineto{\pgfqpoint{0.632298in}{0.616605in}}%
\pgfpathlineto{\pgfqpoint{0.653523in}{0.720410in}}%
\pgfpathlineto{\pgfqpoint{0.694261in}{0.842436in}}%
\pgfpathlineto{\pgfqpoint{0.759789in}{0.977811in}}%
\pgfpathlineto{\pgfqpoint{0.860417in}{1.110578in}}%
\pgfpathlineto{\pgfqpoint{1.004004in}{1.240365in}}%
\pgfpathlineto{\pgfqpoint{1.185428in}{1.359040in}}%
\pgfpathlineto{\pgfqpoint{1.401437in}{1.457288in}}%
\pgfpathlineto{\pgfqpoint{1.633715in}{1.532534in}}%
\pgfpathlineto{\pgfqpoint{1.860395in}{1.580898in}}%
\pgfpathlineto{\pgfqpoint{2.033126in}{1.601944in}}%
\pgfpathlineto{\pgfqpoint{2.033126in}{1.601944in}}%
\pgfusepath{stroke}%
\end{pgfscope}%
\begin{pgfscope}%
\pgfsetrectcap%
\pgfsetmiterjoin%
\pgfsetlinewidth{0.803000pt}%
\definecolor{currentstroke}{rgb}{0.000000,0.000000,0.000000}%
\pgfsetstrokecolor{currentstroke}%
\pgfsetdash{}{0pt}%
\pgfpathmoveto{\pgfqpoint{0.553581in}{0.499444in}}%
\pgfpathlineto{\pgfqpoint{0.553581in}{1.654444in}}%
\pgfusepath{stroke}%
\end{pgfscope}%
\begin{pgfscope}%
\pgfsetrectcap%
\pgfsetmiterjoin%
\pgfsetlinewidth{0.803000pt}%
\definecolor{currentstroke}{rgb}{0.000000,0.000000,0.000000}%
\pgfsetstrokecolor{currentstroke}%
\pgfsetdash{}{0pt}%
\pgfpathmoveto{\pgfqpoint{2.103581in}{0.499444in}}%
\pgfpathlineto{\pgfqpoint{2.103581in}{1.654444in}}%
\pgfusepath{stroke}%
\end{pgfscope}%
\begin{pgfscope}%
\pgfsetrectcap%
\pgfsetmiterjoin%
\pgfsetlinewidth{0.803000pt}%
\definecolor{currentstroke}{rgb}{0.000000,0.000000,0.000000}%
\pgfsetstrokecolor{currentstroke}%
\pgfsetdash{}{0pt}%
\pgfpathmoveto{\pgfqpoint{0.553581in}{0.499444in}}%
\pgfpathlineto{\pgfqpoint{2.103581in}{0.499444in}}%
\pgfusepath{stroke}%
\end{pgfscope}%
\begin{pgfscope}%
\pgfsetrectcap%
\pgfsetmiterjoin%
\pgfsetlinewidth{0.803000pt}%
\definecolor{currentstroke}{rgb}{0.000000,0.000000,0.000000}%
\pgfsetstrokecolor{currentstroke}%
\pgfsetdash{}{0pt}%
\pgfpathmoveto{\pgfqpoint{0.553581in}{1.654444in}}%
\pgfpathlineto{\pgfqpoint{2.103581in}{1.654444in}}%
\pgfusepath{stroke}%
\end{pgfscope}%
\begin{pgfscope}%
\pgfsetbuttcap%
\pgfsetmiterjoin%
\definecolor{currentfill}{rgb}{1.000000,1.000000,1.000000}%
\pgfsetfillcolor{currentfill}%
\pgfsetfillopacity{0.800000}%
\pgfsetlinewidth{1.003750pt}%
\definecolor{currentstroke}{rgb}{0.800000,0.800000,0.800000}%
\pgfsetstrokecolor{currentstroke}%
\pgfsetstrokeopacity{0.800000}%
\pgfsetdash{}{0pt}%
\pgfpathmoveto{\pgfqpoint{0.832747in}{0.568889in}}%
\pgfpathlineto{\pgfqpoint{2.006358in}{0.568889in}}%
\pgfpathquadraticcurveto{\pgfqpoint{2.034136in}{0.568889in}}{\pgfqpoint{2.034136in}{0.596666in}}%
\pgfpathlineto{\pgfqpoint{2.034136in}{0.776388in}}%
\pgfpathquadraticcurveto{\pgfqpoint{2.034136in}{0.804166in}}{\pgfqpoint{2.006358in}{0.804166in}}%
\pgfpathlineto{\pgfqpoint{0.832747in}{0.804166in}}%
\pgfpathquadraticcurveto{\pgfqpoint{0.804970in}{0.804166in}}{\pgfqpoint{0.804970in}{0.776388in}}%
\pgfpathlineto{\pgfqpoint{0.804970in}{0.596666in}}%
\pgfpathquadraticcurveto{\pgfqpoint{0.804970in}{0.568889in}}{\pgfqpoint{0.832747in}{0.568889in}}%
\pgfpathlineto{\pgfqpoint{0.832747in}{0.568889in}}%
\pgfpathclose%
\pgfusepath{stroke,fill}%
\end{pgfscope}%
\begin{pgfscope}%
\pgfsetrectcap%
\pgfsetroundjoin%
\pgfsetlinewidth{1.505625pt}%
\definecolor{currentstroke}{rgb}{0.000000,0.000000,0.000000}%
\pgfsetstrokecolor{currentstroke}%
\pgfsetdash{}{0pt}%
\pgfpathmoveto{\pgfqpoint{0.860525in}{0.700000in}}%
\pgfpathlineto{\pgfqpoint{0.999414in}{0.700000in}}%
\pgfpathlineto{\pgfqpoint{1.138303in}{0.700000in}}%
\pgfusepath{stroke}%
\end{pgfscope}%
\begin{pgfscope}%
\definecolor{textcolor}{rgb}{0.000000,0.000000,0.000000}%
\pgfsetstrokecolor{textcolor}%
\pgfsetfillcolor{textcolor}%
\pgftext[x=1.249414in,y=0.651388in,left,base]{\color{textcolor}\rmfamily\fontsize{10.000000}{12.000000}\selectfont AUC=0.758}%
\end{pgfscope}%
\end{pgfpicture}%
\makeatother%
\endgroup%

\end{tabular}

\

\

Other stuff

\


%%%
\begin{comment}
If we set the discrimination threshold about $0.7$, the model would classify almost all of the samples, both positive and negative class, correctly, with about the same number of false positives (sending an ambulance when one is not needed, negative class samples with $p > 0.7$) and false negatives (not sending an ambulance when one is needed, positive class samples with $p < 0.7$).  If we (as a society) were willing to tolerate more false positives, we could set the discrimination threshold lower, and if budgets were tighter we could increase the $p$ threshold.  

The table below gives the number of true negatives (TN), false positives (FP), false negatives (FN), and true positives (TP) for the 499,496 samples in the test set, along with the precision and recall values, for different discrimination thresholds $p$.  The precision is the proportion of ambulances we sent that were needed, and the recall is the proportion of ambulances needed that we sent.  

$$\text{Precision} = \frac{TP}{FP+TP}, \qquad \text{Recall} = \frac{TP}{FN + TP}$$

\begin{center}
\begin{tabular}{rrrrrrrrrrrrrr}
\toprule
$p$ &   TN &       FP &      FN &      TP &  Precision &   Recall \\
\midrule
0.50 &  346,776 &   73,794 &       1 &  78,925 &  0.52 &  1.00       \\
0.60 &  390,335 &   30,235 &      89 &  78,837 &  0.72 &  1.00  \\
0.70 & 411,040 &    9,530 &   2,838 &  76,088 &  0.89 &  0.96 \\
0.80 & 418,739 &    1,831 &  19,174 &  59,752 &  0.97 &  0.76  \\
0.90 & 420,496 &       74 &  53,736 &  25,190 &  1.00 &  0.32 & \\
\bottomrule
\end{tabular}
\end{center}

\end{comment}
%%%



