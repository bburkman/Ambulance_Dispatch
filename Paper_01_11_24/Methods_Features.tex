\subsection{Choosing Features}
\label{features}

The \verb|Accident|, \verb|Vehicle|, and \verb|Person| files of the CRSS dataset 2016-2021 have 170 unique features.  

First we want to narrow the features to those that are relevant, of good quality, and knowable at the at the time of the automated notification (before any eyewitness reports).  Some features, like Vehicle Identification Number (VIN), have no predictive value.  Other features have missing data for more than 20\% of samples.  Some features, like drug and alcohol test results, are unknowable at the time of the automated notification.

Having data available for instantaneous analysis when the crash notification comes in is not free, and some features are more expensive than others.  A city thinking of implementing a recommendation system for immediate dispatch will need to decide how much to spend to have the data available, and whether the more expensive features increase the quality of the models enough to be worth the cost.  We categorized the features as ``Easy,'' ``Medium,'' and ``Hard,'' which can also be called ``Free,'' ``Expensive,'' and ``Problematic.''   The ``Easy'' features are those the dispatcher already has, like day of week, time of day, weather, and urban/rural.  The ``Medium'' features add details about the location (intersection, speed limit, interstate highway) and information the cell service company probably has about the primary user of the phone (age and sex).  To have the medium features instantaneously available would require coordination of many resources and be expensive to set up.  The ``Hard'' features are much more problematic, requiring more coordination of public and private records, and introduce privacy and data security issues.  Hard features include whether the location is a work zone, the likely vehicle driven by the primary user of the cell phone, and, if there are multiple automated notifications from the same location, how many crash persons are likely to be involved.   

We note here our simplifying assumption (\S \ref{simplifying_assumptions}) that we will have complete and accurate data for each automated notification.  Also, we will test three combinations of features but have not done more detailed testing to see which individual features or groups of features are most or least useful in predicting whether a crash person needs an ambulance.  

See 
\verb|Ambulance_Dispatch_01_Get_Data.ipynb|  
for a list of the excluded features.

See \verb|Ambulance_Dispatch_03_Bin_Data.ipynb|
for a list of the features we used for imputation of missing data in CRSS.


See
\verb|Ambulance_Dispatch_07_Build_Models.ipynb|.
for the complete list of the features used in the Easy, Medium, and Hard model building.